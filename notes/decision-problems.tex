\chapter{Decision and Learning Theory}
\label{ch:decision-problems}
\only<article>{This chapter deals with simple decision problems,
  whereby a decision maker (DM) makes a simple choice among many. In
  some of this problems the DM has to make a decision after first
  observing some side-information. Then the DM uses a \emph{decision
    rule} to assign a probability to each possible decision for each
  possible side-information. However, designing the decision rule is
  not trivial, as it relies on previously collected data. A
  higher-level decision includes choosing the decision rule
  itself. The problems of classification and regression fall within
  this framework. While most steps in the process can be automated and
  formalised, a lot of decisions are actual design choices made by
  humans. This creates scope for errors and misinterpretation of
  results.

  In this chapter, we shall formalise all these simple decision
  problems from the point of view of statistical decision theory. The
  first question is, given a real world application, what type of
  decision problem does it map to? Then, what kind of machine learning
  algorithms can we use to solve it? What are the underlying
  assumptions and how valid are our conclusions?  }

\section{Hierarchies of decision making problems}
\label{sec:decision-problems}
\only<presentation>{
  \begin{frame}
    \tableofcontents[ 
    currentsection, 
    hideothersubsections, 
    sectionstyle=show/shaded
    ] 
  \end{frame}
}


\only<article>{ All machine learning problems are essentially decision
  problems. This essentially means replacing some human decisions with
  machine decisions. One of the simplest decision problems is
  classification, where you want an algorithm to decide the correct
  class of some data, but even within this simple framework there is a
  multitude of decisions to be made. The first is how to frame the
  classification problem the first place. The second is how to
  collect, process and annotate the data. The third is choosing the
  type of classification model to use. The fourth is how to use the
  collected data to find an optimal classifier within the selected
  type. After all this has been done, there is the problem of
  classifying new data. In this course, we will take a holistic view
  of the problem, and consider each problem in turn, starting from the
  lowest level and working our way up.}


\subsection{Simple decision problems}
\begin{frame}
  \frametitle{Preferences}
  \only<article>{The simplest decision problem involves selecting one item from a set of choices, such as in the following examples}  
  \begin{example}
    \begin{block}{Food}
      \begin{itemize}
      \item[A] McDonald's cheeseburger
      \item[B] Surstromming
      \item[C] Oatmeal
      \end{itemize}
    \end{block}
    \begin{block}{Money}
      \begin{itemize}
      \item[A] 10,000,000 CHF
      \item[B] 10,000,000 USD
      \item[C] 1,000 BTC
      \end{itemize}
    \end{block}
    \begin{block}{Entertainment}
      \begin{itemize}
      \item[A] 1 Week in Venice
      \item[B] 1 Week Hoch-Tour in the Alps
      \item[C] 1 Week Transatlantic Cruise
      \end{itemize}
    \end{block}
  \end{example}
\end{frame}

\begin{frame}
  \frametitle{Rewards and utilities}
  \only<article>{In the decision theoretic framework, the things we
    receive are called rewards, and we assign a utility value to each
    one of them, showing which one we prefer.}
  \begin{itemize}
  \item Each choice is called a \alert{reward} $r \in \CR$.
  \item There is a \alert{utility function} $U : \CR \to \Reals$, assigning values to reward.
  \item We (weakly) prefer $A$ to $B$ iff $U(A) \geq U(B)$.
  \end{itemize}
  \only<article>{In each case, given $U$ the choice between each reward is trivial. We just select the reward:
    \[
    r^* \in \argmax_r U(r)
    \]
    The main difficult is actually selecting the appropriate utility
    function. In a behavioural context, we simply assume that humans
    act with respect to a specific utility function. However, figuring
    out this function from behavioural data is non trivial. ven when
    this assumption is correct, individuals do not have a common
    utility function.  }
  \begin{exercise}
    From your individual preferences, derive a \alert{common utility
      function} that reflects everybody's preferences in the class for
    each of the three examples. Is there a simple algorithm for
    deciding this? Would you consider the outcome fair?
  \end{exercise}
\end{frame}

\begin{frame}
  \frametitle{Preferences among random outcomes}
  \begin{example}
    Would you rather \ldots
    \begin{itemize}
    \item[A] Have 100 EUR now?
    \item[B] Flip a coin, and get 200 EUR if it comes heads?
    \end{itemize}    
  \end{example}
  \uncover<2->{
    \begin{block}{The expected utility hypothesis}
      Rational decision makers prefer choice $A$ to $B$ if
      \[
      \E(U | A) \geq \E(U | B),
      \]
      where the expected utility is
      \[
      \E(U | A) = \sum_r U(r) \Pr(r | A).
      \]
    \end{block}
    In the above example, $r \in \{0, 100, 200\}$ and $U(r)$ is
    increasing, and the coin is fair.
  }
  \begin{block}{Risk\index{risk} and monetary rewards}
    \only<article>{When $r \in \Reals$, as in the case of monetary rewards, we can use the shape of the utility function determines the amount of risk aversion. In particular:}
    \begin{itemize}
    \item<3-> If $U$ is convex, we are risk-seeking. \only<article>{In the example above, we would prefer B to A, as the expected utility of B would be higher than A, even though they give the same amount of money on average.}
    \item<4-> If $U$ is linear, we are risk neutral. \only<article>{In the example above, we would be indifferent between $A$ and $B$, as the expected amount of money is the same as the amount of money we get.}
    \item<5-> If $U$ is concave, we are risk-averse. \only<article>{Hence, in the example above, we prefer A.}
    \end{itemize}
  \end{block}
  \only<article>{This idea of risk can be used with any other utility function. We can simply replace the original utility function $U$ with a monotonic function $f(U)$ to achieve risk-sensitive behaviour. However, this is not the only risk-sensitive approach possible.}
\end{frame}


\begin{frame}
  \frametitle{Uncertain rewards}
  \only<article>{However, in real life, there are many cases where we can only choose between uncertain outcomes. The simplest example are lottery tickets, where rewards are essentially random. However, in many cases the rewards are not really random, but simply uncertain. In those cases it is useful to represent our uncertainty with probabilities as well, even though there is nothing really random.}
  \begin{itemize}
  \item Decisions $\decision \in \Decision$
  \item Each choice is called a \alert{reward} $r \in \CR$.
  \item There is a \alert{utility function} $U : \CR \to \Reals$, assigning values to reward.
  \item We (weakly) prefer $A$ to $B$ iff $U(A) \geq U(B)$.
  \end{itemize}

  \begin{example}
    \begin{columns}
      \begin{column}{0.5\textwidth}
        You are going to work, and it might rain.  What do you do?
        \begin{itemize}
        \item $\decision_1$: Take the umbrella.
        \item $\decision_2$: Risk it!
        \item $\outcome_1$: rain
        \item $\outcome_2$: dry
        \end{itemize}
      \end{column}
      \begin{column}{0.5\textwidth}
        \begin{table}
          \centering
          \begin{tabular}{c|c|c}
            $\Rew(\outcome,\decision)$ & $\decision_1$ & $\decision_2$ \\ %ro: U has only one argument.
            \hline
            $\outcome_1$ & dry, carrying umbrella & wet\\
            $\outcome_2$ & dry, carrying umbrella & dry\\
            \hline
            \hline
            $U[\Rew(\outcome,\decision)]$ & $\decision_1$ & $\decision_2$ \\
            \hline
            $\outcome_1$ & 0 & -10\\
            $\outcome_2$ & 0 & 1
          \end{tabular}
          \caption{Rewards and utilities.}
          \label{tab:rain-utility-function}
        \end{table}

        \begin{itemize}
        \item<2-> $\max_\decision \min_\outcome U = 0$
        \item<3-> $\min_\outcome \max_\decision U = 0$
        \end{itemize}
      \end{column}

    \end{columns}
  \end{example}
\end{frame}



\begin{frame}
  \frametitle{Expected utility}
  \[
  \E (U \mid a) = \sum_r U[\Rew(\outcome, \decision)] \Pr(\outcome \mid \decision)
  \]
  \begin{example}%ro: rather an exercise?
    You are going to work, and it might rain. The forecast said that
    the probability of rain $(\outcome_1)$ was $20\%$. What do you do?
    \begin{itemize}
    \item $\decision_1$: Take the umbrella.
    \item $\decision_2$: Risk it!
    \end{itemize}
    \begin{table}
      \centering
      \begin{tabular}{c|c|c}
        $\Rew(\outcome,\decision)$ & $\decision_1$ & $\decision_2$ \\ %ro: U has only one argument.
        \hline
        $\outcome_1$ & dry, carrying umbrella & wet\\
        $\outcome_2$ & dry, carrying umbrella & dry\\
        \hline
        \hline
        $U[\Rew(\outcome,\decision)]$ & $\decision_1$ & $\decision_2$ \\
        \hline
        $\outcome_1$ & 0 & -10\\
        $\outcome_2$ & 0 & 1\\
        \hline
        \hline
        $\E(U \mid \decision)$ & 0 &  -1.2 \\ 
      \end{tabular}
      \caption{Rewards, utilities, expected utility for $20\%$ probability of rain.}
      \label{tab:rain-utility-function}
    \end{table}
  \end{example}
\end{frame}





\subsection{Decision rules}

\only<article>{We now move from simple decisions to decisions that
  depend on some observation. We shall start with a simple problem in applied meteorology. Then we will discuss hypothesis testing as a decision making problem. Finally, we will go through an exercise in Bayesian methods for classification.}

\begin{frame}
  \frametitle{Bayes decision rules}
  Consider the case where outcomes are independent of decisions:
  \[
  \util (\bel, \decision) \defn \sum_{\model}  \util (\model, \decision) \bel(\model)
  \]
  This corresponds e.g. to the case where $\bel(\model)$ is the belief about an unknown world.
  \begin{definition}[Bayes utility]
    \label{def:bayes-utility}
    The maximising decision for $\bel$ has an expected utility equal to:
    \begin{equation}
      \BUtil(\bel) \defn \max_{\decision \in \Decision} \util (\bel, \decision).
      \label{eq:bayes-utility}
    \end{equation}
  \end{definition}
\end{frame}




\begin{frame}
  \frametitle{The $n$-meteorologists problem}
  \index{$n$-meteorologists|textbf}
  \only<article>{Of course, we may not always just be interested in classification performance in terms of predicting the most likely class. It strongly depends on the problem we are actually wanting to solve. In  biometric authentication, for example, we want to guard against the unlikely event that an impostor will successfully be authenticated. Even if the decision rule that always says 'OK' has the lowest classification error in practice, the expected cost of impostors means that the optimal decision rule must sometimes say 'Failed' even if this leads to false rejections sometimes.}
  \begin{exercise}
    \only<presentation>{
      \only<1>{
        \begin{itemize}
        \item Meteorological models $\CM = \set{\model_1, \ldots, \model_n}$
        \item Rain predictions at time $t$: $p_{t,\model} \defn  P_{\model}(x_t = \textrm{rain})$.
        \item Prior probability $\bel(\model) = 1/n$ for each model.
        \item Should we take the umbrella?
        \end{itemize}
      }
    }
    \only<article>{Assume you have $n$ meteorologists. At each day $t$, each meteorologist $i$ gives a probability $p_{t,\model_i}\defn P_{\model_i}(x_t = \textrm{rain})$ for rain. Consider the case of there being three meteorologists, and each one making the following prediction for the coming week. Start with a uniform prior $\bel(\model) = 1/3$ for each model.}
    {
      \begin{table}[h]
        \begin{tabular}{c|l|l|l|l|l|l|l}
          &M&T&W&T&F&S&S\\
          \hline
          CNN & 0.5 & 0.6 & 0.7 & 0.9 & 0.5 & 0.3 & 0.1\\
          SMHI & 0.3 & 0.7 & 0.8 & 0.9 & 0.5 & 0.2 & 0.1\\
          YR & 0.6 & 0.9 & 0.8 & 0.5 & 0.4 & 0.1 & 0.1\\
          \hline
          Rain? & Y & Y & Y & N & Y & N & N
        \end{tabular}
        \caption{Predictions by three different entities for the probability of rain on a particular day, along with whether or not it actually rained.}
        \label{tab:meteorologists}
      \end{table}
    }
    \uncover<2->{
      \begin{enumerate}
      \item<2-> What is your belief about the quality of each meteorologist after each day? 
      \item<3-> What is your belief about the probability of rain each day? 
        \[
        P_\bel(x_t = \textrm{rain} \mid x_1, x_2, \ldots x_{t-1})
        =
        \sum_{\model \in \Model} P_\model(x_t = \textrm{rain} \mid x_1, x_2, \ldots x_{t-1})
        \bel(\model \mid x_1, x_2, \ldots x_{t-1}) 
        \]
      \item<4-> Assume you can decide whether or not to go running each
        day. If you go running and it does not rain, your utility is 1. If
        it rains, it's -10. If you don't go running, your utility is
        0. What is the decision maximising utility in expectation (with respect to the posterior) each
        day?
      \end{enumerate}
    }
    \label{ex:meteorologists}
  \end{exercise}
\end{frame}


\section{Beliefs and probabilities}
\only<presentation>{
  \begin{frame}
    \tableofcontents[ 
    currentsection, 
    hideothersubsections, 
    sectionstyle=show/shaded
    ] 
  \end{frame}
}


\only<article>{Probability can be used to describe purely chance events, as in for example quantum physics. However, it is mostly used to describe uncertain events, such as the outcome of a dice roll or a coin flip, which only appear random. In fact, one can take it even further than that, and use it to model subjective uncertainty about any arbitrary event. Although probabilities are not the only way in which we can quantify uncertainty, it is a simple enough model, and with a rich enough history in mathematics, statistics, computer science and engineering that it is the most useful.}
\begin{frame}
  \frametitle{Uncertainty}
  \only<presentation>{
    \begin{itemize}
    \item We cannot perfectly predict the future.
    \item We cannot know for sure what happened in the past.
    \item How can we quantify this uncertainty?
    \item Probabilities!
    \end{itemize}
  }
  \only<article>{
  In this book, we use probability to quantify uncertainty. Consider a collection of possible events or statements $\Sigma$, which includes a special event $\Omega$, which is always true, so that for any event $A$, we have $A \subset \Omega$. For every events $A$, $B$ we assign some probability $P(A)$ so that, $P(A) > P(B)$ only if we think that $A$ is more likely to happen than $B$.

  Now, because we want to be able to reason about every possible combination of events in $\Sigma$, we should be able to combine them in logical statements. If $A, B$ are sets, it is natural to think of ``A or B'' as $A \cup B$ and ``A and B'' as $A \cap B$. It is also natural to think of ``not A'' as $\Omega \setminus A$. If we consider all combinations of events possible, we arrive at what is called a $\sigma$-algebra:
}
  \only<article>{
    \begin{definition}[$\sigma$-algebra]
      \index{$\sigma$-algebra}
      A $\sigma$-algebra $\Sigma$ on a set $\Omega$ has the following properties:
      \begin{enumerate}[(a)]
      \item $\Omega \in \Sigma$.
      \item If $A \in \Sigma$ then $A \subset \Omega$.
      \item If $A, B \in \Sigma$ then $A \cup B \in \Sigma$.
      \item If $A \in \Sigma$ then $\Omega \setminus A \in \Sigma$.
      \end{enumerate}
    \end{definition}
    For a more complete introduction to basic probability concepts, take a look at \citet{intro-prob:ber-tsi},\citet[Appendix B]{dimitrakakis-ortner:dmuurl-book}, or \citet{degroot:optimalstatisticaldecisions}
  }
  \only<article>{
    While we said that we assign higher probability to events we think are more likely, we would like our probability to obey certain rules. 
  }
  \begin{block}{Axioms of probability}
    \only<article>{Let $\Omega$ be the certain event, and $\Sigma$ is an appropriate $\sigma$-algebra on $\Omega$.}
    A probability measure $P$ on $(\Omega, \Sigma)$ has the following properties:
    \begin{enumerate}
    \item<2-> The probability of the certain event is $P(\Omega) = 1$
    \item<3->The probability of the impossible event is
      $P(\emptyset) = 0$
    \item<4->The probability of any event $A \in \Sigma$ is $0 \leq P(A) \leq 1$.
    \item<5-> If $A, B$ are disjoint, i.e. $A \cap B = \emptyset$, meaning
      that they cannot happen at the same time, then
      \[
      P(A \cup B) = P(A) + P(B)
      \]
    \end{enumerate}
  \end{block}
\end{frame}

\begin{frame}
  \only<article>{ Sometimes we would like to calculate the probability
    of some event $A$ happening given that we know that some other
    event $B$ has happened. For this we need to first define the idea
    of conditional probability.  }
  \begin{definition}[Conditional probability]
    The probability of $A$ happening if we know that $B$ has happened
    is defined to be:
    \[
    P(A \mid B) \defn \frac{P(A \cap B) }{P(B)}.
    \]
  \end{definition}
  \only<1>{
    Conditional probabilities obey the same rules as probabilities. }
  \only<article>{
    Here, the probability measure of any event $A$ given $B$ is defined to be the probability of the intersection of of the events divided by the second event.
    We can rewrite this definition as follows, by using the definition for $P(B \mid A)$}
  \begin{block}{Bayes's theorem}
    For $P(A_1 \cup A_2)  = 1$, $A_1 \cap A_2 = \emptyset$,
    \[
    P(A_i \mid B)
    \uncover<2->{= \frac{P(B \mid A_i) P(A_i)}{P(B)}}
    \uncover<3->{= \frac{P(B \mid A_i) P(A_i)}{P(B \mid A_1) P(A_1) + P(B \mid A_2) P(A_2)}}
    \]
  \end{block}
  \uncover<4->{
    \begin{example}[probability of rain]
      What is the probability of rain given a forecast $x_1$ or $x_2$?
      \begin{columns}
        \begin{column}{0.33\textwidth}
          \begin{table}[H]
            \centering
            \begin{tabular}{c|c}
              $\outcome_1$: rain & $P(\outcome_1) = 80\%$ \\
              $\outcome_2$: dry & $P(\outcome_2) = 20\%$
            \end{tabular}
            \caption{Prior probability of rain tomorrow}
          \end{table}
        \end{column}
        
        \begin{column}{0.33\textwidth}
          \uncover<5->{
            \begin{table}[H]
              \centering
              \begin{tabular}{c|c}
                $x_1$: rain & $P(x_1 \mid \outcome_1) = 90\%$ \\
                $x_2$: dry & $P(x_2 \mid \outcome_2) = 50\%$
              \end{tabular}
              \caption{Probability the forecast is correct}
            \end{table}
          }
        \end{column}
        \uncover<6->{
          \begin{column}{0.33\textwidth}
            \begin{table}[H]
              \centering
              \begin{tabular}{c}
                $P(\outcome_1 \mid x_1) = 87.8\%$ \\
                $P(\outcome_1 \mid x_2) = 44.4\%$
              \end{tabular}
              \caption{Probability that it will rain given the forecast}
            \end{table}
          }
        \end{column}
      \end{columns}
    \end{example}
  }
\end{frame}


\begin{frame}
  \frametitle{Classification in terms of conditional probabilities}
  \only<1>{
    \only<presentation>{
      \begin{itemize}
      \item Features $x_t \in \CX$.
      \item Class label $y_t \in \CY$.
      \item Probability model $P_\model(x_t \mid y_t)$.
      \item Prior class probability $P_\model(y_t = c)$.
      \end{itemize}
    }
  }
  \only<article>{
    Conditional probability naturally appears in classification problems. Given a new example vector of data $x_t \in \CX$, we would like to calculate the probability of different classes $c \in \CY$ given the data, $P_\model(y_t = c \mid x_t)$.  
    If we somehow obtained the distribution of data $P_\model(x_t \mid y_t)$ for each possible class, as well as the prior class probability $P_\model(y_t = c)$, 
    from Bayes's theorem, we see that we can obtain the probability of the class:
  }
  \only<1>{
    \[
    P_\model(y_t = c \mid x_t) = \frac{P_\model(x_t \mid y_t = c) P_\model(y_t = c)}{\sum_{c' \in \CY} P_\model(x_t \mid y_t = c') P_\model(y_t = c')}
    \]
  }
  \only<article>{
    for any class $c$. This directly gives us a method for classifying new data, as long as we have a way to obtain $P_\model(x_t \mid y_t)$ and $P_\model(y_t)$.
  }
  \only<1>{
    \begin{figure}[H]
      \centering
      \begin{tikzpicture}
        \node[RV] at (0,0) (x) {$y_t$};
        \node[RV] at (0,2) (y) {$x_t$};
        \node[RV] at (1,1) (m) {$\model$};
        \draw[->] (x) to (y);
        \draw[->] (m) to (x); \draw[->] (m) to (y);
      \end{tikzpicture}
      \caption{A generative classification model. $\model$ identifies the model (paramter). $x_t$ are the features and $y_t$ the class label of the $t$-th example.}
    \end{figure}
  }
  \begin{figure}[H]
    \centering
    \begin{subfigure}{\fwidth}
      \includegraphics[width=\textwidth]{../figures/equal-variance}
      \caption{Equal prior and variance}
    \end{subfigure}
    \begin{subfigure}{\fwidth}
      \includegraphics[width=\textwidth]{../figures/unequal-variance}
      \caption{Unequal variance}
    \end{subfigure}
    \begin{subfigure}{\fwidth}
      \includegraphics[width=\textwidth]{../figures/unequal-prior}
      \caption{Unequal prior}
    \end{subfigure}
    \caption{The effect of changing variance and prior on the classification decision when we assume a normal distribution.}
    \label{fig:normal-generative}
  \end{figure}
  \begin{example}[Normal distribution]
    A simple example is when $x_t$ is normally distributed in a matter that depends on the class.  Figure~\ref{fig:normal-generative} shows the distribution of $x_t$ for two different classes, with means of $-1$ and $+1$ respectively, for three different case. In the first case, both classes have variance of 1, and we assume the same prior probability for both
    \[
    x_t \mid y_t = 0 \sim \Normal(-1,1),
    \qquad
    x_t \mid y_t = 1 \sim \Normal(1,1)
    \]
    \[
    x_t \mid y_t = 0 \sim \Normal(-1,1),
    \qquad
    x_t \mid y_t = 1 \sim \Normal(1,1)
    \]

  \end{example}
  
  \uncover<5>{
    \alert{But how can we get a probability model in the first place?}
  }
\end{frame}


\begin{frame}
  \frametitle{Subjective probability}
  \only<article>{While probabilities apply to truly random events, they are also useful for representing subjective uncertainty. In this course, we will use a special symbol for subjective probability, $\bel$.}
  \begin{block}{Subjective probability measure $\bel$}
    \begin{itemize}
    \item If we think event $A$ is more likely than $B$, then $\bel(A) > \bel(B)$.
    \item Usual rules of probability apply:
      \begin{enumerate}
      \item $\bel(A) \in [0,1]$.
      \item $\bel(\emptyset) = 0$.
      \item If $A \cap B = \emptyset$, then $\bel(A \cup B) = \bel(A) + \bel(B)$.
      \end{enumerate}
    \end{itemize}
  \end{block}
\end{frame}


\begin{frame}
  \frametitle{Bayesian inference illustration}
\only<article>{
Here is a simple example to illustrate the idea of Bayesian inference. Imagine you are a rat in the maze and you need to get to the cheese. In the example below, you believe that the maze lay out is only one of two possibilities $\model_1, \model_2$. Initially you might believe that the first model is more likely than the second, so you decide upon some function $\bel$ so that $\bel(\model_1) > \bel(\model_2)$, for example $\bel(\model_1) = 2/3$, $\bel(\model_2) = 1/3$. Now you start acting in the maze, collecting data $h$ consisting of your history of observation in this maze. The data gives a different amount of \emph{evidence} for each model of the world, which we can write as $P_\model(h)$. In the end, we want to combine this evidence with our initial belief to arrive at a conclusion. In the Bayesian setting, all of these quantities are probabilities, and the conclusion is simply the conditional probability of the model given the data.}
  \begin{columns}
    \begin{column}{0.7\textwidth}
      \begin{block}{Use a subjective belief $\bel(\model)$ on $\Model$}
        \begin{itemize}
        \item<1-> \alert{Prior} belief $\bel(\model)$ represents our initial uncertainty.
        \item<2-> We \alert{observe history} $h$.
        \item<3->Each possible $\model$ assigns a \alert{probability} $P_\model(h)$ to $h$.
        \item<4-> We can use this to \alert{update} our belief via Bayes' theorem to obtain the \alert{posterior} belief:
          \[
          \bel(\model \mid h) \propto P_\model(h) \bel(\model)
          \tag{conclusion = evidence $\times$ prior}
          \]
        \end{itemize}
      \end{block}
    \end{column}
    \begin{column}{0.3\textwidth}
      \centering
      \uncover<1->{\includegraphics[width=0.5\fwidth]{../figures/rl_worlds}
        \\
        prior
      }
      \\
      \uncover<2->{\includegraphics[width=0.5\fwidth]{../figures/rl_observations}
        \\
        evidence
      }
      \\
      \uncover<4->{\includegraphics[width=0.5\fwidth]{../figures/rl_worlds2}
        \\ 
        conclusion
      }
    \end{column}
  \end{columns}
\end{frame}




\subsection{Probability and Bayesian inference}
\only<article>{One of the most important methods in machine learning
  and statistics is that of Bayesian inference.  This is the most
  fundamental method of drawing conclusions from data and explicit
  prior assumptions. In Bayesian inference, prior assumptions are
  represented as a probabilities on a space of hypotheses. Each
  hypothesis is seen as a probabilistic model of all possible data
  that we can see.}

\only<article>{Frequently, we want to draw conclusions from data. However, the conclusions are never solely inferred from data, but also depend on prior assumptions about reality.}




\begin{frame}
  \frametitle{Some examples}

  \begin{example}
    John claims to be a medium. He throws a coin $n$ times and predicts its value always correctly. Should we believe that he is a medium?
    \begin{itemize}
    \item $\model_1$: John is a medium.
    \item $\model_0$: John is not a medium.
    \end{itemize}
  \end{example}
  The answer depends on what we \alert{expect} a medium to be able to do, and how likely we thought he'd be a medium in the first place.

  \only<article>{
    \begin{example}
      Traces of DNA are found at a murder scene. We perform a DNA test against a database of $10^4$ citizens registered to be living in the area. We know that the probability of a false positive (that is, the test finding a match by mistake) is $10^{-6}$. If there is a match in the database, does that mean that the citizen was at the scene of the crime?
    \end{example}
  }
\end{frame}





\begin{frame}
  \frametitle{Bayesian inference}
  \only<article>{
    Now let us apply this idea to our specific problem. We already have the probability of the observation for each model, but we just need to define a \emph{prior probability} for each model. Since this is usually completely subjective, we give it another symbol.
  }
  \only<article>{
    \begin{block}{Prior probability}
      The prior probability $\bel$ on a set of models $\Model$ specifies our subjective belief $\bel(\model)$ that each model is true.\footnote{More generally $\bel$ is a probability measure.}
    \end{block}
  }
  \only<article>{
    This allows us to calculate the probability of John being a medium, given the data:
    \[
    \bel(\model_1 \mid \bx) = \frac{\Pr(\bx \mid \model_1) \bel(\model_1)}{\Pr_\bel(\bx)},
    \]
    where
    \[
    \Pr_\bel(\bx) \defn \Pr(\bx \mid \model_1) \bel(\model_1) + \Pr(\bx \mid \model_0) \bel(\model_0).
    \]
    The only thing left to specify is $\bel(\model_1)$, the probability that John is a medium before seeing the data. This is our subjective prior belief that mediums exist and that John is one of them.
    More generally, we can think of Bayesian inference as follows: }
\only<article>{
  \begin{itemize}
  \item<1-> We start with a set of mutually exclusive models $\Model = \{\model_1, \ldots, \model_k\}$.
  \item<2->Each model $\model$ is represented by a specific probabilistic model for any possible data $x$, that is
    $P_\model(x) \equiv \Pr(x \mid \model)$.
  \item<3-> For each model, we have a prior probability $\bel(\model)$ that it is correct.
  \item<4-> After observing the data, we can calculate a posterior probability that the model is correct:
    \[
    \bel(\model \mid x) = \frac{\Pr(x \mid \model) \bel(\model)}{\sum_{\model' \in \Model} \Pr(x \mid \model') \bel(\model')}
    = \frac{P_\model(x) \bel(\model)}{\sum_{\model' \in \Model} P_{\model'} (x) \bel(\model')}.
    \]
  \end{itemize}
}
\only<presentation>{
  \begin{block}{Family of models $\Model = \{\model_1, \ldots, \model_k\}$}
    Defines a family of probabilities for \alert{any} data $x$:
    \[
     \{P_\model | \model \in \Model\}, \qquad P_\model(x) \equiv \Pr(x \mid \model).
    \]
  \end{block}
  \only<2->{
    \begin{block}{Prior belief $\bel$ over models $\Model$}
      $\bel(\model)$ is a \alert{distribution} how much we believe it is correct before seeing any data.
    \end{block}
  }
  \only<3->{
    \begin{block}{Posterior belief $\bel(\model \mid x)$ over models after seeing $x$}
      \[
      \bel(\model \mid x)
= \frac{P_\model(x) \bel(\model)}{\sum_{\model' \in \Model} P_{\model'} (x) \bel(\model')}
\only<4->{
  = \frac{\Pr(x \mid \model) \bel(\model)}{\sum_{\model' \in \Model} \Pr(x \mid \model') \bel(\model')}.
}
       \]
    \end{block}
  }
}
  \only<5->{
    \begin{exampleblock}{Interpretation}
      \begin{itemize}
      \item $\CM$: Set of all possible models that could describe the data.
      \item $P_\model(x)$: Probability of $x$ under model $\model$.
      \item Alternative notation $\Pr(x \mid \model)$: Probability of $x$ given that model $\model$ is correct.
      \item $\bel(\model)$: Our belief, before seeing the data, that $\model$ is correct.
      \item $\bel(\model \mid x)$: Our belief, aftering seeing the data, that $\model$ is correct.
      \end{itemize}
    \end{exampleblock}
    \only<article>{It must be emphasized that $P_\model(x) = \Pr(x \mid \model)$ as they are simply two different notations for the same thing. In words the first can be seen as the probability that model $\model$ assigns to data $x$, while the second as the probability of $x$ if $\model$ is the true model.}
  }
  \only<article>{
    Combining the prior belief with evidence is key in this procedure. Our posterior belief can then be used as a new prior belief when we get more evidence.}
\end{frame}
\begin{frame}
  \begin{exercise}[Continued example for medium]
    \only<article>{ Now let us apply this idea to our specific
      problem. We first make an independence assumption. In particular, we can assume that success and failure comes from a Bernoulli distribution with a parameter depending on the model.}
    \begin{align}
      P_{\model} (x) &= \prod_{t=1}^n P_{\model} (x_t).
                       \tag{independence property}
    \end{align}
    \only<article>{We first need to specify how well a medium could predict. Let's assume that a true medium would be able to predict perfectly, and that a non-medium would only predict randomly. This leads to the following models:}
    \uncover<2->{
    \begin{align}
      P_{\model_1}(x_t = 1) &= 1, &P_{\model_1}(x_t = 0) &= 0.
                                                           \tag{true medium model}
      \\
      P_{\model_0}(x_t = 1) &= 1/2, &P_{\model_0}(x_t = 0) &= 1/2.
                                                             \tag{non-medium model}
    \end{align}
  }
    \only<article>{
      The only thing left to specify is $\bel(\model_1)$, the probability
      that John is a medium before seeing the data. This is our
      subjective prior belief that mediums exist and that John is one of
      them.}
    \uncover<3->{
      \begin{align}
        \bel(\model_0) &= 1/2,   &  \bel(\model_1) &= 1/2.
                                                     \tag{prior belief}
      \end{align}
    }
    \only<article>{Combining the prior belief with evidence is key in this
      procedure. Our posterior belief can then be used as a new prior
      belief when we get more evidence.  }
    \uncover<4>{
      \begin{align}
        \bel(\model_1 \mid x) & = \frac{P_{\model_1}(x)
                                \bel(\model_1)}{\Pr_\bel(x)} \tag{posterior belief}
        \\
        \Pr_\bel(x) &\defn P_{\model_1}(x) \bel(\model_1) + P_{\model_0}(x) \bel(\model_0).
                      \tag{marginal distribution}
      \end{align}
    }
    If a classmate correctly predicts 4 coin tosses, what is your belief they are a medium? 
  \end{exercise}
\end{frame}


\begin{frame}
  \frametitle{Sequential update of beliefs}
  \only<article>{Assume you have $n$ meteorologists. At each day $t$, each meteorologist $i$ gives a probability $p_{t,\model_i}\defn P_{\model_i}(x_t = \textrm{rain})$ for rain. Consider the case of there being three meteorologists, and each one making the following prediction for the coming week. Start with a uniform prior $\bel(\model) = 1/3$ for each model.}
  {
    \begin{table}[h]
      \begin{tabular}{c|l|l|l|l|l|l|l}
        &M&T&W&T&F&S&S\\
        \hline
        CNN & 0.5 & 0.6 & 0.7 & 0.9 & 0.5 & 0.3 & 0.1\\
        SMHI & 0.3 & 0.7 & 0.8 & 0.9 & 0.5 & 0.2 & 0.1\\
        YR & 0.6 & 0.9 & 0.8 & 0.5 & 0.4 & 0.1 & 0.1\\
        \hline
        Rain? & Y & Y & Y & N & Y & N & N
      \end{tabular}
      \caption{Predictions by three different entities for the probability of rain on a particular day, along with whether or not it actually rained.}
      \label{tab:meteorologists}
    \end{table}
  }
  \begin{exercise}
    \begin{itemize}
    \item $n$ meteorological stations $\cset{\mdp_i}{i=1, \ldots,n}$
    \item The $i$-th station predicts rain $P_{\mdp_i}(x_t \mid x_1, \ldots, x_{t-1})$.
    \item Let $\bel_t(\mdp)$ be our belief at time $t$.
      Derive the next-step belief
      $\bel_{t+1}(\mdp) \defn  \bel_t(\mdp | x_{t})$ in terms of the current belief $\bel_t$.
    \item Write a python function that computes this posterior
    \end{itemize}
  \end{exercise}
  \uncover<2->{
    \[
    \bel_{t+1}(\mdp)
    \defn
    \bel_t(\mdp | x_{t})
    =
    \frac{P_\mdp(x_t \mid x_1, \ldots, x_{t-1}) \bel_t(\mdp)}
    {\sum_{\mdp'} P_{\mdp'}(x_t \mid x_1, \ldots, x_{t-1}) \bel_t(\mdp')}
    \]
  }
\end{frame}



\begin{frame}[label=beta-example]
  \frametitle{Bayesian inference for Bernoulli distributions}
  \only<1>{
    \begin{block}{Estimating a coin's bias}
      A fair coin comes heads $50\%$ of the time. 
      We want to test an unknown coin, which we think may not be completely fair. 
    \end{block}
  }
  \only<1,2>{
    \begin{figure}[h]
      \centering
      \includegraphics[width=\textwidth]{../figures/beta-prior}
      \caption{Prior belief $\bel$ about the coin bias $\theta$.}
    \end{figure}
  }
  \only<2>{
    For a sequence of throws $x_t \in \{0,1\}$,
    \[
    P_\theta(x) \propto \prod_t \theta^{x_t} (1 - \theta)^{1 - x_t}
    = \theta^{\textrm{\#Heads}} (1 - \theta)^{\textrm{\#Tails}}
    \]
  }
  \only<3>{
    \begin{figure}[h]
      \centering
      \includegraphics[width=\textwidth]{../figures/beta-likelihood}
      \caption{Prior belief $\bel$ about the coin bias $\theta$ and likelihood of $\theta$ for the data.}
    \end{figure}
    Say we throw the coin 100 times and obtain 70 heads. Then we plot the \alert{likelihood} $P_\theta(x)$ of different models.
  }
  \only<4>{
    \begin{figure}[h]
      \centering
      \includegraphics[width=\textwidth]{../figures/beta-posterior}
      \caption{Prior belief $\bel(\theta)$ about the coin bias $\theta$, likelihood of $\theta$ for the data, and posterior belief $\bel(\theta \mid x)$}
    \end{figure}
    From these, we calculate a \alert{posterior} distribution over the correct models. This represents our conclusion given our prior and the data.
  }
  \only<article>{If the prior distribution is described by the so-called Beta density
    \[
    f(\theta \mid \alpha, \beta) \propto \theta^{\alpha -1} (1 - \theta)^{\beta -1}
    \]
    where $\alpha, \beta$ describe the shape of the Beta distribution.
  }
\end{frame}

\subsubsection{Riemann and Lebesgue integrals}
\only<article>{
  Since $\bel$ is a probability measure over models $\Param$, it is always simple to write the posterior probability given some data $x$ in the following terms when $\Param$ is discrete (finite or countable):
  \[
  \bel(\param \mid x) = \frac{P_\param(x) \bel(\param)}{\sum_{\param'} P_{\param'(x) \bel(\param')}.}
  \]
  However, in many situations $\Param$ is uncountable, i.e. $\Param \subset \Reals^k$. Then, as both the prior $\bel$ and the posterior $\bel(\cdot \mid x)$ have to be probability measures on $\Param$, they are defined over subsets of $\Param$. This means that we can write the posterior in terms of Lebesgue integrals:
  \[
  \bel(B \mid x) = \frac{\int_B P_\param(x) \dd \bel(\param)}{\int_\Param P_\param(x) \dd \bel(\param)}.
  \]
  Alternatively, we can abuse notation and use $\bel(\param)$ to describe a density, so that the \emindex{posterior density} is written in terms of a Riemann integral:
  \[
  \bel(\param \mid x) \defn \frac{P_\param(x) \bel(\param) \dd \param}{\int_{\Param} P_{\param'}(x) \bel(\param') \param'}.
  \]
  However, the advantage of the Lebesgue integral notation is that it works the same no matter whether $\Param$ is discrete or continuous.
}

\subsection{Distributions and information theory}
It is very useful to define a divergence between probability measures. For simplicity, we give those first in discrete outcomes, before the general case.
\begin{equation}
  D(P\|Q) \defn \sum_{\omega \in \Omega} P(\omega) \ln \frac{P(\omega)}{Q(\omega)}
  \label{eq:kl-divergence}
\end{equation}
For the general case, where $P, Q$ are probability measures on $(\Omega, \Sigma)$, we can write this as $D(P\|Q) \defn \int_{\Omega} dP(\omega) \ln \frac{dP(\omega)}{dQ(\omega)}$.
Another common divergence is the total variation,
\begin{equation}
  \|P - Q\|_{TV} = \frac{1}{2}\|P - Q\|_1 \frac{1}{2} \sum_{\omega \in \Omega} |P(\omega) - Q(\omega)|.
\end{equation}
For the general case, this is defined as
\[
  \|P - Q\|_{TV} = \sup_{A \in \Sigma} |P(A) - Q(A)|.
\]
Those divergences are related by \emindex{Pinsker's inequality}
\begin{equation}
  \label{eq:pinsker}
  \|P - Q\|_{TV} \leq \sqrt{\frac{1}{2} D(P\|Q).
\end{equation}

Both of these are special cases of $f$-divergences, defined as:
\begin{equation}
  D_f(P\|Q) \defn \E_Q[f(dP/dQ)].
  \label{eq:f-divergence}
\end{equation}
It is easy to see that the KL-divergence is obtained by
$f(t) = t \ln t$ and the total variation by
$f(t) = \frac{1}{2}|t - 1|$.

Any divergence between distributions satisfies the information- processing inequality. In particular, if $P, Q$ are measures on $\Omega$, and we define a new random variable $f : \Omega \to X$ with corresponding probability measures $P_f, Q_f$ so that $P_f(A) = P(\cset{\omega \in \Omega}{f(\omega) \in A}$ then we have that
\begin{equation}
  \label{eq:inf-proc}
  D(P_f\|Q_f) \leq D(P \| Q).
\end{equation}

\begin{frame}
  \frametitle{Learning outcomes}
  \begin{block}{Understanding}
    \begin{itemize}
    \item The axioms of probability, marginals and conditional distributions.
    \item The philosophical underpinnings of Bayesianism.
    \item The simple conjugate model for Bernoulli distributions.
    \end{itemize}
  \end{block}
  
  \begin{block}{Skills}
    \begin{itemize}
    \item Be able to calculate with probabilities using the marginal and conditional definitions and Bayes rule.
    \item Being able to implement a simple Bayesian inference algorithm in Python.
    \end{itemize}
  \end{block}

  \begin{block}{Reflection}
    \begin{itemize}
    \item How useful is the Bayesian representation of uncertainty?
    \item How restrictive is the need to select a prior distribution?
    \item Can you think of another way to explicitly represent uncertainty in a way that can incorporate new evidence?
    \end{itemize}
  \end{block}
  
\end{frame}

%%% Local Variables:
%%% mode: latex
%%% TeX-engine: xetex
%%% TeX-master: "notes.tex"
%%% End:
 % Bayesian inference
\subsection{Statistical testing\textsuperscript{*}}
\only<article>{A common type of decision problem is a statistical test. This arises when we have a set of possible candidate models $\CM$ and we need to be able to decide which model to select after we see the evidence.
  Many times, there is only one model under consideration, $\model_0$, the so-called \alert{null hypothesis}. Then, our only decision is whether or not to accept or reject this hypothesis.}
\begin{frame}
  \frametitle{Simple hypothesis testing}
  \only<article>{Let us start with the simple case of needing to compare two models.}
  \begin{block}{The simple hypothesis test as a decision problem}
    \begin{itemize}
    \item $\CM = \{\model_0, \model_1\}$
    \item $a_0$: Accept model $\model_0$
    \item $a_1$: Accept model $\model_1$
    \end{itemize}
    \begin{table}[H]
      \begin{tabular}{c|cc}
        $\util$& $\model_0$& $\model_1$\\\hline
        $a_0$ & 1 & 0\\
        $a_1$ & 0 & 1
      \end{tabular}
      \caption{Example utility function for simple hypothesis tests.}
    \end{table}
    \only<article>{There is no reason for us to be restricted to this utility function. As it is diagonal, it effectively treats both types of errors in the same way.}
  \end{block}

  \begin{example}[Continuation of the medium example]
    \begin{itemize}
    \item $\model_1$: that John is a medium.
    \item $\model_0$: that John is not a medium.
    \end{itemize}
    \only<article>{
      Let $x_t$ be $0$ if John makes an incorrect prediction at time $t$ and $x_t = 1$ if he makes a correct prediction. Let us once more assume a Bernoulli model, so that John's claim that he can predict our tosses perfectly means that for a sequence of tosses $\bx = x_1, \ldots, x_n$,
      \[
        P_{\model_1}(\bx) = \begin{cases}
          1, & x_t = 1 \forall t \in [n]\\
          0, & \exists t \in [n] : x_t = 0.
        \end{cases}
      \]
      That is, the probability of perfectly correct predictions is 1, and that of one or more incorrect prediction is 0. For the other model, we can assume that all draws are independently and identically distributed from a fair coin. Consequently, no matter what John's predictions are, we have that:
      \[
        P_{\model_0}(\bx = 1 \ldots 1) = 2^{-n}.
      \]
      So, for the given example, as stated, we have the following facts:
      \begin{itemize}
      \item If John makes one or more mistakes, then $\Pr(\bx \mid \model_1) = 0$ and $\Pr(\bx \mid \model_0) = 2^{-n}$. Thus, we should perhaps say that then John is not a medium
      \item If John makes no mistakes at all, then 
        \begin{align}
          \Pr(\bx = 1, \ldots, 1 \mid \model_1) &= 1,
          &
            \Pr(\bx = 1, \ldots, 1 \mid \model_0) &= 2^{-n}.
        \end{align}
      \end{itemize}
      Now we can calculate the posterior distribution, which is
      \[
        \bel(\model_1 \mid \bx = 1, \ldots, 1) = \frac{1 \times \bel(\model_1)}{1 \times \bel(\model_1) + 2^{-n} (1 - \bel(\model_1))}.
      \]
      Our expected utility for taking action $a_0$ is actually
    }
    \[
      \E_\bel(\util \mid a_0) = 1 \times \bel(\model_0 \mid \bx) + 0 \times \bel(\model_1 \mid \bx), \qquad
      \E_\bel(\util \mid a_1) = 0 \times \bel(\model_0 \mid \bx) + 1 \times \bel(\model_1 \mid \bx)
    \]
  \end{example}
  
\end{frame}


\begin{frame}
  \frametitle{Null hypothesis test}
    \index{Null-Hypothesis test}
  Many times, there is only one model under consideration, $\model_0$, the so-called \alert{null hypothesis}. \only<article>{ This happens when, for example, we have no simple way of defining an appropriate alternative. Consider the example of the medium: How should we expect a medium to predict? Then, our only decision is whether or not to accept or reject this hypothesis.}
  \begin{block}{The null hypothesis test as a decision problem}
    \begin{itemize}
    \item $a_0$: Accept model $\model_0$
    \item $a_1$: Reject model $\model_0$
    \end{itemize}
  \end{block}

  \begin{example}{Construction of the test for the medium}
    \index{policy!for statistical testing}
    \begin{itemize}
    \item<2-> $\model_0$ is simply the $\Bernoulli(1/2)$ model: responses are by chance.
    \item<3-> We need to design a policy $\pol(a \mid \bx)$ that accepts or rejects depending on the data.
    \item<4-> Since there is no alternative model, we can only construct this policy according to its properties when $\model_0$ is true.
    \item<5-> In particular, we can fix a policy that only chooses $a_1$ when $\model_0$ is true a proportion $\delta$ of the time.
    \item<6-> This can be done by construcing a threshold test from the inverse-CDF.
    \end{itemize}
  \end{example}
\end{frame}
\begin{frame}
  \frametitle{Using $p$-values to construct statistical tests}
  \begin{definition}[Null statistical test]
    \only<article>{
      A statistical test $\pol$ is a decision rule for accepting or rejecting a hypothesis on the basis of evidence. A $p$-value test rejects a hypothesis whenever the value of the statistic $f(x)$ is smaller than a threshold.}
    The statistic $f : \CX \to [0,1]$ is  designed to have the property:
    \[
      P_{\model_0}(\cset{x}{f(x) \leq \delta}) = \delta.
    \]
    If our decision rule is:
    \[
      \pol(a \mid x) =
      \begin{cases}
        a_0, & f(x) \geq \delta\\
        a_1, & f(x) < \delta,
      \end{cases}
    \]
    the probability of rejecting the null hypothesis when it is true is exactly $\delta$.
  \end{definition}
  \only<presentation>{The value of the statistic $f(x)$, otherwise known as the \alert{$p$-value}, is uninformative.}
  \only<article>{This is because, by definition, $f(x)$ has a uniform distribution under $\model_0$. Hence the value of $f(x)$ itself is uninformative: high and low values are equally likely. In theory we should simply choose $\delta$ before seeing the data and just accept or reject based on whether $f(x) \leq \delta$. However nobody does that in practice, meaning that $p$-values are used incorrectly. Better not to use them at all, if uncertain about their meaning.}
\end{frame}
\begin{frame}
  \begin{block}{Issues with $p$-values}
    \begin{itemize}
    \item They only measure quality of fit \alert{on the data}.
    \item Not robust to model misspecification. \only<article>{For example, zero-mean testing using the $\chi^2$-test has a normality assumption.}
    \item They ignore effect sizes. \only<article>{For example, a linear analysis may determine that there is a significant deviation from zero-mean, but with only a small effect size of 0.01. Thus, reporting only the $p$-value is misleading}
    \item They do not consider prior information. 
    \item They do not represent the probability of having made an error. \only<article>{In particular, a $p$-value of $\delta$ does not mean that the probability that the null hypothesis is false given the data $x$, is $\delta$, i.e. $\delta \neq \Pr(\neg \model_0 \mid x)$.}
    \item The null-rejection error probability is the same irrespective of the amount of data (by design).
    \end{itemize}
  \end{block}
\end{frame}

\begin{frame}
  \only<article>{Let us consider the example of the medium.}
  \begin{block}{$p$-values for the medium example}
    \begin{itemize}
    \item<2->$\model_0$ is simply the $\Bernoulli(1/2)$ model:
      responses are by chance. 
    \item<3->CDF: $P_{\model_0}(N \leq n \mid K = 100)$ \only<article> {is the probability of at most $N$ successes if we throw the coin 100 times. This is in fact the cumulative probability function of the binomial distribution. Recall that the binomial represents the distribution for the number of successes of independent experiments, each following a Bernoulli distribution.}
    \item<4->ICDF:  the number of successes that will happen with probability at least $\delta$
    \item<5->e.g. we'll get at most 50 successes a proportion $\delta = 1/2$ of the time.
    \item<6>Using the (inverse) CDF we can construct a policy $\pol$ that selects $a_1$ when $\model_0$ is true only a $\delta$ portion of the time, for any choice of $\delta$.
    \end{itemize}
  \end{block}
  \begin{columns}
    \setlength\fheight{0.33\columnwidth}
    \setlength\fwidth{0.33\columnwidth}
    \begin{column}{0.5\textwidth}
      \only<3,4,5,6>{\input{../figures/binomial-cdf.tikz}}      
    \end{column}
    \begin{column}{0.5\textwidth}
      \only<4,5,6>{\input{../figures/binomial-icdf.tikz}}
    \end{column}
  \end{columns}    
\end{frame}



\begin{frame}
  \frametitle{Constructing a Null-Hypothesis test with frequentist properties}
  \index{Null-Hypothesis test!frequentist}
  \begin{block}{The test statistic}
    We want the test to reflect that we don't have a significant number of failures.
    \[
      f(x) = 1 - \textrm{binocdf}(\sum_{t=1}^n x_t, n, 0.5)
    \]
  \end{block}
  \begin{alertblock}{What $f(x)$ is and is not}
    \begin{itemize}
    \item It is a \textbf{statistic} which is $\leq \delta$ a $\delta$ portion of the time when $\model_0$ is true.
    \item It is \textbf{not} the probability of observing $x$ under $\model_0$.
    \item It is \textbf{not} the probability of $\model_0$ given $x$.
    \end{itemize}
  \end{alertblock}
\end{frame}
\begin{frame}
  \begin{exercise}
    \begin{itemize}
    \item<1-> Let us throw a coin 8 times, and try and predict the outcome.
    \item<2-> Select a $p$-value threshold so that $\delta = 0.05$. 
      For 8 throws, this corresponds to \uncover<3->{$ > 6$ successes or $\geq 87.5\%$ success rate}.
    \item<3-> Let's calculate the $p$-value for each one of you
    \item<4-> What is the rejection performance of the test?
    \end{itemize}
    \setlength\fheight{0.25\columnwidth}
    \setlength\fwidth{0.5\columnwidth}
    \only<2,3>{
      \begin{figure}[H]
        \centering
        \input{../figures/p-value-example-rejection-threshold.tikz}
        \caption{Here we see how the rejection threshold, in terms of the success rate, changes with the number of throws to achieve an error rate of $\delta = 0.05$.}
      \end{figure}
      \only<article>{As the amount of throws goes to infinity, the threshold converges to $0.5$. This means that a statistically significant difference from the null hypothesis can be obtained, even when the actual model from which the data is drawn is only slightly different from 0.5.}
    }
    \only<4>{
      \begin{figure}[H]
        \centering
        \input{../figures/p-value-example-rejection.tikz}
        \caption{Here we see the rejection rate of the null hypothesis ($\model_0$) for two cases. Firstly, for the case when $\model_0$ is true. Secondly, when the data is generated from $\Bernoulli(0.55)$.}
      \end{figure}
      \only<article>{As we see, this method keeps its promise: the null is only rejected 0.05 of the time when it's true. We can also examine how often the null is rejected when it is false... but what should we compare against? Here we are generating data from a $\Bernoulli(0.55)$ model, and we can see the rejection of the null increases with the amount of data. This is called the \alert{power} of the test with respect to the $\Bernoulli(0.55)$ distribution. }
    }
  \end{exercise}
\end{frame}

\begin{frame}
  \begin{alertblock}{Statistical power and false discovery.}
    Beyond not rejecting the null when it's true, we also want:
    \begin{itemize}
    \item High power: Rejecting the null when it is false.
    \item Low false discovery rate: Accepting the null when it is true.
    \end{itemize}
  \end{alertblock}
  \begin{block}{Power}
    The power depends on what hypothesis we use as an alternative.
    \only<article>{This implies that we cannot simply consider a plain null hypothesis test, but must formulate a specific alternative hypothesis. }
  \end{block}

  \begin{block}{False discovery rate}
    False discovery depends on how likely it is \alert{a priori} that the null is false.
    \only<article>{This implies that we need to consider a prior probability for the null hypothesis being true.}
  \end{block}

  \only<article>{Both of these problems suggest that a Bayesian approach might be more suitable. Firstly, it allows us to consider an infinite number of possible alternative models as the alternative hypothesis, through Bayesian model averaging. Secondly, it allows us to specify prior probabilities for each alternative. This is especially important when we consider some effects unlikely.}
\end{frame}

\begin{frame}
  \frametitle{The Bayesian Null-Hypothesis test}
  \index{Null-Hypothesis test!Bayesian}
  \begin{example}
    \only<article>{In this example, we construct a null-hypothesis test in the Bayesian framework. This requires that we select the observation and action space and utility function. First, we consider only two models. In the first $\model_0$, we assume that coin tosses are fair. The alternative hypothesis $\model_1$ is that coin tosses are from some Bernoulli distribution with an unknown parameter $\theta$.}
    \begin{enumerate}
    \item $\model_0$: $\Bernoulli(1/2)$. \only<article>{This is the
        same as the null hypothesis test.}
    \item $\model_1$: $\Bernoulli(\theta)$,
      $\theta \sim \Uniform([0,1])$. \only<article>{This is an
        extension of the simple hypothesis test, with an alternative
        hypothesis that says ``the data comes from an arbitrary
        Bernoulli model''.}
    \item Actions $a_0, a_1$ indicating whether we accept $\model_0$ or $\model_1$ respectively.
    \item Set $\util(a_i, \model_j) = \ind{i =
        j}$.
      \only<article>{This choice makes sense if we care equally about
        either type of error.}
    \item Set $\bel(\model_i) = 1/2$. \only<article>{Here we place an
        equal probability in both models.}
    \item Calculate $\bel(\model \mid x)$.
    \item Choose action $a_i$, where $i = \argmax_{j} \E[\util | a = a_j]$.
      \only<article>{
        In this case, $\argmax_j \bel(\model_j \mid x) = \argmax_{j} \E[\util | a = a_j]$
        because $\E[\util | a = a_j] = \sum_i \bel(\model_i \mid x) \util(\model_i, a_j)$
      }
    \end{enumerate}
    \label{ex:bayesian-compound-hypothesis-test}
  \end{example}
  \begin{block}{Bayesian model averaging for the alternative model $\model_1$}
    \only<article>{In this scenario, $\model_0$ is a simple point model, e.g. corresponding to a $\Bernoulli(1/2)$. However $\model_1$ is a marginal distribution integrated over many models, e.g. a $Beta$ distribution over Bernoulli parameters.}
    \begin{align}
      P_{\model_1}(x) &= \int_\Param B_{\param}(x) \dd \beta(\param) \\
      \bel(\model_0 \mid x) &= \frac{P_{\model_0}(x) \bel(\model_0)}
                              {P_{\model_0}(x) \bel(\model_0) + P_{\model_1}(x) \bel(\model_1)}
    \end{align}
  \end{block}
\end{frame}
\begin{frame}
  \only<1>{
    \begin{figure}[H]
      \centering
      \input{../figures/p-value-example-posterior.tikz}
      \caption{Here we see the convergence of the posterior probability.}
    \end{figure}
    \only<article>{As can be seen in the figure above, in both cases, the posterior converges to the correct value, so it can be used to indicate our confidence that the null is true.}
  }
  \only<2>{
    \begin{figure}[H]
      \centering
      \input{../figures/p-value-example-null-posterior.tikz}
      \caption{Comparison of the rejection probability for the null and the Bayesian test when $\model_0$ is true.}
    \end{figure}
    \only<article>{Now we can use this Bayesian test, with uniform prior, to see how well it performs. While the plain null hypothesis test has a fixed rejection rate of $0.05$, the Bayesian test's rejection rate converges to 0 as we collect more data.}
  }
  \only<3>{
    \begin{figure}[H]
      \centering
      \input{../figures/p-value-example-true-posterior.tikz}
      \caption{Comparison of the rejection probability for the null and the Bayesian test when $\model_1$ is true.}
    \end{figure}
    \only<article>{However, both methods are able to reject the null hypothesis more often when it is false, as long as we have more data.}
  }
\end{frame}

\begin{frame}
  \frametitle{Concentration inequalities and confidence intervals}
  \only<article>{
    There are a number of inequalities from probability theory that allow us to construct high-probability confidence intervals. The most well-known of those is Hoeffding's inequality \index{Hoeffding inequality|textbf}, the simplest version of which is the following:
    \begin{lemma}[Hoeffding's inequality]
      \label{lem:hoeffding}
      Let $x_1, \ldots, x_n$ be a series of random variables, $x_i \in [0, 1]$. Then it holds that, for the empirical mean:
      \[
        \mu_n \defn \frac{1}{n} sum_{t=1}^n x_t
      \]
      \begin{equation}
        \Pr(\mu_n \geq \E \mu_n+ \epsilon) \leq e^{-2n\epsilon^2}.
        \label{eq:hoeffding}
      \end{equation}
    \end{lemma}
    When $x_t$ are i.i.d, $\E \mu_n = \E x_t$. This allows us to construct an interval of size $\epsilon$ around the true mean. This can generalise to a two-sided interval:
    \[
      \Pr(|\mu_n - \E \mu_n| \geq \epsilon) \leq 2e^{-2n\epsilon^2}.
    \]
    We can also rewrite the equation to say that, with probability at least $1 - \delta$
    \[
      |\mu_n - \E \mu_n| \leq \sqrt{\frac{\ln 2/\delta}{2n}}
    \]
  }
\end{frame}
\begin{frame}
  \frametitle{Further reading}
  \begin{block}{Points of significance (Nature Methods)}
    \begin{itemize}
    \item Importance of being uncertain \url{https://www.nature.com/articles/nmeth.2613}
    \item Error bars \url{https://www.nature.com/articles/nmeth.2659}
    \item P values and the search for significance \url{https://www.nature.com/articles/nmeth.4120}
    \item Bayes' theorem \url{https://www.nature.com/articles/nmeth.3335}
    \item Sampling distributions and the bootstrap \url{https://www.nature.com/articles/nmeth.3414}
    \end{itemize}
  \end{block}
\end{frame}



%%% Local Variables:
%%% mode: latex
%%% TeX-master: "notes"
%%% End:
 % Missing
\section{Formalising classification problems}
\only<article>{
  One of the simplest decision problems is classification. At the simplest level, this is the problem of observing some data point $x_t \in \CX$ and making a decision about what class $\CY$ it belongs to. Typically, a fixed classifier is defined as a decision rule $\pi(a | x)$ making decisions $a \in \CA$, where the decision space includes the class labels, so that if we observe some point $x_t$ and choose $a_t = 1$, we essentially declare that $y_t = 1$.

  Typically, we wish to have a classification policy that minimises classification error.
}
\begin{frame}
  \frametitle{Deciding a class given a model}
  \only<article>{In the simplest classification problem, we observe some features $x_t$ and want to make a guess $\decision_t$ about the true class label $y_t$. Assuming we have some probabilistic model $P_\model(y_t \mid x_t)$, we want to define a decision rule $\pol(\decision_t \mid x_t)$ that is optimal, in the sense that it maximises expected utility for $P_\model$.}
  \begin{itemize}
  \item Features $x_t \in \CX$.
  \item Label $y_t \in \CY$.
  \item Decisions $\decision_t \in \CA$.
  \item Decision rule $\pol(\decision_t \mid x_t)$ assigns probabilities to actions.
  \end{itemize}
  
  \begin{block}{Standard classification problem}
    \only<article>{In the simplest case, the set of decisions we make are the same as the set of classes}
    \[
    \CA = \CY, \qquad
    U(\decision, y) = \ind{\decision = y}
    \]
  \end{block}

  \begin{exercise}
    If we have a model $P_\model(y_t \mid x_t)$, and a suitable $U$, what is the optimal decision to make?
  \end{exercise}
  \only<presentation>{
    \uncover<2->{
      \[
      \decision_t \in \argmax_{\decision \in \Decision} \sum_y P_\model(y_t = y \mid x_t) \util(\decision, y)
      \]
    }
    \uncover<3>{
      For standard classification,
      \[
      \decision_t \in \argmax_{\decision \in \Decision} P_\model(y_t = \decision \mid x_t)
      \]
    }
  }
\end{frame}


\begin{frame}
  \frametitle{Deciding the class given a model family}
  \begin{itemize}
  \item Training data $\Training = \cset{(x_i, y_i)}{i=1, \ldots, \ndata}$
  \item Models $\cset{P_\model}{\model \in \Model}$.
  \item Prior $\bel$ on $\Model$.
  \end{itemize}
  \only<article>{Similarly to our example with the meteorological stations, we can define a posterior distribution over models.}
  \begin{block}{Posterior over classification models}
    \[
    \bel(\model \mid \Training) = \frac{P_\model(y_1, \ldots, y_\ndata \mid
      x_1, \ldots, x_\ndata) \bel(\model)} {\sum_{\model' \in \Model}
      P_{\model'}(y_1, \ldots, y_\ndata \mid x_1, \ldots, x_\ndata)
      \bel(\model')}
    \]
    \only<article>{
      This posterior form can be seen as weighing each model according to how well they can predict the class labels. It is a correct form as long as, for every pair of models $\model, \model'$ we have that $P_\model(x_1, \ldots, x_\ndata) = P_{\model'}(x_1, \ldots, x_\ndata)$. This assumption can be easily satisfied without specifying a particular model for the $x$.}
    \only<2>{
      If not dealing with time-series data, we assume independence between $x_t$:
      \[
      P_\model(y_1, \ldots, y_\ndata \mid  x_1, \ldots, x_\ndata)
      = \prod_{i=1}^T P_\model(y_i \mid x_i)
      \]
    }
  \end{block}
  \uncover<3->{
    \begin{block}{The \alert{Bayes rule} for maximising $\E_\bel(\util \mid a, x_t, \Training)$}
      The decision rule simply chooses the action:
      \begin{align}
        \decision_t &\in
                      \argmax_{\decision \in \Decision}
                      \sum_{y}  \alert<4>{\sum_{\model \in
                      \Model}  P_\model(y_t = y \mid x_t) \bel(\model \mid
                      \Training)} 
                      \util(\decision, y)
                      \only<5>{
        \\ &=
             \argmax_{\decision \in \Decision}
             \sum_{y} \Pr_{\bel \mid \Training}(y_t = y \mid x_t) 
             \util(\decision, y)
             }
      \end{align}
    \end{block}
  }
  \uncover<4->{
    We can rewrite this by calculating the posterior marginal marginal label probability
    \[
    \Pr_{\bel \mid \Training}(y_t \mid x_t) \defn
    \Pr_{\bel}(y_t \mid x_t, \Training) = 
    \sum_{\model \in \Model} P_\model(y_t \mid x_t) \bel(\model \mid \Training).
    \]
  }

\end{frame}

\begin{frame}
  \frametitle{Approximating the model}
  \begin{block}{Full Bayesian approach for infinite $\Model$}
    Here $\bel$ can be a probability density function and 
    \[
    \bel(\model \mid \Training)  = P_\model(\Training)  \bel(\model)  / \Pr_\bel(\Training),
    \qquad
    \Pr_\bel(\Training) = \int_{\Model} P_\model(\Training)  \bel(\model)  \dd,
    \]
    can be hard to calculate.
  \end{block}
  \onslide<2->{
    \begin{block}{Maximum a posteriori model}
      \index{MAP inference}
      \index{Maximum a posteriori|see MAP inference}
      We only choose a single model through the following optimisation:
      \[
      \MAP(\bel, \Training) 
      \only<2>{
        = \argmax_{\model \in \Model} P_\model(\Training)  \bel(\model) 
      }
      \only<3>{
        = \argmax_{\model \in \Model}
        \overbrace{\ln P_\model(\Training)}^{\textrm{goodness of fit}}  + \underbrace{\ln \bel(\model)}_{\textrm{regulariser}}.
      }
      \]
      \only<article>{You can think of the goodness of fit as how well the model fits the training data, while the regulariser term simply weighs models according to some criterion. Typically, lower weights are used for more complex models.}
    \end{block}
  }
\end{frame}

\input{naive-bayes} % naive Bayes classifiers
\input{knn}  % knn, reproducability and bootstrapping
\input{ann} % linear models and stochastic gradient descent



\begin{frame}
  \frametitle{Learning outcomes}
  \begin{block}{Understanding}
    \begin{itemize}
    \item Preferences, utilities and the expected utility principle.
    \item Hypothesis testing and classification as decision problems.
    \item How to interpret $p$-values Bayesian tests.
    \item The MAP approximation to full Bayesian inference.
    \end{itemize}
  \end{block}
  
  \begin{block}{Skills}
    \begin{itemize}
    \item Being able to implement an optimal decision rule for a given utility and probability.
    \item Being able to construct a simple null hypothesis test.
    \end{itemize}
  \end{block}

  \begin{block}{Reflection}
    \begin{itemize}
    \item When would expected utility maximisation not be a good idea?
    \item What does a $p$ value represent when you see it in a paper?
    \item Can we prevent high false discovery rates when using $p$ values?
    \item When is the MAP approximation good?
    \end{itemize}
  \end{block}
  
\end{frame}






%%% Local Variables:
%%% mode: latex
%%% TeX-master: "notes"
%%% End:

