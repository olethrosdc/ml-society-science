\section{Reproducibility}

\begin{frame}
  \frametitle{Reproducibility}
  \begin{itemize}
  \item We use our machine learning algorithm $\alg$ on some data $x$ and get result $y = \alg(x)$.
  \item What does this result mean?
  \item Would we get the same result again?
  \end{itemize}
  \only<2>{\includegraphics[width=\textwidth]{../figures/2016-election}}
\end{frame}

\subsection{Confidence intervals and $p$-values}
\begin{frame}
  \frametitle{The fallacy of $p$-values}
  \only<article>{A $p$-value has a uniform distribution under the null hypothesis. More precisely,let $x$ is our data and $H_0$ is our null hypothesis. If $H_0$ is true then $x \sim P_0$. Let}
  \[
    f(x) \tag{$p$-statistic}
  \]
  \only<article>{be a statistic on the data designed to have the property:}
  \[
    P_0(\cset{x}{f(x) \leq p}) = p.
  \]
  \only<article>{This ensures that the probability of rejecting the null hypothesis when it is true is actually $p$. But note that this is the definition of the uniform distribution, so $f(x)$ has a uniform distribution under $H_0$. Hence the value of $f(x)$ itself is uninformative. In theory we should simply choose $p$ before seeing the data and just accept or reject based on whether $f(x) \leq p$. However nobody does that in practice, meaning that $p$-values are used incorrectly. Better not to use them at all.}
\end{frame}


\begin{frame}
  \frametitle{Confidence intervals}
  \only<article>{Similarly, a confidence interval $I$ tells us the probability of an underlying value $\theta$ to be estimated being in $\theta \in I$, if our modelling assumption is correct.}
  \begin{block}[Confidence interval under a model $\mu$]
    \only<article>{
      The interval $I(\delta)$ for an estimator $\hat{\theta}$ of a parameter $\theta$ from data $x$ under the modelling assumptions $\mu$ is simply a set in $\Theta$ such that the true parameter lies in $I(\delta)$ with probability $1 - \delta$, i.e.
    }
    \[
      \Pr_\mu(\hat{\theta} \in I(\theta, \delta)) \geq 1 - \delta.
    \]
  \end{block}
  This is a property of the estimator $\hat{\theta} : \CX \to \Theta$ and tells us \emph{nothing direct} about our specific estimate $\hat{\theta(x)}$. \only<article>{It only tells that if we repeatedly apply this produre, we will only get an estimate outside of the interval a $\delta$-fraction of the time.}
\end{frame}

\subsection{Bayesian credible intervals}


\subsection{Model mismatch}

\subsection{Boot-strapping}

\subsection{Cross-validation}

\subsection{Independent replication}



%%% Local Variables:
%%% mode: latex
%%% TeX-master: "notes"
%%% End:
