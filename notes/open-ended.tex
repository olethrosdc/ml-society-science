
\section{Simulation selection (Nov 6)}

Firstly, visit \url{https://ai.googleblog.com/2020/02/ml-fairness-gym-tool-for-exploring-long.html} and  \url{https://github.com/google/ml-fairness-gym} and start working on one of the dynamic environments there. In particular, I recommend you study the credit score environment. Using the environments requires the open-AI gym \url{https://gym.openai.com/}

Describe the variables into the following categories:
\begin{itemize}
\item Features $x_t$, that e.g. describe the patient $t$ in a medical database
\item Actions $a_t$ that describe whatever intervention has been taken, if applicable. There is always at least one action, even if not included in the dataset. E.g. for a diagnostic database, the action would be 'perform the diagnosis'.
\item Outcomes $y_t$ that describe what happens after the action has been taken.
\item The utility function $U$ that describes what  $y_t$ that describe what happens after the action has been taken.
\end{itemize}


\textbf{Simulation selection and analysis (Nov 6)} Here you should describe the simulation, and analyse the data it produces with a simple, default policy for making decisions.

\textbf{Improved policies (Nov 20)} Here you should come up with an algorithm that produces improved policies in some sense.

\textbf{Final report (Dec 6)} In the final report, make sure to communicate everything related to reproducibility, fairness and privacy of your methodology and the policy you have derived.







%%% Local Variables:
%%% mode: latex
%%% TeX-master: "open-ended-project"
%%% End:
