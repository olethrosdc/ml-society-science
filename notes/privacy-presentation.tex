\documentclass{beamer}


\mode<presentation>{
  % \useinnertheme{rectangles}
  \useoutertheme{infolines}
  % \usecolortheme{crane}
  % \usecolortheme{rose}
}
%% Pre-amble - commonly defined macros.

%% Packages
\usepackage{amsmath}
\usepackage{amsfonts}
\usepackage{amssymb}
\usepackage{amsbsy}
\usepackage{isomath}
\usepackage{amsthm}
\usepackage{dsfont}
%%\usepackage{theorem}
\usepackage{algorithm}
\usepackage{algorithmicx}
\usepackage{algpseudocode}
\usepackage{mathrsfs}
\usepackage{epsfig}
\usepackage{subcaption}
\usepackage{makeidx} 
\usepackage{colortbl}
\usepackage{enumerate}
\usepackage{multirow}
\usepackage{listings}
\usepackage{pgfplots}
\newlength\fheight
\newlength\fwidth
\only<presentation>{
\setlength\fheight{0.5\columnwidth}
\setlength\fwidth{0.5\columnwidth}
}
\only<article>{
\setlength\fheight{0.25\textwidth}
\setlength\fwidth{0.25\textwidth}
}
\usepackage[sort&compress,comma,super]{natbib}
\def\newblock{} % To avoid a compilation error about a function \newblock undefined
\usepackage{hyperref}

\setbeamertemplate{theorems}[numbered] 
\mode<presentation>{
\theoremstyle{plain}
\newtheorem{assumption}{Assumption}
\theoremstyle{definition}
\newtheorem{exercise}{Exercise}
\theoremstyle{remark}
\newtheorem{remark}{Remark}
}

\numberwithin{equation}{section} 
\mode<article>{
\theoremstyle{plain}
%    \newtheorem{assumption}{Assumption}[section]
\newtheorem{lemma}{Lemma}[section]
\newtheorem{theorem}{Theorem}[section]
\newtheorem{corollary}{Corollary}[section]
\theoremstyle{definition}
\newtheorem{definition}{Definition}[section]
\theoremstyle{remark}
\newtheorem{remark}{Remark}[section]

%\theoremstyle{plain} \newtheorem{remark}{Remark}[section]
%\theoremstyle{plain} \newtheorem{definition}{Definition}[section]
\theoremstyle{plain} \newtheorem{assumption}{Assumption}[section]

%%% Examples %%%%
\newtheoremstyle{example}  % Name
{1em}       % Space above 
{1em}       % Space below
{\small}      % Body font
{}          % Indent amount 
{\scshape}  % Theorem head font
{.}         % Punctuation after theorem head
{.5em}      % Space after theorem head
{}          % Theorem head spec
\theoremstyle{example}
\newtheorem{example}{Example}
\newtheorem{exercise}{Exercise}

\usepackage{framed}
\renewenvironment{block}[1]
{\framed \par \textbf{#1} \newline}
{\par \endframed}

\renewenvironment{exampleblock}[1]
{\framed \par \textit{#1} \newline \bigskip}
{\par \endframed}

\renewenvironment{alertblock}[1]
{\framed \par \textit{\textbf{#1}} \newline \hrule \bigskip}
{\par \endframed}
}


%\theoremstyle{plain} \newtheorem{conjecture}{Conjecture}[section]
%\theoremstyle{plain} \newtheorem{theorem}{Theorem}[section]
%\theoremstyle{plain} \newtheorem{proposition}{Proposition}[section]
%\theoremstyle{plain} \newtheorem{lemma}{Lemma}[section]
%\theoremstyle{plain} \newtheorem{corollary}{Corollary}[section]


%\newenvironment{proof}[1][Proof]{\begin{trivlist}
%\item[\hskip \labelsep {\bfseries #1}]}{\end{trivlist}}
%\newcommand{\qed}{\nobreak \ifvmode \relax \else
%      \ifdim\lastskip<1.5em \hskip-\lastskip
%      \hskip1.5em plus0em minus0.5em \fi \nobreak
%      \vrule height0.5em width0.5em depth0.25em\fi}

\newcommand \indexmargin[1] {\marginpar{\emph{#1}}\index{#1}}
\newcommand \marginref[2] {\marginpar{\emph{#1}}\emph{#1}\index{#2}}
\newcommand \emindex[1] {\emph{#1}\marginpar{\emph{#1}}\index{#1}}


\newcommand \E {\mathop{\mbox{\ensuremath{\mathbb{E}}}}\nolimits}
\newcommand \hE {\hat{\mathop{\mbox{\ensuremath{\mathbb{E}}}}\nolimits}}
\renewcommand \Pr {\mathop{\mbox{\ensuremath{\mathbb{P}}}}\nolimits}
\newcommand \given {\mathrel{|}}
\newcommand \gvn {|}
\newcommand \eq {{=}}


%% Special characters
\newcommand\Reals {{\mathbb{R}}}
\newcommand\Naturals {{\mathbb{N}}} 
\newcommand\Simplex {\mathbold{\Delta}}

\newcommand \FB {{\mathfrak{B}}}
\newcommand \FD {{\mathfrak{D}}}
\newcommand \FF {{\mathfrak{F}}}
\newcommand \FM {{\mathfrak{M}}}
\newcommand \FK {{\mathfrak{K}}}
\newcommand \FJ {{\mathfrak{J}}}
\newcommand \FL {{\mathfrak{L}}}
\newcommand \FO {{\mathfrak{O}}}
\newcommand \FS {{\mathfrak{S}}}
\newcommand \FT {{\mathfrak{T}}}
\newcommand \FP {{\mathfrak{P}}}
\newcommand \FR {{\mathfrak{R}}}


\newcommand \CA {{\mathcal{A}}}
\newcommand \CB {{\mathcal{B}}}
\newcommand \CC {{\mathcal{C}}}
\newcommand \CD {{\mathcal{D}}}
\newcommand \CE {{\mathcal{E}}}
\newcommand \CF {{\mathcal{F}}}
\newcommand \CG {{\mathcal{G}}}
\newcommand \CH {{\mathcal{H}}}
\newcommand \CJ {{\mathcal{J}}}
\newcommand \CL {{\mathcal{L}}}
\newcommand \CM {{\mathcal{M}}}
\newcommand \CN {{\mathcal{N}}}
\newcommand \CO {{\mathcal{O}}}
\newcommand \CP {{\mathcal{P}}}
\newcommand \CQ {{\mathcal{Q}}}
\newcommand \CR {{\mathcal{R}}}
\newcommand \CS {{\mathcal{S}}}
\newcommand \CT {{\mathcal{T}}}
\newcommand \CU {{\mathcal{U}}}
\newcommand \CV {{\mathcal{V}}}
\newcommand \CW {{\mathcal{W}}}
\newcommand \CX {{\mathcal{X}}}
\newcommand \CY {{\mathcal{Y}}}
\newcommand \CZ {{\mathcal{Z}}}

\newcommand \BA {{\mathbb{A}}}
\newcommand \BI {{\mathbb{I}}}
\newcommand \BS {{\mathbb{S}}}

\newcommand \bx {{\mathbf{x}}}
\newcommand \by {{\mathbf{y}}}
\newcommand \bu {{\mathbf{u}}}
\newcommand \bw {{\mathbf{w}}}
\newcommand \ba {{\mathbf{a}}}
\newcommand \bz {{\mathbf{z}}}
\newcommand \bat {{\mathbf{a}_t}}
\newcommand \bh {{\mathbf{h}}}
\newcommand \bo {{\mathbf{o}}}
\newcommand \bp {{\mathbf{p}}}
\newcommand \bs {{\mathbf{s}}}
\newcommand \br {{\mathbf{r}}}

\newcommand \SA {\mathscr{A}}
\newcommand \SB {\mathscr{B}}
\newcommand \SC {\mathscr{C}}
\newcommand \SF {\mathscr{F}}
\newcommand \SG {\mathscr{G}}
\newcommand \SH {\mathscr{H}}
\newcommand \SJ {\mathscr{J}}
\newcommand \SL {\mathscr{L}}
\newcommand \SP {\mathscr{P}}
\newcommand \SR {\mathscr{R}}
%%\newcommand \SS {\mathscr{S}}
\newcommand \ST {\mathscr{T}}
\newcommand \SU {\mathscr{U}}
\newcommand \SV {\mathscr{V}}
\newcommand \SW {\mathscr{W}}

\newcommand \hM {\widehat{M}}

\newcommand \KL[2] {\mathbb{D}\left( #1 \| #2 \right)}


%\newcommand \p {\partial}

\newcommand \then{\Rightarrow}
\newcommand \defn {\mathrel{\triangleq}}
%\newcommand \StateSet {{\CQ}}


%% Commands

\newcommand \argmax{\mathop{\rm arg\,max}}
\newcommand \argmin{\mathop{\rm arg\,min}}
\newcommand \dtan{\mathop{\rm dtan}}
\newcommand \sgn{\mathop{\rm sgn}}
\newcommand \trace{\mathop{\rm tr}}

\newcommand \onenorm[1]{\left\|#1\right\|_1}
\newcommand \pnorm[2]{\left\|#1\right\|_{#2}}
\newcommand \inftynorm[1]{\left\right\|#1\|_\infty}
\newcommand \norm[1]{\left\|#1\right\|}

%%\newcommand \defn {\triangleq}
%%\newcommand \defn {\equiv}
%%\newcommand \defn {\coloneq}
%%\newcommand \defn {\stackrel{\text{\tiny def}}{=}}
%%\newcommand \defn {\stackrel{\text{def}}{\hbox{\equalsfill}}}

\DeclareMathAlphabet{\mathpzc}{OT1}{pzc}{m}{it}

\newcommand \Normal {\mathop{\mathpzc{N}}\nolimits}
\newcommand \Poisson {\mathop{\mathpzc{Poisson}}\nolimits}
\newcommand \Multinomial {\mathop{\mathpzc{Multinomial}}\nolimits}
\newcommand \Dirichlet {\mathop{\mathpzc{Dirichlet}}\nolimits}
\newcommand \Student {\mathop{\mathpzc{Student}}\nolimits}
\newcommand \Bernoulli {\mathop{\mathpzc{Bernoulli}}\nolimits}
\newcommand \BetaDist   {\mathop{\mathpzc{Beta}}\nolimits}
\newcommand \Singular   {\mathop{\mathpzc{D}}\nolimits}
\newcommand \GammaDist {\mathop{\mathpzc{Gamma}}\nolimits}
\newcommand \Softmax{\mathop{\mathpzc{Softmax}}\nolimits}
\newcommand \Exp{\mathop{\mathpzc{Exp}}\nolimits}
\newcommand \Uniform{\mathop{\mathpzc{Unif}}\nolimits}
\newcommand \Laplace {\mathop{\mathpzc{Laplace}}\nolimits}

\newcommand \Param {\Theta}
\newcommand \param {\theta}
\newcommand \vparam {\vectorsym{\theta}}
\newcommand \mparam {\matrixsym{\Theta}}
\newcommand \Hyperparam {\Phi}
\newcommand \hyperparam {\phi}
\newcommand \family {\mathcal{F}}
\newcommand{\ie}{\emph{i.e.}\xspace}
\newcommand{\eg}{\emph{e.g.}\xspace}
\newcommand{\etal}{\emph{et al.}\xspace}
\newcommand{\constg}{}
\newcommand{\Bel}{\Xi}
\newcommand \Bay {\ensuremath{\mathscr{B}}}
\newcommand \Adv {\ensuremath{\mathscr{A}}}


\newcommand \Borel[1] {\FF(#1)}
\newcommand \Probs[1] {\FM(#1)}


\newcommand \pol {\pi}
\newcommand \Pol {\Pi}
\newcommand \mdp {\mu}
\newcommand \MDP {\CM}
\newcommand \meanMDP {{\bar{\mdp}_\xi}}

\newcommand {\msqr} {\vrule height0.33cm width0.44cm}
\newcommand {\bsqr} {\vrule height0.55cm width0.66cm}

\newcommand\ind[1]{\mathop{\mbox{\ensuremath{\mathbb{I}}}}\left\{#1\right\}}
\newcommand\Ind{\mbox{\bf{I}}}

\newcommand\dd{\,\mathrm{d}}

\newcommand \seq[2]{#1^{#2}}
\newcommand \pseq[3]{#1_{#2}^{#3}}
\newcommand \sam[2]{#1^{(#2)}}
\newcommand \transpose[1] {#1^\top}
\newcommand\set[1] {\left\{#1\right\}}
\newcommand\tuple[1] {\left\langle #1\right\rangle}
\newcommand\cset[2] {\left\{#1 ~\middle|~ #2\right\}}
\newcommand \ceil[1]{\left\lceil #1 \right\rceil}





\newcommand{\indep}{\mathrel{\text{\scalebox{1.07}{$\perp\mkern-10mu\perp$}}}}


\newcommand \eqlike {\eqsim}
\newcommand \gtlike {\succ}
\newcommand \ltlike {\prec}
\newcommand \gelike {\succsim}
\newcommand \lelike {\precsim}

\newcommand \eqpref {\eqsim^*}
\newcommand \gtpref {\succ^*}
\newcommand \ltpref {\prec^*}
\newcommand \gepref {\succsim^*}
\newcommand \lepref {\precsim^*}

\newcommand \util {U}
\newcommand \BUtil {U^*}
\newcommand \MUtil {\matrixsym{U}}
\newcommand \risk {\sigma}
\newcommand \Brisk {\sigma^*}
\newcommand \Loss {\ell}
\newcommand \Regret {L}
\newcommand \regret {\ell}
\newcommand \Reward {\SR}
\newcommand \reward {r}
\newcommand \vreward {\vectorsym{r}}
\newcommand \Rew {\rho}
\newcommand \outcome {\omega}
\newcommand \Outcome {\Omega}
\newcommand \act {a}
\newcommand \Act {\CA}
\newcommand \decision {a}
\newcommand \Decision {\mathcal{A}}
\newcommand \dec {\delta}
\newcommand \Dec {\mathscr{D}}


\newcommand {\MH} {\matrixsym{H}}

\newcommand \alg {\lambda}
\newcommand \Alg {\Lambda}
\newcommand \KNN {\textsc{k-NN}}

\newcommand \model {\mu}
\newcommand \MAP {\model_{\textrm{MAP}}}
\newcommand \Model {\CM}
\newcommand \Datasets {\CD}
\newcommand \Data {D}
\newcommand \Training {D_T}
\newcommand \Holdout {D_H}
\newcommand \Testing {D^*}
\newcommand \error {\epsilon}
\newcommand \obs {x}
\newcommand \Obs {\CX}
\newcommand \Att {\CA}
\newcommand \att {a}
\newcommand \attv {v}
\newcommand \Attv {\CV}
\newcommand \cls {y}
\newcommand \Cls {\CY}
\newcommand \Entropy {\mathbb{H}}
\newcommand \Gain {\mathbb{G}}


\newcommand \IDThree {\texttt{ID3}}

\newcommand \nactions {A}
\newcommand \nclasses {C}
\newcommand \nstates{S}
\newcommand \nobservations {N}
\newcommand \ndata{T}

\newcommand \figwidth {0.6\textwidth}
\newcommand \figheight {0.4\textwidth}

\newcommand \eye {\matrixsym{I}}
\newcommand \MA {\matrixsym{A}}
\newcommand \MX {\matrixsym{X}}
\newcommand \MY {\matrixsym{Y}}
\newcommand \MB {\matrixsym{B}}
\newcommand \MV {\matrixsym{V}}
\newcommand \MW {\matrixsym{W}}
\newcommand \MP {\matrixsym{P}}
\newcommand \vg {\vectorsym{\gamma}}
\newcommand \vp {\vectorsym{p}}
\newcommand \vs {\vectorsym{s}}
\newcommand \vx {\vectorsym{x}}
\newcommand \vr {\vectorsym{r}}
\newcommand \vm {\vectorsym{m}}
\newcommand \vb {\vectorsym{b}}
\newcommand \vt {\vectorsym{\theta}}

\newcommand \pn[1] {\vx_{[#1]}}

\newcommand \basis {f}
\newcommand \bel {\xi}
\newcommand \hyper {\omega}
\newcommand \mbel {\bel^D}
\newcommand \pbel {\bel^C}

\newcommand \pmean {\matrixsym{M}}
\newcommand \pcov {\matrixsym{C}}
\newcommand \pwish {\matrixsym{W}}
\newcommand \porder {n}

\newcommand \Syx {\matrixsym{\Sigma}_{yx}}
\newcommand \Sxx {\matrixsym{\Sigma}_{xx}}
\newcommand \Syy {\matrixsym{\Sigma}_{yy}}
\newcommand \Symx {\matrixsym{\Sigma}_{y\mid x}}

\newcommand \trans {\matrixsym{P}}
\newcommand \ident {\matrixsym{I}}

\newcommand \noise {\vectorsym{\varepsilon}}

\newcommand \pt {p_t}


\newcommand \CSet {G}
\newcommand \Parent[1] {\mathfrak{P}(#1)}
\newcommand \Children[1] {\mathfrak{C}(#1)}
\newcommand \Ancestors[1] {\mathfrak{A}(#1)}
\newcommand \Descendants[1] {\mathfrak{D}(#1)}
\newcommand \metric[2] {\nu(#1, #2)}
\newcommand \zooming {\zeta}
\newcommand \depth[1] {d(#1)}

\newcommand \sensitivity[1] {\mathbb{L}\left(#1\right)}
\newcommand \disc {\gamma}
\newcommand \Value {V}
\newcommand \val {\vectorsym{v}}
\newcommand \Vals {\mathcal{V}}
\newcommand \qval {\vectorsym{q}}
\newcommand \Qvals {\mathcal{Q}}
\newcommand \blm {\mathscr{L}}
\newcommand \tdm {\mathscr{D}}
\newcommand \pim {\mathscr{B}}



\newcommand \dist[2]{D\left(#1 ~\middle\|~ #2\right)}

\newcommand \Ae {A_\epsilon^\hist}

\newcommand \lrdist[2]{d_{lr}(#1, #2)}
\newcommand \xdistChar{\rho}
\newcommand \xdist[2]{\xdistChar(#1, #2)}
\newcommand \pdist[2]{\kappa(#1, #2)}
\newcommand{\constScale}{\omega}
\newcommand{\constScaleB}{\kappa}

\newcommand \fields[1]{\sigma(#1)}

\newcommand \hist {h}

\newcommand \abs[1] {\left|#1\right|}

\newcommand{\errorband}[5][]{ % x column, y column, error column, optional argument for setting style of the area plot
\pgfplotstableread[col sep=comma, skip first n=2]{#2}\datatable
% Lower bound (invisible plot)
\addplot [draw=none, stack plots=y, forget plot] table [
x={#3},
y expr=\thisrow{#4}-\thisrow{#5}
] {\datatable};

% Stack twice the error, draw as area plot
\addplot [draw=none, fill=gray!40, stack plots=y, area legend, #1] table [
x={#3},
y expr=2*\thisrow{#5}
] {\datatable} \closedcycle;

% Reset stack using invisible plot
\addplot [forget plot, stack plots=y,draw=none] table [x={#3}, y expr=-(\thisrow{#4}+\thisrow{#5})] {\datatable};
}


%%% macros to make things smalller
% For comparison, the existing overlap macros:
% \def\llap#1{\hbox to 0pt{\hss#1}}
% \def\rlap#1{\hbox to 0pt{#1\hss}}
\def\clap#1{\hbox to 0pt{\hss#1\hss}}
\def\mathllap{\mathpalette\mathllapinternal}
\def\mathrlap{\mathpalette\mathrlapinternal}
\def\mathclap{\mathpalette\mathclapinternal}
\def\mathllapinternal#1#2{%
\llap{$\mathsurround=0pt#1{#2}$}}
\def\mathrlapinternal#1#2{%
\rlap{$\mathsurround=0pt#1{#2}$}}
\def\mathclapinternal#1#2{%
\clap{$\mathsurround=0pt#1{#2}$}}


\usepackage{tikz}

%\usetikzlibrary{external}
%\tikzexternalize[prefix=tikz/]
\usepackage{gnuplot-lua-tikz}


\usetikzlibrary{automata}
\usetikzlibrary{topaths}
\usetikzlibrary{shapes}
\usetikzlibrary{arrows}
\usetikzlibrary{decorations.markings}
\usetikzlibrary{intersections}
\usetikzlibrary{backgrounds}


\tikzstyle{utility}=[diamond,draw=black,draw=blue!50,fill=blue!10,inner sep=0mm, minimum size=8mm]
\tikzstyle{select}=[rectangle,draw=black,draw=blue!50,fill=blue!10,inner sep=0mm, minimum size=6mm]
\tikzstyle{hidden}=[dashed,draw=black,fill=red!10]
\tikzstyle{RV}=[circle,draw=black,draw=blue!50,fill=blue!10,inner sep=0mm, minimum size=6mm]
\tikzstyle{place}=[circle,draw=black,draw=blue!50,fill=blue!20,inner sep=0mm, minimum size=9mm]
\tikzstyle{select}=[rectangle,draw=black,draw=blue!50,fill=blue!20,inner sep=0mm, minimum size=6mm]
\tikzstyle{transition}=[rectangle,draw=black!50,fill=black!20,thick]
\tikzstyle{observed}=[circle,draw=black,draw=blue!50,fill=blue!10,inner sep=0mm, minimum size=6mm]
\tikzstyle{someset}=[circle,draw=black,minimum size=8mm]

\tikzstyle{known}=[rectangle,draw=green!50,fill=green!20,thick]
\tikzstyle{queried}=[rectangle,draw=blue!50,fill=blue!20,thick]
%\tikzstyle{transition}=[rectangle,draw=black!50,fill=black!20,thick]

\tikzstyle{thickarrow}=[->, >=latex, line width=15pt, green!50]
\tikzstyle{medarrow}=[->, >=latex,  line width=5pt]
\tikzstyle{arrow}=[->,>=triangle 60]

\tikzset{every picture/.style={
    line width=1
  }
}

\definecolor{dark-green}{rgb}{0,0.5,0}






\title{Privacy}
\author[C. Dimitrakakis]{Christos Dimitrakakis}
\begin{document}

\begin{frame}
  \titlepage
\end{frame}

\section{Introduction}
\only<presentation>{
  \begin{frame}
    \tableofcontents[ 
    currentsection, 
    hideothersubsections, 
    sectionstyle=show/shaded
    ] 
  \end{frame}
}

\begin{frame}
  \centering
  \includegraphics[height=\textheight]{../figures/smbc-the-problem}
\end{frame}
\begin{frame}
  \begin{block}{Privacy in statitical disclosure.}
    \only<article>{ Consider a researcher wishing to collect data for a  statistical analysis. As long as the analysis is eventually
      published,\footnote{If somebody knows that the analysis is being
        conducted, however, they could still learn something private from the fact that the analysis has /emph{not} been published.} this
      creates two types of possible privacy violations.}
    \begin{itemize}
    \item Public analysis of sensitive data.
    \item Publication of ``anonymised'' data.
    \end{itemize}
    \only<article>{ The first is direct exposure of sensitive data
      through publication of the analysis, if for example the study is
      about something such as drug use. The second is through
      publication of ``anonymised'' versions of the dataset, for
      example by removing names and addresses, which create
      opportunities for linkage attacks.}
  \end{block}
  \begin{alertblock}{Not about cryptography}
    \only<article>{
      The problems we are considering can not be solved through cryptographic means. Cryptography provides:}
    \begin{itemize}
    \item Secure communication and computation.
    \item Authentication and verification.
    \end{itemize}
    \only<article>{These are useful to establish secure channels with somebody that we trust. However, the issue that}
  \end{alertblock}

  \begin{block}{An issue of trust}
    \only<article>{Fundamentally, privacy in statistics is an issue of trust. The analyst, whether it be a human statistician, or an automated service provided by a company, will use your data to make decisions. You must also decide:}
    \begin{itemize}
    \item Who to trust and how much.
    \item With what data to trust them.
    \item What you want out of the service.
    \end{itemize}
    \only<article>{These are difficult questions and are hard to quantify, hence in this course we are assuming that we have already decided how to answer them, and we simply want to find an appropriate methodology for achieving a good result.}
  \end{block}
\end{frame}

\section{Database access models}
\only<presentation>{
  \begin{frame}
    \tableofcontents[ 
    currentsection, 
    hideothersubsections, 
    sectionstyle=show/shaded
    ] 
  \end{frame}
}

\begin{frame}
  \frametitle{Databases}
  \begin{example}[Typical relational database in a tax office]
    \begin{table}[H]
      \centering
  \begin{tabular}{l|l|l|l|l|l|l}
    ID & Name &  Salary & Deposits & Age & Postcode & Profession\\
    \hline
    1959060783 & Mike Pence & 150,000 & 1e6 & 60 & 1001 & Politician\\
    1946061408 & Donald Trump & 300,000 & -1e9 & 72 & 1001 & Rentier\\
    2100010101 & A. B. Student & 10,000 & 100,000 & 40 & 1001 & Time Traveller
  \end{tabular}
\end{table}
\end{example}

\only<1>{
  \begin{block}{Database access}
    \begin{itemize}
    \item When owning the database: Direct look-up.
    \item When accessing a server etc: Query model.
    \end{itemize}
  \end{block}
}
\only<2>{
  \begin{figure}[H]
    \centering
    \begin{tikzpicture}
        \node[rectangle] at (0,0) (python) {Python program};
        \node[rectangle] at (8,0) (database) {Database System};
        \draw[thickarrow, bend right]   (python) to node[black]{Query} (database) ;
        \draw[thickarrow, bend right]   (database) to node[black]{response} (python) ;
      \end{tikzpicture}
    \label{fig:database-access}
    \caption{Database access model}
  \end{figure}
}
  
\end{frame}

\begin{frame}
  \frametitle{Queries in SQL}
  \begin{block}{The \texttt{SELECT} statement}
    \begin{itemize}
    \item \texttt{SELECT column1, column2 FROM table;}
      \only<article>{This selects only some columns from the table}
    \item \texttt{SELECT * FROM table;}
      \only<article>{This selects all the columns from the table}
    \end{itemize}
  \end{block}

  \begin{block}{Selecting rows}
    \texttt{SELECT * FROM table WHERE column = value;}
  \end{block}

  \begin{exampleblock}{Arithmetic queries}
    \only<article>{Here are some example SQL statements}
    \begin{itemize}
    \item  \texttt{SELECT COUNT(column) FROM table WHERE condition;}
      \only<article>{This allows you to count the number of rows matching \texttt{condition}}
    \item  \texttt{SELECT AVG(column) FROM table WHERE condition;}
      \only<article>{This lets you to count the number of rows matching \texttt{condition}}
    \item  \texttt{SELECT SUM(column) FROM table WHERE condition;}
      \only<article>{This is used to sum up the values in a column.}
    \end{itemize}
  \end{exampleblock}

\end{frame}



%%% Local Variables:
%%% mode: latex
%%% TeX-master: "notes"
%%% End:
 % data base access model

\only<article>{ Consider a researcher wishing to collect data for a
  statistical analysis. As long as the analysis is eventually
  published,\footnote{If somebody knows that the analysis is being
    conducted, however, they could still learn something private from
    the fact that the analysis has /emph{not} been published.} this
  creates two types of possible privacy violations. The first is
  direct exposure of sensitive data through publication of the
  analysis, if for example the study is about something such as drug
  use. The second is through publication of ``anonymised'' versions of
  the dataset, which create opportunities for linkage attacks.}

\section{Privacy in databases}
\begin{frame}
  \frametitle{Anonymisation}
  \only<article>{If we wish to publish a database, frequently we need to protect identities of people involved. The simplest method for doing that is simply erasing directly identifying information. However, this does not really work most of the time, especially since attackers can have side-information that can reveal the identities of individuals in the original data.}
  
  \begin{example}[Typical relational database in Tinder]
    \begin{table}[H]
      \begin{tabular}{l|l|l|l|l|l|l}
        Birthday & Name & Height  & Weight & Age & Postcode & Profession\\
        \hline
        06/07 & \only<1>{Mike Pence} & 190 & 80 & 60-70 & 1001 & Politician\\
        06/14 & \only<1>{Donald Trump} & 185 & 110 & 70+ & 1001 & Rentier\\
        01/01 & \only<1>{A. B. Student} & 170 & 70 & 40-60 & 6732 & Time Traveller
      \end{tabular} 
    \end{table}
  \end{example}

  \only<2>{The simple act of hiding or using random identifiers is called anonymisation.}
  \only<article>{However this is generally insufficient as other identifying information may be used to re-identify individuals in the data.}
\end{frame}


\begin{frame}
  \frametitle{Record linkage}
  \only<article>{In particular, anonymisation is not enough as record linkage can allow you to still extract information using data from another database through \emph{quasi-identifiers}.}

  \only<1>{
    \def\firstcircle{(0,0) circle (7em)}
    \def\secondcircle{(3,0) circle (7em)}
    
    \begin{figure}[H]
      \centering
      \begin{tikzpicture}

        \draw \firstcircle node[text width=7em] {Ethnicity\newline
          Date\newline Diagnosis \newline Procedure \newline
          Medication \newline Charge }; \draw \secondcircle node [text
        width=2em, align=right] {Name \newline Address \newline
          Registration \newline Party \newline Lastvote};
        \begin{scope}
          \clip \firstcircle; \fill[red] \secondcircle;
        \end{scope}
        \node [text width=4em] (QI) at (1.5, 0) {Postcode \newline
          Birthdate \newline Sex}; \node [text width=4em] (qi-text) at
        (1.5, -3) {Quasi-identifiers}; \path[->]<1-> (qi-text) edge
        [bend left] (QI);
        % Now we want to highlight the intersection of the first and
        % the
        % second circle:


      \end{tikzpicture}
      
      \caption{An example of two datasets, one containing sensitive and the other public information. The two datasets can be linked and individuals identified through the use of quasi-identifiers.}
      \label{fig:quasi-identifiers}
    \end{figure}
  }
  
  \begin{example}[Typical relational database in a tax office]
    \begin{table}[H]
      \begin{tabular}{l|l|l|l|l|l|l}
        ID & Name &  Salary & Deposits & Age & Postcode & Profession\\
        \hline
        1959060783 & Mike Pence & 150,000 & 1e6 & 60 & 1001 & Politician\\
        1946061408 & Donald Trump & 300,000 & -1e9 & 72 & 1001 & Rentier\\
        2100010101 & A. B. Student & 10,000 & 100,000 & 40 & 6732 & Time Traveller
      \end{tabular}
    \end{table}
  \end{example}
  
  \begin{example}[Typical relational database in Tinder]
    \begin{table}[H]
      \begin{tabular}{l|l|l|l|l|l|l}
        Birthday & Name & Height  & Weight & Age & Postcode & Profession\\
        \hline
        06/07 & & 190 & 80 & 60-70 & 1001 & Politician\\
        06/14 &  & 185 & 110 & 70+ & 1001 & Rentier\\
        01/01 &  & 170 & 70 & 40-60 & 6732 & Time Traveller
      \end{tabular}
    \end{table}
  \end{example}
\end{frame}

\section{$k$-anonymity}

\begin{frame}
  \frametitle{$k$-anonymity}
  \begin{figure}[H]
    \centering \subfigure[Samarati]{\includegraphics[width=0.25\textwidth]{../figures/samarati}}
    \subfigure[Sweeney]{\includegraphics[width=0.25\textwidth]{../figures/sweeney}}
  \end{figure}
  \only<article>{The concept of $k$-anonymity was introduced by~\citet{samarati1998protecting} and provides good guarantees when accessing a single database}

  \begin{definition}[$k$-anonymity]
    A database provides $k$-anonymity if for every person in the database is indistinguishable from $k-1$ persons with respect to \emph{quasi-identifiers}.
  \end{definition}
  \alert{It's the analyst's job to define quasi-identifiers}
  
\end{frame}

\begin{frame}
  \only<1>{
    \begin{table}[H]
      \begin{tabular}{l|l|l|l|l|l|l}
        Birthday & Name & Height  & Weight & Age & Postcode & Profession\\
        \hline
        06/07 & Mike Pence & 190 & 80 & 60+ & 1001 & Politician\\
        06/14 & Donald Trump & 185 & 110 & 60+ & 1001 & Rentier\\
        06/12 & John Bolton & 170 & 60 & 60+ & 1243 & Politician\\
        01/01 & A. B. Student & 170 & 70 & 40-60 & 6732 & Time Traveller\\
        05/08 & Li Yang & 175 & 72 & 30-40 & 6910 & Time Traveller
      \end{tabular}
      \caption{1-anonymity.}
    \end{table}

  }
  \only<presentation>{
    \only<2>{
      \begin{tabular}{l|l|l|l|l|l|l}
        Birthday & Name & Height  & Weight & Age & Postcode & Profession\\
        \hline
        06/07 &  & 190 & 80 & 60+ & 1001 & Politician\\
        06/14 &  & 185 & 110 & 60+ & 1001 & Rentier\\
        06/12 &  & 170 & 60 & 60+ & 1243 & Politician\\
        01/01 &  & 170 & 70 & 40-60 & 6732 & Time Traveller\\
        05/08 &  & 175 & 72 & 30-40 & 6910 & Policeman
      \end{tabular}
      1-anonymity
    }

    \only<3>{
      \begin{tabular}{l|l|l|l|l|l|l}
        Birthday & Name & Height  & Weight & Age & Postcode & Profession\\
        \hline
        06/07 &  & 180-190 & 80+ & 60+ & 1* & \\
        06/14 &  & 180-190 & 80+ & 60+ & 1* &\\
        06/12 &  & 170-180 & 60-80 & 60+ & 1* & \\
        01/01 &  & 170-180 & 60-80 & 20-60 & 6* &\\
        05/08 &  & 170-180 & 60-80 & 20-60 & 6* & 
      \end{tabular}
      1-anonymity
    }
  }
  \only<4>{
    \begin{table}[H]
      \begin{tabular}{l|l|l|l|l|l|l}
        Birthday & Name & Height  & Weight & Age & Postcode & Profession\\
        \hline
                 &  & 180-190 & 80+ & 60+ & 1* & \\
                 &  & 180-190 & 80+ & 60+ & 1* &\\
                 &  & 170-180 & 60-80 & 60+ & 1* & \\
                 &  & 170-180 & 60-80 & 20-60 & 6* &\\
                 &  & 170-180 & 60-80 & 20-60 & 6* & 
      \end{tabular}
      \caption{2-anonymity: the database can be partitioned in sets of at least 2 records}
    \end{table}
  }

  \only<article>{However, with enough information, somebody may still be able to infer something about the indivduals}
\end{frame}





\section{Differential privacy}
\only<article>{While $k$-anonymity can protect against specific re-identification attacks when used with care, it says little about what to do when the adversary has a lot of power. For example, if the  adversary knows the data of everybody that has participated in the database,  it is trivial for them to infer what our own data is. Differential privacy offers protection against adversaries with unlimited side-information or computational power. Informally, an algorithmic computation is differentially-private if an adversary cannot distinguish two similar database based on the result of the computation. While the notion of similarity is for the analyst to defined, a common is to say that two databases are similar when they are identical apart from the data of one person.}

\begin{frame}
  \begin{figure}[H]
    \begin{tikzpicture}
      \node[label=left:$x$] at (0,0) (data) {\includegraphics[width=0.2\columnwidth]{../figures/medical}};

      \node[label=$x_1$] at (-2,3)(patient1) {\includegraphics[width=0.1\columnwidth]{../figures/me-recent}};
      \uncover<3->{
        \node[label=$x_2$] at (2,3) (patient2) {\includegraphics[width=0.2\columnwidth]{../figures/judge}};
      }
      \uncover<4->{
        \node[label=$a$] at (4,0)   (statistics) {\includegraphics[width=0.3\columnwidth]{../figures/coronary-disease}};
      }
      \uncover<2->{
        \draw[->] (patient1) -- (data);
      }
      \uncover<3->{
        \draw[->] (patient2) -- (data);
      }
      \uncover<4->{
        \draw[->] (data) -- node[above]{$\pol$} (statistics);
      }
      \uncover<5->{
        \draw[line width=5, red, ->] (statistics) -- (patient2);
      }
    \end{tikzpicture}
    \caption{If two people contribute their data $x = (x_1, x_2)$ to a medical database, and an algorithm $\pol$ computes some public output $a$ from $x$, then it should be hard infer anything about the data from the public output.}
  \end{figure}

\end{frame}

\begin{frame}
  \frametitle{Privacy desiderata}
  \only<article>{
    Consider a scenario where $n$ persons give their data $x_1, \ldots, x_n$ to an analyst. This analyst then performs some calculation $f(x)$ on the data and published the result. The following properties are desirable from a general standpoint.

    \paragraph{Anonymity.} Individual participation in the study remains a secret. From the release of the calculations results, nobody can significantly increase their probability of identifying an individual in the database.

    \paragraph{Secrecy.} The data of individuals is not revealed. The release does not significantly increase the probability of inferring individual's information $x_i$.

    \paragraph{Side-information.} Even if an adversary has arbitrary side-information, he cannot use that to amplify the amount of knowledge he would have obtained from the release.

    \paragraph{Utility.} The released result has, with high probability, only a small error relative to a calculation that does not attempt to safeguard privacy.
  }
  \only<presentation>{
    We wish to calculate something on some private data and publish a \alert{privacy-preserving}, but \alert{useful}, version of the result.
    \begin{itemize}
    \item Anonymity: Individual participation remains hidden.
    \item Secrecy: Individual data $x_i$ is not revealed.
    \item Side-information: Linkage attacks are not possible.
    \item Utility: The calculation remains useful.
    \end{itemize}
  }
\end{frame}

\begin{frame}
  \frametitle{Example: The prevalence of drug use in sport}
  
  \only<article>{
    Let's say you need to perform a statistical analysis of the drug-use habits of athletes. Obviously, even if you promise the athlete not to reveal their information, you still might not convince them. Yet, you'd like them to be truthful. The trick is to allow them to randomly change their answers, so that you can't be \emph{sure} if they take drugs, no matter what they answer.
  }

  \only<presentation>{
    \begin{itemize}
    \item $n$ athletes
    \item Ask whether they have doped in the past year.
    \item Aim: calculate \% of doping.
    \item How can we get truthful / accurate results?
    \end{itemize}
  }
  \only<2>{
    \begin{block}{Algorithm for randomising responses about drug use}
      \begin{enumerate}
      \item Flip a coin.
      \item If it comes heads, respond \texttt{Yes}.
      \item Otherwise, respond truthfully.
      \end{enumerate}
    \end{block}

    If the rate of positive responses is $p$, everybody follows the protocol, and the coin is fair, what is the true rate $q$ of drug use?
  }
  \only<presentation>{
    \uncover<3>{
      \[
      p = 1/2 + q/2 \Rightarrow q = 1/2
      \]
    }
  }
  \only<article>{The problem with this approach, of course, is that we are effectively throwing away half of our data. In particular, if we repeated the experiment with a coin that came heads at a rate $\epsilon$, then our error bounds would scale as $O(1/\sqrt{\epsilon n})$ for $n$ data points.}
\end{frame}

\begin{frame}
  \frametitle{The randomised response mechanism}
  \only<article>{The above idea can be generalised. Consider we have data $x_1, \ldots, x_n$ from $n$ users and we transform it randomly to $y_1, \ldots, y_n$ using the following mapping.}
  \begin{definition}[Randomised response]
    The $i$-th user, whose data is $x_i \in \{0,1\}$ , responds with $a_i \in \{0, 1\}$ with probability
    \[
    \pol(a_i = j \mid x_i = k) = p,  \qquad  \pol(a_i = k \mid x_i = k) = 1 - p,
    \]
    where $j \neq k$.
  \end{definition}

  \uncover<2->{Given the complete data $x$, the mechanism's output is $a = (a_1, \ldots, a_n)$.}
  \uncover<3->{Since the algorithm independently calculates a new value for each data entry, the output is
    \[
    \pol(a \mid x) = \prod_i \pol(a_i \mid x_i)
    \]
  }

  \only<article>{This mechanism satisfies so-called $\epsilon$-differential privacy, which we will define later.}

\end{frame}

\begin{frame}
  \frametitle{The local privacy model}
  \begin{figure}[H]
    \centering
    \begin{tikzpicture}
      \node[RV] at (0,0) (x1) {$x_1$};
      \node[RV] at (0,1) (x2) {$x_2$};
      \node[RV] at (0,2) (xn) {$x_n$};
      \node[select] at (1,-1) (pol) {$\pol$};
      \node[RV] at (2,0) (a1) {$a_1$};
      \node[RV] at (2,1) (a2) {$a_2$};
      \node[RV] at (2,2) (an) {$a_n$};
      \draw[->] (x1) -- (a1);
      \draw[->] (x2) -- (a2);
      \draw[->] (xn) -- (an);
      \draw[->] (pol) -- (a1);
      \draw[->] (pol) -- (a2);
      \draw[->] (pol) -- (an);
    \end{tikzpicture}
    
    \caption{The local privacy model}
    \label{fig:local-privacy}
  \end{figure}
\end{frame}

\begin{frame}
  \frametitle{Differential privacy.}
  \includegraphics[width=0.2\textwidth]{../figures/dwork} \hspace{1em}
  \includegraphics[width=0.2\textwidth]{../figures/mcsherry} \hspace{1em}
  \includegraphics[width=0.2\textwidth]{../figures/nissim} \hspace{1em}
  \includegraphics[width=0.2\textwidth]{../figures/smith}
  \only<article>{Now let us take a look at a way to characterise the  the inherent privacy properties of algorithms. This is called differential privacy, and it can be seen as a bound on the information an adversary with arbitrary power or side-information could extract from a computation.}
  
  \begin{definition}[$\epsilon$-Differential Privacy]
    A stochastic algorithm $\pol : \CX \to \CA$, where $\CX$ is endowed with a neighbourhood relation $N$, is said to be $\epsilon$-differentially private if
    \begin{equation}
      \label{eq:epsilon-dp}
      \left|\ln \frac{\pol(a \mid x)}{\pol(a \mid x')}\right| \leq \epsilon , \qquad \forall x N x'.
    \end{equation}
  \end{definition}
  
  \only<article>{Typically, algorithms are applied to datasets $x = (x_1, \ldots, x_n)$ composed of the data of $n$ individuals. Thus, all privacy guarantees relate to the data contributed by these individuals. 

    In this context, two datasets are usually called neighbouring if $x = (x_1, \ldots, x_{i-1}, x_i, x_{i+1} x_n)$ and 
    $x' = (x_1, \ldots, x_{i-1}, x_{i+1} x_n)$, i.e. if one dataset is missing an element.
    
    A slightly weaker definition of neighbourhood is to say that $x N x'$ if $x' = (x_1, \ldots, x_{i-1}, x'_i, x_{i+1} x_n)$, i.e. if one dataset has an altered element.

  }
\end{frame}

\begin{frame}
  \only<article>{
    \begin{remark}
      Any differentially private algorithm must be stochastic.
    \end{remark}

    To prove that this is necessary, consider the example of counting how many people take drugs in a competition. If the adversary only doesn't know whether you in particular take drugs, but knows whether everybody else takes drugs, it's trivial to discover your own drug habits by looking at the total. This is because in this case, $f(x) = \sum_i x_i$ and the adversary knows $x_i$ for all $i \neq j$. Then, by observing $f(x)$, he can recover $x_j = f(x) - \sum_{i \neq j} x_i$. Consequently, it is not possible to protect against adversaries with arbitrary side information without stochasticity.}
  \begin{remark}
    The randomised response mechanism with $p \leq 1/2$ is $\ln \frac{1 - p}{p}$-DP.
  \end{remark}
  \begin{proof}
    Consider $x = (x_1, \ldots, x_i,  \ldots, x_n)$, $x' = (x_1, \ldots, x'_i,  \ldots, x_n)$. Then
    \begin{align*}
      \pol(a \mid x)
      \uncover<2->{&= \prod_i \pol(a_i \mid x_i)}
      \uncover<3->{\\ &= \pol(a_j \mid x_j') \prod_{i \neq j} \pol(a_i \mid x_i) }
      \uncover<4->{\\ &\leq \frac{p}{1 - p} \pol(a_j \mid x_j) \prod_{i \neq j} \pol(a_i \mid x_i) }
      \uncover<5>{\\ &= \frac{1-p}{p} \pol(a \mid x')}
    \end{align*}
    \only<4>{$\pol(a_j = k\mid x_j = k) = 1 - p$ so the ratio is $\max\{(1-p)/p, p/(1 - p)\} \leq (1 - p)/p$ for $p \leq 1/2$.}
  \end{proof}
\end{frame}

\begin{frame}
  \begin{figure}[H]
    \centering
    \begin{tikzpicture}
      \node[rectangle] at (0,0) (python) {Python program};
      \node[rectangle] at (8,0) (database) {Database System};
      \draw[->, >=latex, blue!20!white, line width=15pt, bend right]   (python) to node[black]{Query $q$} (database) ;
      \draw[->, >=latex, blue!20!white, line width=15pt, bend right]   (database) to node[black]{Private response $a$} (python) ;
    \end{tikzpicture}
    \label{fig:database-access}
    \caption{Private database access model}
  \end{figure}
  \begin{block}{Response policy}
    The  policy defines a distribution over responses $a$ given the data $x$ and the query $q$.
    \[
    \pol(a \mid x, q)
    \]
  \end{block}
\end{frame}

\begin{frame}
  \frametitle{Differentially private queries}
  \begin{block}{The \texttt{DP-SELECT} statement}
    \begin{itemize}
    \item \texttt{DP-SELECT column1, column2 FROM table;}
      \only<article>{This selects only some columns from the table}
    \item \texttt{DP-SELECT * FROM table;}
      \only<article>{This selects all the columns from the table}
    \end{itemize}
  \end{block}

  \begin{block}{Selecting rows}
    \texttt{SELECT * FROM table WHERE column = value;}
  \end{block}

  \begin{exampleblock}{Arithmetic queries}
    \only<article>{Here are some example SQL statements}
    \begin{itemize}
    \item  \texttt{DP-SELECT COUNT(column) FROM table WHERE condition;}
      \only<article>{This allows you to count the number of rows matching \texttt{condition}}
    \item  \texttt{DP-SELECT AVG(column) FROM table WHERE condition;}
      \only<article>{This lets you to count the number of rows matching \texttt{condition}}
    \item  \texttt{DP-SELECT SUM(column) FROM table WHERE condition;}
      \only<article>{This is used to sum up the values in a column.}
    \end{itemize}
  \end{exampleblock}

  \begin{alertblock}{Cumulative privacy loss}
    Depending on the DP scheme, each query answered may leak privacy.
    \only<article>{In particular, if we always respond with an $\epsilon$-DP mechanism, after $T$ queries our privacy guarantee is $T \epsilon$. There exist mechanisms that do not respond to each query independently, which can bound the total privacy loss.}
  \end{alertblock}
\end{frame}

\begin{frame}
  \frametitle{The Laplace mechanism.}
  \only<article>{
    A simple method to obtain a differentially private algorithm from a deterministic function $f : \CX \to \Reals$, is to use additive noise, so that the output of the algorithm is simply 
    \[
    a = f(x) + \omega, \qquad \omega \sim \Laplace.
    \]
    The amount of noise added, together with the smoothness of the function $f$, determine the amount of privacy we have.
  }
  \begin{definition}[The Laplace mechanism]
    For any function $f : \CX \to \Reals$, 
    \begin{equation}
      \label{eq:laplace-mechanism}
      \pol(a \mid x) = \Laplace(f(x), \lambda),
    \end{equation}
    where the Laplace density is defined as
    \[
    p(\omega \mid \mu, \lambda) = \frac{1}{2 \lambda} \exp\left(-\frac{|\omega - \mu|}{\lambda}\right).
    \]
  \end{definition}
  \only<article>{Here, $\Laplace(\mu, \lambda)$ is the density $f(x) = \frac{\lambda}{2} \exp(-\lambda |x - \mu|)$}.
\end{frame}

\begin{frame}
  \begin{example}[Calculating the average salary]
    \begin{itemize}
    \item The $i$-th person receives salary $x_i$
    \item We wish to calculate the average salary in a private manner.
    \end{itemize}
  \end{example}
  \begin{block}{Local privacy model}
    \begin{itemize}
    \item Obtain $y_i = x_i + \omega$, where $\omega \sim \Laplace(\lambda)$.
    \item Return $a = n^{-1} \sum_{i=1}^n y_i$.
    \end{itemize}
  \end{block}
  \begin{block}{Centralised privacy model}
     Return $a = n^{-1} \sum_{i=1}^n x_i + \omega$, where $\omega \sim \Laplace(\lambda')$.
  \end{block}
  
  How should we add noise in order to guarantee privacy?
\end{frame}


\begin{frame}
  \frametitle{DP properties of the Laplace mechanism}
  \begin{definition}[Sensitivity]
    The sensitivity of a function $f$ is
    \[
    \sensitivity{f} \defn \sup_{x N x'} |f(x) - f(x')|
    \]
    \only<article>{
      If we define a metric $d$, so that $d(x, x') = 1$ for $x N x'$, then:
      \[
       |f(x) - f(x')| \leq \sensitivity{f} d(x, x'),
      \]
      i.e. $f$ is $\sensivity{f}$-Lipschitz with respect to $d$.
    }
  \end{definition}
  \begin{theorem}
    The Laplace mechanism on a function $f$ ran with $\Laplace(\lambda)$ is $\sensitivity{f} / \lambda$-DP.
  \end{theorem}
  \begin{proof}
    \begin{align*}
      \frac{\pol(a \mid x)}{\pol(a \mid x')}
      &=
      \frac{e^{|a - f(x')|/\lambda}}{e^{|a - f(x)|/\lambda}}
      \leq
      \frac{e^{|a - f(x)|/\lambda + \sensitivity{f}/\lambda}}{e^{|a - f(x)|/\lambda}}
        = e^{\sensitivity{f} / \lambda}
    \end{align*}
  \end{proof}
\end{frame}

\begin{frame}
  Here let is assume $x_i \in [0, M]$ for all $i$.
  \begin{block}{Laplace in the local privacy model}
    The sensitivity of the individual data is $M$, so to obtain $\epsilon$-DP we need to use $\lambda = M / \epsilon$.

    The variance of $a$ is $M / \epsilon \sqrt{n}$.
  \end{block}
  \begin{block}{Laplace in the centralised privacy model}
    The sensitivity of $f$ is $M / n$, so we need to use $\lambda = M / n\epsilon$.

    The variance of $a$ is $M / \epsilon n$.
  \end{block}
\end{frame}

\begin{frame}
  \frametitle{The Exponential Mechanism.}
  \only<article>{
    Here we assume that we can answer queries $q$, whereby each possible answer $a$ to the query has a different utility to the DM: $\util(q, a, x)$.
    Let $\sensitivity{\util(q)} \defn \sup_{x N x'} |\util(q, a, x) -\util(q, a, x)|$ denote the sensitivity of a query. Then the following mechanism is $\epsilon$-differentially private.
  }
  \begin{definition}[The Exponential mechanism]
    For any utility function $\util : \CQ \times \CA \times \CX \to \Reals$, define the policy
    \begin{equation}
      \label{eq:exponential-mechanism}
      \pol(a \mid x) \defn \frac{e^{\epsilon \util(q, a, x) / \sensitivity{ \util(q)}}}{\sum_{a'} e^{\epsilon \util(q, a', x) / \sensitivity{\util(q)}}}
    \end{equation}
  \end{definition}
  \only<article>{
    Clearly, when $\epsilon \to 0$, this mechanism is uniformly random. When $\epsilon \to \infty$ the action maximising $\util(q,a,x)$ is always chosen.
  }
\end{frame}



%%% Local Variables:
%%% mode: latex
%%% TeX-engine: xetex
%%% TeX-master: "notes"
%%% End:

 %data bases


\end{document}


%%% Local Variables:
%%% mode: latex
%%% TeX-master: t
%%% End:
