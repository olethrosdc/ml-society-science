\chapter{Simple decision problems}

\only<article>{This chapter deals with simple decision problems, whereby a decision maker (DM) makes a simple choice among many. In some of this problems the DM has to make a decision after first observing some side-information. Then the DM uses a \emph{decision rule} to assign a probability to each possible decision for each possible side-information. However, designing the decision rule is not trivial , as it relies on previously collected data. A higher-level decision includes choosing the decision rule itself. The problems of classification and regression fall within this framework. While most steps in the process can be automated and formalised, a lot of decisions are actual design choices made by humans. This creates scope for errors and misinterpretation of results.

In this chapter, we shall formalise all these simple decision problems from the point of view of statistical decision theory. The first question is, given a real world application, what type of decision problem does it map to? Then, what kind of machine learning algorithms can we use to solve it? What are the underlying assumptions and how valid are our conclusions? Does application of the method have a societal impact? In particular, is there scope for privacy violations, reinforcing racial biases or creating other negative externalities?
}
\section{Introduction to machine learning}
 

\only<article>{
 What are the central problems in machine learning?

 Problems in machine learning are similar to problems in science.
 Scientists must plan experiments intelligently and collect data.
 The must be able to use the data to verify a different hypothesis.
 More generally, they must be able to make decisions under
 uncertainty (Without uncertainty, there would be no need to gather more data).
 Similar problems appear in more mundane tasks, like learning to drive a car.
}
\only<presentation>{
 \begin{frame}
   \frametitle{Scientific applications}
   \centering
   \begin{columns}
     \begin{column}{0.5\textwidth}
       \centering
       \includegraphics[width=0.8\columnwidth]{../figures/climate.jpg}\\
       \includegraphics[width=\columnwidth]{../figures/networks-2.jpg}
     \end{column}
     \begin{column}{0.5\textwidth}
       \includegraphics[width=\columnwidth]{../figures/dark_matter.jpg}
       \\
       \includegraphics[width=\columnwidth]{../figures/protein.jpg}
     \end{column}
   \end{columns}
  \only<2>{
    \begin{tikzpicture}[remember picture,overlay]
      \draw[fill=black,opacity=0.75] 
      (current page.north east) rectangle (current page.south west);
      \node at (current page.center) {
        {\Huge \alert{Interpretability, Reproducibility}}
      };
    \end{tikzpicture}}
\end{frame}
}

\only<article>{
 For that reason, science is a very natural application area for
 machine learning.  We can model the effects of climate change and
 how to mitigate it; discover structure in social networks; map
 the existence of dark matter in the universe by intelligently
 shifting through weak gravitational lens data, and not only study
 the mechanisms of protein folding, but discover methods to
 synthesize new drugs.

 We must be careful, however. In many cases we need to be able to
 interpret what our model tells us. We also must make sure that
 the any results we obtain are reproducible. This is something
 that we shall emphasize in this course.
}

\only<presentation>{
\begin{frame}
  \frametitle{Pervasive ``intelligent'' systems}
  \begin{columns}
    \begin{column}{0.3\textwidth}
      \centering
      \includegraphics[width=\textwidth]{../figures/echo-home.jpg}
      \\
      Home assistants

      \vspace{\fill}

      \bigskip

      \includegraphics[width=\textwidth]{../figures/tesla.jpg}
      \\
      Autonomous vehicles
    \end{column}
    \begin{column}{0.3\textwidth}
      \centering 
      \includegraphics[width=\textwidth]{../figures/web-ads.png}
      \\
      Web advertising

      \vspace{\fill}

      \bigskip

      \includegraphics[width=\textwidth]{../figures/uber-here-maps.jpg}
      \\
      Ridesharing
    \end{column}
    \begin{column}{0.3\textwidth}
      \centering 
      \\
      \includegraphics[width=\textwidth,clip = true, trim=0 0 0 42.5cm]{../figures/lending.pdf}
      \\
      Lending

      \vspace{\fill}

      \bigskip

      \includegraphics[width=\textwidth]{../figures/algorithms-public.jpg}
      \\
      Public policy
    \end{column}
  \end{columns}
  \only<2>{
    \begin{tikzpicture}[remember picture,overlay]
      \draw[fill=black,opacity=0.75] 
      (current page.north east) rectangle (current page.south west);
      \node at (current page.center) {
        {\Huge \alert{Privacy, Fairness, Safety}}
      };
    \end{tikzpicture}}
\end{frame}
}

\only<article>{
 While machine learning models in science are typically carefully
 handcrafted by scientists and experts in machine learning and
 statistics, this is not typically the case in everyday
 applications. Nevertheless, well-known or home-grown machine
 learning models are being deployed across the application
 spectrum. This involve home assistants that try and get you want,
 web advertising, which tries to find new things for you to want,
 lending, which tries to optimally lend you money so that you buy
 what you didn't need before. We also have autonomous vehicles,
 which take you were you want to go, and ridesharing services,
 which do the same thing, but use humans instead. Finally, there
 are many applications in public policy, such as crime prevention,
 justice, and disease control which use machine learning.  In all
 those cases, we have to worry about a great many things that are
 outside the scope of the machine learning problems itself. These
 are (a) privacy: you don't want your data used in ways that you have
 not consented to (b) fairness: you don't want minorities to be
 disadvantaged and (c) safety: you don't want your car to crash.
}

\subsection{Data analysis}

\only<article>{
  To make the above more concrete, let's have a look at a number of problems. 
}

\only<presentation>{
\begin{frame}
  \centering
  \Huge{What can machine learning do?}
\end{frame}
}
\begin{frame}
  \frametitle{Can machines learn from data?}
  \begin{center}
    \only<1>{\includegraphics[width=0.8\textwidth]{../figures/text-cloud}
      \\

      {\large An unsupervised learning problem: topic modelling}
    }
    \only<2>{\includegraphics[width=0.8\textwidth]{../figures/Face-Recognition}
      \\

      {\large A supervised learning problem: object recognition}
    }
  \end{center}
\end{frame}


\only<article>{
You can use machine learning just to analyse, or find structure in
the data. This is generally called unsupervised learning. One such
example is topic modelling, where you let the algorithm find topics
from a corpus of text.  These days machines are used to learn from
in many applications.  These include speech recognition, facial
authentication, weather prediction, etc. In general, in these
problems we are given a \emph{labelled} dataset with, say, example
images from each class. Unfortunately this does not scale very
well, because obtaining labels is expensive.

This is partially how science works, because what we need to do
is to find a general rule of nature from data. Starting from some
hypothesis and some data, we reach a conclusion. However, many
times we may need to actively experiment to obtain more data,
perhaps because we found that our model is wrong.
}



\begin{frame}
  \frametitle{Can machines learn from their mistakes?}
  \begin{center}
    \includegraphics[width=0.7\textwidth]{../figures/rl_interaction}
  \end{center}
  \begin{block}{Reinforcement learning}
    Take actions $a_1, \ldots, a_t$, so as to maximise utility
    $U = \sum_{t=1}^T r_t$
  \end{block}
\end{frame}


\only<article>{
So, what happens when we make a mistake? Can we somehow recognise
it? Humans and other animals can actually learn from their
mistakes. Consider the proverbial rat in the maze. At some
intervals, the experimenter places some cheese in there, and the
rat must do a series of actions to obtain it, such as navigating
the maze and pulling some levers. It doesn't know how to get to
the cheese easily, but it slowly learns the layout of the maze
through observation, and in the end, through trial-and-error it
is able to get to the cheese very efficiently.

We can formalise this as a reinforcement learning problem, where
the rat takes a series of actions; at each step it also obtains a
reward, let's say equal to 0 when it has no cheese, and 1 when it
eats cheese. Then we can declare that the rat's utility is the sum
of all rewards over time, i.e. the total amount of cheese it can
eat before it dies. The rat needs to explore the environment in order to be able to
get to the cheese. 

An example in robotics is trying to teach a
robot to flip pancakes. One easy thing we can try is to show the robot
how to do it, and then let it just copy the demonstrated
movement. However, this doesn't work! The robot needs to explore
variations of the movement, until it manages to successfully flip
pancakes. Again, we can formulate this as a reinforcement learning
problem, with a reward that is high whenever the pancake's position is
flipped, and on the pan; and low everywhere else. Then the robot can
learn to perform this behaviour through trial and error. It's
important to note that in this example, merely demonstration is not
enough. Neither is reinforcement learning enough. The same thing is
true for the recent success of AlphaGo in beating a master human:
apart from planning, they used both demonstration data and self-play,
so that it could learn through trial and error.  }

\begin{frame}
  \frametitle{Can machines make complex plans?}
  \begin{center}
    \includegraphics[width=0.8\textwidth]{../figures/619px-FloorGoban}
  \end{center}
\end{frame}


\only<article>{
 I suppose the first question is whether machines can plan
 ahead. Indeed, even for large problems, such as Go, machines can
 now perform at least as well as top-rated humans. How is this
 achieved?
}

\begin{frame}
  \frametitle{Machines can make complex plans!}
  \begin{center}
    \includegraphics[width=0.8\textwidth]{../figures/Tic-tac-toe-game-tree}
  \end{center}
\end{frame}


\only<article>{
The basic construction is the planning tree. This is an enumeration
of all possible future events. If a complete enumeration is
impossible, a partial tree is constructed. However this requires
evaluating non-terminal game positions. In the old times, this was
done with heuristics, but now this is data-driven, both through the
use of expert databases, and through self-play and reinforcement
learning.
}


\subsection{Experiment design}

\only<presentation>{
  \begin{frame}
    \centering
    \Huge{The scientific process as machine learning}
  \end{frame}
  \begin{frame}
    \centering
    \includegraphics[width=\textwidth]{../figures/Las_Vegas_slot_machines}
  \end{frame}
}


\only<article>{
An example that typifies trial and error learning are bandit
problems. Imagine that you are in a Casino and you wish to
maximise the amount of money you make during the night. There are
a lot of machines to play. If you knew which one was the best,
then you'd just play it all night long. However, you must also
spend time trying out different machines, in order to get an
estimate of how much money each one gives out. The trade off
between trying out different machines and playing the one you
currently think is best is called the exploration-exploitation
trade-off and it appears in many problems of experiment design for
science.
}


\begin{frame}
  \frametitle{Adam, the robot scientist}
  \centering
  \includegraphics[width=0.8\textwidth]{../figures/robot-scientist}
\end{frame}


\only<article>{
 Let's say we want to build a robot scientist and tell it to
 discover a cure for cancer. What does the scientist do and how can the robot replicate it??
}



\begin{frame}
  \frametitle{Drug discovery}
  \centering
  \includegraphics[width=\columnwidth]{../figures/drug-discovery-000}
\end{frame}


\only<article>{
Simplifying the problem a bit, consider that you have a large
number of drug candidates for cancer and you wish to discover
those that are active against it. The ideas is that you select
some of them, then screen them, to sort them into active and
inactive. However, there are too many drugs to screen, so the
process is interactive. At each cycle, we select some drugs to
screen, classify them, and then use this information to select
more drugs to screen. This cycle, consequently has two parts:
1. Selecting some drugs given our current knowledge.
2. Updating our knowledge given new evidence.
}


\begin{frame}
  \frametitle{Drawing conclusions from results}
  \centering
  \begin{tikzpicture}[line width=2pt]
    \node at (0,0) (bt) {hypothesis};
    \node[select] at (0,2) (at) {experiment};
    \node[utility] at (3,-2) (rt) {result};
    \draw[blue,->] (at) -- (rt);
    \node at (4,0) (bt2) {conclusion};
    \draw[red,->] (at) -- (bt2);
    \draw[red,->] (bt) -- (bt2);
    \draw[red,->] (rt) -- (bt2);
  \end{tikzpicture}
\end{frame}

\only<article>{
  In general, we would like to have some method which can draw
  conclusions from results. This involves starting with a
  hypothesis, performing an experiment to verify or refute it,
  obtain some experimental result; and then concluding for or
  against the hypothesis. Here the arrows show dependencies
  between these variables. So what do we mean by "hypothesis" in this case?
}

\subsection{Bayesian inference.}
\begin{frame}
  \frametitle{Tycho Brahe's minute eye measurements}
  \begin{columns}
    \begin{column}{0.5\textwidth}
      \includegraphics[width=0.5\textwidth]{../figures/circular-orbits}
    \end{column}
    \begin{column}{0.5\textwidth}
      \includegraphics[width=0.5\textwidth]{../figures/tycho-observations}
    \end{column}
  \end{columns}
  \begin{itemize}
  \item Hypothesis: Circular orbits
  \item Conclusion: \alert{Specific} circular orbits
  \end{itemize}
\end{frame}


\only<article>{
  Let's take the example of planetary orbits. Here Tycho famously
  spent 20 years experimentally measuring the location of Mars. He
  had a hypothesis: that planetary orbits were circular, but he
  didn't know which were the right orbits. When he tried to fit his data to this hypothesis, he concluded a specific circular orbit for Mars \ldots around Earth.
}


\begin{frame}
  \frametitle{Johannes Kepler's alternative hypothesis}
  \begin{columns}
    \begin{column}{0.5\textwidth}
      \includegraphics[width=0.5\textwidth]{../figures/orbits}
    \end{column}
    \begin{column}{0.5\textwidth}
      \includegraphics[width=0.5\textwidth]{../figures/tycho-observations}
    \end{column}
  \end{columns}
  \begin{itemize}
  \item Hypothesis: Circular \alert{or} elliptic orbits
  \item Conclusion: Specific \alert{elliptic} orbits
  \end{itemize}
\end{frame}


\only<article>{
Kepler had a more general hypothesis: that orbits could be
circular or elliptic, and he actually accepted that the planets
orbited the sun. This led him to the broadly correct model of all
planets being in elliptical orbits around the sun. However, the
actual verification that all things do not revolve around earth,
requires different experiments.
}


\begin{frame}
  \frametitle{200 years later, Gauss formalised this statistically}
  \begin{columns}
    \begin{column}{0.5\textwidth}
      \includegraphics[width=\columnwidth]{../figures/gauss-diagram}
    \end{column}
    \begin{column}{0.5\textwidth}
      \includegraphics[width=\columnwidth]{../figures/SeptemberTable}
    \end{column}
  \end{columns}
\end{frame}


\only<article>{
Later on, Gauss collected even more experimental data to calculate the orbit of Ceres. He did this using one of the first formal statistical methods; this allowed him to avoid cheating (like Kepler did, to accentuate his finding that orbits were elliptical).
}


\begin{frame}
  \frametitle{A warning: The dead salmon mirage}
  \includegraphics[width=\textwidth]{../figures/fmri-salmon}
\end{frame}


\only<article>{
It is quite easy to draw the wrong conclusions from applying
machine learning / statistics to your data. For example, it was
fashionable to perform fMRI studies in humans to see whether some
neurons have a particular functional role. There were even
articles saying that "we found the neurons encoding for Angelina
Jolie". So some scientists tried to replicate those results. They
took a dead salmon, and put it an fMRI scanner. They checked its
brain activity when it was shown images of happy or sad
people. Perhaps surprisingly, they found an area of the brain that
was correlated with the pictures - so it seemed, as though the
dead salmon could distinguish photos of happy people from sad
ones. However, this was all due to a misapplication of
statistics. In this course, we will try and teach you to avoid
such mistakes.
}


\begin{frame}
  \frametitle{Planning future experiments}
  \centering
  \begin{tikzpicture}[line width=2pt]
    \node at (0,0) (bt) {hypothesis};
    \node[select] at (0,2) (at) {experiment};
    \node[utility] at (3,-2) (rt) {result};
    \draw[blue,->] (at) -- (rt);
    \node at (4,0) (bt2) {conclusion};
    \draw[red,->] (at) -- (bt2);
    \draw[red,->] (bt) -- (bt2);
    \draw[red,->] (rt) -- (bt2);
  \end{tikzpicture}
\end{frame}

\only<article>{
I mentioned before that we must decide what experiment to do. This is indeed difficult, especially in setting such as drug discovery where the number of experiments is huge.  However, conceptually, there is a simple and elegant solution to this problem.
}


\begin{frame}
  \frametitle{Planning experiments is like Tic-Tac-Toe}
  \begin{center}
    \includegraphics[width=\textwidth]{../figures/Tic-tac-toe-game-tree}
  \end{center}
\end{frame}


\only<article>{
  The basic idea is to think of experiment design as a game between the scientist and Nature. At every step, the scientist plays an X to  denote an experiment. Then Nature responds with an Observation. The main difference from a game is that Nature is (probably) not adversarial. We can also generalise this idea to problems in robotics, etc.
}

\only<presentation>{
\begin{frame}
  \frametitle{Eve, another robot scientist}
  \centering \movie{\includegraphics[width=\textwidth]{../figures/eve.jpg}}{Eve-video.mp4}
  Discovered a malaria drug
\end{frame}
}
\only<article>{
 These kinds of techniques, coming from the reinforcement learning literature have been successfully used at the university of Manchester to create a robot, called Eve, that recently (re)-discovered a malaria drug.
}

\subsection{Course overview}

\begin{frame}
  \frametitle{Machine learning in practice}
  \begin{block}{Avoiding pitfalls}
    \begin{itemize}
    \item Choosing hypotheses.
    \item Correctly interpreting conclusions.
    \item Using a good testing methodology.
    \end{itemize}
  \end{block}
  \begin{block}{Machine learning in society}
    \begin{itemize}
    \item<alert@2> Privacy \uncover<2->{--- Medical data.}
    \item<alert@3> Fairness \uncover<3->{--- Credit risk.}
    \item<alert@4> Safety \uncover<4->{--- Autonomous vehicles.}
    \end{itemize}
  \end{block}
\end{frame}

\only<article>{
One of the things we want to do in this course is teach you to
avoid common pitfalls.

Now I want to get into a different track. So far everything has
been about pure research, but now machine learning is pervasive:
Our phones, cars, watches, bathrooms, kettles are connected to the
internet and send a continuous stream of data to companies. In
addition, many companies and government actors use machine
learning algorithms to make or support decisions. This creates a
number of problems in privacy, fairness and safety.
}


\begin{frame}
  \frametitle{Technical topics}
  
  \begin{block}{Machine learning problems}
    \begin{itemize}
    \item Unsupervised learning.
    \item Supervised learning.
    \item Reinforcement learning.
    \end{itemize}
  \end{block}

  \begin{block}{Machine learning tools}
    \begin{itemize}
    \item Stochastic optimisation and neural networks.
    \item Probabilistic inference and Bayesian networks.
    \item Markov decision processes.
    \end{itemize}
  \end{block}
\end{frame}

\begin{frame}
  \frametitle{Course structure}
  \begin{block}{Module structure}
    \begin{itemize}
    \item \alert{Activity}-based, hands-on.
    \item Background reading \alert{before} class.
    \item Mini-lecture with \alert{Q/A} at each class.
    \item Small \alert{group project} in second half of class.
    \end{itemize}
  \end{block}

  \begin{block}{Modules}
    \begin{itemize}
    \item Medical diagnostics.
    \item Speech recognition.
    \item Recommendation systems.
    \item Helicopter flight.
    \end{itemize}
  \end{block}
\end{frame}


\section{Nearest neighbours}
\begin{frame}
  \frametitle{Discriminating between diseases}
  % Title: glps_renderer figure
% Creator: GL2PS 1.3.8, (C) 1999-2012 C. Geuzaine
% For: Octave
% CreationDate: Fri Jun 16 12:38:10 2017
\begin{pgfpicture}
\pgfsetlinewidth{0.01pt}
\color[rgb]{1.000000,1.000000,1.000000}
\pgfpathmoveto{\pgfpoint{41.600006pt}{205.577454pt}}
\pgflineto{\pgfpoint{289.600037pt}{140.777435pt}}
\pgflineto{\pgfpoint{41.600006pt}{140.777435pt}}
\pgfpathclose
\pgfusepath{fill,stroke}
\pgfpathmoveto{\pgfpoint{41.600006pt}{205.577454pt}}
\pgflineto{\pgfpoint{289.600037pt}{205.577454pt}}
\pgflineto{\pgfpoint{289.600037pt}{140.777435pt}}
\pgfpathclose
\pgfusepath{fill,stroke}
\pgfpathmoveto{\pgfpoint{41.600006pt}{91.199989pt}}
\pgflineto{\pgfpoint{289.600037pt}{26.399979pt}}
\pgflineto{\pgfpoint{41.600006pt}{26.399979pt}}
\pgfpathclose
\pgfusepath{fill,stroke}
\pgfpathmoveto{\pgfpoint{41.600006pt}{91.199989pt}}
\pgflineto{\pgfpoint{289.600037pt}{91.199989pt}}
\pgflineto{\pgfpoint{289.600037pt}{26.399979pt}}
\pgfpathclose
\pgfusepath{fill,stroke}
\color[rgb]{1.000000,0.000000,0.000000}
\pgfpathmoveto{\pgfpoint{287.608032pt}{140.777435pt}}
\pgflineto{\pgfpoint{288.604004pt}{140.777435pt}}
\pgflineto{\pgfpoint{288.604004pt}{142.317886pt}}
\pgfpathclose
\pgfusepath{fill,stroke}
\pgfpathmoveto{\pgfpoint{289.600037pt}{141.234161pt}}
\pgflineto{\pgfpoint{288.604004pt}{142.317886pt}}
\pgflineto{\pgfpoint{288.604004pt}{140.777435pt}}
\pgfpathclose
\pgfusepath{fill,stroke}
\pgfpathmoveto{\pgfpoint{289.600037pt}{140.777435pt}}
\pgflineto{\pgfpoint{289.600037pt}{141.234161pt}}
\pgflineto{\pgfpoint{288.604004pt}{140.777435pt}}
\pgfpathclose
\pgfusepath{fill,stroke}
\pgfpathmoveto{\pgfpoint{283.624115pt}{140.777435pt}}
\pgflineto{\pgfpoint{284.620087pt}{140.777435pt}}
\pgflineto{\pgfpoint{284.620087pt}{140.819778pt}}
\pgfpathclose
\pgfusepath{fill,stroke}
\pgfpathmoveto{\pgfpoint{285.616089pt}{141.090790pt}}
\pgflineto{\pgfpoint{284.620087pt}{140.819778pt}}
\pgflineto{\pgfpoint{284.620087pt}{140.777435pt}}
\pgfpathclose
\pgfusepath{fill,stroke}
\pgfpathmoveto{\pgfpoint{285.616089pt}{140.777435pt}}
\pgflineto{\pgfpoint{285.616089pt}{141.090790pt}}
\pgflineto{\pgfpoint{284.620087pt}{140.777435pt}}
\pgfpathclose
\pgfusepath{fill,stroke}
\pgfpathmoveto{\pgfpoint{286.612061pt}{140.777435pt}}
\pgflineto{\pgfpoint{285.616089pt}{141.090790pt}}
\pgflineto{\pgfpoint{285.616089pt}{140.777435pt}}
\pgfpathclose
\pgfusepath{fill,stroke}
\pgfpathmoveto{\pgfpoint{275.656250pt}{140.777435pt}}
\pgflineto{\pgfpoint{276.652222pt}{140.777435pt}}
\pgflineto{\pgfpoint{276.652222pt}{141.883286pt}}
\pgfpathclose
\pgfusepath{fill,stroke}
\pgfpathmoveto{\pgfpoint{277.648193pt}{145.700470pt}}
\pgflineto{\pgfpoint{276.652222pt}{141.883286pt}}
\pgflineto{\pgfpoint{276.652222pt}{140.777435pt}}
\pgfpathclose
\pgfusepath{fill,stroke}
\pgfpathmoveto{\pgfpoint{277.648193pt}{140.777435pt}}
\pgflineto{\pgfpoint{277.648193pt}{145.700470pt}}
\pgflineto{\pgfpoint{276.652222pt}{140.777435pt}}
\pgfpathclose
\pgfusepath{fill,stroke}
\pgfpathmoveto{\pgfpoint{278.644196pt}{141.350510pt}}
\pgflineto{\pgfpoint{277.648193pt}{145.700470pt}}
\pgflineto{\pgfpoint{277.648193pt}{140.777435pt}}
\pgfpathclose
\pgfusepath{fill,stroke}
\pgfpathmoveto{\pgfpoint{278.644196pt}{140.777435pt}}
\pgflineto{\pgfpoint{278.644196pt}{141.350510pt}}
\pgflineto{\pgfpoint{277.648193pt}{140.777435pt}}
\pgfpathclose
\pgfusepath{fill,stroke}
\pgfpathmoveto{\pgfpoint{279.640167pt}{141.685425pt}}
\pgflineto{\pgfpoint{278.644196pt}{141.350510pt}}
\pgflineto{\pgfpoint{278.644196pt}{140.777435pt}}
\pgfpathclose
\pgfusepath{fill,stroke}
\pgfpathmoveto{\pgfpoint{279.640167pt}{140.777435pt}}
\pgflineto{\pgfpoint{279.640167pt}{141.685425pt}}
\pgflineto{\pgfpoint{278.644196pt}{140.777435pt}}
\pgfpathclose
\pgfusepath{fill,stroke}
\pgfpathmoveto{\pgfpoint{280.636139pt}{140.839111pt}}
\pgflineto{\pgfpoint{279.640167pt}{141.685425pt}}
\pgflineto{\pgfpoint{279.640167pt}{140.777435pt}}
\pgfpathclose
\pgfusepath{fill,stroke}
\pgfpathmoveto{\pgfpoint{280.636139pt}{140.777435pt}}
\pgflineto{\pgfpoint{280.636139pt}{140.839111pt}}
\pgflineto{\pgfpoint{279.640167pt}{140.777435pt}}
\pgfpathclose
\pgfusepath{fill,stroke}
\pgfpathmoveto{\pgfpoint{281.632141pt}{140.783401pt}}
\pgflineto{\pgfpoint{280.636139pt}{140.839111pt}}
\pgflineto{\pgfpoint{280.636139pt}{140.777435pt}}
\pgfpathclose
\pgfusepath{fill,stroke}
\pgfpathmoveto{\pgfpoint{281.632141pt}{140.777435pt}}
\pgflineto{\pgfpoint{281.632141pt}{140.783401pt}}
\pgflineto{\pgfpoint{280.636139pt}{140.777435pt}}
\pgfpathclose
\pgfusepath{fill,stroke}
\pgfpathmoveto{\pgfpoint{282.628113pt}{141.247986pt}}
\pgflineto{\pgfpoint{281.632141pt}{140.783401pt}}
\pgflineto{\pgfpoint{281.632141pt}{140.777435pt}}
\pgfpathclose
\pgfusepath{fill,stroke}
\pgfpathmoveto{\pgfpoint{282.628113pt}{140.777435pt}}
\pgflineto{\pgfpoint{282.628113pt}{141.247986pt}}
\pgflineto{\pgfpoint{281.632141pt}{140.777435pt}}
\pgfpathclose
\pgfusepath{fill,stroke}
\pgfpathmoveto{\pgfpoint{283.624115pt}{140.777435pt}}
\pgflineto{\pgfpoint{282.628113pt}{141.247986pt}}
\pgflineto{\pgfpoint{282.628113pt}{140.777435pt}}
\pgfpathclose
\pgfusepath{fill,stroke}
\pgfpathmoveto{\pgfpoint{262.708435pt}{140.777435pt}}
\pgflineto{\pgfpoint{263.704407pt}{140.777435pt}}
\pgflineto{\pgfpoint{263.704407pt}{149.672318pt}}
\pgfpathclose
\pgfusepath{fill,stroke}
\pgfpathmoveto{\pgfpoint{264.700409pt}{150.470245pt}}
\pgflineto{\pgfpoint{263.704407pt}{149.672318pt}}
\pgflineto{\pgfpoint{263.704407pt}{140.777435pt}}
\pgfpathclose
\pgfusepath{fill,stroke}
\pgfpathmoveto{\pgfpoint{264.700409pt}{140.777435pt}}
\pgflineto{\pgfpoint{264.700409pt}{150.470245pt}}
\pgflineto{\pgfpoint{263.704407pt}{140.777435pt}}
\pgfpathclose
\pgfusepath{fill,stroke}
\pgfpathmoveto{\pgfpoint{265.696411pt}{146.772659pt}}
\pgflineto{\pgfpoint{264.700409pt}{150.470245pt}}
\pgflineto{\pgfpoint{264.700409pt}{140.777435pt}}
\pgfpathclose
\pgfusepath{fill,stroke}
\pgfpathmoveto{\pgfpoint{265.696411pt}{140.777435pt}}
\pgflineto{\pgfpoint{265.696411pt}{146.772659pt}}
\pgflineto{\pgfpoint{264.700409pt}{140.777435pt}}
\pgfpathclose
\pgfusepath{fill,stroke}
\pgfpathmoveto{\pgfpoint{266.692383pt}{141.012543pt}}
\pgflineto{\pgfpoint{265.696411pt}{146.772659pt}}
\pgflineto{\pgfpoint{265.696411pt}{140.777435pt}}
\pgfpathclose
\pgfusepath{fill,stroke}
\pgfpathmoveto{\pgfpoint{266.692383pt}{140.777435pt}}
\pgflineto{\pgfpoint{266.692383pt}{141.012543pt}}
\pgflineto{\pgfpoint{265.696411pt}{140.777435pt}}
\pgfpathclose
\pgfusepath{fill,stroke}
\pgfpathmoveto{\pgfpoint{267.688354pt}{140.947601pt}}
\pgflineto{\pgfpoint{266.692383pt}{141.012543pt}}
\pgflineto{\pgfpoint{266.692383pt}{140.777435pt}}
\pgfpathclose
\pgfusepath{fill,stroke}
\pgfpathmoveto{\pgfpoint{267.688354pt}{140.777435pt}}
\pgflineto{\pgfpoint{267.688354pt}{140.947601pt}}
\pgflineto{\pgfpoint{266.692383pt}{140.777435pt}}
\pgfpathclose
\pgfusepath{fill,stroke}
\pgfpathmoveto{\pgfpoint{268.684326pt}{165.780823pt}}
\pgflineto{\pgfpoint{267.688354pt}{140.947601pt}}
\pgflineto{\pgfpoint{267.688354pt}{140.777435pt}}
\pgfpathclose
\pgfusepath{fill,stroke}
\pgfpathmoveto{\pgfpoint{268.684326pt}{140.777435pt}}
\pgflineto{\pgfpoint{268.684326pt}{165.780823pt}}
\pgflineto{\pgfpoint{267.688354pt}{140.777435pt}}
\pgfpathclose
\pgfusepath{fill,stroke}
\pgfpathmoveto{\pgfpoint{269.680328pt}{142.178589pt}}
\pgflineto{\pgfpoint{268.684326pt}{165.780823pt}}
\pgflineto{\pgfpoint{268.684326pt}{140.777435pt}}
\pgfpathclose
\pgfusepath{fill,stroke}
\pgfpathmoveto{\pgfpoint{269.680328pt}{140.777435pt}}
\pgflineto{\pgfpoint{269.680328pt}{142.178589pt}}
\pgflineto{\pgfpoint{268.684326pt}{140.777435pt}}
\pgfpathclose
\pgfusepath{fill,stroke}
\pgfpathmoveto{\pgfpoint{270.676331pt}{140.852936pt}}
\pgflineto{\pgfpoint{269.680328pt}{142.178589pt}}
\pgflineto{\pgfpoint{269.680328pt}{140.777435pt}}
\pgfpathclose
\pgfusepath{fill,stroke}
\pgfpathmoveto{\pgfpoint{270.676331pt}{140.777435pt}}
\pgflineto{\pgfpoint{270.676331pt}{140.852936pt}}
\pgflineto{\pgfpoint{269.680328pt}{140.777435pt}}
\pgfpathclose
\pgfusepath{fill,stroke}
\pgfpathmoveto{\pgfpoint{271.672302pt}{149.853119pt}}
\pgflineto{\pgfpoint{270.676331pt}{140.852936pt}}
\pgflineto{\pgfpoint{270.676331pt}{140.777435pt}}
\pgfpathclose
\pgfusepath{fill,stroke}
\pgfpathmoveto{\pgfpoint{271.672302pt}{140.777435pt}}
\pgflineto{\pgfpoint{271.672302pt}{149.853119pt}}
\pgflineto{\pgfpoint{270.676331pt}{140.777435pt}}
\pgfpathclose
\pgfusepath{fill,stroke}
\pgfpathmoveto{\pgfpoint{272.668274pt}{167.198486pt}}
\pgflineto{\pgfpoint{271.672302pt}{149.853119pt}}
\pgflineto{\pgfpoint{271.672302pt}{140.777435pt}}
\pgfpathclose
\pgfusepath{fill,stroke}
\pgfpathmoveto{\pgfpoint{272.668274pt}{140.777435pt}}
\pgflineto{\pgfpoint{272.668274pt}{167.198486pt}}
\pgflineto{\pgfpoint{271.672302pt}{140.777435pt}}
\pgfpathclose
\pgfusepath{fill,stroke}
\pgfpathmoveto{\pgfpoint{273.664276pt}{140.777435pt}}
\pgflineto{\pgfpoint{272.668274pt}{167.198486pt}}
\pgflineto{\pgfpoint{272.668274pt}{140.777435pt}}
\pgfpathclose
\pgfusepath{fill,stroke}
\pgfpathmoveto{\pgfpoint{255.736542pt}{140.777435pt}}
\pgflineto{\pgfpoint{256.732544pt}{140.777435pt}}
\pgflineto{\pgfpoint{256.732544pt}{160.055664pt}}
\pgfpathclose
\pgfusepath{fill,stroke}
\pgfpathmoveto{\pgfpoint{257.728516pt}{145.276688pt}}
\pgflineto{\pgfpoint{256.732544pt}{160.055664pt}}
\pgflineto{\pgfpoint{256.732544pt}{140.777435pt}}
\pgfpathclose
\pgfusepath{fill,stroke}
\pgfpathmoveto{\pgfpoint{257.728516pt}{140.777435pt}}
\pgflineto{\pgfpoint{257.728516pt}{145.276688pt}}
\pgflineto{\pgfpoint{256.732544pt}{140.777435pt}}
\pgfpathclose
\pgfusepath{fill,stroke}
\pgfpathmoveto{\pgfpoint{258.724518pt}{145.143265pt}}
\pgflineto{\pgfpoint{257.728516pt}{145.276688pt}}
\pgflineto{\pgfpoint{257.728516pt}{140.777435pt}}
\pgfpathclose
\pgfusepath{fill,stroke}
\pgfpathmoveto{\pgfpoint{258.724518pt}{140.777435pt}}
\pgflineto{\pgfpoint{258.724518pt}{145.143265pt}}
\pgflineto{\pgfpoint{257.728516pt}{140.777435pt}}
\pgfpathclose
\pgfusepath{fill,stroke}
\pgfpathmoveto{\pgfpoint{259.720490pt}{141.107025pt}}
\pgflineto{\pgfpoint{258.724518pt}{145.143265pt}}
\pgflineto{\pgfpoint{258.724518pt}{140.777435pt}}
\pgfpathclose
\pgfusepath{fill,stroke}
\pgfpathmoveto{\pgfpoint{259.720490pt}{140.777435pt}}
\pgflineto{\pgfpoint{259.720490pt}{141.107025pt}}
\pgflineto{\pgfpoint{258.724518pt}{140.777435pt}}
\pgfpathclose
\pgfusepath{fill,stroke}
\pgfpathmoveto{\pgfpoint{260.716492pt}{140.785751pt}}
\pgflineto{\pgfpoint{259.720490pt}{141.107025pt}}
\pgflineto{\pgfpoint{259.720490pt}{140.777435pt}}
\pgfpathclose
\pgfusepath{fill,stroke}
\pgfpathmoveto{\pgfpoint{260.716492pt}{140.777435pt}}
\pgflineto{\pgfpoint{260.716492pt}{140.785751pt}}
\pgflineto{\pgfpoint{259.720490pt}{140.777435pt}}
\pgfpathclose
\pgfusepath{fill,stroke}
\pgfpathmoveto{\pgfpoint{261.712463pt}{140.881470pt}}
\pgflineto{\pgfpoint{260.716492pt}{140.785751pt}}
\pgflineto{\pgfpoint{260.716492pt}{140.777435pt}}
\pgfpathclose
\pgfusepath{fill,stroke}
\pgfpathmoveto{\pgfpoint{261.712463pt}{140.777435pt}}
\pgflineto{\pgfpoint{261.712463pt}{140.881470pt}}
\pgflineto{\pgfpoint{260.716492pt}{140.777435pt}}
\pgfpathclose
\pgfusepath{fill,stroke}
\pgfpathmoveto{\pgfpoint{262.708435pt}{140.777435pt}}
\pgflineto{\pgfpoint{261.712463pt}{140.881470pt}}
\pgflineto{\pgfpoint{261.712463pt}{140.777435pt}}
\pgfpathclose
\pgfusepath{fill,stroke}
\pgfpathmoveto{\pgfpoint{252.748611pt}{140.777435pt}}
\pgflineto{\pgfpoint{253.744598pt}{205.642242pt}}
\pgflineto{\pgfpoint{253.234100pt}{205.642242pt}}
\pgfpathclose
\pgfusepath{fill,stroke}
\pgfpathmoveto{\pgfpoint{252.748611pt}{140.777435pt}}
\pgflineto{\pgfpoint{253.744598pt}{140.777435pt}}
\pgflineto{\pgfpoint{253.744598pt}{205.642242pt}}
\pgfpathclose
\pgfusepath{fill,stroke}
\pgfpathmoveto{\pgfpoint{254.325424pt}{205.642242pt}}
\pgflineto{\pgfpoint{253.744598pt}{205.642242pt}}
\pgflineto{\pgfpoint{253.744598pt}{140.777435pt}}
\pgfpathclose
\pgfusepath{fill,stroke}
\pgfpathmoveto{\pgfpoint{254.740570pt}{140.777435pt}}
\pgflineto{\pgfpoint{254.325424pt}{205.642242pt}}
\pgflineto{\pgfpoint{253.744598pt}{140.777435pt}}
\pgfpathclose
\pgfusepath{fill,stroke}
\pgfpathmoveto{\pgfpoint{254.740570pt}{140.777435pt}}
\pgflineto{\pgfpoint{254.740570pt}{205.642242pt}}
\pgflineto{\pgfpoint{254.325424pt}{205.642242pt}}
\pgfpathclose
\pgfusepath{fill,stroke}
\pgfpathmoveto{\pgfpoint{255.736542pt}{140.777435pt}}
\pgflineto{\pgfpoint{254.740570pt}{205.642242pt}}
\pgflineto{\pgfpoint{254.740570pt}{140.777435pt}}
\pgfpathclose
\pgfusepath{fill,stroke}
\pgfpathmoveto{\pgfpoint{255.736542pt}{140.777435pt}}
\pgflineto{\pgfpoint{255.155716pt}{205.642242pt}}
\pgflineto{\pgfpoint{254.740570pt}{205.642242pt}}
\pgfpathclose
\pgfusepath{fill,stroke}
\pgfpathmoveto{\pgfpoint{244.780731pt}{140.777435pt}}
\pgflineto{\pgfpoint{245.776718pt}{140.777435pt}}
\pgflineto{\pgfpoint{245.776718pt}{140.921082pt}}
\pgfpathclose
\pgfusepath{fill,stroke}
\pgfpathmoveto{\pgfpoint{246.231064pt}{205.642242pt}}
\pgflineto{\pgfpoint{245.776718pt}{140.921082pt}}
\pgflineto{\pgfpoint{245.776718pt}{140.777435pt}}
\pgfpathclose
\pgfusepath{fill,stroke}
\pgfpathmoveto{\pgfpoint{246.231064pt}{205.642242pt}}
\pgflineto{\pgfpoint{246.230515pt}{205.642242pt}}
\pgflineto{\pgfpoint{245.776718pt}{140.921082pt}}
\pgfpathclose
\pgfusepath{fill,stroke}
\pgfpathmoveto{\pgfpoint{246.772705pt}{140.777435pt}}
\pgflineto{\pgfpoint{246.231064pt}{205.642242pt}}
\pgflineto{\pgfpoint{245.776718pt}{140.777435pt}}
\pgfpathclose
\pgfusepath{fill,stroke}
\pgfpathmoveto{\pgfpoint{246.772705pt}{140.777435pt}}
\pgflineto{\pgfpoint{246.772705pt}{205.642242pt}}
\pgflineto{\pgfpoint{246.231064pt}{205.642242pt}}
\pgfpathclose
\pgfusepath{fill,stroke}
\pgfpathmoveto{\pgfpoint{247.768677pt}{144.035980pt}}
\pgflineto{\pgfpoint{246.772705pt}{205.642242pt}}
\pgflineto{\pgfpoint{246.772705pt}{140.777435pt}}
\pgfpathclose
\pgfusepath{fill,stroke}
\pgfpathmoveto{\pgfpoint{247.768677pt}{144.035980pt}}
\pgflineto{\pgfpoint{247.327042pt}{205.642242pt}}
\pgflineto{\pgfpoint{246.772705pt}{205.642242pt}}
\pgfpathclose
\pgfusepath{fill,stroke}
\pgfpathmoveto{\pgfpoint{247.768677pt}{140.777435pt}}
\pgflineto{\pgfpoint{247.768677pt}{144.035980pt}}
\pgflineto{\pgfpoint{246.772705pt}{140.777435pt}}
\pgfpathclose
\pgfusepath{fill,stroke}
\pgfpathmoveto{\pgfpoint{248.764679pt}{140.987183pt}}
\pgflineto{\pgfpoint{247.768677pt}{144.035980pt}}
\pgflineto{\pgfpoint{247.768677pt}{140.777435pt}}
\pgfpathclose
\pgfusepath{fill,stroke}
\pgfpathmoveto{\pgfpoint{248.764679pt}{140.777435pt}}
\pgflineto{\pgfpoint{248.764679pt}{140.987183pt}}
\pgflineto{\pgfpoint{247.768677pt}{140.777435pt}}
\pgfpathclose
\pgfusepath{fill,stroke}
\pgfpathmoveto{\pgfpoint{249.760651pt}{140.861908pt}}
\pgflineto{\pgfpoint{248.764679pt}{140.987183pt}}
\pgflineto{\pgfpoint{248.764679pt}{140.777435pt}}
\pgfpathclose
\pgfusepath{fill,stroke}
\pgfpathmoveto{\pgfpoint{249.760651pt}{140.777435pt}}
\pgflineto{\pgfpoint{249.760651pt}{140.861908pt}}
\pgflineto{\pgfpoint{248.764679pt}{140.777435pt}}
\pgfpathclose
\pgfusepath{fill,stroke}
\pgfpathmoveto{\pgfpoint{250.756638pt}{143.997910pt}}
\pgflineto{\pgfpoint{249.760651pt}{140.861908pt}}
\pgflineto{\pgfpoint{249.760651pt}{140.777435pt}}
\pgfpathclose
\pgfusepath{fill,stroke}
\pgfpathmoveto{\pgfpoint{250.756638pt}{140.777435pt}}
\pgflineto{\pgfpoint{250.756638pt}{143.997910pt}}
\pgflineto{\pgfpoint{249.760651pt}{140.777435pt}}
\pgfpathclose
\pgfusepath{fill,stroke}
\pgfpathmoveto{\pgfpoint{251.752625pt}{140.777435pt}}
\pgflineto{\pgfpoint{250.756638pt}{143.997910pt}}
\pgflineto{\pgfpoint{250.756638pt}{140.777435pt}}
\pgfpathclose
\pgfusepath{fill,stroke}
\pgfpathmoveto{\pgfpoint{241.792786pt}{140.777435pt}}
\pgflineto{\pgfpoint{242.788757pt}{140.777435pt}}
\pgflineto{\pgfpoint{242.788757pt}{141.050537pt}}
\pgfpathclose
\pgfusepath{fill,stroke}
\pgfpathmoveto{\pgfpoint{243.784744pt}{140.826157pt}}
\pgflineto{\pgfpoint{242.788757pt}{141.050537pt}}
\pgflineto{\pgfpoint{242.788757pt}{140.777435pt}}
\pgfpathclose
\pgfusepath{fill,stroke}
\pgfpathmoveto{\pgfpoint{243.784744pt}{140.777435pt}}
\pgflineto{\pgfpoint{243.784744pt}{140.826157pt}}
\pgflineto{\pgfpoint{242.788757pt}{140.777435pt}}
\pgfpathclose
\pgfusepath{fill,stroke}
\pgfpathmoveto{\pgfpoint{244.780731pt}{140.777435pt}}
\pgflineto{\pgfpoint{243.784744pt}{140.826157pt}}
\pgflineto{\pgfpoint{243.784744pt}{140.777435pt}}
\pgfpathclose
\pgfusepath{fill,stroke}
\pgfpathmoveto{\pgfpoint{239.800812pt}{140.777435pt}}
\pgflineto{\pgfpoint{240.796814pt}{140.777435pt}}
\pgflineto{\pgfpoint{240.796814pt}{141.635849pt}}
\pgfpathclose
\pgfusepath{fill,stroke}
\pgfpathmoveto{\pgfpoint{241.792786pt}{140.777435pt}}
\pgflineto{\pgfpoint{240.796814pt}{141.635849pt}}
\pgflineto{\pgfpoint{240.796814pt}{140.777435pt}}
\pgfpathclose
\pgfusepath{fill,stroke}
\pgfpathmoveto{\pgfpoint{232.828934pt}{140.777435pt}}
\pgflineto{\pgfpoint{233.824921pt}{140.777435pt}}
\pgflineto{\pgfpoint{233.824921pt}{143.663483pt}}
\pgfpathclose
\pgfusepath{fill,stroke}
\pgfpathmoveto{\pgfpoint{234.820892pt}{144.424850pt}}
\pgflineto{\pgfpoint{233.824921pt}{143.663483pt}}
\pgflineto{\pgfpoint{233.824921pt}{140.777435pt}}
\pgfpathclose
\pgfusepath{fill,stroke}
\pgfpathmoveto{\pgfpoint{234.820892pt}{140.777435pt}}
\pgflineto{\pgfpoint{234.820892pt}{144.424850pt}}
\pgflineto{\pgfpoint{233.824921pt}{140.777435pt}}
\pgfpathclose
\pgfusepath{fill,stroke}
\pgfpathmoveto{\pgfpoint{235.816864pt}{141.070068pt}}
\pgflineto{\pgfpoint{234.820892pt}{144.424850pt}}
\pgflineto{\pgfpoint{234.820892pt}{140.777435pt}}
\pgfpathclose
\pgfusepath{fill,stroke}
\pgfpathmoveto{\pgfpoint{235.816864pt}{140.777435pt}}
\pgflineto{\pgfpoint{235.816864pt}{141.070068pt}}
\pgflineto{\pgfpoint{234.820892pt}{140.777435pt}}
\pgfpathclose
\pgfusepath{fill,stroke}
\pgfpathmoveto{\pgfpoint{236.812866pt}{141.387177pt}}
\pgflineto{\pgfpoint{235.816864pt}{141.070068pt}}
\pgflineto{\pgfpoint{235.816864pt}{140.777435pt}}
\pgfpathclose
\pgfusepath{fill,stroke}
\pgfpathmoveto{\pgfpoint{236.812866pt}{140.777435pt}}
\pgflineto{\pgfpoint{236.812866pt}{141.387177pt}}
\pgflineto{\pgfpoint{235.816864pt}{140.777435pt}}
\pgfpathclose
\pgfusepath{fill,stroke}
\pgfpathmoveto{\pgfpoint{237.808838pt}{141.605896pt}}
\pgflineto{\pgfpoint{236.812866pt}{141.387177pt}}
\pgflineto{\pgfpoint{236.812866pt}{140.777435pt}}
\pgfpathclose
\pgfusepath{fill,stroke}
\pgfpathmoveto{\pgfpoint{237.808838pt}{140.777435pt}}
\pgflineto{\pgfpoint{237.808838pt}{141.605896pt}}
\pgflineto{\pgfpoint{236.812866pt}{140.777435pt}}
\pgfpathclose
\pgfusepath{fill,stroke}
\pgfpathmoveto{\pgfpoint{238.804825pt}{141.892487pt}}
\pgflineto{\pgfpoint{237.808838pt}{141.605896pt}}
\pgflineto{\pgfpoint{237.808838pt}{140.777435pt}}
\pgfpathclose
\pgfusepath{fill,stroke}
\pgfpathmoveto{\pgfpoint{238.804825pt}{140.777435pt}}
\pgflineto{\pgfpoint{238.804825pt}{141.892487pt}}
\pgflineto{\pgfpoint{237.808838pt}{140.777435pt}}
\pgfpathclose
\pgfusepath{fill,stroke}
\pgfpathmoveto{\pgfpoint{239.800812pt}{140.777435pt}}
\pgflineto{\pgfpoint{238.804825pt}{141.892487pt}}
\pgflineto{\pgfpoint{238.804825pt}{140.777435pt}}
\pgfpathclose
\pgfusepath{fill,stroke}
\pgfpathmoveto{\pgfpoint{226.853027pt}{140.777435pt}}
\pgflineto{\pgfpoint{227.849014pt}{140.777435pt}}
\pgflineto{\pgfpoint{227.849014pt}{141.269379pt}}
\pgfpathclose
\pgfusepath{fill,stroke}
\pgfpathmoveto{\pgfpoint{228.845001pt}{143.487183pt}}
\pgflineto{\pgfpoint{227.849014pt}{141.269379pt}}
\pgflineto{\pgfpoint{227.849014pt}{140.777435pt}}
\pgfpathclose
\pgfusepath{fill,stroke}
\pgfpathmoveto{\pgfpoint{228.845001pt}{140.777435pt}}
\pgflineto{\pgfpoint{228.845001pt}{143.487183pt}}
\pgflineto{\pgfpoint{227.849014pt}{140.777435pt}}
\pgfpathclose
\pgfusepath{fill,stroke}
\pgfpathmoveto{\pgfpoint{229.840973pt}{143.864670pt}}
\pgflineto{\pgfpoint{228.845001pt}{143.487183pt}}
\pgflineto{\pgfpoint{228.845001pt}{140.777435pt}}
\pgfpathclose
\pgfusepath{fill,stroke}
\pgfpathmoveto{\pgfpoint{229.840973pt}{140.777435pt}}
\pgflineto{\pgfpoint{229.840973pt}{143.864670pt}}
\pgflineto{\pgfpoint{228.845001pt}{140.777435pt}}
\pgfpathclose
\pgfusepath{fill,stroke}
\pgfpathmoveto{\pgfpoint{230.836945pt}{141.428329pt}}
\pgflineto{\pgfpoint{229.840973pt}{143.864670pt}}
\pgflineto{\pgfpoint{229.840973pt}{140.777435pt}}
\pgfpathclose
\pgfusepath{fill,stroke}
\pgfpathmoveto{\pgfpoint{230.836945pt}{140.777435pt}}
\pgflineto{\pgfpoint{230.836945pt}{141.428329pt}}
\pgflineto{\pgfpoint{229.840973pt}{140.777435pt}}
\pgfpathclose
\pgfusepath{fill,stroke}
\pgfpathmoveto{\pgfpoint{231.832932pt}{162.511902pt}}
\pgflineto{\pgfpoint{230.836945pt}{141.428329pt}}
\pgflineto{\pgfpoint{230.836945pt}{140.777435pt}}
\pgfpathclose
\pgfusepath{fill,stroke}
\pgfpathmoveto{\pgfpoint{231.832932pt}{140.777435pt}}
\pgflineto{\pgfpoint{231.832932pt}{162.511902pt}}
\pgflineto{\pgfpoint{230.836945pt}{140.777435pt}}
\pgfpathclose
\pgfusepath{fill,stroke}
\pgfpathmoveto{\pgfpoint{232.828934pt}{140.777435pt}}
\pgflineto{\pgfpoint{231.832932pt}{162.511902pt}}
\pgflineto{\pgfpoint{231.832932pt}{140.777435pt}}
\pgfpathclose
\pgfusepath{fill,stroke}
\pgfpathmoveto{\pgfpoint{222.869080pt}{140.777435pt}}
\pgflineto{\pgfpoint{223.865082pt}{140.777435pt}}
\pgflineto{\pgfpoint{223.865082pt}{142.648438pt}}
\pgfpathclose
\pgfusepath{fill,stroke}
\pgfpathmoveto{\pgfpoint{224.861053pt}{140.814316pt}}
\pgflineto{\pgfpoint{223.865082pt}{142.648438pt}}
\pgflineto{\pgfpoint{223.865082pt}{140.777435pt}}
\pgfpathclose
\pgfusepath{fill,stroke}
\pgfpathmoveto{\pgfpoint{224.861053pt}{140.777435pt}}
\pgflineto{\pgfpoint{224.861053pt}{140.814316pt}}
\pgflineto{\pgfpoint{223.865082pt}{140.777435pt}}
\pgfpathclose
\pgfusepath{fill,stroke}
\pgfpathmoveto{\pgfpoint{225.857040pt}{142.893295pt}}
\pgflineto{\pgfpoint{224.861053pt}{140.814316pt}}
\pgflineto{\pgfpoint{224.861053pt}{140.777435pt}}
\pgfpathclose
\pgfusepath{fill,stroke}
\pgfpathmoveto{\pgfpoint{225.857040pt}{140.777435pt}}
\pgflineto{\pgfpoint{225.857040pt}{142.893295pt}}
\pgflineto{\pgfpoint{224.861053pt}{140.777435pt}}
\pgfpathclose
\pgfusepath{fill,stroke}
\pgfpathmoveto{\pgfpoint{226.853027pt}{140.777435pt}}
\pgflineto{\pgfpoint{225.857040pt}{142.893295pt}}
\pgflineto{\pgfpoint{225.857040pt}{140.777435pt}}
\pgfpathclose
\pgfusepath{fill,stroke}
\pgfpathmoveto{\pgfpoint{218.885147pt}{140.777435pt}}
\pgflineto{\pgfpoint{219.881134pt}{140.777435pt}}
\pgflineto{\pgfpoint{219.881134pt}{141.240646pt}}
\pgfpathclose
\pgfusepath{fill,stroke}
\pgfpathmoveto{\pgfpoint{220.877121pt}{140.777435pt}}
\pgflineto{\pgfpoint{219.881134pt}{141.240646pt}}
\pgflineto{\pgfpoint{219.881134pt}{140.777435pt}}
\pgfpathclose
\pgfusepath{fill,stroke}
\pgfpathmoveto{\pgfpoint{216.893188pt}{140.777435pt}}
\pgflineto{\pgfpoint{217.889160pt}{140.777435pt}}
\pgflineto{\pgfpoint{217.889160pt}{144.868698pt}}
\pgfpathclose
\pgfusepath{fill,stroke}
\pgfpathmoveto{\pgfpoint{218.885147pt}{140.777435pt}}
\pgflineto{\pgfpoint{217.889160pt}{144.868698pt}}
\pgflineto{\pgfpoint{217.889160pt}{140.777435pt}}
\pgfpathclose
\pgfusepath{fill,stroke}
\pgfpathmoveto{\pgfpoint{207.929337pt}{140.777435pt}}
\pgflineto{\pgfpoint{208.925323pt}{140.777435pt}}
\pgflineto{\pgfpoint{208.925323pt}{168.977936pt}}
\pgfpathclose
\pgfusepath{fill,stroke}
\pgfpathmoveto{\pgfpoint{209.716339pt}{205.642242pt}}
\pgflineto{\pgfpoint{208.925323pt}{168.977936pt}}
\pgflineto{\pgfpoint{208.925323pt}{140.777435pt}}
\pgfpathclose
\pgfusepath{fill,stroke}
\pgfpathmoveto{\pgfpoint{209.716339pt}{205.642242pt}}
\pgflineto{\pgfpoint{209.608246pt}{205.642242pt}}
\pgflineto{\pgfpoint{208.925323pt}{168.977936pt}}
\pgfpathclose
\pgfusepath{fill,stroke}
\pgfpathmoveto{\pgfpoint{209.921295pt}{140.777435pt}}
\pgflineto{\pgfpoint{209.716339pt}{205.642242pt}}
\pgflineto{\pgfpoint{208.925323pt}{140.777435pt}}
\pgfpathclose
\pgfusepath{fill,stroke}
\pgfpathmoveto{\pgfpoint{209.921295pt}{140.777435pt}}
\pgflineto{\pgfpoint{209.921295pt}{205.642242pt}}
\pgflineto{\pgfpoint{209.716339pt}{205.642242pt}}
\pgfpathclose
\pgfusepath{fill,stroke}
\pgfpathmoveto{\pgfpoint{210.917267pt}{156.156219pt}}
\pgflineto{\pgfpoint{209.921295pt}{205.642242pt}}
\pgflineto{\pgfpoint{209.921295pt}{140.777435pt}}
\pgfpathclose
\pgfusepath{fill,stroke}
\pgfpathmoveto{\pgfpoint{210.917267pt}{156.156219pt}}
\pgflineto{\pgfpoint{210.173798pt}{205.642242pt}}
\pgflineto{\pgfpoint{209.921295pt}{205.642242pt}}
\pgfpathclose
\pgfusepath{fill,stroke}
\pgfpathmoveto{\pgfpoint{210.917267pt}{140.777435pt}}
\pgflineto{\pgfpoint{210.917267pt}{156.156219pt}}
\pgflineto{\pgfpoint{209.921295pt}{140.777435pt}}
\pgfpathclose
\pgfusepath{fill,stroke}
\pgfpathmoveto{\pgfpoint{211.913269pt}{142.768951pt}}
\pgflineto{\pgfpoint{210.917267pt}{156.156219pt}}
\pgflineto{\pgfpoint{210.917267pt}{140.777435pt}}
\pgfpathclose
\pgfusepath{fill,stroke}
\pgfpathmoveto{\pgfpoint{211.913269pt}{140.777435pt}}
\pgflineto{\pgfpoint{211.913269pt}{142.768951pt}}
\pgflineto{\pgfpoint{210.917267pt}{140.777435pt}}
\pgfpathclose
\pgfusepath{fill,stroke}
\pgfpathmoveto{\pgfpoint{212.909241pt}{140.796997pt}}
\pgflineto{\pgfpoint{211.913269pt}{142.768951pt}}
\pgflineto{\pgfpoint{211.913269pt}{140.777435pt}}
\pgfpathclose
\pgfusepath{fill,stroke}
\pgfpathmoveto{\pgfpoint{212.909241pt}{140.777435pt}}
\pgflineto{\pgfpoint{212.909241pt}{140.796997pt}}
\pgflineto{\pgfpoint{211.913269pt}{140.777435pt}}
\pgfpathclose
\pgfusepath{fill,stroke}
\pgfpathmoveto{\pgfpoint{213.905228pt}{147.912155pt}}
\pgflineto{\pgfpoint{212.909241pt}{140.796997pt}}
\pgflineto{\pgfpoint{212.909241pt}{140.777435pt}}
\pgfpathclose
\pgfusepath{fill,stroke}
\pgfpathmoveto{\pgfpoint{213.905228pt}{140.777435pt}}
\pgflineto{\pgfpoint{213.905228pt}{147.912155pt}}
\pgflineto{\pgfpoint{212.909241pt}{140.777435pt}}
\pgfpathclose
\pgfusepath{fill,stroke}
\pgfpathmoveto{\pgfpoint{214.901215pt}{141.340469pt}}
\pgflineto{\pgfpoint{213.905228pt}{147.912155pt}}
\pgflineto{\pgfpoint{213.905228pt}{140.777435pt}}
\pgfpathclose
\pgfusepath{fill,stroke}
\pgfpathmoveto{\pgfpoint{214.901215pt}{140.777435pt}}
\pgflineto{\pgfpoint{214.901215pt}{141.340469pt}}
\pgflineto{\pgfpoint{213.905228pt}{140.777435pt}}
\pgfpathclose
\pgfusepath{fill,stroke}
\pgfpathmoveto{\pgfpoint{215.897217pt}{146.916656pt}}
\pgflineto{\pgfpoint{214.901215pt}{141.340469pt}}
\pgflineto{\pgfpoint{214.901215pt}{140.777435pt}}
\pgfpathclose
\pgfusepath{fill,stroke}
\pgfpathmoveto{\pgfpoint{215.897217pt}{140.777435pt}}
\pgflineto{\pgfpoint{215.897217pt}{146.916656pt}}
\pgflineto{\pgfpoint{214.901215pt}{140.777435pt}}
\pgfpathclose
\pgfusepath{fill,stroke}
\pgfpathmoveto{\pgfpoint{216.893188pt}{140.777435pt}}
\pgflineto{\pgfpoint{215.897217pt}{146.916656pt}}
\pgflineto{\pgfpoint{215.897217pt}{140.777435pt}}
\pgfpathclose
\pgfusepath{fill,stroke}
\pgfpathmoveto{\pgfpoint{204.941376pt}{140.777435pt}}
\pgflineto{\pgfpoint{205.937347pt}{140.777435pt}}
\pgflineto{\pgfpoint{205.937347pt}{144.822937pt}}
\pgfpathclose
\pgfusepath{fill,stroke}
\pgfpathmoveto{\pgfpoint{206.933334pt}{140.777435pt}}
\pgflineto{\pgfpoint{205.937347pt}{144.822937pt}}
\pgflineto{\pgfpoint{205.937347pt}{140.777435pt}}
\pgfpathclose
\pgfusepath{fill,stroke}
\pgfpathmoveto{\pgfpoint{199.961456pt}{140.777435pt}}
\pgflineto{\pgfpoint{200.957443pt}{140.777435pt}}
\pgflineto{\pgfpoint{200.957443pt}{141.105011pt}}
\pgfpathclose
\pgfusepath{fill,stroke}
\pgfpathmoveto{\pgfpoint{201.953430pt}{140.916779pt}}
\pgflineto{\pgfpoint{200.957443pt}{141.105011pt}}
\pgflineto{\pgfpoint{200.957443pt}{140.777435pt}}
\pgfpathclose
\pgfusepath{fill,stroke}
\pgfpathmoveto{\pgfpoint{201.953430pt}{140.777435pt}}
\pgflineto{\pgfpoint{201.953430pt}{140.916779pt}}
\pgflineto{\pgfpoint{200.957443pt}{140.777435pt}}
\pgfpathclose
\pgfusepath{fill,stroke}
\pgfpathmoveto{\pgfpoint{202.949402pt}{140.777435pt}}
\pgflineto{\pgfpoint{201.953430pt}{140.916779pt}}
\pgflineto{\pgfpoint{201.953430pt}{140.777435pt}}
\pgfpathclose
\pgfusepath{fill,stroke}
\pgfpathmoveto{\pgfpoint{188.009659pt}{140.777435pt}}
\pgflineto{\pgfpoint{189.005630pt}{140.777435pt}}
\pgflineto{\pgfpoint{189.005630pt}{140.805771pt}}
\pgfpathclose
\pgfusepath{fill,stroke}
\pgfpathmoveto{\pgfpoint{190.001617pt}{149.564423pt}}
\pgflineto{\pgfpoint{189.005630pt}{140.805771pt}}
\pgflineto{\pgfpoint{189.005630pt}{140.777435pt}}
\pgfpathclose
\pgfusepath{fill,stroke}
\pgfpathmoveto{\pgfpoint{190.001617pt}{140.777435pt}}
\pgflineto{\pgfpoint{190.001617pt}{149.564423pt}}
\pgflineto{\pgfpoint{189.005630pt}{140.777435pt}}
\pgfpathclose
\pgfusepath{fill,stroke}
\pgfpathmoveto{\pgfpoint{190.997604pt}{140.863358pt}}
\pgflineto{\pgfpoint{190.001617pt}{149.564423pt}}
\pgflineto{\pgfpoint{190.001617pt}{140.777435pt}}
\pgfpathclose
\pgfusepath{fill,stroke}
\pgfpathmoveto{\pgfpoint{190.997604pt}{140.777435pt}}
\pgflineto{\pgfpoint{190.997604pt}{140.863358pt}}
\pgflineto{\pgfpoint{190.001617pt}{140.777435pt}}
\pgfpathclose
\pgfusepath{fill,stroke}
\pgfpathmoveto{\pgfpoint{191.993591pt}{142.184387pt}}
\pgflineto{\pgfpoint{190.997604pt}{140.863358pt}}
\pgflineto{\pgfpoint{190.997604pt}{140.777435pt}}
\pgfpathclose
\pgfusepath{fill,stroke}
\pgfpathmoveto{\pgfpoint{191.993591pt}{140.777435pt}}
\pgflineto{\pgfpoint{191.993591pt}{142.184387pt}}
\pgflineto{\pgfpoint{190.997604pt}{140.777435pt}}
\pgfpathclose
\pgfusepath{fill,stroke}
\pgfpathmoveto{\pgfpoint{192.989563pt}{195.151352pt}}
\pgflineto{\pgfpoint{191.993591pt}{142.184387pt}}
\pgflineto{\pgfpoint{191.993591pt}{140.777435pt}}
\pgfpathclose
\pgfusepath{fill,stroke}
\pgfpathmoveto{\pgfpoint{192.989563pt}{140.777435pt}}
\pgflineto{\pgfpoint{192.989563pt}{195.151352pt}}
\pgflineto{\pgfpoint{191.993591pt}{140.777435pt}}
\pgfpathclose
\pgfusepath{fill,stroke}
\pgfpathmoveto{\pgfpoint{193.985565pt}{141.529373pt}}
\pgflineto{\pgfpoint{192.989563pt}{195.151352pt}}
\pgflineto{\pgfpoint{192.989563pt}{140.777435pt}}
\pgfpathclose
\pgfusepath{fill,stroke}
\pgfpathmoveto{\pgfpoint{193.985565pt}{140.777435pt}}
\pgflineto{\pgfpoint{193.985565pt}{141.529373pt}}
\pgflineto{\pgfpoint{192.989563pt}{140.777435pt}}
\pgfpathclose
\pgfusepath{fill,stroke}
\pgfpathmoveto{\pgfpoint{194.981537pt}{140.831909pt}}
\pgflineto{\pgfpoint{193.985565pt}{141.529373pt}}
\pgflineto{\pgfpoint{193.985565pt}{140.777435pt}}
\pgfpathclose
\pgfusepath{fill,stroke}
\pgfpathmoveto{\pgfpoint{194.981537pt}{140.777435pt}}
\pgflineto{\pgfpoint{194.981537pt}{140.831909pt}}
\pgflineto{\pgfpoint{193.985565pt}{140.777435pt}}
\pgfpathclose
\pgfusepath{fill,stroke}
\pgfpathmoveto{\pgfpoint{195.977524pt}{149.854721pt}}
\pgflineto{\pgfpoint{194.981537pt}{140.831909pt}}
\pgflineto{\pgfpoint{194.981537pt}{140.777435pt}}
\pgfpathclose
\pgfusepath{fill,stroke}
\pgfpathmoveto{\pgfpoint{195.977524pt}{140.777435pt}}
\pgflineto{\pgfpoint{195.977524pt}{149.854721pt}}
\pgflineto{\pgfpoint{194.981537pt}{140.777435pt}}
\pgfpathclose
\pgfusepath{fill,stroke}
\pgfpathmoveto{\pgfpoint{196.973511pt}{144.710114pt}}
\pgflineto{\pgfpoint{195.977524pt}{149.854721pt}}
\pgflineto{\pgfpoint{195.977524pt}{140.777435pt}}
\pgfpathclose
\pgfusepath{fill,stroke}
\pgfpathmoveto{\pgfpoint{196.973511pt}{140.777435pt}}
\pgflineto{\pgfpoint{196.973511pt}{144.710114pt}}
\pgflineto{\pgfpoint{195.977524pt}{140.777435pt}}
\pgfpathclose
\pgfusepath{fill,stroke}
\pgfpathmoveto{\pgfpoint{197.969498pt}{144.257416pt}}
\pgflineto{\pgfpoint{196.973511pt}{144.710114pt}}
\pgflineto{\pgfpoint{196.973511pt}{140.777435pt}}
\pgfpathclose
\pgfusepath{fill,stroke}
\pgfpathmoveto{\pgfpoint{197.969498pt}{140.777435pt}}
\pgflineto{\pgfpoint{197.969498pt}{144.257416pt}}
\pgflineto{\pgfpoint{196.973511pt}{140.777435pt}}
\pgfpathclose
\pgfusepath{fill,stroke}
\pgfpathmoveto{\pgfpoint{198.965469pt}{197.820709pt}}
\pgflineto{\pgfpoint{197.969498pt}{144.257416pt}}
\pgflineto{\pgfpoint{197.969498pt}{140.777435pt}}
\pgfpathclose
\pgfusepath{fill,stroke}
\pgfpathmoveto{\pgfpoint{198.965469pt}{140.777435pt}}
\pgflineto{\pgfpoint{198.965469pt}{197.820709pt}}
\pgflineto{\pgfpoint{197.969498pt}{140.777435pt}}
\pgfpathclose
\pgfusepath{fill,stroke}
\pgfpathmoveto{\pgfpoint{199.961456pt}{140.777435pt}}
\pgflineto{\pgfpoint{198.965469pt}{197.820709pt}}
\pgflineto{\pgfpoint{198.965469pt}{140.777435pt}}
\pgfpathclose
\pgfusepath{fill,stroke}
\pgfpathmoveto{\pgfpoint{186.017685pt}{140.777435pt}}
\pgflineto{\pgfpoint{187.013672pt}{140.777435pt}}
\pgflineto{\pgfpoint{187.013672pt}{141.098480pt}}
\pgfpathclose
\pgfusepath{fill,stroke}
\pgfpathmoveto{\pgfpoint{188.009659pt}{140.777435pt}}
\pgflineto{\pgfpoint{187.013672pt}{141.098480pt}}
\pgflineto{\pgfpoint{187.013672pt}{140.777435pt}}
\pgfpathclose
\pgfusepath{fill,stroke}
\pgfpathmoveto{\pgfpoint{182.033752pt}{140.777435pt}}
\pgflineto{\pgfpoint{183.029724pt}{140.777435pt}}
\pgflineto{\pgfpoint{183.029724pt}{142.399628pt}}
\pgfpathclose
\pgfusepath{fill,stroke}
\pgfpathmoveto{\pgfpoint{184.025711pt}{140.777435pt}}
\pgflineto{\pgfpoint{183.029724pt}{142.399628pt}}
\pgflineto{\pgfpoint{183.029724pt}{140.777435pt}}
\pgfpathclose
\pgfusepath{fill,stroke}
\pgfpathmoveto{\pgfpoint{176.057846pt}{140.777435pt}}
\pgflineto{\pgfpoint{177.053818pt}{140.777435pt}}
\pgflineto{\pgfpoint{177.053818pt}{144.404053pt}}
\pgfpathclose
\pgfusepath{fill,stroke}
\pgfpathmoveto{\pgfpoint{178.049805pt}{144.353317pt}}
\pgflineto{\pgfpoint{177.053818pt}{144.404053pt}}
\pgflineto{\pgfpoint{177.053818pt}{140.777435pt}}
\pgfpathclose
\pgfusepath{fill,stroke}
\pgfpathmoveto{\pgfpoint{178.049805pt}{140.777435pt}}
\pgflineto{\pgfpoint{178.049805pt}{144.353317pt}}
\pgflineto{\pgfpoint{177.053818pt}{140.777435pt}}
\pgfpathclose
\pgfusepath{fill,stroke}
\pgfpathmoveto{\pgfpoint{179.045792pt}{140.963043pt}}
\pgflineto{\pgfpoint{178.049805pt}{144.353317pt}}
\pgflineto{\pgfpoint{178.049805pt}{140.777435pt}}
\pgfpathclose
\pgfusepath{fill,stroke}
\pgfpathmoveto{\pgfpoint{179.045792pt}{140.777435pt}}
\pgflineto{\pgfpoint{179.045792pt}{140.963043pt}}
\pgflineto{\pgfpoint{178.049805pt}{140.777435pt}}
\pgfpathclose
\pgfusepath{fill,stroke}
\pgfpathmoveto{\pgfpoint{180.041779pt}{140.930664pt}}
\pgflineto{\pgfpoint{179.045792pt}{140.963043pt}}
\pgflineto{\pgfpoint{179.045792pt}{140.777435pt}}
\pgfpathclose
\pgfusepath{fill,stroke}
\pgfpathmoveto{\pgfpoint{180.041779pt}{140.777435pt}}
\pgflineto{\pgfpoint{180.041779pt}{140.930664pt}}
\pgflineto{\pgfpoint{179.045792pt}{140.777435pt}}
\pgfpathclose
\pgfusepath{fill,stroke}
\pgfpathmoveto{\pgfpoint{181.037766pt}{141.619095pt}}
\pgflineto{\pgfpoint{180.041779pt}{140.930664pt}}
\pgflineto{\pgfpoint{180.041779pt}{140.777435pt}}
\pgfpathclose
\pgfusepath{fill,stroke}
\pgfpathmoveto{\pgfpoint{181.037766pt}{140.777435pt}}
\pgflineto{\pgfpoint{181.037766pt}{141.619095pt}}
\pgflineto{\pgfpoint{180.041779pt}{140.777435pt}}
\pgfpathclose
\pgfusepath{fill,stroke}
\pgfpathmoveto{\pgfpoint{182.033752pt}{140.777435pt}}
\pgflineto{\pgfpoint{181.037766pt}{141.619095pt}}
\pgflineto{\pgfpoint{181.037766pt}{140.777435pt}}
\pgfpathclose
\pgfusepath{fill,stroke}
\pgfpathmoveto{\pgfpoint{169.085953pt}{140.777435pt}}
\pgflineto{\pgfpoint{170.081940pt}{140.777435pt}}
\pgflineto{\pgfpoint{170.081940pt}{146.720169pt}}
\pgfpathclose
\pgfusepath{fill,stroke}
\pgfpathmoveto{\pgfpoint{171.077911pt}{145.686310pt}}
\pgflineto{\pgfpoint{170.081940pt}{146.720169pt}}
\pgflineto{\pgfpoint{170.081940pt}{140.777435pt}}
\pgfpathclose
\pgfusepath{fill,stroke}
\pgfpathmoveto{\pgfpoint{171.077911pt}{140.777435pt}}
\pgflineto{\pgfpoint{171.077911pt}{145.686310pt}}
\pgflineto{\pgfpoint{170.081940pt}{140.777435pt}}
\pgfpathclose
\pgfusepath{fill,stroke}
\pgfpathmoveto{\pgfpoint{172.073914pt}{149.736984pt}}
\pgflineto{\pgfpoint{171.077911pt}{145.686310pt}}
\pgflineto{\pgfpoint{171.077911pt}{140.777435pt}}
\pgfpathclose
\pgfusepath{fill,stroke}
\pgfpathmoveto{\pgfpoint{172.073914pt}{140.777435pt}}
\pgflineto{\pgfpoint{172.073914pt}{149.736984pt}}
\pgflineto{\pgfpoint{171.077911pt}{140.777435pt}}
\pgfpathclose
\pgfusepath{fill,stroke}
\pgfpathmoveto{\pgfpoint{173.069885pt}{148.531342pt}}
\pgflineto{\pgfpoint{172.073914pt}{149.736984pt}}
\pgflineto{\pgfpoint{172.073914pt}{140.777435pt}}
\pgfpathclose
\pgfusepath{fill,stroke}
\pgfpathmoveto{\pgfpoint{173.069885pt}{140.777435pt}}
\pgflineto{\pgfpoint{173.069885pt}{148.531342pt}}
\pgflineto{\pgfpoint{172.073914pt}{140.777435pt}}
\pgfpathclose
\pgfusepath{fill,stroke}
\pgfpathmoveto{\pgfpoint{174.065872pt}{141.515808pt}}
\pgflineto{\pgfpoint{173.069885pt}{148.531342pt}}
\pgflineto{\pgfpoint{173.069885pt}{140.777435pt}}
\pgfpathclose
\pgfusepath{fill,stroke}
\pgfpathmoveto{\pgfpoint{174.065872pt}{140.777435pt}}
\pgflineto{\pgfpoint{174.065872pt}{141.515808pt}}
\pgflineto{\pgfpoint{173.069885pt}{140.777435pt}}
\pgfpathclose
\pgfusepath{fill,stroke}
\pgfpathmoveto{\pgfpoint{175.061859pt}{141.323135pt}}
\pgflineto{\pgfpoint{174.065872pt}{141.515808pt}}
\pgflineto{\pgfpoint{174.065872pt}{140.777435pt}}
\pgfpathclose
\pgfusepath{fill,stroke}
\pgfpathmoveto{\pgfpoint{175.061859pt}{140.777435pt}}
\pgflineto{\pgfpoint{175.061859pt}{141.323135pt}}
\pgflineto{\pgfpoint{174.065872pt}{140.777435pt}}
\pgfpathclose
\pgfusepath{fill,stroke}
\pgfpathmoveto{\pgfpoint{176.057846pt}{140.777435pt}}
\pgflineto{\pgfpoint{175.061859pt}{141.323135pt}}
\pgflineto{\pgfpoint{175.061859pt}{140.777435pt}}
\pgfpathclose
\pgfusepath{fill,stroke}
\pgfpathmoveto{\pgfpoint{166.098007pt}{140.777435pt}}
\pgflineto{\pgfpoint{167.093994pt}{140.777435pt}}
\pgflineto{\pgfpoint{167.093994pt}{141.434799pt}}
\pgfpathclose
\pgfusepath{fill,stroke}
\pgfpathmoveto{\pgfpoint{167.417542pt}{205.642242pt}}
\pgflineto{\pgfpoint{167.093994pt}{141.434799pt}}
\pgflineto{\pgfpoint{167.093994pt}{140.777435pt}}
\pgfpathclose
\pgfusepath{fill,stroke}
\pgfpathmoveto{\pgfpoint{167.417542pt}{205.642242pt}}
\pgflineto{\pgfpoint{167.415314pt}{205.642242pt}}
\pgflineto{\pgfpoint{167.093994pt}{141.434799pt}}
\pgfpathclose
\pgfusepath{fill,stroke}
\pgfpathmoveto{\pgfpoint{168.089966pt}{140.777435pt}}
\pgflineto{\pgfpoint{167.417542pt}{205.642242pt}}
\pgflineto{\pgfpoint{167.093994pt}{140.777435pt}}
\pgfpathclose
\pgfusepath{fill,stroke}
\pgfpathmoveto{\pgfpoint{168.089966pt}{140.777435pt}}
\pgflineto{\pgfpoint{168.089966pt}{205.642242pt}}
\pgflineto{\pgfpoint{167.417542pt}{205.642242pt}}
\pgfpathclose
\pgfusepath{fill,stroke}
\pgfpathmoveto{\pgfpoint{169.085953pt}{140.777435pt}}
\pgflineto{\pgfpoint{168.089966pt}{205.642242pt}}
\pgflineto{\pgfpoint{168.089966pt}{140.777435pt}}
\pgfpathclose
\pgfusepath{fill,stroke}
\pgfpathmoveto{\pgfpoint{169.085953pt}{140.777435pt}}
\pgflineto{\pgfpoint{168.762405pt}{205.642242pt}}
\pgflineto{\pgfpoint{168.089966pt}{205.642242pt}}
\pgfpathclose
\pgfusepath{fill,stroke}
\pgfpathmoveto{\pgfpoint{163.110062pt}{140.777435pt}}
\pgflineto{\pgfpoint{164.106033pt}{140.777435pt}}
\pgflineto{\pgfpoint{164.106033pt}{194.127029pt}}
\pgfpathclose
\pgfusepath{fill,stroke}
\pgfpathmoveto{\pgfpoint{165.102020pt}{140.828079pt}}
\pgflineto{\pgfpoint{164.106033pt}{194.127029pt}}
\pgflineto{\pgfpoint{164.106033pt}{140.777435pt}}
\pgfpathclose
\pgfusepath{fill,stroke}
\pgfpathmoveto{\pgfpoint{165.102020pt}{140.777435pt}}
\pgflineto{\pgfpoint{165.102020pt}{140.828079pt}}
\pgflineto{\pgfpoint{164.106033pt}{140.777435pt}}
\pgfpathclose
\pgfusepath{fill,stroke}
\pgfpathmoveto{\pgfpoint{166.098007pt}{140.777435pt}}
\pgflineto{\pgfpoint{165.102020pt}{140.828079pt}}
\pgflineto{\pgfpoint{165.102020pt}{140.777435pt}}
\pgfpathclose
\pgfusepath{fill,stroke}
\pgfpathmoveto{\pgfpoint{159.126114pt}{140.777435pt}}
\pgflineto{\pgfpoint{160.122101pt}{140.777435pt}}
\pgflineto{\pgfpoint{160.122101pt}{158.193100pt}}
\pgfpathclose
\pgfusepath{fill,stroke}
\pgfpathmoveto{\pgfpoint{161.118088pt}{140.795273pt}}
\pgflineto{\pgfpoint{160.122101pt}{158.193100pt}}
\pgflineto{\pgfpoint{160.122101pt}{140.777435pt}}
\pgfpathclose
\pgfusepath{fill,stroke}
\pgfpathmoveto{\pgfpoint{161.118088pt}{140.777435pt}}
\pgflineto{\pgfpoint{161.118088pt}{140.795273pt}}
\pgflineto{\pgfpoint{160.122101pt}{140.777435pt}}
\pgfpathclose
\pgfusepath{fill,stroke}
\pgfpathmoveto{\pgfpoint{162.114075pt}{141.404877pt}}
\pgflineto{\pgfpoint{161.118088pt}{140.795273pt}}
\pgflineto{\pgfpoint{161.118088pt}{140.777435pt}}
\pgfpathclose
\pgfusepath{fill,stroke}
\pgfpathmoveto{\pgfpoint{162.114075pt}{140.777435pt}}
\pgflineto{\pgfpoint{162.114075pt}{141.404877pt}}
\pgflineto{\pgfpoint{161.118088pt}{140.777435pt}}
\pgfpathclose
\pgfusepath{fill,stroke}
\pgfpathmoveto{\pgfpoint{163.110062pt}{140.777435pt}}
\pgflineto{\pgfpoint{162.114075pt}{141.404877pt}}
\pgflineto{\pgfpoint{162.114075pt}{140.777435pt}}
\pgfpathclose
\pgfusepath{fill,stroke}
\pgfpathmoveto{\pgfpoint{156.138168pt}{140.777435pt}}
\pgflineto{\pgfpoint{157.134155pt}{140.777435pt}}
\pgflineto{\pgfpoint{157.134155pt}{140.926575pt}}
\pgfpathclose
\pgfusepath{fill,stroke}
\pgfpathmoveto{\pgfpoint{158.130127pt}{140.979492pt}}
\pgflineto{\pgfpoint{157.134155pt}{140.926575pt}}
\pgflineto{\pgfpoint{157.134155pt}{140.777435pt}}
\pgfpathclose
\pgfusepath{fill,stroke}
\pgfpathmoveto{\pgfpoint{158.130127pt}{140.777435pt}}
\pgflineto{\pgfpoint{158.130127pt}{140.979492pt}}
\pgflineto{\pgfpoint{157.134155pt}{140.777435pt}}
\pgfpathclose
\pgfusepath{fill,stroke}
\pgfpathmoveto{\pgfpoint{159.126114pt}{140.777435pt}}
\pgflineto{\pgfpoint{158.130127pt}{140.979492pt}}
\pgflineto{\pgfpoint{158.130127pt}{140.777435pt}}
\pgfpathclose
\pgfusepath{fill,stroke}
\pgfpathmoveto{\pgfpoint{154.146194pt}{140.777435pt}}
\pgflineto{\pgfpoint{155.142181pt}{140.777435pt}}
\pgflineto{\pgfpoint{155.142181pt}{195.815277pt}}
\pgfpathclose
\pgfusepath{fill,stroke}
\pgfpathmoveto{\pgfpoint{156.138168pt}{140.777435pt}}
\pgflineto{\pgfpoint{155.142181pt}{195.815277pt}}
\pgflineto{\pgfpoint{155.142181pt}{140.777435pt}}
\pgfpathclose
\pgfusepath{fill,stroke}
\pgfpathmoveto{\pgfpoint{150.162262pt}{140.777435pt}}
\pgflineto{\pgfpoint{151.158249pt}{140.777435pt}}
\pgflineto{\pgfpoint{151.158249pt}{141.561523pt}}
\pgfpathclose
\pgfusepath{fill,stroke}
\pgfpathmoveto{\pgfpoint{152.154221pt}{144.148651pt}}
\pgflineto{\pgfpoint{151.158249pt}{141.561523pt}}
\pgflineto{\pgfpoint{151.158249pt}{140.777435pt}}
\pgfpathclose
\pgfusepath{fill,stroke}
\pgfpathmoveto{\pgfpoint{152.154221pt}{140.777435pt}}
\pgflineto{\pgfpoint{152.154221pt}{144.148651pt}}
\pgflineto{\pgfpoint{151.158249pt}{140.777435pt}}
\pgfpathclose
\pgfusepath{fill,stroke}
\pgfpathmoveto{\pgfpoint{153.150208pt}{143.520859pt}}
\pgflineto{\pgfpoint{152.154221pt}{144.148651pt}}
\pgflineto{\pgfpoint{152.154221pt}{140.777435pt}}
\pgfpathclose
\pgfusepath{fill,stroke}
\pgfpathmoveto{\pgfpoint{153.150208pt}{140.777435pt}}
\pgflineto{\pgfpoint{153.150208pt}{143.520859pt}}
\pgflineto{\pgfpoint{152.154221pt}{140.777435pt}}
\pgfpathclose
\pgfusepath{fill,stroke}
\pgfpathmoveto{\pgfpoint{154.146194pt}{140.777435pt}}
\pgflineto{\pgfpoint{153.150208pt}{143.520859pt}}
\pgflineto{\pgfpoint{153.150208pt}{140.777435pt}}
\pgfpathclose
\pgfusepath{fill,stroke}
\pgfpathmoveto{\pgfpoint{144.186356pt}{140.777435pt}}
\pgflineto{\pgfpoint{145.182343pt}{140.777435pt}}
\pgflineto{\pgfpoint{145.182343pt}{141.507751pt}}
\pgfpathclose
\pgfusepath{fill,stroke}
\pgfpathmoveto{\pgfpoint{146.178314pt}{141.062988pt}}
\pgflineto{\pgfpoint{145.182343pt}{141.507751pt}}
\pgflineto{\pgfpoint{145.182343pt}{140.777435pt}}
\pgfpathclose
\pgfusepath{fill,stroke}
\pgfpathmoveto{\pgfpoint{146.178314pt}{140.777435pt}}
\pgflineto{\pgfpoint{146.178314pt}{141.062988pt}}
\pgflineto{\pgfpoint{145.182343pt}{140.777435pt}}
\pgfpathclose
\pgfusepath{fill,stroke}
\pgfpathmoveto{\pgfpoint{147.174316pt}{143.409790pt}}
\pgflineto{\pgfpoint{146.178314pt}{141.062988pt}}
\pgflineto{\pgfpoint{146.178314pt}{140.777435pt}}
\pgfpathclose
\pgfusepath{fill,stroke}
\pgfpathmoveto{\pgfpoint{147.174316pt}{140.777435pt}}
\pgflineto{\pgfpoint{147.174316pt}{143.409790pt}}
\pgflineto{\pgfpoint{146.178314pt}{140.777435pt}}
\pgfpathclose
\pgfusepath{fill,stroke}
\pgfpathmoveto{\pgfpoint{148.170288pt}{140.777435pt}}
\pgflineto{\pgfpoint{147.174316pt}{143.409790pt}}
\pgflineto{\pgfpoint{147.174316pt}{140.777435pt}}
\pgfpathclose
\pgfusepath{fill,stroke}
\pgfpathmoveto{\pgfpoint{142.194382pt}{140.777435pt}}
\pgflineto{\pgfpoint{143.190369pt}{140.777435pt}}
\pgflineto{\pgfpoint{143.190369pt}{141.972321pt}}
\pgfpathclose
\pgfusepath{fill,stroke}
\pgfpathmoveto{\pgfpoint{144.186356pt}{140.777435pt}}
\pgflineto{\pgfpoint{143.190369pt}{141.972321pt}}
\pgflineto{\pgfpoint{143.190369pt}{140.777435pt}}
\pgfpathclose
\pgfusepath{fill,stroke}
\pgfpathmoveto{\pgfpoint{136.218475pt}{140.777435pt}}
\pgflineto{\pgfpoint{137.214478pt}{140.777435pt}}
\pgflineto{\pgfpoint{137.214478pt}{144.493912pt}}
\pgfpathclose
\pgfusepath{fill,stroke}
\pgfpathmoveto{\pgfpoint{138.210449pt}{140.824677pt}}
\pgflineto{\pgfpoint{137.214478pt}{144.493912pt}}
\pgflineto{\pgfpoint{137.214478pt}{140.777435pt}}
\pgfpathclose
\pgfusepath{fill,stroke}
\pgfpathmoveto{\pgfpoint{138.210449pt}{140.777435pt}}
\pgflineto{\pgfpoint{138.210449pt}{140.824677pt}}
\pgflineto{\pgfpoint{137.214478pt}{140.777435pt}}
\pgfpathclose
\pgfusepath{fill,stroke}
\pgfpathmoveto{\pgfpoint{139.206436pt}{155.555008pt}}
\pgflineto{\pgfpoint{138.210449pt}{140.824677pt}}
\pgflineto{\pgfpoint{138.210449pt}{140.777435pt}}
\pgfpathclose
\pgfusepath{fill,stroke}
\pgfpathmoveto{\pgfpoint{139.206436pt}{140.777435pt}}
\pgflineto{\pgfpoint{139.206436pt}{155.555008pt}}
\pgflineto{\pgfpoint{138.210449pt}{140.777435pt}}
\pgfpathclose
\pgfusepath{fill,stroke}
\pgfpathmoveto{\pgfpoint{140.202423pt}{141.880554pt}}
\pgflineto{\pgfpoint{139.206436pt}{155.555008pt}}
\pgflineto{\pgfpoint{139.206436pt}{140.777435pt}}
\pgfpathclose
\pgfusepath{fill,stroke}
\pgfpathmoveto{\pgfpoint{140.202423pt}{140.777435pt}}
\pgflineto{\pgfpoint{140.202423pt}{141.880554pt}}
\pgflineto{\pgfpoint{139.206436pt}{140.777435pt}}
\pgfpathclose
\pgfusepath{fill,stroke}
\pgfpathmoveto{\pgfpoint{141.198410pt}{141.096649pt}}
\pgflineto{\pgfpoint{140.202423pt}{141.880554pt}}
\pgflineto{\pgfpoint{140.202423pt}{140.777435pt}}
\pgfpathclose
\pgfusepath{fill,stroke}
\pgfpathmoveto{\pgfpoint{141.198410pt}{140.777435pt}}
\pgflineto{\pgfpoint{141.198410pt}{141.096649pt}}
\pgflineto{\pgfpoint{140.202423pt}{140.777435pt}}
\pgfpathclose
\pgfusepath{fill,stroke}
\pgfpathmoveto{\pgfpoint{142.194382pt}{140.777435pt}}
\pgflineto{\pgfpoint{141.198410pt}{141.096649pt}}
\pgflineto{\pgfpoint{141.198410pt}{140.777435pt}}
\pgfpathclose
\pgfusepath{fill,stroke}
\pgfpathmoveto{\pgfpoint{134.226517pt}{140.777435pt}}
\pgflineto{\pgfpoint{135.222504pt}{140.777435pt}}
\pgflineto{\pgfpoint{135.222504pt}{141.478424pt}}
\pgfpathclose
\pgfusepath{fill,stroke}
\pgfpathmoveto{\pgfpoint{136.218475pt}{140.777435pt}}
\pgflineto{\pgfpoint{135.222504pt}{141.478424pt}}
\pgflineto{\pgfpoint{135.222504pt}{140.777435pt}}
\pgfpathclose
\pgfusepath{fill,stroke}
\pgfpathmoveto{\pgfpoint{128.250610pt}{140.777435pt}}
\pgflineto{\pgfpoint{129.246597pt}{140.777435pt}}
\pgflineto{\pgfpoint{129.246597pt}{146.889862pt}}
\pgfpathclose
\pgfusepath{fill,stroke}
\pgfpathmoveto{\pgfpoint{130.242584pt}{141.305695pt}}
\pgflineto{\pgfpoint{129.246597pt}{146.889862pt}}
\pgflineto{\pgfpoint{129.246597pt}{140.777435pt}}
\pgfpathclose
\pgfusepath{fill,stroke}
\pgfpathmoveto{\pgfpoint{130.242584pt}{140.777435pt}}
\pgflineto{\pgfpoint{130.242584pt}{141.305695pt}}
\pgflineto{\pgfpoint{129.246597pt}{140.777435pt}}
\pgfpathclose
\pgfusepath{fill,stroke}
\pgfpathmoveto{\pgfpoint{131.238571pt}{142.924377pt}}
\pgflineto{\pgfpoint{130.242584pt}{141.305695pt}}
\pgflineto{\pgfpoint{130.242584pt}{140.777435pt}}
\pgfpathclose
\pgfusepath{fill,stroke}
\pgfpathmoveto{\pgfpoint{131.238571pt}{140.777435pt}}
\pgflineto{\pgfpoint{131.238571pt}{142.924377pt}}
\pgflineto{\pgfpoint{130.242584pt}{140.777435pt}}
\pgfpathclose
\pgfusepath{fill,stroke}
\pgfpathmoveto{\pgfpoint{132.234558pt}{140.784973pt}}
\pgflineto{\pgfpoint{131.238571pt}{142.924377pt}}
\pgflineto{\pgfpoint{131.238571pt}{140.777435pt}}
\pgfpathclose
\pgfusepath{fill,stroke}
\pgfpathmoveto{\pgfpoint{132.234558pt}{140.777435pt}}
\pgflineto{\pgfpoint{132.234558pt}{140.784973pt}}
\pgflineto{\pgfpoint{131.238571pt}{140.777435pt}}
\pgfpathclose
\pgfusepath{fill,stroke}
\pgfpathmoveto{\pgfpoint{133.230530pt}{140.777435pt}}
\pgflineto{\pgfpoint{132.234558pt}{140.784973pt}}
\pgflineto{\pgfpoint{132.234558pt}{140.777435pt}}
\pgfpathclose
\pgfusepath{fill,stroke}
\pgfpathmoveto{\pgfpoint{125.262665pt}{140.777435pt}}
\pgflineto{\pgfpoint{126.258652pt}{140.777435pt}}
\pgflineto{\pgfpoint{126.258652pt}{141.187393pt}}
\pgfpathclose
\pgfusepath{fill,stroke}
\pgfpathmoveto{\pgfpoint{127.254631pt}{158.624695pt}}
\pgflineto{\pgfpoint{126.258652pt}{141.187393pt}}
\pgflineto{\pgfpoint{126.258652pt}{140.777435pt}}
\pgfpathclose
\pgfusepath{fill,stroke}
\pgfpathmoveto{\pgfpoint{127.254631pt}{140.777435pt}}
\pgflineto{\pgfpoint{127.254631pt}{158.624695pt}}
\pgflineto{\pgfpoint{126.258652pt}{140.777435pt}}
\pgfpathclose
\pgfusepath{fill,stroke}
\pgfpathmoveto{\pgfpoint{128.250610pt}{140.777435pt}}
\pgflineto{\pgfpoint{127.254631pt}{158.624695pt}}
\pgflineto{\pgfpoint{127.254631pt}{140.777435pt}}
\pgfpathclose
\pgfusepath{fill,stroke}
\pgfpathmoveto{\pgfpoint{122.274712pt}{140.777435pt}}
\pgflineto{\pgfpoint{123.270691pt}{140.777435pt}}
\pgflineto{\pgfpoint{123.270691pt}{141.045074pt}}
\pgfpathclose
\pgfusepath{fill,stroke}
\pgfpathmoveto{\pgfpoint{124.266678pt}{140.777435pt}}
\pgflineto{\pgfpoint{123.270691pt}{141.045074pt}}
\pgflineto{\pgfpoint{123.270691pt}{140.777435pt}}
\pgfpathclose
\pgfusepath{fill,stroke}
\pgfpathmoveto{\pgfpoint{120.282745pt}{140.777435pt}}
\pgflineto{\pgfpoint{121.278725pt}{140.777435pt}}
\pgflineto{\pgfpoint{121.278725pt}{141.992584pt}}
\pgfpathclose
\pgfusepath{fill,stroke}
\pgfpathmoveto{\pgfpoint{122.274712pt}{140.777435pt}}
\pgflineto{\pgfpoint{121.278725pt}{141.992584pt}}
\pgflineto{\pgfpoint{121.278725pt}{140.777435pt}}
\pgfpathclose
\pgfusepath{fill,stroke}
\pgfpathmoveto{\pgfpoint{115.302826pt}{140.777435pt}}
\pgflineto{\pgfpoint{116.298813pt}{140.777435pt}}
\pgflineto{\pgfpoint{116.298813pt}{141.607086pt}}
\pgfpathclose
\pgfusepath{fill,stroke}
\pgfpathmoveto{\pgfpoint{117.294792pt}{143.559708pt}}
\pgflineto{\pgfpoint{116.298813pt}{141.607086pt}}
\pgflineto{\pgfpoint{116.298813pt}{140.777435pt}}
\pgfpathclose
\pgfusepath{fill,stroke}
\pgfpathmoveto{\pgfpoint{117.294792pt}{140.777435pt}}
\pgflineto{\pgfpoint{117.294792pt}{143.559708pt}}
\pgflineto{\pgfpoint{116.298813pt}{140.777435pt}}
\pgfpathclose
\pgfusepath{fill,stroke}
\pgfpathmoveto{\pgfpoint{118.290779pt}{140.848618pt}}
\pgflineto{\pgfpoint{117.294792pt}{143.559708pt}}
\pgflineto{\pgfpoint{117.294792pt}{140.777435pt}}
\pgfpathclose
\pgfusepath{fill,stroke}
\pgfpathmoveto{\pgfpoint{118.290779pt}{140.777435pt}}
\pgflineto{\pgfpoint{118.290779pt}{140.848618pt}}
\pgflineto{\pgfpoint{117.294792pt}{140.777435pt}}
\pgfpathclose
\pgfusepath{fill,stroke}
\pgfpathmoveto{\pgfpoint{119.286758pt}{140.997116pt}}
\pgflineto{\pgfpoint{118.290779pt}{140.848618pt}}
\pgflineto{\pgfpoint{118.290779pt}{140.777435pt}}
\pgfpathclose
\pgfusepath{fill,stroke}
\pgfpathmoveto{\pgfpoint{119.286758pt}{140.777435pt}}
\pgflineto{\pgfpoint{119.286758pt}{140.997116pt}}
\pgflineto{\pgfpoint{118.290779pt}{140.777435pt}}
\pgfpathclose
\pgfusepath{fill,stroke}
\pgfpathmoveto{\pgfpoint{120.282745pt}{140.777435pt}}
\pgflineto{\pgfpoint{119.286758pt}{140.997116pt}}
\pgflineto{\pgfpoint{119.286758pt}{140.777435pt}}
\pgfpathclose
\pgfusepath{fill,stroke}
\pgfpathmoveto{\pgfpoint{111.318893pt}{140.777435pt}}
\pgflineto{\pgfpoint{112.314873pt}{140.777435pt}}
\pgflineto{\pgfpoint{112.314873pt}{142.841614pt}}
\pgfpathclose
\pgfusepath{fill,stroke}
\pgfpathmoveto{\pgfpoint{113.310852pt}{141.335541pt}}
\pgflineto{\pgfpoint{112.314873pt}{142.841614pt}}
\pgflineto{\pgfpoint{112.314873pt}{140.777435pt}}
\pgfpathclose
\pgfusepath{fill,stroke}
\pgfpathmoveto{\pgfpoint{113.310852pt}{140.777435pt}}
\pgflineto{\pgfpoint{113.310852pt}{141.335541pt}}
\pgflineto{\pgfpoint{112.314873pt}{140.777435pt}}
\pgfpathclose
\pgfusepath{fill,stroke}
\pgfpathmoveto{\pgfpoint{114.306839pt}{143.259003pt}}
\pgflineto{\pgfpoint{113.310852pt}{141.335541pt}}
\pgflineto{\pgfpoint{113.310852pt}{140.777435pt}}
\pgfpathclose
\pgfusepath{fill,stroke}
\pgfpathmoveto{\pgfpoint{114.306839pt}{140.777435pt}}
\pgflineto{\pgfpoint{114.306839pt}{143.259003pt}}
\pgflineto{\pgfpoint{113.310852pt}{140.777435pt}}
\pgfpathclose
\pgfusepath{fill,stroke}
\pgfpathmoveto{\pgfpoint{115.302826pt}{140.777435pt}}
\pgflineto{\pgfpoint{114.306839pt}{143.259003pt}}
\pgflineto{\pgfpoint{114.306839pt}{140.777435pt}}
\pgfpathclose
\pgfusepath{fill,stroke}
\pgfpathmoveto{\pgfpoint{107.334953pt}{140.777435pt}}
\pgflineto{\pgfpoint{108.330933pt}{140.777435pt}}
\pgflineto{\pgfpoint{108.330933pt}{156.427597pt}}
\pgfpathclose
\pgfusepath{fill,stroke}
\pgfpathmoveto{\pgfpoint{109.326920pt}{140.854156pt}}
\pgflineto{\pgfpoint{108.330933pt}{156.427597pt}}
\pgflineto{\pgfpoint{108.330933pt}{140.777435pt}}
\pgfpathclose
\pgfusepath{fill,stroke}
\pgfpathmoveto{\pgfpoint{109.326920pt}{140.777435pt}}
\pgflineto{\pgfpoint{109.326920pt}{140.854156pt}}
\pgflineto{\pgfpoint{108.330933pt}{140.777435pt}}
\pgfpathclose
\pgfusepath{fill,stroke}
\pgfpathmoveto{\pgfpoint{110.322906pt}{143.840607pt}}
\pgflineto{\pgfpoint{109.326920pt}{140.854156pt}}
\pgflineto{\pgfpoint{109.326920pt}{140.777435pt}}
\pgfpathclose
\pgfusepath{fill,stroke}
\pgfpathmoveto{\pgfpoint{110.322906pt}{140.777435pt}}
\pgflineto{\pgfpoint{110.322906pt}{143.840607pt}}
\pgflineto{\pgfpoint{109.326920pt}{140.777435pt}}
\pgfpathclose
\pgfusepath{fill,stroke}
\pgfpathmoveto{\pgfpoint{111.318893pt}{140.777435pt}}
\pgflineto{\pgfpoint{110.322906pt}{143.840607pt}}
\pgflineto{\pgfpoint{110.322906pt}{140.777435pt}}
\pgfpathclose
\pgfusepath{fill,stroke}
\pgfpathmoveto{\pgfpoint{103.351013pt}{140.777435pt}}
\pgflineto{\pgfpoint{104.347000pt}{140.777435pt}}
\pgflineto{\pgfpoint{104.347000pt}{152.852020pt}}
\pgfpathclose
\pgfusepath{fill,stroke}
\pgfpathmoveto{\pgfpoint{105.342987pt}{140.777435pt}}
\pgflineto{\pgfpoint{104.347000pt}{152.852020pt}}
\pgflineto{\pgfpoint{104.347000pt}{140.777435pt}}
\pgfpathclose
\pgfusepath{fill,stroke}
\pgfpathmoveto{\pgfpoint{94.387161pt}{140.777435pt}}
\pgflineto{\pgfpoint{95.383141pt}{140.777435pt}}
\pgflineto{\pgfpoint{95.383141pt}{141.040619pt}}
\pgfpathclose
\pgfusepath{fill,stroke}
\pgfpathmoveto{\pgfpoint{96.379128pt}{141.052612pt}}
\pgflineto{\pgfpoint{95.383141pt}{141.040619pt}}
\pgflineto{\pgfpoint{95.383141pt}{140.777435pt}}
\pgfpathclose
\pgfusepath{fill,stroke}
\pgfpathmoveto{\pgfpoint{96.379128pt}{140.777435pt}}
\pgflineto{\pgfpoint{96.379128pt}{141.052612pt}}
\pgflineto{\pgfpoint{95.383141pt}{140.777435pt}}
\pgfpathclose
\pgfusepath{fill,stroke}
\pgfpathmoveto{\pgfpoint{97.375107pt}{142.021271pt}}
\pgflineto{\pgfpoint{96.379128pt}{141.052612pt}}
\pgflineto{\pgfpoint{96.379128pt}{140.777435pt}}
\pgfpathclose
\pgfusepath{fill,stroke}
\pgfpathmoveto{\pgfpoint{97.375107pt}{140.777435pt}}
\pgflineto{\pgfpoint{97.375107pt}{142.021271pt}}
\pgflineto{\pgfpoint{96.379128pt}{140.777435pt}}
\pgfpathclose
\pgfusepath{fill,stroke}
\pgfpathmoveto{\pgfpoint{98.371094pt}{187.727631pt}}
\pgflineto{\pgfpoint{97.375107pt}{142.021271pt}}
\pgflineto{\pgfpoint{97.375107pt}{140.777435pt}}
\pgfpathclose
\pgfusepath{fill,stroke}
\pgfpathmoveto{\pgfpoint{98.371094pt}{140.777435pt}}
\pgflineto{\pgfpoint{98.371094pt}{187.727631pt}}
\pgflineto{\pgfpoint{97.375107pt}{140.777435pt}}
\pgfpathclose
\pgfusepath{fill,stroke}
\pgfpathmoveto{\pgfpoint{99.367081pt}{143.174622pt}}
\pgflineto{\pgfpoint{98.371094pt}{187.727631pt}}
\pgflineto{\pgfpoint{98.371094pt}{140.777435pt}}
\pgfpathclose
\pgfusepath{fill,stroke}
\pgfpathmoveto{\pgfpoint{99.367081pt}{140.777435pt}}
\pgflineto{\pgfpoint{99.367081pt}{143.174622pt}}
\pgflineto{\pgfpoint{98.371094pt}{140.777435pt}}
\pgfpathclose
\pgfusepath{fill,stroke}
\pgfpathmoveto{\pgfpoint{100.363068pt}{142.326782pt}}
\pgflineto{\pgfpoint{99.367081pt}{143.174622pt}}
\pgflineto{\pgfpoint{99.367081pt}{140.777435pt}}
\pgfpathclose
\pgfusepath{fill,stroke}
\pgfpathmoveto{\pgfpoint{100.363068pt}{140.777435pt}}
\pgflineto{\pgfpoint{100.363068pt}{142.326782pt}}
\pgflineto{\pgfpoint{99.367081pt}{140.777435pt}}
\pgfpathclose
\pgfusepath{fill,stroke}
\pgfpathmoveto{\pgfpoint{101.359047pt}{141.473938pt}}
\pgflineto{\pgfpoint{100.363068pt}{142.326782pt}}
\pgflineto{\pgfpoint{100.363068pt}{140.777435pt}}
\pgfpathclose
\pgfusepath{fill,stroke}
\pgfpathmoveto{\pgfpoint{101.359047pt}{140.777435pt}}
\pgflineto{\pgfpoint{101.359047pt}{141.473938pt}}
\pgflineto{\pgfpoint{100.363068pt}{140.777435pt}}
\pgfpathclose
\pgfusepath{fill,stroke}
\pgfpathmoveto{\pgfpoint{102.355034pt}{153.173050pt}}
\pgflineto{\pgfpoint{101.359047pt}{141.473938pt}}
\pgflineto{\pgfpoint{101.359047pt}{140.777435pt}}
\pgfpathclose
\pgfusepath{fill,stroke}
\pgfpathmoveto{\pgfpoint{102.355034pt}{140.777435pt}}
\pgflineto{\pgfpoint{102.355034pt}{153.173050pt}}
\pgflineto{\pgfpoint{101.359047pt}{140.777435pt}}
\pgfpathclose
\pgfusepath{fill,stroke}
\pgfpathmoveto{\pgfpoint{103.351013pt}{140.777435pt}}
\pgflineto{\pgfpoint{102.355034pt}{153.173050pt}}
\pgflineto{\pgfpoint{102.355034pt}{140.777435pt}}
\pgfpathclose
\pgfusepath{fill,stroke}
\pgfpathmoveto{\pgfpoint{91.399208pt}{140.777435pt}}
\pgflineto{\pgfpoint{92.395187pt}{140.777435pt}}
\pgflineto{\pgfpoint{92.395187pt}{140.789795pt}}
\pgfpathclose
\pgfusepath{fill,stroke}
\pgfpathmoveto{\pgfpoint{93.391174pt}{141.209564pt}}
\pgflineto{\pgfpoint{92.395187pt}{140.789795pt}}
\pgflineto{\pgfpoint{92.395187pt}{140.777435pt}}
\pgfpathclose
\pgfusepath{fill,stroke}
\pgfpathmoveto{\pgfpoint{93.391174pt}{140.777435pt}}
\pgflineto{\pgfpoint{93.391174pt}{141.209564pt}}
\pgflineto{\pgfpoint{92.395187pt}{140.777435pt}}
\pgfpathclose
\pgfusepath{fill,stroke}
\pgfpathmoveto{\pgfpoint{94.387161pt}{140.777435pt}}
\pgflineto{\pgfpoint{93.391174pt}{141.209564pt}}
\pgflineto{\pgfpoint{93.391174pt}{140.777435pt}}
\pgfpathclose
\pgfusepath{fill,stroke}
\pgfpathmoveto{\pgfpoint{86.419289pt}{140.777435pt}}
\pgflineto{\pgfpoint{87.415276pt}{140.777435pt}}
\pgflineto{\pgfpoint{87.415276pt}{152.008636pt}}
\pgfpathclose
\pgfusepath{fill,stroke}
\pgfpathmoveto{\pgfpoint{88.411255pt}{163.097900pt}}
\pgflineto{\pgfpoint{87.415276pt}{152.008636pt}}
\pgflineto{\pgfpoint{87.415276pt}{140.777435pt}}
\pgfpathclose
\pgfusepath{fill,stroke}
\pgfpathmoveto{\pgfpoint{88.411255pt}{140.777435pt}}
\pgflineto{\pgfpoint{88.411255pt}{163.097900pt}}
\pgflineto{\pgfpoint{87.415276pt}{140.777435pt}}
\pgfpathclose
\pgfusepath{fill,stroke}
\pgfpathmoveto{\pgfpoint{89.407242pt}{147.608887pt}}
\pgflineto{\pgfpoint{88.411255pt}{163.097900pt}}
\pgflineto{\pgfpoint{88.411255pt}{140.777435pt}}
\pgfpathclose
\pgfusepath{fill,stroke}
\pgfpathmoveto{\pgfpoint{89.407242pt}{140.777435pt}}
\pgflineto{\pgfpoint{89.407242pt}{147.608887pt}}
\pgflineto{\pgfpoint{88.411255pt}{140.777435pt}}
\pgfpathclose
\pgfusepath{fill,stroke}
\pgfpathmoveto{\pgfpoint{90.403221pt}{189.355881pt}}
\pgflineto{\pgfpoint{89.407242pt}{147.608887pt}}
\pgflineto{\pgfpoint{89.407242pt}{140.777435pt}}
\pgfpathclose
\pgfusepath{fill,stroke}
\pgfpathmoveto{\pgfpoint{90.403221pt}{140.777435pt}}
\pgflineto{\pgfpoint{90.403221pt}{189.355881pt}}
\pgflineto{\pgfpoint{89.407242pt}{140.777435pt}}
\pgfpathclose
\pgfusepath{fill,stroke}
\pgfpathmoveto{\pgfpoint{91.399208pt}{140.777435pt}}
\pgflineto{\pgfpoint{90.403221pt}{189.355881pt}}
\pgflineto{\pgfpoint{90.403221pt}{140.777435pt}}
\pgfpathclose
\pgfusepath{fill,stroke}
\pgfpathmoveto{\pgfpoint{78.451424pt}{140.777435pt}}
\pgflineto{\pgfpoint{79.447403pt}{140.777435pt}}
\pgflineto{\pgfpoint{79.447403pt}{164.721649pt}}
\pgfpathclose
\pgfusepath{fill,stroke}
\pgfpathmoveto{\pgfpoint{80.443390pt}{141.021835pt}}
\pgflineto{\pgfpoint{79.447403pt}{164.721649pt}}
\pgflineto{\pgfpoint{79.447403pt}{140.777435pt}}
\pgfpathclose
\pgfusepath{fill,stroke}
\pgfpathmoveto{\pgfpoint{80.443390pt}{140.777435pt}}
\pgflineto{\pgfpoint{80.443390pt}{141.021835pt}}
\pgflineto{\pgfpoint{79.447403pt}{140.777435pt}}
\pgfpathclose
\pgfusepath{fill,stroke}
\pgfpathmoveto{\pgfpoint{81.439369pt}{141.797729pt}}
\pgflineto{\pgfpoint{80.443390pt}{141.021835pt}}
\pgflineto{\pgfpoint{80.443390pt}{140.777435pt}}
\pgfpathclose
\pgfusepath{fill,stroke}
\pgfpathmoveto{\pgfpoint{81.439369pt}{140.777435pt}}
\pgflineto{\pgfpoint{81.439369pt}{141.797729pt}}
\pgflineto{\pgfpoint{80.443390pt}{140.777435pt}}
\pgfpathclose
\pgfusepath{fill,stroke}
\pgfpathmoveto{\pgfpoint{82.435356pt}{141.784378pt}}
\pgflineto{\pgfpoint{81.439369pt}{141.797729pt}}
\pgflineto{\pgfpoint{81.439369pt}{140.777435pt}}
\pgfpathclose
\pgfusepath{fill,stroke}
\pgfpathmoveto{\pgfpoint{82.435356pt}{140.777435pt}}
\pgflineto{\pgfpoint{82.435356pt}{141.784378pt}}
\pgflineto{\pgfpoint{81.439369pt}{140.777435pt}}
\pgfpathclose
\pgfusepath{fill,stroke}
\pgfpathmoveto{\pgfpoint{83.431335pt}{147.373245pt}}
\pgflineto{\pgfpoint{82.435356pt}{141.784378pt}}
\pgflineto{\pgfpoint{82.435356pt}{140.777435pt}}
\pgfpathclose
\pgfusepath{fill,stroke}
\pgfpathmoveto{\pgfpoint{83.431335pt}{140.777435pt}}
\pgflineto{\pgfpoint{83.431335pt}{147.373245pt}}
\pgflineto{\pgfpoint{82.435356pt}{140.777435pt}}
\pgfpathclose
\pgfusepath{fill,stroke}
\pgfpathmoveto{\pgfpoint{84.427322pt}{140.853500pt}}
\pgflineto{\pgfpoint{83.431335pt}{147.373245pt}}
\pgflineto{\pgfpoint{83.431335pt}{140.777435pt}}
\pgfpathclose
\pgfusepath{fill,stroke}
\pgfpathmoveto{\pgfpoint{84.427322pt}{140.777435pt}}
\pgflineto{\pgfpoint{84.427322pt}{140.853500pt}}
\pgflineto{\pgfpoint{83.431335pt}{140.777435pt}}
\pgfpathclose
\pgfusepath{fill,stroke}
\pgfpathmoveto{\pgfpoint{85.423309pt}{140.987396pt}}
\pgflineto{\pgfpoint{84.427322pt}{140.853500pt}}
\pgflineto{\pgfpoint{84.427322pt}{140.777435pt}}
\pgfpathclose
\pgfusepath{fill,stroke}
\pgfpathmoveto{\pgfpoint{85.423309pt}{140.777435pt}}
\pgflineto{\pgfpoint{85.423309pt}{140.987396pt}}
\pgflineto{\pgfpoint{84.427322pt}{140.777435pt}}
\pgfpathclose
\pgfusepath{fill,stroke}
\pgfpathmoveto{\pgfpoint{86.419289pt}{140.777435pt}}
\pgflineto{\pgfpoint{85.423309pt}{140.987396pt}}
\pgflineto{\pgfpoint{85.423309pt}{140.777435pt}}
\pgfpathclose
\pgfusepath{fill,stroke}
\pgfpathmoveto{\pgfpoint{73.471497pt}{140.777435pt}}
\pgflineto{\pgfpoint{74.467484pt}{140.777435pt}}
\pgflineto{\pgfpoint{74.467484pt}{141.606003pt}}
\pgfpathclose
\pgfusepath{fill,stroke}
\pgfpathmoveto{\pgfpoint{75.463470pt}{145.536850pt}}
\pgflineto{\pgfpoint{74.467484pt}{141.606003pt}}
\pgflineto{\pgfpoint{74.467484pt}{140.777435pt}}
\pgfpathclose
\pgfusepath{fill,stroke}
\pgfpathmoveto{\pgfpoint{75.463470pt}{140.777435pt}}
\pgflineto{\pgfpoint{75.463470pt}{145.536850pt}}
\pgflineto{\pgfpoint{74.467484pt}{140.777435pt}}
\pgfpathclose
\pgfusepath{fill,stroke}
\pgfpathmoveto{\pgfpoint{76.459442pt}{143.382019pt}}
\pgflineto{\pgfpoint{75.463470pt}{145.536850pt}}
\pgflineto{\pgfpoint{75.463470pt}{140.777435pt}}
\pgfpathclose
\pgfusepath{fill,stroke}
\pgfpathmoveto{\pgfpoint{76.459442pt}{140.777435pt}}
\pgflineto{\pgfpoint{76.459442pt}{143.382019pt}}
\pgflineto{\pgfpoint{75.463470pt}{140.777435pt}}
\pgfpathclose
\pgfusepath{fill,stroke}
\pgfpathmoveto{\pgfpoint{77.455437pt}{142.594910pt}}
\pgflineto{\pgfpoint{76.459442pt}{143.382019pt}}
\pgflineto{\pgfpoint{76.459442pt}{140.777435pt}}
\pgfpathclose
\pgfusepath{fill,stroke}
\pgfpathmoveto{\pgfpoint{77.455437pt}{140.777435pt}}
\pgflineto{\pgfpoint{77.455437pt}{142.594910pt}}
\pgflineto{\pgfpoint{76.459442pt}{140.777435pt}}
\pgfpathclose
\pgfusepath{fill,stroke}
\pgfpathmoveto{\pgfpoint{78.451424pt}{140.777435pt}}
\pgflineto{\pgfpoint{77.455437pt}{142.594910pt}}
\pgflineto{\pgfpoint{77.455437pt}{140.777435pt}}
\pgfpathclose
\pgfusepath{fill,stroke}
\pgfpathmoveto{\pgfpoint{68.491577pt}{140.777435pt}}
\pgflineto{\pgfpoint{69.487564pt}{140.777435pt}}
\pgflineto{\pgfpoint{69.487564pt}{149.515793pt}}
\pgfpathclose
\pgfusepath{fill,stroke}
\pgfpathmoveto{\pgfpoint{70.483551pt}{141.832703pt}}
\pgflineto{\pgfpoint{69.487564pt}{149.515793pt}}
\pgflineto{\pgfpoint{69.487564pt}{140.777435pt}}
\pgfpathclose
\pgfusepath{fill,stroke}
\pgfpathmoveto{\pgfpoint{70.483551pt}{140.777435pt}}
\pgflineto{\pgfpoint{70.483551pt}{141.832703pt}}
\pgflineto{\pgfpoint{69.487564pt}{140.777435pt}}
\pgfpathclose
\pgfusepath{fill,stroke}
\pgfpathmoveto{\pgfpoint{71.479530pt}{190.848999pt}}
\pgflineto{\pgfpoint{70.483551pt}{141.832703pt}}
\pgflineto{\pgfpoint{70.483551pt}{140.777435pt}}
\pgfpathclose
\pgfusepath{fill,stroke}
\pgfpathmoveto{\pgfpoint{71.479530pt}{140.777435pt}}
\pgflineto{\pgfpoint{71.479530pt}{190.848999pt}}
\pgflineto{\pgfpoint{70.483551pt}{140.777435pt}}
\pgfpathclose
\pgfusepath{fill,stroke}
\pgfpathmoveto{\pgfpoint{72.475510pt}{148.664902pt}}
\pgflineto{\pgfpoint{71.479530pt}{190.848999pt}}
\pgflineto{\pgfpoint{71.479530pt}{140.777435pt}}
\pgfpathclose
\pgfusepath{fill,stroke}
\pgfpathmoveto{\pgfpoint{72.475510pt}{140.777435pt}}
\pgflineto{\pgfpoint{72.475510pt}{148.664902pt}}
\pgflineto{\pgfpoint{71.479530pt}{140.777435pt}}
\pgfpathclose
\pgfusepath{fill,stroke}
\pgfpathmoveto{\pgfpoint{73.471497pt}{140.777435pt}}
\pgflineto{\pgfpoint{72.475510pt}{148.664902pt}}
\pgflineto{\pgfpoint{72.475510pt}{140.777435pt}}
\pgfpathclose
\pgfusepath{fill,stroke}
\pgfpathmoveto{\pgfpoint{50.563873pt}{140.777435pt}}
\pgflineto{\pgfpoint{51.559845pt}{140.777435pt}}
\pgflineto{\pgfpoint{51.559845pt}{146.765289pt}}
\pgfpathclose
\pgfusepath{fill,stroke}
\pgfpathmoveto{\pgfpoint{52.555840pt}{142.530624pt}}
\pgflineto{\pgfpoint{51.559845pt}{146.765289pt}}
\pgflineto{\pgfpoint{51.559845pt}{140.777435pt}}
\pgfpathclose
\pgfusepath{fill,stroke}
\pgfpathmoveto{\pgfpoint{52.555840pt}{140.777435pt}}
\pgflineto{\pgfpoint{52.555840pt}{142.530624pt}}
\pgflineto{\pgfpoint{51.559845pt}{140.777435pt}}
\pgfpathclose
\pgfusepath{fill,stroke}
\pgfpathmoveto{\pgfpoint{53.551819pt}{146.137848pt}}
\pgflineto{\pgfpoint{52.555840pt}{142.530624pt}}
\pgflineto{\pgfpoint{52.555840pt}{140.777435pt}}
\pgfpathclose
\pgfusepath{fill,stroke}
\pgfpathmoveto{\pgfpoint{53.551819pt}{140.777435pt}}
\pgflineto{\pgfpoint{53.551819pt}{146.137848pt}}
\pgflineto{\pgfpoint{52.555840pt}{140.777435pt}}
\pgfpathclose
\pgfusepath{fill,stroke}
\pgfpathmoveto{\pgfpoint{54.547806pt}{142.199554pt}}
\pgflineto{\pgfpoint{53.551819pt}{146.137848pt}}
\pgflineto{\pgfpoint{53.551819pt}{140.777435pt}}
\pgfpathclose
\pgfusepath{fill,stroke}
\pgfpathmoveto{\pgfpoint{54.547806pt}{140.777435pt}}
\pgflineto{\pgfpoint{54.547806pt}{142.199554pt}}
\pgflineto{\pgfpoint{53.551819pt}{140.777435pt}}
\pgfpathclose
\pgfusepath{fill,stroke}
\pgfpathmoveto{\pgfpoint{55.543785pt}{196.136078pt}}
\pgflineto{\pgfpoint{54.547806pt}{142.199554pt}}
\pgflineto{\pgfpoint{54.547806pt}{140.777435pt}}
\pgfpathclose
\pgfusepath{fill,stroke}
\pgfpathmoveto{\pgfpoint{55.543785pt}{140.777435pt}}
\pgflineto{\pgfpoint{55.543785pt}{196.136078pt}}
\pgflineto{\pgfpoint{54.547806pt}{140.777435pt}}
\pgfpathclose
\pgfusepath{fill,stroke}
\pgfpathmoveto{\pgfpoint{56.539772pt}{168.281158pt}}
\pgflineto{\pgfpoint{55.543785pt}{196.136078pt}}
\pgflineto{\pgfpoint{55.543785pt}{140.777435pt}}
\pgfpathclose
\pgfusepath{fill,stroke}
\pgfpathmoveto{\pgfpoint{56.539772pt}{140.777435pt}}
\pgflineto{\pgfpoint{56.539772pt}{168.281158pt}}
\pgflineto{\pgfpoint{55.543785pt}{140.777435pt}}
\pgfpathclose
\pgfusepath{fill,stroke}
\pgfpathmoveto{\pgfpoint{57.535751pt}{147.868973pt}}
\pgflineto{\pgfpoint{56.539772pt}{168.281158pt}}
\pgflineto{\pgfpoint{56.539772pt}{140.777435pt}}
\pgfpathclose
\pgfusepath{fill,stroke}
\pgfpathmoveto{\pgfpoint{57.535751pt}{140.777435pt}}
\pgflineto{\pgfpoint{57.535751pt}{147.868973pt}}
\pgflineto{\pgfpoint{56.539772pt}{140.777435pt}}
\pgfpathclose
\pgfusepath{fill,stroke}
\pgfpathmoveto{\pgfpoint{58.531738pt}{141.160339pt}}
\pgflineto{\pgfpoint{57.535751pt}{147.868973pt}}
\pgflineto{\pgfpoint{57.535751pt}{140.777435pt}}
\pgfpathclose
\pgfusepath{fill,stroke}
\pgfpathmoveto{\pgfpoint{58.531738pt}{140.777435pt}}
\pgflineto{\pgfpoint{58.531738pt}{141.160339pt}}
\pgflineto{\pgfpoint{57.535751pt}{140.777435pt}}
\pgfpathclose
\pgfusepath{fill,stroke}
\pgfpathmoveto{\pgfpoint{59.527725pt}{141.894562pt}}
\pgflineto{\pgfpoint{58.531738pt}{141.160339pt}}
\pgflineto{\pgfpoint{58.531738pt}{140.777435pt}}
\pgfpathclose
\pgfusepath{fill,stroke}
\pgfpathmoveto{\pgfpoint{59.527725pt}{140.777435pt}}
\pgflineto{\pgfpoint{59.527725pt}{141.894562pt}}
\pgflineto{\pgfpoint{58.531738pt}{140.777435pt}}
\pgfpathclose
\pgfusepath{fill,stroke}
\pgfpathmoveto{\pgfpoint{60.523712pt}{142.042740pt}}
\pgflineto{\pgfpoint{59.527725pt}{141.894562pt}}
\pgflineto{\pgfpoint{59.527725pt}{140.777435pt}}
\pgfpathclose
\pgfusepath{fill,stroke}
\pgfpathmoveto{\pgfpoint{60.523712pt}{140.777435pt}}
\pgflineto{\pgfpoint{60.523712pt}{142.042740pt}}
\pgflineto{\pgfpoint{59.527725pt}{140.777435pt}}
\pgfpathclose
\pgfusepath{fill,stroke}
\pgfpathmoveto{\pgfpoint{61.519691pt}{141.419281pt}}
\pgflineto{\pgfpoint{60.523712pt}{142.042740pt}}
\pgflineto{\pgfpoint{60.523712pt}{140.777435pt}}
\pgfpathclose
\pgfusepath{fill,stroke}
\pgfpathmoveto{\pgfpoint{61.519691pt}{140.777435pt}}
\pgflineto{\pgfpoint{61.519691pt}{141.419281pt}}
\pgflineto{\pgfpoint{60.523712pt}{140.777435pt}}
\pgfpathclose
\pgfusepath{fill,stroke}
\pgfpathmoveto{\pgfpoint{62.515678pt}{202.358673pt}}
\pgflineto{\pgfpoint{61.519691pt}{141.419281pt}}
\pgflineto{\pgfpoint{61.519691pt}{140.777435pt}}
\pgfpathclose
\pgfusepath{fill,stroke}
\pgfpathmoveto{\pgfpoint{62.515678pt}{140.777435pt}}
\pgflineto{\pgfpoint{62.515678pt}{202.358673pt}}
\pgflineto{\pgfpoint{61.519691pt}{140.777435pt}}
\pgfpathclose
\pgfusepath{fill,stroke}
\pgfpathmoveto{\pgfpoint{63.511658pt}{143.879868pt}}
\pgflineto{\pgfpoint{62.515678pt}{202.358673pt}}
\pgflineto{\pgfpoint{62.515678pt}{140.777435pt}}
\pgfpathclose
\pgfusepath{fill,stroke}
\pgfpathmoveto{\pgfpoint{63.511658pt}{140.777435pt}}
\pgflineto{\pgfpoint{63.511658pt}{143.879868pt}}
\pgflineto{\pgfpoint{62.515678pt}{140.777435pt}}
\pgfpathclose
\pgfusepath{fill,stroke}
\pgfpathmoveto{\pgfpoint{64.507637pt}{144.233887pt}}
\pgflineto{\pgfpoint{63.511658pt}{143.879868pt}}
\pgflineto{\pgfpoint{63.511658pt}{140.777435pt}}
\pgfpathclose
\pgfusepath{fill,stroke}
\pgfpathmoveto{\pgfpoint{64.507637pt}{140.777435pt}}
\pgflineto{\pgfpoint{64.507637pt}{144.233887pt}}
\pgflineto{\pgfpoint{63.511658pt}{140.777435pt}}
\pgfpathclose
\pgfusepath{fill,stroke}
\pgfpathmoveto{\pgfpoint{65.503624pt}{142.070892pt}}
\pgflineto{\pgfpoint{64.507637pt}{144.233887pt}}
\pgflineto{\pgfpoint{64.507637pt}{140.777435pt}}
\pgfpathclose
\pgfusepath{fill,stroke}
\pgfpathmoveto{\pgfpoint{65.503624pt}{140.777435pt}}
\pgflineto{\pgfpoint{65.503624pt}{142.070892pt}}
\pgflineto{\pgfpoint{64.507637pt}{140.777435pt}}
\pgfpathclose
\pgfusepath{fill,stroke}
\pgfpathmoveto{\pgfpoint{66.499619pt}{141.913147pt}}
\pgflineto{\pgfpoint{65.503624pt}{142.070892pt}}
\pgflineto{\pgfpoint{65.503624pt}{140.777435pt}}
\pgfpathclose
\pgfusepath{fill,stroke}
\pgfpathmoveto{\pgfpoint{66.499619pt}{140.777435pt}}
\pgflineto{\pgfpoint{66.499619pt}{141.913147pt}}
\pgflineto{\pgfpoint{65.503624pt}{140.777435pt}}
\pgfpathclose
\pgfusepath{fill,stroke}
\pgfpathmoveto{\pgfpoint{67.495590pt}{140.777435pt}}
\pgflineto{\pgfpoint{66.499619pt}{141.913147pt}}
\pgflineto{\pgfpoint{66.499619pt}{140.777435pt}}
\pgfpathclose
\pgfusepath{fill,stroke}
\pgfpathmoveto{\pgfpoint{46.579933pt}{140.777435pt}}
\pgflineto{\pgfpoint{47.575912pt}{140.777435pt}}
\pgflineto{\pgfpoint{47.575912pt}{141.646210pt}}
\pgfpathclose
\pgfusepath{fill,stroke}
\pgfpathmoveto{\pgfpoint{48.571899pt}{141.530609pt}}
\pgflineto{\pgfpoint{47.575912pt}{141.646210pt}}
\pgflineto{\pgfpoint{47.575912pt}{140.777435pt}}
\pgfpathclose
\pgfusepath{fill,stroke}
\pgfpathmoveto{\pgfpoint{48.571899pt}{140.777435pt}}
\pgflineto{\pgfpoint{48.571899pt}{141.530609pt}}
\pgflineto{\pgfpoint{47.575912pt}{140.777435pt}}
\pgfpathclose
\pgfusepath{fill,stroke}
\pgfpathmoveto{\pgfpoint{49.567879pt}{143.458160pt}}
\pgflineto{\pgfpoint{48.571899pt}{141.530609pt}}
\pgflineto{\pgfpoint{48.571899pt}{140.777435pt}}
\pgfpathclose
\pgfusepath{fill,stroke}
\pgfpathmoveto{\pgfpoint{49.567879pt}{140.777435pt}}
\pgflineto{\pgfpoint{49.567879pt}{143.458160pt}}
\pgflineto{\pgfpoint{48.571899pt}{140.777435pt}}
\pgfpathclose
\pgfusepath{fill,stroke}
\pgfpathmoveto{\pgfpoint{50.563873pt}{140.777435pt}}
\pgflineto{\pgfpoint{49.567879pt}{143.458160pt}}
\pgflineto{\pgfpoint{49.567879pt}{140.777435pt}}
\pgfpathclose
\pgfusepath{fill,stroke}
\pgfpathmoveto{\pgfpoint{44.587967pt}{140.777435pt}}
\pgflineto{\pgfpoint{45.583946pt}{140.777435pt}}
\pgflineto{\pgfpoint{45.583946pt}{148.946091pt}}
\pgfpathclose
\pgfusepath{fill,stroke}
\pgfpathmoveto{\pgfpoint{46.579933pt}{140.777435pt}}
\pgflineto{\pgfpoint{45.583946pt}{148.946091pt}}
\pgflineto{\pgfpoint{45.583946pt}{140.777435pt}}
\pgfpathclose
\pgfusepath{fill,stroke}
\pgfpathmoveto{\pgfpoint{42.595993pt}{140.777435pt}}
\pgflineto{\pgfpoint{43.591980pt}{140.777435pt}}
\pgflineto{\pgfpoint{43.591980pt}{186.324295pt}}
\pgfpathclose
\pgfusepath{fill,stroke}
\pgfpathmoveto{\pgfpoint{44.587967pt}{140.777435pt}}
\pgflineto{\pgfpoint{43.591980pt}{186.324295pt}}
\pgflineto{\pgfpoint{43.591980pt}{140.777435pt}}
\pgfpathclose
\pgfusepath{fill,stroke}
\color[rgb]{0.000000,1.000000,0.000000}
\pgfpathmoveto{\pgfpoint{284.620087pt}{26.399979pt}}
\pgflineto{\pgfpoint{285.616089pt}{26.399979pt}}
\pgflineto{\pgfpoint{285.616089pt}{26.474419pt}}
\pgfpathclose
\pgfusepath{fill,stroke}
\pgfpathmoveto{\pgfpoint{286.612061pt}{26.399979pt}}
\pgflineto{\pgfpoint{285.616089pt}{26.474419pt}}
\pgflineto{\pgfpoint{285.616089pt}{26.399979pt}}
\pgfpathclose
\pgfusepath{fill,stroke}
\pgfpathmoveto{\pgfpoint{281.632141pt}{26.399979pt}}
\pgflineto{\pgfpoint{282.628113pt}{26.399979pt}}
\pgflineto{\pgfpoint{282.628113pt}{29.464996pt}}
\pgfpathclose
\pgfusepath{fill,stroke}
\pgfpathmoveto{\pgfpoint{283.624115pt}{28.196754pt}}
\pgflineto{\pgfpoint{282.628113pt}{29.464996pt}}
\pgflineto{\pgfpoint{282.628113pt}{26.399979pt}}
\pgfpathclose
\pgfusepath{fill,stroke}
\pgfpathmoveto{\pgfpoint{283.624115pt}{26.399979pt}}
\pgflineto{\pgfpoint{283.624115pt}{28.196754pt}}
\pgflineto{\pgfpoint{282.628113pt}{26.399979pt}}
\pgfpathclose
\pgfusepath{fill,stroke}
\pgfpathmoveto{\pgfpoint{284.620087pt}{26.399979pt}}
\pgflineto{\pgfpoint{283.624115pt}{28.196754pt}}
\pgflineto{\pgfpoint{283.624115pt}{26.399979pt}}
\pgfpathclose
\pgfusepath{fill,stroke}
\pgfpathmoveto{\pgfpoint{277.648193pt}{26.399979pt}}
\pgflineto{\pgfpoint{278.644196pt}{26.399979pt}}
\pgflineto{\pgfpoint{278.644196pt}{30.203644pt}}
\pgfpathclose
\pgfusepath{fill,stroke}
\pgfpathmoveto{\pgfpoint{279.640167pt}{35.907738pt}}
\pgflineto{\pgfpoint{278.644196pt}{30.203644pt}}
\pgflineto{\pgfpoint{278.644196pt}{26.399979pt}}
\pgfpathclose
\pgfusepath{fill,stroke}
\pgfpathmoveto{\pgfpoint{279.640167pt}{26.399979pt}}
\pgflineto{\pgfpoint{279.640167pt}{35.907738pt}}
\pgflineto{\pgfpoint{278.644196pt}{26.399979pt}}
\pgfpathclose
\pgfusepath{fill,stroke}
\pgfpathmoveto{\pgfpoint{280.636139pt}{26.399979pt}}
\pgflineto{\pgfpoint{279.640167pt}{35.907738pt}}
\pgflineto{\pgfpoint{279.640167pt}{26.399979pt}}
\pgfpathclose
\pgfusepath{fill,stroke}
\pgfpathmoveto{\pgfpoint{267.688354pt}{26.399979pt}}
\pgflineto{\pgfpoint{268.684326pt}{26.399979pt}}
\pgflineto{\pgfpoint{268.684326pt}{26.647873pt}}
\pgfpathclose
\pgfusepath{fill,stroke}
\pgfpathmoveto{\pgfpoint{269.680328pt}{26.596115pt}}
\pgflineto{\pgfpoint{268.684326pt}{26.647873pt}}
\pgflineto{\pgfpoint{268.684326pt}{26.399979pt}}
\pgfpathclose
\pgfusepath{fill,stroke}
\pgfpathmoveto{\pgfpoint{269.680328pt}{26.399979pt}}
\pgflineto{\pgfpoint{269.680328pt}{26.596115pt}}
\pgflineto{\pgfpoint{268.684326pt}{26.399979pt}}
\pgfpathclose
\pgfusepath{fill,stroke}
\pgfpathmoveto{\pgfpoint{270.676331pt}{27.976303pt}}
\pgflineto{\pgfpoint{269.680328pt}{26.596115pt}}
\pgflineto{\pgfpoint{269.680328pt}{26.399979pt}}
\pgfpathclose
\pgfusepath{fill,stroke}
\pgfpathmoveto{\pgfpoint{270.676331pt}{26.399979pt}}
\pgflineto{\pgfpoint{270.676331pt}{27.976303pt}}
\pgflineto{\pgfpoint{269.680328pt}{26.399979pt}}
\pgfpathclose
\pgfusepath{fill,stroke}
\pgfpathmoveto{\pgfpoint{271.672302pt}{42.660454pt}}
\pgflineto{\pgfpoint{270.676331pt}{27.976303pt}}
\pgflineto{\pgfpoint{270.676331pt}{26.399979pt}}
\pgfpathclose
\pgfusepath{fill,stroke}
\pgfpathmoveto{\pgfpoint{271.672302pt}{26.399979pt}}
\pgflineto{\pgfpoint{271.672302pt}{42.660454pt}}
\pgflineto{\pgfpoint{270.676331pt}{26.399979pt}}
\pgfpathclose
\pgfusepath{fill,stroke}
\pgfpathmoveto{\pgfpoint{272.668274pt}{26.867470pt}}
\pgflineto{\pgfpoint{271.672302pt}{42.660454pt}}
\pgflineto{\pgfpoint{271.672302pt}{26.399979pt}}
\pgfpathclose
\pgfusepath{fill,stroke}
\pgfpathmoveto{\pgfpoint{272.668274pt}{26.399979pt}}
\pgflineto{\pgfpoint{272.668274pt}{26.867470pt}}
\pgflineto{\pgfpoint{271.672302pt}{26.399979pt}}
\pgfpathclose
\pgfusepath{fill,stroke}
\pgfpathmoveto{\pgfpoint{273.664276pt}{26.399979pt}}
\pgflineto{\pgfpoint{272.668274pt}{26.867470pt}}
\pgflineto{\pgfpoint{272.668274pt}{26.399979pt}}
\pgfpathclose
\pgfusepath{fill,stroke}
\pgfpathmoveto{\pgfpoint{265.696411pt}{26.399979pt}}
\pgflineto{\pgfpoint{266.692383pt}{26.399979pt}}
\pgflineto{\pgfpoint{266.692383pt}{26.914734pt}}
\pgfpathclose
\pgfusepath{fill,stroke}
\pgfpathmoveto{\pgfpoint{267.688354pt}{26.399979pt}}
\pgflineto{\pgfpoint{266.692383pt}{26.914734pt}}
\pgflineto{\pgfpoint{266.692383pt}{26.399979pt}}
\pgfpathclose
\pgfusepath{fill,stroke}
\pgfpathmoveto{\pgfpoint{263.704407pt}{26.399979pt}}
\pgflineto{\pgfpoint{264.700409pt}{26.399979pt}}
\pgflineto{\pgfpoint{264.700409pt}{26.932861pt}}
\pgfpathclose
\pgfusepath{fill,stroke}
\pgfpathmoveto{\pgfpoint{265.696411pt}{26.399979pt}}
\pgflineto{\pgfpoint{264.700409pt}{26.932861pt}}
\pgflineto{\pgfpoint{264.700409pt}{26.399979pt}}
\pgfpathclose
\pgfusepath{fill,stroke}
\pgfpathmoveto{\pgfpoint{260.716492pt}{26.399979pt}}
\pgflineto{\pgfpoint{261.712463pt}{26.399979pt}}
\pgflineto{\pgfpoint{261.712463pt}{27.286751pt}}
\pgfpathclose
\pgfusepath{fill,stroke}
\pgfpathmoveto{\pgfpoint{262.708435pt}{26.399979pt}}
\pgflineto{\pgfpoint{261.712463pt}{27.286751pt}}
\pgflineto{\pgfpoint{261.712463pt}{26.399979pt}}
\pgfpathclose
\pgfusepath{fill,stroke}
\pgfpathmoveto{\pgfpoint{257.728516pt}{26.399979pt}}
\pgflineto{\pgfpoint{258.724518pt}{26.399979pt}}
\pgflineto{\pgfpoint{258.724518pt}{26.886185pt}}
\pgfpathclose
\pgfusepath{fill,stroke}
\pgfpathmoveto{\pgfpoint{259.720490pt}{26.399979pt}}
\pgflineto{\pgfpoint{258.724518pt}{26.886185pt}}
\pgflineto{\pgfpoint{258.724518pt}{26.399979pt}}
\pgfpathclose
\pgfusepath{fill,stroke}
\pgfpathmoveto{\pgfpoint{253.744598pt}{26.399979pt}}
\pgflineto{\pgfpoint{254.740570pt}{26.399979pt}}
\pgflineto{\pgfpoint{254.740570pt}{27.508163pt}}
\pgfpathclose
\pgfusepath{fill,stroke}
\pgfpathmoveto{\pgfpoint{255.736542pt}{26.399979pt}}
\pgflineto{\pgfpoint{254.740570pt}{27.508163pt}}
\pgflineto{\pgfpoint{254.740570pt}{26.399979pt}}
\pgfpathclose
\pgfusepath{fill,stroke}
\pgfpathmoveto{\pgfpoint{248.764679pt}{26.399979pt}}
\pgflineto{\pgfpoint{249.760651pt}{26.399979pt}}
\pgflineto{\pgfpoint{249.760651pt}{26.474113pt}}
\pgfpathclose
\pgfusepath{fill,stroke}
\pgfpathmoveto{\pgfpoint{250.756638pt}{26.708969pt}}
\pgflineto{\pgfpoint{249.760651pt}{26.474113pt}}
\pgflineto{\pgfpoint{249.760651pt}{26.399979pt}}
\pgfpathclose
\pgfusepath{fill,stroke}
\pgfpathmoveto{\pgfpoint{250.756638pt}{26.399979pt}}
\pgflineto{\pgfpoint{250.756638pt}{26.708969pt}}
\pgflineto{\pgfpoint{249.760651pt}{26.399979pt}}
\pgfpathclose
\pgfusepath{fill,stroke}
\pgfpathmoveto{\pgfpoint{251.752625pt}{26.399979pt}}
\pgflineto{\pgfpoint{250.756638pt}{26.708969pt}}
\pgflineto{\pgfpoint{250.756638pt}{26.399979pt}}
\pgfpathclose
\pgfusepath{fill,stroke}
\pgfpathmoveto{\pgfpoint{244.780731pt}{26.399979pt}}
\pgflineto{\pgfpoint{245.776718pt}{26.399979pt}}
\pgflineto{\pgfpoint{245.776718pt}{26.477646pt}}
\pgfpathclose
\pgfusepath{fill,stroke}
\pgfpathmoveto{\pgfpoint{246.772705pt}{26.550812pt}}
\pgflineto{\pgfpoint{245.776718pt}{26.477646pt}}
\pgflineto{\pgfpoint{245.776718pt}{26.399979pt}}
\pgfpathclose
\pgfusepath{fill,stroke}
\pgfpathmoveto{\pgfpoint{246.772705pt}{26.399979pt}}
\pgflineto{\pgfpoint{246.772705pt}{26.550812pt}}
\pgflineto{\pgfpoint{245.776718pt}{26.399979pt}}
\pgfpathclose
\pgfusepath{fill,stroke}
\pgfpathmoveto{\pgfpoint{247.768677pt}{41.200562pt}}
\pgflineto{\pgfpoint{246.772705pt}{26.550812pt}}
\pgflineto{\pgfpoint{246.772705pt}{26.399979pt}}
\pgfpathclose
\pgfusepath{fill,stroke}
\pgfpathmoveto{\pgfpoint{247.768677pt}{26.399979pt}}
\pgflineto{\pgfpoint{247.768677pt}{41.200562pt}}
\pgflineto{\pgfpoint{246.772705pt}{26.399979pt}}
\pgfpathclose
\pgfusepath{fill,stroke}
\pgfpathmoveto{\pgfpoint{248.764679pt}{26.399979pt}}
\pgflineto{\pgfpoint{247.768677pt}{41.200562pt}}
\pgflineto{\pgfpoint{247.768677pt}{26.399979pt}}
\pgfpathclose
\pgfusepath{fill,stroke}
\pgfpathmoveto{\pgfpoint{241.792786pt}{26.399979pt}}
\pgflineto{\pgfpoint{242.788757pt}{26.399979pt}}
\pgflineto{\pgfpoint{242.788757pt}{26.819893pt}}
\pgfpathclose
\pgfusepath{fill,stroke}
\pgfpathmoveto{\pgfpoint{243.784744pt}{27.221687pt}}
\pgflineto{\pgfpoint{242.788757pt}{26.819893pt}}
\pgflineto{\pgfpoint{242.788757pt}{26.399979pt}}
\pgfpathclose
\pgfusepath{fill,stroke}
\pgfpathmoveto{\pgfpoint{243.784744pt}{26.399979pt}}
\pgflineto{\pgfpoint{243.784744pt}{27.221687pt}}
\pgflineto{\pgfpoint{242.788757pt}{26.399979pt}}
\pgfpathclose
\pgfusepath{fill,stroke}
\pgfpathmoveto{\pgfpoint{244.780731pt}{26.399979pt}}
\pgflineto{\pgfpoint{243.784744pt}{27.221687pt}}
\pgflineto{\pgfpoint{243.784744pt}{26.399979pt}}
\pgfpathclose
\pgfusepath{fill,stroke}
\pgfpathmoveto{\pgfpoint{235.816864pt}{26.399979pt}}
\pgflineto{\pgfpoint{236.812866pt}{26.399979pt}}
\pgflineto{\pgfpoint{236.812866pt}{35.730797pt}}
\pgfpathclose
\pgfusepath{fill,stroke}
\pgfpathmoveto{\pgfpoint{237.808838pt}{26.516998pt}}
\pgflineto{\pgfpoint{236.812866pt}{35.730797pt}}
\pgflineto{\pgfpoint{236.812866pt}{26.399979pt}}
\pgfpathclose
\pgfusepath{fill,stroke}
\pgfpathmoveto{\pgfpoint{237.808838pt}{26.399979pt}}
\pgflineto{\pgfpoint{237.808838pt}{26.516998pt}}
\pgflineto{\pgfpoint{236.812866pt}{26.399979pt}}
\pgfpathclose
\pgfusepath{fill,stroke}
\pgfpathmoveto{\pgfpoint{238.804825pt}{26.399979pt}}
\pgflineto{\pgfpoint{237.808838pt}{26.516998pt}}
\pgflineto{\pgfpoint{237.808838pt}{26.399979pt}}
\pgfpathclose
\pgfusepath{fill,stroke}
\pgfpathmoveto{\pgfpoint{227.849014pt}{26.399979pt}}
\pgflineto{\pgfpoint{228.845001pt}{26.399979pt}}
\pgflineto{\pgfpoint{228.845001pt}{27.012459pt}}
\pgfpathclose
\pgfusepath{fill,stroke}
\pgfpathmoveto{\pgfpoint{229.840973pt}{26.558662pt}}
\pgflineto{\pgfpoint{228.845001pt}{27.012459pt}}
\pgflineto{\pgfpoint{228.845001pt}{26.399979pt}}
\pgfpathclose
\pgfusepath{fill,stroke}
\pgfpathmoveto{\pgfpoint{229.840973pt}{26.399979pt}}
\pgflineto{\pgfpoint{229.840973pt}{26.558662pt}}
\pgflineto{\pgfpoint{228.845001pt}{26.399979pt}}
\pgfpathclose
\pgfusepath{fill,stroke}
\pgfpathmoveto{\pgfpoint{230.836945pt}{26.682472pt}}
\pgflineto{\pgfpoint{229.840973pt}{26.558662pt}}
\pgflineto{\pgfpoint{229.840973pt}{26.399979pt}}
\pgfpathclose
\pgfusepath{fill,stroke}
\pgfpathmoveto{\pgfpoint{230.836945pt}{26.399979pt}}
\pgflineto{\pgfpoint{230.836945pt}{26.682472pt}}
\pgflineto{\pgfpoint{229.840973pt}{26.399979pt}}
\pgfpathclose
\pgfusepath{fill,stroke}
\pgfpathmoveto{\pgfpoint{231.832932pt}{26.399979pt}}
\pgflineto{\pgfpoint{230.836945pt}{26.682472pt}}
\pgflineto{\pgfpoint{230.836945pt}{26.399979pt}}
\pgfpathclose
\pgfusepath{fill,stroke}
\pgfpathmoveto{\pgfpoint{224.861053pt}{26.399979pt}}
\pgflineto{\pgfpoint{225.857040pt}{26.399979pt}}
\pgflineto{\pgfpoint{225.857040pt}{28.125343pt}}
\pgfpathclose
\pgfusepath{fill,stroke}
\pgfpathmoveto{\pgfpoint{226.853027pt}{26.399979pt}}
\pgflineto{\pgfpoint{225.857040pt}{28.125343pt}}
\pgflineto{\pgfpoint{225.857040pt}{26.399979pt}}
\pgfpathclose
\pgfusepath{fill,stroke}
\pgfpathmoveto{\pgfpoint{216.893188pt}{26.399979pt}}
\pgflineto{\pgfpoint{217.889160pt}{26.399979pt}}
\pgflineto{\pgfpoint{217.889160pt}{66.077660pt}}
\pgfpathclose
\pgfusepath{fill,stroke}
\pgfpathmoveto{\pgfpoint{218.885147pt}{26.399979pt}}
\pgflineto{\pgfpoint{217.889160pt}{66.077660pt}}
\pgflineto{\pgfpoint{217.889160pt}{26.399979pt}}
\pgfpathclose
\pgfusepath{fill,stroke}
\pgfpathmoveto{\pgfpoint{213.905228pt}{26.399979pt}}
\pgflineto{\pgfpoint{214.901215pt}{26.399979pt}}
\pgflineto{\pgfpoint{214.901215pt}{27.407372pt}}
\pgfpathclose
\pgfusepath{fill,stroke}
\pgfpathmoveto{\pgfpoint{215.897217pt}{26.495361pt}}
\pgflineto{\pgfpoint{214.901215pt}{27.407372pt}}
\pgflineto{\pgfpoint{214.901215pt}{26.399979pt}}
\pgfpathclose
\pgfusepath{fill,stroke}
\pgfpathmoveto{\pgfpoint{215.897217pt}{26.399979pt}}
\pgflineto{\pgfpoint{215.897217pt}{26.495361pt}}
\pgflineto{\pgfpoint{214.901215pt}{26.399979pt}}
\pgfpathclose
\pgfusepath{fill,stroke}
\pgfpathmoveto{\pgfpoint{216.893188pt}{26.399979pt}}
\pgflineto{\pgfpoint{215.897217pt}{26.495361pt}}
\pgflineto{\pgfpoint{215.897217pt}{26.399979pt}}
\pgfpathclose
\pgfusepath{fill,stroke}
\pgfpathmoveto{\pgfpoint{209.921295pt}{26.399979pt}}
\pgflineto{\pgfpoint{210.917267pt}{26.399979pt}}
\pgflineto{\pgfpoint{210.917267pt}{56.628197pt}}
\pgfpathclose
\pgfusepath{fill,stroke}
\pgfpathmoveto{\pgfpoint{211.913269pt}{30.417183pt}}
\pgflineto{\pgfpoint{210.917267pt}{56.628197pt}}
\pgflineto{\pgfpoint{210.917267pt}{26.399979pt}}
\pgfpathclose
\pgfusepath{fill,stroke}
\pgfpathmoveto{\pgfpoint{211.913269pt}{26.399979pt}}
\pgflineto{\pgfpoint{211.913269pt}{30.417183pt}}
\pgflineto{\pgfpoint{210.917267pt}{26.399979pt}}
\pgfpathclose
\pgfusepath{fill,stroke}
\pgfpathmoveto{\pgfpoint{212.909241pt}{47.374290pt}}
\pgflineto{\pgfpoint{211.913269pt}{30.417183pt}}
\pgflineto{\pgfpoint{211.913269pt}{26.399979pt}}
\pgfpathclose
\pgfusepath{fill,stroke}
\pgfpathmoveto{\pgfpoint{212.909241pt}{26.399979pt}}
\pgflineto{\pgfpoint{212.909241pt}{47.374290pt}}
\pgflineto{\pgfpoint{211.913269pt}{26.399979pt}}
\pgfpathclose
\pgfusepath{fill,stroke}
\pgfpathmoveto{\pgfpoint{213.905228pt}{26.399979pt}}
\pgflineto{\pgfpoint{212.909241pt}{47.374290pt}}
\pgflineto{\pgfpoint{212.909241pt}{26.399979pt}}
\pgfpathclose
\pgfusepath{fill,stroke}
\pgfpathmoveto{\pgfpoint{207.929337pt}{26.399979pt}}
\pgflineto{\pgfpoint{208.925323pt}{26.399979pt}}
\pgflineto{\pgfpoint{208.925323pt}{29.200577pt}}
\pgfpathclose
\pgfusepath{fill,stroke}
\pgfpathmoveto{\pgfpoint{209.921295pt}{26.399979pt}}
\pgflineto{\pgfpoint{208.925323pt}{29.200577pt}}
\pgflineto{\pgfpoint{208.925323pt}{26.399979pt}}
\pgfpathclose
\pgfusepath{fill,stroke}
\pgfpathmoveto{\pgfpoint{204.941376pt}{26.399979pt}}
\pgflineto{\pgfpoint{205.937347pt}{26.399979pt}}
\pgflineto{\pgfpoint{205.937347pt}{29.413773pt}}
\pgfpathclose
\pgfusepath{fill,stroke}
\pgfpathmoveto{\pgfpoint{206.933334pt}{28.372978pt}}
\pgflineto{\pgfpoint{205.937347pt}{29.413773pt}}
\pgflineto{\pgfpoint{205.937347pt}{26.399979pt}}
\pgfpathclose
\pgfusepath{fill,stroke}
\pgfpathmoveto{\pgfpoint{206.933334pt}{26.399979pt}}
\pgflineto{\pgfpoint{206.933334pt}{28.372978pt}}
\pgflineto{\pgfpoint{205.937347pt}{26.399979pt}}
\pgfpathclose
\pgfusepath{fill,stroke}
\pgfpathmoveto{\pgfpoint{207.929337pt}{26.399979pt}}
\pgflineto{\pgfpoint{206.933334pt}{28.372978pt}}
\pgflineto{\pgfpoint{206.933334pt}{26.399979pt}}
\pgfpathclose
\pgfusepath{fill,stroke}
\pgfpathmoveto{\pgfpoint{202.949402pt}{26.399979pt}}
\pgflineto{\pgfpoint{203.945404pt}{26.399979pt}}
\pgflineto{\pgfpoint{203.945404pt}{26.673592pt}}
\pgfpathclose
\pgfusepath{fill,stroke}
\pgfpathmoveto{\pgfpoint{204.941376pt}{26.399979pt}}
\pgflineto{\pgfpoint{203.945404pt}{26.673592pt}}
\pgflineto{\pgfpoint{203.945404pt}{26.399979pt}}
\pgfpathclose
\pgfusepath{fill,stroke}
\pgfpathmoveto{\pgfpoint{196.973511pt}{26.399979pt}}
\pgflineto{\pgfpoint{197.969498pt}{26.399979pt}}
\pgflineto{\pgfpoint{197.969498pt}{27.127144pt}}
\pgfpathclose
\pgfusepath{fill,stroke}
\pgfpathmoveto{\pgfpoint{198.965469pt}{26.399979pt}}
\pgflineto{\pgfpoint{197.969498pt}{27.127144pt}}
\pgflineto{\pgfpoint{197.969498pt}{26.399979pt}}
\pgfpathclose
\pgfusepath{fill,stroke}
\pgfpathmoveto{\pgfpoint{193.985565pt}{26.399979pt}}
\pgflineto{\pgfpoint{194.981537pt}{26.399979pt}}
\pgflineto{\pgfpoint{194.981537pt}{28.872696pt}}
\pgfpathclose
\pgfusepath{fill,stroke}
\pgfpathmoveto{\pgfpoint{195.977524pt}{27.054726pt}}
\pgflineto{\pgfpoint{194.981537pt}{28.872696pt}}
\pgflineto{\pgfpoint{194.981537pt}{26.399979pt}}
\pgfpathclose
\pgfusepath{fill,stroke}
\pgfpathmoveto{\pgfpoint{195.977524pt}{26.399979pt}}
\pgflineto{\pgfpoint{195.977524pt}{27.054726pt}}
\pgflineto{\pgfpoint{194.981537pt}{26.399979pt}}
\pgfpathclose
\pgfusepath{fill,stroke}
\pgfpathmoveto{\pgfpoint{196.973511pt}{26.399979pt}}
\pgflineto{\pgfpoint{195.977524pt}{27.054726pt}}
\pgflineto{\pgfpoint{195.977524pt}{26.399979pt}}
\pgfpathclose
\pgfusepath{fill,stroke}
\pgfpathmoveto{\pgfpoint{190.997604pt}{26.399979pt}}
\pgflineto{\pgfpoint{191.993591pt}{26.399979pt}}
\pgflineto{\pgfpoint{191.993591pt}{31.034332pt}}
\pgfpathclose
\pgfusepath{fill,stroke}
\pgfpathmoveto{\pgfpoint{192.989563pt}{26.399979pt}}
\pgflineto{\pgfpoint{191.993591pt}{31.034332pt}}
\pgflineto{\pgfpoint{191.993591pt}{26.399979pt}}
\pgfpathclose
\pgfusepath{fill,stroke}
\pgfpathmoveto{\pgfpoint{189.005630pt}{26.399979pt}}
\pgflineto{\pgfpoint{190.001617pt}{26.399979pt}}
\pgflineto{\pgfpoint{190.001617pt}{34.817444pt}}
\pgfpathclose
\pgfusepath{fill,stroke}
\pgfpathmoveto{\pgfpoint{190.997604pt}{26.399979pt}}
\pgflineto{\pgfpoint{190.001617pt}{34.817444pt}}
\pgflineto{\pgfpoint{190.001617pt}{26.399979pt}}
\pgfpathclose
\pgfusepath{fill,stroke}
\pgfpathmoveto{\pgfpoint{182.033752pt}{26.399979pt}}
\pgflineto{\pgfpoint{183.029724pt}{26.399979pt}}
\pgflineto{\pgfpoint{183.029724pt}{26.463455pt}}
\pgfpathclose
\pgfusepath{fill,stroke}
\pgfpathmoveto{\pgfpoint{184.025711pt}{26.399979pt}}
\pgflineto{\pgfpoint{183.029724pt}{26.463455pt}}
\pgflineto{\pgfpoint{183.029724pt}{26.399979pt}}
\pgfpathclose
\pgfusepath{fill,stroke}
\pgfpathmoveto{\pgfpoint{177.053818pt}{26.399979pt}}
\pgflineto{\pgfpoint{178.049805pt}{91.264786pt}}
\pgflineto{\pgfpoint{177.581543pt}{91.264786pt}}
\pgfpathclose
\pgfusepath{fill,stroke}
\pgfpathmoveto{\pgfpoint{177.053818pt}{26.399979pt}}
\pgflineto{\pgfpoint{178.049805pt}{26.399979pt}}
\pgflineto{\pgfpoint{178.049805pt}{91.264786pt}}
\pgfpathclose
\pgfusepath{fill,stroke}
\pgfpathmoveto{\pgfpoint{179.045792pt}{26.504753pt}}
\pgflineto{\pgfpoint{178.049805pt}{91.264786pt}}
\pgflineto{\pgfpoint{178.049805pt}{26.399979pt}}
\pgfpathclose
\pgfusepath{fill,stroke}
\pgfpathmoveto{\pgfpoint{179.045792pt}{26.504753pt}}
\pgflineto{\pgfpoint{178.518478pt}{91.264793pt}}
\pgflineto{\pgfpoint{178.049805pt}{91.264786pt}}
\pgfpathclose
\pgfusepath{fill,stroke}
\pgfpathmoveto{\pgfpoint{179.045792pt}{26.399979pt}}
\pgflineto{\pgfpoint{179.045792pt}{26.504753pt}}
\pgflineto{\pgfpoint{178.049805pt}{26.399979pt}}
\pgfpathclose
\pgfusepath{fill,stroke}
\pgfpathmoveto{\pgfpoint{180.041779pt}{30.030716pt}}
\pgflineto{\pgfpoint{179.045792pt}{26.504753pt}}
\pgflineto{\pgfpoint{179.045792pt}{26.399979pt}}
\pgfpathclose
\pgfusepath{fill,stroke}
\pgfpathmoveto{\pgfpoint{180.041779pt}{26.399979pt}}
\pgflineto{\pgfpoint{180.041779pt}{30.030716pt}}
\pgflineto{\pgfpoint{179.045792pt}{26.399979pt}}
\pgfpathclose
\pgfusepath{fill,stroke}
\pgfpathmoveto{\pgfpoint{181.037766pt}{26.399979pt}}
\pgflineto{\pgfpoint{180.041779pt}{30.030716pt}}
\pgflineto{\pgfpoint{180.041779pt}{26.399979pt}}
\pgfpathclose
\pgfusepath{fill,stroke}
\pgfpathmoveto{\pgfpoint{169.085953pt}{26.399979pt}}
\pgflineto{\pgfpoint{170.081940pt}{26.399979pt}}
\pgflineto{\pgfpoint{170.081940pt}{26.839508pt}}
\pgfpathclose
\pgfusepath{fill,stroke}
\pgfpathmoveto{\pgfpoint{171.077911pt}{26.399979pt}}
\pgflineto{\pgfpoint{170.081940pt}{26.839508pt}}
\pgflineto{\pgfpoint{170.081940pt}{26.399979pt}}
\pgfpathclose
\pgfusepath{fill,stroke}
\pgfpathmoveto{\pgfpoint{166.098007pt}{26.399979pt}}
\pgflineto{\pgfpoint{167.093994pt}{26.399979pt}}
\pgflineto{\pgfpoint{167.093994pt}{30.536880pt}}
\pgfpathclose
\pgfusepath{fill,stroke}
\pgfpathmoveto{\pgfpoint{168.089966pt}{26.447098pt}}
\pgflineto{\pgfpoint{167.093994pt}{30.536880pt}}
\pgflineto{\pgfpoint{167.093994pt}{26.399979pt}}
\pgfpathclose
\pgfusepath{fill,stroke}
\pgfpathmoveto{\pgfpoint{168.089966pt}{26.399979pt}}
\pgflineto{\pgfpoint{168.089966pt}{26.447098pt}}
\pgflineto{\pgfpoint{167.093994pt}{26.399979pt}}
\pgfpathclose
\pgfusepath{fill,stroke}
\pgfpathmoveto{\pgfpoint{169.085953pt}{26.399979pt}}
\pgflineto{\pgfpoint{168.089966pt}{26.447098pt}}
\pgflineto{\pgfpoint{168.089966pt}{26.399979pt}}
\pgfpathclose
\pgfusepath{fill,stroke}
\pgfpathmoveto{\pgfpoint{164.106033pt}{26.399979pt}}
\pgflineto{\pgfpoint{165.102020pt}{26.399979pt}}
\pgflineto{\pgfpoint{165.102020pt}{26.737396pt}}
\pgfpathclose
\pgfusepath{fill,stroke}
\pgfpathmoveto{\pgfpoint{166.098007pt}{26.399979pt}}
\pgflineto{\pgfpoint{165.102020pt}{26.737396pt}}
\pgflineto{\pgfpoint{165.102020pt}{26.399979pt}}
\pgfpathclose
\pgfusepath{fill,stroke}
\pgfpathmoveto{\pgfpoint{160.122101pt}{26.399979pt}}
\pgflineto{\pgfpoint{161.118088pt}{26.399979pt}}
\pgflineto{\pgfpoint{161.118088pt}{26.483330pt}}
\pgfpathclose
\pgfusepath{fill,stroke}
\pgfpathmoveto{\pgfpoint{162.114075pt}{26.604111pt}}
\pgflineto{\pgfpoint{161.118088pt}{26.483330pt}}
\pgflineto{\pgfpoint{161.118088pt}{26.399979pt}}
\pgfpathclose
\pgfusepath{fill,stroke}
\pgfpathmoveto{\pgfpoint{162.114075pt}{26.399979pt}}
\pgflineto{\pgfpoint{162.114075pt}{26.604111pt}}
\pgflineto{\pgfpoint{161.118088pt}{26.399979pt}}
\pgfpathclose
\pgfusepath{fill,stroke}
\pgfpathmoveto{\pgfpoint{163.110062pt}{28.794250pt}}
\pgflineto{\pgfpoint{162.114075pt}{26.604111pt}}
\pgflineto{\pgfpoint{162.114075pt}{26.399979pt}}
\pgfpathclose
\pgfusepath{fill,stroke}
\pgfpathmoveto{\pgfpoint{163.110062pt}{26.399979pt}}
\pgflineto{\pgfpoint{163.110062pt}{28.794250pt}}
\pgflineto{\pgfpoint{162.114075pt}{26.399979pt}}
\pgfpathclose
\pgfusepath{fill,stroke}
\pgfpathmoveto{\pgfpoint{164.106033pt}{26.399979pt}}
\pgflineto{\pgfpoint{163.110062pt}{28.794250pt}}
\pgflineto{\pgfpoint{163.110062pt}{26.399979pt}}
\pgfpathclose
\pgfusepath{fill,stroke}
\pgfpathmoveto{\pgfpoint{156.138168pt}{26.399979pt}}
\pgflineto{\pgfpoint{157.134155pt}{26.399979pt}}
\pgflineto{\pgfpoint{157.134155pt}{26.492256pt}}
\pgfpathclose
\pgfusepath{fill,stroke}
\pgfpathmoveto{\pgfpoint{158.130127pt}{26.480469pt}}
\pgflineto{\pgfpoint{157.134155pt}{26.492256pt}}
\pgflineto{\pgfpoint{157.134155pt}{26.399979pt}}
\pgfpathclose
\pgfusepath{fill,stroke}
\pgfpathmoveto{\pgfpoint{158.130127pt}{26.399979pt}}
\pgflineto{\pgfpoint{158.130127pt}{26.480469pt}}
\pgflineto{\pgfpoint{157.134155pt}{26.399979pt}}
\pgfpathclose
\pgfusepath{fill,stroke}
\pgfpathmoveto{\pgfpoint{159.126114pt}{26.399979pt}}
\pgflineto{\pgfpoint{158.130127pt}{26.480469pt}}
\pgflineto{\pgfpoint{158.130127pt}{26.399979pt}}
\pgfpathclose
\pgfusepath{fill,stroke}
\pgfpathmoveto{\pgfpoint{154.146194pt}{26.399979pt}}
\pgflineto{\pgfpoint{155.142181pt}{26.399979pt}}
\pgflineto{\pgfpoint{155.142181pt}{28.098244pt}}
\pgfpathclose
\pgfusepath{fill,stroke}
\pgfpathmoveto{\pgfpoint{156.138168pt}{26.399979pt}}
\pgflineto{\pgfpoint{155.142181pt}{28.098244pt}}
\pgflineto{\pgfpoint{155.142181pt}{26.399979pt}}
\pgfpathclose
\pgfusepath{fill,stroke}
\pgfpathmoveto{\pgfpoint{150.162262pt}{26.399979pt}}
\pgflineto{\pgfpoint{151.158249pt}{26.399979pt}}
\pgflineto{\pgfpoint{151.158249pt}{26.439781pt}}
\pgfpathclose
\pgfusepath{fill,stroke}
\pgfpathmoveto{\pgfpoint{152.154221pt}{26.910515pt}}
\pgflineto{\pgfpoint{151.158249pt}{26.439781pt}}
\pgflineto{\pgfpoint{151.158249pt}{26.399979pt}}
\pgfpathclose
\pgfusepath{fill,stroke}
\pgfpathmoveto{\pgfpoint{152.154221pt}{26.399979pt}}
\pgflineto{\pgfpoint{152.154221pt}{26.910515pt}}
\pgflineto{\pgfpoint{151.158249pt}{26.399979pt}}
\pgfpathclose
\pgfusepath{fill,stroke}
\pgfpathmoveto{\pgfpoint{153.150208pt}{26.399979pt}}
\pgflineto{\pgfpoint{152.154221pt}{26.910515pt}}
\pgflineto{\pgfpoint{152.154221pt}{26.399979pt}}
\pgfpathclose
\pgfusepath{fill,stroke}
\pgfpathmoveto{\pgfpoint{145.182343pt}{26.399979pt}}
\pgflineto{\pgfpoint{146.178314pt}{26.399979pt}}
\pgflineto{\pgfpoint{146.178314pt}{26.535568pt}}
\pgfpathclose
\pgfusepath{fill,stroke}
\pgfpathmoveto{\pgfpoint{147.174316pt}{26.399979pt}}
\pgflineto{\pgfpoint{146.178314pt}{26.535568pt}}
\pgflineto{\pgfpoint{146.178314pt}{26.399979pt}}
\pgfpathclose
\pgfusepath{fill,stroke}
\pgfpathmoveto{\pgfpoint{142.194382pt}{26.399979pt}}
\pgflineto{\pgfpoint{143.190369pt}{26.399979pt}}
\pgflineto{\pgfpoint{143.190369pt}{31.207901pt}}
\pgfpathclose
\pgfusepath{fill,stroke}
\pgfpathmoveto{\pgfpoint{144.186356pt}{26.399979pt}}
\pgflineto{\pgfpoint{143.190369pt}{31.207901pt}}
\pgflineto{\pgfpoint{143.190369pt}{26.399979pt}}
\pgfpathclose
\pgfusepath{fill,stroke}
\pgfpathmoveto{\pgfpoint{138.210449pt}{26.399979pt}}
\pgflineto{\pgfpoint{139.206436pt}{26.399979pt}}
\pgflineto{\pgfpoint{139.206436pt}{27.485779pt}}
\pgfpathclose
\pgfusepath{fill,stroke}
\pgfpathmoveto{\pgfpoint{140.202423pt}{27.782883pt}}
\pgflineto{\pgfpoint{139.206436pt}{27.485779pt}}
\pgflineto{\pgfpoint{139.206436pt}{26.399979pt}}
\pgfpathclose
\pgfusepath{fill,stroke}
\pgfpathmoveto{\pgfpoint{140.202423pt}{26.399979pt}}
\pgflineto{\pgfpoint{140.202423pt}{27.782883pt}}
\pgflineto{\pgfpoint{139.206436pt}{26.399979pt}}
\pgfpathclose
\pgfusepath{fill,stroke}
\pgfpathmoveto{\pgfpoint{141.198410pt}{26.399979pt}}
\pgflineto{\pgfpoint{140.202423pt}{27.782883pt}}
\pgflineto{\pgfpoint{140.202423pt}{26.399979pt}}
\pgfpathclose
\pgfusepath{fill,stroke}
\pgfpathmoveto{\pgfpoint{125.262665pt}{26.399979pt}}
\pgflineto{\pgfpoint{126.258652pt}{26.399979pt}}
\pgflineto{\pgfpoint{126.258652pt}{28.660576pt}}
\pgfpathclose
\pgfusepath{fill,stroke}
\pgfpathmoveto{\pgfpoint{127.254631pt}{29.838959pt}}
\pgflineto{\pgfpoint{126.258652pt}{28.660576pt}}
\pgflineto{\pgfpoint{126.258652pt}{26.399979pt}}
\pgfpathclose
\pgfusepath{fill,stroke}
\pgfpathmoveto{\pgfpoint{127.254631pt}{26.399979pt}}
\pgflineto{\pgfpoint{127.254631pt}{29.838959pt}}
\pgflineto{\pgfpoint{126.258652pt}{26.399979pt}}
\pgfpathclose
\pgfusepath{fill,stroke}
\pgfpathmoveto{\pgfpoint{128.250610pt}{26.614319pt}}
\pgflineto{\pgfpoint{127.254631pt}{29.838959pt}}
\pgflineto{\pgfpoint{127.254631pt}{26.399979pt}}
\pgfpathclose
\pgfusepath{fill,stroke}
\pgfpathmoveto{\pgfpoint{128.250610pt}{26.399979pt}}
\pgflineto{\pgfpoint{128.250610pt}{26.614319pt}}
\pgflineto{\pgfpoint{127.254631pt}{26.399979pt}}
\pgfpathclose
\pgfusepath{fill,stroke}
\pgfpathmoveto{\pgfpoint{129.246597pt}{26.451225pt}}
\pgflineto{\pgfpoint{128.250610pt}{26.614319pt}}
\pgflineto{\pgfpoint{128.250610pt}{26.399979pt}}
\pgfpathclose
\pgfusepath{fill,stroke}
\pgfpathmoveto{\pgfpoint{129.246597pt}{26.399979pt}}
\pgflineto{\pgfpoint{129.246597pt}{26.451225pt}}
\pgflineto{\pgfpoint{128.250610pt}{26.399979pt}}
\pgfpathclose
\pgfusepath{fill,stroke}
\pgfpathmoveto{\pgfpoint{130.242584pt}{26.430634pt}}
\pgflineto{\pgfpoint{129.246597pt}{26.451225pt}}
\pgflineto{\pgfpoint{129.246597pt}{26.399979pt}}
\pgfpathclose
\pgfusepath{fill,stroke}
\pgfpathmoveto{\pgfpoint{130.242584pt}{26.399979pt}}
\pgflineto{\pgfpoint{130.242584pt}{26.430634pt}}
\pgflineto{\pgfpoint{129.246597pt}{26.399979pt}}
\pgfpathclose
\pgfusepath{fill,stroke}
\pgfpathmoveto{\pgfpoint{131.238571pt}{27.958008pt}}
\pgflineto{\pgfpoint{130.242584pt}{26.430634pt}}
\pgflineto{\pgfpoint{130.242584pt}{26.399979pt}}
\pgfpathclose
\pgfusepath{fill,stroke}
\pgfpathmoveto{\pgfpoint{131.238571pt}{26.399979pt}}
\pgflineto{\pgfpoint{131.238571pt}{27.958008pt}}
\pgflineto{\pgfpoint{130.242584pt}{26.399979pt}}
\pgfpathclose
\pgfusepath{fill,stroke}
\pgfpathmoveto{\pgfpoint{132.234558pt}{27.682930pt}}
\pgflineto{\pgfpoint{131.238571pt}{27.958008pt}}
\pgflineto{\pgfpoint{131.238571pt}{26.399979pt}}
\pgfpathclose
\pgfusepath{fill,stroke}
\pgfpathmoveto{\pgfpoint{132.234558pt}{26.399979pt}}
\pgflineto{\pgfpoint{132.234558pt}{27.682930pt}}
\pgflineto{\pgfpoint{131.238571pt}{26.399979pt}}
\pgfpathclose
\pgfusepath{fill,stroke}
\pgfpathmoveto{\pgfpoint{133.230530pt}{26.399979pt}}
\pgflineto{\pgfpoint{132.234558pt}{27.682930pt}}
\pgflineto{\pgfpoint{132.234558pt}{26.399979pt}}
\pgfpathclose
\pgfusepath{fill,stroke}
\pgfpathmoveto{\pgfpoint{122.274712pt}{26.399979pt}}
\pgflineto{\pgfpoint{123.270691pt}{26.399979pt}}
\pgflineto{\pgfpoint{123.270691pt}{26.975601pt}}
\pgfpathclose
\pgfusepath{fill,stroke}
\pgfpathmoveto{\pgfpoint{124.266678pt}{26.399979pt}}
\pgflineto{\pgfpoint{123.270691pt}{26.975601pt}}
\pgflineto{\pgfpoint{123.270691pt}{26.399979pt}}
\pgfpathclose
\pgfusepath{fill,stroke}
\pgfpathmoveto{\pgfpoint{120.282745pt}{26.399979pt}}
\pgflineto{\pgfpoint{121.278725pt}{26.399979pt}}
\pgflineto{\pgfpoint{121.278725pt}{27.190407pt}}
\pgfpathclose
\pgfusepath{fill,stroke}
\pgfpathmoveto{\pgfpoint{122.274712pt}{26.399979pt}}
\pgflineto{\pgfpoint{121.278725pt}{27.190407pt}}
\pgflineto{\pgfpoint{121.278725pt}{26.399979pt}}
\pgfpathclose
\pgfusepath{fill,stroke}
\pgfpathmoveto{\pgfpoint{115.302826pt}{26.399979pt}}
\pgflineto{\pgfpoint{116.298813pt}{26.399979pt}}
\pgflineto{\pgfpoint{116.298813pt}{29.134193pt}}
\pgfpathclose
\pgfusepath{fill,stroke}
\pgfpathmoveto{\pgfpoint{117.294792pt}{26.815964pt}}
\pgflineto{\pgfpoint{116.298813pt}{29.134193pt}}
\pgflineto{\pgfpoint{116.298813pt}{26.399979pt}}
\pgfpathclose
\pgfusepath{fill,stroke}
\pgfpathmoveto{\pgfpoint{117.294792pt}{26.399979pt}}
\pgflineto{\pgfpoint{117.294792pt}{26.815964pt}}
\pgflineto{\pgfpoint{116.298813pt}{26.399979pt}}
\pgfpathclose
\pgfusepath{fill,stroke}
\pgfpathmoveto{\pgfpoint{117.967636pt}{91.264793pt}}
\pgflineto{\pgfpoint{117.294792pt}{26.815964pt}}
\pgflineto{\pgfpoint{117.294792pt}{26.399979pt}}
\pgfpathclose
\pgfusepath{fill,stroke}
\pgfpathmoveto{\pgfpoint{117.967636pt}{91.264793pt}}
\pgflineto{\pgfpoint{117.966232pt}{91.264786pt}}
\pgflineto{\pgfpoint{117.294792pt}{26.815964pt}}
\pgfpathclose
\pgfusepath{fill,stroke}
\pgfpathmoveto{\pgfpoint{118.290779pt}{26.399979pt}}
\pgflineto{\pgfpoint{117.967636pt}{91.264793pt}}
\pgflineto{\pgfpoint{117.294792pt}{26.399979pt}}
\pgfpathclose
\pgfusepath{fill,stroke}
\pgfpathmoveto{\pgfpoint{118.290779pt}{26.399979pt}}
\pgflineto{\pgfpoint{118.290779pt}{91.264793pt}}
\pgflineto{\pgfpoint{117.967636pt}{91.264793pt}}
\pgfpathclose
\pgfusepath{fill,stroke}
\pgfpathmoveto{\pgfpoint{119.286758pt}{26.399979pt}}
\pgflineto{\pgfpoint{118.290779pt}{91.264793pt}}
\pgflineto{\pgfpoint{118.290779pt}{26.399979pt}}
\pgfpathclose
\pgfusepath{fill,stroke}
\pgfpathmoveto{\pgfpoint{119.286758pt}{26.399979pt}}
\pgflineto{\pgfpoint{118.613922pt}{91.264793pt}}
\pgflineto{\pgfpoint{118.290779pt}{91.264793pt}}
\pgfpathclose
\pgfusepath{fill,stroke}
\pgfpathmoveto{\pgfpoint{111.318893pt}{26.399979pt}}
\pgflineto{\pgfpoint{112.314873pt}{26.399979pt}}
\pgflineto{\pgfpoint{112.314873pt}{37.362213pt}}
\pgfpathclose
\pgfusepath{fill,stroke}
\pgfpathmoveto{\pgfpoint{113.310852pt}{26.517036pt}}
\pgflineto{\pgfpoint{112.314873pt}{37.362213pt}}
\pgflineto{\pgfpoint{112.314873pt}{26.399979pt}}
\pgfpathclose
\pgfusepath{fill,stroke}
\pgfpathmoveto{\pgfpoint{113.310852pt}{26.399979pt}}
\pgflineto{\pgfpoint{113.310852pt}{26.517036pt}}
\pgflineto{\pgfpoint{112.314873pt}{26.399979pt}}
\pgfpathclose
\pgfusepath{fill,stroke}
\pgfpathmoveto{\pgfpoint{114.306839pt}{26.399979pt}}
\pgflineto{\pgfpoint{113.310852pt}{26.517036pt}}
\pgflineto{\pgfpoint{113.310852pt}{26.399979pt}}
\pgfpathclose
\pgfusepath{fill,stroke}
\pgfpathmoveto{\pgfpoint{107.334953pt}{26.399979pt}}
\pgflineto{\pgfpoint{108.330933pt}{26.399979pt}}
\pgflineto{\pgfpoint{108.330933pt}{27.150108pt}}
\pgfpathclose
\pgfusepath{fill,stroke}
\pgfpathmoveto{\pgfpoint{109.326920pt}{28.632256pt}}
\pgflineto{\pgfpoint{108.330933pt}{27.150108pt}}
\pgflineto{\pgfpoint{108.330933pt}{26.399979pt}}
\pgfpathclose
\pgfusepath{fill,stroke}
\pgfpathmoveto{\pgfpoint{109.326920pt}{26.399979pt}}
\pgflineto{\pgfpoint{109.326920pt}{28.632256pt}}
\pgflineto{\pgfpoint{108.330933pt}{26.399979pt}}
\pgfpathclose
\pgfusepath{fill,stroke}
\pgfpathmoveto{\pgfpoint{110.322906pt}{26.399979pt}}
\pgflineto{\pgfpoint{109.326920pt}{28.632256pt}}
\pgflineto{\pgfpoint{109.326920pt}{26.399979pt}}
\pgfpathclose
\pgfusepath{fill,stroke}
\pgfpathmoveto{\pgfpoint{103.351013pt}{26.399979pt}}
\pgflineto{\pgfpoint{104.347000pt}{26.399979pt}}
\pgflineto{\pgfpoint{104.347000pt}{35.689026pt}}
\pgfpathclose
\pgfusepath{fill,stroke}
\pgfpathmoveto{\pgfpoint{105.342987pt}{26.399979pt}}
\pgflineto{\pgfpoint{104.347000pt}{35.689026pt}}
\pgflineto{\pgfpoint{104.347000pt}{26.399979pt}}
\pgfpathclose
\pgfusepath{fill,stroke}
\pgfpathmoveto{\pgfpoint{101.359047pt}{26.399979pt}}
\pgflineto{\pgfpoint{102.355034pt}{26.399979pt}}
\pgflineto{\pgfpoint{102.355034pt}{26.959930pt}}
\pgfpathclose
\pgfusepath{fill,stroke}
\pgfpathmoveto{\pgfpoint{103.351013pt}{26.399979pt}}
\pgflineto{\pgfpoint{102.355034pt}{26.959930pt}}
\pgflineto{\pgfpoint{102.355034pt}{26.399979pt}}
\pgfpathclose
\pgfusepath{fill,stroke}
\pgfpathmoveto{\pgfpoint{97.375107pt}{26.399979pt}}
\pgflineto{\pgfpoint{98.371094pt}{26.399979pt}}
\pgflineto{\pgfpoint{98.371094pt}{47.782463pt}}
\pgfpathclose
\pgfusepath{fill,stroke}
\pgfpathmoveto{\pgfpoint{99.367081pt}{54.046951pt}}
\pgflineto{\pgfpoint{98.371094pt}{47.782463pt}}
\pgflineto{\pgfpoint{98.371094pt}{26.399979pt}}
\pgfpathclose
\pgfusepath{fill,stroke}
\pgfpathmoveto{\pgfpoint{99.367081pt}{26.399979pt}}
\pgflineto{\pgfpoint{99.367081pt}{54.046951pt}}
\pgflineto{\pgfpoint{98.371094pt}{26.399979pt}}
\pgfpathclose
\pgfusepath{fill,stroke}
\pgfpathmoveto{\pgfpoint{100.363068pt}{32.883598pt}}
\pgflineto{\pgfpoint{99.367081pt}{54.046951pt}}
\pgflineto{\pgfpoint{99.367081pt}{26.399979pt}}
\pgfpathclose
\pgfusepath{fill,stroke}
\pgfpathmoveto{\pgfpoint{100.363068pt}{26.399979pt}}
\pgflineto{\pgfpoint{100.363068pt}{32.883598pt}}
\pgflineto{\pgfpoint{99.367081pt}{26.399979pt}}
\pgfpathclose
\pgfusepath{fill,stroke}
\pgfpathmoveto{\pgfpoint{101.359047pt}{26.399979pt}}
\pgflineto{\pgfpoint{100.363068pt}{32.883598pt}}
\pgflineto{\pgfpoint{100.363068pt}{26.399979pt}}
\pgfpathclose
\pgfusepath{fill,stroke}
\pgfpathmoveto{\pgfpoint{91.399208pt}{26.399979pt}}
\pgflineto{\pgfpoint{92.395187pt}{26.399979pt}}
\pgflineto{\pgfpoint{92.395187pt}{26.889816pt}}
\pgfpathclose
\pgfusepath{fill,stroke}
\pgfpathmoveto{\pgfpoint{93.391174pt}{27.006203pt}}
\pgflineto{\pgfpoint{92.395187pt}{26.889816pt}}
\pgflineto{\pgfpoint{92.395187pt}{26.399979pt}}
\pgfpathclose
\pgfusepath{fill,stroke}
\pgfpathmoveto{\pgfpoint{93.391174pt}{26.399979pt}}
\pgflineto{\pgfpoint{93.391174pt}{27.006203pt}}
\pgflineto{\pgfpoint{92.395187pt}{26.399979pt}}
\pgfpathclose
\pgfusepath{fill,stroke}
\pgfpathmoveto{\pgfpoint{94.387161pt}{26.399979pt}}
\pgflineto{\pgfpoint{93.391174pt}{27.006203pt}}
\pgflineto{\pgfpoint{93.391174pt}{26.399979pt}}
\pgfpathclose
\pgfusepath{fill,stroke}
\pgfpathmoveto{\pgfpoint{89.407242pt}{26.399979pt}}
\pgflineto{\pgfpoint{90.403221pt}{26.399979pt}}
\pgflineto{\pgfpoint{90.403221pt}{26.454811pt}}
\pgfpathclose
\pgfusepath{fill,stroke}
\pgfpathmoveto{\pgfpoint{91.399208pt}{26.399979pt}}
\pgflineto{\pgfpoint{90.403221pt}{26.454811pt}}
\pgflineto{\pgfpoint{90.403221pt}{26.399979pt}}
\pgfpathclose
\pgfusepath{fill,stroke}
\pgfpathmoveto{\pgfpoint{87.415276pt}{26.399979pt}}
\pgflineto{\pgfpoint{88.411255pt}{26.399979pt}}
\pgflineto{\pgfpoint{88.411255pt}{31.944191pt}}
\pgfpathclose
\pgfusepath{fill,stroke}
\pgfpathmoveto{\pgfpoint{89.407242pt}{26.399979pt}}
\pgflineto{\pgfpoint{88.411255pt}{31.944191pt}}
\pgflineto{\pgfpoint{88.411255pt}{26.399979pt}}
\pgfpathclose
\pgfusepath{fill,stroke}
\pgfpathmoveto{\pgfpoint{85.423309pt}{26.399979pt}}
\pgflineto{\pgfpoint{86.419289pt}{26.399979pt}}
\pgflineto{\pgfpoint{86.419289pt}{26.871178pt}}
\pgfpathclose
\pgfusepath{fill,stroke}
\pgfpathmoveto{\pgfpoint{87.415276pt}{26.399979pt}}
\pgflineto{\pgfpoint{86.419289pt}{26.871178pt}}
\pgflineto{\pgfpoint{86.419289pt}{26.399979pt}}
\pgfpathclose
\pgfusepath{fill,stroke}
\pgfpathmoveto{\pgfpoint{79.447403pt}{26.399979pt}}
\pgflineto{\pgfpoint{80.443390pt}{26.399979pt}}
\pgflineto{\pgfpoint{80.443390pt}{28.913742pt}}
\pgfpathclose
\pgfusepath{fill,stroke}
\pgfpathmoveto{\pgfpoint{81.439369pt}{28.404305pt}}
\pgflineto{\pgfpoint{80.443390pt}{28.913742pt}}
\pgflineto{\pgfpoint{80.443390pt}{26.399979pt}}
\pgfpathclose
\pgfusepath{fill,stroke}
\pgfpathmoveto{\pgfpoint{81.439369pt}{26.399979pt}}
\pgflineto{\pgfpoint{81.439369pt}{28.404305pt}}
\pgflineto{\pgfpoint{80.443390pt}{26.399979pt}}
\pgfpathclose
\pgfusepath{fill,stroke}
\pgfpathmoveto{\pgfpoint{82.435356pt}{28.310295pt}}
\pgflineto{\pgfpoint{81.439369pt}{28.404305pt}}
\pgflineto{\pgfpoint{81.439369pt}{26.399979pt}}
\pgfpathclose
\pgfusepath{fill,stroke}
\pgfpathmoveto{\pgfpoint{82.435356pt}{26.399979pt}}
\pgflineto{\pgfpoint{82.435356pt}{28.310295pt}}
\pgflineto{\pgfpoint{81.439369pt}{26.399979pt}}
\pgfpathclose
\pgfusepath{fill,stroke}
\pgfpathmoveto{\pgfpoint{83.431335pt}{26.399979pt}}
\pgflineto{\pgfpoint{82.435356pt}{28.310295pt}}
\pgflineto{\pgfpoint{82.435356pt}{26.399979pt}}
\pgfpathclose
\pgfusepath{fill,stroke}
\pgfpathmoveto{\pgfpoint{73.471497pt}{26.399979pt}}
\pgflineto{\pgfpoint{74.467484pt}{26.399979pt}}
\pgflineto{\pgfpoint{74.467484pt}{26.868034pt}}
\pgfpathclose
\pgfusepath{fill,stroke}
\pgfpathmoveto{\pgfpoint{75.463470pt}{26.470932pt}}
\pgflineto{\pgfpoint{74.467484pt}{26.868034pt}}
\pgflineto{\pgfpoint{74.467484pt}{26.399979pt}}
\pgfpathclose
\pgfusepath{fill,stroke}
\pgfpathmoveto{\pgfpoint{75.463470pt}{26.399979pt}}
\pgflineto{\pgfpoint{75.463470pt}{26.470932pt}}
\pgflineto{\pgfpoint{74.467484pt}{26.399979pt}}
\pgfpathclose
\pgfusepath{fill,stroke}
\pgfpathmoveto{\pgfpoint{76.459442pt}{26.399979pt}}
\pgflineto{\pgfpoint{75.463470pt}{26.470932pt}}
\pgflineto{\pgfpoint{75.463470pt}{26.399979pt}}
\pgfpathclose
\pgfusepath{fill,stroke}
\pgfpathmoveto{\pgfpoint{70.483551pt}{26.399979pt}}
\pgflineto{\pgfpoint{71.479530pt}{26.399979pt}}
\pgflineto{\pgfpoint{71.479530pt}{52.145065pt}}
\pgfpathclose
\pgfusepath{fill,stroke}
\pgfpathmoveto{\pgfpoint{72.475510pt}{31.143661pt}}
\pgflineto{\pgfpoint{71.479530pt}{52.145065pt}}
\pgflineto{\pgfpoint{71.479530pt}{26.399979pt}}
\pgfpathclose
\pgfusepath{fill,stroke}
\pgfpathmoveto{\pgfpoint{72.475510pt}{26.399979pt}}
\pgflineto{\pgfpoint{72.475510pt}{31.143661pt}}
\pgflineto{\pgfpoint{71.479530pt}{26.399979pt}}
\pgfpathclose
\pgfusepath{fill,stroke}
\pgfpathmoveto{\pgfpoint{73.471497pt}{26.399979pt}}
\pgflineto{\pgfpoint{72.475510pt}{31.143661pt}}
\pgflineto{\pgfpoint{72.475510pt}{26.399979pt}}
\pgfpathclose
\pgfusepath{fill,stroke}
\pgfpathmoveto{\pgfpoint{67.495590pt}{26.399979pt}}
\pgflineto{\pgfpoint{68.491577pt}{26.399979pt}}
\pgflineto{\pgfpoint{68.491577pt}{26.687355pt}}
\pgfpathclose
\pgfusepath{fill,stroke}
\pgfpathmoveto{\pgfpoint{69.487564pt}{27.948082pt}}
\pgflineto{\pgfpoint{68.491577pt}{26.687355pt}}
\pgflineto{\pgfpoint{68.491577pt}{26.399979pt}}
\pgfpathclose
\pgfusepath{fill,stroke}
\pgfpathmoveto{\pgfpoint{69.487564pt}{26.399979pt}}
\pgflineto{\pgfpoint{69.487564pt}{27.948082pt}}
\pgflineto{\pgfpoint{68.491577pt}{26.399979pt}}
\pgfpathclose
\pgfusepath{fill,stroke}
\pgfpathmoveto{\pgfpoint{70.483551pt}{26.399979pt}}
\pgflineto{\pgfpoint{69.487564pt}{27.948082pt}}
\pgflineto{\pgfpoint{69.487564pt}{26.399979pt}}
\pgfpathclose
\pgfusepath{fill,stroke}
\pgfpathmoveto{\pgfpoint{63.511658pt}{26.399979pt}}
\pgflineto{\pgfpoint{64.507637pt}{26.399979pt}}
\pgflineto{\pgfpoint{64.507637pt}{26.602051pt}}
\pgfpathclose
\pgfusepath{fill,stroke}
\pgfpathmoveto{\pgfpoint{65.503624pt}{27.212898pt}}
\pgflineto{\pgfpoint{64.507637pt}{26.602051pt}}
\pgflineto{\pgfpoint{64.507637pt}{26.399979pt}}
\pgfpathclose
\pgfusepath{fill,stroke}
\pgfpathmoveto{\pgfpoint{65.503624pt}{26.399979pt}}
\pgflineto{\pgfpoint{65.503624pt}{27.212898pt}}
\pgflineto{\pgfpoint{64.507637pt}{26.399979pt}}
\pgfpathclose
\pgfusepath{fill,stroke}
\pgfpathmoveto{\pgfpoint{66.499619pt}{29.321190pt}}
\pgflineto{\pgfpoint{65.503624pt}{27.212898pt}}
\pgflineto{\pgfpoint{65.503624pt}{26.399979pt}}
\pgfpathclose
\pgfusepath{fill,stroke}
\pgfpathmoveto{\pgfpoint{66.499619pt}{26.399979pt}}
\pgflineto{\pgfpoint{66.499619pt}{29.321190pt}}
\pgflineto{\pgfpoint{65.503624pt}{26.399979pt}}
\pgfpathclose
\pgfusepath{fill,stroke}
\pgfpathmoveto{\pgfpoint{67.495590pt}{26.399979pt}}
\pgflineto{\pgfpoint{66.499619pt}{29.321190pt}}
\pgflineto{\pgfpoint{66.499619pt}{26.399979pt}}
\pgfpathclose
\pgfusepath{fill,stroke}
\pgfpathmoveto{\pgfpoint{61.519691pt}{26.399979pt}}
\pgflineto{\pgfpoint{62.515678pt}{26.399979pt}}
\pgflineto{\pgfpoint{62.515678pt}{26.488380pt}}
\pgfpathclose
\pgfusepath{fill,stroke}
\pgfpathmoveto{\pgfpoint{63.511658pt}{26.399979pt}}
\pgflineto{\pgfpoint{62.515678pt}{26.488380pt}}
\pgflineto{\pgfpoint{62.515678pt}{26.399979pt}}
\pgfpathclose
\pgfusepath{fill,stroke}
\pgfpathmoveto{\pgfpoint{57.535751pt}{26.399979pt}}
\pgflineto{\pgfpoint{58.531738pt}{26.399979pt}}
\pgflineto{\pgfpoint{58.531738pt}{27.217796pt}}
\pgfpathclose
\pgfusepath{fill,stroke}
\pgfpathmoveto{\pgfpoint{59.527725pt}{26.723701pt}}
\pgflineto{\pgfpoint{58.531738pt}{27.217796pt}}
\pgflineto{\pgfpoint{58.531738pt}{26.399979pt}}
\pgfpathclose
\pgfusepath{fill,stroke}
\pgfpathmoveto{\pgfpoint{59.527725pt}{26.399979pt}}
\pgflineto{\pgfpoint{59.527725pt}{26.723701pt}}
\pgflineto{\pgfpoint{58.531738pt}{26.399979pt}}
\pgfpathclose
\pgfusepath{fill,stroke}
\pgfpathmoveto{\pgfpoint{60.523712pt}{26.399979pt}}
\pgflineto{\pgfpoint{59.527725pt}{26.723701pt}}
\pgflineto{\pgfpoint{59.527725pt}{26.399979pt}}
\pgfpathclose
\pgfusepath{fill,stroke}
\pgfpathmoveto{\pgfpoint{54.547806pt}{26.399979pt}}
\pgflineto{\pgfpoint{55.543785pt}{26.399979pt}}
\pgflineto{\pgfpoint{55.543785pt}{26.577354pt}}
\pgfpathclose
\pgfusepath{fill,stroke}
\pgfpathmoveto{\pgfpoint{56.539772pt}{65.803238pt}}
\pgflineto{\pgfpoint{55.543785pt}{26.577354pt}}
\pgflineto{\pgfpoint{55.543785pt}{26.399979pt}}
\pgfpathclose
\pgfusepath{fill,stroke}
\pgfpathmoveto{\pgfpoint{56.539772pt}{26.399979pt}}
\pgflineto{\pgfpoint{56.539772pt}{65.803238pt}}
\pgflineto{\pgfpoint{55.543785pt}{26.399979pt}}
\pgfpathclose
\pgfusepath{fill,stroke}
\pgfpathmoveto{\pgfpoint{57.535751pt}{26.399979pt}}
\pgflineto{\pgfpoint{56.539772pt}{65.803238pt}}
\pgflineto{\pgfpoint{56.539772pt}{26.399979pt}}
\pgfpathclose
\pgfusepath{fill,stroke}
\pgfpathmoveto{\pgfpoint{52.555840pt}{26.399979pt}}
\pgflineto{\pgfpoint{53.551819pt}{26.399979pt}}
\pgflineto{\pgfpoint{53.551819pt}{32.692734pt}}
\pgfpathclose
\pgfusepath{fill,stroke}
\pgfpathmoveto{\pgfpoint{54.547806pt}{26.399979pt}}
\pgflineto{\pgfpoint{53.551819pt}{32.692734pt}}
\pgflineto{\pgfpoint{53.551819pt}{26.399979pt}}
\pgfpathclose
\pgfusepath{fill,stroke}
\pgfpathmoveto{\pgfpoint{46.579933pt}{26.399979pt}}
\pgflineto{\pgfpoint{47.575912pt}{26.399979pt}}
\pgflineto{\pgfpoint{47.575912pt}{27.559311pt}}
\pgfpathclose
\pgfusepath{fill,stroke}
\pgfpathmoveto{\pgfpoint{48.571899pt}{26.399979pt}}
\pgflineto{\pgfpoint{47.575912pt}{27.559311pt}}
\pgflineto{\pgfpoint{47.575912pt}{26.399979pt}}
\pgfpathclose
\pgfusepath{fill,stroke}
\pgfpathmoveto{\pgfpoint{42.595993pt}{26.399979pt}}
\pgflineto{\pgfpoint{43.591980pt}{26.399979pt}}
\pgflineto{\pgfpoint{43.591980pt}{33.198105pt}}
\pgfpathclose
\pgfusepath{fill,stroke}
\pgfpathmoveto{\pgfpoint{44.587967pt}{26.399979pt}}
\pgflineto{\pgfpoint{43.591980pt}{33.198105pt}}
\pgflineto{\pgfpoint{43.591980pt}{26.399979pt}}
\pgfpathclose
\pgfusepath{fill,stroke}
\color[rgb]{0.000000,0.000000,0.000000}
\pgfsetlinewidth{0.500000pt}
\pgfsetdash{{16pt}{0pt}}{0pt}
\pgfpathmoveto{\pgfpoint{289.600037pt}{140.777435pt}}
\pgflineto{\pgfpoint{41.600006pt}{140.777435pt}}
\pgfusepath{stroke}
\pgfpathmoveto{\pgfpoint{289.600037pt}{205.577454pt}}
\pgflineto{\pgfpoint{41.600006pt}{205.577454pt}}
\pgfusepath{stroke}
\pgfpathmoveto{\pgfpoint{41.600006pt}{205.577454pt}}
\pgflineto{\pgfpoint{41.600006pt}{140.777435pt}}
\pgfusepath{stroke}
\pgfpathmoveto{\pgfpoint{289.600037pt}{205.577454pt}}
\pgflineto{\pgfpoint{289.600037pt}{140.777435pt}}
\pgfusepath{stroke}
\pgfpathmoveto{\pgfpoint{90.403221pt}{143.249802pt}}
\pgflineto{\pgfpoint{90.403221pt}{140.777435pt}}
\pgfusepath{stroke}
\pgfpathmoveto{\pgfpoint{90.403221pt}{203.105072pt}}
\pgflineto{\pgfpoint{90.403221pt}{205.577454pt}}
\pgfusepath{stroke}
\pgfpathmoveto{\pgfpoint{140.202423pt}{143.249802pt}}
\pgflineto{\pgfpoint{140.202423pt}{140.777435pt}}
\pgfusepath{stroke}
\pgfpathmoveto{\pgfpoint{140.202423pt}{203.105072pt}}
\pgflineto{\pgfpoint{140.202423pt}{205.577454pt}}
\pgfusepath{stroke}
\pgfpathmoveto{\pgfpoint{190.001617pt}{143.249802pt}}
\pgflineto{\pgfpoint{190.001617pt}{140.777435pt}}
\pgfusepath{stroke}
\pgfpathmoveto{\pgfpoint{190.001617pt}{203.105072pt}}
\pgflineto{\pgfpoint{190.001617pt}{205.577454pt}}
\pgfusepath{stroke}
\pgfpathmoveto{\pgfpoint{239.800812pt}{143.249802pt}}
\pgflineto{\pgfpoint{239.800812pt}{140.777435pt}}
\pgfusepath{stroke}
\pgfpathmoveto{\pgfpoint{239.800812pt}{203.105072pt}}
\pgflineto{\pgfpoint{239.800812pt}{205.577454pt}}
\pgfusepath{stroke}
\pgfpathmoveto{\pgfpoint{289.600037pt}{143.249802pt}}
\pgflineto{\pgfpoint{289.600037pt}{140.777435pt}}
\pgfusepath{stroke}
\pgfpathmoveto{\pgfpoint{289.600037pt}{203.105072pt}}
\pgflineto{\pgfpoint{289.600037pt}{205.577454pt}}
\pgfusepath{stroke}
{
\pgftransformshift{\pgfpoint{90.403229pt}{135.792816pt}}
\pgfnode{rectangle}{north}{\fontsize{10}{0}\selectfont\textcolor[rgb]{0,0,0}{{50}}}{}{\pgfusepath{discard}}}
{
\pgftransformshift{\pgfpoint{140.202423pt}{135.792816pt}}
\pgfnode{rectangle}{north}{\fontsize{10}{0}\selectfont\textcolor[rgb]{0,0,0}{{100}}}{}{\pgfusepath{discard}}}
{
\pgftransformshift{\pgfpoint{190.001617pt}{135.792816pt}}
\pgfnode{rectangle}{north}{\fontsize{10}{0}\selectfont\textcolor[rgb]{0,0,0}{{150}}}{}{\pgfusepath{discard}}}
{
\pgftransformshift{\pgfpoint{239.800812pt}{135.792816pt}}
\pgfnode{rectangle}{north}{\fontsize{10}{0}\selectfont\textcolor[rgb]{0,0,0}{{200}}}{}{\pgfusepath{discard}}}
{
\pgftransformshift{\pgfpoint{289.600037pt}{135.792816pt}}
\pgfnode{rectangle}{north}{\fontsize{10}{0}\selectfont\textcolor[rgb]{0,0,0}{{250}}}{}{\pgfusepath{discard}}}
\pgfpathmoveto{\pgfpoint{44.080009pt}{140.777435pt}}
\pgflineto{\pgfpoint{41.600006pt}{140.777435pt}}
\pgfusepath{stroke}
\pgfpathmoveto{\pgfpoint{287.119995pt}{140.777435pt}}
\pgflineto{\pgfpoint{289.600037pt}{140.777435pt}}
\pgfusepath{stroke}
\pgfpathmoveto{\pgfpoint{44.080009pt}{153.737442pt}}
\pgflineto{\pgfpoint{41.600006pt}{153.737442pt}}
\pgfusepath{stroke}
\pgfpathmoveto{\pgfpoint{287.119995pt}{153.737442pt}}
\pgflineto{\pgfpoint{289.600037pt}{153.737442pt}}
\pgfusepath{stroke}
\pgfpathmoveto{\pgfpoint{44.080009pt}{166.697433pt}}
\pgflineto{\pgfpoint{41.600006pt}{166.697433pt}}
\pgfusepath{stroke}
\pgfpathmoveto{\pgfpoint{287.119995pt}{166.697433pt}}
\pgflineto{\pgfpoint{289.600037pt}{166.697433pt}}
\pgfusepath{stroke}
\pgfpathmoveto{\pgfpoint{44.080009pt}{179.657440pt}}
\pgflineto{\pgfpoint{41.600006pt}{179.657440pt}}
\pgfusepath{stroke}
\pgfpathmoveto{\pgfpoint{287.119995pt}{179.657440pt}}
\pgflineto{\pgfpoint{289.600037pt}{179.657440pt}}
\pgfusepath{stroke}
\pgfpathmoveto{\pgfpoint{44.080009pt}{192.617432pt}}
\pgflineto{\pgfpoint{41.600006pt}{192.617432pt}}
\pgfusepath{stroke}
\pgfpathmoveto{\pgfpoint{287.119995pt}{192.617432pt}}
\pgflineto{\pgfpoint{289.600037pt}{192.617432pt}}
\pgfusepath{stroke}
\pgfpathmoveto{\pgfpoint{44.080009pt}{205.577454pt}}
\pgflineto{\pgfpoint{41.600006pt}{205.577454pt}}
\pgfusepath{stroke}
\pgfpathmoveto{\pgfpoint{287.119995pt}{205.577454pt}}
\pgflineto{\pgfpoint{289.600037pt}{205.577454pt}}
\pgfusepath{stroke}
{
\pgftransformshift{\pgfpoint{36.600006pt}{140.777435pt}}
\pgfnode{rectangle}{east}{\fontsize{10}{0}\selectfont\textcolor[rgb]{0,0,0}{{0}}}{}{\pgfusepath{discard}}}
{
\pgftransformshift{\pgfpoint{36.600006pt}{153.737442pt}}
\pgfnode{rectangle}{east}{\fontsize{10}{0}\selectfont\textcolor[rgb]{0,0,0}{{2e+06}}}{}{\pgfusepath{discard}}}
{
\pgftransformshift{\pgfpoint{36.600006pt}{166.697433pt}}
\pgfnode{rectangle}{east}{\fontsize{10}{0}\selectfont\textcolor[rgb]{0,0,0}{{4e+06}}}{}{\pgfusepath{discard}}}
{
\pgftransformshift{\pgfpoint{36.600006pt}{179.657440pt}}
\pgfnode{rectangle}{east}{\fontsize{10}{0}\selectfont\textcolor[rgb]{0,0,0}{{6e+06}}}{}{\pgfusepath{discard}}}
{
\pgftransformshift{\pgfpoint{36.600006pt}{192.617432pt}}
\pgfnode{rectangle}{east}{\fontsize{10}{0}\selectfont\textcolor[rgb]{0,0,0}{{8e+06}}}{}{\pgfusepath{discard}}}
{
\pgftransformshift{\pgfpoint{36.600006pt}{205.577454pt}}
\pgfnode{rectangle}{east}{\fontsize{10}{0}\selectfont\textcolor[rgb]{0,0,0}{{1e+07}}}{}{\pgfusepath{discard}}}
\pgfsetlinewidth{0.000100pt}
\pgfsetdash{}{0pt}
\pgfpathmoveto{\pgfpoint{43.591980pt}{140.777435pt}}
\pgflineto{\pgfpoint{44.587967pt}{140.777435pt}}
\pgfusepath{stroke}
\pgfpathmoveto{\pgfpoint{42.595993pt}{140.777435pt}}
\pgflineto{\pgfpoint{43.591980pt}{140.777435pt}}
\pgfusepath{stroke}
\pgfpathmoveto{\pgfpoint{43.591980pt}{186.324295pt}}
\pgflineto{\pgfpoint{42.595993pt}{140.777435pt}}
\pgfusepath{stroke}
\pgfpathmoveto{\pgfpoint{44.587967pt}{140.777435pt}}
\pgflineto{\pgfpoint{43.591980pt}{186.324295pt}}
\pgfusepath{stroke}
\pgfpathmoveto{\pgfpoint{45.583946pt}{140.777435pt}}
\pgflineto{\pgfpoint{46.579933pt}{140.777435pt}}
\pgfusepath{stroke}
\pgfpathmoveto{\pgfpoint{44.587967pt}{140.777435pt}}
\pgflineto{\pgfpoint{45.583946pt}{140.777435pt}}
\pgfusepath{stroke}
\pgfpathmoveto{\pgfpoint{45.583946pt}{148.946091pt}}
\pgflineto{\pgfpoint{44.587967pt}{140.777435pt}}
\pgfusepath{stroke}
\pgfpathmoveto{\pgfpoint{46.579933pt}{140.777435pt}}
\pgflineto{\pgfpoint{45.583946pt}{148.946091pt}}
\pgfusepath{stroke}
\pgfpathmoveto{\pgfpoint{49.567879pt}{140.777435pt}}
\pgflineto{\pgfpoint{50.563873pt}{140.777435pt}}
\pgfusepath{stroke}
\pgfpathmoveto{\pgfpoint{48.571899pt}{140.777435pt}}
\pgflineto{\pgfpoint{49.567879pt}{140.777435pt}}
\pgfusepath{stroke}
\pgfpathmoveto{\pgfpoint{47.575912pt}{140.777435pt}}
\pgflineto{\pgfpoint{48.571899pt}{140.777435pt}}
\pgfusepath{stroke}
\pgfpathmoveto{\pgfpoint{46.579933pt}{140.777435pt}}
\pgflineto{\pgfpoint{47.575912pt}{140.777435pt}}
\pgfusepath{stroke}
\pgfpathmoveto{\pgfpoint{47.575912pt}{141.646210pt}}
\pgflineto{\pgfpoint{46.579933pt}{140.777435pt}}
\pgfusepath{stroke}
\pgfpathmoveto{\pgfpoint{48.571899pt}{141.530609pt}}
\pgflineto{\pgfpoint{47.575912pt}{141.646210pt}}
\pgfusepath{stroke}
\pgfpathmoveto{\pgfpoint{49.567879pt}{143.458160pt}}
\pgflineto{\pgfpoint{48.571899pt}{141.530609pt}}
\pgfusepath{stroke}
\pgfpathmoveto{\pgfpoint{50.563873pt}{140.777435pt}}
\pgflineto{\pgfpoint{49.567879pt}{143.458160pt}}
\pgfusepath{stroke}
\pgfpathmoveto{\pgfpoint{66.499619pt}{140.777435pt}}
\pgflineto{\pgfpoint{67.495590pt}{140.777435pt}}
\pgfusepath{stroke}
\pgfpathmoveto{\pgfpoint{65.503624pt}{140.777435pt}}
\pgflineto{\pgfpoint{66.499619pt}{140.777435pt}}
\pgfusepath{stroke}
\pgfpathmoveto{\pgfpoint{64.507637pt}{140.777435pt}}
\pgflineto{\pgfpoint{65.503624pt}{140.777435pt}}
\pgfusepath{stroke}
\pgfpathmoveto{\pgfpoint{63.511658pt}{140.777435pt}}
\pgflineto{\pgfpoint{64.507637pt}{140.777435pt}}
\pgfusepath{stroke}
\pgfpathmoveto{\pgfpoint{62.515678pt}{140.777435pt}}
\pgflineto{\pgfpoint{63.511658pt}{140.777435pt}}
\pgfusepath{stroke}
\pgfpathmoveto{\pgfpoint{61.519691pt}{140.777435pt}}
\pgflineto{\pgfpoint{62.515678pt}{140.777435pt}}
\pgfusepath{stroke}
\pgfpathmoveto{\pgfpoint{60.523712pt}{140.777435pt}}
\pgflineto{\pgfpoint{61.519691pt}{140.777435pt}}
\pgfusepath{stroke}
\pgfpathmoveto{\pgfpoint{59.527725pt}{140.777435pt}}
\pgflineto{\pgfpoint{60.523712pt}{140.777435pt}}
\pgfusepath{stroke}
\pgfpathmoveto{\pgfpoint{58.531738pt}{140.777435pt}}
\pgflineto{\pgfpoint{59.527725pt}{140.777435pt}}
\pgfusepath{stroke}
\pgfpathmoveto{\pgfpoint{57.535751pt}{140.777435pt}}
\pgflineto{\pgfpoint{58.531738pt}{140.777435pt}}
\pgfusepath{stroke}
\pgfpathmoveto{\pgfpoint{56.539772pt}{140.777435pt}}
\pgflineto{\pgfpoint{57.535751pt}{140.777435pt}}
\pgfusepath{stroke}
\pgfpathmoveto{\pgfpoint{55.543785pt}{140.777435pt}}
\pgflineto{\pgfpoint{56.539772pt}{140.777435pt}}
\pgfusepath{stroke}
\pgfpathmoveto{\pgfpoint{54.547806pt}{140.777435pt}}
\pgflineto{\pgfpoint{55.543785pt}{140.777435pt}}
\pgfusepath{stroke}
\pgfpathmoveto{\pgfpoint{53.551819pt}{140.777435pt}}
\pgflineto{\pgfpoint{54.547806pt}{140.777435pt}}
\pgfusepath{stroke}
\pgfpathmoveto{\pgfpoint{52.555840pt}{140.777435pt}}
\pgflineto{\pgfpoint{53.551819pt}{140.777435pt}}
\pgfusepath{stroke}
\pgfpathmoveto{\pgfpoint{51.559845pt}{140.777435pt}}
\pgflineto{\pgfpoint{52.555840pt}{140.777435pt}}
\pgfusepath{stroke}
\pgfpathmoveto{\pgfpoint{50.563873pt}{140.777435pt}}
\pgflineto{\pgfpoint{51.559845pt}{140.777435pt}}
\pgfusepath{stroke}
\pgfpathmoveto{\pgfpoint{51.559845pt}{146.765289pt}}
\pgflineto{\pgfpoint{50.563873pt}{140.777435pt}}
\pgfusepath{stroke}
\pgfpathmoveto{\pgfpoint{52.555840pt}{142.530624pt}}
\pgflineto{\pgfpoint{51.559845pt}{146.765289pt}}
\pgfusepath{stroke}
\pgfpathmoveto{\pgfpoint{53.551819pt}{146.137848pt}}
\pgflineto{\pgfpoint{52.555840pt}{142.530624pt}}
\pgfusepath{stroke}
\pgfpathmoveto{\pgfpoint{54.547806pt}{142.199554pt}}
\pgflineto{\pgfpoint{53.551819pt}{146.137848pt}}
\pgfusepath{stroke}
\pgfpathmoveto{\pgfpoint{55.543785pt}{196.136078pt}}
\pgflineto{\pgfpoint{54.547806pt}{142.199554pt}}
\pgfusepath{stroke}
\pgfpathmoveto{\pgfpoint{56.539772pt}{168.281158pt}}
\pgflineto{\pgfpoint{55.543785pt}{196.136078pt}}
\pgfusepath{stroke}
\pgfpathmoveto{\pgfpoint{57.535751pt}{147.868973pt}}
\pgflineto{\pgfpoint{56.539772pt}{168.281158pt}}
\pgfusepath{stroke}
\pgfpathmoveto{\pgfpoint{58.531738pt}{141.160339pt}}
\pgflineto{\pgfpoint{57.535751pt}{147.868973pt}}
\pgfusepath{stroke}
\pgfpathmoveto{\pgfpoint{59.527725pt}{141.894562pt}}
\pgflineto{\pgfpoint{58.531738pt}{141.160339pt}}
\pgfusepath{stroke}
\pgfpathmoveto{\pgfpoint{60.523712pt}{142.042740pt}}
\pgflineto{\pgfpoint{59.527725pt}{141.894562pt}}
\pgfusepath{stroke}
\pgfpathmoveto{\pgfpoint{61.519691pt}{141.419281pt}}
\pgflineto{\pgfpoint{60.523712pt}{142.042740pt}}
\pgfusepath{stroke}
\pgfpathmoveto{\pgfpoint{62.515678pt}{202.358673pt}}
\pgflineto{\pgfpoint{61.519691pt}{141.419281pt}}
\pgfusepath{stroke}
\pgfpathmoveto{\pgfpoint{63.511658pt}{143.879868pt}}
\pgflineto{\pgfpoint{62.515678pt}{202.358673pt}}
\pgfusepath{stroke}
\pgfpathmoveto{\pgfpoint{64.507637pt}{144.233887pt}}
\pgflineto{\pgfpoint{63.511658pt}{143.879868pt}}
\pgfusepath{stroke}
\pgfpathmoveto{\pgfpoint{65.503624pt}{142.070892pt}}
\pgflineto{\pgfpoint{64.507637pt}{144.233887pt}}
\pgfusepath{stroke}
\pgfpathmoveto{\pgfpoint{66.499619pt}{141.913147pt}}
\pgflineto{\pgfpoint{65.503624pt}{142.070892pt}}
\pgfusepath{stroke}
\pgfpathmoveto{\pgfpoint{67.495590pt}{140.777435pt}}
\pgflineto{\pgfpoint{66.499619pt}{141.913147pt}}
\pgfusepath{stroke}
\pgfpathmoveto{\pgfpoint{72.475510pt}{140.777435pt}}
\pgflineto{\pgfpoint{73.471497pt}{140.777435pt}}
\pgfusepath{stroke}
\pgfpathmoveto{\pgfpoint{71.479530pt}{140.777435pt}}
\pgflineto{\pgfpoint{72.475510pt}{140.777435pt}}
\pgfusepath{stroke}
\pgfpathmoveto{\pgfpoint{70.483551pt}{140.777435pt}}
\pgflineto{\pgfpoint{71.479530pt}{140.777435pt}}
\pgfusepath{stroke}
\pgfpathmoveto{\pgfpoint{69.487564pt}{140.777435pt}}
\pgflineto{\pgfpoint{70.483551pt}{140.777435pt}}
\pgfusepath{stroke}
\pgfpathmoveto{\pgfpoint{68.491577pt}{140.777435pt}}
\pgflineto{\pgfpoint{69.487564pt}{140.777435pt}}
\pgfusepath{stroke}
\pgfpathmoveto{\pgfpoint{69.487564pt}{149.515793pt}}
\pgflineto{\pgfpoint{68.491577pt}{140.777435pt}}
\pgfusepath{stroke}
\pgfpathmoveto{\pgfpoint{70.483551pt}{141.832703pt}}
\pgflineto{\pgfpoint{69.487564pt}{149.515793pt}}
\pgfusepath{stroke}
\pgfpathmoveto{\pgfpoint{71.479530pt}{190.848999pt}}
\pgflineto{\pgfpoint{70.483551pt}{141.832703pt}}
\pgfusepath{stroke}
\pgfpathmoveto{\pgfpoint{72.475510pt}{148.664902pt}}
\pgflineto{\pgfpoint{71.479530pt}{190.848999pt}}
\pgfusepath{stroke}
\pgfpathmoveto{\pgfpoint{73.471497pt}{140.777435pt}}
\pgflineto{\pgfpoint{72.475510pt}{148.664902pt}}
\pgfusepath{stroke}
\pgfpathmoveto{\pgfpoint{77.455437pt}{140.777435pt}}
\pgflineto{\pgfpoint{78.451424pt}{140.777435pt}}
\pgfusepath{stroke}
\pgfpathmoveto{\pgfpoint{76.459442pt}{140.777435pt}}
\pgflineto{\pgfpoint{77.455437pt}{140.777435pt}}
\pgfusepath{stroke}
\pgfpathmoveto{\pgfpoint{75.463470pt}{140.777435pt}}
\pgflineto{\pgfpoint{76.459442pt}{140.777435pt}}
\pgfusepath{stroke}
\pgfpathmoveto{\pgfpoint{74.467484pt}{140.777435pt}}
\pgflineto{\pgfpoint{75.463470pt}{140.777435pt}}
\pgfusepath{stroke}
\pgfpathmoveto{\pgfpoint{73.471497pt}{140.777435pt}}
\pgflineto{\pgfpoint{74.467484pt}{140.777435pt}}
\pgfusepath{stroke}
\pgfpathmoveto{\pgfpoint{74.467484pt}{141.606003pt}}
\pgflineto{\pgfpoint{73.471497pt}{140.777435pt}}
\pgfusepath{stroke}
\pgfpathmoveto{\pgfpoint{75.463470pt}{145.536850pt}}
\pgflineto{\pgfpoint{74.467484pt}{141.606003pt}}
\pgfusepath{stroke}
\pgfpathmoveto{\pgfpoint{76.459442pt}{143.382019pt}}
\pgflineto{\pgfpoint{75.463470pt}{145.536850pt}}
\pgfusepath{stroke}
\pgfpathmoveto{\pgfpoint{77.455437pt}{142.594910pt}}
\pgflineto{\pgfpoint{76.459442pt}{143.382019pt}}
\pgfusepath{stroke}
\pgfpathmoveto{\pgfpoint{78.451424pt}{140.777435pt}}
\pgflineto{\pgfpoint{77.455437pt}{142.594910pt}}
\pgfusepath{stroke}
\pgfpathmoveto{\pgfpoint{85.423309pt}{140.777435pt}}
\pgflineto{\pgfpoint{86.419289pt}{140.777435pt}}
\pgfusepath{stroke}
\pgfpathmoveto{\pgfpoint{84.427322pt}{140.777435pt}}
\pgflineto{\pgfpoint{85.423309pt}{140.777435pt}}
\pgfusepath{stroke}
\pgfpathmoveto{\pgfpoint{83.431335pt}{140.777435pt}}
\pgflineto{\pgfpoint{84.427322pt}{140.777435pt}}
\pgfusepath{stroke}
\pgfpathmoveto{\pgfpoint{82.435356pt}{140.777435pt}}
\pgflineto{\pgfpoint{83.431335pt}{140.777435pt}}
\pgfusepath{stroke}
\pgfpathmoveto{\pgfpoint{81.439369pt}{140.777435pt}}
\pgflineto{\pgfpoint{82.435356pt}{140.777435pt}}
\pgfusepath{stroke}
\pgfpathmoveto{\pgfpoint{80.443390pt}{140.777435pt}}
\pgflineto{\pgfpoint{81.439369pt}{140.777435pt}}
\pgfusepath{stroke}
\pgfpathmoveto{\pgfpoint{79.447403pt}{140.777435pt}}
\pgflineto{\pgfpoint{80.443390pt}{140.777435pt}}
\pgfusepath{stroke}
\pgfpathmoveto{\pgfpoint{78.451424pt}{140.777435pt}}
\pgflineto{\pgfpoint{79.447403pt}{140.777435pt}}
\pgfusepath{stroke}
\pgfpathmoveto{\pgfpoint{79.447403pt}{164.721649pt}}
\pgflineto{\pgfpoint{78.451424pt}{140.777435pt}}
\pgfusepath{stroke}
\pgfpathmoveto{\pgfpoint{80.443390pt}{141.021835pt}}
\pgflineto{\pgfpoint{79.447403pt}{164.721649pt}}
\pgfusepath{stroke}
\pgfpathmoveto{\pgfpoint{81.439369pt}{141.797729pt}}
\pgflineto{\pgfpoint{80.443390pt}{141.021835pt}}
\pgfusepath{stroke}
\pgfpathmoveto{\pgfpoint{82.435356pt}{141.784378pt}}
\pgflineto{\pgfpoint{81.439369pt}{141.797729pt}}
\pgfusepath{stroke}
\pgfpathmoveto{\pgfpoint{83.431335pt}{147.373245pt}}
\pgflineto{\pgfpoint{82.435356pt}{141.784378pt}}
\pgfusepath{stroke}
\pgfpathmoveto{\pgfpoint{84.427322pt}{140.853500pt}}
\pgflineto{\pgfpoint{83.431335pt}{147.373245pt}}
\pgfusepath{stroke}
\pgfpathmoveto{\pgfpoint{85.423309pt}{140.987396pt}}
\pgflineto{\pgfpoint{84.427322pt}{140.853500pt}}
\pgfusepath{stroke}
\pgfpathmoveto{\pgfpoint{86.419289pt}{140.777435pt}}
\pgflineto{\pgfpoint{85.423309pt}{140.987396pt}}
\pgfusepath{stroke}
\pgfpathmoveto{\pgfpoint{90.403221pt}{140.777435pt}}
\pgflineto{\pgfpoint{91.399208pt}{140.777435pt}}
\pgfusepath{stroke}
\pgfpathmoveto{\pgfpoint{89.407242pt}{140.777435pt}}
\pgflineto{\pgfpoint{90.403221pt}{140.777435pt}}
\pgfusepath{stroke}
\pgfpathmoveto{\pgfpoint{88.411255pt}{140.777435pt}}
\pgflineto{\pgfpoint{89.407242pt}{140.777435pt}}
\pgfusepath{stroke}
\pgfpathmoveto{\pgfpoint{87.415276pt}{140.777435pt}}
\pgflineto{\pgfpoint{88.411255pt}{140.777435pt}}
\pgfusepath{stroke}
\pgfpathmoveto{\pgfpoint{86.419289pt}{140.777435pt}}
\pgflineto{\pgfpoint{87.415276pt}{140.777435pt}}
\pgfusepath{stroke}
\pgfpathmoveto{\pgfpoint{87.415276pt}{152.008636pt}}
\pgflineto{\pgfpoint{86.419289pt}{140.777435pt}}
\pgfusepath{stroke}
\pgfpathmoveto{\pgfpoint{88.411255pt}{163.097900pt}}
\pgflineto{\pgfpoint{87.415276pt}{152.008636pt}}
\pgfusepath{stroke}
\pgfpathmoveto{\pgfpoint{89.407242pt}{147.608887pt}}
\pgflineto{\pgfpoint{88.411255pt}{163.097900pt}}
\pgfusepath{stroke}
\pgfpathmoveto{\pgfpoint{90.403221pt}{189.355881pt}}
\pgflineto{\pgfpoint{89.407242pt}{147.608887pt}}
\pgfusepath{stroke}
\pgfpathmoveto{\pgfpoint{91.399208pt}{140.777435pt}}
\pgflineto{\pgfpoint{90.403221pt}{189.355881pt}}
\pgfusepath{stroke}
\pgfpathmoveto{\pgfpoint{93.391174pt}{140.777435pt}}
\pgflineto{\pgfpoint{94.387161pt}{140.777435pt}}
\pgfusepath{stroke}
\pgfpathmoveto{\pgfpoint{92.395187pt}{140.777435pt}}
\pgflineto{\pgfpoint{93.391174pt}{140.777435pt}}
\pgfusepath{stroke}
\pgfpathmoveto{\pgfpoint{91.399208pt}{140.777435pt}}
\pgflineto{\pgfpoint{92.395187pt}{140.777435pt}}
\pgfusepath{stroke}
\pgfpathmoveto{\pgfpoint{92.395187pt}{140.789795pt}}
\pgflineto{\pgfpoint{91.399208pt}{140.777435pt}}
\pgfusepath{stroke}
\pgfpathmoveto{\pgfpoint{93.391174pt}{141.209564pt}}
\pgflineto{\pgfpoint{92.395187pt}{140.789795pt}}
\pgfusepath{stroke}
\pgfpathmoveto{\pgfpoint{94.387161pt}{140.777435pt}}
\pgflineto{\pgfpoint{93.391174pt}{141.209564pt}}
\pgfusepath{stroke}
\pgfpathmoveto{\pgfpoint{102.355034pt}{140.777435pt}}
\pgflineto{\pgfpoint{103.351013pt}{140.777435pt}}
\pgfusepath{stroke}
\pgfpathmoveto{\pgfpoint{101.359047pt}{140.777435pt}}
\pgflineto{\pgfpoint{102.355034pt}{140.777435pt}}
\pgfusepath{stroke}
\pgfpathmoveto{\pgfpoint{100.363068pt}{140.777435pt}}
\pgflineto{\pgfpoint{101.359047pt}{140.777435pt}}
\pgfusepath{stroke}
\pgfpathmoveto{\pgfpoint{99.367081pt}{140.777435pt}}
\pgflineto{\pgfpoint{100.363068pt}{140.777435pt}}
\pgfusepath{stroke}
\pgfpathmoveto{\pgfpoint{98.371094pt}{140.777435pt}}
\pgflineto{\pgfpoint{99.367081pt}{140.777435pt}}
\pgfusepath{stroke}
\pgfpathmoveto{\pgfpoint{97.375107pt}{140.777435pt}}
\pgflineto{\pgfpoint{98.371094pt}{140.777435pt}}
\pgfusepath{stroke}
\pgfpathmoveto{\pgfpoint{96.379128pt}{140.777435pt}}
\pgflineto{\pgfpoint{97.375107pt}{140.777435pt}}
\pgfusepath{stroke}
\pgfpathmoveto{\pgfpoint{95.383141pt}{140.777435pt}}
\pgflineto{\pgfpoint{96.379128pt}{140.777435pt}}
\pgfusepath{stroke}
\pgfpathmoveto{\pgfpoint{94.387161pt}{140.777435pt}}
\pgflineto{\pgfpoint{95.383141pt}{140.777435pt}}
\pgfusepath{stroke}
\pgfpathmoveto{\pgfpoint{95.383141pt}{141.040619pt}}
\pgflineto{\pgfpoint{94.387161pt}{140.777435pt}}
\pgfusepath{stroke}
\pgfpathmoveto{\pgfpoint{96.379128pt}{141.052612pt}}
\pgflineto{\pgfpoint{95.383141pt}{141.040619pt}}
\pgfusepath{stroke}
\pgfpathmoveto{\pgfpoint{97.375107pt}{142.021271pt}}
\pgflineto{\pgfpoint{96.379128pt}{141.052612pt}}
\pgfusepath{stroke}
\pgfpathmoveto{\pgfpoint{98.371094pt}{187.727631pt}}
\pgflineto{\pgfpoint{97.375107pt}{142.021271pt}}
\pgfusepath{stroke}
\pgfpathmoveto{\pgfpoint{99.367081pt}{143.174622pt}}
\pgflineto{\pgfpoint{98.371094pt}{187.727631pt}}
\pgfusepath{stroke}
\pgfpathmoveto{\pgfpoint{100.363068pt}{142.326782pt}}
\pgflineto{\pgfpoint{99.367081pt}{143.174622pt}}
\pgfusepath{stroke}
\pgfpathmoveto{\pgfpoint{101.359047pt}{141.473938pt}}
\pgflineto{\pgfpoint{100.363068pt}{142.326782pt}}
\pgfusepath{stroke}
\pgfpathmoveto{\pgfpoint{102.355034pt}{153.173050pt}}
\pgflineto{\pgfpoint{101.359047pt}{141.473938pt}}
\pgfusepath{stroke}
\pgfpathmoveto{\pgfpoint{103.351013pt}{140.777435pt}}
\pgflineto{\pgfpoint{102.355034pt}{153.173050pt}}
\pgfusepath{stroke}
\pgfpathmoveto{\pgfpoint{104.347000pt}{140.777435pt}}
\pgflineto{\pgfpoint{105.342987pt}{140.777435pt}}
\pgfusepath{stroke}
\pgfpathmoveto{\pgfpoint{103.351013pt}{140.777435pt}}
\pgflineto{\pgfpoint{104.347000pt}{140.777435pt}}
\pgfusepath{stroke}
\pgfpathmoveto{\pgfpoint{104.347000pt}{152.852020pt}}
\pgflineto{\pgfpoint{103.351013pt}{140.777435pt}}
\pgfusepath{stroke}
\pgfpathmoveto{\pgfpoint{105.342987pt}{140.777435pt}}
\pgflineto{\pgfpoint{104.347000pt}{152.852020pt}}
\pgfusepath{stroke}
\pgfpathmoveto{\pgfpoint{110.322906pt}{140.777435pt}}
\pgflineto{\pgfpoint{111.318893pt}{140.777435pt}}
\pgfusepath{stroke}
\pgfpathmoveto{\pgfpoint{109.326920pt}{140.777435pt}}
\pgflineto{\pgfpoint{110.322906pt}{140.777435pt}}
\pgfusepath{stroke}
\pgfpathmoveto{\pgfpoint{108.330933pt}{140.777435pt}}
\pgflineto{\pgfpoint{109.326920pt}{140.777435pt}}
\pgfusepath{stroke}
\pgfpathmoveto{\pgfpoint{107.334953pt}{140.777435pt}}
\pgflineto{\pgfpoint{108.330933pt}{140.777435pt}}
\pgfusepath{stroke}
\pgfpathmoveto{\pgfpoint{108.330933pt}{156.427597pt}}
\pgflineto{\pgfpoint{107.334953pt}{140.777435pt}}
\pgfusepath{stroke}
\pgfpathmoveto{\pgfpoint{109.326920pt}{140.854156pt}}
\pgflineto{\pgfpoint{108.330933pt}{156.427597pt}}
\pgfusepath{stroke}
\pgfpathmoveto{\pgfpoint{110.322906pt}{143.840607pt}}
\pgflineto{\pgfpoint{109.326920pt}{140.854156pt}}
\pgfusepath{stroke}
\pgfpathmoveto{\pgfpoint{111.318893pt}{140.777435pt}}
\pgflineto{\pgfpoint{110.322906pt}{143.840607pt}}
\pgfusepath{stroke}
\pgfpathmoveto{\pgfpoint{114.306839pt}{140.777435pt}}
\pgflineto{\pgfpoint{115.302826pt}{140.777435pt}}
\pgfusepath{stroke}
\pgfpathmoveto{\pgfpoint{113.310852pt}{140.777435pt}}
\pgflineto{\pgfpoint{114.306839pt}{140.777435pt}}
\pgfusepath{stroke}
\pgfpathmoveto{\pgfpoint{112.314873pt}{140.777435pt}}
\pgflineto{\pgfpoint{113.310852pt}{140.777435pt}}
\pgfusepath{stroke}
\pgfpathmoveto{\pgfpoint{111.318893pt}{140.777435pt}}
\pgflineto{\pgfpoint{112.314873pt}{140.777435pt}}
\pgfusepath{stroke}
\pgfpathmoveto{\pgfpoint{112.314873pt}{142.841614pt}}
\pgflineto{\pgfpoint{111.318893pt}{140.777435pt}}
\pgfusepath{stroke}
\pgfpathmoveto{\pgfpoint{113.310852pt}{141.335541pt}}
\pgflineto{\pgfpoint{112.314873pt}{142.841614pt}}
\pgfusepath{stroke}
\pgfpathmoveto{\pgfpoint{114.306839pt}{143.259003pt}}
\pgflineto{\pgfpoint{113.310852pt}{141.335541pt}}
\pgfusepath{stroke}
\pgfpathmoveto{\pgfpoint{115.302826pt}{140.777435pt}}
\pgflineto{\pgfpoint{114.306839pt}{143.259003pt}}
\pgfusepath{stroke}
\pgfpathmoveto{\pgfpoint{119.286758pt}{140.777435pt}}
\pgflineto{\pgfpoint{120.282745pt}{140.777435pt}}
\pgfusepath{stroke}
\pgfpathmoveto{\pgfpoint{118.290779pt}{140.777435pt}}
\pgflineto{\pgfpoint{119.286758pt}{140.777435pt}}
\pgfusepath{stroke}
\pgfpathmoveto{\pgfpoint{117.294792pt}{140.777435pt}}
\pgflineto{\pgfpoint{118.290779pt}{140.777435pt}}
\pgfusepath{stroke}
\pgfpathmoveto{\pgfpoint{116.298813pt}{140.777435pt}}
\pgflineto{\pgfpoint{117.294792pt}{140.777435pt}}
\pgfusepath{stroke}
\pgfpathmoveto{\pgfpoint{115.302826pt}{140.777435pt}}
\pgflineto{\pgfpoint{116.298813pt}{140.777435pt}}
\pgfusepath{stroke}
\pgfpathmoveto{\pgfpoint{116.298813pt}{141.607086pt}}
\pgflineto{\pgfpoint{115.302826pt}{140.777435pt}}
\pgfusepath{stroke}
\pgfpathmoveto{\pgfpoint{117.294792pt}{143.559708pt}}
\pgflineto{\pgfpoint{116.298813pt}{141.607086pt}}
\pgfusepath{stroke}
\pgfpathmoveto{\pgfpoint{118.290779pt}{140.848618pt}}
\pgflineto{\pgfpoint{117.294792pt}{143.559708pt}}
\pgfusepath{stroke}
\pgfpathmoveto{\pgfpoint{119.286758pt}{140.997116pt}}
\pgflineto{\pgfpoint{118.290779pt}{140.848618pt}}
\pgfusepath{stroke}
\pgfpathmoveto{\pgfpoint{120.282745pt}{140.777435pt}}
\pgflineto{\pgfpoint{119.286758pt}{140.997116pt}}
\pgfusepath{stroke}
\pgfpathmoveto{\pgfpoint{121.278725pt}{140.777435pt}}
\pgflineto{\pgfpoint{122.274712pt}{140.777435pt}}
\pgfusepath{stroke}
\pgfpathmoveto{\pgfpoint{120.282745pt}{140.777435pt}}
\pgflineto{\pgfpoint{121.278725pt}{140.777435pt}}
\pgfusepath{stroke}
\pgfpathmoveto{\pgfpoint{121.278725pt}{141.992584pt}}
\pgflineto{\pgfpoint{120.282745pt}{140.777435pt}}
\pgfusepath{stroke}
\pgfpathmoveto{\pgfpoint{122.274712pt}{140.777435pt}}
\pgflineto{\pgfpoint{121.278725pt}{141.992584pt}}
\pgfusepath{stroke}
\pgfpathmoveto{\pgfpoint{123.270691pt}{140.777435pt}}
\pgflineto{\pgfpoint{124.266678pt}{140.777435pt}}
\pgfusepath{stroke}
\pgfpathmoveto{\pgfpoint{122.274712pt}{140.777435pt}}
\pgflineto{\pgfpoint{123.270691pt}{140.777435pt}}
\pgfusepath{stroke}
\pgfpathmoveto{\pgfpoint{123.270691pt}{141.045074pt}}
\pgflineto{\pgfpoint{122.274712pt}{140.777435pt}}
\pgfusepath{stroke}
\pgfpathmoveto{\pgfpoint{124.266678pt}{140.777435pt}}
\pgflineto{\pgfpoint{123.270691pt}{141.045074pt}}
\pgfusepath{stroke}
\pgfpathmoveto{\pgfpoint{127.254631pt}{140.777435pt}}
\pgflineto{\pgfpoint{128.250610pt}{140.777435pt}}
\pgfusepath{stroke}
\pgfpathmoveto{\pgfpoint{126.258652pt}{140.777435pt}}
\pgflineto{\pgfpoint{127.254631pt}{140.777435pt}}
\pgfusepath{stroke}
\pgfpathmoveto{\pgfpoint{125.262665pt}{140.777435pt}}
\pgflineto{\pgfpoint{126.258652pt}{140.777435pt}}
\pgfusepath{stroke}
\pgfpathmoveto{\pgfpoint{126.258652pt}{141.187393pt}}
\pgflineto{\pgfpoint{125.262665pt}{140.777435pt}}
\pgfusepath{stroke}
\pgfpathmoveto{\pgfpoint{127.254631pt}{158.624695pt}}
\pgflineto{\pgfpoint{126.258652pt}{141.187393pt}}
\pgfusepath{stroke}
\pgfpathmoveto{\pgfpoint{128.250610pt}{140.777435pt}}
\pgflineto{\pgfpoint{127.254631pt}{158.624695pt}}
\pgfusepath{stroke}
\pgfpathmoveto{\pgfpoint{132.234558pt}{140.777435pt}}
\pgflineto{\pgfpoint{133.230530pt}{140.777435pt}}
\pgfusepath{stroke}
\pgfpathmoveto{\pgfpoint{131.238571pt}{140.777435pt}}
\pgflineto{\pgfpoint{132.234558pt}{140.777435pt}}
\pgfusepath{stroke}
\pgfpathmoveto{\pgfpoint{130.242584pt}{140.777435pt}}
\pgflineto{\pgfpoint{131.238571pt}{140.777435pt}}
\pgfusepath{stroke}
\pgfpathmoveto{\pgfpoint{129.246597pt}{140.777435pt}}
\pgflineto{\pgfpoint{130.242584pt}{140.777435pt}}
\pgfusepath{stroke}
\pgfpathmoveto{\pgfpoint{128.250610pt}{140.777435pt}}
\pgflineto{\pgfpoint{129.246597pt}{140.777435pt}}
\pgfusepath{stroke}
\pgfpathmoveto{\pgfpoint{129.246597pt}{146.889862pt}}
\pgflineto{\pgfpoint{128.250610pt}{140.777435pt}}
\pgfusepath{stroke}
\pgfpathmoveto{\pgfpoint{130.242584pt}{141.305695pt}}
\pgflineto{\pgfpoint{129.246597pt}{146.889862pt}}
\pgfusepath{stroke}
\pgfpathmoveto{\pgfpoint{131.238571pt}{142.924377pt}}
\pgflineto{\pgfpoint{130.242584pt}{141.305695pt}}
\pgfusepath{stroke}
\pgfpathmoveto{\pgfpoint{132.234558pt}{140.784973pt}}
\pgflineto{\pgfpoint{131.238571pt}{142.924377pt}}
\pgfusepath{stroke}
\pgfpathmoveto{\pgfpoint{133.230530pt}{140.777435pt}}
\pgflineto{\pgfpoint{132.234558pt}{140.784973pt}}
\pgfusepath{stroke}
\pgfpathmoveto{\pgfpoint{135.222504pt}{140.777435pt}}
\pgflineto{\pgfpoint{136.218475pt}{140.777435pt}}
\pgfusepath{stroke}
\pgfpathmoveto{\pgfpoint{134.226517pt}{140.777435pt}}
\pgflineto{\pgfpoint{135.222504pt}{140.777435pt}}
\pgfusepath{stroke}
\pgfpathmoveto{\pgfpoint{135.222504pt}{141.478424pt}}
\pgflineto{\pgfpoint{134.226517pt}{140.777435pt}}
\pgfusepath{stroke}
\pgfpathmoveto{\pgfpoint{136.218475pt}{140.777435pt}}
\pgflineto{\pgfpoint{135.222504pt}{141.478424pt}}
\pgfusepath{stroke}
\pgfpathmoveto{\pgfpoint{141.198410pt}{140.777435pt}}
\pgflineto{\pgfpoint{142.194382pt}{140.777435pt}}
\pgfusepath{stroke}
\pgfpathmoveto{\pgfpoint{140.202423pt}{140.777435pt}}
\pgflineto{\pgfpoint{141.198410pt}{140.777435pt}}
\pgfusepath{stroke}
\pgfpathmoveto{\pgfpoint{139.206436pt}{140.777435pt}}
\pgflineto{\pgfpoint{140.202423pt}{140.777435pt}}
\pgfusepath{stroke}
\pgfpathmoveto{\pgfpoint{138.210449pt}{140.777435pt}}
\pgflineto{\pgfpoint{139.206436pt}{140.777435pt}}
\pgfusepath{stroke}
\pgfpathmoveto{\pgfpoint{137.214478pt}{140.777435pt}}
\pgflineto{\pgfpoint{138.210449pt}{140.777435pt}}
\pgfusepath{stroke}
\pgfpathmoveto{\pgfpoint{136.218475pt}{140.777435pt}}
\pgflineto{\pgfpoint{137.214478pt}{140.777435pt}}
\pgfusepath{stroke}
\pgfpathmoveto{\pgfpoint{137.214478pt}{144.493912pt}}
\pgflineto{\pgfpoint{136.218475pt}{140.777435pt}}
\pgfusepath{stroke}
\pgfpathmoveto{\pgfpoint{138.210449pt}{140.824677pt}}
\pgflineto{\pgfpoint{137.214478pt}{144.493912pt}}
\pgfusepath{stroke}
\pgfpathmoveto{\pgfpoint{139.206436pt}{155.555008pt}}
\pgflineto{\pgfpoint{138.210449pt}{140.824677pt}}
\pgfusepath{stroke}
\pgfpathmoveto{\pgfpoint{140.202423pt}{141.880554pt}}
\pgflineto{\pgfpoint{139.206436pt}{155.555008pt}}
\pgfusepath{stroke}
\pgfpathmoveto{\pgfpoint{141.198410pt}{141.096649pt}}
\pgflineto{\pgfpoint{140.202423pt}{141.880554pt}}
\pgfusepath{stroke}
\pgfpathmoveto{\pgfpoint{142.194382pt}{140.777435pt}}
\pgflineto{\pgfpoint{141.198410pt}{141.096649pt}}
\pgfusepath{stroke}
\pgfpathmoveto{\pgfpoint{143.190369pt}{140.777435pt}}
\pgflineto{\pgfpoint{144.186356pt}{140.777435pt}}
\pgfusepath{stroke}
\pgfpathmoveto{\pgfpoint{142.194382pt}{140.777435pt}}
\pgflineto{\pgfpoint{143.190369pt}{140.777435pt}}
\pgfusepath{stroke}
\pgfpathmoveto{\pgfpoint{143.190369pt}{141.972321pt}}
\pgflineto{\pgfpoint{142.194382pt}{140.777435pt}}
\pgfusepath{stroke}
\pgfpathmoveto{\pgfpoint{144.186356pt}{140.777435pt}}
\pgflineto{\pgfpoint{143.190369pt}{141.972321pt}}
\pgfusepath{stroke}
\pgfpathmoveto{\pgfpoint{147.174316pt}{140.777435pt}}
\pgflineto{\pgfpoint{148.170288pt}{140.777435pt}}
\pgfusepath{stroke}
\pgfpathmoveto{\pgfpoint{146.178314pt}{140.777435pt}}
\pgflineto{\pgfpoint{147.174316pt}{140.777435pt}}
\pgfusepath{stroke}
\pgfpathmoveto{\pgfpoint{145.182343pt}{140.777435pt}}
\pgflineto{\pgfpoint{146.178314pt}{140.777435pt}}
\pgfusepath{stroke}
\pgfpathmoveto{\pgfpoint{144.186356pt}{140.777435pt}}
\pgflineto{\pgfpoint{145.182343pt}{140.777435pt}}
\pgfusepath{stroke}
\pgfpathmoveto{\pgfpoint{145.182343pt}{141.507751pt}}
\pgflineto{\pgfpoint{144.186356pt}{140.777435pt}}
\pgfusepath{stroke}
\pgfpathmoveto{\pgfpoint{146.178314pt}{141.062988pt}}
\pgflineto{\pgfpoint{145.182343pt}{141.507751pt}}
\pgfusepath{stroke}
\pgfpathmoveto{\pgfpoint{147.174316pt}{143.409790pt}}
\pgflineto{\pgfpoint{146.178314pt}{141.062988pt}}
\pgfusepath{stroke}
\pgfpathmoveto{\pgfpoint{148.170288pt}{140.777435pt}}
\pgflineto{\pgfpoint{147.174316pt}{143.409790pt}}
\pgfusepath{stroke}
\pgfpathmoveto{\pgfpoint{153.150208pt}{140.777435pt}}
\pgflineto{\pgfpoint{154.146194pt}{140.777435pt}}
\pgfusepath{stroke}
\pgfpathmoveto{\pgfpoint{152.154221pt}{140.777435pt}}
\pgflineto{\pgfpoint{153.150208pt}{140.777435pt}}
\pgfusepath{stroke}
\pgfpathmoveto{\pgfpoint{151.158249pt}{140.777435pt}}
\pgflineto{\pgfpoint{152.154221pt}{140.777435pt}}
\pgfusepath{stroke}
\pgfpathmoveto{\pgfpoint{150.162262pt}{140.777435pt}}
\pgflineto{\pgfpoint{151.158249pt}{140.777435pt}}
\pgfusepath{stroke}
\pgfpathmoveto{\pgfpoint{151.158249pt}{141.561523pt}}
\pgflineto{\pgfpoint{150.162262pt}{140.777435pt}}
\pgfusepath{stroke}
\pgfpathmoveto{\pgfpoint{152.154221pt}{144.148651pt}}
\pgflineto{\pgfpoint{151.158249pt}{141.561523pt}}
\pgfusepath{stroke}
\pgfpathmoveto{\pgfpoint{153.150208pt}{143.520859pt}}
\pgflineto{\pgfpoint{152.154221pt}{144.148651pt}}
\pgfusepath{stroke}
\pgfpathmoveto{\pgfpoint{154.146194pt}{140.777435pt}}
\pgflineto{\pgfpoint{153.150208pt}{143.520859pt}}
\pgfusepath{stroke}
\pgfpathmoveto{\pgfpoint{155.142181pt}{140.777435pt}}
\pgflineto{\pgfpoint{156.138168pt}{140.777435pt}}
\pgfusepath{stroke}
\pgfpathmoveto{\pgfpoint{154.146194pt}{140.777435pt}}
\pgflineto{\pgfpoint{155.142181pt}{140.777435pt}}
\pgfusepath{stroke}
\pgfpathmoveto{\pgfpoint{155.142181pt}{195.815277pt}}
\pgflineto{\pgfpoint{154.146194pt}{140.777435pt}}
\pgfusepath{stroke}
\pgfpathmoveto{\pgfpoint{156.138168pt}{140.777435pt}}
\pgflineto{\pgfpoint{155.142181pt}{195.815277pt}}
\pgfusepath{stroke}
\pgfpathmoveto{\pgfpoint{158.130127pt}{140.777435pt}}
\pgflineto{\pgfpoint{159.126114pt}{140.777435pt}}
\pgfusepath{stroke}
\pgfpathmoveto{\pgfpoint{157.134155pt}{140.777435pt}}
\pgflineto{\pgfpoint{158.130127pt}{140.777435pt}}
\pgfusepath{stroke}
\pgfpathmoveto{\pgfpoint{156.138168pt}{140.777435pt}}
\pgflineto{\pgfpoint{157.134155pt}{140.777435pt}}
\pgfusepath{stroke}
\pgfpathmoveto{\pgfpoint{157.134155pt}{140.926575pt}}
\pgflineto{\pgfpoint{156.138168pt}{140.777435pt}}
\pgfusepath{stroke}
\pgfpathmoveto{\pgfpoint{158.130127pt}{140.979492pt}}
\pgflineto{\pgfpoint{157.134155pt}{140.926575pt}}
\pgfusepath{stroke}
\pgfpathmoveto{\pgfpoint{159.126114pt}{140.777435pt}}
\pgflineto{\pgfpoint{158.130127pt}{140.979492pt}}
\pgfusepath{stroke}
\pgfpathmoveto{\pgfpoint{162.114075pt}{140.777435pt}}
\pgflineto{\pgfpoint{163.110062pt}{140.777435pt}}
\pgfusepath{stroke}
\pgfpathmoveto{\pgfpoint{161.118088pt}{140.777435pt}}
\pgflineto{\pgfpoint{162.114075pt}{140.777435pt}}
\pgfusepath{stroke}
\pgfpathmoveto{\pgfpoint{160.122101pt}{140.777435pt}}
\pgflineto{\pgfpoint{161.118088pt}{140.777435pt}}
\pgfusepath{stroke}
\pgfpathmoveto{\pgfpoint{159.126114pt}{140.777435pt}}
\pgflineto{\pgfpoint{160.122101pt}{140.777435pt}}
\pgfusepath{stroke}
\pgfpathmoveto{\pgfpoint{160.122101pt}{158.193100pt}}
\pgflineto{\pgfpoint{159.126114pt}{140.777435pt}}
\pgfusepath{stroke}
\pgfpathmoveto{\pgfpoint{161.118088pt}{140.795273pt}}
\pgflineto{\pgfpoint{160.122101pt}{158.193100pt}}
\pgfusepath{stroke}
\pgfpathmoveto{\pgfpoint{162.114075pt}{141.404877pt}}
\pgflineto{\pgfpoint{161.118088pt}{140.795273pt}}
\pgfusepath{stroke}
\pgfpathmoveto{\pgfpoint{163.110062pt}{140.777435pt}}
\pgflineto{\pgfpoint{162.114075pt}{141.404877pt}}
\pgfusepath{stroke}
\pgfpathmoveto{\pgfpoint{165.102020pt}{140.777435pt}}
\pgflineto{\pgfpoint{166.098007pt}{140.777435pt}}
\pgfusepath{stroke}
\pgfpathmoveto{\pgfpoint{164.106033pt}{140.777435pt}}
\pgflineto{\pgfpoint{165.102020pt}{140.777435pt}}
\pgfusepath{stroke}
\pgfpathmoveto{\pgfpoint{163.110062pt}{140.777435pt}}
\pgflineto{\pgfpoint{164.106033pt}{140.777435pt}}
\pgfusepath{stroke}
\pgfpathmoveto{\pgfpoint{164.106033pt}{194.127029pt}}
\pgflineto{\pgfpoint{163.110062pt}{140.777435pt}}
\pgfusepath{stroke}
\pgfpathmoveto{\pgfpoint{165.102020pt}{140.828079pt}}
\pgflineto{\pgfpoint{164.106033pt}{194.127029pt}}
\pgfusepath{stroke}
\pgfpathmoveto{\pgfpoint{166.098007pt}{140.777435pt}}
\pgflineto{\pgfpoint{165.102020pt}{140.828079pt}}
\pgfusepath{stroke}
\pgfpathmoveto{\pgfpoint{168.089966pt}{140.777435pt}}
\pgflineto{\pgfpoint{169.085953pt}{140.777435pt}}
\pgfusepath{stroke}
\pgfpathmoveto{\pgfpoint{167.093994pt}{140.777435pt}}
\pgflineto{\pgfpoint{168.089966pt}{140.777435pt}}
\pgfusepath{stroke}
\pgfpathmoveto{\pgfpoint{166.098007pt}{140.777435pt}}
\pgflineto{\pgfpoint{167.093994pt}{140.777435pt}}
\pgfusepath{stroke}
\pgfpathmoveto{\pgfpoint{167.093994pt}{141.434799pt}}
\pgflineto{\pgfpoint{166.098007pt}{140.777435pt}}
\pgfusepath{stroke}
\pgfpathmoveto{\pgfpoint{167.415314pt}{205.642242pt}}
\pgflineto{\pgfpoint{167.093994pt}{141.434799pt}}
\pgfusepath{stroke}
\pgfpathmoveto{\pgfpoint{169.085953pt}{140.777435pt}}
\pgflineto{\pgfpoint{168.762405pt}{205.642242pt}}
\pgfusepath{stroke}
\pgfpathmoveto{\pgfpoint{175.061859pt}{140.777435pt}}
\pgflineto{\pgfpoint{176.057846pt}{140.777435pt}}
\pgfusepath{stroke}
\pgfpathmoveto{\pgfpoint{174.065872pt}{140.777435pt}}
\pgflineto{\pgfpoint{175.061859pt}{140.777435pt}}
\pgfusepath{stroke}
\pgfpathmoveto{\pgfpoint{173.069885pt}{140.777435pt}}
\pgflineto{\pgfpoint{174.065872pt}{140.777435pt}}
\pgfusepath{stroke}
\pgfpathmoveto{\pgfpoint{172.073914pt}{140.777435pt}}
\pgflineto{\pgfpoint{173.069885pt}{140.777435pt}}
\pgfusepath{stroke}
\pgfpathmoveto{\pgfpoint{171.077911pt}{140.777435pt}}
\pgflineto{\pgfpoint{172.073914pt}{140.777435pt}}
\pgfusepath{stroke}
\pgfpathmoveto{\pgfpoint{170.081940pt}{140.777435pt}}
\pgflineto{\pgfpoint{171.077911pt}{140.777435pt}}
\pgfusepath{stroke}
\pgfpathmoveto{\pgfpoint{169.085953pt}{140.777435pt}}
\pgflineto{\pgfpoint{170.081940pt}{140.777435pt}}
\pgfusepath{stroke}
\pgfpathmoveto{\pgfpoint{170.081940pt}{146.720169pt}}
\pgflineto{\pgfpoint{169.085953pt}{140.777435pt}}
\pgfusepath{stroke}
\pgfpathmoveto{\pgfpoint{171.077911pt}{145.686310pt}}
\pgflineto{\pgfpoint{170.081940pt}{146.720169pt}}
\pgfusepath{stroke}
\pgfpathmoveto{\pgfpoint{172.073914pt}{149.736984pt}}
\pgflineto{\pgfpoint{171.077911pt}{145.686310pt}}
\pgfusepath{stroke}
\pgfpathmoveto{\pgfpoint{173.069885pt}{148.531342pt}}
\pgflineto{\pgfpoint{172.073914pt}{149.736984pt}}
\pgfusepath{stroke}
\pgfpathmoveto{\pgfpoint{174.065872pt}{141.515808pt}}
\pgflineto{\pgfpoint{173.069885pt}{148.531342pt}}
\pgfusepath{stroke}
\pgfpathmoveto{\pgfpoint{175.061859pt}{141.323135pt}}
\pgflineto{\pgfpoint{174.065872pt}{141.515808pt}}
\pgfusepath{stroke}
\pgfpathmoveto{\pgfpoint{176.057846pt}{140.777435pt}}
\pgflineto{\pgfpoint{175.061859pt}{141.323135pt}}
\pgfusepath{stroke}
\pgfpathmoveto{\pgfpoint{181.037766pt}{140.777435pt}}
\pgflineto{\pgfpoint{182.033752pt}{140.777435pt}}
\pgfusepath{stroke}
\pgfpathmoveto{\pgfpoint{180.041779pt}{140.777435pt}}
\pgflineto{\pgfpoint{181.037766pt}{140.777435pt}}
\pgfusepath{stroke}
\pgfpathmoveto{\pgfpoint{179.045792pt}{140.777435pt}}
\pgflineto{\pgfpoint{180.041779pt}{140.777435pt}}
\pgfusepath{stroke}
\pgfpathmoveto{\pgfpoint{178.049805pt}{140.777435pt}}
\pgflineto{\pgfpoint{179.045792pt}{140.777435pt}}
\pgfusepath{stroke}
\pgfpathmoveto{\pgfpoint{177.053818pt}{140.777435pt}}
\pgflineto{\pgfpoint{178.049805pt}{140.777435pt}}
\pgfusepath{stroke}
\pgfpathmoveto{\pgfpoint{176.057846pt}{140.777435pt}}
\pgflineto{\pgfpoint{177.053818pt}{140.777435pt}}
\pgfusepath{stroke}
\pgfpathmoveto{\pgfpoint{177.053818pt}{144.404053pt}}
\pgflineto{\pgfpoint{176.057846pt}{140.777435pt}}
\pgfusepath{stroke}
\pgfpathmoveto{\pgfpoint{178.049805pt}{144.353317pt}}
\pgflineto{\pgfpoint{177.053818pt}{144.404053pt}}
\pgfusepath{stroke}
\pgfpathmoveto{\pgfpoint{179.045792pt}{140.963043pt}}
\pgflineto{\pgfpoint{178.049805pt}{144.353317pt}}
\pgfusepath{stroke}
\pgfpathmoveto{\pgfpoint{180.041779pt}{140.930664pt}}
\pgflineto{\pgfpoint{179.045792pt}{140.963043pt}}
\pgfusepath{stroke}
\pgfpathmoveto{\pgfpoint{181.037766pt}{141.619095pt}}
\pgflineto{\pgfpoint{180.041779pt}{140.930664pt}}
\pgfusepath{stroke}
\pgfpathmoveto{\pgfpoint{182.033752pt}{140.777435pt}}
\pgflineto{\pgfpoint{181.037766pt}{141.619095pt}}
\pgfusepath{stroke}
\pgfpathmoveto{\pgfpoint{183.029724pt}{140.777435pt}}
\pgflineto{\pgfpoint{184.025711pt}{140.777435pt}}
\pgfusepath{stroke}
\pgfpathmoveto{\pgfpoint{182.033752pt}{140.777435pt}}
\pgflineto{\pgfpoint{183.029724pt}{140.777435pt}}
\pgfusepath{stroke}
\pgfpathmoveto{\pgfpoint{183.029724pt}{142.399628pt}}
\pgflineto{\pgfpoint{182.033752pt}{140.777435pt}}
\pgfusepath{stroke}
\pgfpathmoveto{\pgfpoint{184.025711pt}{140.777435pt}}
\pgflineto{\pgfpoint{183.029724pt}{142.399628pt}}
\pgfusepath{stroke}
\pgfpathmoveto{\pgfpoint{187.013672pt}{140.777435pt}}
\pgflineto{\pgfpoint{188.009659pt}{140.777435pt}}
\pgfusepath{stroke}
\pgfpathmoveto{\pgfpoint{186.017685pt}{140.777435pt}}
\pgflineto{\pgfpoint{187.013672pt}{140.777435pt}}
\pgfusepath{stroke}
\pgfpathmoveto{\pgfpoint{187.013672pt}{141.098480pt}}
\pgflineto{\pgfpoint{186.017685pt}{140.777435pt}}
\pgfusepath{stroke}
\pgfpathmoveto{\pgfpoint{188.009659pt}{140.777435pt}}
\pgflineto{\pgfpoint{187.013672pt}{141.098480pt}}
\pgfusepath{stroke}
\pgfpathmoveto{\pgfpoint{198.965469pt}{140.777435pt}}
\pgflineto{\pgfpoint{199.961456pt}{140.777435pt}}
\pgfusepath{stroke}
\pgfpathmoveto{\pgfpoint{197.969498pt}{140.777435pt}}
\pgflineto{\pgfpoint{198.965469pt}{140.777435pt}}
\pgfusepath{stroke}
\pgfpathmoveto{\pgfpoint{196.973511pt}{140.777435pt}}
\pgflineto{\pgfpoint{197.969498pt}{140.777435pt}}
\pgfusepath{stroke}
\pgfpathmoveto{\pgfpoint{195.977524pt}{140.777435pt}}
\pgflineto{\pgfpoint{196.973511pt}{140.777435pt}}
\pgfusepath{stroke}
\pgfpathmoveto{\pgfpoint{194.981537pt}{140.777435pt}}
\pgflineto{\pgfpoint{195.977524pt}{140.777435pt}}
\pgfusepath{stroke}
\pgfpathmoveto{\pgfpoint{193.985565pt}{140.777435pt}}
\pgflineto{\pgfpoint{194.981537pt}{140.777435pt}}
\pgfusepath{stroke}
\pgfpathmoveto{\pgfpoint{192.989563pt}{140.777435pt}}
\pgflineto{\pgfpoint{193.985565pt}{140.777435pt}}
\pgfusepath{stroke}
\pgfpathmoveto{\pgfpoint{191.993591pt}{140.777435pt}}
\pgflineto{\pgfpoint{192.989563pt}{140.777435pt}}
\pgfusepath{stroke}
\pgfpathmoveto{\pgfpoint{190.997604pt}{140.777435pt}}
\pgflineto{\pgfpoint{191.993591pt}{140.777435pt}}
\pgfusepath{stroke}
\pgfpathmoveto{\pgfpoint{190.001617pt}{140.777435pt}}
\pgflineto{\pgfpoint{190.997604pt}{140.777435pt}}
\pgfusepath{stroke}
\pgfpathmoveto{\pgfpoint{189.005630pt}{140.777435pt}}
\pgflineto{\pgfpoint{190.001617pt}{140.777435pt}}
\pgfusepath{stroke}
\pgfpathmoveto{\pgfpoint{188.009659pt}{140.777435pt}}
\pgflineto{\pgfpoint{189.005630pt}{140.777435pt}}
\pgfusepath{stroke}
\pgfpathmoveto{\pgfpoint{189.005630pt}{140.805771pt}}
\pgflineto{\pgfpoint{188.009659pt}{140.777435pt}}
\pgfusepath{stroke}
\pgfpathmoveto{\pgfpoint{190.001617pt}{149.564423pt}}
\pgflineto{\pgfpoint{189.005630pt}{140.805771pt}}
\pgfusepath{stroke}
\pgfpathmoveto{\pgfpoint{190.997604pt}{140.863358pt}}
\pgflineto{\pgfpoint{190.001617pt}{149.564423pt}}
\pgfusepath{stroke}
\pgfpathmoveto{\pgfpoint{191.993591pt}{142.184387pt}}
\pgflineto{\pgfpoint{190.997604pt}{140.863358pt}}
\pgfusepath{stroke}
\pgfpathmoveto{\pgfpoint{192.989563pt}{195.151352pt}}
\pgflineto{\pgfpoint{191.993591pt}{142.184387pt}}
\pgfusepath{stroke}
\pgfpathmoveto{\pgfpoint{193.985565pt}{141.529373pt}}
\pgflineto{\pgfpoint{192.989563pt}{195.151352pt}}
\pgfusepath{stroke}
\pgfpathmoveto{\pgfpoint{194.981537pt}{140.831909pt}}
\pgflineto{\pgfpoint{193.985565pt}{141.529373pt}}
\pgfusepath{stroke}
\pgfpathmoveto{\pgfpoint{195.977524pt}{149.854721pt}}
\pgflineto{\pgfpoint{194.981537pt}{140.831909pt}}
\pgfusepath{stroke}
\pgfpathmoveto{\pgfpoint{196.973511pt}{144.710114pt}}
\pgflineto{\pgfpoint{195.977524pt}{149.854721pt}}
\pgfusepath{stroke}
\pgfpathmoveto{\pgfpoint{197.969498pt}{144.257416pt}}
\pgflineto{\pgfpoint{196.973511pt}{144.710114pt}}
\pgfusepath{stroke}
\pgfpathmoveto{\pgfpoint{198.965469pt}{197.820709pt}}
\pgflineto{\pgfpoint{197.969498pt}{144.257416pt}}
\pgfusepath{stroke}
\pgfpathmoveto{\pgfpoint{199.961456pt}{140.777435pt}}
\pgflineto{\pgfpoint{198.965469pt}{197.820709pt}}
\pgfusepath{stroke}
\pgfpathmoveto{\pgfpoint{201.953430pt}{140.777435pt}}
\pgflineto{\pgfpoint{202.949402pt}{140.777435pt}}
\pgfusepath{stroke}
\pgfpathmoveto{\pgfpoint{200.957443pt}{140.777435pt}}
\pgflineto{\pgfpoint{201.953430pt}{140.777435pt}}
\pgfusepath{stroke}
\pgfpathmoveto{\pgfpoint{199.961456pt}{140.777435pt}}
\pgflineto{\pgfpoint{200.957443pt}{140.777435pt}}
\pgfusepath{stroke}
\pgfpathmoveto{\pgfpoint{200.957443pt}{141.105011pt}}
\pgflineto{\pgfpoint{199.961456pt}{140.777435pt}}
\pgfusepath{stroke}
\pgfpathmoveto{\pgfpoint{201.953430pt}{140.916779pt}}
\pgflineto{\pgfpoint{200.957443pt}{141.105011pt}}
\pgfusepath{stroke}
\pgfpathmoveto{\pgfpoint{202.949402pt}{140.777435pt}}
\pgflineto{\pgfpoint{201.953430pt}{140.916779pt}}
\pgfusepath{stroke}
\pgfpathmoveto{\pgfpoint{205.937347pt}{140.777435pt}}
\pgflineto{\pgfpoint{206.933334pt}{140.777435pt}}
\pgfusepath{stroke}
\pgfpathmoveto{\pgfpoint{204.941376pt}{140.777435pt}}
\pgflineto{\pgfpoint{205.937347pt}{140.777435pt}}
\pgfusepath{stroke}
\pgfpathmoveto{\pgfpoint{205.937347pt}{144.822937pt}}
\pgflineto{\pgfpoint{204.941376pt}{140.777435pt}}
\pgfusepath{stroke}
\pgfpathmoveto{\pgfpoint{206.933334pt}{140.777435pt}}
\pgflineto{\pgfpoint{205.937347pt}{144.822937pt}}
\pgfusepath{stroke}
\pgfpathmoveto{\pgfpoint{215.897217pt}{140.777435pt}}
\pgflineto{\pgfpoint{216.893188pt}{140.777435pt}}
\pgfusepath{stroke}
\pgfpathmoveto{\pgfpoint{214.901215pt}{140.777435pt}}
\pgflineto{\pgfpoint{215.897217pt}{140.777435pt}}
\pgfusepath{stroke}
\pgfpathmoveto{\pgfpoint{213.905228pt}{140.777435pt}}
\pgflineto{\pgfpoint{214.901215pt}{140.777435pt}}
\pgfusepath{stroke}
\pgfpathmoveto{\pgfpoint{212.909241pt}{140.777435pt}}
\pgflineto{\pgfpoint{213.905228pt}{140.777435pt}}
\pgfusepath{stroke}
\pgfpathmoveto{\pgfpoint{211.913269pt}{140.777435pt}}
\pgflineto{\pgfpoint{212.909241pt}{140.777435pt}}
\pgfusepath{stroke}
\pgfpathmoveto{\pgfpoint{210.917267pt}{140.777435pt}}
\pgflineto{\pgfpoint{211.913269pt}{140.777435pt}}
\pgfusepath{stroke}
\pgfpathmoveto{\pgfpoint{209.921295pt}{140.777435pt}}
\pgflineto{\pgfpoint{210.917267pt}{140.777435pt}}
\pgfusepath{stroke}
\pgfpathmoveto{\pgfpoint{208.925323pt}{140.777435pt}}
\pgflineto{\pgfpoint{209.921295pt}{140.777435pt}}
\pgfusepath{stroke}
\pgfpathmoveto{\pgfpoint{207.929337pt}{140.777435pt}}
\pgflineto{\pgfpoint{208.925323pt}{140.777435pt}}
\pgfusepath{stroke}
\pgfpathmoveto{\pgfpoint{208.925323pt}{168.977936pt}}
\pgflineto{\pgfpoint{207.929337pt}{140.777435pt}}
\pgfusepath{stroke}
\pgfpathmoveto{\pgfpoint{209.608246pt}{205.642242pt}}
\pgflineto{\pgfpoint{208.925323pt}{168.977936pt}}
\pgfusepath{stroke}
\pgfpathmoveto{\pgfpoint{210.917267pt}{156.156219pt}}
\pgflineto{\pgfpoint{210.173798pt}{205.642242pt}}
\pgfusepath{stroke}
\pgfpathmoveto{\pgfpoint{211.913269pt}{142.768951pt}}
\pgflineto{\pgfpoint{210.917267pt}{156.156219pt}}
\pgfusepath{stroke}
\pgfpathmoveto{\pgfpoint{212.909241pt}{140.796997pt}}
\pgflineto{\pgfpoint{211.913269pt}{142.768951pt}}
\pgfusepath{stroke}
\pgfpathmoveto{\pgfpoint{213.905228pt}{147.912155pt}}
\pgflineto{\pgfpoint{212.909241pt}{140.796997pt}}
\pgfusepath{stroke}
\pgfpathmoveto{\pgfpoint{214.901215pt}{141.340469pt}}
\pgflineto{\pgfpoint{213.905228pt}{147.912155pt}}
\pgfusepath{stroke}
\pgfpathmoveto{\pgfpoint{215.897217pt}{146.916656pt}}
\pgflineto{\pgfpoint{214.901215pt}{141.340469pt}}
\pgfusepath{stroke}
\pgfpathmoveto{\pgfpoint{216.893188pt}{140.777435pt}}
\pgflineto{\pgfpoint{215.897217pt}{146.916656pt}}
\pgfusepath{stroke}
\pgfpathmoveto{\pgfpoint{217.889160pt}{140.777435pt}}
\pgflineto{\pgfpoint{218.885147pt}{140.777435pt}}
\pgfusepath{stroke}
\pgfpathmoveto{\pgfpoint{216.893188pt}{140.777435pt}}
\pgflineto{\pgfpoint{217.889160pt}{140.777435pt}}
\pgfusepath{stroke}
\pgfpathmoveto{\pgfpoint{217.889160pt}{144.868698pt}}
\pgflineto{\pgfpoint{216.893188pt}{140.777435pt}}
\pgfusepath{stroke}
\pgfpathmoveto{\pgfpoint{218.885147pt}{140.777435pt}}
\pgflineto{\pgfpoint{217.889160pt}{144.868698pt}}
\pgfusepath{stroke}
\pgfpathmoveto{\pgfpoint{219.881134pt}{140.777435pt}}
\pgflineto{\pgfpoint{220.877121pt}{140.777435pt}}
\pgfusepath{stroke}
\pgfpathmoveto{\pgfpoint{218.885147pt}{140.777435pt}}
\pgflineto{\pgfpoint{219.881134pt}{140.777435pt}}
\pgfusepath{stroke}
\pgfpathmoveto{\pgfpoint{219.881134pt}{141.240646pt}}
\pgflineto{\pgfpoint{218.885147pt}{140.777435pt}}
\pgfusepath{stroke}
\pgfpathmoveto{\pgfpoint{220.877121pt}{140.777435pt}}
\pgflineto{\pgfpoint{219.881134pt}{141.240646pt}}
\pgfusepath{stroke}
\pgfpathmoveto{\pgfpoint{225.857040pt}{140.777435pt}}
\pgflineto{\pgfpoint{226.853027pt}{140.777435pt}}
\pgfusepath{stroke}
\pgfpathmoveto{\pgfpoint{224.861053pt}{140.777435pt}}
\pgflineto{\pgfpoint{225.857040pt}{140.777435pt}}
\pgfusepath{stroke}
\pgfpathmoveto{\pgfpoint{223.865082pt}{140.777435pt}}
\pgflineto{\pgfpoint{224.861053pt}{140.777435pt}}
\pgfusepath{stroke}
\pgfpathmoveto{\pgfpoint{222.869080pt}{140.777435pt}}
\pgflineto{\pgfpoint{223.865082pt}{140.777435pt}}
\pgfusepath{stroke}
\pgfpathmoveto{\pgfpoint{223.865082pt}{142.648438pt}}
\pgflineto{\pgfpoint{222.869080pt}{140.777435pt}}
\pgfusepath{stroke}
\pgfpathmoveto{\pgfpoint{224.861053pt}{140.814316pt}}
\pgflineto{\pgfpoint{223.865082pt}{142.648438pt}}
\pgfusepath{stroke}
\pgfpathmoveto{\pgfpoint{225.857040pt}{142.893295pt}}
\pgflineto{\pgfpoint{224.861053pt}{140.814316pt}}
\pgfusepath{stroke}
\pgfpathmoveto{\pgfpoint{226.853027pt}{140.777435pt}}
\pgflineto{\pgfpoint{225.857040pt}{142.893295pt}}
\pgfusepath{stroke}
\pgfpathmoveto{\pgfpoint{231.832932pt}{140.777435pt}}
\pgflineto{\pgfpoint{232.828934pt}{140.777435pt}}
\pgfusepath{stroke}
\pgfpathmoveto{\pgfpoint{230.836945pt}{140.777435pt}}
\pgflineto{\pgfpoint{231.832932pt}{140.777435pt}}
\pgfusepath{stroke}
\pgfpathmoveto{\pgfpoint{229.840973pt}{140.777435pt}}
\pgflineto{\pgfpoint{230.836945pt}{140.777435pt}}
\pgfusepath{stroke}
\pgfpathmoveto{\pgfpoint{228.845001pt}{140.777435pt}}
\pgflineto{\pgfpoint{229.840973pt}{140.777435pt}}
\pgfusepath{stroke}
\pgfpathmoveto{\pgfpoint{227.849014pt}{140.777435pt}}
\pgflineto{\pgfpoint{228.845001pt}{140.777435pt}}
\pgfusepath{stroke}
\pgfpathmoveto{\pgfpoint{226.853027pt}{140.777435pt}}
\pgflineto{\pgfpoint{227.849014pt}{140.777435pt}}
\pgfusepath{stroke}
\pgfpathmoveto{\pgfpoint{227.849014pt}{141.269379pt}}
\pgflineto{\pgfpoint{226.853027pt}{140.777435pt}}
\pgfusepath{stroke}
\pgfpathmoveto{\pgfpoint{228.845001pt}{143.487183pt}}
\pgflineto{\pgfpoint{227.849014pt}{141.269379pt}}
\pgfusepath{stroke}
\pgfpathmoveto{\pgfpoint{229.840973pt}{143.864670pt}}
\pgflineto{\pgfpoint{228.845001pt}{143.487183pt}}
\pgfusepath{stroke}
\pgfpathmoveto{\pgfpoint{230.836945pt}{141.428329pt}}
\pgflineto{\pgfpoint{229.840973pt}{143.864670pt}}
\pgfusepath{stroke}
\pgfpathmoveto{\pgfpoint{231.832932pt}{162.511902pt}}
\pgflineto{\pgfpoint{230.836945pt}{141.428329pt}}
\pgfusepath{stroke}
\pgfpathmoveto{\pgfpoint{232.828934pt}{140.777435pt}}
\pgflineto{\pgfpoint{231.832932pt}{162.511902pt}}
\pgfusepath{stroke}
\pgfpathmoveto{\pgfpoint{238.804825pt}{140.777435pt}}
\pgflineto{\pgfpoint{239.800812pt}{140.777435pt}}
\pgfusepath{stroke}
\pgfpathmoveto{\pgfpoint{237.808838pt}{140.777435pt}}
\pgflineto{\pgfpoint{238.804825pt}{140.777435pt}}
\pgfusepath{stroke}
\pgfpathmoveto{\pgfpoint{236.812866pt}{140.777435pt}}
\pgflineto{\pgfpoint{237.808838pt}{140.777435pt}}
\pgfusepath{stroke}
\pgfpathmoveto{\pgfpoint{235.816864pt}{140.777435pt}}
\pgflineto{\pgfpoint{236.812866pt}{140.777435pt}}
\pgfusepath{stroke}
\pgfpathmoveto{\pgfpoint{234.820892pt}{140.777435pt}}
\pgflineto{\pgfpoint{235.816864pt}{140.777435pt}}
\pgfusepath{stroke}
\pgfpathmoveto{\pgfpoint{233.824921pt}{140.777435pt}}
\pgflineto{\pgfpoint{234.820892pt}{140.777435pt}}
\pgfusepath{stroke}
\pgfpathmoveto{\pgfpoint{232.828934pt}{140.777435pt}}
\pgflineto{\pgfpoint{233.824921pt}{140.777435pt}}
\pgfusepath{stroke}
\pgfpathmoveto{\pgfpoint{233.824921pt}{143.663483pt}}
\pgflineto{\pgfpoint{232.828934pt}{140.777435pt}}
\pgfusepath{stroke}
\pgfpathmoveto{\pgfpoint{234.820892pt}{144.424850pt}}
\pgflineto{\pgfpoint{233.824921pt}{143.663483pt}}
\pgfusepath{stroke}
\pgfpathmoveto{\pgfpoint{235.816864pt}{141.070068pt}}
\pgflineto{\pgfpoint{234.820892pt}{144.424850pt}}
\pgfusepath{stroke}
\pgfpathmoveto{\pgfpoint{236.812866pt}{141.387177pt}}
\pgflineto{\pgfpoint{235.816864pt}{141.070068pt}}
\pgfusepath{stroke}
\pgfpathmoveto{\pgfpoint{237.808838pt}{141.605896pt}}
\pgflineto{\pgfpoint{236.812866pt}{141.387177pt}}
\pgfusepath{stroke}
\pgfpathmoveto{\pgfpoint{238.804825pt}{141.892487pt}}
\pgflineto{\pgfpoint{237.808838pt}{141.605896pt}}
\pgfusepath{stroke}
\pgfpathmoveto{\pgfpoint{239.800812pt}{140.777435pt}}
\pgflineto{\pgfpoint{238.804825pt}{141.892487pt}}
\pgfusepath{stroke}
\pgfpathmoveto{\pgfpoint{240.796814pt}{140.777435pt}}
\pgflineto{\pgfpoint{241.792786pt}{140.777435pt}}
\pgfusepath{stroke}
\pgfpathmoveto{\pgfpoint{239.800812pt}{140.777435pt}}
\pgflineto{\pgfpoint{240.796814pt}{140.777435pt}}
\pgfusepath{stroke}
\pgfpathmoveto{\pgfpoint{240.796814pt}{141.635849pt}}
\pgflineto{\pgfpoint{239.800812pt}{140.777435pt}}
\pgfusepath{stroke}
\pgfpathmoveto{\pgfpoint{241.792786pt}{140.777435pt}}
\pgflineto{\pgfpoint{240.796814pt}{141.635849pt}}
\pgfusepath{stroke}
\pgfpathmoveto{\pgfpoint{243.784744pt}{140.777435pt}}
\pgflineto{\pgfpoint{244.780731pt}{140.777435pt}}
\pgfusepath{stroke}
\pgfpathmoveto{\pgfpoint{242.788757pt}{140.777435pt}}
\pgflineto{\pgfpoint{243.784744pt}{140.777435pt}}
\pgfusepath{stroke}
\pgfpathmoveto{\pgfpoint{241.792786pt}{140.777435pt}}
\pgflineto{\pgfpoint{242.788757pt}{140.777435pt}}
\pgfusepath{stroke}
\pgfpathmoveto{\pgfpoint{242.788757pt}{141.050537pt}}
\pgflineto{\pgfpoint{241.792786pt}{140.777435pt}}
\pgfusepath{stroke}
\pgfpathmoveto{\pgfpoint{243.784744pt}{140.826157pt}}
\pgflineto{\pgfpoint{242.788757pt}{141.050537pt}}
\pgfusepath{stroke}
\pgfpathmoveto{\pgfpoint{244.780731pt}{140.777435pt}}
\pgflineto{\pgfpoint{243.784744pt}{140.826157pt}}
\pgfusepath{stroke}
\pgfpathmoveto{\pgfpoint{250.756638pt}{140.777435pt}}
\pgflineto{\pgfpoint{251.752625pt}{140.777435pt}}
\pgfusepath{stroke}
\pgfpathmoveto{\pgfpoint{249.760651pt}{140.777435pt}}
\pgflineto{\pgfpoint{250.756638pt}{140.777435pt}}
\pgfusepath{stroke}
\pgfpathmoveto{\pgfpoint{248.764679pt}{140.777435pt}}
\pgflineto{\pgfpoint{249.760651pt}{140.777435pt}}
\pgfusepath{stroke}
\pgfpathmoveto{\pgfpoint{247.768677pt}{140.777435pt}}
\pgflineto{\pgfpoint{248.764679pt}{140.777435pt}}
\pgfusepath{stroke}
\pgfpathmoveto{\pgfpoint{246.772705pt}{140.777435pt}}
\pgflineto{\pgfpoint{247.768677pt}{140.777435pt}}
\pgfusepath{stroke}
\pgfpathmoveto{\pgfpoint{245.776718pt}{140.777435pt}}
\pgflineto{\pgfpoint{246.772705pt}{140.777435pt}}
\pgfusepath{stroke}
\pgfpathmoveto{\pgfpoint{244.780731pt}{140.777435pt}}
\pgflineto{\pgfpoint{245.776718pt}{140.777435pt}}
\pgfusepath{stroke}
\pgfpathmoveto{\pgfpoint{245.776718pt}{140.921082pt}}
\pgflineto{\pgfpoint{244.780731pt}{140.777435pt}}
\pgfusepath{stroke}
\pgfpathmoveto{\pgfpoint{246.230515pt}{205.642242pt}}
\pgflineto{\pgfpoint{245.776718pt}{140.921082pt}}
\pgfusepath{stroke}
\pgfpathmoveto{\pgfpoint{247.768677pt}{144.035980pt}}
\pgflineto{\pgfpoint{247.327026pt}{205.642242pt}}
\pgfusepath{stroke}
\pgfpathmoveto{\pgfpoint{248.764679pt}{140.987183pt}}
\pgflineto{\pgfpoint{247.768677pt}{144.035980pt}}
\pgfusepath{stroke}
\pgfpathmoveto{\pgfpoint{249.760651pt}{140.861908pt}}
\pgflineto{\pgfpoint{248.764679pt}{140.987183pt}}
\pgfusepath{stroke}
\pgfpathmoveto{\pgfpoint{250.756638pt}{143.997910pt}}
\pgflineto{\pgfpoint{249.760651pt}{140.861908pt}}
\pgfusepath{stroke}
\pgfpathmoveto{\pgfpoint{251.752625pt}{140.777435pt}}
\pgflineto{\pgfpoint{250.756638pt}{143.997910pt}}
\pgfusepath{stroke}
\pgfpathmoveto{\pgfpoint{254.740570pt}{140.777435pt}}
\pgflineto{\pgfpoint{255.736542pt}{140.777435pt}}
\pgfusepath{stroke}
\pgfpathmoveto{\pgfpoint{253.744598pt}{140.777435pt}}
\pgflineto{\pgfpoint{254.740570pt}{140.777435pt}}
\pgfusepath{stroke}
\pgfpathmoveto{\pgfpoint{252.748611pt}{140.777435pt}}
\pgflineto{\pgfpoint{253.744598pt}{140.777435pt}}
\pgfusepath{stroke}
\pgfpathmoveto{\pgfpoint{253.234100pt}{205.642242pt}}
\pgflineto{\pgfpoint{252.748611pt}{140.777435pt}}
\pgfusepath{stroke}
\pgfpathmoveto{\pgfpoint{255.736542pt}{140.777435pt}}
\pgflineto{\pgfpoint{255.155716pt}{205.642242pt}}
\pgfusepath{stroke}
\pgfpathmoveto{\pgfpoint{261.712463pt}{140.777435pt}}
\pgflineto{\pgfpoint{262.708435pt}{140.777435pt}}
\pgfusepath{stroke}
\pgfpathmoveto{\pgfpoint{260.716492pt}{140.777435pt}}
\pgflineto{\pgfpoint{261.712463pt}{140.777435pt}}
\pgfusepath{stroke}
\pgfpathmoveto{\pgfpoint{259.720490pt}{140.777435pt}}
\pgflineto{\pgfpoint{260.716492pt}{140.777435pt}}
\pgfusepath{stroke}
\pgfpathmoveto{\pgfpoint{258.724518pt}{140.777435pt}}
\pgflineto{\pgfpoint{259.720490pt}{140.777435pt}}
\pgfusepath{stroke}
\pgfpathmoveto{\pgfpoint{257.728516pt}{140.777435pt}}
\pgflineto{\pgfpoint{258.724518pt}{140.777435pt}}
\pgfusepath{stroke}
\pgfpathmoveto{\pgfpoint{256.732544pt}{140.777435pt}}
\pgflineto{\pgfpoint{257.728516pt}{140.777435pt}}
\pgfusepath{stroke}
\pgfpathmoveto{\pgfpoint{255.736542pt}{140.777435pt}}
\pgflineto{\pgfpoint{256.732544pt}{140.777435pt}}
\pgfusepath{stroke}
\pgfpathmoveto{\pgfpoint{256.732544pt}{160.055664pt}}
\pgflineto{\pgfpoint{255.736542pt}{140.777435pt}}
\pgfusepath{stroke}
\pgfpathmoveto{\pgfpoint{257.728516pt}{145.276688pt}}
\pgflineto{\pgfpoint{256.732544pt}{160.055664pt}}
\pgfusepath{stroke}
\pgfpathmoveto{\pgfpoint{258.724518pt}{145.143265pt}}
\pgflineto{\pgfpoint{257.728516pt}{145.276688pt}}
\pgfusepath{stroke}
\pgfpathmoveto{\pgfpoint{259.720490pt}{141.107025pt}}
\pgflineto{\pgfpoint{258.724518pt}{145.143265pt}}
\pgfusepath{stroke}
\pgfpathmoveto{\pgfpoint{260.716492pt}{140.785751pt}}
\pgflineto{\pgfpoint{259.720490pt}{141.107025pt}}
\pgfusepath{stroke}
\pgfpathmoveto{\pgfpoint{261.712463pt}{140.881470pt}}
\pgflineto{\pgfpoint{260.716492pt}{140.785751pt}}
\pgfusepath{stroke}
\pgfpathmoveto{\pgfpoint{262.708435pt}{140.777435pt}}
\pgflineto{\pgfpoint{261.712463pt}{140.881470pt}}
\pgfusepath{stroke}
\pgfpathmoveto{\pgfpoint{272.668274pt}{140.777435pt}}
\pgflineto{\pgfpoint{273.664276pt}{140.777435pt}}
\pgfusepath{stroke}
\pgfpathmoveto{\pgfpoint{271.672302pt}{140.777435pt}}
\pgflineto{\pgfpoint{272.668274pt}{140.777435pt}}
\pgfusepath{stroke}
\pgfpathmoveto{\pgfpoint{270.676331pt}{140.777435pt}}
\pgflineto{\pgfpoint{271.672302pt}{140.777435pt}}
\pgfusepath{stroke}
\pgfpathmoveto{\pgfpoint{269.680328pt}{140.777435pt}}
\pgflineto{\pgfpoint{270.676331pt}{140.777435pt}}
\pgfusepath{stroke}
\pgfpathmoveto{\pgfpoint{268.684326pt}{140.777435pt}}
\pgflineto{\pgfpoint{269.680328pt}{140.777435pt}}
\pgfusepath{stroke}
\pgfpathmoveto{\pgfpoint{267.688354pt}{140.777435pt}}
\pgflineto{\pgfpoint{268.684326pt}{140.777435pt}}
\pgfusepath{stroke}
\pgfpathmoveto{\pgfpoint{266.692383pt}{140.777435pt}}
\pgflineto{\pgfpoint{267.688354pt}{140.777435pt}}
\pgfusepath{stroke}
\pgfpathmoveto{\pgfpoint{265.696411pt}{140.777435pt}}
\pgflineto{\pgfpoint{266.692383pt}{140.777435pt}}
\pgfusepath{stroke}
\pgfpathmoveto{\pgfpoint{264.700409pt}{140.777435pt}}
\pgflineto{\pgfpoint{265.696411pt}{140.777435pt}}
\pgfusepath{stroke}
\pgfpathmoveto{\pgfpoint{263.704407pt}{140.777435pt}}
\pgflineto{\pgfpoint{264.700409pt}{140.777435pt}}
\pgfusepath{stroke}
\pgfpathmoveto{\pgfpoint{262.708435pt}{140.777435pt}}
\pgflineto{\pgfpoint{263.704407pt}{140.777435pt}}
\pgfusepath{stroke}
\pgfpathmoveto{\pgfpoint{263.704407pt}{149.672318pt}}
\pgflineto{\pgfpoint{262.708435pt}{140.777435pt}}
\pgfusepath{stroke}
\pgfpathmoveto{\pgfpoint{264.700409pt}{150.470245pt}}
\pgflineto{\pgfpoint{263.704407pt}{149.672318pt}}
\pgfusepath{stroke}
\pgfpathmoveto{\pgfpoint{265.696411pt}{146.772659pt}}
\pgflineto{\pgfpoint{264.700409pt}{150.470245pt}}
\pgfusepath{stroke}
\pgfpathmoveto{\pgfpoint{266.692383pt}{141.012543pt}}
\pgflineto{\pgfpoint{265.696411pt}{146.772659pt}}
\pgfusepath{stroke}
\pgfpathmoveto{\pgfpoint{267.688354pt}{140.947601pt}}
\pgflineto{\pgfpoint{266.692383pt}{141.012543pt}}
\pgfusepath{stroke}
\pgfpathmoveto{\pgfpoint{268.684326pt}{165.780823pt}}
\pgflineto{\pgfpoint{267.688354pt}{140.947601pt}}
\pgfusepath{stroke}
\pgfpathmoveto{\pgfpoint{269.680328pt}{142.178589pt}}
\pgflineto{\pgfpoint{268.684326pt}{165.780823pt}}
\pgfusepath{stroke}
\pgfpathmoveto{\pgfpoint{270.676331pt}{140.852936pt}}
\pgflineto{\pgfpoint{269.680328pt}{142.178589pt}}
\pgfusepath{stroke}
\pgfpathmoveto{\pgfpoint{271.672302pt}{149.853119pt}}
\pgflineto{\pgfpoint{270.676331pt}{140.852936pt}}
\pgfusepath{stroke}
\pgfpathmoveto{\pgfpoint{272.668274pt}{167.198486pt}}
\pgflineto{\pgfpoint{271.672302pt}{149.853119pt}}
\pgfusepath{stroke}
\pgfpathmoveto{\pgfpoint{273.664276pt}{140.777435pt}}
\pgflineto{\pgfpoint{272.668274pt}{167.198486pt}}
\pgfusepath{stroke}
\pgfpathmoveto{\pgfpoint{282.628113pt}{140.777435pt}}
\pgflineto{\pgfpoint{283.624115pt}{140.777435pt}}
\pgfusepath{stroke}
\pgfpathmoveto{\pgfpoint{281.632141pt}{140.777435pt}}
\pgflineto{\pgfpoint{282.628113pt}{140.777435pt}}
\pgfusepath{stroke}
\pgfpathmoveto{\pgfpoint{280.636139pt}{140.777435pt}}
\pgflineto{\pgfpoint{281.632141pt}{140.777435pt}}
\pgfusepath{stroke}
\pgfpathmoveto{\pgfpoint{279.640167pt}{140.777435pt}}
\pgflineto{\pgfpoint{280.636139pt}{140.777435pt}}
\pgfusepath{stroke}
\pgfpathmoveto{\pgfpoint{278.644196pt}{140.777435pt}}
\pgflineto{\pgfpoint{279.640167pt}{140.777435pt}}
\pgfusepath{stroke}
\pgfpathmoveto{\pgfpoint{277.648193pt}{140.777435pt}}
\pgflineto{\pgfpoint{278.644196pt}{140.777435pt}}
\pgfusepath{stroke}
\pgfpathmoveto{\pgfpoint{276.652222pt}{140.777435pt}}
\pgflineto{\pgfpoint{277.648193pt}{140.777435pt}}
\pgfusepath{stroke}
\pgfpathmoveto{\pgfpoint{275.656250pt}{140.777435pt}}
\pgflineto{\pgfpoint{276.652222pt}{140.777435pt}}
\pgfusepath{stroke}
\pgfpathmoveto{\pgfpoint{276.652222pt}{141.883286pt}}
\pgflineto{\pgfpoint{275.656250pt}{140.777435pt}}
\pgfusepath{stroke}
\pgfpathmoveto{\pgfpoint{277.648193pt}{145.700470pt}}
\pgflineto{\pgfpoint{276.652222pt}{141.883286pt}}
\pgfusepath{stroke}
\pgfpathmoveto{\pgfpoint{278.644196pt}{141.350510pt}}
\pgflineto{\pgfpoint{277.648193pt}{145.700470pt}}
\pgfusepath{stroke}
\pgfpathmoveto{\pgfpoint{279.640167pt}{141.685425pt}}
\pgflineto{\pgfpoint{278.644196pt}{141.350510pt}}
\pgfusepath{stroke}
\pgfpathmoveto{\pgfpoint{280.636139pt}{140.839111pt}}
\pgflineto{\pgfpoint{279.640167pt}{141.685425pt}}
\pgfusepath{stroke}
\pgfpathmoveto{\pgfpoint{281.632141pt}{140.783401pt}}
\pgflineto{\pgfpoint{280.636139pt}{140.839111pt}}
\pgfusepath{stroke}
\pgfpathmoveto{\pgfpoint{282.628113pt}{141.247986pt}}
\pgflineto{\pgfpoint{281.632141pt}{140.783401pt}}
\pgfusepath{stroke}
\pgfpathmoveto{\pgfpoint{283.624115pt}{140.777435pt}}
\pgflineto{\pgfpoint{282.628113pt}{141.247986pt}}
\pgfusepath{stroke}
\pgfpathmoveto{\pgfpoint{285.616089pt}{140.777435pt}}
\pgflineto{\pgfpoint{286.612061pt}{140.777435pt}}
\pgfusepath{stroke}
\pgfpathmoveto{\pgfpoint{284.620087pt}{140.777435pt}}
\pgflineto{\pgfpoint{285.616089pt}{140.777435pt}}
\pgfusepath{stroke}
\pgfpathmoveto{\pgfpoint{283.624115pt}{140.777435pt}}
\pgflineto{\pgfpoint{284.620087pt}{140.777435pt}}
\pgfusepath{stroke}
\pgfpathmoveto{\pgfpoint{284.620087pt}{140.819778pt}}
\pgflineto{\pgfpoint{283.624115pt}{140.777435pt}}
\pgfusepath{stroke}
\pgfpathmoveto{\pgfpoint{285.616089pt}{141.090790pt}}
\pgflineto{\pgfpoint{284.620087pt}{140.819778pt}}
\pgfusepath{stroke}
\pgfpathmoveto{\pgfpoint{286.612061pt}{140.777435pt}}
\pgflineto{\pgfpoint{285.616089pt}{141.090790pt}}
\pgfusepath{stroke}
\pgfpathmoveto{\pgfpoint{288.604004pt}{140.777435pt}}
\pgflineto{\pgfpoint{289.600037pt}{140.777435pt}}
\pgfusepath{stroke}
\pgfpathmoveto{\pgfpoint{287.608032pt}{140.777435pt}}
\pgflineto{\pgfpoint{288.604004pt}{140.777435pt}}
\pgfusepath{stroke}
\pgfpathmoveto{\pgfpoint{288.604004pt}{142.317886pt}}
\pgflineto{\pgfpoint{287.608032pt}{140.777435pt}}
\pgfusepath{stroke}
\pgfpathmoveto{\pgfpoint{289.600037pt}{141.234161pt}}
\pgflineto{\pgfpoint{288.604004pt}{142.317886pt}}
\pgfusepath{stroke}
\pgfpathmoveto{\pgfpoint{289.600037pt}{140.777435pt}}
\pgflineto{\pgfpoint{289.600037pt}{141.234161pt}}
\pgfusepath{stroke}
\color[rgb]{0.000000,0.000000,1.000000}
\pgfsetlinewidth{2.000000pt}
\pgfpathmoveto{\pgfpoint{42.595993pt}{140.777435pt}}
\pgflineto{\pgfpoint{41.600006pt}{140.777435pt}}
\pgfusepath{stroke}
\pgfpathmoveto{\pgfpoint{43.591980pt}{142.291229pt}}
\pgflineto{\pgfpoint{42.595993pt}{140.777435pt}}
\pgfusepath{stroke}
\pgfpathmoveto{\pgfpoint{44.587967pt}{140.777435pt}}
\pgflineto{\pgfpoint{43.591980pt}{142.291229pt}}
\pgfusepath{stroke}
\pgfpathmoveto{\pgfpoint{45.583946pt}{140.910431pt}}
\pgflineto{\pgfpoint{44.587967pt}{140.777435pt}}
\pgfusepath{stroke}
\pgfpathmoveto{\pgfpoint{46.579933pt}{140.777435pt}}
\pgflineto{\pgfpoint{45.583946pt}{140.910431pt}}
\pgfusepath{stroke}
\pgfpathmoveto{\pgfpoint{47.575912pt}{140.789795pt}}
\pgflineto{\pgfpoint{46.579933pt}{140.777435pt}}
\pgfusepath{stroke}
\pgfpathmoveto{\pgfpoint{48.571899pt}{140.788818pt}}
\pgflineto{\pgfpoint{47.575912pt}{140.789795pt}}
\pgfusepath{stroke}
\pgfpathmoveto{\pgfpoint{49.567879pt}{140.835007pt}}
\pgflineto{\pgfpoint{48.571899pt}{140.788818pt}}
\pgfusepath{stroke}
\pgfpathmoveto{\pgfpoint{50.563873pt}{140.777435pt}}
\pgflineto{\pgfpoint{49.567879pt}{140.835007pt}}
\pgfusepath{stroke}
\pgfpathmoveto{\pgfpoint{51.559845pt}{140.826508pt}}
\pgflineto{\pgfpoint{50.563873pt}{140.777435pt}}
\pgfusepath{stroke}
\pgfpathmoveto{\pgfpoint{52.555840pt}{140.814529pt}}
\pgflineto{\pgfpoint{51.559845pt}{140.826508pt}}
\pgfusepath{stroke}
\pgfpathmoveto{\pgfpoint{53.551819pt}{141.029465pt}}
\pgflineto{\pgfpoint{52.555840pt}{140.814529pt}}
\pgfusepath{stroke}
\pgfpathmoveto{\pgfpoint{54.547806pt}{140.800766pt}}
\pgflineto{\pgfpoint{53.551819pt}{141.029465pt}}
\pgfusepath{stroke}
\pgfpathmoveto{\pgfpoint{55.543785pt}{141.743744pt}}
\pgflineto{\pgfpoint{54.547806pt}{140.800766pt}}
\pgfusepath{stroke}
\pgfpathmoveto{\pgfpoint{56.539772pt}{143.031479pt}}
\pgflineto{\pgfpoint{55.543785pt}{141.743744pt}}
\pgfusepath{stroke}
\pgfpathmoveto{\pgfpoint{57.535751pt}{140.964340pt}}
\pgflineto{\pgfpoint{56.539772pt}{143.031479pt}}
\pgfusepath{stroke}
\pgfpathmoveto{\pgfpoint{58.531738pt}{140.780151pt}}
\pgflineto{\pgfpoint{57.535751pt}{140.964340pt}}
\pgfusepath{stroke}
\pgfpathmoveto{\pgfpoint{59.527725pt}{140.855103pt}}
\pgflineto{\pgfpoint{58.531738pt}{140.780151pt}}
\pgfusepath{stroke}
\pgfpathmoveto{\pgfpoint{60.523712pt}{140.787338pt}}
\pgflineto{\pgfpoint{59.527725pt}{140.855103pt}}
\pgfusepath{stroke}
\pgfpathmoveto{\pgfpoint{61.519691pt}{140.782700pt}}
\pgflineto{\pgfpoint{60.523712pt}{140.787338pt}}
\pgfusepath{stroke}
\pgfpathmoveto{\pgfpoint{62.515678pt}{148.013794pt}}
\pgflineto{\pgfpoint{61.519691pt}{140.782700pt}}
\pgfusepath{stroke}
\pgfpathmoveto{\pgfpoint{63.511658pt}{140.808426pt}}
\pgflineto{\pgfpoint{62.515678pt}{148.013794pt}}
\pgfusepath{stroke}
\pgfpathmoveto{\pgfpoint{64.507637pt}{140.848022pt}}
\pgflineto{\pgfpoint{63.511658pt}{140.808426pt}}
\pgfusepath{stroke}
\pgfpathmoveto{\pgfpoint{65.503624pt}{140.813904pt}}
\pgflineto{\pgfpoint{64.507637pt}{140.848022pt}}
\pgfusepath{stroke}
\pgfpathmoveto{\pgfpoint{66.499619pt}{140.822861pt}}
\pgflineto{\pgfpoint{65.503624pt}{140.813904pt}}
\pgfusepath{stroke}
\pgfpathmoveto{\pgfpoint{67.495590pt}{140.777435pt}}
\pgflineto{\pgfpoint{66.499619pt}{140.822861pt}}
\pgfusepath{stroke}
\pgfpathmoveto{\pgfpoint{68.491577pt}{140.777435pt}}
\pgflineto{\pgfpoint{67.495590pt}{140.777435pt}}
\pgfusepath{stroke}
\pgfpathmoveto{\pgfpoint{69.487564pt}{140.957489pt}}
\pgflineto{\pgfpoint{68.491577pt}{140.777435pt}}
\pgfusepath{stroke}
\pgfpathmoveto{\pgfpoint{70.483551pt}{140.814529pt}}
\pgflineto{\pgfpoint{69.487564pt}{140.957489pt}}
\pgfusepath{stroke}
\pgfpathmoveto{\pgfpoint{71.479530pt}{141.474945pt}}
\pgflineto{\pgfpoint{70.483551pt}{140.814529pt}}
\pgfusepath{stroke}
\pgfpathmoveto{\pgfpoint{72.475510pt}{140.931519pt}}
\pgflineto{\pgfpoint{71.479530pt}{141.474945pt}}
\pgfusepath{stroke}
\pgfpathmoveto{\pgfpoint{73.471497pt}{140.777435pt}}
\pgflineto{\pgfpoint{72.475510pt}{140.931519pt}}
\pgfusepath{stroke}
\pgfpathmoveto{\pgfpoint{74.467484pt}{140.789795pt}}
\pgflineto{\pgfpoint{73.471497pt}{140.777435pt}}
\pgfusepath{stroke}
\pgfpathmoveto{\pgfpoint{75.463470pt}{140.988373pt}}
\pgflineto{\pgfpoint{74.467484pt}{140.789795pt}}
\pgfusepath{stroke}
\pgfpathmoveto{\pgfpoint{76.459442pt}{140.912750pt}}
\pgflineto{\pgfpoint{75.463470pt}{140.988373pt}}
\pgfusepath{stroke}
\pgfpathmoveto{\pgfpoint{77.455437pt}{140.810455pt}}
\pgflineto{\pgfpoint{76.459442pt}{140.912750pt}}
\pgfusepath{stroke}
\pgfpathmoveto{\pgfpoint{78.451424pt}{140.777435pt}}
\pgflineto{\pgfpoint{77.455437pt}{140.810455pt}}
\pgfusepath{stroke}
\pgfpathmoveto{\pgfpoint{79.447403pt}{142.779175pt}}
\pgflineto{\pgfpoint{78.451424pt}{140.777435pt}}
\pgfusepath{stroke}
\pgfpathmoveto{\pgfpoint{80.443390pt}{140.781250pt}}
\pgflineto{\pgfpoint{79.447403pt}{142.779175pt}}
\pgfusepath{stroke}
\pgfpathmoveto{\pgfpoint{81.439369pt}{140.786606pt}}
\pgflineto{\pgfpoint{80.443390pt}{140.781250pt}}
\pgfusepath{stroke}
\pgfpathmoveto{\pgfpoint{82.435356pt}{140.785400pt}}
\pgflineto{\pgfpoint{81.439369pt}{140.786606pt}}
\pgfusepath{stroke}
\pgfpathmoveto{\pgfpoint{83.431335pt}{141.342422pt}}
\pgflineto{\pgfpoint{82.435356pt}{140.785400pt}}
\pgfusepath{stroke}
\pgfpathmoveto{\pgfpoint{84.427322pt}{140.777969pt}}
\pgflineto{\pgfpoint{83.431335pt}{141.342422pt}}
\pgfusepath{stroke}
\pgfpathmoveto{\pgfpoint{85.423309pt}{140.779861pt}}
\pgflineto{\pgfpoint{84.427322pt}{140.777969pt}}
\pgfusepath{stroke}
\pgfpathmoveto{\pgfpoint{86.419289pt}{140.777435pt}}
\pgflineto{\pgfpoint{85.423309pt}{140.779861pt}}
\pgfusepath{stroke}
\pgfpathmoveto{\pgfpoint{87.415276pt}{141.062363pt}}
\pgflineto{\pgfpoint{86.419289pt}{140.777435pt}}
\pgfusepath{stroke}
\pgfpathmoveto{\pgfpoint{88.411255pt}{141.826538pt}}
\pgflineto{\pgfpoint{87.415276pt}{141.062363pt}}
\pgfusepath{stroke}
\pgfpathmoveto{\pgfpoint{89.407242pt}{141.208725pt}}
\pgflineto{\pgfpoint{88.411255pt}{141.826538pt}}
\pgfusepath{stroke}
\pgfpathmoveto{\pgfpoint{90.403221pt}{141.961151pt}}
\pgflineto{\pgfpoint{89.407242pt}{141.208725pt}}
\pgfusepath{stroke}
\pgfpathmoveto{\pgfpoint{91.399208pt}{140.777435pt}}
\pgflineto{\pgfpoint{90.403221pt}{141.961151pt}}
\pgfusepath{stroke}
\pgfpathmoveto{\pgfpoint{92.395187pt}{140.777527pt}}
\pgflineto{\pgfpoint{91.399208pt}{140.777435pt}}
\pgfusepath{stroke}
\pgfpathmoveto{\pgfpoint{93.391174pt}{140.785782pt}}
\pgflineto{\pgfpoint{92.395187pt}{140.777527pt}}
\pgfusepath{stroke}
\pgfpathmoveto{\pgfpoint{94.387161pt}{140.777435pt}}
\pgflineto{\pgfpoint{93.391174pt}{140.785782pt}}
\pgfusepath{stroke}
\pgfpathmoveto{\pgfpoint{95.383141pt}{140.780762pt}}
\pgflineto{\pgfpoint{94.387161pt}{140.777435pt}}
\pgfusepath{stroke}
\pgfpathmoveto{\pgfpoint{96.379128pt}{140.781006pt}}
\pgflineto{\pgfpoint{95.383141pt}{140.780762pt}}
\pgfusepath{stroke}
\pgfpathmoveto{\pgfpoint{97.375107pt}{140.790512pt}}
\pgflineto{\pgfpoint{96.379128pt}{140.781006pt}}
\pgfusepath{stroke}
\pgfpathmoveto{\pgfpoint{98.371094pt}{141.396454pt}}
\pgflineto{\pgfpoint{97.375107pt}{140.790512pt}}
\pgfusepath{stroke}
\pgfpathmoveto{\pgfpoint{99.367081pt}{140.828232pt}}
\pgflineto{\pgfpoint{98.371094pt}{141.396454pt}}
\pgfusepath{stroke}
\pgfpathmoveto{\pgfpoint{100.363068pt}{140.788437pt}}
\pgflineto{\pgfpoint{99.367081pt}{140.828232pt}}
\pgfusepath{stroke}
\pgfpathmoveto{\pgfpoint{101.359047pt}{140.782562pt}}
\pgflineto{\pgfpoint{100.363068pt}{140.788437pt}}
\pgfusepath{stroke}
\pgfpathmoveto{\pgfpoint{102.355034pt}{140.893219pt}}
\pgflineto{\pgfpoint{101.359047pt}{140.782562pt}}
\pgfusepath{stroke}
\pgfpathmoveto{\pgfpoint{103.351013pt}{140.777435pt}}
\pgflineto{\pgfpoint{102.355034pt}{140.893219pt}}
\pgfusepath{stroke}
\pgfpathmoveto{\pgfpoint{104.347000pt}{141.194229pt}}
\pgflineto{\pgfpoint{103.351013pt}{140.777435pt}}
\pgfusepath{stroke}
\pgfpathmoveto{\pgfpoint{105.342987pt}{140.777435pt}}
\pgflineto{\pgfpoint{104.347000pt}{141.194229pt}}
\pgfusepath{stroke}
\pgfpathmoveto{\pgfpoint{106.338966pt}{140.777435pt}}
\pgflineto{\pgfpoint{105.342987pt}{140.777435pt}}
\pgfusepath{stroke}
\pgfpathmoveto{\pgfpoint{107.334953pt}{140.777435pt}}
\pgflineto{\pgfpoint{106.338966pt}{140.777435pt}}
\pgfusepath{stroke}
\pgfpathmoveto{\pgfpoint{108.330933pt}{141.164261pt}}
\pgflineto{\pgfpoint{107.334953pt}{140.777435pt}}
\pgfusepath{stroke}
\pgfpathmoveto{\pgfpoint{109.326920pt}{140.777985pt}}
\pgflineto{\pgfpoint{108.330933pt}{141.164261pt}}
\pgfusepath{stroke}
\pgfpathmoveto{\pgfpoint{110.322906pt}{140.811081pt}}
\pgflineto{\pgfpoint{109.326920pt}{140.777985pt}}
\pgfusepath{stroke}
\pgfpathmoveto{\pgfpoint{111.318893pt}{140.777435pt}}
\pgflineto{\pgfpoint{110.322906pt}{140.811081pt}}
\pgfusepath{stroke}
\pgfpathmoveto{\pgfpoint{112.314873pt}{140.798782pt}}
\pgflineto{\pgfpoint{111.318893pt}{140.777435pt}}
\pgfusepath{stroke}
\pgfpathmoveto{\pgfpoint{113.310852pt}{140.789963pt}}
\pgflineto{\pgfpoint{112.314873pt}{140.798782pt}}
\pgfusepath{stroke}
\pgfpathmoveto{\pgfpoint{114.306839pt}{140.825577pt}}
\pgflineto{\pgfpoint{113.310852pt}{140.789963pt}}
\pgfusepath{stroke}
\pgfpathmoveto{\pgfpoint{115.302826pt}{140.777435pt}}
\pgflineto{\pgfpoint{114.306839pt}{140.825577pt}}
\pgfusepath{stroke}
\pgfpathmoveto{\pgfpoint{116.298813pt}{140.784241pt}}
\pgflineto{\pgfpoint{115.302826pt}{140.777435pt}}
\pgfusepath{stroke}
\pgfpathmoveto{\pgfpoint{117.294792pt}{140.853699pt}}
\pgflineto{\pgfpoint{116.298813pt}{140.784241pt}}
\pgfusepath{stroke}
\pgfpathmoveto{\pgfpoint{118.290779pt}{140.778458pt}}
\pgflineto{\pgfpoint{117.294792pt}{140.853699pt}}
\pgfusepath{stroke}
\pgfpathmoveto{\pgfpoint{119.286758pt}{140.778992pt}}
\pgflineto{\pgfpoint{118.290779pt}{140.778458pt}}
\pgfusepath{stroke}
\pgfpathmoveto{\pgfpoint{120.282745pt}{140.777435pt}}
\pgflineto{\pgfpoint{119.286758pt}{140.778992pt}}
\pgfusepath{stroke}
\pgfpathmoveto{\pgfpoint{121.278725pt}{140.821732pt}}
\pgflineto{\pgfpoint{120.282745pt}{140.777435pt}}
\pgfusepath{stroke}
\pgfpathmoveto{\pgfpoint{122.274712pt}{140.777435pt}}
\pgflineto{\pgfpoint{121.278725pt}{140.821732pt}}
\pgfusepath{stroke}
\pgfpathmoveto{\pgfpoint{123.270691pt}{140.781097pt}}
\pgflineto{\pgfpoint{122.274712pt}{140.777435pt}}
\pgfusepath{stroke}
\pgfpathmoveto{\pgfpoint{124.266678pt}{140.777435pt}}
\pgflineto{\pgfpoint{123.270691pt}{140.781097pt}}
\pgfusepath{stroke}
\pgfpathmoveto{\pgfpoint{125.262665pt}{140.777435pt}}
\pgflineto{\pgfpoint{124.266678pt}{140.777435pt}}
\pgfusepath{stroke}
\pgfpathmoveto{\pgfpoint{126.258652pt}{140.787552pt}}
\pgflineto{\pgfpoint{125.262665pt}{140.777435pt}}
\pgfusepath{stroke}
\pgfpathmoveto{\pgfpoint{127.254631pt}{142.211929pt}}
\pgflineto{\pgfpoint{126.258652pt}{140.787552pt}}
\pgfusepath{stroke}
\pgfpathmoveto{\pgfpoint{128.250610pt}{140.777435pt}}
\pgflineto{\pgfpoint{127.254631pt}{142.211929pt}}
\pgfusepath{stroke}
\pgfpathmoveto{\pgfpoint{129.246597pt}{140.893631pt}}
\pgflineto{\pgfpoint{128.250610pt}{140.777435pt}}
\pgfusepath{stroke}
\pgfpathmoveto{\pgfpoint{130.242584pt}{140.790909pt}}
\pgflineto{\pgfpoint{129.246597pt}{140.893631pt}}
\pgfusepath{stroke}
\pgfpathmoveto{\pgfpoint{131.238571pt}{140.817093pt}}
\pgflineto{\pgfpoint{130.242584pt}{140.790909pt}}
\pgfusepath{stroke}
\pgfpathmoveto{\pgfpoint{132.234558pt}{140.777496pt}}
\pgflineto{\pgfpoint{131.238571pt}{140.817093pt}}
\pgfusepath{stroke}
\pgfpathmoveto{\pgfpoint{133.230530pt}{140.777435pt}}
\pgflineto{\pgfpoint{132.234558pt}{140.777496pt}}
\pgfusepath{stroke}
\pgfpathmoveto{\pgfpoint{134.226517pt}{140.777435pt}}
\pgflineto{\pgfpoint{133.230530pt}{140.777435pt}}
\pgfusepath{stroke}
\pgfpathmoveto{\pgfpoint{135.222504pt}{140.782410pt}}
\pgflineto{\pgfpoint{134.226517pt}{140.777435pt}}
\pgfusepath{stroke}
\pgfpathmoveto{\pgfpoint{136.218475pt}{140.777435pt}}
\pgflineto{\pgfpoint{135.222504pt}{140.782410pt}}
\pgfusepath{stroke}
\pgfpathmoveto{\pgfpoint{137.214478pt}{140.868332pt}}
\pgflineto{\pgfpoint{136.218475pt}{140.777435pt}}
\pgfusepath{stroke}
\pgfpathmoveto{\pgfpoint{138.210449pt}{140.777908pt}}
\pgflineto{\pgfpoint{137.214478pt}{140.868332pt}}
\pgfusepath{stroke}
\pgfpathmoveto{\pgfpoint{139.206436pt}{141.098373pt}}
\pgflineto{\pgfpoint{138.210449pt}{140.777908pt}}
\pgfusepath{stroke}
\pgfpathmoveto{\pgfpoint{140.202423pt}{140.810928pt}}
\pgflineto{\pgfpoint{139.206436pt}{141.098373pt}}
\pgfusepath{stroke}
\pgfpathmoveto{\pgfpoint{141.198410pt}{140.779694pt}}
\pgflineto{\pgfpoint{140.202423pt}{140.810928pt}}
\pgfusepath{stroke}
\pgfpathmoveto{\pgfpoint{142.194382pt}{140.777435pt}}
\pgflineto{\pgfpoint{141.198410pt}{140.779694pt}}
\pgfusepath{stroke}
\pgfpathmoveto{\pgfpoint{143.190369pt}{140.818817pt}}
\pgflineto{\pgfpoint{142.194382pt}{140.777435pt}}
\pgfusepath{stroke}
\pgfpathmoveto{\pgfpoint{144.186356pt}{140.777435pt}}
\pgflineto{\pgfpoint{143.190369pt}{140.818817pt}}
\pgfusepath{stroke}
\pgfpathmoveto{\pgfpoint{145.182343pt}{140.783417pt}}
\pgflineto{\pgfpoint{144.186356pt}{140.777435pt}}
\pgfusepath{stroke}
\pgfpathmoveto{\pgfpoint{146.178314pt}{140.779510pt}}
\pgflineto{\pgfpoint{145.182343pt}{140.783417pt}}
\pgfusepath{stroke}
\pgfpathmoveto{\pgfpoint{147.174316pt}{140.817200pt}}
\pgflineto{\pgfpoint{146.178314pt}{140.779510pt}}
\pgfusepath{stroke}
\pgfpathmoveto{\pgfpoint{148.170288pt}{140.777435pt}}
\pgflineto{\pgfpoint{147.174316pt}{140.817200pt}}
\pgfusepath{stroke}
\pgfpathmoveto{\pgfpoint{149.166275pt}{140.777435pt}}
\pgflineto{\pgfpoint{148.170288pt}{140.777435pt}}
\pgfusepath{stroke}
\pgfpathmoveto{\pgfpoint{150.162262pt}{140.777435pt}}
\pgflineto{\pgfpoint{149.166275pt}{140.777435pt}}
\pgfusepath{stroke}
\pgfpathmoveto{\pgfpoint{151.158249pt}{140.784195pt}}
\pgflineto{\pgfpoint{150.162262pt}{140.777435pt}}
\pgfusepath{stroke}
\pgfpathmoveto{\pgfpoint{152.154221pt}{140.816559pt}}
\pgflineto{\pgfpoint{151.158249pt}{140.784195pt}}
\pgfusepath{stroke}
\pgfpathmoveto{\pgfpoint{153.150208pt}{140.826141pt}}
\pgflineto{\pgfpoint{152.154221pt}{140.816559pt}}
\pgfusepath{stroke}
\pgfpathmoveto{\pgfpoint{154.146194pt}{140.777435pt}}
\pgflineto{\pgfpoint{153.150208pt}{140.826141pt}}
\pgfusepath{stroke}
\pgfpathmoveto{\pgfpoint{155.142181pt}{145.339951pt}}
\pgflineto{\pgfpoint{154.146194pt}{140.777435pt}}
\pgfusepath{stroke}
\pgfpathmoveto{\pgfpoint{156.138168pt}{140.777435pt}}
\pgflineto{\pgfpoint{155.142181pt}{145.339951pt}}
\pgfusepath{stroke}
\pgfpathmoveto{\pgfpoint{157.134155pt}{140.779984pt}}
\pgflineto{\pgfpoint{156.138168pt}{140.777435pt}}
\pgfusepath{stroke}
\pgfpathmoveto{\pgfpoint{158.130127pt}{140.780472pt}}
\pgflineto{\pgfpoint{157.134155pt}{140.779984pt}}
\pgfusepath{stroke}
\pgfpathmoveto{\pgfpoint{159.126114pt}{140.777435pt}}
\pgflineto{\pgfpoint{158.130127pt}{140.780472pt}}
\pgfusepath{stroke}
\pgfpathmoveto{\pgfpoint{160.122101pt}{140.912704pt}}
\pgflineto{\pgfpoint{159.126114pt}{140.777435pt}}
\pgfusepath{stroke}
\pgfpathmoveto{\pgfpoint{161.118088pt}{140.777557pt}}
\pgflineto{\pgfpoint{160.122101pt}{140.912704pt}}
\pgfusepath{stroke}
\pgfpathmoveto{\pgfpoint{162.114075pt}{140.788925pt}}
\pgflineto{\pgfpoint{161.118088pt}{140.777557pt}}
\pgfusepath{stroke}
\pgfpathmoveto{\pgfpoint{163.110062pt}{140.777435pt}}
\pgflineto{\pgfpoint{162.114075pt}{140.788925pt}}
\pgfusepath{stroke}
\pgfpathmoveto{\pgfpoint{164.106033pt}{141.622849pt}}
\pgflineto{\pgfpoint{163.110062pt}{140.777435pt}}
\pgfusepath{stroke}
\pgfpathmoveto{\pgfpoint{165.102020pt}{140.777802pt}}
\pgflineto{\pgfpoint{164.106033pt}{141.622849pt}}
\pgfusepath{stroke}
\pgfpathmoveto{\pgfpoint{166.098007pt}{140.777435pt}}
\pgflineto{\pgfpoint{165.102020pt}{140.777802pt}}
\pgfusepath{stroke}
\pgfpathmoveto{\pgfpoint{167.093994pt}{140.785172pt}}
\pgflineto{\pgfpoint{166.098007pt}{140.777435pt}}
\pgfusepath{stroke}
\pgfpathmoveto{\pgfpoint{168.089966pt}{165.120728pt}}
\pgflineto{\pgfpoint{167.093994pt}{140.785172pt}}
\pgfusepath{stroke}
\pgfpathmoveto{\pgfpoint{169.085953pt}{140.777435pt}}
\pgflineto{\pgfpoint{168.089966pt}{165.120728pt}}
\pgfusepath{stroke}
\pgfpathmoveto{\pgfpoint{170.081940pt}{141.018890pt}}
\pgflineto{\pgfpoint{169.085953pt}{140.777435pt}}
\pgfusepath{stroke}
\pgfpathmoveto{\pgfpoint{171.077911pt}{140.882141pt}}
\pgflineto{\pgfpoint{170.081940pt}{141.018890pt}}
\pgfusepath{stroke}
\pgfpathmoveto{\pgfpoint{172.073914pt}{141.041473pt}}
\pgflineto{\pgfpoint{171.077911pt}{140.882141pt}}
\pgfusepath{stroke}
\pgfpathmoveto{\pgfpoint{173.069885pt}{140.914948pt}}
\pgflineto{\pgfpoint{172.073914pt}{141.041473pt}}
\pgfusepath{stroke}
\pgfpathmoveto{\pgfpoint{174.065872pt}{140.782669pt}}
\pgflineto{\pgfpoint{173.069885pt}{140.914948pt}}
\pgfusepath{stroke}
\pgfpathmoveto{\pgfpoint{175.061859pt}{140.783112pt}}
\pgflineto{\pgfpoint{174.065872pt}{140.782669pt}}
\pgfusepath{stroke}
\pgfpathmoveto{\pgfpoint{176.057846pt}{140.777435pt}}
\pgflineto{\pgfpoint{175.061859pt}{140.783112pt}}
\pgfusepath{stroke}
\pgfpathmoveto{\pgfpoint{177.053818pt}{140.905045pt}}
\pgflineto{\pgfpoint{176.057846pt}{140.777435pt}}
\pgfusepath{stroke}
\pgfpathmoveto{\pgfpoint{178.049805pt}{140.866837pt}}
\pgflineto{\pgfpoint{177.053818pt}{140.905045pt}}
\pgfusepath{stroke}
\pgfpathmoveto{\pgfpoint{179.045792pt}{140.781509pt}}
\pgflineto{\pgfpoint{178.049805pt}{140.866837pt}}
\pgfusepath{stroke}
\pgfpathmoveto{\pgfpoint{180.041779pt}{140.779846pt}}
\pgflineto{\pgfpoint{179.045792pt}{140.781509pt}}
\pgfusepath{stroke}
\pgfpathmoveto{\pgfpoint{181.037766pt}{140.783401pt}}
\pgflineto{\pgfpoint{180.041779pt}{140.779846pt}}
\pgfusepath{stroke}
\pgfpathmoveto{\pgfpoint{182.033752pt}{140.777435pt}}
\pgflineto{\pgfpoint{181.037766pt}{140.783401pt}}
\pgfusepath{stroke}
\pgfpathmoveto{\pgfpoint{183.029724pt}{140.817596pt}}
\pgflineto{\pgfpoint{182.033752pt}{140.777435pt}}
\pgfusepath{stroke}
\pgfpathmoveto{\pgfpoint{184.025711pt}{140.777435pt}}
\pgflineto{\pgfpoint{183.029724pt}{140.817596pt}}
\pgfusepath{stroke}
\pgfpathmoveto{\pgfpoint{185.021698pt}{140.777435pt}}
\pgflineto{\pgfpoint{184.025711pt}{140.777435pt}}
\pgfusepath{stroke}
\pgfpathmoveto{\pgfpoint{186.017685pt}{140.777435pt}}
\pgflineto{\pgfpoint{185.021698pt}{140.777435pt}}
\pgfusepath{stroke}
\pgfpathmoveto{\pgfpoint{187.013672pt}{140.781067pt}}
\pgflineto{\pgfpoint{186.017685pt}{140.777435pt}}
\pgfusepath{stroke}
\pgfpathmoveto{\pgfpoint{188.009659pt}{140.777435pt}}
\pgflineto{\pgfpoint{187.013672pt}{140.781067pt}}
\pgfusepath{stroke}
\pgfpathmoveto{\pgfpoint{189.005630pt}{140.777634pt}}
\pgflineto{\pgfpoint{188.009659pt}{140.777435pt}}
\pgfusepath{stroke}
\pgfpathmoveto{\pgfpoint{190.001617pt}{141.032303pt}}
\pgflineto{\pgfpoint{189.005630pt}{140.777634pt}}
\pgfusepath{stroke}
\pgfpathmoveto{\pgfpoint{190.997604pt}{140.778046pt}}
\pgflineto{\pgfpoint{190.001617pt}{141.032303pt}}
\pgfusepath{stroke}
\pgfpathmoveto{\pgfpoint{191.993591pt}{140.805786pt}}
\pgflineto{\pgfpoint{190.997604pt}{140.778046pt}}
\pgfusepath{stroke}
\pgfpathmoveto{\pgfpoint{192.989563pt}{141.541718pt}}
\pgflineto{\pgfpoint{191.993591pt}{140.805786pt}}
\pgfusepath{stroke}
\pgfpathmoveto{\pgfpoint{193.985565pt}{140.790375pt}}
\pgflineto{\pgfpoint{192.989563pt}{141.541718pt}}
\pgfusepath{stroke}
\pgfpathmoveto{\pgfpoint{194.981537pt}{140.778259pt}}
\pgflineto{\pgfpoint{193.985565pt}{140.790375pt}}
\pgfusepath{stroke}
\pgfpathmoveto{\pgfpoint{195.977524pt}{141.314301pt}}
\pgflineto{\pgfpoint{194.981537pt}{140.778259pt}}
\pgfusepath{stroke}
\pgfpathmoveto{\pgfpoint{196.973511pt}{140.944916pt}}
\pgflineto{\pgfpoint{195.977524pt}{141.314301pt}}
\pgfusepath{stroke}
\pgfpathmoveto{\pgfpoint{197.969498pt}{140.809601pt}}
\pgflineto{\pgfpoint{196.973511pt}{140.944916pt}}
\pgfusepath{stroke}
\pgfpathmoveto{\pgfpoint{198.965469pt}{149.605469pt}}
\pgflineto{\pgfpoint{197.969498pt}{140.809601pt}}
\pgfusepath{stroke}
\pgfpathmoveto{\pgfpoint{199.961456pt}{140.777435pt}}
\pgflineto{\pgfpoint{198.965469pt}{149.605469pt}}
\pgfusepath{stroke}
\pgfpathmoveto{\pgfpoint{200.957443pt}{140.786819pt}}
\pgflineto{\pgfpoint{199.961456pt}{140.777435pt}}
\pgfusepath{stroke}
\pgfpathmoveto{\pgfpoint{201.953430pt}{140.778427pt}}
\pgflineto{\pgfpoint{200.957443pt}{140.786819pt}}
\pgfusepath{stroke}
\pgfpathmoveto{\pgfpoint{202.949402pt}{140.777435pt}}
\pgflineto{\pgfpoint{201.953430pt}{140.778427pt}}
\pgfusepath{stroke}
\pgfpathmoveto{\pgfpoint{203.945404pt}{140.777435pt}}
\pgflineto{\pgfpoint{202.949402pt}{140.777435pt}}
\pgfusepath{stroke}
\pgfpathmoveto{\pgfpoint{204.941376pt}{140.777435pt}}
\pgflineto{\pgfpoint{203.945404pt}{140.777435pt}}
\pgfusepath{stroke}
\pgfpathmoveto{\pgfpoint{205.937347pt}{140.854156pt}}
\pgflineto{\pgfpoint{204.941376pt}{140.777435pt}}
\pgfusepath{stroke}
\pgfpathmoveto{\pgfpoint{206.933334pt}{140.777435pt}}
\pgflineto{\pgfpoint{205.937347pt}{140.854156pt}}
\pgfusepath{stroke}
\pgfpathmoveto{\pgfpoint{207.929337pt}{140.777435pt}}
\pgflineto{\pgfpoint{206.933334pt}{140.777435pt}}
\pgfusepath{stroke}
\pgfpathmoveto{\pgfpoint{208.925323pt}{141.508636pt}}
\pgflineto{\pgfpoint{207.929337pt}{140.777435pt}}
\pgfusepath{stroke}
\pgfpathmoveto{\pgfpoint{209.921295pt}{143.205780pt}}
\pgflineto{\pgfpoint{208.925323pt}{141.508636pt}}
\pgfusepath{stroke}
\pgfpathmoveto{\pgfpoint{210.917267pt}{141.750488pt}}
\pgflineto{\pgfpoint{209.921295pt}{143.205780pt}}
\pgfusepath{stroke}
\pgfpathmoveto{\pgfpoint{211.913269pt}{140.869766pt}}
\pgflineto{\pgfpoint{210.917267pt}{141.750488pt}}
\pgfusepath{stroke}
\pgfpathmoveto{\pgfpoint{212.909241pt}{140.777573pt}}
\pgflineto{\pgfpoint{211.913269pt}{140.869766pt}}
\pgfusepath{stroke}
\pgfpathmoveto{\pgfpoint{213.905228pt}{140.910294pt}}
\pgflineto{\pgfpoint{212.909241pt}{140.777573pt}}
\pgfusepath{stroke}
\pgfpathmoveto{\pgfpoint{214.901215pt}{140.781433pt}}
\pgflineto{\pgfpoint{213.905228pt}{140.910294pt}}
\pgfusepath{stroke}
\pgfpathmoveto{\pgfpoint{215.897217pt}{141.024170pt}}
\pgflineto{\pgfpoint{214.901215pt}{140.781433pt}}
\pgfusepath{stroke}
\pgfpathmoveto{\pgfpoint{216.893188pt}{140.777435pt}}
\pgflineto{\pgfpoint{215.897217pt}{141.024170pt}}
\pgfusepath{stroke}
\pgfpathmoveto{\pgfpoint{217.889160pt}{140.945007pt}}
\pgflineto{\pgfpoint{216.893188pt}{140.777435pt}}
\pgfusepath{stroke}
\pgfpathmoveto{\pgfpoint{218.885147pt}{140.777435pt}}
\pgflineto{\pgfpoint{217.889160pt}{140.945007pt}}
\pgfusepath{stroke}
\pgfpathmoveto{\pgfpoint{219.881134pt}{140.780746pt}}
\pgflineto{\pgfpoint{218.885147pt}{140.777435pt}}
\pgfusepath{stroke}
\pgfpathmoveto{\pgfpoint{220.877121pt}{140.777435pt}}
\pgflineto{\pgfpoint{219.881134pt}{140.780746pt}}
\pgfusepath{stroke}
\pgfpathmoveto{\pgfpoint{221.873108pt}{140.777435pt}}
\pgflineto{\pgfpoint{220.877121pt}{140.777435pt}}
\pgfusepath{stroke}
\pgfpathmoveto{\pgfpoint{222.869080pt}{140.777435pt}}
\pgflineto{\pgfpoint{221.873108pt}{140.777435pt}}
\pgfusepath{stroke}
\pgfpathmoveto{\pgfpoint{223.865082pt}{140.800522pt}}
\pgflineto{\pgfpoint{222.869080pt}{140.777435pt}}
\pgfusepath{stroke}
\pgfpathmoveto{\pgfpoint{224.861053pt}{140.777695pt}}
\pgflineto{\pgfpoint{223.865082pt}{140.800522pt}}
\pgfusepath{stroke}
\pgfpathmoveto{\pgfpoint{225.857040pt}{140.916656pt}}
\pgflineto{\pgfpoint{224.861053pt}{140.777695pt}}
\pgfusepath{stroke}
\pgfpathmoveto{\pgfpoint{226.853027pt}{140.777435pt}}
\pgflineto{\pgfpoint{225.857040pt}{140.916656pt}}
\pgfusepath{stroke}
\pgfpathmoveto{\pgfpoint{227.849014pt}{140.782928pt}}
\pgflineto{\pgfpoint{226.853027pt}{140.777435pt}}
\pgfusepath{stroke}
\pgfpathmoveto{\pgfpoint{228.845001pt}{140.833771pt}}
\pgflineto{\pgfpoint{227.849014pt}{140.782928pt}}
\pgfusepath{stroke}
\pgfpathmoveto{\pgfpoint{229.840973pt}{140.894608pt}}
\pgflineto{\pgfpoint{228.845001pt}{140.833771pt}}
\pgfusepath{stroke}
\pgfpathmoveto{\pgfpoint{230.836945pt}{140.789825pt}}
\pgflineto{\pgfpoint{229.840973pt}{140.894608pt}}
\pgfusepath{stroke}
\pgfpathmoveto{\pgfpoint{231.832932pt}{142.041382pt}}
\pgflineto{\pgfpoint{230.836945pt}{140.789825pt}}
\pgfusepath{stroke}
\pgfpathmoveto{\pgfpoint{232.828934pt}{140.777435pt}}
\pgflineto{\pgfpoint{231.832932pt}{142.041382pt}}
\pgfusepath{stroke}
\pgfpathmoveto{\pgfpoint{233.824921pt}{140.823318pt}}
\pgflineto{\pgfpoint{232.828934pt}{140.777435pt}}
\pgfusepath{stroke}
\pgfpathmoveto{\pgfpoint{234.820892pt}{140.922363pt}}
\pgflineto{\pgfpoint{233.824921pt}{140.823318pt}}
\pgfusepath{stroke}
\pgfpathmoveto{\pgfpoint{235.816864pt}{140.779510pt}}
\pgflineto{\pgfpoint{234.820892pt}{140.922363pt}}
\pgfusepath{stroke}
\pgfpathmoveto{\pgfpoint{236.812866pt}{140.790619pt}}
\pgflineto{\pgfpoint{235.816864pt}{140.779510pt}}
\pgfusepath{stroke}
\pgfpathmoveto{\pgfpoint{237.808838pt}{140.799713pt}}
\pgflineto{\pgfpoint{236.812866pt}{140.790619pt}}
\pgfusepath{stroke}
\pgfpathmoveto{\pgfpoint{238.804825pt}{140.785339pt}}
\pgflineto{\pgfpoint{237.808838pt}{140.799713pt}}
\pgfusepath{stroke}
\pgfpathmoveto{\pgfpoint{239.800812pt}{140.777435pt}}
\pgflineto{\pgfpoint{238.804825pt}{140.785339pt}}
\pgfusepath{stroke}
\pgfpathmoveto{\pgfpoint{240.796814pt}{140.796509pt}}
\pgflineto{\pgfpoint{239.800812pt}{140.777435pt}}
\pgfusepath{stroke}
\pgfpathmoveto{\pgfpoint{241.792786pt}{140.777435pt}}
\pgflineto{\pgfpoint{240.796814pt}{140.796509pt}}
\pgfusepath{stroke}
\pgfpathmoveto{\pgfpoint{242.788757pt}{140.780701pt}}
\pgflineto{\pgfpoint{241.792786pt}{140.777435pt}}
\pgfusepath{stroke}
\pgfpathmoveto{\pgfpoint{243.784744pt}{140.777786pt}}
\pgflineto{\pgfpoint{242.788757pt}{140.780701pt}}
\pgfusepath{stroke}
\pgfpathmoveto{\pgfpoint{244.780731pt}{140.777435pt}}
\pgflineto{\pgfpoint{243.784744pt}{140.777786pt}}
\pgfusepath{stroke}
\pgfpathmoveto{\pgfpoint{245.776718pt}{140.778488pt}}
\pgflineto{\pgfpoint{244.780731pt}{140.777435pt}}
\pgfusepath{stroke}
\pgfpathmoveto{\pgfpoint{246.772705pt}{143.280365pt}}
\pgflineto{\pgfpoint{245.776718pt}{140.778488pt}}
\pgfusepath{stroke}
\pgfpathmoveto{\pgfpoint{247.768677pt}{140.830551pt}}
\pgflineto{\pgfpoint{246.772705pt}{143.280365pt}}
\pgfusepath{stroke}
\pgfpathmoveto{\pgfpoint{248.764679pt}{140.781235pt}}
\pgflineto{\pgfpoint{247.768677pt}{140.830551pt}}
\pgfusepath{stroke}
\pgfpathmoveto{\pgfpoint{249.760651pt}{140.778198pt}}
\pgflineto{\pgfpoint{248.764679pt}{140.781235pt}}
\pgfusepath{stroke}
\pgfpathmoveto{\pgfpoint{250.756638pt}{140.823486pt}}
\pgflineto{\pgfpoint{249.760651pt}{140.778198pt}}
\pgfusepath{stroke}
\pgfpathmoveto{\pgfpoint{251.752625pt}{140.777435pt}}
\pgflineto{\pgfpoint{250.756638pt}{140.823486pt}}
\pgfusepath{stroke}
\pgfpathmoveto{\pgfpoint{252.748611pt}{140.777435pt}}
\pgflineto{\pgfpoint{251.752625pt}{140.777435pt}}
\pgfusepath{stroke}
\pgfpathmoveto{\pgfpoint{253.744598pt}{142.920593pt}}
\pgflineto{\pgfpoint{252.748611pt}{140.777435pt}}
\pgfusepath{stroke}
\pgfpathmoveto{\pgfpoint{254.740570pt}{142.519669pt}}
\pgflineto{\pgfpoint{253.744598pt}{142.920593pt}}
\pgfusepath{stroke}
\pgfpathmoveto{\pgfpoint{255.736542pt}{140.777435pt}}
\pgflineto{\pgfpoint{254.740570pt}{142.519669pt}}
\pgfusepath{stroke}
\pgfpathmoveto{\pgfpoint{256.732544pt}{143.110718pt}}
\pgflineto{\pgfpoint{255.736542pt}{140.777435pt}}
\pgfusepath{stroke}
\pgfpathmoveto{\pgfpoint{257.728516pt}{140.907501pt}}
\pgflineto{\pgfpoint{256.732544pt}{143.110718pt}}
\pgfusepath{stroke}
\pgfpathmoveto{\pgfpoint{258.724518pt}{140.840500pt}}
\pgflineto{\pgfpoint{257.728516pt}{140.907501pt}}
\pgfusepath{stroke}
\pgfpathmoveto{\pgfpoint{259.720490pt}{140.779770pt}}
\pgflineto{\pgfpoint{258.724518pt}{140.840500pt}}
\pgfusepath{stroke}
\pgfpathmoveto{\pgfpoint{260.716492pt}{140.777496pt}}
\pgflineto{\pgfpoint{259.720490pt}{140.779770pt}}
\pgfusepath{stroke}
\pgfpathmoveto{\pgfpoint{261.712463pt}{140.779022pt}}
\pgflineto{\pgfpoint{260.716492pt}{140.777496pt}}
\pgfusepath{stroke}
\pgfpathmoveto{\pgfpoint{262.708435pt}{140.777435pt}}
\pgflineto{\pgfpoint{261.712463pt}{140.779022pt}}
\pgfusepath{stroke}
\pgfpathmoveto{\pgfpoint{263.704407pt}{140.927124pt}}
\pgflineto{\pgfpoint{262.708435pt}{140.777435pt}}
\pgfusepath{stroke}
\pgfpathmoveto{\pgfpoint{264.700409pt}{141.153885pt}}
\pgflineto{\pgfpoint{263.704407pt}{140.927124pt}}
\pgfusepath{stroke}
\pgfpathmoveto{\pgfpoint{265.696411pt}{140.951752pt}}
\pgflineto{\pgfpoint{264.700409pt}{141.153885pt}}
\pgfusepath{stroke}
\pgfpathmoveto{\pgfpoint{266.692383pt}{140.785080pt}}
\pgflineto{\pgfpoint{265.696411pt}{140.951752pt}}
\pgfusepath{stroke}
\pgfpathmoveto{\pgfpoint{267.688354pt}{140.779419pt}}
\pgflineto{\pgfpoint{266.692383pt}{140.785080pt}}
\pgfusepath{stroke}
\pgfpathmoveto{\pgfpoint{268.684326pt}{140.976303pt}}
\pgflineto{\pgfpoint{267.688354pt}{140.779419pt}}
\pgfusepath{stroke}
\pgfpathmoveto{\pgfpoint{269.680328pt}{140.824692pt}}
\pgflineto{\pgfpoint{268.684326pt}{140.976303pt}}
\pgfusepath{stroke}
\pgfpathmoveto{\pgfpoint{270.676331pt}{140.778473pt}}
\pgflineto{\pgfpoint{269.680328pt}{140.824692pt}}
\pgfusepath{stroke}
\pgfpathmoveto{\pgfpoint{271.672302pt}{141.361237pt}}
\pgflineto{\pgfpoint{270.676331pt}{140.778473pt}}
\pgfusepath{stroke}
\pgfpathmoveto{\pgfpoint{272.668274pt}{141.409683pt}}
\pgflineto{\pgfpoint{271.672302pt}{141.361237pt}}
\pgfusepath{stroke}
\pgfpathmoveto{\pgfpoint{273.664276pt}{140.777435pt}}
\pgflineto{\pgfpoint{272.668274pt}{141.409683pt}}
\pgfusepath{stroke}
\pgfpathmoveto{\pgfpoint{274.660248pt}{140.777435pt}}
\pgflineto{\pgfpoint{273.664276pt}{140.777435pt}}
\pgfusepath{stroke}
\pgfpathmoveto{\pgfpoint{275.656250pt}{140.777435pt}}
\pgflineto{\pgfpoint{274.660248pt}{140.777435pt}}
\pgfusepath{stroke}
\pgfpathmoveto{\pgfpoint{276.652222pt}{140.792587pt}}
\pgflineto{\pgfpoint{275.656250pt}{140.777435pt}}
\pgfusepath{stroke}
\pgfpathmoveto{\pgfpoint{277.648193pt}{140.928909pt}}
\pgflineto{\pgfpoint{276.652222pt}{140.792587pt}}
\pgfusepath{stroke}
\pgfpathmoveto{\pgfpoint{278.644196pt}{140.783813pt}}
\pgflineto{\pgfpoint{277.648193pt}{140.928909pt}}
\pgfusepath{stroke}
\pgfpathmoveto{\pgfpoint{279.640167pt}{140.790070pt}}
\pgflineto{\pgfpoint{278.644196pt}{140.783813pt}}
\pgfusepath{stroke}
\pgfpathmoveto{\pgfpoint{280.636139pt}{140.777878pt}}
\pgflineto{\pgfpoint{279.640167pt}{140.790070pt}}
\pgfusepath{stroke}
\pgfpathmoveto{\pgfpoint{281.632141pt}{140.777481pt}}
\pgflineto{\pgfpoint{280.636139pt}{140.777878pt}}
\pgfusepath{stroke}
\pgfpathmoveto{\pgfpoint{282.628113pt}{140.782150pt}}
\pgflineto{\pgfpoint{281.632141pt}{140.777481pt}}
\pgfusepath{stroke}
\pgfpathmoveto{\pgfpoint{283.624115pt}{140.777435pt}}
\pgflineto{\pgfpoint{282.628113pt}{140.782150pt}}
\pgfusepath{stroke}
\pgfpathmoveto{\pgfpoint{284.620087pt}{140.777740pt}}
\pgflineto{\pgfpoint{283.624115pt}{140.777435pt}}
\pgfusepath{stroke}
\pgfpathmoveto{\pgfpoint{285.616089pt}{140.784164pt}}
\pgflineto{\pgfpoint{284.620087pt}{140.777740pt}}
\pgfusepath{stroke}
\pgfpathmoveto{\pgfpoint{286.612061pt}{140.777435pt}}
\pgflineto{\pgfpoint{285.616089pt}{140.784164pt}}
\pgfusepath{stroke}
\pgfpathmoveto{\pgfpoint{287.608032pt}{140.777435pt}}
\pgflineto{\pgfpoint{286.612061pt}{140.777435pt}}
\pgfusepath{stroke}
\pgfpathmoveto{\pgfpoint{288.604004pt}{140.798889pt}}
\pgflineto{\pgfpoint{287.608032pt}{140.777435pt}}
\pgfusepath{stroke}
\pgfpathmoveto{\pgfpoint{289.600037pt}{140.790405pt}}
\pgflineto{\pgfpoint{288.604004pt}{140.798889pt}}
\pgfusepath{stroke}
\color[rgb]{0.000000,0.000000,0.000000}
\pgfsetlinewidth{0.500000pt}
\pgfsetdash{{16pt}{0pt}}{0pt}
\pgfpathmoveto{\pgfpoint{289.600037pt}{26.399979pt}}
\pgflineto{\pgfpoint{41.600006pt}{26.399979pt}}
\pgfusepath{stroke}
\pgfpathmoveto{\pgfpoint{289.600037pt}{91.199989pt}}
\pgflineto{\pgfpoint{41.600006pt}{91.199989pt}}
\pgfusepath{stroke}
\pgfpathmoveto{\pgfpoint{41.600006pt}{91.199989pt}}
\pgflineto{\pgfpoint{41.600006pt}{26.399979pt}}
\pgfusepath{stroke}
\pgfpathmoveto{\pgfpoint{289.600037pt}{91.199989pt}}
\pgflineto{\pgfpoint{289.600037pt}{26.399979pt}}
\pgfusepath{stroke}
\pgfpathmoveto{\pgfpoint{90.403221pt}{28.872353pt}}
\pgflineto{\pgfpoint{90.403221pt}{26.399979pt}}
\pgfusepath{stroke}
\pgfpathmoveto{\pgfpoint{90.403221pt}{88.727623pt}}
\pgflineto{\pgfpoint{90.403221pt}{91.199989pt}}
\pgfusepath{stroke}
\pgfpathmoveto{\pgfpoint{140.202423pt}{28.872353pt}}
\pgflineto{\pgfpoint{140.202423pt}{26.399979pt}}
\pgfusepath{stroke}
\pgfpathmoveto{\pgfpoint{140.202423pt}{88.727623pt}}
\pgflineto{\pgfpoint{140.202423pt}{91.199989pt}}
\pgfusepath{stroke}
\pgfpathmoveto{\pgfpoint{190.001617pt}{28.872353pt}}
\pgflineto{\pgfpoint{190.001617pt}{26.399979pt}}
\pgfusepath{stroke}
\pgfpathmoveto{\pgfpoint{190.001617pt}{88.727623pt}}
\pgflineto{\pgfpoint{190.001617pt}{91.199989pt}}
\pgfusepath{stroke}
\pgfpathmoveto{\pgfpoint{239.800812pt}{28.872353pt}}
\pgflineto{\pgfpoint{239.800812pt}{26.399979pt}}
\pgfusepath{stroke}
\pgfpathmoveto{\pgfpoint{239.800812pt}{88.727623pt}}
\pgflineto{\pgfpoint{239.800812pt}{91.199989pt}}
\pgfusepath{stroke}
\pgfpathmoveto{\pgfpoint{289.600037pt}{28.872353pt}}
\pgflineto{\pgfpoint{289.600037pt}{26.399979pt}}
\pgfusepath{stroke}
\pgfpathmoveto{\pgfpoint{289.600037pt}{88.727623pt}}
\pgflineto{\pgfpoint{289.600037pt}{91.199989pt}}
\pgfusepath{stroke}
{
\pgftransformshift{\pgfpoint{90.403229pt}{21.415367pt}}
\pgfnode{rectangle}{north}{\fontsize{10}{0}\selectfont\textcolor[rgb]{0,0,0}{{50}}}{}{\pgfusepath{discard}}}
{
\pgftransformshift{\pgfpoint{140.202423pt}{21.415367pt}}
\pgfnode{rectangle}{north}{\fontsize{10}{0}\selectfont\textcolor[rgb]{0,0,0}{{100}}}{}{\pgfusepath{discard}}}
{
\pgftransformshift{\pgfpoint{190.001617pt}{21.415367pt}}
\pgfnode{rectangle}{north}{\fontsize{10}{0}\selectfont\textcolor[rgb]{0,0,0}{{150}}}{}{\pgfusepath{discard}}}
{
\pgftransformshift{\pgfpoint{239.800812pt}{21.415367pt}}
\pgfnode{rectangle}{north}{\fontsize{10}{0}\selectfont\textcolor[rgb]{0,0,0}{{200}}}{}{\pgfusepath{discard}}}
{
\pgftransformshift{\pgfpoint{289.600037pt}{21.415367pt}}
\pgfnode{rectangle}{north}{\fontsize{10}{0}\selectfont\textcolor[rgb]{0,0,0}{{250}}}{}{\pgfusepath{discard}}}
\pgfpathmoveto{\pgfpoint{44.080009pt}{26.399979pt}}
\pgflineto{\pgfpoint{41.600006pt}{26.399979pt}}
\pgfusepath{stroke}
\pgfpathmoveto{\pgfpoint{287.119995pt}{26.399979pt}}
\pgflineto{\pgfpoint{289.600037pt}{26.399979pt}}
\pgfusepath{stroke}
\pgfpathmoveto{\pgfpoint{44.080009pt}{39.359985pt}}
\pgflineto{\pgfpoint{41.600006pt}{39.359985pt}}
\pgfusepath{stroke}
\pgfpathmoveto{\pgfpoint{287.119995pt}{39.359985pt}}
\pgflineto{\pgfpoint{289.600037pt}{39.359985pt}}
\pgfusepath{stroke}
\pgfpathmoveto{\pgfpoint{44.080009pt}{52.319984pt}}
\pgflineto{\pgfpoint{41.600006pt}{52.319984pt}}
\pgfusepath{stroke}
\pgfpathmoveto{\pgfpoint{287.119995pt}{52.319984pt}}
\pgflineto{\pgfpoint{289.600037pt}{52.319984pt}}
\pgfusepath{stroke}
\pgfpathmoveto{\pgfpoint{44.080009pt}{65.279991pt}}
\pgflineto{\pgfpoint{41.600006pt}{65.279991pt}}
\pgfusepath{stroke}
\pgfpathmoveto{\pgfpoint{287.119995pt}{65.279991pt}}
\pgflineto{\pgfpoint{289.600037pt}{65.279991pt}}
\pgfusepath{stroke}
\pgfpathmoveto{\pgfpoint{44.080009pt}{78.239990pt}}
\pgflineto{\pgfpoint{41.600006pt}{78.239990pt}}
\pgfusepath{stroke}
\pgfpathmoveto{\pgfpoint{287.119995pt}{78.239990pt}}
\pgflineto{\pgfpoint{289.600037pt}{78.239990pt}}
\pgfusepath{stroke}
\pgfpathmoveto{\pgfpoint{44.080009pt}{91.199989pt}}
\pgflineto{\pgfpoint{41.600006pt}{91.199989pt}}
\pgfusepath{stroke}
\pgfpathmoveto{\pgfpoint{287.119995pt}{91.199989pt}}
\pgflineto{\pgfpoint{289.600037pt}{91.199989pt}}
\pgfusepath{stroke}
{
\pgftransformshift{\pgfpoint{36.600006pt}{26.399979pt}}
\pgfnode{rectangle}{east}{\fontsize{10}{0}\selectfont\textcolor[rgb]{0,0,0}{{0}}}{}{\pgfusepath{discard}}}
{
\pgftransformshift{\pgfpoint{36.600006pt}{39.359978pt}}
\pgfnode{rectangle}{east}{\fontsize{10}{0}\selectfont\textcolor[rgb]{0,0,0}{{2e+06}}}{}{\pgfusepath{discard}}}
{
\pgftransformshift{\pgfpoint{36.600006pt}{52.319984pt}}
\pgfnode{rectangle}{east}{\fontsize{10}{0}\selectfont\textcolor[rgb]{0,0,0}{{4e+06}}}{}{\pgfusepath{discard}}}
{
\pgftransformshift{\pgfpoint{36.600006pt}{65.279991pt}}
\pgfnode{rectangle}{east}{\fontsize{10}{0}\selectfont\textcolor[rgb]{0,0,0}{{6e+06}}}{}{\pgfusepath{discard}}}
{
\pgftransformshift{\pgfpoint{36.600006pt}{78.239990pt}}
\pgfnode{rectangle}{east}{\fontsize{10}{0}\selectfont\textcolor[rgb]{0,0,0}{{8e+06}}}{}{\pgfusepath{discard}}}
{
\pgftransformshift{\pgfpoint{36.600006pt}{91.199989pt}}
\pgfnode{rectangle}{east}{\fontsize{10}{0}\selectfont\textcolor[rgb]{0,0,0}{{1e+07}}}{}{\pgfusepath{discard}}}
\pgfsetlinewidth{0.000100pt}
\pgfsetdash{}{0pt}
\pgfpathmoveto{\pgfpoint{43.591980pt}{26.399979pt}}
\pgflineto{\pgfpoint{44.587967pt}{26.399979pt}}
\pgfusepath{stroke}
\pgfpathmoveto{\pgfpoint{42.595993pt}{26.399979pt}}
\pgflineto{\pgfpoint{43.591980pt}{26.399979pt}}
\pgfusepath{stroke}
\pgfpathmoveto{\pgfpoint{43.591980pt}{33.198105pt}}
\pgflineto{\pgfpoint{42.595993pt}{26.399979pt}}
\pgfusepath{stroke}
\pgfpathmoveto{\pgfpoint{44.587967pt}{26.399979pt}}
\pgflineto{\pgfpoint{43.591980pt}{33.198105pt}}
\pgfusepath{stroke}
\pgfpathmoveto{\pgfpoint{47.575912pt}{26.399979pt}}
\pgflineto{\pgfpoint{48.571899pt}{26.399979pt}}
\pgfusepath{stroke}
\pgfpathmoveto{\pgfpoint{46.579933pt}{26.399979pt}}
\pgflineto{\pgfpoint{47.575912pt}{26.399979pt}}
\pgfusepath{stroke}
\pgfpathmoveto{\pgfpoint{47.575912pt}{27.559311pt}}
\pgflineto{\pgfpoint{46.579933pt}{26.399979pt}}
\pgfusepath{stroke}
\pgfpathmoveto{\pgfpoint{48.571899pt}{26.399979pt}}
\pgflineto{\pgfpoint{47.575912pt}{27.559311pt}}
\pgfusepath{stroke}
\pgfpathmoveto{\pgfpoint{53.551819pt}{26.399979pt}}
\pgflineto{\pgfpoint{54.547806pt}{26.399979pt}}
\pgfusepath{stroke}
\pgfpathmoveto{\pgfpoint{52.555840pt}{26.399979pt}}
\pgflineto{\pgfpoint{53.551819pt}{26.399979pt}}
\pgfusepath{stroke}
\pgfpathmoveto{\pgfpoint{53.551819pt}{32.692734pt}}
\pgflineto{\pgfpoint{52.555840pt}{26.399979pt}}
\pgfusepath{stroke}
\pgfpathmoveto{\pgfpoint{54.547806pt}{26.399979pt}}
\pgflineto{\pgfpoint{53.551819pt}{32.692734pt}}
\pgfusepath{stroke}
\pgfpathmoveto{\pgfpoint{56.539772pt}{26.399979pt}}
\pgflineto{\pgfpoint{57.535751pt}{26.399979pt}}
\pgfusepath{stroke}
\pgfpathmoveto{\pgfpoint{55.543785pt}{26.399979pt}}
\pgflineto{\pgfpoint{56.539772pt}{26.399979pt}}
\pgfusepath{stroke}
\pgfpathmoveto{\pgfpoint{54.547806pt}{26.399979pt}}
\pgflineto{\pgfpoint{55.543785pt}{26.399979pt}}
\pgfusepath{stroke}
\pgfpathmoveto{\pgfpoint{55.543785pt}{26.577354pt}}
\pgflineto{\pgfpoint{54.547806pt}{26.399979pt}}
\pgfusepath{stroke}
\pgfpathmoveto{\pgfpoint{56.539772pt}{65.803238pt}}
\pgflineto{\pgfpoint{55.543785pt}{26.577354pt}}
\pgfusepath{stroke}
\pgfpathmoveto{\pgfpoint{57.535751pt}{26.399979pt}}
\pgflineto{\pgfpoint{56.539772pt}{65.803238pt}}
\pgfusepath{stroke}
\pgfpathmoveto{\pgfpoint{59.527725pt}{26.399979pt}}
\pgflineto{\pgfpoint{60.523712pt}{26.399979pt}}
\pgfusepath{stroke}
\pgfpathmoveto{\pgfpoint{58.531738pt}{26.399979pt}}
\pgflineto{\pgfpoint{59.527725pt}{26.399979pt}}
\pgfusepath{stroke}
\pgfpathmoveto{\pgfpoint{57.535751pt}{26.399979pt}}
\pgflineto{\pgfpoint{58.531738pt}{26.399979pt}}
\pgfusepath{stroke}
\pgfpathmoveto{\pgfpoint{58.531738pt}{27.217796pt}}
\pgflineto{\pgfpoint{57.535751pt}{26.399979pt}}
\pgfusepath{stroke}
\pgfpathmoveto{\pgfpoint{59.527725pt}{26.723701pt}}
\pgflineto{\pgfpoint{58.531738pt}{27.217796pt}}
\pgfusepath{stroke}
\pgfpathmoveto{\pgfpoint{60.523712pt}{26.399979pt}}
\pgflineto{\pgfpoint{59.527725pt}{26.723701pt}}
\pgfusepath{stroke}
\pgfpathmoveto{\pgfpoint{62.515678pt}{26.399979pt}}
\pgflineto{\pgfpoint{63.511658pt}{26.399979pt}}
\pgfusepath{stroke}
\pgfpathmoveto{\pgfpoint{61.519691pt}{26.399979pt}}
\pgflineto{\pgfpoint{62.515678pt}{26.399979pt}}
\pgfusepath{stroke}
\pgfpathmoveto{\pgfpoint{62.515678pt}{26.488380pt}}
\pgflineto{\pgfpoint{61.519691pt}{26.399979pt}}
\pgfusepath{stroke}
\pgfpathmoveto{\pgfpoint{63.511658pt}{26.399979pt}}
\pgflineto{\pgfpoint{62.515678pt}{26.488380pt}}
\pgfusepath{stroke}
\pgfpathmoveto{\pgfpoint{66.499619pt}{26.399979pt}}
\pgflineto{\pgfpoint{67.495590pt}{26.399979pt}}
\pgfusepath{stroke}
\pgfpathmoveto{\pgfpoint{65.503624pt}{26.399979pt}}
\pgflineto{\pgfpoint{66.499619pt}{26.399979pt}}
\pgfusepath{stroke}
\pgfpathmoveto{\pgfpoint{64.507637pt}{26.399979pt}}
\pgflineto{\pgfpoint{65.503624pt}{26.399979pt}}
\pgfusepath{stroke}
\pgfpathmoveto{\pgfpoint{63.511658pt}{26.399979pt}}
\pgflineto{\pgfpoint{64.507637pt}{26.399979pt}}
\pgfusepath{stroke}
\pgfpathmoveto{\pgfpoint{64.507637pt}{26.602051pt}}
\pgflineto{\pgfpoint{63.511658pt}{26.399979pt}}
\pgfusepath{stroke}
\pgfpathmoveto{\pgfpoint{65.503624pt}{27.212898pt}}
\pgflineto{\pgfpoint{64.507637pt}{26.602051pt}}
\pgfusepath{stroke}
\pgfpathmoveto{\pgfpoint{66.499619pt}{29.321190pt}}
\pgflineto{\pgfpoint{65.503624pt}{27.212898pt}}
\pgfusepath{stroke}
\pgfpathmoveto{\pgfpoint{67.495590pt}{26.399979pt}}
\pgflineto{\pgfpoint{66.499619pt}{29.321190pt}}
\pgfusepath{stroke}
\pgfpathmoveto{\pgfpoint{69.487564pt}{26.399979pt}}
\pgflineto{\pgfpoint{70.483551pt}{26.399979pt}}
\pgfusepath{stroke}
\pgfpathmoveto{\pgfpoint{68.491577pt}{26.399979pt}}
\pgflineto{\pgfpoint{69.487564pt}{26.399979pt}}
\pgfusepath{stroke}
\pgfpathmoveto{\pgfpoint{67.495590pt}{26.399979pt}}
\pgflineto{\pgfpoint{68.491577pt}{26.399979pt}}
\pgfusepath{stroke}
\pgfpathmoveto{\pgfpoint{68.491577pt}{26.687355pt}}
\pgflineto{\pgfpoint{67.495590pt}{26.399979pt}}
\pgfusepath{stroke}
\pgfpathmoveto{\pgfpoint{69.487564pt}{27.948082pt}}
\pgflineto{\pgfpoint{68.491577pt}{26.687355pt}}
\pgfusepath{stroke}
\pgfpathmoveto{\pgfpoint{70.483551pt}{26.399979pt}}
\pgflineto{\pgfpoint{69.487564pt}{27.948082pt}}
\pgfusepath{stroke}
\pgfpathmoveto{\pgfpoint{72.475510pt}{26.399979pt}}
\pgflineto{\pgfpoint{73.471497pt}{26.399979pt}}
\pgfusepath{stroke}
\pgfpathmoveto{\pgfpoint{71.479530pt}{26.399979pt}}
\pgflineto{\pgfpoint{72.475510pt}{26.399979pt}}
\pgfusepath{stroke}
\pgfpathmoveto{\pgfpoint{70.483551pt}{26.399979pt}}
\pgflineto{\pgfpoint{71.479530pt}{26.399979pt}}
\pgfusepath{stroke}
\pgfpathmoveto{\pgfpoint{71.479530pt}{52.145065pt}}
\pgflineto{\pgfpoint{70.483551pt}{26.399979pt}}
\pgfusepath{stroke}
\pgfpathmoveto{\pgfpoint{72.475510pt}{31.143661pt}}
\pgflineto{\pgfpoint{71.479530pt}{52.145065pt}}
\pgfusepath{stroke}
\pgfpathmoveto{\pgfpoint{73.471497pt}{26.399979pt}}
\pgflineto{\pgfpoint{72.475510pt}{31.143661pt}}
\pgfusepath{stroke}
\pgfpathmoveto{\pgfpoint{75.463470pt}{26.399979pt}}
\pgflineto{\pgfpoint{76.459442pt}{26.399979pt}}
\pgfusepath{stroke}
\pgfpathmoveto{\pgfpoint{74.467484pt}{26.399979pt}}
\pgflineto{\pgfpoint{75.463470pt}{26.399979pt}}
\pgfusepath{stroke}
\pgfpathmoveto{\pgfpoint{73.471497pt}{26.399979pt}}
\pgflineto{\pgfpoint{74.467484pt}{26.399979pt}}
\pgfusepath{stroke}
\pgfpathmoveto{\pgfpoint{74.467484pt}{26.868034pt}}
\pgflineto{\pgfpoint{73.471497pt}{26.399979pt}}
\pgfusepath{stroke}
\pgfpathmoveto{\pgfpoint{75.463470pt}{26.470932pt}}
\pgflineto{\pgfpoint{74.467484pt}{26.868034pt}}
\pgfusepath{stroke}
\pgfpathmoveto{\pgfpoint{76.459442pt}{26.399979pt}}
\pgflineto{\pgfpoint{75.463470pt}{26.470932pt}}
\pgfusepath{stroke}
\pgfpathmoveto{\pgfpoint{82.435356pt}{26.399979pt}}
\pgflineto{\pgfpoint{83.431335pt}{26.399979pt}}
\pgfusepath{stroke}
\pgfpathmoveto{\pgfpoint{81.439369pt}{26.399979pt}}
\pgflineto{\pgfpoint{82.435356pt}{26.399979pt}}
\pgfusepath{stroke}
\pgfpathmoveto{\pgfpoint{80.443390pt}{26.399979pt}}
\pgflineto{\pgfpoint{81.439369pt}{26.399979pt}}
\pgfusepath{stroke}
\pgfpathmoveto{\pgfpoint{79.447403pt}{26.399979pt}}
\pgflineto{\pgfpoint{80.443390pt}{26.399979pt}}
\pgfusepath{stroke}
\pgfpathmoveto{\pgfpoint{80.443390pt}{28.913742pt}}
\pgflineto{\pgfpoint{79.447403pt}{26.399979pt}}
\pgfusepath{stroke}
\pgfpathmoveto{\pgfpoint{81.439369pt}{28.404305pt}}
\pgflineto{\pgfpoint{80.443390pt}{28.913742pt}}
\pgfusepath{stroke}
\pgfpathmoveto{\pgfpoint{82.435356pt}{28.310295pt}}
\pgflineto{\pgfpoint{81.439369pt}{28.404305pt}}
\pgfusepath{stroke}
\pgfpathmoveto{\pgfpoint{83.431335pt}{26.399979pt}}
\pgflineto{\pgfpoint{82.435356pt}{28.310295pt}}
\pgfusepath{stroke}
\pgfpathmoveto{\pgfpoint{86.419289pt}{26.399979pt}}
\pgflineto{\pgfpoint{87.415276pt}{26.399979pt}}
\pgfusepath{stroke}
\pgfpathmoveto{\pgfpoint{85.423309pt}{26.399979pt}}
\pgflineto{\pgfpoint{86.419289pt}{26.399979pt}}
\pgfusepath{stroke}
\pgfpathmoveto{\pgfpoint{86.419289pt}{26.871178pt}}
\pgflineto{\pgfpoint{85.423309pt}{26.399979pt}}
\pgfusepath{stroke}
\pgfpathmoveto{\pgfpoint{87.415276pt}{26.399979pt}}
\pgflineto{\pgfpoint{86.419289pt}{26.871178pt}}
\pgfusepath{stroke}
\pgfpathmoveto{\pgfpoint{88.411255pt}{26.399979pt}}
\pgflineto{\pgfpoint{89.407242pt}{26.399979pt}}
\pgfusepath{stroke}
\pgfpathmoveto{\pgfpoint{87.415276pt}{26.399979pt}}
\pgflineto{\pgfpoint{88.411255pt}{26.399979pt}}
\pgfusepath{stroke}
\pgfpathmoveto{\pgfpoint{88.411255pt}{31.944191pt}}
\pgflineto{\pgfpoint{87.415276pt}{26.399979pt}}
\pgfusepath{stroke}
\pgfpathmoveto{\pgfpoint{89.407242pt}{26.399979pt}}
\pgflineto{\pgfpoint{88.411255pt}{31.944191pt}}
\pgfusepath{stroke}
\pgfpathmoveto{\pgfpoint{90.403221pt}{26.399979pt}}
\pgflineto{\pgfpoint{91.399208pt}{26.399979pt}}
\pgfusepath{stroke}
\pgfpathmoveto{\pgfpoint{89.407242pt}{26.399979pt}}
\pgflineto{\pgfpoint{90.403221pt}{26.399979pt}}
\pgfusepath{stroke}
\pgfpathmoveto{\pgfpoint{90.403221pt}{26.454811pt}}
\pgflineto{\pgfpoint{89.407242pt}{26.399979pt}}
\pgfusepath{stroke}
\pgfpathmoveto{\pgfpoint{91.399208pt}{26.399979pt}}
\pgflineto{\pgfpoint{90.403221pt}{26.454811pt}}
\pgfusepath{stroke}
\pgfpathmoveto{\pgfpoint{93.391174pt}{26.399979pt}}
\pgflineto{\pgfpoint{94.387161pt}{26.399979pt}}
\pgfusepath{stroke}
\pgfpathmoveto{\pgfpoint{92.395187pt}{26.399979pt}}
\pgflineto{\pgfpoint{93.391174pt}{26.399979pt}}
\pgfusepath{stroke}
\pgfpathmoveto{\pgfpoint{91.399208pt}{26.399979pt}}
\pgflineto{\pgfpoint{92.395187pt}{26.399979pt}}
\pgfusepath{stroke}
\pgfpathmoveto{\pgfpoint{92.395187pt}{26.889816pt}}
\pgflineto{\pgfpoint{91.399208pt}{26.399979pt}}
\pgfusepath{stroke}
\pgfpathmoveto{\pgfpoint{93.391174pt}{27.006203pt}}
\pgflineto{\pgfpoint{92.395187pt}{26.889816pt}}
\pgfusepath{stroke}
\pgfpathmoveto{\pgfpoint{94.387161pt}{26.399979pt}}
\pgflineto{\pgfpoint{93.391174pt}{27.006203pt}}
\pgfusepath{stroke}
\pgfpathmoveto{\pgfpoint{100.363068pt}{26.399979pt}}
\pgflineto{\pgfpoint{101.359047pt}{26.399979pt}}
\pgfusepath{stroke}
\pgfpathmoveto{\pgfpoint{99.367081pt}{26.399979pt}}
\pgflineto{\pgfpoint{100.363068pt}{26.399979pt}}
\pgfusepath{stroke}
\pgfpathmoveto{\pgfpoint{98.371094pt}{26.399979pt}}
\pgflineto{\pgfpoint{99.367081pt}{26.399979pt}}
\pgfusepath{stroke}
\pgfpathmoveto{\pgfpoint{97.375107pt}{26.399979pt}}
\pgflineto{\pgfpoint{98.371094pt}{26.399979pt}}
\pgfusepath{stroke}
\pgfpathmoveto{\pgfpoint{98.371094pt}{47.782463pt}}
\pgflineto{\pgfpoint{97.375107pt}{26.399979pt}}
\pgfusepath{stroke}
\pgfpathmoveto{\pgfpoint{99.367081pt}{54.046951pt}}
\pgflineto{\pgfpoint{98.371094pt}{47.782463pt}}
\pgfusepath{stroke}
\pgfpathmoveto{\pgfpoint{100.363068pt}{32.883598pt}}
\pgflineto{\pgfpoint{99.367081pt}{54.046951pt}}
\pgfusepath{stroke}
\pgfpathmoveto{\pgfpoint{101.359047pt}{26.399979pt}}
\pgflineto{\pgfpoint{100.363068pt}{32.883598pt}}
\pgfusepath{stroke}
\pgfpathmoveto{\pgfpoint{102.355034pt}{26.399979pt}}
\pgflineto{\pgfpoint{103.351013pt}{26.399979pt}}
\pgfusepath{stroke}
\pgfpathmoveto{\pgfpoint{101.359047pt}{26.399979pt}}
\pgflineto{\pgfpoint{102.355034pt}{26.399979pt}}
\pgfusepath{stroke}
\pgfpathmoveto{\pgfpoint{102.355034pt}{26.959930pt}}
\pgflineto{\pgfpoint{101.359047pt}{26.399979pt}}
\pgfusepath{stroke}
\pgfpathmoveto{\pgfpoint{103.351013pt}{26.399979pt}}
\pgflineto{\pgfpoint{102.355034pt}{26.959930pt}}
\pgfusepath{stroke}
\pgfpathmoveto{\pgfpoint{104.347000pt}{26.399979pt}}
\pgflineto{\pgfpoint{105.342987pt}{26.399979pt}}
\pgfusepath{stroke}
\pgfpathmoveto{\pgfpoint{103.351013pt}{26.399979pt}}
\pgflineto{\pgfpoint{104.347000pt}{26.399979pt}}
\pgfusepath{stroke}
\pgfpathmoveto{\pgfpoint{104.347000pt}{35.689026pt}}
\pgflineto{\pgfpoint{103.351013pt}{26.399979pt}}
\pgfusepath{stroke}
\pgfpathmoveto{\pgfpoint{105.342987pt}{26.399979pt}}
\pgflineto{\pgfpoint{104.347000pt}{35.689026pt}}
\pgfusepath{stroke}
\pgfpathmoveto{\pgfpoint{109.326920pt}{26.399979pt}}
\pgflineto{\pgfpoint{110.322906pt}{26.399979pt}}
\pgfusepath{stroke}
\pgfpathmoveto{\pgfpoint{108.330933pt}{26.399979pt}}
\pgflineto{\pgfpoint{109.326920pt}{26.399979pt}}
\pgfusepath{stroke}
\pgfpathmoveto{\pgfpoint{107.334953pt}{26.399979pt}}
\pgflineto{\pgfpoint{108.330933pt}{26.399979pt}}
\pgfusepath{stroke}
\pgfpathmoveto{\pgfpoint{108.330933pt}{27.150108pt}}
\pgflineto{\pgfpoint{107.334953pt}{26.399979pt}}
\pgfusepath{stroke}
\pgfpathmoveto{\pgfpoint{109.326920pt}{28.632256pt}}
\pgflineto{\pgfpoint{108.330933pt}{27.150108pt}}
\pgfusepath{stroke}
\pgfpathmoveto{\pgfpoint{110.322906pt}{26.399979pt}}
\pgflineto{\pgfpoint{109.326920pt}{28.632256pt}}
\pgfusepath{stroke}
\pgfpathmoveto{\pgfpoint{113.310852pt}{26.399979pt}}
\pgflineto{\pgfpoint{114.306839pt}{26.399979pt}}
\pgfusepath{stroke}
\pgfpathmoveto{\pgfpoint{112.314873pt}{26.399979pt}}
\pgflineto{\pgfpoint{113.310852pt}{26.399979pt}}
\pgfusepath{stroke}
\pgfpathmoveto{\pgfpoint{111.318893pt}{26.399979pt}}
\pgflineto{\pgfpoint{112.314873pt}{26.399979pt}}
\pgfusepath{stroke}
\pgfpathmoveto{\pgfpoint{112.314873pt}{37.362213pt}}
\pgflineto{\pgfpoint{111.318893pt}{26.399979pt}}
\pgfusepath{stroke}
\pgfpathmoveto{\pgfpoint{113.310852pt}{26.517036pt}}
\pgflineto{\pgfpoint{112.314873pt}{37.362213pt}}
\pgfusepath{stroke}
\pgfpathmoveto{\pgfpoint{114.306839pt}{26.399979pt}}
\pgflineto{\pgfpoint{113.310852pt}{26.517036pt}}
\pgfusepath{stroke}
\pgfpathmoveto{\pgfpoint{118.290779pt}{26.399979pt}}
\pgflineto{\pgfpoint{119.286758pt}{26.399979pt}}
\pgfusepath{stroke}
\pgfpathmoveto{\pgfpoint{117.294792pt}{26.399979pt}}
\pgflineto{\pgfpoint{118.290779pt}{26.399979pt}}
\pgfusepath{stroke}
\pgfpathmoveto{\pgfpoint{116.298813pt}{26.399979pt}}
\pgflineto{\pgfpoint{117.294792pt}{26.399979pt}}
\pgfusepath{stroke}
\pgfpathmoveto{\pgfpoint{115.302826pt}{26.399979pt}}
\pgflineto{\pgfpoint{116.298813pt}{26.399979pt}}
\pgfusepath{stroke}
\pgfpathmoveto{\pgfpoint{116.298813pt}{29.134193pt}}
\pgflineto{\pgfpoint{115.302826pt}{26.399979pt}}
\pgfusepath{stroke}
\pgfpathmoveto{\pgfpoint{117.294792pt}{26.815964pt}}
\pgflineto{\pgfpoint{116.298813pt}{29.134193pt}}
\pgfusepath{stroke}
\pgfpathmoveto{\pgfpoint{117.966232pt}{91.264786pt}}
\pgflineto{\pgfpoint{117.294792pt}{26.815964pt}}
\pgfusepath{stroke}
\pgfpathmoveto{\pgfpoint{119.286758pt}{26.399979pt}}
\pgflineto{\pgfpoint{118.613922pt}{91.264793pt}}
\pgfusepath{stroke}
\pgfpathmoveto{\pgfpoint{121.278725pt}{26.399979pt}}
\pgflineto{\pgfpoint{122.274712pt}{26.399979pt}}
\pgfusepath{stroke}
\pgfpathmoveto{\pgfpoint{120.282745pt}{26.399979pt}}
\pgflineto{\pgfpoint{121.278725pt}{26.399979pt}}
\pgfusepath{stroke}
\pgfpathmoveto{\pgfpoint{121.278725pt}{27.190407pt}}
\pgflineto{\pgfpoint{120.282745pt}{26.399979pt}}
\pgfusepath{stroke}
\pgfpathmoveto{\pgfpoint{122.274712pt}{26.399979pt}}
\pgflineto{\pgfpoint{121.278725pt}{27.190407pt}}
\pgfusepath{stroke}
\pgfpathmoveto{\pgfpoint{123.270691pt}{26.399979pt}}
\pgflineto{\pgfpoint{124.266678pt}{26.399979pt}}
\pgfusepath{stroke}
\pgfpathmoveto{\pgfpoint{122.274712pt}{26.399979pt}}
\pgflineto{\pgfpoint{123.270691pt}{26.399979pt}}
\pgfusepath{stroke}
\pgfpathmoveto{\pgfpoint{123.270691pt}{26.975601pt}}
\pgflineto{\pgfpoint{122.274712pt}{26.399979pt}}
\pgfusepath{stroke}
\pgfpathmoveto{\pgfpoint{124.266678pt}{26.399979pt}}
\pgflineto{\pgfpoint{123.270691pt}{26.975601pt}}
\pgfusepath{stroke}
\pgfpathmoveto{\pgfpoint{132.234558pt}{26.399979pt}}
\pgflineto{\pgfpoint{133.230530pt}{26.399979pt}}
\pgfusepath{stroke}
\pgfpathmoveto{\pgfpoint{131.238571pt}{26.399979pt}}
\pgflineto{\pgfpoint{132.234558pt}{26.399979pt}}
\pgfusepath{stroke}
\pgfpathmoveto{\pgfpoint{130.242584pt}{26.399979pt}}
\pgflineto{\pgfpoint{131.238571pt}{26.399979pt}}
\pgfusepath{stroke}
\pgfpathmoveto{\pgfpoint{129.246597pt}{26.399979pt}}
\pgflineto{\pgfpoint{130.242584pt}{26.399979pt}}
\pgfusepath{stroke}
\pgfpathmoveto{\pgfpoint{128.250610pt}{26.399979pt}}
\pgflineto{\pgfpoint{129.246597pt}{26.399979pt}}
\pgfusepath{stroke}
\pgfpathmoveto{\pgfpoint{127.254631pt}{26.399979pt}}
\pgflineto{\pgfpoint{128.250610pt}{26.399979pt}}
\pgfusepath{stroke}
\pgfpathmoveto{\pgfpoint{126.258652pt}{26.399979pt}}
\pgflineto{\pgfpoint{127.254631pt}{26.399979pt}}
\pgfusepath{stroke}
\pgfpathmoveto{\pgfpoint{125.262665pt}{26.399979pt}}
\pgflineto{\pgfpoint{126.258652pt}{26.399979pt}}
\pgfusepath{stroke}
\pgfpathmoveto{\pgfpoint{126.258652pt}{28.660576pt}}
\pgflineto{\pgfpoint{125.262665pt}{26.399979pt}}
\pgfusepath{stroke}
\pgfpathmoveto{\pgfpoint{127.254631pt}{29.838959pt}}
\pgflineto{\pgfpoint{126.258652pt}{28.660576pt}}
\pgfusepath{stroke}
\pgfpathmoveto{\pgfpoint{128.250610pt}{26.614319pt}}
\pgflineto{\pgfpoint{127.254631pt}{29.838959pt}}
\pgfusepath{stroke}
\pgfpathmoveto{\pgfpoint{129.246597pt}{26.451225pt}}
\pgflineto{\pgfpoint{128.250610pt}{26.614319pt}}
\pgfusepath{stroke}
\pgfpathmoveto{\pgfpoint{130.242584pt}{26.430634pt}}
\pgflineto{\pgfpoint{129.246597pt}{26.451225pt}}
\pgfusepath{stroke}
\pgfpathmoveto{\pgfpoint{131.238571pt}{27.958008pt}}
\pgflineto{\pgfpoint{130.242584pt}{26.430634pt}}
\pgfusepath{stroke}
\pgfpathmoveto{\pgfpoint{132.234558pt}{27.682930pt}}
\pgflineto{\pgfpoint{131.238571pt}{27.958008pt}}
\pgfusepath{stroke}
\pgfpathmoveto{\pgfpoint{133.230530pt}{26.399979pt}}
\pgflineto{\pgfpoint{132.234558pt}{27.682930pt}}
\pgfusepath{stroke}
\pgfpathmoveto{\pgfpoint{140.202423pt}{26.399979pt}}
\pgflineto{\pgfpoint{141.198410pt}{26.399979pt}}
\pgfusepath{stroke}
\pgfpathmoveto{\pgfpoint{139.206436pt}{26.399979pt}}
\pgflineto{\pgfpoint{140.202423pt}{26.399979pt}}
\pgfusepath{stroke}
\pgfpathmoveto{\pgfpoint{138.210449pt}{26.399979pt}}
\pgflineto{\pgfpoint{139.206436pt}{26.399979pt}}
\pgfusepath{stroke}
\pgfpathmoveto{\pgfpoint{139.206436pt}{27.485779pt}}
\pgflineto{\pgfpoint{138.210449pt}{26.399979pt}}
\pgfusepath{stroke}
\pgfpathmoveto{\pgfpoint{140.202423pt}{27.782883pt}}
\pgflineto{\pgfpoint{139.206436pt}{27.485779pt}}
\pgfusepath{stroke}
\pgfpathmoveto{\pgfpoint{141.198410pt}{26.399979pt}}
\pgflineto{\pgfpoint{140.202423pt}{27.782883pt}}
\pgfusepath{stroke}
\pgfpathmoveto{\pgfpoint{143.190369pt}{26.399979pt}}
\pgflineto{\pgfpoint{144.186356pt}{26.399979pt}}
\pgfusepath{stroke}
\pgfpathmoveto{\pgfpoint{142.194382pt}{26.399979pt}}
\pgflineto{\pgfpoint{143.190369pt}{26.399979pt}}
\pgfusepath{stroke}
\pgfpathmoveto{\pgfpoint{143.190369pt}{31.207901pt}}
\pgflineto{\pgfpoint{142.194382pt}{26.399979pt}}
\pgfusepath{stroke}
\pgfpathmoveto{\pgfpoint{144.186356pt}{26.399979pt}}
\pgflineto{\pgfpoint{143.190369pt}{31.207901pt}}
\pgfusepath{stroke}
\pgfpathmoveto{\pgfpoint{146.178314pt}{26.399979pt}}
\pgflineto{\pgfpoint{147.174316pt}{26.399979pt}}
\pgfusepath{stroke}
\pgfpathmoveto{\pgfpoint{145.182343pt}{26.399979pt}}
\pgflineto{\pgfpoint{146.178314pt}{26.399979pt}}
\pgfusepath{stroke}
\pgfpathmoveto{\pgfpoint{146.178314pt}{26.535568pt}}
\pgflineto{\pgfpoint{145.182343pt}{26.399979pt}}
\pgfusepath{stroke}
\pgfpathmoveto{\pgfpoint{147.174316pt}{26.399979pt}}
\pgflineto{\pgfpoint{146.178314pt}{26.535568pt}}
\pgfusepath{stroke}
\pgfpathmoveto{\pgfpoint{152.154221pt}{26.399979pt}}
\pgflineto{\pgfpoint{153.150208pt}{26.399979pt}}
\pgfusepath{stroke}
\pgfpathmoveto{\pgfpoint{151.158249pt}{26.399979pt}}
\pgflineto{\pgfpoint{152.154221pt}{26.399979pt}}
\pgfusepath{stroke}
\pgfpathmoveto{\pgfpoint{150.162262pt}{26.399979pt}}
\pgflineto{\pgfpoint{151.158249pt}{26.399979pt}}
\pgfusepath{stroke}
\pgfpathmoveto{\pgfpoint{151.158249pt}{26.439781pt}}
\pgflineto{\pgfpoint{150.162262pt}{26.399979pt}}
\pgfusepath{stroke}
\pgfpathmoveto{\pgfpoint{152.154221pt}{26.910515pt}}
\pgflineto{\pgfpoint{151.158249pt}{26.439781pt}}
\pgfusepath{stroke}
\pgfpathmoveto{\pgfpoint{153.150208pt}{26.399979pt}}
\pgflineto{\pgfpoint{152.154221pt}{26.910515pt}}
\pgfusepath{stroke}
\pgfpathmoveto{\pgfpoint{155.142181pt}{26.399979pt}}
\pgflineto{\pgfpoint{156.138168pt}{26.399979pt}}
\pgfusepath{stroke}
\pgfpathmoveto{\pgfpoint{154.146194pt}{26.399979pt}}
\pgflineto{\pgfpoint{155.142181pt}{26.399979pt}}
\pgfusepath{stroke}
\pgfpathmoveto{\pgfpoint{155.142181pt}{28.098244pt}}
\pgflineto{\pgfpoint{154.146194pt}{26.399979pt}}
\pgfusepath{stroke}
\pgfpathmoveto{\pgfpoint{156.138168pt}{26.399979pt}}
\pgflineto{\pgfpoint{155.142181pt}{28.098244pt}}
\pgfusepath{stroke}
\pgfpathmoveto{\pgfpoint{158.130127pt}{26.399979pt}}
\pgflineto{\pgfpoint{159.126114pt}{26.399979pt}}
\pgfusepath{stroke}
\pgfpathmoveto{\pgfpoint{157.134155pt}{26.399979pt}}
\pgflineto{\pgfpoint{158.130127pt}{26.399979pt}}
\pgfusepath{stroke}
\pgfpathmoveto{\pgfpoint{156.138168pt}{26.399979pt}}
\pgflineto{\pgfpoint{157.134155pt}{26.399979pt}}
\pgfusepath{stroke}
\pgfpathmoveto{\pgfpoint{157.134155pt}{26.492256pt}}
\pgflineto{\pgfpoint{156.138168pt}{26.399979pt}}
\pgfusepath{stroke}
\pgfpathmoveto{\pgfpoint{158.130127pt}{26.480469pt}}
\pgflineto{\pgfpoint{157.134155pt}{26.492256pt}}
\pgfusepath{stroke}
\pgfpathmoveto{\pgfpoint{159.126114pt}{26.399979pt}}
\pgflineto{\pgfpoint{158.130127pt}{26.480469pt}}
\pgfusepath{stroke}
\pgfpathmoveto{\pgfpoint{163.110062pt}{26.399979pt}}
\pgflineto{\pgfpoint{164.106033pt}{26.399979pt}}
\pgfusepath{stroke}
\pgfpathmoveto{\pgfpoint{162.114075pt}{26.399979pt}}
\pgflineto{\pgfpoint{163.110062pt}{26.399979pt}}
\pgfusepath{stroke}
\pgfpathmoveto{\pgfpoint{161.118088pt}{26.399979pt}}
\pgflineto{\pgfpoint{162.114075pt}{26.399979pt}}
\pgfusepath{stroke}
\pgfpathmoveto{\pgfpoint{160.122101pt}{26.399979pt}}
\pgflineto{\pgfpoint{161.118088pt}{26.399979pt}}
\pgfusepath{stroke}
\pgfpathmoveto{\pgfpoint{161.118088pt}{26.483330pt}}
\pgflineto{\pgfpoint{160.122101pt}{26.399979pt}}
\pgfusepath{stroke}
\pgfpathmoveto{\pgfpoint{162.114075pt}{26.604111pt}}
\pgflineto{\pgfpoint{161.118088pt}{26.483330pt}}
\pgfusepath{stroke}
\pgfpathmoveto{\pgfpoint{163.110062pt}{28.794250pt}}
\pgflineto{\pgfpoint{162.114075pt}{26.604111pt}}
\pgfusepath{stroke}
\pgfpathmoveto{\pgfpoint{164.106033pt}{26.399979pt}}
\pgflineto{\pgfpoint{163.110062pt}{28.794250pt}}
\pgfusepath{stroke}
\pgfpathmoveto{\pgfpoint{165.102020pt}{26.399979pt}}
\pgflineto{\pgfpoint{166.098007pt}{26.399979pt}}
\pgfusepath{stroke}
\pgfpathmoveto{\pgfpoint{164.106033pt}{26.399979pt}}
\pgflineto{\pgfpoint{165.102020pt}{26.399979pt}}
\pgfusepath{stroke}
\pgfpathmoveto{\pgfpoint{165.102020pt}{26.737396pt}}
\pgflineto{\pgfpoint{164.106033pt}{26.399979pt}}
\pgfusepath{stroke}
\pgfpathmoveto{\pgfpoint{166.098007pt}{26.399979pt}}
\pgflineto{\pgfpoint{165.102020pt}{26.737396pt}}
\pgfusepath{stroke}
\pgfpathmoveto{\pgfpoint{168.089966pt}{26.399979pt}}
\pgflineto{\pgfpoint{169.085953pt}{26.399979pt}}
\pgfusepath{stroke}
\pgfpathmoveto{\pgfpoint{167.093994pt}{26.399979pt}}
\pgflineto{\pgfpoint{168.089966pt}{26.399979pt}}
\pgfusepath{stroke}
\pgfpathmoveto{\pgfpoint{166.098007pt}{26.399979pt}}
\pgflineto{\pgfpoint{167.093994pt}{26.399979pt}}
\pgfusepath{stroke}
\pgfpathmoveto{\pgfpoint{167.093994pt}{30.536880pt}}
\pgflineto{\pgfpoint{166.098007pt}{26.399979pt}}
\pgfusepath{stroke}
\pgfpathmoveto{\pgfpoint{168.089966pt}{26.447098pt}}
\pgflineto{\pgfpoint{167.093994pt}{30.536880pt}}
\pgfusepath{stroke}
\pgfpathmoveto{\pgfpoint{169.085953pt}{26.399979pt}}
\pgflineto{\pgfpoint{168.089966pt}{26.447098pt}}
\pgfusepath{stroke}
\pgfpathmoveto{\pgfpoint{170.081940pt}{26.399979pt}}
\pgflineto{\pgfpoint{171.077911pt}{26.399979pt}}
\pgfusepath{stroke}
\pgfpathmoveto{\pgfpoint{169.085953pt}{26.399979pt}}
\pgflineto{\pgfpoint{170.081940pt}{26.399979pt}}
\pgfusepath{stroke}
\pgfpathmoveto{\pgfpoint{170.081940pt}{26.839508pt}}
\pgflineto{\pgfpoint{169.085953pt}{26.399979pt}}
\pgfusepath{stroke}
\pgfpathmoveto{\pgfpoint{171.077911pt}{26.399979pt}}
\pgflineto{\pgfpoint{170.081940pt}{26.839508pt}}
\pgfusepath{stroke}
\pgfpathmoveto{\pgfpoint{180.041779pt}{26.399979pt}}
\pgflineto{\pgfpoint{181.037766pt}{26.399979pt}}
\pgfusepath{stroke}
\pgfpathmoveto{\pgfpoint{179.045792pt}{26.399979pt}}
\pgflineto{\pgfpoint{180.041779pt}{26.399979pt}}
\pgfusepath{stroke}
\pgfpathmoveto{\pgfpoint{178.049805pt}{26.399979pt}}
\pgflineto{\pgfpoint{179.045792pt}{26.399979pt}}
\pgfusepath{stroke}
\pgfpathmoveto{\pgfpoint{177.053818pt}{26.399979pt}}
\pgflineto{\pgfpoint{178.049805pt}{26.399979pt}}
\pgfusepath{stroke}
\pgfpathmoveto{\pgfpoint{177.581543pt}{91.264786pt}}
\pgflineto{\pgfpoint{177.053818pt}{26.399979pt}}
\pgfusepath{stroke}
\pgfpathmoveto{\pgfpoint{179.045792pt}{26.504753pt}}
\pgflineto{\pgfpoint{178.518478pt}{91.264793pt}}
\pgfusepath{stroke}
\pgfpathmoveto{\pgfpoint{180.041779pt}{30.030716pt}}
\pgflineto{\pgfpoint{179.045792pt}{26.504753pt}}
\pgfusepath{stroke}
\pgfpathmoveto{\pgfpoint{181.037766pt}{26.399979pt}}
\pgflineto{\pgfpoint{180.041779pt}{30.030716pt}}
\pgfusepath{stroke}
\pgfpathmoveto{\pgfpoint{183.029724pt}{26.399979pt}}
\pgflineto{\pgfpoint{184.025711pt}{26.399979pt}}
\pgfusepath{stroke}
\pgfpathmoveto{\pgfpoint{182.033752pt}{26.399979pt}}
\pgflineto{\pgfpoint{183.029724pt}{26.399979pt}}
\pgfusepath{stroke}
\pgfpathmoveto{\pgfpoint{183.029724pt}{26.463455pt}}
\pgflineto{\pgfpoint{182.033752pt}{26.399979pt}}
\pgfusepath{stroke}
\pgfpathmoveto{\pgfpoint{184.025711pt}{26.399979pt}}
\pgflineto{\pgfpoint{183.029724pt}{26.463455pt}}
\pgfusepath{stroke}
\pgfpathmoveto{\pgfpoint{190.001617pt}{26.399979pt}}
\pgflineto{\pgfpoint{190.997604pt}{26.399979pt}}
\pgfusepath{stroke}
\pgfpathmoveto{\pgfpoint{189.005630pt}{26.399979pt}}
\pgflineto{\pgfpoint{190.001617pt}{26.399979pt}}
\pgfusepath{stroke}
\pgfpathmoveto{\pgfpoint{190.001617pt}{34.817444pt}}
\pgflineto{\pgfpoint{189.005630pt}{26.399979pt}}
\pgfusepath{stroke}
\pgfpathmoveto{\pgfpoint{190.997604pt}{26.399979pt}}
\pgflineto{\pgfpoint{190.001617pt}{34.817444pt}}
\pgfusepath{stroke}
\pgfpathmoveto{\pgfpoint{191.993591pt}{26.399979pt}}
\pgflineto{\pgfpoint{192.989563pt}{26.399979pt}}
\pgfusepath{stroke}
\pgfpathmoveto{\pgfpoint{190.997604pt}{26.399979pt}}
\pgflineto{\pgfpoint{191.993591pt}{26.399979pt}}
\pgfusepath{stroke}
\pgfpathmoveto{\pgfpoint{191.993591pt}{31.034332pt}}
\pgflineto{\pgfpoint{190.997604pt}{26.399979pt}}
\pgfusepath{stroke}
\pgfpathmoveto{\pgfpoint{192.989563pt}{26.399979pt}}
\pgflineto{\pgfpoint{191.993591pt}{31.034332pt}}
\pgfusepath{stroke}
\pgfpathmoveto{\pgfpoint{195.977524pt}{26.399979pt}}
\pgflineto{\pgfpoint{196.973511pt}{26.399979pt}}
\pgfusepath{stroke}
\pgfpathmoveto{\pgfpoint{194.981537pt}{26.399979pt}}
\pgflineto{\pgfpoint{195.977524pt}{26.399979pt}}
\pgfusepath{stroke}
\pgfpathmoveto{\pgfpoint{193.985565pt}{26.399979pt}}
\pgflineto{\pgfpoint{194.981537pt}{26.399979pt}}
\pgfusepath{stroke}
\pgfpathmoveto{\pgfpoint{194.981537pt}{28.872696pt}}
\pgflineto{\pgfpoint{193.985565pt}{26.399979pt}}
\pgfusepath{stroke}
\pgfpathmoveto{\pgfpoint{195.977524pt}{27.054726pt}}
\pgflineto{\pgfpoint{194.981537pt}{28.872696pt}}
\pgfusepath{stroke}
\pgfpathmoveto{\pgfpoint{196.973511pt}{26.399979pt}}
\pgflineto{\pgfpoint{195.977524pt}{27.054726pt}}
\pgfusepath{stroke}
\pgfpathmoveto{\pgfpoint{197.969498pt}{26.399979pt}}
\pgflineto{\pgfpoint{198.965469pt}{26.399979pt}}
\pgfusepath{stroke}
\pgfpathmoveto{\pgfpoint{196.973511pt}{26.399979pt}}
\pgflineto{\pgfpoint{197.969498pt}{26.399979pt}}
\pgfusepath{stroke}
\pgfpathmoveto{\pgfpoint{197.969498pt}{27.127144pt}}
\pgflineto{\pgfpoint{196.973511pt}{26.399979pt}}
\pgfusepath{stroke}
\pgfpathmoveto{\pgfpoint{198.965469pt}{26.399979pt}}
\pgflineto{\pgfpoint{197.969498pt}{27.127144pt}}
\pgfusepath{stroke}
\pgfpathmoveto{\pgfpoint{203.945404pt}{26.399979pt}}
\pgflineto{\pgfpoint{204.941376pt}{26.399979pt}}
\pgfusepath{stroke}
\pgfpathmoveto{\pgfpoint{202.949402pt}{26.399979pt}}
\pgflineto{\pgfpoint{203.945404pt}{26.399979pt}}
\pgfusepath{stroke}
\pgfpathmoveto{\pgfpoint{203.945404pt}{26.673592pt}}
\pgflineto{\pgfpoint{202.949402pt}{26.399979pt}}
\pgfusepath{stroke}
\pgfpathmoveto{\pgfpoint{204.941376pt}{26.399979pt}}
\pgflineto{\pgfpoint{203.945404pt}{26.673592pt}}
\pgfusepath{stroke}
\pgfpathmoveto{\pgfpoint{206.933334pt}{26.399979pt}}
\pgflineto{\pgfpoint{207.929337pt}{26.399979pt}}
\pgfusepath{stroke}
\pgfpathmoveto{\pgfpoint{205.937347pt}{26.399979pt}}
\pgflineto{\pgfpoint{206.933334pt}{26.399979pt}}
\pgfusepath{stroke}
\pgfpathmoveto{\pgfpoint{204.941376pt}{26.399979pt}}
\pgflineto{\pgfpoint{205.937347pt}{26.399979pt}}
\pgfusepath{stroke}
\pgfpathmoveto{\pgfpoint{205.937347pt}{29.413773pt}}
\pgflineto{\pgfpoint{204.941376pt}{26.399979pt}}
\pgfusepath{stroke}
\pgfpathmoveto{\pgfpoint{206.933334pt}{28.372978pt}}
\pgflineto{\pgfpoint{205.937347pt}{29.413773pt}}
\pgfusepath{stroke}
\pgfpathmoveto{\pgfpoint{207.929337pt}{26.399979pt}}
\pgflineto{\pgfpoint{206.933334pt}{28.372978pt}}
\pgfusepath{stroke}
\pgfpathmoveto{\pgfpoint{208.925323pt}{26.399979pt}}
\pgflineto{\pgfpoint{209.921295pt}{26.399979pt}}
\pgfusepath{stroke}
\pgfpathmoveto{\pgfpoint{207.929337pt}{26.399979pt}}
\pgflineto{\pgfpoint{208.925323pt}{26.399979pt}}
\pgfusepath{stroke}
\pgfpathmoveto{\pgfpoint{208.925323pt}{29.200577pt}}
\pgflineto{\pgfpoint{207.929337pt}{26.399979pt}}
\pgfusepath{stroke}
\pgfpathmoveto{\pgfpoint{209.921295pt}{26.399979pt}}
\pgflineto{\pgfpoint{208.925323pt}{29.200577pt}}
\pgfusepath{stroke}
\pgfpathmoveto{\pgfpoint{212.909241pt}{26.399979pt}}
\pgflineto{\pgfpoint{213.905228pt}{26.399979pt}}
\pgfusepath{stroke}
\pgfpathmoveto{\pgfpoint{211.913269pt}{26.399979pt}}
\pgflineto{\pgfpoint{212.909241pt}{26.399979pt}}
\pgfusepath{stroke}
\pgfpathmoveto{\pgfpoint{210.917267pt}{26.399979pt}}
\pgflineto{\pgfpoint{211.913269pt}{26.399979pt}}
\pgfusepath{stroke}
\pgfpathmoveto{\pgfpoint{209.921295pt}{26.399979pt}}
\pgflineto{\pgfpoint{210.917267pt}{26.399979pt}}
\pgfusepath{stroke}
\pgfpathmoveto{\pgfpoint{210.917267pt}{56.628197pt}}
\pgflineto{\pgfpoint{209.921295pt}{26.399979pt}}
\pgfusepath{stroke}
\pgfpathmoveto{\pgfpoint{211.913269pt}{30.417183pt}}
\pgflineto{\pgfpoint{210.917267pt}{56.628197pt}}
\pgfusepath{stroke}
\pgfpathmoveto{\pgfpoint{212.909241pt}{47.374290pt}}
\pgflineto{\pgfpoint{211.913269pt}{30.417183pt}}
\pgfusepath{stroke}
\pgfpathmoveto{\pgfpoint{213.905228pt}{26.399979pt}}
\pgflineto{\pgfpoint{212.909241pt}{47.374290pt}}
\pgfusepath{stroke}
\pgfpathmoveto{\pgfpoint{215.897217pt}{26.399979pt}}
\pgflineto{\pgfpoint{216.893188pt}{26.399979pt}}
\pgfusepath{stroke}
\pgfpathmoveto{\pgfpoint{214.901215pt}{26.399979pt}}
\pgflineto{\pgfpoint{215.897217pt}{26.399979pt}}
\pgfusepath{stroke}
\pgfpathmoveto{\pgfpoint{213.905228pt}{26.399979pt}}
\pgflineto{\pgfpoint{214.901215pt}{26.399979pt}}
\pgfusepath{stroke}
\pgfpathmoveto{\pgfpoint{214.901215pt}{27.407372pt}}
\pgflineto{\pgfpoint{213.905228pt}{26.399979pt}}
\pgfusepath{stroke}
\pgfpathmoveto{\pgfpoint{215.897217pt}{26.495361pt}}
\pgflineto{\pgfpoint{214.901215pt}{27.407372pt}}
\pgfusepath{stroke}
\pgfpathmoveto{\pgfpoint{216.893188pt}{26.399979pt}}
\pgflineto{\pgfpoint{215.897217pt}{26.495361pt}}
\pgfusepath{stroke}
\pgfpathmoveto{\pgfpoint{217.889160pt}{26.399979pt}}
\pgflineto{\pgfpoint{218.885147pt}{26.399979pt}}
\pgfusepath{stroke}
\pgfpathmoveto{\pgfpoint{216.893188pt}{26.399979pt}}
\pgflineto{\pgfpoint{217.889160pt}{26.399979pt}}
\pgfusepath{stroke}
\pgfpathmoveto{\pgfpoint{217.889160pt}{66.077660pt}}
\pgflineto{\pgfpoint{216.893188pt}{26.399979pt}}
\pgfusepath{stroke}
\pgfpathmoveto{\pgfpoint{218.885147pt}{26.399979pt}}
\pgflineto{\pgfpoint{217.889160pt}{66.077660pt}}
\pgfusepath{stroke}
\pgfpathmoveto{\pgfpoint{225.857040pt}{26.399979pt}}
\pgflineto{\pgfpoint{226.853027pt}{26.399979pt}}
\pgfusepath{stroke}
\pgfpathmoveto{\pgfpoint{224.861053pt}{26.399979pt}}
\pgflineto{\pgfpoint{225.857040pt}{26.399979pt}}
\pgfusepath{stroke}
\pgfpathmoveto{\pgfpoint{225.857040pt}{28.125343pt}}
\pgflineto{\pgfpoint{224.861053pt}{26.399979pt}}
\pgfusepath{stroke}
\pgfpathmoveto{\pgfpoint{226.853027pt}{26.399979pt}}
\pgflineto{\pgfpoint{225.857040pt}{28.125343pt}}
\pgfusepath{stroke}
\pgfpathmoveto{\pgfpoint{230.836945pt}{26.399979pt}}
\pgflineto{\pgfpoint{231.832932pt}{26.399979pt}}
\pgfusepath{stroke}
\pgfpathmoveto{\pgfpoint{229.840973pt}{26.399979pt}}
\pgflineto{\pgfpoint{230.836945pt}{26.399979pt}}
\pgfusepath{stroke}
\pgfpathmoveto{\pgfpoint{228.845001pt}{26.399979pt}}
\pgflineto{\pgfpoint{229.840973pt}{26.399979pt}}
\pgfusepath{stroke}
\pgfpathmoveto{\pgfpoint{227.849014pt}{26.399979pt}}
\pgflineto{\pgfpoint{228.845001pt}{26.399979pt}}
\pgfusepath{stroke}
\pgfpathmoveto{\pgfpoint{228.845001pt}{27.012459pt}}
\pgflineto{\pgfpoint{227.849014pt}{26.399979pt}}
\pgfusepath{stroke}
\pgfpathmoveto{\pgfpoint{229.840973pt}{26.558662pt}}
\pgflineto{\pgfpoint{228.845001pt}{27.012459pt}}
\pgfusepath{stroke}
\pgfpathmoveto{\pgfpoint{230.836945pt}{26.682472pt}}
\pgflineto{\pgfpoint{229.840973pt}{26.558662pt}}
\pgfusepath{stroke}
\pgfpathmoveto{\pgfpoint{231.832932pt}{26.399979pt}}
\pgflineto{\pgfpoint{230.836945pt}{26.682472pt}}
\pgfusepath{stroke}
\pgfpathmoveto{\pgfpoint{237.808838pt}{26.399979pt}}
\pgflineto{\pgfpoint{238.804825pt}{26.399979pt}}
\pgfusepath{stroke}
\pgfpathmoveto{\pgfpoint{236.812866pt}{26.399979pt}}
\pgflineto{\pgfpoint{237.808838pt}{26.399979pt}}
\pgfusepath{stroke}
\pgfpathmoveto{\pgfpoint{235.816864pt}{26.399979pt}}
\pgflineto{\pgfpoint{236.812866pt}{26.399979pt}}
\pgfusepath{stroke}
\pgfpathmoveto{\pgfpoint{236.812866pt}{35.730797pt}}
\pgflineto{\pgfpoint{235.816864pt}{26.399979pt}}
\pgfusepath{stroke}
\pgfpathmoveto{\pgfpoint{237.808838pt}{26.516998pt}}
\pgflineto{\pgfpoint{236.812866pt}{35.730797pt}}
\pgfusepath{stroke}
\pgfpathmoveto{\pgfpoint{238.804825pt}{26.399979pt}}
\pgflineto{\pgfpoint{237.808838pt}{26.516998pt}}
\pgfusepath{stroke}
\pgfpathmoveto{\pgfpoint{243.784744pt}{26.399979pt}}
\pgflineto{\pgfpoint{244.780731pt}{26.399979pt}}
\pgfusepath{stroke}
\pgfpathmoveto{\pgfpoint{242.788757pt}{26.399979pt}}
\pgflineto{\pgfpoint{243.784744pt}{26.399979pt}}
\pgfusepath{stroke}
\pgfpathmoveto{\pgfpoint{241.792786pt}{26.399979pt}}
\pgflineto{\pgfpoint{242.788757pt}{26.399979pt}}
\pgfusepath{stroke}
\pgfpathmoveto{\pgfpoint{242.788757pt}{26.819893pt}}
\pgflineto{\pgfpoint{241.792786pt}{26.399979pt}}
\pgfusepath{stroke}
\pgfpathmoveto{\pgfpoint{243.784744pt}{27.221687pt}}
\pgflineto{\pgfpoint{242.788757pt}{26.819893pt}}
\pgfusepath{stroke}
\pgfpathmoveto{\pgfpoint{244.780731pt}{26.399979pt}}
\pgflineto{\pgfpoint{243.784744pt}{27.221687pt}}
\pgfusepath{stroke}
\pgfpathmoveto{\pgfpoint{247.768677pt}{26.399979pt}}
\pgflineto{\pgfpoint{248.764679pt}{26.399979pt}}
\pgfusepath{stroke}
\pgfpathmoveto{\pgfpoint{246.772705pt}{26.399979pt}}
\pgflineto{\pgfpoint{247.768677pt}{26.399979pt}}
\pgfusepath{stroke}
\pgfpathmoveto{\pgfpoint{245.776718pt}{26.399979pt}}
\pgflineto{\pgfpoint{246.772705pt}{26.399979pt}}
\pgfusepath{stroke}
\pgfpathmoveto{\pgfpoint{244.780731pt}{26.399979pt}}
\pgflineto{\pgfpoint{245.776718pt}{26.399979pt}}
\pgfusepath{stroke}
\pgfpathmoveto{\pgfpoint{245.776718pt}{26.477646pt}}
\pgflineto{\pgfpoint{244.780731pt}{26.399979pt}}
\pgfusepath{stroke}
\pgfpathmoveto{\pgfpoint{246.772705pt}{26.550812pt}}
\pgflineto{\pgfpoint{245.776718pt}{26.477646pt}}
\pgfusepath{stroke}
\pgfpathmoveto{\pgfpoint{247.768677pt}{41.200562pt}}
\pgflineto{\pgfpoint{246.772705pt}{26.550812pt}}
\pgfusepath{stroke}
\pgfpathmoveto{\pgfpoint{248.764679pt}{26.399979pt}}
\pgflineto{\pgfpoint{247.768677pt}{41.200562pt}}
\pgfusepath{stroke}
\pgfpathmoveto{\pgfpoint{250.756638pt}{26.399979pt}}
\pgflineto{\pgfpoint{251.752625pt}{26.399979pt}}
\pgfusepath{stroke}
\pgfpathmoveto{\pgfpoint{249.760651pt}{26.399979pt}}
\pgflineto{\pgfpoint{250.756638pt}{26.399979pt}}
\pgfusepath{stroke}
\pgfpathmoveto{\pgfpoint{248.764679pt}{26.399979pt}}
\pgflineto{\pgfpoint{249.760651pt}{26.399979pt}}
\pgfusepath{stroke}
\pgfpathmoveto{\pgfpoint{249.760651pt}{26.474113pt}}
\pgflineto{\pgfpoint{248.764679pt}{26.399979pt}}
\pgfusepath{stroke}
\pgfpathmoveto{\pgfpoint{250.756638pt}{26.708969pt}}
\pgflineto{\pgfpoint{249.760651pt}{26.474113pt}}
\pgfusepath{stroke}
\pgfpathmoveto{\pgfpoint{251.752625pt}{26.399979pt}}
\pgflineto{\pgfpoint{250.756638pt}{26.708969pt}}
\pgfusepath{stroke}
\pgfpathmoveto{\pgfpoint{254.740570pt}{26.399979pt}}
\pgflineto{\pgfpoint{255.736542pt}{26.399979pt}}
\pgfusepath{stroke}
\pgfpathmoveto{\pgfpoint{253.744598pt}{26.399979pt}}
\pgflineto{\pgfpoint{254.740570pt}{26.399979pt}}
\pgfusepath{stroke}
\pgfpathmoveto{\pgfpoint{254.740570pt}{27.508163pt}}
\pgflineto{\pgfpoint{253.744598pt}{26.399979pt}}
\pgfusepath{stroke}
\pgfpathmoveto{\pgfpoint{255.736542pt}{26.399979pt}}
\pgflineto{\pgfpoint{254.740570pt}{27.508163pt}}
\pgfusepath{stroke}
\pgfpathmoveto{\pgfpoint{258.724518pt}{26.399979pt}}
\pgflineto{\pgfpoint{259.720490pt}{26.399979pt}}
\pgfusepath{stroke}
\pgfpathmoveto{\pgfpoint{257.728516pt}{26.399979pt}}
\pgflineto{\pgfpoint{258.724518pt}{26.399979pt}}
\pgfusepath{stroke}
\pgfpathmoveto{\pgfpoint{258.724518pt}{26.886185pt}}
\pgflineto{\pgfpoint{257.728516pt}{26.399979pt}}
\pgfusepath{stroke}
\pgfpathmoveto{\pgfpoint{259.720490pt}{26.399979pt}}
\pgflineto{\pgfpoint{258.724518pt}{26.886185pt}}
\pgfusepath{stroke}
\pgfpathmoveto{\pgfpoint{261.712463pt}{26.399979pt}}
\pgflineto{\pgfpoint{262.708435pt}{26.399979pt}}
\pgfusepath{stroke}
\pgfpathmoveto{\pgfpoint{260.716492pt}{26.399979pt}}
\pgflineto{\pgfpoint{261.712463pt}{26.399979pt}}
\pgfusepath{stroke}
\pgfpathmoveto{\pgfpoint{261.712463pt}{27.286751pt}}
\pgflineto{\pgfpoint{260.716492pt}{26.399979pt}}
\pgfusepath{stroke}
\pgfpathmoveto{\pgfpoint{262.708435pt}{26.399979pt}}
\pgflineto{\pgfpoint{261.712463pt}{27.286751pt}}
\pgfusepath{stroke}
\pgfpathmoveto{\pgfpoint{264.700409pt}{26.399979pt}}
\pgflineto{\pgfpoint{265.696411pt}{26.399979pt}}
\pgfusepath{stroke}
\pgfpathmoveto{\pgfpoint{263.704407pt}{26.399979pt}}
\pgflineto{\pgfpoint{264.700409pt}{26.399979pt}}
\pgfusepath{stroke}
\pgfpathmoveto{\pgfpoint{264.700409pt}{26.932861pt}}
\pgflineto{\pgfpoint{263.704407pt}{26.399979pt}}
\pgfusepath{stroke}
\pgfpathmoveto{\pgfpoint{265.696411pt}{26.399979pt}}
\pgflineto{\pgfpoint{264.700409pt}{26.932861pt}}
\pgfusepath{stroke}
\pgfpathmoveto{\pgfpoint{266.692383pt}{26.399979pt}}
\pgflineto{\pgfpoint{267.688354pt}{26.399979pt}}
\pgfusepath{stroke}
\pgfpathmoveto{\pgfpoint{265.696411pt}{26.399979pt}}
\pgflineto{\pgfpoint{266.692383pt}{26.399979pt}}
\pgfusepath{stroke}
\pgfpathmoveto{\pgfpoint{266.692383pt}{26.914734pt}}
\pgflineto{\pgfpoint{265.696411pt}{26.399979pt}}
\pgfusepath{stroke}
\pgfpathmoveto{\pgfpoint{267.688354pt}{26.399979pt}}
\pgflineto{\pgfpoint{266.692383pt}{26.914734pt}}
\pgfusepath{stroke}
\pgfpathmoveto{\pgfpoint{272.668274pt}{26.399979pt}}
\pgflineto{\pgfpoint{273.664276pt}{26.399979pt}}
\pgfusepath{stroke}
\pgfpathmoveto{\pgfpoint{271.672302pt}{26.399979pt}}
\pgflineto{\pgfpoint{272.668274pt}{26.399979pt}}
\pgfusepath{stroke}
\pgfpathmoveto{\pgfpoint{270.676331pt}{26.399979pt}}
\pgflineto{\pgfpoint{271.672302pt}{26.399979pt}}
\pgfusepath{stroke}
\pgfpathmoveto{\pgfpoint{269.680328pt}{26.399979pt}}
\pgflineto{\pgfpoint{270.676331pt}{26.399979pt}}
\pgfusepath{stroke}
\pgfpathmoveto{\pgfpoint{268.684326pt}{26.399979pt}}
\pgflineto{\pgfpoint{269.680328pt}{26.399979pt}}
\pgfusepath{stroke}
\pgfpathmoveto{\pgfpoint{267.688354pt}{26.399979pt}}
\pgflineto{\pgfpoint{268.684326pt}{26.399979pt}}
\pgfusepath{stroke}
\pgfpathmoveto{\pgfpoint{268.684326pt}{26.647873pt}}
\pgflineto{\pgfpoint{267.688354pt}{26.399979pt}}
\pgfusepath{stroke}
\pgfpathmoveto{\pgfpoint{269.680328pt}{26.596115pt}}
\pgflineto{\pgfpoint{268.684326pt}{26.647873pt}}
\pgfusepath{stroke}
\pgfpathmoveto{\pgfpoint{270.676331pt}{27.976303pt}}
\pgflineto{\pgfpoint{269.680328pt}{26.596115pt}}
\pgfusepath{stroke}
\pgfpathmoveto{\pgfpoint{271.672302pt}{42.660454pt}}
\pgflineto{\pgfpoint{270.676331pt}{27.976303pt}}
\pgfusepath{stroke}
\pgfpathmoveto{\pgfpoint{272.668274pt}{26.867470pt}}
\pgflineto{\pgfpoint{271.672302pt}{42.660454pt}}
\pgfusepath{stroke}
\pgfpathmoveto{\pgfpoint{273.664276pt}{26.399979pt}}
\pgflineto{\pgfpoint{272.668274pt}{26.867470pt}}
\pgfusepath{stroke}
\pgfpathmoveto{\pgfpoint{279.640167pt}{26.399979pt}}
\pgflineto{\pgfpoint{280.636139pt}{26.399979pt}}
\pgfusepath{stroke}
\pgfpathmoveto{\pgfpoint{278.644196pt}{26.399979pt}}
\pgflineto{\pgfpoint{279.640167pt}{26.399979pt}}
\pgfusepath{stroke}
\pgfpathmoveto{\pgfpoint{277.648193pt}{26.399979pt}}
\pgflineto{\pgfpoint{278.644196pt}{26.399979pt}}
\pgfusepath{stroke}
\pgfpathmoveto{\pgfpoint{278.644196pt}{30.203644pt}}
\pgflineto{\pgfpoint{277.648193pt}{26.399979pt}}
\pgfusepath{stroke}
\pgfpathmoveto{\pgfpoint{279.640167pt}{35.907738pt}}
\pgflineto{\pgfpoint{278.644196pt}{30.203644pt}}
\pgfusepath{stroke}
\pgfpathmoveto{\pgfpoint{280.636139pt}{26.399979pt}}
\pgflineto{\pgfpoint{279.640167pt}{35.907738pt}}
\pgfusepath{stroke}
\pgfpathmoveto{\pgfpoint{283.624115pt}{26.399979pt}}
\pgflineto{\pgfpoint{284.620087pt}{26.399979pt}}
\pgfusepath{stroke}
\pgfpathmoveto{\pgfpoint{282.628113pt}{26.399979pt}}
\pgflineto{\pgfpoint{283.624115pt}{26.399979pt}}
\pgfusepath{stroke}
\pgfpathmoveto{\pgfpoint{281.632141pt}{26.399979pt}}
\pgflineto{\pgfpoint{282.628113pt}{26.399979pt}}
\pgfusepath{stroke}
\pgfpathmoveto{\pgfpoint{282.628113pt}{29.464996pt}}
\pgflineto{\pgfpoint{281.632141pt}{26.399979pt}}
\pgfusepath{stroke}
\pgfpathmoveto{\pgfpoint{283.624115pt}{28.196754pt}}
\pgflineto{\pgfpoint{282.628113pt}{29.464996pt}}
\pgfusepath{stroke}
\pgfpathmoveto{\pgfpoint{284.620087pt}{26.399979pt}}
\pgflineto{\pgfpoint{283.624115pt}{28.196754pt}}
\pgfusepath{stroke}
\pgfpathmoveto{\pgfpoint{285.616089pt}{26.399979pt}}
\pgflineto{\pgfpoint{286.612061pt}{26.399979pt}}
\pgfusepath{stroke}
\pgfpathmoveto{\pgfpoint{284.620087pt}{26.399979pt}}
\pgflineto{\pgfpoint{285.616089pt}{26.399979pt}}
\pgfusepath{stroke}
\pgfpathmoveto{\pgfpoint{285.616089pt}{26.474419pt}}
\pgflineto{\pgfpoint{284.620087pt}{26.399979pt}}
\pgfusepath{stroke}
\pgfpathmoveto{\pgfpoint{286.612061pt}{26.399979pt}}
\pgflineto{\pgfpoint{285.616089pt}{26.474419pt}}
\pgfusepath{stroke}
\color[rgb]{0.000000,0.000000,1.000000}
\pgfsetlinewidth{2.000000pt}
\pgfpathmoveto{\pgfpoint{42.595993pt}{26.399979pt}}
\pgflineto{\pgfpoint{41.600006pt}{26.399979pt}}
\pgfusepath{stroke}
\pgfpathmoveto{\pgfpoint{43.591980pt}{26.721756pt}}
\pgflineto{\pgfpoint{42.595993pt}{26.399979pt}}
\pgfusepath{stroke}
\pgfpathmoveto{\pgfpoint{44.587967pt}{26.399979pt}}
\pgflineto{\pgfpoint{43.591980pt}{26.721756pt}}
\pgfusepath{stroke}
\pgfpathmoveto{\pgfpoint{45.583946pt}{26.399979pt}}
\pgflineto{\pgfpoint{44.587967pt}{26.399979pt}}
\pgfusepath{stroke}
\pgfpathmoveto{\pgfpoint{46.579933pt}{26.399979pt}}
\pgflineto{\pgfpoint{45.583946pt}{26.399979pt}}
\pgfusepath{stroke}
\pgfpathmoveto{\pgfpoint{47.575912pt}{26.440331pt}}
\pgflineto{\pgfpoint{46.579933pt}{26.399979pt}}
\pgfusepath{stroke}
\pgfpathmoveto{\pgfpoint{48.571899pt}{26.399979pt}}
\pgflineto{\pgfpoint{47.575912pt}{26.440331pt}}
\pgfusepath{stroke}
\pgfpathmoveto{\pgfpoint{49.567879pt}{26.399979pt}}
\pgflineto{\pgfpoint{48.571899pt}{26.399979pt}}
\pgfusepath{stroke}
\pgfpathmoveto{\pgfpoint{50.563873pt}{26.399979pt}}
\pgflineto{\pgfpoint{49.567879pt}{26.399979pt}}
\pgfusepath{stroke}
\pgfpathmoveto{\pgfpoint{51.559845pt}{26.399979pt}}
\pgflineto{\pgfpoint{50.563873pt}{26.399979pt}}
\pgfusepath{stroke}
\pgfpathmoveto{\pgfpoint{52.555840pt}{26.399979pt}}
\pgflineto{\pgfpoint{51.559845pt}{26.399979pt}}
\pgfusepath{stroke}
\pgfpathmoveto{\pgfpoint{53.551819pt}{27.127853pt}}
\pgflineto{\pgfpoint{52.555840pt}{26.399979pt}}
\pgfusepath{stroke}
\pgfpathmoveto{\pgfpoint{54.547806pt}{26.399979pt}}
\pgflineto{\pgfpoint{53.551819pt}{27.127853pt}}
\pgfusepath{stroke}
\pgfpathmoveto{\pgfpoint{55.543785pt}{26.408897pt}}
\pgflineto{\pgfpoint{54.547806pt}{26.399979pt}}
\pgfusepath{stroke}
\pgfpathmoveto{\pgfpoint{56.539772pt}{28.370888pt}}
\pgflineto{\pgfpoint{55.543785pt}{26.408897pt}}
\pgfusepath{stroke}
\pgfpathmoveto{\pgfpoint{57.535751pt}{26.399979pt}}
\pgflineto{\pgfpoint{56.539772pt}{28.370888pt}}
\pgfusepath{stroke}
\pgfpathmoveto{\pgfpoint{58.531738pt}{26.430054pt}}
\pgflineto{\pgfpoint{57.535751pt}{26.399979pt}}
\pgfusepath{stroke}
\pgfpathmoveto{\pgfpoint{59.527725pt}{26.413170pt}}
\pgflineto{\pgfpoint{58.531738pt}{26.430054pt}}
\pgfusepath{stroke}
\pgfpathmoveto{\pgfpoint{60.523712pt}{26.399979pt}}
\pgflineto{\pgfpoint{59.527725pt}{26.413170pt}}
\pgfusepath{stroke}
\pgfpathmoveto{\pgfpoint{61.519691pt}{26.399979pt}}
\pgflineto{\pgfpoint{60.523712pt}{26.399979pt}}
\pgfusepath{stroke}
\pgfpathmoveto{\pgfpoint{62.515678pt}{26.401459pt}}
\pgflineto{\pgfpoint{61.519691pt}{26.399979pt}}
\pgfusepath{stroke}
\pgfpathmoveto{\pgfpoint{63.511658pt}{26.399979pt}}
\pgflineto{\pgfpoint{62.515678pt}{26.401459pt}}
\pgfusepath{stroke}
\pgfpathmoveto{\pgfpoint{64.507637pt}{26.409256pt}}
\pgflineto{\pgfpoint{63.511658pt}{26.399979pt}}
\pgfusepath{stroke}
\pgfpathmoveto{\pgfpoint{65.503624pt}{26.424644pt}}
\pgflineto{\pgfpoint{64.507637pt}{26.409256pt}}
\pgfusepath{stroke}
\pgfpathmoveto{\pgfpoint{66.499619pt}{26.614471pt}}
\pgflineto{\pgfpoint{65.503624pt}{26.424644pt}}
\pgfusepath{stroke}
\pgfpathmoveto{\pgfpoint{67.495590pt}{26.399979pt}}
\pgflineto{\pgfpoint{66.499619pt}{26.614471pt}}
\pgfusepath{stroke}
\pgfpathmoveto{\pgfpoint{68.491577pt}{26.404778pt}}
\pgflineto{\pgfpoint{67.495590pt}{26.399979pt}}
\pgfusepath{stroke}
\pgfpathmoveto{\pgfpoint{69.487564pt}{26.463905pt}}
\pgflineto{\pgfpoint{68.491577pt}{26.404778pt}}
\pgfusepath{stroke}
\pgfpathmoveto{\pgfpoint{70.483551pt}{26.399979pt}}
\pgflineto{\pgfpoint{69.487564pt}{26.463905pt}}
\pgfusepath{stroke}
\pgfpathmoveto{\pgfpoint{71.479530pt}{29.584541pt}}
\pgflineto{\pgfpoint{70.483551pt}{26.399979pt}}
\pgfusepath{stroke}
\pgfpathmoveto{\pgfpoint{72.475510pt}{26.503563pt}}
\pgflineto{\pgfpoint{71.479530pt}{29.584541pt}}
\pgfusepath{stroke}
\pgfpathmoveto{\pgfpoint{73.471497pt}{26.399979pt}}
\pgflineto{\pgfpoint{72.475510pt}{26.503563pt}}
\pgfusepath{stroke}
\pgfpathmoveto{\pgfpoint{74.467484pt}{26.431870pt}}
\pgflineto{\pgfpoint{73.471497pt}{26.399979pt}}
\pgfusepath{stroke}
\pgfpathmoveto{\pgfpoint{75.463470pt}{26.402107pt}}
\pgflineto{\pgfpoint{74.467484pt}{26.431870pt}}
\pgfusepath{stroke}
\pgfpathmoveto{\pgfpoint{76.459442pt}{26.399979pt}}
\pgflineto{\pgfpoint{75.463470pt}{26.402107pt}}
\pgfusepath{stroke}
\pgfpathmoveto{\pgfpoint{77.455437pt}{26.399979pt}}
\pgflineto{\pgfpoint{76.459442pt}{26.399979pt}}
\pgfusepath{stroke}
\pgfpathmoveto{\pgfpoint{78.451424pt}{26.399979pt}}
\pgflineto{\pgfpoint{77.455437pt}{26.399979pt}}
\pgfusepath{stroke}
\pgfpathmoveto{\pgfpoint{79.447403pt}{26.399979pt}}
\pgflineto{\pgfpoint{78.451424pt}{26.399979pt}}
\pgfusepath{stroke}
\pgfpathmoveto{\pgfpoint{80.443390pt}{26.744049pt}}
\pgflineto{\pgfpoint{79.447403pt}{26.399979pt}}
\pgfusepath{stroke}
\pgfpathmoveto{\pgfpoint{81.439369pt}{26.646423pt}}
\pgflineto{\pgfpoint{80.443390pt}{26.744049pt}}
\pgfusepath{stroke}
\pgfpathmoveto{\pgfpoint{82.435356pt}{26.442352pt}}
\pgflineto{\pgfpoint{81.439369pt}{26.646423pt}}
\pgfusepath{stroke}
\pgfpathmoveto{\pgfpoint{83.431335pt}{26.399979pt}}
\pgflineto{\pgfpoint{82.435356pt}{26.442352pt}}
\pgfusepath{stroke}
\pgfpathmoveto{\pgfpoint{84.427322pt}{26.399979pt}}
\pgflineto{\pgfpoint{83.431335pt}{26.399979pt}}
\pgfusepath{stroke}
\pgfpathmoveto{\pgfpoint{85.423309pt}{26.399979pt}}
\pgflineto{\pgfpoint{84.427322pt}{26.399979pt}}
\pgfusepath{stroke}
\pgfpathmoveto{\pgfpoint{86.419289pt}{26.407837pt}}
\pgflineto{\pgfpoint{85.423309pt}{26.399979pt}}
\pgfusepath{stroke}
\pgfpathmoveto{\pgfpoint{87.415276pt}{26.399979pt}}
\pgflineto{\pgfpoint{86.419289pt}{26.407837pt}}
\pgfusepath{stroke}
\pgfpathmoveto{\pgfpoint{88.411255pt}{26.617722pt}}
\pgflineto{\pgfpoint{87.415276pt}{26.399979pt}}
\pgfusepath{stroke}
\pgfpathmoveto{\pgfpoint{89.407242pt}{26.399979pt}}
\pgflineto{\pgfpoint{88.411255pt}{26.617722pt}}
\pgfusepath{stroke}
\pgfpathmoveto{\pgfpoint{90.403221pt}{26.400894pt}}
\pgflineto{\pgfpoint{89.407242pt}{26.399979pt}}
\pgfusepath{stroke}
\pgfpathmoveto{\pgfpoint{91.399208pt}{26.399979pt}}
\pgflineto{\pgfpoint{90.403221pt}{26.400894pt}}
\pgfusepath{stroke}
\pgfpathmoveto{\pgfpoint{92.395187pt}{26.413033pt}}
\pgflineto{\pgfpoint{91.399208pt}{26.399979pt}}
\pgfusepath{stroke}
\pgfpathmoveto{\pgfpoint{93.391174pt}{26.413887pt}}
\pgflineto{\pgfpoint{92.395187pt}{26.413033pt}}
\pgfusepath{stroke}
\pgfpathmoveto{\pgfpoint{94.387161pt}{26.399979pt}}
\pgflineto{\pgfpoint{93.391174pt}{26.413887pt}}
\pgfusepath{stroke}
\pgfpathmoveto{\pgfpoint{95.383141pt}{26.399979pt}}
\pgflineto{\pgfpoint{94.387161pt}{26.399979pt}}
\pgfusepath{stroke}
\pgfpathmoveto{\pgfpoint{96.379128pt}{26.399979pt}}
\pgflineto{\pgfpoint{95.383141pt}{26.399979pt}}
\pgfusepath{stroke}
\pgfpathmoveto{\pgfpoint{97.375107pt}{26.399979pt}}
\pgflineto{\pgfpoint{96.379128pt}{26.399979pt}}
\pgfusepath{stroke}
\pgfpathmoveto{\pgfpoint{98.371094pt}{27.362320pt}}
\pgflineto{\pgfpoint{97.375107pt}{26.399979pt}}
\pgfusepath{stroke}
\pgfpathmoveto{\pgfpoint{99.367081pt}{30.703705pt}}
\pgflineto{\pgfpoint{98.371094pt}{27.362320pt}}
\pgfusepath{stroke}
\pgfpathmoveto{\pgfpoint{100.363068pt}{26.755974pt}}
\pgflineto{\pgfpoint{99.367081pt}{30.703705pt}}
\pgfusepath{stroke}
\pgfpathmoveto{\pgfpoint{101.359047pt}{26.399979pt}}
\pgflineto{\pgfpoint{100.363068pt}{26.755974pt}}
\pgfusepath{stroke}
\pgfpathmoveto{\pgfpoint{102.355034pt}{26.436134pt}}
\pgflineto{\pgfpoint{101.359047pt}{26.399979pt}}
\pgfusepath{stroke}
\pgfpathmoveto{\pgfpoint{103.351013pt}{26.399979pt}}
\pgflineto{\pgfpoint{102.355034pt}{26.436134pt}}
\pgfusepath{stroke}
\pgfpathmoveto{\pgfpoint{104.347000pt}{27.051666pt}}
\pgflineto{\pgfpoint{103.351013pt}{26.399979pt}}
\pgfusepath{stroke}
\pgfpathmoveto{\pgfpoint{105.342987pt}{26.399979pt}}
\pgflineto{\pgfpoint{104.347000pt}{27.051666pt}}
\pgfusepath{stroke}
\pgfpathmoveto{\pgfpoint{106.338966pt}{26.399979pt}}
\pgflineto{\pgfpoint{105.342987pt}{26.399979pt}}
\pgfusepath{stroke}
\pgfpathmoveto{\pgfpoint{107.334953pt}{26.399979pt}}
\pgflineto{\pgfpoint{106.338966pt}{26.399979pt}}
\pgfusepath{stroke}
\pgfpathmoveto{\pgfpoint{108.330933pt}{26.421722pt}}
\pgflineto{\pgfpoint{107.334953pt}{26.399979pt}}
\pgfusepath{stroke}
\pgfpathmoveto{\pgfpoint{109.326920pt}{26.454926pt}}
\pgflineto{\pgfpoint{108.330933pt}{26.421722pt}}
\pgfusepath{stroke}
\pgfpathmoveto{\pgfpoint{110.322906pt}{26.399979pt}}
\pgflineto{\pgfpoint{109.326920pt}{26.454926pt}}
\pgfusepath{stroke}
\pgfpathmoveto{\pgfpoint{111.318893pt}{26.399979pt}}
\pgflineto{\pgfpoint{110.322906pt}{26.399979pt}}
\pgfusepath{stroke}
\pgfpathmoveto{\pgfpoint{112.314873pt}{27.245140pt}}
\pgflineto{\pgfpoint{111.318893pt}{26.399979pt}}
\pgfusepath{stroke}
\pgfpathmoveto{\pgfpoint{113.310852pt}{26.404144pt}}
\pgflineto{\pgfpoint{112.314873pt}{27.245140pt}}
\pgfusepath{stroke}
\pgfpathmoveto{\pgfpoint{114.306839pt}{26.399979pt}}
\pgflineto{\pgfpoint{113.310852pt}{26.404144pt}}
\pgfusepath{stroke}
\pgfpathmoveto{\pgfpoint{115.302826pt}{26.399979pt}}
\pgflineto{\pgfpoint{114.306839pt}{26.399979pt}}
\pgfusepath{stroke}
\pgfpathmoveto{\pgfpoint{116.298813pt}{26.625999pt}}
\pgflineto{\pgfpoint{115.302826pt}{26.399979pt}}
\pgfusepath{stroke}
\pgfpathmoveto{\pgfpoint{117.294792pt}{26.408501pt}}
\pgflineto{\pgfpoint{116.298813pt}{26.625999pt}}
\pgfusepath{stroke}
\pgfpathmoveto{\pgfpoint{118.290779pt}{41.526093pt}}
\pgflineto{\pgfpoint{117.294792pt}{26.408501pt}}
\pgfusepath{stroke}
\pgfpathmoveto{\pgfpoint{119.286758pt}{26.399979pt}}
\pgflineto{\pgfpoint{118.290779pt}{41.526093pt}}
\pgfusepath{stroke}
\pgfpathmoveto{\pgfpoint{120.282745pt}{26.399979pt}}
\pgflineto{\pgfpoint{119.286758pt}{26.399979pt}}
\pgfusepath{stroke}
\pgfpathmoveto{\pgfpoint{121.278725pt}{26.441711pt}}
\pgflineto{\pgfpoint{120.282745pt}{26.399979pt}}
\pgfusepath{stroke}
\pgfpathmoveto{\pgfpoint{122.274712pt}{26.399979pt}}
\pgflineto{\pgfpoint{121.278725pt}{26.441711pt}}
\pgfusepath{stroke}
\pgfpathmoveto{\pgfpoint{123.270691pt}{26.409576pt}}
\pgflineto{\pgfpoint{122.274712pt}{26.399979pt}}
\pgfusepath{stroke}
\pgfpathmoveto{\pgfpoint{124.266678pt}{26.399979pt}}
\pgflineto{\pgfpoint{123.270691pt}{26.409576pt}}
\pgfusepath{stroke}
\pgfpathmoveto{\pgfpoint{125.262665pt}{26.399979pt}}
\pgflineto{\pgfpoint{124.266678pt}{26.399979pt}}
\pgfusepath{stroke}
\pgfpathmoveto{\pgfpoint{126.258652pt}{26.571793pt}}
\pgflineto{\pgfpoint{125.262665pt}{26.399979pt}}
\pgfusepath{stroke}
\pgfpathmoveto{\pgfpoint{127.254631pt}{26.565819pt}}
\pgflineto{\pgfpoint{126.258652pt}{26.571793pt}}
\pgfusepath{stroke}
\pgfpathmoveto{\pgfpoint{128.250610pt}{26.405869pt}}
\pgflineto{\pgfpoint{127.254631pt}{26.565819pt}}
\pgfusepath{stroke}
\pgfpathmoveto{\pgfpoint{129.246597pt}{26.400833pt}}
\pgflineto{\pgfpoint{128.250610pt}{26.405869pt}}
\pgfusepath{stroke}
\pgfpathmoveto{\pgfpoint{130.242584pt}{26.400490pt}}
\pgflineto{\pgfpoint{129.246597pt}{26.400833pt}}
\pgfusepath{stroke}
\pgfpathmoveto{\pgfpoint{131.238571pt}{26.425949pt}}
\pgflineto{\pgfpoint{130.242584pt}{26.400490pt}}
\pgfusepath{stroke}
\pgfpathmoveto{\pgfpoint{132.234558pt}{26.445862pt}}
\pgflineto{\pgfpoint{131.238571pt}{26.425949pt}}
\pgfusepath{stroke}
\pgfpathmoveto{\pgfpoint{133.230530pt}{26.399979pt}}
\pgflineto{\pgfpoint{132.234558pt}{26.445862pt}}
\pgfusepath{stroke}
\pgfpathmoveto{\pgfpoint{134.226517pt}{26.399979pt}}
\pgflineto{\pgfpoint{133.230530pt}{26.399979pt}}
\pgfusepath{stroke}
\pgfpathmoveto{\pgfpoint{135.222504pt}{26.399979pt}}
\pgflineto{\pgfpoint{134.226517pt}{26.399979pt}}
\pgfusepath{stroke}
\pgfpathmoveto{\pgfpoint{136.218475pt}{26.399979pt}}
\pgflineto{\pgfpoint{135.222504pt}{26.399979pt}}
\pgfusepath{stroke}
\pgfpathmoveto{\pgfpoint{137.214478pt}{26.399979pt}}
\pgflineto{\pgfpoint{136.218475pt}{26.399979pt}}
\pgfusepath{stroke}
\pgfpathmoveto{\pgfpoint{138.210449pt}{26.399979pt}}
\pgflineto{\pgfpoint{137.214478pt}{26.399979pt}}
\pgfusepath{stroke}
\pgfpathmoveto{\pgfpoint{139.206436pt}{26.449463pt}}
\pgflineto{\pgfpoint{138.210449pt}{26.399979pt}}
\pgfusepath{stroke}
\pgfpathmoveto{\pgfpoint{140.202423pt}{26.439690pt}}
\pgflineto{\pgfpoint{139.206436pt}{26.449463pt}}
\pgfusepath{stroke}
\pgfpathmoveto{\pgfpoint{141.198410pt}{26.399979pt}}
\pgflineto{\pgfpoint{140.202423pt}{26.439690pt}}
\pgfusepath{stroke}
\pgfpathmoveto{\pgfpoint{142.194382pt}{26.399979pt}}
\pgflineto{\pgfpoint{141.198410pt}{26.399979pt}}
\pgfusepath{stroke}
\pgfpathmoveto{\pgfpoint{143.190369pt}{26.747536pt}}
\pgflineto{\pgfpoint{142.194382pt}{26.399979pt}}
\pgfusepath{stroke}
\pgfpathmoveto{\pgfpoint{144.186356pt}{26.399979pt}}
\pgflineto{\pgfpoint{143.190369pt}{26.747536pt}}
\pgfusepath{stroke}
\pgfpathmoveto{\pgfpoint{145.182343pt}{26.399979pt}}
\pgflineto{\pgfpoint{144.186356pt}{26.399979pt}}
\pgfusepath{stroke}
\pgfpathmoveto{\pgfpoint{146.178314pt}{26.402245pt}}
\pgflineto{\pgfpoint{145.182343pt}{26.399979pt}}
\pgfusepath{stroke}
\pgfpathmoveto{\pgfpoint{147.174316pt}{26.399979pt}}
\pgflineto{\pgfpoint{146.178314pt}{26.402245pt}}
\pgfusepath{stroke}
\pgfpathmoveto{\pgfpoint{148.170288pt}{26.399979pt}}
\pgflineto{\pgfpoint{147.174316pt}{26.399979pt}}
\pgfusepath{stroke}
\pgfpathmoveto{\pgfpoint{149.166275pt}{26.399979pt}}
\pgflineto{\pgfpoint{148.170288pt}{26.399979pt}}
\pgfusepath{stroke}
\pgfpathmoveto{\pgfpoint{150.162262pt}{26.399979pt}}
\pgflineto{\pgfpoint{149.166275pt}{26.399979pt}}
\pgfusepath{stroke}
\pgfpathmoveto{\pgfpoint{151.158249pt}{26.400650pt}}
\pgflineto{\pgfpoint{150.162262pt}{26.399979pt}}
\pgfusepath{stroke}
\pgfpathmoveto{\pgfpoint{152.154221pt}{26.413498pt}}
\pgflineto{\pgfpoint{151.158249pt}{26.400650pt}}
\pgfusepath{stroke}
\pgfpathmoveto{\pgfpoint{153.150208pt}{26.399979pt}}
\pgflineto{\pgfpoint{152.154221pt}{26.413498pt}}
\pgfusepath{stroke}
\pgfpathmoveto{\pgfpoint{154.146194pt}{26.399979pt}}
\pgflineto{\pgfpoint{153.150208pt}{26.399979pt}}
\pgfusepath{stroke}
\pgfpathmoveto{\pgfpoint{155.142181pt}{26.428284pt}}
\pgflineto{\pgfpoint{154.146194pt}{26.399979pt}}
\pgfusepath{stroke}
\pgfpathmoveto{\pgfpoint{156.138168pt}{26.399979pt}}
\pgflineto{\pgfpoint{155.142181pt}{26.428284pt}}
\pgfusepath{stroke}
\pgfpathmoveto{\pgfpoint{157.134155pt}{26.402534pt}}
\pgflineto{\pgfpoint{156.138168pt}{26.399979pt}}
\pgfusepath{stroke}
\pgfpathmoveto{\pgfpoint{158.130127pt}{26.404060pt}}
\pgflineto{\pgfpoint{157.134155pt}{26.402534pt}}
\pgfusepath{stroke}
\pgfpathmoveto{\pgfpoint{159.126114pt}{26.399979pt}}
\pgflineto{\pgfpoint{158.130127pt}{26.404060pt}}
\pgfusepath{stroke}
\pgfpathmoveto{\pgfpoint{160.122101pt}{26.399979pt}}
\pgflineto{\pgfpoint{159.126114pt}{26.399979pt}}
\pgfusepath{stroke}
\pgfpathmoveto{\pgfpoint{161.118088pt}{26.401367pt}}
\pgflineto{\pgfpoint{160.122101pt}{26.399979pt}}
\pgfusepath{stroke}
\pgfpathmoveto{\pgfpoint{162.114075pt}{26.404663pt}}
\pgflineto{\pgfpoint{161.118088pt}{26.401367pt}}
\pgfusepath{stroke}
\pgfpathmoveto{\pgfpoint{163.110062pt}{26.439888pt}}
\pgflineto{\pgfpoint{162.114075pt}{26.404663pt}}
\pgfusepath{stroke}
\pgfpathmoveto{\pgfpoint{164.106033pt}{26.399979pt}}
\pgflineto{\pgfpoint{163.110062pt}{26.439888pt}}
\pgfusepath{stroke}
\pgfpathmoveto{\pgfpoint{165.102020pt}{26.405602pt}}
\pgflineto{\pgfpoint{164.106033pt}{26.399979pt}}
\pgfusepath{stroke}
\pgfpathmoveto{\pgfpoint{166.098007pt}{26.399979pt}}
\pgflineto{\pgfpoint{165.102020pt}{26.405602pt}}
\pgfusepath{stroke}
\pgfpathmoveto{\pgfpoint{167.093994pt}{26.491440pt}}
\pgflineto{\pgfpoint{166.098007pt}{26.399979pt}}
\pgfusepath{stroke}
\pgfpathmoveto{\pgfpoint{168.089966pt}{26.401459pt}}
\pgflineto{\pgfpoint{167.093994pt}{26.491440pt}}
\pgfusepath{stroke}
\pgfpathmoveto{\pgfpoint{169.085953pt}{26.399979pt}}
\pgflineto{\pgfpoint{168.089966pt}{26.401459pt}}
\pgfusepath{stroke}
\pgfpathmoveto{\pgfpoint{170.081940pt}{26.417023pt}}
\pgflineto{\pgfpoint{169.085953pt}{26.399979pt}}
\pgfusepath{stroke}
\pgfpathmoveto{\pgfpoint{171.077911pt}{26.399979pt}}
\pgflineto{\pgfpoint{170.081940pt}{26.417023pt}}
\pgfusepath{stroke}
\pgfpathmoveto{\pgfpoint{172.073914pt}{26.399979pt}}
\pgflineto{\pgfpoint{171.077911pt}{26.399979pt}}
\pgfusepath{stroke}
\pgfpathmoveto{\pgfpoint{173.069885pt}{26.399979pt}}
\pgflineto{\pgfpoint{172.073914pt}{26.399979pt}}
\pgfusepath{stroke}
\pgfpathmoveto{\pgfpoint{174.065872pt}{26.399979pt}}
\pgflineto{\pgfpoint{173.069885pt}{26.399979pt}}
\pgfusepath{stroke}
\pgfpathmoveto{\pgfpoint{175.061859pt}{26.399979pt}}
\pgflineto{\pgfpoint{174.065872pt}{26.399979pt}}
\pgfusepath{stroke}
\pgfpathmoveto{\pgfpoint{176.057846pt}{26.399979pt}}
\pgflineto{\pgfpoint{175.061859pt}{26.399979pt}}
\pgfusepath{stroke}
\pgfpathmoveto{\pgfpoint{177.053818pt}{26.399979pt}}
\pgflineto{\pgfpoint{176.057846pt}{26.399979pt}}
\pgfusepath{stroke}
\pgfpathmoveto{\pgfpoint{178.049805pt}{38.817703pt}}
\pgflineto{\pgfpoint{177.053818pt}{26.399979pt}}
\pgfusepath{stroke}
\pgfpathmoveto{\pgfpoint{179.045792pt}{26.402115pt}}
\pgflineto{\pgfpoint{178.049805pt}{38.817703pt}}
\pgfusepath{stroke}
\pgfpathmoveto{\pgfpoint{180.041779pt}{26.478966pt}}
\pgflineto{\pgfpoint{179.045792pt}{26.402115pt}}
\pgfusepath{stroke}
\pgfpathmoveto{\pgfpoint{181.037766pt}{26.399979pt}}
\pgflineto{\pgfpoint{180.041779pt}{26.478966pt}}
\pgfusepath{stroke}
\pgfpathmoveto{\pgfpoint{182.033752pt}{26.399979pt}}
\pgflineto{\pgfpoint{181.037766pt}{26.399979pt}}
\pgfusepath{stroke}
\pgfpathmoveto{\pgfpoint{183.029724pt}{26.401039pt}}
\pgflineto{\pgfpoint{182.033752pt}{26.399979pt}}
\pgfusepath{stroke}
\pgfpathmoveto{\pgfpoint{184.025711pt}{26.399979pt}}
\pgflineto{\pgfpoint{183.029724pt}{26.401039pt}}
\pgfusepath{stroke}
\pgfpathmoveto{\pgfpoint{185.021698pt}{26.399979pt}}
\pgflineto{\pgfpoint{184.025711pt}{26.399979pt}}
\pgfusepath{stroke}
\pgfpathmoveto{\pgfpoint{186.017685pt}{26.399979pt}}
\pgflineto{\pgfpoint{185.021698pt}{26.399979pt}}
\pgfusepath{stroke}
\pgfpathmoveto{\pgfpoint{187.013672pt}{26.399979pt}}
\pgflineto{\pgfpoint{186.017685pt}{26.399979pt}}
\pgfusepath{stroke}
\pgfpathmoveto{\pgfpoint{188.009659pt}{26.399979pt}}
\pgflineto{\pgfpoint{187.013672pt}{26.399979pt}}
\pgfusepath{stroke}
\pgfpathmoveto{\pgfpoint{189.005630pt}{26.399979pt}}
\pgflineto{\pgfpoint{188.009659pt}{26.399979pt}}
\pgfusepath{stroke}
\pgfpathmoveto{\pgfpoint{190.001617pt}{27.040497pt}}
\pgflineto{\pgfpoint{189.005630pt}{26.399979pt}}
\pgfusepath{stroke}
\pgfpathmoveto{\pgfpoint{190.997604pt}{26.399979pt}}
\pgflineto{\pgfpoint{190.001617pt}{27.040497pt}}
\pgfusepath{stroke}
\pgfpathmoveto{\pgfpoint{191.993591pt}{26.562309pt}}
\pgflineto{\pgfpoint{190.997604pt}{26.399979pt}}
\pgfusepath{stroke}
\pgfpathmoveto{\pgfpoint{192.989563pt}{26.399979pt}}
\pgflineto{\pgfpoint{191.993591pt}{26.562309pt}}
\pgfusepath{stroke}
\pgfpathmoveto{\pgfpoint{193.985565pt}{26.399979pt}}
\pgflineto{\pgfpoint{192.989563pt}{26.399979pt}}
\pgfusepath{stroke}
\pgfpathmoveto{\pgfpoint{194.981537pt}{26.700401pt}}
\pgflineto{\pgfpoint{193.985565pt}{26.399979pt}}
\pgfusepath{stroke}
\pgfpathmoveto{\pgfpoint{195.977524pt}{26.438347pt}}
\pgflineto{\pgfpoint{194.981537pt}{26.700401pt}}
\pgfusepath{stroke}
\pgfpathmoveto{\pgfpoint{196.973511pt}{26.399979pt}}
\pgflineto{\pgfpoint{195.977524pt}{26.438347pt}}
\pgfusepath{stroke}
\pgfpathmoveto{\pgfpoint{197.969498pt}{26.425056pt}}
\pgflineto{\pgfpoint{196.973511pt}{26.399979pt}}
\pgfusepath{stroke}
\pgfpathmoveto{\pgfpoint{198.965469pt}{26.399979pt}}
\pgflineto{\pgfpoint{197.969498pt}{26.425056pt}}
\pgfusepath{stroke}
\pgfpathmoveto{\pgfpoint{199.961456pt}{26.399979pt}}
\pgflineto{\pgfpoint{198.965469pt}{26.399979pt}}
\pgfusepath{stroke}
\pgfpathmoveto{\pgfpoint{200.957443pt}{26.399979pt}}
\pgflineto{\pgfpoint{199.961456pt}{26.399979pt}}
\pgfusepath{stroke}
\pgfpathmoveto{\pgfpoint{201.953430pt}{26.399979pt}}
\pgflineto{\pgfpoint{200.957443pt}{26.399979pt}}
\pgfusepath{stroke}
\pgfpathmoveto{\pgfpoint{202.949402pt}{26.399979pt}}
\pgflineto{\pgfpoint{201.953430pt}{26.399979pt}}
\pgfusepath{stroke}
\pgfpathmoveto{\pgfpoint{203.945404pt}{26.404549pt}}
\pgflineto{\pgfpoint{202.949402pt}{26.399979pt}}
\pgfusepath{stroke}
\pgfpathmoveto{\pgfpoint{204.941376pt}{26.399979pt}}
\pgflineto{\pgfpoint{203.945404pt}{26.404549pt}}
\pgfusepath{stroke}
\pgfpathmoveto{\pgfpoint{205.937347pt}{26.498993pt}}
\pgflineto{\pgfpoint{204.941376pt}{26.399979pt}}
\pgfusepath{stroke}
\pgfpathmoveto{\pgfpoint{206.933334pt}{26.624886pt}}
\pgflineto{\pgfpoint{205.937347pt}{26.498993pt}}
\pgfusepath{stroke}
\pgfpathmoveto{\pgfpoint{207.929337pt}{26.399979pt}}
\pgflineto{\pgfpoint{206.933334pt}{26.624886pt}}
\pgfusepath{stroke}
\pgfpathmoveto{\pgfpoint{208.925323pt}{26.499222pt}}
\pgflineto{\pgfpoint{207.929337pt}{26.399979pt}}
\pgfusepath{stroke}
\pgfpathmoveto{\pgfpoint{209.921295pt}{26.399979pt}}
\pgflineto{\pgfpoint{208.925323pt}{26.499222pt}}
\pgfusepath{stroke}
\pgfpathmoveto{\pgfpoint{210.917267pt}{28.758682pt}}
\pgflineto{\pgfpoint{209.921295pt}{26.399979pt}}
\pgfusepath{stroke}
\pgfpathmoveto{\pgfpoint{211.913269pt}{26.580238pt}}
\pgflineto{\pgfpoint{210.917267pt}{28.758682pt}}
\pgfusepath{stroke}
\pgfpathmoveto{\pgfpoint{212.909241pt}{27.484299pt}}
\pgflineto{\pgfpoint{211.913269pt}{26.580238pt}}
\pgfusepath{stroke}
\pgfpathmoveto{\pgfpoint{213.905228pt}{26.399979pt}}
\pgflineto{\pgfpoint{212.909241pt}{27.484299pt}}
\pgfusepath{stroke}
\pgfpathmoveto{\pgfpoint{214.901215pt}{26.416771pt}}
\pgflineto{\pgfpoint{213.905228pt}{26.399979pt}}
\pgfusepath{stroke}
\pgfpathmoveto{\pgfpoint{215.897217pt}{26.401573pt}}
\pgflineto{\pgfpoint{214.901215pt}{26.416771pt}}
\pgfusepath{stroke}
\pgfpathmoveto{\pgfpoint{216.893188pt}{26.399979pt}}
\pgflineto{\pgfpoint{215.897217pt}{26.401573pt}}
\pgfusepath{stroke}
\pgfpathmoveto{\pgfpoint{217.889160pt}{30.564590pt}}
\pgflineto{\pgfpoint{216.893188pt}{26.399979pt}}
\pgfusepath{stroke}
\pgfpathmoveto{\pgfpoint{218.885147pt}{26.399979pt}}
\pgflineto{\pgfpoint{217.889160pt}{30.564590pt}}
\pgfusepath{stroke}
\pgfpathmoveto{\pgfpoint{219.881134pt}{26.399979pt}}
\pgflineto{\pgfpoint{218.885147pt}{26.399979pt}}
\pgfusepath{stroke}
\pgfpathmoveto{\pgfpoint{220.877121pt}{26.399979pt}}
\pgflineto{\pgfpoint{219.881134pt}{26.399979pt}}
\pgfusepath{stroke}
\pgfpathmoveto{\pgfpoint{221.873108pt}{26.399979pt}}
\pgflineto{\pgfpoint{220.877121pt}{26.399979pt}}
\pgfusepath{stroke}
\pgfpathmoveto{\pgfpoint{222.869080pt}{26.399979pt}}
\pgflineto{\pgfpoint{221.873108pt}{26.399979pt}}
\pgfusepath{stroke}
\pgfpathmoveto{\pgfpoint{223.865082pt}{26.399979pt}}
\pgflineto{\pgfpoint{222.869080pt}{26.399979pt}}
\pgfusepath{stroke}
\pgfpathmoveto{\pgfpoint{224.861053pt}{26.399979pt}}
\pgflineto{\pgfpoint{223.865082pt}{26.399979pt}}
\pgfusepath{stroke}
\pgfpathmoveto{\pgfpoint{225.857040pt}{26.482925pt}}
\pgflineto{\pgfpoint{224.861053pt}{26.399979pt}}
\pgfusepath{stroke}
\pgfpathmoveto{\pgfpoint{226.853027pt}{26.399979pt}}
\pgflineto{\pgfpoint{225.857040pt}{26.482925pt}}
\pgfusepath{stroke}
\pgfpathmoveto{\pgfpoint{227.849014pt}{26.399979pt}}
\pgflineto{\pgfpoint{226.853027pt}{26.399979pt}}
\pgfusepath{stroke}
\pgfpathmoveto{\pgfpoint{228.845001pt}{26.419968pt}}
\pgflineto{\pgfpoint{227.849014pt}{26.399979pt}}
\pgfusepath{stroke}
\pgfpathmoveto{\pgfpoint{229.840973pt}{26.402626pt}}
\pgflineto{\pgfpoint{228.845001pt}{26.419968pt}}
\pgfusepath{stroke}
\pgfpathmoveto{\pgfpoint{230.836945pt}{26.404686pt}}
\pgflineto{\pgfpoint{229.840973pt}{26.402626pt}}
\pgfusepath{stroke}
\pgfpathmoveto{\pgfpoint{231.832932pt}{26.399979pt}}
\pgflineto{\pgfpoint{230.836945pt}{26.404686pt}}
\pgfusepath{stroke}
\pgfpathmoveto{\pgfpoint{232.828934pt}{26.399979pt}}
\pgflineto{\pgfpoint{231.832932pt}{26.399979pt}}
\pgfusepath{stroke}
\pgfpathmoveto{\pgfpoint{233.824921pt}{26.399979pt}}
\pgflineto{\pgfpoint{232.828934pt}{26.399979pt}}
\pgfusepath{stroke}
\pgfpathmoveto{\pgfpoint{234.820892pt}{26.399979pt}}
\pgflineto{\pgfpoint{233.824921pt}{26.399979pt}}
\pgfusepath{stroke}
\pgfpathmoveto{\pgfpoint{235.816864pt}{26.399979pt}}
\pgflineto{\pgfpoint{234.820892pt}{26.399979pt}}
\pgfusepath{stroke}
\pgfpathmoveto{\pgfpoint{236.812866pt}{27.336357pt}}
\pgflineto{\pgfpoint{235.816864pt}{26.399979pt}}
\pgfusepath{stroke}
\pgfpathmoveto{\pgfpoint{237.808838pt}{26.403709pt}}
\pgflineto{\pgfpoint{236.812866pt}{27.336357pt}}
\pgfusepath{stroke}
\pgfpathmoveto{\pgfpoint{238.804825pt}{26.399979pt}}
\pgflineto{\pgfpoint{237.808838pt}{26.403709pt}}
\pgfusepath{stroke}
\pgfpathmoveto{\pgfpoint{239.800812pt}{26.399979pt}}
\pgflineto{\pgfpoint{238.804825pt}{26.399979pt}}
\pgfusepath{stroke}
\pgfpathmoveto{\pgfpoint{240.796814pt}{26.399979pt}}
\pgflineto{\pgfpoint{239.800812pt}{26.399979pt}}
\pgfusepath{stroke}
\pgfpathmoveto{\pgfpoint{241.792786pt}{26.399979pt}}
\pgflineto{\pgfpoint{240.796814pt}{26.399979pt}}
\pgfusepath{stroke}
\pgfpathmoveto{\pgfpoint{242.788757pt}{26.408813pt}}
\pgflineto{\pgfpoint{241.792786pt}{26.399979pt}}
\pgfusepath{stroke}
\pgfpathmoveto{\pgfpoint{243.784744pt}{26.426201pt}}
\pgflineto{\pgfpoint{242.788757pt}{26.408813pt}}
\pgfusepath{stroke}
\pgfpathmoveto{\pgfpoint{244.780731pt}{26.399979pt}}
\pgflineto{\pgfpoint{243.784744pt}{26.426201pt}}
\pgfusepath{stroke}
\pgfpathmoveto{\pgfpoint{245.776718pt}{26.401276pt}}
\pgflineto{\pgfpoint{244.780731pt}{26.399979pt}}
\pgfusepath{stroke}
\pgfpathmoveto{\pgfpoint{246.772705pt}{26.412910pt}}
\pgflineto{\pgfpoint{245.776718pt}{26.401276pt}}
\pgfusepath{stroke}
\pgfpathmoveto{\pgfpoint{247.768677pt}{28.808250pt}}
\pgflineto{\pgfpoint{246.772705pt}{26.412910pt}}
\pgfusepath{stroke}
\pgfpathmoveto{\pgfpoint{248.764679pt}{26.399979pt}}
\pgflineto{\pgfpoint{247.768677pt}{28.808250pt}}
\pgfusepath{stroke}
\pgfpathmoveto{\pgfpoint{249.760651pt}{26.401222pt}}
\pgflineto{\pgfpoint{248.764679pt}{26.399979pt}}
\pgfusepath{stroke}
\pgfpathmoveto{\pgfpoint{250.756638pt}{26.407082pt}}
\pgflineto{\pgfpoint{249.760651pt}{26.401222pt}}
\pgfusepath{stroke}
\pgfpathmoveto{\pgfpoint{251.752625pt}{26.399979pt}}
\pgflineto{\pgfpoint{250.756638pt}{26.407082pt}}
\pgfusepath{stroke}
\pgfpathmoveto{\pgfpoint{252.748611pt}{26.399979pt}}
\pgflineto{\pgfpoint{251.752625pt}{26.399979pt}}
\pgfusepath{stroke}
\pgfpathmoveto{\pgfpoint{253.744598pt}{26.399979pt}}
\pgflineto{\pgfpoint{252.748611pt}{26.399979pt}}
\pgfusepath{stroke}
\pgfpathmoveto{\pgfpoint{254.740570pt}{26.418449pt}}
\pgflineto{\pgfpoint{253.744598pt}{26.399979pt}}
\pgfusepath{stroke}
\pgfpathmoveto{\pgfpoint{255.736542pt}{26.399979pt}}
\pgflineto{\pgfpoint{254.740570pt}{26.418449pt}}
\pgfusepath{stroke}
\pgfpathmoveto{\pgfpoint{256.732544pt}{26.399979pt}}
\pgflineto{\pgfpoint{255.736542pt}{26.399979pt}}
\pgfusepath{stroke}
\pgfpathmoveto{\pgfpoint{257.728516pt}{26.399979pt}}
\pgflineto{\pgfpoint{256.732544pt}{26.399979pt}}
\pgfusepath{stroke}
\pgfpathmoveto{\pgfpoint{258.724518pt}{26.417419pt}}
\pgflineto{\pgfpoint{257.728516pt}{26.399979pt}}
\pgfusepath{stroke}
\pgfpathmoveto{\pgfpoint{259.720490pt}{26.399979pt}}
\pgflineto{\pgfpoint{258.724518pt}{26.417419pt}}
\pgfusepath{stroke}
\pgfpathmoveto{\pgfpoint{260.716492pt}{26.399979pt}}
\pgflineto{\pgfpoint{259.720490pt}{26.399979pt}}
\pgfusepath{stroke}
\pgfpathmoveto{\pgfpoint{261.712463pt}{26.456230pt}}
\pgflineto{\pgfpoint{260.716492pt}{26.399979pt}}
\pgfusepath{stroke}
\pgfpathmoveto{\pgfpoint{262.708435pt}{26.399979pt}}
\pgflineto{\pgfpoint{261.712463pt}{26.456230pt}}
\pgfusepath{stroke}
\pgfpathmoveto{\pgfpoint{263.704407pt}{26.399979pt}}
\pgflineto{\pgfpoint{262.708435pt}{26.399979pt}}
\pgfusepath{stroke}
\pgfpathmoveto{\pgfpoint{264.700409pt}{26.422783pt}}
\pgflineto{\pgfpoint{263.704407pt}{26.399979pt}}
\pgfusepath{stroke}
\pgfpathmoveto{\pgfpoint{265.696411pt}{26.399979pt}}
\pgflineto{\pgfpoint{264.700409pt}{26.422783pt}}
\pgfusepath{stroke}
\pgfpathmoveto{\pgfpoint{266.692383pt}{26.434059pt}}
\pgflineto{\pgfpoint{265.696411pt}{26.399979pt}}
\pgfusepath{stroke}
\pgfpathmoveto{\pgfpoint{267.688354pt}{26.399979pt}}
\pgflineto{\pgfpoint{266.692383pt}{26.434059pt}}
\pgfusepath{stroke}
\pgfpathmoveto{\pgfpoint{268.684326pt}{26.407372pt}}
\pgflineto{\pgfpoint{267.688354pt}{26.399979pt}}
\pgfusepath{stroke}
\pgfpathmoveto{\pgfpoint{269.680328pt}{26.407188pt}}
\pgflineto{\pgfpoint{268.684326pt}{26.407372pt}}
\pgfusepath{stroke}
\pgfpathmoveto{\pgfpoint{270.676331pt}{26.458199pt}}
\pgflineto{\pgfpoint{269.680328pt}{26.407188pt}}
\pgfusepath{stroke}
\pgfpathmoveto{\pgfpoint{271.672302pt}{28.359955pt}}
\pgflineto{\pgfpoint{270.676331pt}{26.458199pt}}
\pgfusepath{stroke}
\pgfpathmoveto{\pgfpoint{272.668274pt}{26.415123pt}}
\pgflineto{\pgfpoint{271.672302pt}{28.359955pt}}
\pgfusepath{stroke}
\pgfpathmoveto{\pgfpoint{273.664276pt}{26.399979pt}}
\pgflineto{\pgfpoint{272.668274pt}{26.415123pt}}
\pgfusepath{stroke}
\pgfpathmoveto{\pgfpoint{274.660248pt}{26.399979pt}}
\pgflineto{\pgfpoint{273.664276pt}{26.399979pt}}
\pgfusepath{stroke}
\pgfpathmoveto{\pgfpoint{275.656250pt}{26.399979pt}}
\pgflineto{\pgfpoint{274.660248pt}{26.399979pt}}
\pgfusepath{stroke}
\pgfpathmoveto{\pgfpoint{276.652222pt}{26.399979pt}}
\pgflineto{\pgfpoint{275.656250pt}{26.399979pt}}
\pgfusepath{stroke}
\pgfpathmoveto{\pgfpoint{277.648193pt}{26.399979pt}}
\pgflineto{\pgfpoint{276.652222pt}{26.399979pt}}
\pgfusepath{stroke}
\pgfpathmoveto{\pgfpoint{278.644196pt}{26.623550pt}}
\pgflineto{\pgfpoint{277.648193pt}{26.399979pt}}
\pgfusepath{stroke}
\pgfpathmoveto{\pgfpoint{279.640167pt}{27.888870pt}}
\pgflineto{\pgfpoint{278.644196pt}{26.623550pt}}
\pgfusepath{stroke}
\pgfpathmoveto{\pgfpoint{280.636139pt}{26.399979pt}}
\pgflineto{\pgfpoint{279.640167pt}{27.888870pt}}
\pgfusepath{stroke}
\pgfpathmoveto{\pgfpoint{281.632141pt}{26.399979pt}}
\pgflineto{\pgfpoint{280.636139pt}{26.399979pt}}
\pgfusepath{stroke}
\pgfpathmoveto{\pgfpoint{282.628113pt}{26.524590pt}}
\pgflineto{\pgfpoint{281.632141pt}{26.399979pt}}
\pgfusepath{stroke}
\pgfpathmoveto{\pgfpoint{283.624115pt}{26.456955pt}}
\pgflineto{\pgfpoint{282.628113pt}{26.524590pt}}
\pgfusepath{stroke}
\pgfpathmoveto{\pgfpoint{284.620087pt}{26.399979pt}}
\pgflineto{\pgfpoint{283.624115pt}{26.456955pt}}
\pgfusepath{stroke}
\pgfpathmoveto{\pgfpoint{285.616089pt}{26.402275pt}}
\pgflineto{\pgfpoint{284.620087pt}{26.399979pt}}
\pgfusepath{stroke}
\pgfpathmoveto{\pgfpoint{286.612061pt}{26.399979pt}}
\pgflineto{\pgfpoint{285.616089pt}{26.402275pt}}
\pgfusepath{stroke}
\pgfpathmoveto{\pgfpoint{287.608032pt}{26.399979pt}}
\pgflineto{\pgfpoint{286.612061pt}{26.399979pt}}
\pgfusepath{stroke}
\pgfpathmoveto{\pgfpoint{288.604004pt}{26.399979pt}}
\pgflineto{\pgfpoint{287.608032pt}{26.399979pt}}
\pgfusepath{stroke}
\pgfpathmoveto{\pgfpoint{289.600037pt}{26.399979pt}}
\pgflineto{\pgfpoint{288.604004pt}{26.399979pt}}
\pgfusepath{stroke}
{
\pgftransformshift{\pgfpoint{165.600006pt}{215.577454pt}}
\pgfnode{rectangle}{south}{\fontsize{10}{0}\selectfont\textcolor[rgb]{0,0,0}{{Spectral statistics VVX strain}}}{}{\pgfusepath{discard}}}
{
\pgftransformshift{\pgfpoint{165.600006pt}{101.199989pt}}
\pgfnode{rectangle}{south}{\fontsize{10}{0}\selectfont\textcolor[rgb]{0,0,0}{{Spectral statistics for BUT }}}{}{\pgfusepath{discard}}}
\end{pgfpicture}

\end{frame}

\only<article>{
  Let's tackle the problem of discriminating between different
disease vectors. Ideally, we'd like to have a simple test that
tells us what ails us. One kind of test is mass spectrometry. This
graph shows spectrometry results for two types of bacteria. There
is plenty of variation within each type, both due to measurement
error and due to changes in the bacterial strains. Here, we plot
the average and maximum energies measured for about 100 different
examples from each strain.
}

\begin{frame}
  \frametitle{Nearest neighbour: the hidden secret of machine learning}
  \input{../figures/separation1.tikz}
\end{frame}

\only<article>{
  Now, is it possible to identify an unknown strain based on this
data? Actually, this is possible. Sometimes, very simple algorithms
work very well. One of the simplest one involves just measuring the
distance between the decsription of a new unknown strain and known
ones. In this visualisation, I projected the 1300-dimensional data
into a 2-dimensional space. Here you can clearly see that it is
possible to separate the two strains. In order to classify a new
point, you just need to see whether it's closer to the train VVT or
BUT.
}

\begin{frame}
\frametitle{Comparing spectral data}
  \only<1>{\input{../figures/difference1.tikz}}
  \only<2>{% Title: glps_renderer figure
% Creator: GL2PS 1.3.8, (C) 1999-2012 C. Geuzaine
% For: Octave
% CreationDate: Fri Jun 16 12:38:10 2017
\begin{pgfpicture}
\pgfsetlinewidth{0.01pt}
\color[rgb]{1.000000,1.000000,1.000000}
\pgfpathmoveto{\pgfpoint{41.600006pt}{222.000000pt}}
\pgflineto{\pgfpoint{289.600037pt}{26.399979pt}}
\pgflineto{\pgfpoint{41.600006pt}{26.399979pt}}
\pgfpathclose
\pgfusepath{fill,stroke}
\pgfpathmoveto{\pgfpoint{41.600006pt}{222.000000pt}}
\pgflineto{\pgfpoint{289.600037pt}{222.000000pt}}
\pgflineto{\pgfpoint{289.600037pt}{26.399979pt}}
\pgfpathclose
\pgfusepath{fill,stroke}
\color[rgb]{1.000000,0.000000,0.000000}
\pgfpathmoveto{\pgfpoint{46.560013pt}{171.656006pt}}
\pgflineto{\pgfpoint{46.560013pt}{65.485382pt}}
\pgflineto{\pgfpoint{51.272934pt}{60.450993pt}}
\pgfpathclose
\pgfusepath{fill,stroke}
\pgfpathmoveto{\pgfpoint{51.272934pt}{60.450993pt}}
\pgflineto{\pgfpoint{51.520004pt}{54.621078pt}}
\pgflineto{\pgfpoint{51.520004pt}{60.187065pt}}
\pgfpathclose
\pgfusepath{fill,stroke}
\pgfpathmoveto{\pgfpoint{52.729080pt}{55.915833pt}}
\pgflineto{\pgfpoint{51.520004pt}{60.187065pt}}
\pgflineto{\pgfpoint{51.520004pt}{54.621078pt}}
\pgfpathclose
\pgfusepath{fill,stroke}
\pgfpathmoveto{\pgfpoint{52.729080pt}{55.915833pt}}
\pgflineto{\pgfpoint{56.480011pt}{42.665169pt}}
\pgflineto{\pgfpoint{56.480011pt}{59.932564pt}}
\pgfpathclose
\pgfusepath{fill,stroke}
\pgfpathmoveto{\pgfpoint{61.440010pt}{105.284721pt}}
\pgflineto{\pgfpoint{56.480011pt}{59.932564pt}}
\pgflineto{\pgfpoint{56.480011pt}{42.665169pt}}
\pgfpathclose
\pgfusepath{fill,stroke}
\pgfpathmoveto{\pgfpoint{61.440010pt}{46.590111pt}}
\pgflineto{\pgfpoint{61.440010pt}{105.284721pt}}
\pgflineto{\pgfpoint{56.480011pt}{42.665169pt}}
\pgfpathclose
\pgfusepath{fill,stroke}
\pgfpathmoveto{\pgfpoint{66.400009pt}{163.107224pt}}
\pgflineto{\pgfpoint{61.440010pt}{105.284721pt}}
\pgflineto{\pgfpoint{61.440010pt}{46.590111pt}}
\pgfpathclose
\pgfusepath{fill,stroke}
\pgfpathmoveto{\pgfpoint{66.400009pt}{63.127323pt}}
\pgflineto{\pgfpoint{66.400009pt}{163.107224pt}}
\pgflineto{\pgfpoint{61.440010pt}{46.590111pt}}
\pgfpathclose
\pgfusepath{fill,stroke}
\pgfpathmoveto{\pgfpoint{69.454124pt}{89.661858pt}}
\pgflineto{\pgfpoint{66.400009pt}{163.107224pt}}
\pgflineto{\pgfpoint{66.400009pt}{63.127323pt}}
\pgfpathclose
\pgfusepath{fill,stroke}
\pgfpathmoveto{\pgfpoint{69.454124pt}{89.661858pt}}
\pgflineto{\pgfpoint{71.360008pt}{43.829178pt}}
\pgflineto{\pgfpoint{71.360008pt}{106.220398pt}}
\pgfpathclose
\pgfusepath{fill,stroke}
\pgfpathmoveto{\pgfpoint{73.884773pt}{93.204117pt}}
\pgflineto{\pgfpoint{71.360008pt}{106.220398pt}}
\pgflineto{\pgfpoint{71.360008pt}{43.829178pt}}
\pgfpathclose
\pgfusepath{fill,stroke}
\pgfpathmoveto{\pgfpoint{73.884773pt}{93.204117pt}}
\pgflineto{\pgfpoint{76.320007pt}{80.649338pt}}
\pgflineto{\pgfpoint{76.320007pt}{140.828415pt}}
\pgfpathclose
\pgfusepath{fill,stroke}
\pgfpathmoveto{\pgfpoint{80.181015pt}{65.915634pt}}
\pgflineto{\pgfpoint{76.320007pt}{140.828415pt}}
\pgflineto{\pgfpoint{76.320007pt}{80.649338pt}}
\pgfpathclose
\pgfusepath{fill,stroke}
\pgfpathmoveto{\pgfpoint{80.181015pt}{65.915634pt}}
\pgflineto{\pgfpoint{81.280014pt}{44.592613pt}}
\pgflineto{\pgfpoint{81.280014pt}{61.721870pt}}
\pgfpathclose
\pgfusepath{fill,stroke}
\pgfpathmoveto{\pgfpoint{85.898277pt}{60.412163pt}}
\pgflineto{\pgfpoint{81.280014pt}{61.721870pt}}
\pgflineto{\pgfpoint{81.280014pt}{44.592613pt}}
\pgfpathclose
\pgfusepath{fill,stroke}
\pgfpathmoveto{\pgfpoint{85.898277pt}{60.412163pt}}
\pgflineto{\pgfpoint{86.240013pt}{60.315250pt}}
\pgflineto{\pgfpoint{86.240013pt}{61.582748pt}}
\pgfpathclose
\pgfusepath{fill,stroke}
\pgfpathmoveto{\pgfpoint{91.200012pt}{82.405769pt}}
\pgflineto{\pgfpoint{86.240013pt}{61.582748pt}}
\pgflineto{\pgfpoint{86.240013pt}{60.315250pt}}
\pgfpathclose
\pgfusepath{fill,stroke}
\pgfpathmoveto{\pgfpoint{91.200012pt}{67.963776pt}}
\pgflineto{\pgfpoint{91.200012pt}{82.405769pt}}
\pgflineto{\pgfpoint{86.240013pt}{60.315250pt}}
\pgfpathclose
\pgfusepath{fill,stroke}
\pgfpathmoveto{\pgfpoint{96.160011pt}{121.447739pt}}
\pgflineto{\pgfpoint{91.200012pt}{82.405769pt}}
\pgflineto{\pgfpoint{91.200012pt}{67.963776pt}}
\pgfpathclose
\pgfusepath{fill,stroke}
\pgfpathmoveto{\pgfpoint{96.160011pt}{39.422333pt}}
\pgflineto{\pgfpoint{96.160011pt}{121.447739pt}}
\pgflineto{\pgfpoint{91.200012pt}{67.963776pt}}
\pgfpathclose
\pgfusepath{fill,stroke}
\pgfpathmoveto{\pgfpoint{101.120010pt}{45.170807pt}}
\pgflineto{\pgfpoint{96.160011pt}{121.447739pt}}
\pgflineto{\pgfpoint{96.160011pt}{39.422333pt}}
\pgfpathclose
\pgfusepath{fill,stroke}
\pgfpathmoveto{\pgfpoint{101.120010pt}{36.805199pt}}
\pgflineto{\pgfpoint{101.120010pt}{45.170807pt}}
\pgflineto{\pgfpoint{96.160011pt}{39.422333pt}}
\pgfpathclose
\pgfusepath{fill,stroke}
\pgfpathmoveto{\pgfpoint{106.080017pt}{93.045525pt}}
\pgflineto{\pgfpoint{101.120010pt}{45.170807pt}}
\pgflineto{\pgfpoint{101.120010pt}{36.805199pt}}
\pgfpathclose
\pgfusepath{fill,stroke}
\pgfpathmoveto{\pgfpoint{106.080017pt}{57.417770pt}}
\pgflineto{\pgfpoint{106.080017pt}{93.045525pt}}
\pgflineto{\pgfpoint{101.120010pt}{36.805199pt}}
\pgfpathclose
\pgfusepath{fill,stroke}
\pgfpathmoveto{\pgfpoint{111.040009pt}{89.065750pt}}
\pgflineto{\pgfpoint{106.080017pt}{93.045525pt}}
\pgflineto{\pgfpoint{106.080017pt}{57.417770pt}}
\pgfpathclose
\pgfusepath{fill,stroke}
\pgfpathmoveto{\pgfpoint{111.040009pt}{48.803902pt}}
\pgflineto{\pgfpoint{111.040009pt}{89.065750pt}}
\pgflineto{\pgfpoint{106.080017pt}{57.417770pt}}
\pgfpathclose
\pgfusepath{fill,stroke}
\pgfpathmoveto{\pgfpoint{116.000015pt}{43.683624pt}}
\pgflineto{\pgfpoint{111.040009pt}{89.065750pt}}
\pgflineto{\pgfpoint{111.040009pt}{48.803902pt}}
\pgfpathclose
\pgfusepath{fill,stroke}
\pgfpathmoveto{\pgfpoint{116.000015pt}{40.934975pt}}
\pgflineto{\pgfpoint{116.000015pt}{43.683624pt}}
\pgflineto{\pgfpoint{111.040009pt}{48.803902pt}}
\pgfpathclose
\pgfusepath{fill,stroke}
\pgfpathmoveto{\pgfpoint{120.960007pt}{149.515289pt}}
\pgflineto{\pgfpoint{116.000015pt}{43.683624pt}}
\pgflineto{\pgfpoint{116.000015pt}{40.934975pt}}
\pgfpathclose
\pgfusepath{fill,stroke}
\pgfpathmoveto{\pgfpoint{120.960007pt}{115.830338pt}}
\pgflineto{\pgfpoint{120.960007pt}{149.515289pt}}
\pgflineto{\pgfpoint{116.000015pt}{40.934975pt}}
\pgfpathclose
\pgfusepath{fill,stroke}
\pgfpathmoveto{\pgfpoint{125.920013pt}{72.071335pt}}
\pgflineto{\pgfpoint{120.960007pt}{149.515289pt}}
\pgflineto{\pgfpoint{120.960007pt}{115.830338pt}}
\pgfpathclose
\pgfusepath{fill,stroke}
\pgfpathmoveto{\pgfpoint{125.920013pt}{66.937149pt}}
\pgflineto{\pgfpoint{125.920013pt}{72.071335pt}}
\pgflineto{\pgfpoint{120.960007pt}{115.830338pt}}
\pgfpathclose
\pgfusepath{fill,stroke}
\pgfpathmoveto{\pgfpoint{127.352821pt}{64.771484pt}}
\pgflineto{\pgfpoint{125.920013pt}{72.071335pt}}
\pgflineto{\pgfpoint{125.920013pt}{66.937149pt}}
\pgfpathclose
\pgfusepath{fill,stroke}
\pgfpathmoveto{\pgfpoint{127.352821pt}{64.771484pt}}
\pgflineto{\pgfpoint{130.880005pt}{46.801132pt}}
\pgflineto{\pgfpoint{130.880005pt}{59.440186pt}}
\pgfpathclose
\pgfusepath{fill,stroke}
\pgfpathmoveto{\pgfpoint{132.276794pt}{53.129593pt}}
\pgflineto{\pgfpoint{130.880005pt}{59.440186pt}}
\pgflineto{\pgfpoint{130.880005pt}{46.801132pt}}
\pgfpathclose
\pgfusepath{fill,stroke}
\pgfpathmoveto{\pgfpoint{132.276794pt}{53.129593pt}}
\pgflineto{\pgfpoint{135.840012pt}{37.031158pt}}
\pgflineto{\pgfpoint{135.840012pt}{69.273598pt}}
\pgfpathclose
\pgfusepath{fill,stroke}
\pgfpathmoveto{\pgfpoint{138.230530pt}{66.341614pt}}
\pgflineto{\pgfpoint{135.840012pt}{69.273598pt}}
\pgflineto{\pgfpoint{135.840012pt}{37.031158pt}}
\pgfpathclose
\pgfusepath{fill,stroke}
\pgfpathmoveto{\pgfpoint{138.230530pt}{66.341614pt}}
\pgflineto{\pgfpoint{140.800003pt}{63.190159pt}}
\pgflineto{\pgfpoint{140.800003pt}{97.846222pt}}
\pgfpathclose
\pgfusepath{fill,stroke}
\pgfpathmoveto{\pgfpoint{145.760010pt}{49.291016pt}}
\pgflineto{\pgfpoint{140.800003pt}{97.846222pt}}
\pgflineto{\pgfpoint{140.800003pt}{63.190159pt}}
\pgfpathclose
\pgfusepath{fill,stroke}
\pgfpathmoveto{\pgfpoint{145.760010pt}{45.087975pt}}
\pgflineto{\pgfpoint{145.760010pt}{49.291016pt}}
\pgflineto{\pgfpoint{140.800003pt}{63.190159pt}}
\pgfpathclose
\pgfusepath{fill,stroke}
\pgfpathmoveto{\pgfpoint{148.486938pt}{53.263680pt}}
\pgflineto{\pgfpoint{145.760010pt}{49.291016pt}}
\pgflineto{\pgfpoint{145.760010pt}{45.087975pt}}
\pgfpathclose
\pgfusepath{fill,stroke}
\pgfpathmoveto{\pgfpoint{148.486938pt}{53.263680pt}}
\pgflineto{\pgfpoint{150.720016pt}{56.516876pt}}
\pgflineto{\pgfpoint{150.720016pt}{59.958729pt}}
\pgfpathclose
\pgfusepath{fill,stroke}
\pgfpathmoveto{\pgfpoint{155.680023pt}{145.320908pt}}
\pgflineto{\pgfpoint{150.720016pt}{59.958729pt}}
\pgflineto{\pgfpoint{150.720016pt}{56.516876pt}}
\pgfpathclose
\pgfusepath{fill,stroke}
\pgfpathmoveto{\pgfpoint{155.680023pt}{39.056114pt}}
\pgflineto{\pgfpoint{155.680023pt}{145.320908pt}}
\pgflineto{\pgfpoint{150.720016pt}{56.516876pt}}
\pgfpathclose
\pgfusepath{fill,stroke}
\pgfpathmoveto{\pgfpoint{158.851837pt}{88.183441pt}}
\pgflineto{\pgfpoint{155.680023pt}{145.320908pt}}
\pgflineto{\pgfpoint{155.680023pt}{39.056114pt}}
\pgfpathclose
\pgfusepath{fill,stroke}
\pgfpathmoveto{\pgfpoint{158.851837pt}{88.183441pt}}
\pgflineto{\pgfpoint{160.640015pt}{55.971481pt}}
\pgflineto{\pgfpoint{160.640015pt}{115.879601pt}}
\pgfpathclose
\pgfusepath{fill,stroke}
\pgfpathmoveto{\pgfpoint{165.274719pt}{64.387100pt}}
\pgflineto{\pgfpoint{160.640015pt}{115.879601pt}}
\pgflineto{\pgfpoint{160.640015pt}{55.971481pt}}
\pgfpathclose
\pgfusepath{fill,stroke}
\pgfpathmoveto{\pgfpoint{165.274719pt}{64.387100pt}}
\pgflineto{\pgfpoint{165.600006pt}{60.773132pt}}
\pgflineto{\pgfpoint{165.600006pt}{64.977737pt}}
\pgfpathclose
\pgfusepath{fill,stroke}
\pgfpathmoveto{\pgfpoint{166.011597pt}{63.599285pt}}
\pgflineto{\pgfpoint{165.600006pt}{64.977737pt}}
\pgflineto{\pgfpoint{165.600006pt}{60.773132pt}}
\pgfpathclose
\pgfusepath{fill,stroke}
\pgfpathmoveto{\pgfpoint{166.011597pt}{63.599285pt}}
\pgflineto{\pgfpoint{170.560013pt}{48.365959pt}}
\pgflineto{\pgfpoint{170.560013pt}{94.831261pt}}
\pgfpathclose
\pgfusepath{fill,stroke}
\pgfpathmoveto{\pgfpoint{174.223160pt}{57.045738pt}}
\pgflineto{\pgfpoint{170.560013pt}{94.831261pt}}
\pgflineto{\pgfpoint{170.560013pt}{48.365959pt}}
\pgfpathclose
\pgfusepath{fill,stroke}
\pgfpathmoveto{\pgfpoint{174.223160pt}{57.045738pt}}
\pgflineto{\pgfpoint{175.520004pt}{43.668625pt}}
\pgflineto{\pgfpoint{175.520004pt}{60.118607pt}}
\pgfpathclose
\pgfusepath{fill,stroke}
\pgfpathmoveto{\pgfpoint{180.480011pt}{75.116745pt}}
\pgflineto{\pgfpoint{175.520004pt}{60.118607pt}}
\pgflineto{\pgfpoint{175.520004pt}{43.668625pt}}
\pgfpathclose
\pgfusepath{fill,stroke}
\pgfpathmoveto{\pgfpoint{180.480011pt}{40.850761pt}}
\pgflineto{\pgfpoint{180.480011pt}{75.116745pt}}
\pgflineto{\pgfpoint{175.520004pt}{43.668625pt}}
\pgfpathclose
\pgfusepath{fill,stroke}
\pgfpathmoveto{\pgfpoint{182.813766pt}{65.866241pt}}
\pgflineto{\pgfpoint{180.480011pt}{75.116745pt}}
\pgflineto{\pgfpoint{180.480011pt}{40.850761pt}}
\pgfpathclose
\pgfusepath{fill,stroke}
\pgfpathmoveto{\pgfpoint{182.813766pt}{65.866241pt}}
\pgflineto{\pgfpoint{185.440018pt}{55.456352pt}}
\pgflineto{\pgfpoint{185.440018pt}{94.016953pt}}
\pgfpathclose
\pgfusepath{fill,stroke}
\pgfpathmoveto{\pgfpoint{190.400024pt}{74.316742pt}}
\pgflineto{\pgfpoint{185.440018pt}{94.016953pt}}
\pgflineto{\pgfpoint{185.440018pt}{55.456352pt}}
\pgfpathclose
\pgfusepath{fill,stroke}
\pgfpathmoveto{\pgfpoint{190.400024pt}{40.646240pt}}
\pgflineto{\pgfpoint{190.400024pt}{74.316742pt}}
\pgflineto{\pgfpoint{185.440018pt}{55.456352pt}}
\pgfpathclose
\pgfusepath{fill,stroke}
\pgfpathmoveto{\pgfpoint{194.088348pt}{48.193687pt}}
\pgflineto{\pgfpoint{190.400024pt}{74.316742pt}}
\pgflineto{\pgfpoint{190.400024pt}{40.646240pt}}
\pgfpathclose
\pgfusepath{fill,stroke}
\pgfpathmoveto{\pgfpoint{194.088348pt}{48.193687pt}}
\pgflineto{\pgfpoint{195.360016pt}{39.186867pt}}
\pgflineto{\pgfpoint{195.360016pt}{50.795929pt}}
\pgfpathclose
\pgfusepath{fill,stroke}
\pgfpathmoveto{\pgfpoint{200.320007pt}{162.896408pt}}
\pgflineto{\pgfpoint{195.360016pt}{50.795929pt}}
\pgflineto{\pgfpoint{195.360016pt}{39.186867pt}}
\pgfpathclose
\pgfusepath{fill,stroke}
\pgfpathmoveto{\pgfpoint{200.320007pt}{111.419617pt}}
\pgflineto{\pgfpoint{200.320007pt}{162.896408pt}}
\pgflineto{\pgfpoint{195.360016pt}{39.186867pt}}
\pgfpathclose
\pgfusepath{fill,stroke}
\pgfpathmoveto{\pgfpoint{205.279999pt}{99.158791pt}}
\pgflineto{\pgfpoint{200.320007pt}{162.896408pt}}
\pgflineto{\pgfpoint{200.320007pt}{111.419617pt}}
\pgfpathclose
\pgfusepath{fill,stroke}
\pgfpathmoveto{\pgfpoint{205.279999pt}{77.342728pt}}
\pgflineto{\pgfpoint{205.279999pt}{99.158791pt}}
\pgflineto{\pgfpoint{200.320007pt}{111.419617pt}}
\pgfpathclose
\pgfusepath{fill,stroke}
\pgfpathmoveto{\pgfpoint{210.147217pt}{64.919937pt}}
\pgflineto{\pgfpoint{205.279999pt}{99.158791pt}}
\pgflineto{\pgfpoint{205.279999pt}{77.342728pt}}
\pgfpathclose
\pgfusepath{fill,stroke}
\pgfpathmoveto{\pgfpoint{210.147217pt}{64.919937pt}}
\pgflineto{\pgfpoint{210.240021pt}{64.267227pt}}
\pgflineto{\pgfpoint{210.240021pt}{64.683121pt}}
\pgfpathclose
\pgfusepath{fill,stroke}
\pgfpathmoveto{\pgfpoint{215.200012pt}{68.944382pt}}
\pgflineto{\pgfpoint{210.240021pt}{64.683121pt}}
\pgflineto{\pgfpoint{210.240021pt}{64.267227pt}}
\pgfpathclose
\pgfusepath{fill,stroke}
\pgfpathmoveto{\pgfpoint{215.200012pt}{36.115891pt}}
\pgflineto{\pgfpoint{215.200012pt}{68.944382pt}}
\pgflineto{\pgfpoint{210.240021pt}{64.267227pt}}
\pgfpathclose
\pgfusepath{fill,stroke}
\pgfpathmoveto{\pgfpoint{220.160019pt}{56.577995pt}}
\pgflineto{\pgfpoint{215.200012pt}{68.944382pt}}
\pgflineto{\pgfpoint{215.200012pt}{36.115891pt}}
\pgfpathclose
\pgfusepath{fill,stroke}
\pgfpathmoveto{\pgfpoint{220.160019pt}{48.298035pt}}
\pgflineto{\pgfpoint{220.160019pt}{56.577995pt}}
\pgflineto{\pgfpoint{215.200012pt}{36.115891pt}}
\pgfpathclose
\pgfusepath{fill,stroke}
\pgfpathmoveto{\pgfpoint{225.120026pt}{63.517673pt}}
\pgflineto{\pgfpoint{220.160019pt}{56.577995pt}}
\pgflineto{\pgfpoint{220.160019pt}{48.298035pt}}
\pgfpathclose
\pgfusepath{fill,stroke}
\pgfpathmoveto{\pgfpoint{225.120026pt}{51.340904pt}}
\pgflineto{\pgfpoint{225.120026pt}{63.517673pt}}
\pgflineto{\pgfpoint{220.160019pt}{48.298035pt}}
\pgfpathclose
\pgfusepath{fill,stroke}
\pgfpathmoveto{\pgfpoint{228.633514pt}{65.886925pt}}
\pgflineto{\pgfpoint{225.120026pt}{63.517673pt}}
\pgflineto{\pgfpoint{225.120026pt}{51.340904pt}}
\pgfpathclose
\pgfusepath{fill,stroke}
\pgfpathmoveto{\pgfpoint{228.633514pt}{65.886925pt}}
\pgflineto{\pgfpoint{230.080017pt}{66.862328pt}}
\pgflineto{\pgfpoint{230.080017pt}{71.875427pt}}
\pgfpathclose
\pgfusepath{fill,stroke}
\pgfpathmoveto{\pgfpoint{231.822998pt}{62.078987pt}}
\pgflineto{\pgfpoint{230.080017pt}{71.875427pt}}
\pgflineto{\pgfpoint{230.080017pt}{66.862328pt}}
\pgfpathclose
\pgfusepath{fill,stroke}
\pgfpathmoveto{\pgfpoint{231.822998pt}{62.078987pt}}
\pgflineto{\pgfpoint{235.040024pt}{43.997490pt}}
\pgflineto{\pgfpoint{235.040024pt}{53.250259pt}}
\pgfpathclose
\pgfusepath{fill,stroke}
\pgfpathmoveto{\pgfpoint{240.000000pt}{75.645462pt}}
\pgflineto{\pgfpoint{235.040024pt}{53.250259pt}}
\pgflineto{\pgfpoint{235.040024pt}{43.997490pt}}
\pgfpathclose
\pgfusepath{fill,stroke}
\pgfpathmoveto{\pgfpoint{240.000000pt}{61.037575pt}}
\pgflineto{\pgfpoint{240.000000pt}{75.645462pt}}
\pgflineto{\pgfpoint{235.040024pt}{43.997490pt}}
\pgfpathclose
\pgfusepath{fill,stroke}
\pgfpathmoveto{\pgfpoint{244.506226pt}{42.584846pt}}
\pgflineto{\pgfpoint{240.000000pt}{75.645462pt}}
\pgflineto{\pgfpoint{240.000000pt}{61.037575pt}}
\pgfpathclose
\pgfusepath{fill,stroke}
\pgfpathmoveto{\pgfpoint{244.506226pt}{42.584846pt}}
\pgflineto{\pgfpoint{244.960022pt}{39.255608pt}}
\pgflineto{\pgfpoint{244.960022pt}{40.726631pt}}
\pgfpathclose
\pgfusepath{fill,stroke}
\pgfpathmoveto{\pgfpoint{249.920013pt}{116.421631pt}}
\pgflineto{\pgfpoint{244.960022pt}{40.726631pt}}
\pgflineto{\pgfpoint{244.960022pt}{39.255608pt}}
\pgfpathclose
\pgfusepath{fill,stroke}
\pgfpathmoveto{\pgfpoint{249.920013pt}{106.094025pt}}
\pgflineto{\pgfpoint{249.920013pt}{116.421631pt}}
\pgflineto{\pgfpoint{244.960022pt}{39.255608pt}}
\pgfpathclose
\pgfusepath{fill,stroke}
\pgfpathmoveto{\pgfpoint{254.880020pt}{215.171143pt}}
\pgflineto{\pgfpoint{249.920013pt}{116.421631pt}}
\pgflineto{\pgfpoint{249.920013pt}{106.094025pt}}
\pgfpathclose
\pgfusepath{fill,stroke}
\pgfpathmoveto{\pgfpoint{254.880020pt}{70.739304pt}}
\pgflineto{\pgfpoint{254.880020pt}{215.171143pt}}
\pgflineto{\pgfpoint{249.920013pt}{106.094025pt}}
\pgfpathclose
\pgfusepath{fill,stroke}
\pgfpathmoveto{\pgfpoint{259.840027pt}{141.531891pt}}
\pgflineto{\pgfpoint{254.880020pt}{215.171143pt}}
\pgflineto{\pgfpoint{254.880020pt}{70.739304pt}}
\pgfpathclose
\pgfusepath{fill,stroke}
\pgfpathmoveto{\pgfpoint{259.840027pt}{73.925293pt}}
\pgflineto{\pgfpoint{259.840027pt}{141.531891pt}}
\pgflineto{\pgfpoint{254.880020pt}{70.739304pt}}
\pgfpathclose
\pgfusepath{fill,stroke}
\pgfpathmoveto{\pgfpoint{264.800018pt}{190.998230pt}}
\pgflineto{\pgfpoint{259.840027pt}{141.531891pt}}
\pgflineto{\pgfpoint{259.840027pt}{73.925293pt}}
\pgfpathclose
\pgfusepath{fill,stroke}
\pgfpathmoveto{\pgfpoint{264.800018pt}{34.848969pt}}
\pgflineto{\pgfpoint{264.800018pt}{190.998230pt}}
\pgflineto{\pgfpoint{259.840027pt}{73.925293pt}}
\pgfpathclose
\pgfusepath{fill,stroke}
\pgfpathmoveto{\pgfpoint{269.487213pt}{69.737061pt}}
\pgflineto{\pgfpoint{264.800018pt}{190.998230pt}}
\pgflineto{\pgfpoint{264.800018pt}{34.848969pt}}
\pgfpathclose
\pgfusepath{fill,stroke}
\pgfpathmoveto{\pgfpoint{269.487213pt}{69.737061pt}}
\pgflineto{\pgfpoint{269.760010pt}{62.679287pt}}
\pgflineto{\pgfpoint{269.760010pt}{71.767647pt}}
\pgfpathclose
\pgfusepath{fill,stroke}
\pgfpathmoveto{\pgfpoint{274.720001pt}{76.811203pt}}
\pgflineto{\pgfpoint{269.760010pt}{71.767647pt}}
\pgflineto{\pgfpoint{269.760010pt}{62.679287pt}}
\pgfpathclose
\pgfusepath{fill,stroke}
\pgfpathmoveto{\pgfpoint{274.720001pt}{37.509773pt}}
\pgflineto{\pgfpoint{274.720001pt}{76.811203pt}}
\pgflineto{\pgfpoint{269.760010pt}{62.679287pt}}
\pgfpathclose
\pgfusepath{fill,stroke}
\pgfpathmoveto{\pgfpoint{279.680023pt}{69.222656pt}}
\pgflineto{\pgfpoint{274.720001pt}{76.811203pt}}
\pgflineto{\pgfpoint{274.720001pt}{37.509773pt}}
\pgfpathclose
\pgfusepath{fill,stroke}
\pgfpathmoveto{\pgfpoint{279.680023pt}{54.383209pt}}
\pgflineto{\pgfpoint{279.680023pt}{69.222656pt}}
\pgflineto{\pgfpoint{274.720001pt}{37.509773pt}}
\pgfpathclose
\pgfusepath{fill,stroke}
\pgfpathmoveto{\pgfpoint{280.898163pt}{61.260727pt}}
\pgflineto{\pgfpoint{279.680023pt}{69.222656pt}}
\pgflineto{\pgfpoint{279.680023pt}{54.383209pt}}
\pgfpathclose
\pgfusepath{fill,stroke}
\pgfpathmoveto{\pgfpoint{280.898163pt}{61.260727pt}}
\pgflineto{\pgfpoint{284.640015pt}{36.803474pt}}
\pgflineto{\pgfpoint{284.640015pt}{82.386887pt}}
\pgfpathclose
\pgfusepath{fill,stroke}
\pgfpathmoveto{\pgfpoint{288.703094pt}{64.960159pt}}
\pgflineto{\pgfpoint{284.640015pt}{82.386887pt}}
\pgflineto{\pgfpoint{284.640015pt}{36.803474pt}}
\pgfpathclose
\pgfusepath{fill,stroke}
\pgfpathmoveto{\pgfpoint{289.600037pt}{71.175858pt}}
\pgflineto{\pgfpoint{288.703094pt}{64.960159pt}}
\pgflineto{\pgfpoint{289.600037pt}{61.113129pt}}
\pgfpathclose
\pgfusepath{fill,stroke}
\color[rgb]{0.000000,0.000000,0.000000}
\pgfsetlinewidth{0.500000pt}
\pgfsetdash{{16pt}{0pt}}{0pt}
\pgfpathmoveto{\pgfpoint{289.600037pt}{26.399979pt}}
\pgflineto{\pgfpoint{41.600006pt}{26.399979pt}}
\pgfusepath{stroke}
\pgfpathmoveto{\pgfpoint{289.600037pt}{222.000000pt}}
\pgflineto{\pgfpoint{41.600006pt}{222.000000pt}}
\pgfusepath{stroke}
\pgfpathmoveto{\pgfpoint{41.600006pt}{222.000000pt}}
\pgflineto{\pgfpoint{41.600006pt}{26.399979pt}}
\pgfusepath{stroke}
\pgfpathmoveto{\pgfpoint{289.600037pt}{222.000000pt}}
\pgflineto{\pgfpoint{289.600037pt}{26.399979pt}}
\pgfusepath{stroke}
\pgfpathmoveto{\pgfpoint{41.600006pt}{28.874924pt}}
\pgflineto{\pgfpoint{41.600006pt}{26.399979pt}}
\pgfusepath{stroke}
\pgfpathmoveto{\pgfpoint{41.600006pt}{219.525085pt}}
\pgflineto{\pgfpoint{41.600006pt}{222.000000pt}}
\pgfusepath{stroke}
\pgfpathmoveto{\pgfpoint{91.200012pt}{28.874924pt}}
\pgflineto{\pgfpoint{91.200012pt}{26.399979pt}}
\pgfusepath{stroke}
\pgfpathmoveto{\pgfpoint{91.200012pt}{219.525085pt}}
\pgflineto{\pgfpoint{91.200012pt}{222.000000pt}}
\pgfusepath{stroke}
\pgfpathmoveto{\pgfpoint{140.800003pt}{28.874924pt}}
\pgflineto{\pgfpoint{140.800003pt}{26.399979pt}}
\pgfusepath{stroke}
\pgfpathmoveto{\pgfpoint{140.800003pt}{219.525085pt}}
\pgflineto{\pgfpoint{140.800003pt}{222.000000pt}}
\pgfusepath{stroke}
\pgfpathmoveto{\pgfpoint{190.400024pt}{28.874924pt}}
\pgflineto{\pgfpoint{190.400024pt}{26.399979pt}}
\pgfusepath{stroke}
\pgfpathmoveto{\pgfpoint{190.400024pt}{219.525085pt}}
\pgflineto{\pgfpoint{190.400024pt}{222.000000pt}}
\pgfusepath{stroke}
\pgfpathmoveto{\pgfpoint{240.000000pt}{28.874924pt}}
\pgflineto{\pgfpoint{240.000000pt}{26.399979pt}}
\pgfusepath{stroke}
\pgfpathmoveto{\pgfpoint{240.000000pt}{219.525085pt}}
\pgflineto{\pgfpoint{240.000000pt}{222.000000pt}}
\pgfusepath{stroke}
\pgfpathmoveto{\pgfpoint{289.600037pt}{28.874924pt}}
\pgflineto{\pgfpoint{289.600037pt}{26.399979pt}}
\pgfusepath{stroke}
\pgfpathmoveto{\pgfpoint{289.600037pt}{219.525085pt}}
\pgflineto{\pgfpoint{289.600037pt}{222.000000pt}}
\pgfusepath{stroke}
{
\pgftransformshift{\pgfpoint{41.600006pt}{21.410187pt}}
\pgfnode{rectangle}{north}{\fontsize{10}{0}\selectfont\textcolor[rgb]{0,0,0}{{0}}}{}{\pgfusepath{discard}}}
{
\pgftransformshift{\pgfpoint{91.200012pt}{21.410187pt}}
\pgfnode{rectangle}{north}{\fontsize{10}{0}\selectfont\textcolor[rgb]{0,0,0}{{10}}}{}{\pgfusepath{discard}}}
{
\pgftransformshift{\pgfpoint{140.800003pt}{21.410187pt}}
\pgfnode{rectangle}{north}{\fontsize{10}{0}\selectfont\textcolor[rgb]{0,0,0}{{20}}}{}{\pgfusepath{discard}}}
{
\pgftransformshift{\pgfpoint{190.400009pt}{21.410187pt}}
\pgfnode{rectangle}{north}{\fontsize{10}{0}\selectfont\textcolor[rgb]{0,0,0}{{30}}}{}{\pgfusepath{discard}}}
{
\pgftransformshift{\pgfpoint{240.000000pt}{21.410187pt}}
\pgfnode{rectangle}{north}{\fontsize{10}{0}\selectfont\textcolor[rgb]{0,0,0}{{40}}}{}{\pgfusepath{discard}}}
{
\pgftransformshift{\pgfpoint{289.600037pt}{21.410187pt}}
\pgfnode{rectangle}{north}{\fontsize{10}{0}\selectfont\textcolor[rgb]{0,0,0}{{50}}}{}{\pgfusepath{discard}}}
\pgfpathmoveto{\pgfpoint{44.080009pt}{26.399979pt}}
\pgflineto{\pgfpoint{41.600006pt}{26.399979pt}}
\pgfusepath{stroke}
\pgfpathmoveto{\pgfpoint{287.120026pt}{26.399979pt}}
\pgflineto{\pgfpoint{289.600037pt}{26.399979pt}}
\pgfusepath{stroke}
\pgfpathmoveto{\pgfpoint{44.080009pt}{58.999985pt}}
\pgflineto{\pgfpoint{41.600006pt}{58.999985pt}}
\pgfusepath{stroke}
\pgfpathmoveto{\pgfpoint{287.120026pt}{58.999985pt}}
\pgflineto{\pgfpoint{289.600037pt}{58.999985pt}}
\pgfusepath{stroke}
\pgfpathmoveto{\pgfpoint{44.080009pt}{91.599991pt}}
\pgflineto{\pgfpoint{41.600006pt}{91.599991pt}}
\pgfusepath{stroke}
\pgfpathmoveto{\pgfpoint{287.120026pt}{91.599991pt}}
\pgflineto{\pgfpoint{289.600037pt}{91.599991pt}}
\pgfusepath{stroke}
\pgfpathmoveto{\pgfpoint{44.080009pt}{124.199997pt}}
\pgflineto{\pgfpoint{41.600006pt}{124.199997pt}}
\pgfusepath{stroke}
\pgfpathmoveto{\pgfpoint{287.120026pt}{124.199997pt}}
\pgflineto{\pgfpoint{289.600037pt}{124.199997pt}}
\pgfusepath{stroke}
\pgfpathmoveto{\pgfpoint{44.080009pt}{156.800003pt}}
\pgflineto{\pgfpoint{41.600006pt}{156.800003pt}}
\pgfusepath{stroke}
\pgfpathmoveto{\pgfpoint{287.120026pt}{156.800003pt}}
\pgflineto{\pgfpoint{289.600037pt}{156.800003pt}}
\pgfusepath{stroke}
\pgfpathmoveto{\pgfpoint{44.080009pt}{189.400009pt}}
\pgflineto{\pgfpoint{41.600006pt}{189.400009pt}}
\pgfusepath{stroke}
\pgfpathmoveto{\pgfpoint{287.120026pt}{189.400009pt}}
\pgflineto{\pgfpoint{289.600037pt}{189.400009pt}}
\pgfusepath{stroke}
\pgfpathmoveto{\pgfpoint{44.080009pt}{222.000000pt}}
\pgflineto{\pgfpoint{41.600006pt}{222.000000pt}}
\pgfusepath{stroke}
\pgfpathmoveto{\pgfpoint{287.120026pt}{222.000000pt}}
\pgflineto{\pgfpoint{289.600037pt}{222.000000pt}}
\pgfusepath{stroke}
{
\pgftransformshift{\pgfpoint{36.600006pt}{26.399979pt}}
\pgfnode{rectangle}{east}{\fontsize{10}{0}\selectfont\textcolor[rgb]{0,0,0}{{0}}}{}{\pgfusepath{discard}}}
{
\pgftransformshift{\pgfpoint{36.600006pt}{58.999985pt}}
\pgfnode{rectangle}{east}{\fontsize{10}{0}\selectfont\textcolor[rgb]{0,0,0}{{1}}}{}{\pgfusepath{discard}}}
{
\pgftransformshift{\pgfpoint{36.600006pt}{91.599991pt}}
\pgfnode{rectangle}{east}{\fontsize{10}{0}\selectfont\textcolor[rgb]{0,0,0}{{2}}}{}{\pgfusepath{discard}}}
{
\pgftransformshift{\pgfpoint{36.600006pt}{124.199989pt}}
\pgfnode{rectangle}{east}{\fontsize{10}{0}\selectfont\textcolor[rgb]{0,0,0}{{3}}}{}{\pgfusepath{discard}}}
{
\pgftransformshift{\pgfpoint{36.600006pt}{156.799988pt}}
\pgfnode{rectangle}{east}{\fontsize{10}{0}\selectfont\textcolor[rgb]{0,0,0}{{4}}}{}{\pgfusepath{discard}}}
{
\pgftransformshift{\pgfpoint{36.600006pt}{189.399994pt}}
\pgfnode{rectangle}{east}{\fontsize{10}{0}\selectfont\textcolor[rgb]{0,0,0}{{5}}}{}{\pgfusepath{discard}}}
{
\pgftransformshift{\pgfpoint{36.600006pt}{222.000000pt}}
\pgfnode{rectangle}{east}{\fontsize{10}{0}\selectfont\textcolor[rgb]{0,0,0}{{6}}}{}{\pgfusepath{discard}}}
\color[rgb]{0.000000,0.000000,1.000000}
\pgfsetdash{}{0pt}
\pgfpathmoveto{\pgfpoint{51.520004pt}{60.187065pt}}
\pgflineto{\pgfpoint{46.560013pt}{65.485382pt}}
\pgfusepath{stroke}
\pgfpathmoveto{\pgfpoint{56.480011pt}{42.665169pt}}
\pgflineto{\pgfpoint{51.520004pt}{60.187065pt}}
\pgfusepath{stroke}
\pgfpathmoveto{\pgfpoint{61.440010pt}{46.590111pt}}
\pgflineto{\pgfpoint{56.480011pt}{42.665169pt}}
\pgfusepath{stroke}
\pgfpathmoveto{\pgfpoint{66.400009pt}{63.127323pt}}
\pgflineto{\pgfpoint{61.440010pt}{46.590111pt}}
\pgfusepath{stroke}
\pgfpathmoveto{\pgfpoint{71.360008pt}{106.220398pt}}
\pgflineto{\pgfpoint{66.400009pt}{63.127323pt}}
\pgfusepath{stroke}
\pgfpathmoveto{\pgfpoint{76.320007pt}{80.649338pt}}
\pgflineto{\pgfpoint{71.360008pt}{106.220398pt}}
\pgfusepath{stroke}
\pgfpathmoveto{\pgfpoint{81.280014pt}{61.721870pt}}
\pgflineto{\pgfpoint{76.320007pt}{80.649338pt}}
\pgfusepath{stroke}
\pgfpathmoveto{\pgfpoint{86.240013pt}{60.315250pt}}
\pgflineto{\pgfpoint{81.280014pt}{61.721870pt}}
\pgfusepath{stroke}
\pgfpathmoveto{\pgfpoint{91.200012pt}{67.963776pt}}
\pgflineto{\pgfpoint{86.240013pt}{60.315250pt}}
\pgfusepath{stroke}
\pgfpathmoveto{\pgfpoint{96.160011pt}{39.422333pt}}
\pgflineto{\pgfpoint{91.200012pt}{67.963776pt}}
\pgfusepath{stroke}
\pgfpathmoveto{\pgfpoint{101.120010pt}{36.805199pt}}
\pgflineto{\pgfpoint{96.160011pt}{39.422333pt}}
\pgfusepath{stroke}
\pgfpathmoveto{\pgfpoint{106.080017pt}{57.417770pt}}
\pgflineto{\pgfpoint{101.120010pt}{36.805199pt}}
\pgfusepath{stroke}
\pgfpathmoveto{\pgfpoint{111.040009pt}{48.803902pt}}
\pgflineto{\pgfpoint{106.080017pt}{57.417770pt}}
\pgfusepath{stroke}
\pgfpathmoveto{\pgfpoint{116.000015pt}{40.934975pt}}
\pgflineto{\pgfpoint{111.040009pt}{48.803902pt}}
\pgfusepath{stroke}
\pgfpathmoveto{\pgfpoint{120.960007pt}{115.830338pt}}
\pgflineto{\pgfpoint{116.000015pt}{40.934975pt}}
\pgfusepath{stroke}
\pgfpathmoveto{\pgfpoint{125.920013pt}{66.937149pt}}
\pgflineto{\pgfpoint{120.960007pt}{115.830338pt}}
\pgfusepath{stroke}
\pgfpathmoveto{\pgfpoint{130.880005pt}{59.440186pt}}
\pgflineto{\pgfpoint{125.920013pt}{66.937149pt}}
\pgfusepath{stroke}
\pgfpathmoveto{\pgfpoint{135.840012pt}{37.031158pt}}
\pgflineto{\pgfpoint{130.880005pt}{59.440186pt}}
\pgfusepath{stroke}
\pgfpathmoveto{\pgfpoint{140.800003pt}{97.846222pt}}
\pgflineto{\pgfpoint{135.840012pt}{37.031158pt}}
\pgfusepath{stroke}
\pgfpathmoveto{\pgfpoint{145.760010pt}{49.291016pt}}
\pgflineto{\pgfpoint{140.800003pt}{97.846222pt}}
\pgfusepath{stroke}
\pgfpathmoveto{\pgfpoint{150.720016pt}{56.516876pt}}
\pgflineto{\pgfpoint{145.760010pt}{49.291016pt}}
\pgfusepath{stroke}
\pgfpathmoveto{\pgfpoint{155.680023pt}{39.056114pt}}
\pgflineto{\pgfpoint{150.720016pt}{56.516876pt}}
\pgfusepath{stroke}
\pgfpathmoveto{\pgfpoint{160.640015pt}{115.879601pt}}
\pgflineto{\pgfpoint{155.680023pt}{39.056114pt}}
\pgfusepath{stroke}
\pgfpathmoveto{\pgfpoint{165.600006pt}{60.773132pt}}
\pgflineto{\pgfpoint{160.640015pt}{115.879601pt}}
\pgfusepath{stroke}
\pgfpathmoveto{\pgfpoint{170.560013pt}{94.831261pt}}
\pgflineto{\pgfpoint{165.600006pt}{60.773132pt}}
\pgfusepath{stroke}
\pgfpathmoveto{\pgfpoint{175.520004pt}{43.668625pt}}
\pgflineto{\pgfpoint{170.560013pt}{94.831261pt}}
\pgfusepath{stroke}
\pgfpathmoveto{\pgfpoint{180.480011pt}{40.850761pt}}
\pgflineto{\pgfpoint{175.520004pt}{43.668625pt}}
\pgfusepath{stroke}
\pgfpathmoveto{\pgfpoint{185.440018pt}{94.016953pt}}
\pgflineto{\pgfpoint{180.480011pt}{40.850761pt}}
\pgfusepath{stroke}
\pgfpathmoveto{\pgfpoint{190.400024pt}{74.316742pt}}
\pgflineto{\pgfpoint{185.440018pt}{94.016953pt}}
\pgfusepath{stroke}
\pgfpathmoveto{\pgfpoint{195.360016pt}{39.186867pt}}
\pgflineto{\pgfpoint{190.400024pt}{74.316742pt}}
\pgfusepath{stroke}
\pgfpathmoveto{\pgfpoint{200.320007pt}{111.419617pt}}
\pgflineto{\pgfpoint{195.360016pt}{39.186867pt}}
\pgfusepath{stroke}
\pgfpathmoveto{\pgfpoint{205.279999pt}{77.342728pt}}
\pgflineto{\pgfpoint{200.320007pt}{111.419617pt}}
\pgfusepath{stroke}
\pgfpathmoveto{\pgfpoint{210.240021pt}{64.683121pt}}
\pgflineto{\pgfpoint{205.279999pt}{77.342728pt}}
\pgfusepath{stroke}
\pgfpathmoveto{\pgfpoint{215.200012pt}{68.944382pt}}
\pgflineto{\pgfpoint{210.240021pt}{64.683121pt}}
\pgfusepath{stroke}
\pgfpathmoveto{\pgfpoint{220.160019pt}{56.577995pt}}
\pgflineto{\pgfpoint{215.200012pt}{68.944382pt}}
\pgfusepath{stroke}
\pgfpathmoveto{\pgfpoint{225.120026pt}{63.517673pt}}
\pgflineto{\pgfpoint{220.160019pt}{56.577995pt}}
\pgfusepath{stroke}
\pgfpathmoveto{\pgfpoint{230.080017pt}{66.862328pt}}
\pgflineto{\pgfpoint{225.120026pt}{63.517673pt}}
\pgfusepath{stroke}
\pgfpathmoveto{\pgfpoint{235.040024pt}{53.250259pt}}
\pgflineto{\pgfpoint{230.080017pt}{66.862328pt}}
\pgfusepath{stroke}
\pgfpathmoveto{\pgfpoint{240.000000pt}{75.645462pt}}
\pgflineto{\pgfpoint{235.040024pt}{53.250259pt}}
\pgfusepath{stroke}
\pgfpathmoveto{\pgfpoint{244.960022pt}{39.255608pt}}
\pgflineto{\pgfpoint{240.000000pt}{75.645462pt}}
\pgfusepath{stroke}
\pgfpathmoveto{\pgfpoint{249.920013pt}{106.094025pt}}
\pgflineto{\pgfpoint{244.960022pt}{39.255608pt}}
\pgfusepath{stroke}
\pgfpathmoveto{\pgfpoint{254.880020pt}{70.739304pt}}
\pgflineto{\pgfpoint{249.920013pt}{106.094025pt}}
\pgfusepath{stroke}
\pgfpathmoveto{\pgfpoint{259.840027pt}{73.925293pt}}
\pgflineto{\pgfpoint{254.880020pt}{70.739304pt}}
\pgfusepath{stroke}
\pgfpathmoveto{\pgfpoint{264.800018pt}{34.848969pt}}
\pgflineto{\pgfpoint{259.840027pt}{73.925293pt}}
\pgfusepath{stroke}
\pgfpathmoveto{\pgfpoint{269.760010pt}{71.767647pt}}
\pgflineto{\pgfpoint{264.800018pt}{34.848969pt}}
\pgfusepath{stroke}
\pgfpathmoveto{\pgfpoint{274.720001pt}{76.811203pt}}
\pgflineto{\pgfpoint{269.760010pt}{71.767647pt}}
\pgfusepath{stroke}
\pgfpathmoveto{\pgfpoint{279.680023pt}{69.222656pt}}
\pgflineto{\pgfpoint{274.720001pt}{76.811203pt}}
\pgfusepath{stroke}
\pgfpathmoveto{\pgfpoint{284.640015pt}{36.803474pt}}
\pgflineto{\pgfpoint{279.680023pt}{69.222656pt}}
\pgfusepath{stroke}
\pgfpathmoveto{\pgfpoint{289.600037pt}{71.175858pt}}
\pgflineto{\pgfpoint{284.640015pt}{36.803474pt}}
\pgfusepath{stroke}
\color[rgb]{0.000000,0.500000,0.000000}
\pgfpathmoveto{\pgfpoint{51.520004pt}{54.621078pt}}
\pgflineto{\pgfpoint{46.560013pt}{171.656006pt}}
\pgfusepath{stroke}
\pgfpathmoveto{\pgfpoint{56.480011pt}{59.932564pt}}
\pgflineto{\pgfpoint{51.520004pt}{54.621078pt}}
\pgfusepath{stroke}
\pgfpathmoveto{\pgfpoint{61.440010pt}{105.284721pt}}
\pgflineto{\pgfpoint{56.480011pt}{59.932564pt}}
\pgfusepath{stroke}
\pgfpathmoveto{\pgfpoint{66.400009pt}{163.107224pt}}
\pgflineto{\pgfpoint{61.440010pt}{105.284721pt}}
\pgfusepath{stroke}
\pgfpathmoveto{\pgfpoint{71.360008pt}{43.829178pt}}
\pgflineto{\pgfpoint{66.400009pt}{163.107224pt}}
\pgfusepath{stroke}
\pgfpathmoveto{\pgfpoint{76.320007pt}{140.828415pt}}
\pgflineto{\pgfpoint{71.360008pt}{43.829178pt}}
\pgfusepath{stroke}
\pgfpathmoveto{\pgfpoint{81.280014pt}{44.592613pt}}
\pgflineto{\pgfpoint{76.320007pt}{140.828415pt}}
\pgfusepath{stroke}
\pgfpathmoveto{\pgfpoint{86.240013pt}{61.582748pt}}
\pgflineto{\pgfpoint{81.280014pt}{44.592613pt}}
\pgfusepath{stroke}
\pgfpathmoveto{\pgfpoint{91.200012pt}{82.405769pt}}
\pgflineto{\pgfpoint{86.240013pt}{61.582748pt}}
\pgfusepath{stroke}
\pgfpathmoveto{\pgfpoint{96.160011pt}{121.447739pt}}
\pgflineto{\pgfpoint{91.200012pt}{82.405769pt}}
\pgfusepath{stroke}
\pgfpathmoveto{\pgfpoint{101.120010pt}{45.170807pt}}
\pgflineto{\pgfpoint{96.160011pt}{121.447739pt}}
\pgfusepath{stroke}
\pgfpathmoveto{\pgfpoint{106.080017pt}{93.045525pt}}
\pgflineto{\pgfpoint{101.120010pt}{45.170807pt}}
\pgfusepath{stroke}
\pgfpathmoveto{\pgfpoint{111.040009pt}{89.065750pt}}
\pgflineto{\pgfpoint{106.080017pt}{93.045525pt}}
\pgfusepath{stroke}
\pgfpathmoveto{\pgfpoint{116.000015pt}{43.683624pt}}
\pgflineto{\pgfpoint{111.040009pt}{89.065750pt}}
\pgfusepath{stroke}
\pgfpathmoveto{\pgfpoint{120.960007pt}{149.515289pt}}
\pgflineto{\pgfpoint{116.000015pt}{43.683624pt}}
\pgfusepath{stroke}
\pgfpathmoveto{\pgfpoint{125.920013pt}{72.071335pt}}
\pgflineto{\pgfpoint{120.960007pt}{149.515289pt}}
\pgfusepath{stroke}
\pgfpathmoveto{\pgfpoint{130.880005pt}{46.801132pt}}
\pgflineto{\pgfpoint{125.920013pt}{72.071335pt}}
\pgfusepath{stroke}
\pgfpathmoveto{\pgfpoint{135.840012pt}{69.273598pt}}
\pgflineto{\pgfpoint{130.880005pt}{46.801132pt}}
\pgfusepath{stroke}
\pgfpathmoveto{\pgfpoint{140.800003pt}{63.190159pt}}
\pgflineto{\pgfpoint{135.840012pt}{69.273598pt}}
\pgfusepath{stroke}
\pgfpathmoveto{\pgfpoint{145.760010pt}{45.087975pt}}
\pgflineto{\pgfpoint{140.800003pt}{63.190159pt}}
\pgfusepath{stroke}
\pgfpathmoveto{\pgfpoint{150.720016pt}{59.958729pt}}
\pgflineto{\pgfpoint{145.760010pt}{45.087975pt}}
\pgfusepath{stroke}
\pgfpathmoveto{\pgfpoint{155.680023pt}{145.320908pt}}
\pgflineto{\pgfpoint{150.720016pt}{59.958729pt}}
\pgfusepath{stroke}
\pgfpathmoveto{\pgfpoint{160.640015pt}{55.971481pt}}
\pgflineto{\pgfpoint{155.680023pt}{145.320908pt}}
\pgfusepath{stroke}
\pgfpathmoveto{\pgfpoint{165.600006pt}{64.977737pt}}
\pgflineto{\pgfpoint{160.640015pt}{55.971481pt}}
\pgfusepath{stroke}
\pgfpathmoveto{\pgfpoint{170.560013pt}{48.365959pt}}
\pgflineto{\pgfpoint{165.600006pt}{64.977737pt}}
\pgfusepath{stroke}
\pgfpathmoveto{\pgfpoint{175.520004pt}{60.118607pt}}
\pgflineto{\pgfpoint{170.560013pt}{48.365959pt}}
\pgfusepath{stroke}
\pgfpathmoveto{\pgfpoint{180.480011pt}{75.116745pt}}
\pgflineto{\pgfpoint{175.520004pt}{60.118607pt}}
\pgfusepath{stroke}
\pgfpathmoveto{\pgfpoint{185.440018pt}{55.456352pt}}
\pgflineto{\pgfpoint{180.480011pt}{75.116745pt}}
\pgfusepath{stroke}
\pgfpathmoveto{\pgfpoint{190.400024pt}{40.646240pt}}
\pgflineto{\pgfpoint{185.440018pt}{55.456352pt}}
\pgfusepath{stroke}
\pgfpathmoveto{\pgfpoint{195.360016pt}{50.795929pt}}
\pgflineto{\pgfpoint{190.400024pt}{40.646240pt}}
\pgfusepath{stroke}
\pgfpathmoveto{\pgfpoint{200.320007pt}{162.896408pt}}
\pgflineto{\pgfpoint{195.360016pt}{50.795929pt}}
\pgfusepath{stroke}
\pgfpathmoveto{\pgfpoint{205.279999pt}{99.158791pt}}
\pgflineto{\pgfpoint{200.320007pt}{162.896408pt}}
\pgfusepath{stroke}
\pgfpathmoveto{\pgfpoint{210.240021pt}{64.267227pt}}
\pgflineto{\pgfpoint{205.279999pt}{99.158791pt}}
\pgfusepath{stroke}
\pgfpathmoveto{\pgfpoint{215.200012pt}{36.115891pt}}
\pgflineto{\pgfpoint{210.240021pt}{64.267227pt}}
\pgfusepath{stroke}
\pgfpathmoveto{\pgfpoint{220.160019pt}{48.298035pt}}
\pgflineto{\pgfpoint{215.200012pt}{36.115891pt}}
\pgfusepath{stroke}
\pgfpathmoveto{\pgfpoint{225.120026pt}{51.340904pt}}
\pgflineto{\pgfpoint{220.160019pt}{48.298035pt}}
\pgfusepath{stroke}
\pgfpathmoveto{\pgfpoint{230.080017pt}{71.875427pt}}
\pgflineto{\pgfpoint{225.120026pt}{51.340904pt}}
\pgfusepath{stroke}
\pgfpathmoveto{\pgfpoint{235.040024pt}{43.997490pt}}
\pgflineto{\pgfpoint{230.080017pt}{71.875427pt}}
\pgfusepath{stroke}
\pgfpathmoveto{\pgfpoint{240.000000pt}{61.037575pt}}
\pgflineto{\pgfpoint{235.040024pt}{43.997490pt}}
\pgfusepath{stroke}
\pgfpathmoveto{\pgfpoint{244.960022pt}{40.726631pt}}
\pgflineto{\pgfpoint{240.000000pt}{61.037575pt}}
\pgfusepath{stroke}
\pgfpathmoveto{\pgfpoint{249.920013pt}{116.421631pt}}
\pgflineto{\pgfpoint{244.960022pt}{40.726631pt}}
\pgfusepath{stroke}
\pgfpathmoveto{\pgfpoint{254.880020pt}{215.171143pt}}
\pgflineto{\pgfpoint{249.920013pt}{116.421631pt}}
\pgfusepath{stroke}
\pgfpathmoveto{\pgfpoint{259.840027pt}{141.531891pt}}
\pgflineto{\pgfpoint{254.880020pt}{215.171143pt}}
\pgfusepath{stroke}
\pgfpathmoveto{\pgfpoint{264.800018pt}{190.998230pt}}
\pgflineto{\pgfpoint{259.840027pt}{141.531891pt}}
\pgfusepath{stroke}
\pgfpathmoveto{\pgfpoint{269.760010pt}{62.679287pt}}
\pgflineto{\pgfpoint{264.800018pt}{190.998230pt}}
\pgfusepath{stroke}
\pgfpathmoveto{\pgfpoint{274.720001pt}{37.509773pt}}
\pgflineto{\pgfpoint{269.760010pt}{62.679287pt}}
\pgfusepath{stroke}
\pgfpathmoveto{\pgfpoint{279.680023pt}{54.383209pt}}
\pgflineto{\pgfpoint{274.720001pt}{37.509773pt}}
\pgfusepath{stroke}
\pgfpathmoveto{\pgfpoint{284.640015pt}{82.386887pt}}
\pgflineto{\pgfpoint{279.680023pt}{54.383209pt}}
\pgfusepath{stroke}
\pgfpathmoveto{\pgfpoint{289.600037pt}{61.113129pt}}
\pgflineto{\pgfpoint{284.640015pt}{82.386887pt}}
\pgfusepath{stroke}
\color[rgb]{0.000000,0.000000,0.000000}
\pgfpathmoveto{\pgfpoint{288.703094pt}{64.960159pt}}
\pgflineto{\pgfpoint{289.600037pt}{61.113129pt}}
\pgfusepath{stroke}
\pgfpathmoveto{\pgfpoint{289.600037pt}{71.175858pt}}
\pgflineto{\pgfpoint{288.703094pt}{64.960159pt}}
\pgfusepath{stroke}
\pgfpathmoveto{\pgfpoint{289.600037pt}{61.113129pt}}
\pgflineto{\pgfpoint{289.600037pt}{71.175858pt}}
\pgfusepath{stroke}
\pgfpathmoveto{\pgfpoint{284.640015pt}{36.803474pt}}
\pgflineto{\pgfpoint{288.703094pt}{64.960159pt}}
\pgfusepath{stroke}
\pgfpathmoveto{\pgfpoint{280.898163pt}{61.260727pt}}
\pgflineto{\pgfpoint{284.640015pt}{36.803474pt}}
\pgfusepath{stroke}
\pgfpathmoveto{\pgfpoint{284.640015pt}{82.386887pt}}
\pgflineto{\pgfpoint{280.898163pt}{61.260727pt}}
\pgfusepath{stroke}
\pgfpathmoveto{\pgfpoint{288.703094pt}{64.960159pt}}
\pgflineto{\pgfpoint{284.640015pt}{82.386887pt}}
\pgfusepath{stroke}
\pgfpathmoveto{\pgfpoint{279.680023pt}{54.383209pt}}
\pgflineto{\pgfpoint{280.898163pt}{61.260727pt}}
\pgfusepath{stroke}
\pgfpathmoveto{\pgfpoint{274.720001pt}{37.509773pt}}
\pgflineto{\pgfpoint{279.680023pt}{54.383209pt}}
\pgfusepath{stroke}
\pgfpathmoveto{\pgfpoint{269.760010pt}{62.679287pt}}
\pgflineto{\pgfpoint{274.720001pt}{37.509773pt}}
\pgfusepath{stroke}
\pgfpathmoveto{\pgfpoint{269.487213pt}{69.737061pt}}
\pgflineto{\pgfpoint{269.760010pt}{62.679287pt}}
\pgfusepath{stroke}
\pgfpathmoveto{\pgfpoint{269.760010pt}{71.767647pt}}
\pgflineto{\pgfpoint{269.487213pt}{69.737061pt}}
\pgfusepath{stroke}
\pgfpathmoveto{\pgfpoint{274.720001pt}{76.811203pt}}
\pgflineto{\pgfpoint{269.760010pt}{71.767647pt}}
\pgfusepath{stroke}
\pgfpathmoveto{\pgfpoint{279.680023pt}{69.222656pt}}
\pgflineto{\pgfpoint{274.720001pt}{76.811203pt}}
\pgfusepath{stroke}
\pgfpathmoveto{\pgfpoint{280.898163pt}{61.260727pt}}
\pgflineto{\pgfpoint{279.680023pt}{69.222656pt}}
\pgfusepath{stroke}
\pgfpathmoveto{\pgfpoint{264.800018pt}{34.848969pt}}
\pgflineto{\pgfpoint{269.487213pt}{69.737061pt}}
\pgfusepath{stroke}
\pgfpathmoveto{\pgfpoint{259.840027pt}{73.925293pt}}
\pgflineto{\pgfpoint{264.800018pt}{34.848969pt}}
\pgfusepath{stroke}
\pgfpathmoveto{\pgfpoint{254.880020pt}{70.739304pt}}
\pgflineto{\pgfpoint{259.840027pt}{73.925293pt}}
\pgfusepath{stroke}
\pgfpathmoveto{\pgfpoint{249.920013pt}{106.094025pt}}
\pgflineto{\pgfpoint{254.880020pt}{70.739304pt}}
\pgfusepath{stroke}
\pgfpathmoveto{\pgfpoint{244.960022pt}{39.255608pt}}
\pgflineto{\pgfpoint{249.920013pt}{106.094025pt}}
\pgfusepath{stroke}
\pgfpathmoveto{\pgfpoint{244.506226pt}{42.584846pt}}
\pgflineto{\pgfpoint{244.960022pt}{39.255608pt}}
\pgfusepath{stroke}
\pgfpathmoveto{\pgfpoint{244.960022pt}{40.726631pt}}
\pgflineto{\pgfpoint{244.506226pt}{42.584846pt}}
\pgfusepath{stroke}
\pgfpathmoveto{\pgfpoint{249.920013pt}{116.421631pt}}
\pgflineto{\pgfpoint{244.960022pt}{40.726631pt}}
\pgfusepath{stroke}
\pgfpathmoveto{\pgfpoint{254.880020pt}{215.171143pt}}
\pgflineto{\pgfpoint{249.920013pt}{116.421631pt}}
\pgfusepath{stroke}
\pgfpathmoveto{\pgfpoint{259.840027pt}{141.531891pt}}
\pgflineto{\pgfpoint{254.880020pt}{215.171143pt}}
\pgfusepath{stroke}
\pgfpathmoveto{\pgfpoint{264.800018pt}{190.998230pt}}
\pgflineto{\pgfpoint{259.840027pt}{141.531891pt}}
\pgfusepath{stroke}
\pgfpathmoveto{\pgfpoint{269.487213pt}{69.737061pt}}
\pgflineto{\pgfpoint{264.800018pt}{190.998230pt}}
\pgfusepath{stroke}
\pgfpathmoveto{\pgfpoint{240.000000pt}{61.037575pt}}
\pgflineto{\pgfpoint{244.506226pt}{42.584846pt}}
\pgfusepath{stroke}
\pgfpathmoveto{\pgfpoint{235.040024pt}{43.997490pt}}
\pgflineto{\pgfpoint{240.000000pt}{61.037575pt}}
\pgfusepath{stroke}
\pgfpathmoveto{\pgfpoint{231.822998pt}{62.078987pt}}
\pgflineto{\pgfpoint{235.040024pt}{43.997490pt}}
\pgfusepath{stroke}
\pgfpathmoveto{\pgfpoint{235.040024pt}{53.250259pt}}
\pgflineto{\pgfpoint{231.822998pt}{62.078987pt}}
\pgfusepath{stroke}
\pgfpathmoveto{\pgfpoint{240.000000pt}{75.645462pt}}
\pgflineto{\pgfpoint{235.040024pt}{53.250259pt}}
\pgfusepath{stroke}
\pgfpathmoveto{\pgfpoint{244.506226pt}{42.584846pt}}
\pgflineto{\pgfpoint{240.000000pt}{75.645462pt}}
\pgfusepath{stroke}
\pgfpathmoveto{\pgfpoint{230.080017pt}{66.862328pt}}
\pgflineto{\pgfpoint{231.822998pt}{62.078987pt}}
\pgfusepath{stroke}
\pgfpathmoveto{\pgfpoint{228.633514pt}{65.886925pt}}
\pgflineto{\pgfpoint{230.080017pt}{66.862328pt}}
\pgfusepath{stroke}
\pgfpathmoveto{\pgfpoint{230.080017pt}{71.875427pt}}
\pgflineto{\pgfpoint{228.633514pt}{65.886925pt}}
\pgfusepath{stroke}
\pgfpathmoveto{\pgfpoint{231.822998pt}{62.078987pt}}
\pgflineto{\pgfpoint{230.080017pt}{71.875427pt}}
\pgfusepath{stroke}
\pgfpathmoveto{\pgfpoint{225.120026pt}{51.340904pt}}
\pgflineto{\pgfpoint{228.633514pt}{65.886925pt}}
\pgfusepath{stroke}
\pgfpathmoveto{\pgfpoint{220.160019pt}{48.298035pt}}
\pgflineto{\pgfpoint{225.120026pt}{51.340904pt}}
\pgfusepath{stroke}
\pgfpathmoveto{\pgfpoint{215.200012pt}{36.115891pt}}
\pgflineto{\pgfpoint{220.160019pt}{48.298035pt}}
\pgfusepath{stroke}
\pgfpathmoveto{\pgfpoint{210.240021pt}{64.267227pt}}
\pgflineto{\pgfpoint{215.200012pt}{36.115891pt}}
\pgfusepath{stroke}
\pgfpathmoveto{\pgfpoint{210.147217pt}{64.919937pt}}
\pgflineto{\pgfpoint{210.240021pt}{64.267227pt}}
\pgfusepath{stroke}
\pgfpathmoveto{\pgfpoint{210.240021pt}{64.683121pt}}
\pgflineto{\pgfpoint{210.147217pt}{64.919937pt}}
\pgfusepath{stroke}
\pgfpathmoveto{\pgfpoint{215.200012pt}{68.944382pt}}
\pgflineto{\pgfpoint{210.240021pt}{64.683121pt}}
\pgfusepath{stroke}
\pgfpathmoveto{\pgfpoint{220.160019pt}{56.577995pt}}
\pgflineto{\pgfpoint{215.200012pt}{68.944382pt}}
\pgfusepath{stroke}
\pgfpathmoveto{\pgfpoint{225.120026pt}{63.517673pt}}
\pgflineto{\pgfpoint{220.160019pt}{56.577995pt}}
\pgfusepath{stroke}
\pgfpathmoveto{\pgfpoint{228.633514pt}{65.886925pt}}
\pgflineto{\pgfpoint{225.120026pt}{63.517673pt}}
\pgfusepath{stroke}
\pgfpathmoveto{\pgfpoint{205.279999pt}{77.342728pt}}
\pgflineto{\pgfpoint{210.147217pt}{64.919937pt}}
\pgfusepath{stroke}
\pgfpathmoveto{\pgfpoint{200.320007pt}{111.419617pt}}
\pgflineto{\pgfpoint{205.279999pt}{77.342728pt}}
\pgfusepath{stroke}
\pgfpathmoveto{\pgfpoint{195.360016pt}{39.186867pt}}
\pgflineto{\pgfpoint{200.320007pt}{111.419617pt}}
\pgfusepath{stroke}
\pgfpathmoveto{\pgfpoint{194.088348pt}{48.193687pt}}
\pgflineto{\pgfpoint{195.360016pt}{39.186867pt}}
\pgfusepath{stroke}
\pgfpathmoveto{\pgfpoint{195.360016pt}{50.795929pt}}
\pgflineto{\pgfpoint{194.088348pt}{48.193687pt}}
\pgfusepath{stroke}
\pgfpathmoveto{\pgfpoint{200.320007pt}{162.896408pt}}
\pgflineto{\pgfpoint{195.360016pt}{50.795929pt}}
\pgfusepath{stroke}
\pgfpathmoveto{\pgfpoint{205.279999pt}{99.158791pt}}
\pgflineto{\pgfpoint{200.320007pt}{162.896408pt}}
\pgfusepath{stroke}
\pgfpathmoveto{\pgfpoint{210.147217pt}{64.919937pt}}
\pgflineto{\pgfpoint{205.279999pt}{99.158791pt}}
\pgfusepath{stroke}
\pgfpathmoveto{\pgfpoint{190.400024pt}{40.646240pt}}
\pgflineto{\pgfpoint{194.088348pt}{48.193687pt}}
\pgfusepath{stroke}
\pgfpathmoveto{\pgfpoint{185.440018pt}{55.456352pt}}
\pgflineto{\pgfpoint{190.400024pt}{40.646240pt}}
\pgfusepath{stroke}
\pgfpathmoveto{\pgfpoint{182.813766pt}{65.866241pt}}
\pgflineto{\pgfpoint{185.440018pt}{55.456352pt}}
\pgfusepath{stroke}
\pgfpathmoveto{\pgfpoint{185.440018pt}{94.016953pt}}
\pgflineto{\pgfpoint{182.813766pt}{65.866241pt}}
\pgfusepath{stroke}
\pgfpathmoveto{\pgfpoint{190.400024pt}{74.316742pt}}
\pgflineto{\pgfpoint{185.440018pt}{94.016953pt}}
\pgfusepath{stroke}
\pgfpathmoveto{\pgfpoint{194.088348pt}{48.193687pt}}
\pgflineto{\pgfpoint{190.400024pt}{74.316742pt}}
\pgfusepath{stroke}
\pgfpathmoveto{\pgfpoint{180.480011pt}{40.850761pt}}
\pgflineto{\pgfpoint{182.813766pt}{65.866241pt}}
\pgfusepath{stroke}
\pgfpathmoveto{\pgfpoint{175.520004pt}{43.668625pt}}
\pgflineto{\pgfpoint{180.480011pt}{40.850761pt}}
\pgfusepath{stroke}
\pgfpathmoveto{\pgfpoint{174.223160pt}{57.045738pt}}
\pgflineto{\pgfpoint{175.520004pt}{43.668625pt}}
\pgfusepath{stroke}
\pgfpathmoveto{\pgfpoint{175.520004pt}{60.118607pt}}
\pgflineto{\pgfpoint{174.223160pt}{57.045738pt}}
\pgfusepath{stroke}
\pgfpathmoveto{\pgfpoint{180.480011pt}{75.116745pt}}
\pgflineto{\pgfpoint{175.520004pt}{60.118607pt}}
\pgfusepath{stroke}
\pgfpathmoveto{\pgfpoint{182.813766pt}{65.866241pt}}
\pgflineto{\pgfpoint{180.480011pt}{75.116745pt}}
\pgfusepath{stroke}
\pgfpathmoveto{\pgfpoint{170.560013pt}{48.365959pt}}
\pgflineto{\pgfpoint{174.223160pt}{57.045738pt}}
\pgfusepath{stroke}
\pgfpathmoveto{\pgfpoint{166.011597pt}{63.599285pt}}
\pgflineto{\pgfpoint{170.560013pt}{48.365959pt}}
\pgfusepath{stroke}
\pgfpathmoveto{\pgfpoint{170.560013pt}{94.831261pt}}
\pgflineto{\pgfpoint{166.011597pt}{63.599285pt}}
\pgfusepath{stroke}
\pgfpathmoveto{\pgfpoint{174.223160pt}{57.045738pt}}
\pgflineto{\pgfpoint{170.560013pt}{94.831261pt}}
\pgfusepath{stroke}
\pgfpathmoveto{\pgfpoint{165.600006pt}{60.773132pt}}
\pgflineto{\pgfpoint{166.011597pt}{63.599285pt}}
\pgfusepath{stroke}
\pgfpathmoveto{\pgfpoint{165.274719pt}{64.387100pt}}
\pgflineto{\pgfpoint{165.600006pt}{60.773132pt}}
\pgfusepath{stroke}
\pgfpathmoveto{\pgfpoint{165.600006pt}{64.977737pt}}
\pgflineto{\pgfpoint{165.274719pt}{64.387100pt}}
\pgfusepath{stroke}
\pgfpathmoveto{\pgfpoint{166.011597pt}{63.599285pt}}
\pgflineto{\pgfpoint{165.600006pt}{64.977737pt}}
\pgfusepath{stroke}
\pgfpathmoveto{\pgfpoint{160.640015pt}{55.971481pt}}
\pgflineto{\pgfpoint{165.274719pt}{64.387100pt}}
\pgfusepath{stroke}
\pgfpathmoveto{\pgfpoint{158.851837pt}{88.183441pt}}
\pgflineto{\pgfpoint{160.640015pt}{55.971481pt}}
\pgfusepath{stroke}
\pgfpathmoveto{\pgfpoint{160.640015pt}{115.879601pt}}
\pgflineto{\pgfpoint{158.851837pt}{88.183441pt}}
\pgfusepath{stroke}
\pgfpathmoveto{\pgfpoint{165.274719pt}{64.387100pt}}
\pgflineto{\pgfpoint{160.640015pt}{115.879601pt}}
\pgfusepath{stroke}
\pgfpathmoveto{\pgfpoint{155.680023pt}{39.056114pt}}
\pgflineto{\pgfpoint{158.851837pt}{88.183441pt}}
\pgfusepath{stroke}
\pgfpathmoveto{\pgfpoint{150.720016pt}{56.516876pt}}
\pgflineto{\pgfpoint{155.680023pt}{39.056114pt}}
\pgfusepath{stroke}
\pgfpathmoveto{\pgfpoint{148.486938pt}{53.263680pt}}
\pgflineto{\pgfpoint{150.720016pt}{56.516876pt}}
\pgfusepath{stroke}
\pgfpathmoveto{\pgfpoint{150.720016pt}{59.958729pt}}
\pgflineto{\pgfpoint{148.486938pt}{53.263680pt}}
\pgfusepath{stroke}
\pgfpathmoveto{\pgfpoint{155.680023pt}{145.320908pt}}
\pgflineto{\pgfpoint{150.720016pt}{59.958729pt}}
\pgfusepath{stroke}
\pgfpathmoveto{\pgfpoint{158.851837pt}{88.183441pt}}
\pgflineto{\pgfpoint{155.680023pt}{145.320908pt}}
\pgfusepath{stroke}
\pgfpathmoveto{\pgfpoint{145.760010pt}{45.087975pt}}
\pgflineto{\pgfpoint{148.486938pt}{53.263680pt}}
\pgfusepath{stroke}
\pgfpathmoveto{\pgfpoint{140.800003pt}{63.190159pt}}
\pgflineto{\pgfpoint{145.760010pt}{45.087975pt}}
\pgfusepath{stroke}
\pgfpathmoveto{\pgfpoint{138.230530pt}{66.341614pt}}
\pgflineto{\pgfpoint{140.800003pt}{63.190159pt}}
\pgfusepath{stroke}
\pgfpathmoveto{\pgfpoint{140.800003pt}{97.846222pt}}
\pgflineto{\pgfpoint{138.230530pt}{66.341614pt}}
\pgfusepath{stroke}
\pgfpathmoveto{\pgfpoint{145.760010pt}{49.291016pt}}
\pgflineto{\pgfpoint{140.800003pt}{97.846222pt}}
\pgfusepath{stroke}
\pgfpathmoveto{\pgfpoint{148.486938pt}{53.263680pt}}
\pgflineto{\pgfpoint{145.760010pt}{49.291016pt}}
\pgfusepath{stroke}
\pgfpathmoveto{\pgfpoint{135.840012pt}{37.031158pt}}
\pgflineto{\pgfpoint{138.230530pt}{66.341614pt}}
\pgfusepath{stroke}
\pgfpathmoveto{\pgfpoint{132.276794pt}{53.129593pt}}
\pgflineto{\pgfpoint{135.840012pt}{37.031158pt}}
\pgfusepath{stroke}
\pgfpathmoveto{\pgfpoint{135.840012pt}{69.273598pt}}
\pgflineto{\pgfpoint{132.276794pt}{53.129593pt}}
\pgfusepath{stroke}
\pgfpathmoveto{\pgfpoint{138.230530pt}{66.341614pt}}
\pgflineto{\pgfpoint{135.840012pt}{69.273598pt}}
\pgfusepath{stroke}
\pgfpathmoveto{\pgfpoint{130.880005pt}{46.801132pt}}
\pgflineto{\pgfpoint{132.276794pt}{53.129593pt}}
\pgfusepath{stroke}
\pgfpathmoveto{\pgfpoint{127.352821pt}{64.771484pt}}
\pgflineto{\pgfpoint{130.880005pt}{46.801132pt}}
\pgfusepath{stroke}
\pgfpathmoveto{\pgfpoint{130.880005pt}{59.440186pt}}
\pgflineto{\pgfpoint{127.352821pt}{64.771484pt}}
\pgfusepath{stroke}
\pgfpathmoveto{\pgfpoint{132.276794pt}{53.129593pt}}
\pgflineto{\pgfpoint{130.880005pt}{59.440186pt}}
\pgfusepath{stroke}
\pgfpathmoveto{\pgfpoint{125.920013pt}{66.937149pt}}
\pgflineto{\pgfpoint{127.352821pt}{64.771484pt}}
\pgfusepath{stroke}
\pgfpathmoveto{\pgfpoint{120.960007pt}{115.830338pt}}
\pgflineto{\pgfpoint{125.920013pt}{66.937149pt}}
\pgfusepath{stroke}
\pgfpathmoveto{\pgfpoint{116.000015pt}{40.934975pt}}
\pgflineto{\pgfpoint{120.960007pt}{115.830338pt}}
\pgfusepath{stroke}
\pgfpathmoveto{\pgfpoint{111.040009pt}{48.803902pt}}
\pgflineto{\pgfpoint{116.000015pt}{40.934975pt}}
\pgfusepath{stroke}
\pgfpathmoveto{\pgfpoint{106.080017pt}{57.417770pt}}
\pgflineto{\pgfpoint{111.040009pt}{48.803902pt}}
\pgfusepath{stroke}
\pgfpathmoveto{\pgfpoint{101.120010pt}{36.805199pt}}
\pgflineto{\pgfpoint{106.080017pt}{57.417770pt}}
\pgfusepath{stroke}
\pgfpathmoveto{\pgfpoint{96.160011pt}{39.422333pt}}
\pgflineto{\pgfpoint{101.120010pt}{36.805199pt}}
\pgfusepath{stroke}
\pgfpathmoveto{\pgfpoint{91.200012pt}{67.963776pt}}
\pgflineto{\pgfpoint{96.160011pt}{39.422333pt}}
\pgfusepath{stroke}
\pgfpathmoveto{\pgfpoint{86.240013pt}{60.315250pt}}
\pgflineto{\pgfpoint{91.200012pt}{67.963776pt}}
\pgfusepath{stroke}
\pgfpathmoveto{\pgfpoint{85.898277pt}{60.412163pt}}
\pgflineto{\pgfpoint{86.240013pt}{60.315250pt}}
\pgfusepath{stroke}
\pgfpathmoveto{\pgfpoint{86.240013pt}{61.582748pt}}
\pgflineto{\pgfpoint{85.898277pt}{60.412163pt}}
\pgfusepath{stroke}
\pgfpathmoveto{\pgfpoint{91.200012pt}{82.405769pt}}
\pgflineto{\pgfpoint{86.240013pt}{61.582748pt}}
\pgfusepath{stroke}
\pgfpathmoveto{\pgfpoint{96.160011pt}{121.447739pt}}
\pgflineto{\pgfpoint{91.200012pt}{82.405769pt}}
\pgfusepath{stroke}
\pgfpathmoveto{\pgfpoint{101.120010pt}{45.170807pt}}
\pgflineto{\pgfpoint{96.160011pt}{121.447739pt}}
\pgfusepath{stroke}
\pgfpathmoveto{\pgfpoint{106.080017pt}{93.045525pt}}
\pgflineto{\pgfpoint{101.120010pt}{45.170807pt}}
\pgfusepath{stroke}
\pgfpathmoveto{\pgfpoint{111.040009pt}{89.065750pt}}
\pgflineto{\pgfpoint{106.080017pt}{93.045525pt}}
\pgfusepath{stroke}
\pgfpathmoveto{\pgfpoint{116.000015pt}{43.683624pt}}
\pgflineto{\pgfpoint{111.040009pt}{89.065750pt}}
\pgfusepath{stroke}
\pgfpathmoveto{\pgfpoint{120.960007pt}{149.515289pt}}
\pgflineto{\pgfpoint{116.000015pt}{43.683624pt}}
\pgfusepath{stroke}
\pgfpathmoveto{\pgfpoint{125.920013pt}{72.071335pt}}
\pgflineto{\pgfpoint{120.960007pt}{149.515289pt}}
\pgfusepath{stroke}
\pgfpathmoveto{\pgfpoint{127.352821pt}{64.771484pt}}
\pgflineto{\pgfpoint{125.920013pt}{72.071335pt}}
\pgfusepath{stroke}
\pgfpathmoveto{\pgfpoint{81.280014pt}{44.592613pt}}
\pgflineto{\pgfpoint{85.898277pt}{60.412163pt}}
\pgfusepath{stroke}
\pgfpathmoveto{\pgfpoint{80.181015pt}{65.915634pt}}
\pgflineto{\pgfpoint{81.280014pt}{44.592613pt}}
\pgfusepath{stroke}
\pgfpathmoveto{\pgfpoint{81.280014pt}{61.721870pt}}
\pgflineto{\pgfpoint{80.181015pt}{65.915634pt}}
\pgfusepath{stroke}
\pgfpathmoveto{\pgfpoint{85.898277pt}{60.412163pt}}
\pgflineto{\pgfpoint{81.280014pt}{61.721870pt}}
\pgfusepath{stroke}
\pgfpathmoveto{\pgfpoint{76.320007pt}{80.649338pt}}
\pgflineto{\pgfpoint{80.181015pt}{65.915634pt}}
\pgfusepath{stroke}
\pgfpathmoveto{\pgfpoint{73.884773pt}{93.204117pt}}
\pgflineto{\pgfpoint{76.320007pt}{80.649338pt}}
\pgfusepath{stroke}
\pgfpathmoveto{\pgfpoint{76.320007pt}{140.828415pt}}
\pgflineto{\pgfpoint{73.884773pt}{93.204117pt}}
\pgfusepath{stroke}
\pgfpathmoveto{\pgfpoint{80.181015pt}{65.915634pt}}
\pgflineto{\pgfpoint{76.320007pt}{140.828415pt}}
\pgfusepath{stroke}
\pgfpathmoveto{\pgfpoint{71.360008pt}{43.829178pt}}
\pgflineto{\pgfpoint{73.884773pt}{93.204117pt}}
\pgfusepath{stroke}
\pgfpathmoveto{\pgfpoint{69.454124pt}{89.661858pt}}
\pgflineto{\pgfpoint{71.360008pt}{43.829178pt}}
\pgfusepath{stroke}
\pgfpathmoveto{\pgfpoint{71.360008pt}{106.220398pt}}
\pgflineto{\pgfpoint{69.454124pt}{89.661858pt}}
\pgfusepath{stroke}
\pgfpathmoveto{\pgfpoint{73.884773pt}{93.204117pt}}
\pgflineto{\pgfpoint{71.360008pt}{106.220398pt}}
\pgfusepath{stroke}
\pgfpathmoveto{\pgfpoint{66.400009pt}{63.127323pt}}
\pgflineto{\pgfpoint{69.454124pt}{89.661858pt}}
\pgfusepath{stroke}
\pgfpathmoveto{\pgfpoint{61.440010pt}{46.590111pt}}
\pgflineto{\pgfpoint{66.400009pt}{63.127323pt}}
\pgfusepath{stroke}
\pgfpathmoveto{\pgfpoint{56.480011pt}{42.665169pt}}
\pgflineto{\pgfpoint{61.440010pt}{46.590111pt}}
\pgfusepath{stroke}
\pgfpathmoveto{\pgfpoint{52.729080pt}{55.915833pt}}
\pgflineto{\pgfpoint{56.480011pt}{42.665169pt}}
\pgfusepath{stroke}
\pgfpathmoveto{\pgfpoint{56.480011pt}{59.932564pt}}
\pgflineto{\pgfpoint{52.729080pt}{55.915833pt}}
\pgfusepath{stroke}
\pgfpathmoveto{\pgfpoint{61.440010pt}{105.284721pt}}
\pgflineto{\pgfpoint{56.480011pt}{59.932564pt}}
\pgfusepath{stroke}
\pgfpathmoveto{\pgfpoint{66.400009pt}{163.107224pt}}
\pgflineto{\pgfpoint{61.440010pt}{105.284721pt}}
\pgfusepath{stroke}
\pgfpathmoveto{\pgfpoint{69.454124pt}{89.661858pt}}
\pgflineto{\pgfpoint{66.400009pt}{163.107224pt}}
\pgfusepath{stroke}
\pgfpathmoveto{\pgfpoint{51.520004pt}{54.621078pt}}
\pgflineto{\pgfpoint{52.729080pt}{55.915833pt}}
\pgfusepath{stroke}
\pgfpathmoveto{\pgfpoint{51.272934pt}{60.450993pt}}
\pgflineto{\pgfpoint{51.520004pt}{54.621078pt}}
\pgfusepath{stroke}
\pgfpathmoveto{\pgfpoint{51.520004pt}{60.187065pt}}
\pgflineto{\pgfpoint{51.272934pt}{60.450993pt}}
\pgfusepath{stroke}
\pgfpathmoveto{\pgfpoint{52.729080pt}{55.915833pt}}
\pgflineto{\pgfpoint{51.520004pt}{60.187065pt}}
\pgfusepath{stroke}
\pgfpathmoveto{\pgfpoint{46.560013pt}{65.485382pt}}
\pgflineto{\pgfpoint{51.272934pt}{60.450993pt}}
\pgfusepath{stroke}
\pgfpathmoveto{\pgfpoint{46.560013pt}{171.656006pt}}
\pgflineto{\pgfpoint{46.560013pt}{65.485382pt}}
\pgfusepath{stroke}
\pgfpathmoveto{\pgfpoint{51.272934pt}{60.450993pt}}
\pgflineto{\pgfpoint{46.560013pt}{171.656006pt}}
\pgfusepath{stroke}
\end{pgfpicture}
}
\end{frame}
\only<article>{
The choice of distance in this kind of algorithm is important,
particularly for very high dimensions. For something like a
spectrogram, one idea is look at the total area of the difference
between two spectral lines. 
}
\begin{frame}
  \frametitle{Nearest neighbour: What type is the new bacterium?}
  % Title: glps_renderer figure
% Creator: GL2PS 1.3.8, (C) 1999-2012 C. Geuzaine
% For: Octave
% CreationDate: Fri Jun 16 12:49:21 2017
\begin{pgfpicture}
\pgfsetlinewidth{0.01pt}
\color[rgb]{1.000000,1.000000,1.000000}
\pgfpathmoveto{\pgfpoint{45.000008pt}{222.000000pt}}
\pgflineto{\pgfpoint{289.600037pt}{26.399979pt}}
\pgflineto{\pgfpoint{45.000008pt}{26.399979pt}}
\pgfpathclose
\pgfusepath{fill,stroke}
\pgfpathmoveto{\pgfpoint{45.000008pt}{222.000000pt}}
\pgflineto{\pgfpoint{289.600037pt}{222.000000pt}}
\pgflineto{\pgfpoint{289.600037pt}{26.399979pt}}
\pgfpathclose
\pgfusepath{fill,stroke}
\pgfpathmoveto{\pgfpoint{260.624542pt}{220.474182pt}}
\pgflineto{\pgfpoint{288.074158pt}{197.039612pt}}
\pgflineto{\pgfpoint{260.624542pt}{197.039612pt}}
\pgfpathclose
\pgfusepath{fill,stroke}
\pgfpathmoveto{\pgfpoint{260.624542pt}{220.474182pt}}
\pgflineto{\pgfpoint{288.074158pt}{220.474182pt}}
\pgflineto{\pgfpoint{288.074158pt}{197.039612pt}}
\pgfpathclose
\pgfusepath{fill,stroke}
\color[rgb]{0.000000,0.000000,0.000000}
\pgfsetlinewidth{0.500000pt}
\pgfsetdash{{16pt}{0pt}}{0pt}
\pgfpathmoveto{\pgfpoint{289.600037pt}{26.399979pt}}
\pgflineto{\pgfpoint{45.000008pt}{26.399979pt}}
\pgfusepath{stroke}
\pgfpathmoveto{\pgfpoint{289.600037pt}{222.000000pt}}
\pgflineto{\pgfpoint{45.000008pt}{222.000000pt}}
\pgfusepath{stroke}
\pgfpathmoveto{\pgfpoint{45.000008pt}{222.000000pt}}
\pgflineto{\pgfpoint{45.000008pt}{26.399979pt}}
\pgfusepath{stroke}
\pgfpathmoveto{\pgfpoint{289.600037pt}{222.000000pt}}
\pgflineto{\pgfpoint{289.600037pt}{26.399979pt}}
\pgfusepath{stroke}
\pgfpathmoveto{\pgfpoint{45.000008pt}{28.840996pt}}
\pgflineto{\pgfpoint{45.000008pt}{26.399979pt}}
\pgfusepath{stroke}
\pgfpathmoveto{\pgfpoint{45.000008pt}{219.558990pt}}
\pgflineto{\pgfpoint{45.000008pt}{222.000000pt}}
\pgfusepath{stroke}
\pgfpathmoveto{\pgfpoint{93.920013pt}{28.840996pt}}
\pgflineto{\pgfpoint{93.920013pt}{26.399979pt}}
\pgfusepath{stroke}
\pgfpathmoveto{\pgfpoint{93.920013pt}{219.558990pt}}
\pgflineto{\pgfpoint{93.920013pt}{222.000000pt}}
\pgfusepath{stroke}
\pgfpathmoveto{\pgfpoint{142.840012pt}{28.840996pt}}
\pgflineto{\pgfpoint{142.840012pt}{26.399979pt}}
\pgfusepath{stroke}
\pgfpathmoveto{\pgfpoint{142.840012pt}{219.558990pt}}
\pgflineto{\pgfpoint{142.840012pt}{222.000000pt}}
\pgfusepath{stroke}
\pgfpathmoveto{\pgfpoint{191.760010pt}{28.840996pt}}
\pgflineto{\pgfpoint{191.760010pt}{26.399979pt}}
\pgfusepath{stroke}
\pgfpathmoveto{\pgfpoint{191.760010pt}{219.558990pt}}
\pgflineto{\pgfpoint{191.760010pt}{222.000000pt}}
\pgfusepath{stroke}
\pgfpathmoveto{\pgfpoint{240.680023pt}{28.840996pt}}
\pgflineto{\pgfpoint{240.680023pt}{26.399979pt}}
\pgfusepath{stroke}
\pgfpathmoveto{\pgfpoint{240.680023pt}{219.558990pt}}
\pgflineto{\pgfpoint{240.680023pt}{222.000000pt}}
\pgfusepath{stroke}
\pgfpathmoveto{\pgfpoint{289.600037pt}{28.840996pt}}
\pgflineto{\pgfpoint{289.600037pt}{26.399979pt}}
\pgfusepath{stroke}
\pgfpathmoveto{\pgfpoint{289.600037pt}{219.558990pt}}
\pgflineto{\pgfpoint{289.600037pt}{222.000000pt}}
\pgfusepath{stroke}
{
\pgftransformshift{\pgfpoint{45.000015pt}{21.410187pt}}
\pgfnode{rectangle}{north}{\fontsize{10}{0}\selectfont\textcolor[rgb]{0,0,0}{{-1.5e+08}}}{}{\pgfusepath{discard}}}
{
\pgftransformshift{\pgfpoint{93.920013pt}{21.410187pt}}
\pgfnode{rectangle}{north}{\fontsize{10}{0}\selectfont\textcolor[rgb]{0,0,0}{{-1e+08}}}{}{\pgfusepath{discard}}}
{
\pgftransformshift{\pgfpoint{142.840012pt}{21.410187pt}}
\pgfnode{rectangle}{north}{\fontsize{10}{0}\selectfont\textcolor[rgb]{0,0,0}{{-5e+07}}}{}{\pgfusepath{discard}}}
{
\pgftransformshift{\pgfpoint{191.760010pt}{21.410187pt}}
\pgfnode{rectangle}{north}{\fontsize{10}{0}\selectfont\textcolor[rgb]{0,0,0}{{0}}}{}{\pgfusepath{discard}}}
{
\pgftransformshift{\pgfpoint{240.680008pt}{21.410187pt}}
\pgfnode{rectangle}{north}{\fontsize{10}{0}\selectfont\textcolor[rgb]{0,0,0}{{5e+07}}}{}{\pgfusepath{discard}}}
{
\pgftransformshift{\pgfpoint{289.600006pt}{21.410187pt}}
\pgfnode{rectangle}{north}{\fontsize{10}{0}\selectfont\textcolor[rgb]{0,0,0}{{1e+08}}}{}{\pgfusepath{discard}}}
\pgfpathmoveto{\pgfpoint{47.442024pt}{26.399979pt}}
\pgflineto{\pgfpoint{45.000008pt}{26.399979pt}}
\pgfusepath{stroke}
\pgfpathmoveto{\pgfpoint{287.158020pt}{26.399979pt}}
\pgflineto{\pgfpoint{289.600037pt}{26.399979pt}}
\pgfusepath{stroke}
\pgfpathmoveto{\pgfpoint{47.442024pt}{75.299988pt}}
\pgflineto{\pgfpoint{45.000008pt}{75.299988pt}}
\pgfusepath{stroke}
\pgfpathmoveto{\pgfpoint{287.158020pt}{75.299988pt}}
\pgflineto{\pgfpoint{289.600037pt}{75.299988pt}}
\pgfusepath{stroke}
\pgfpathmoveto{\pgfpoint{47.442024pt}{124.199989pt}}
\pgflineto{\pgfpoint{45.000008pt}{124.199989pt}}
\pgfusepath{stroke}
\pgfpathmoveto{\pgfpoint{287.158020pt}{124.199989pt}}
\pgflineto{\pgfpoint{289.600037pt}{124.199989pt}}
\pgfusepath{stroke}
\pgfpathmoveto{\pgfpoint{47.442024pt}{173.099991pt}}
\pgflineto{\pgfpoint{45.000008pt}{173.099991pt}}
\pgfusepath{stroke}
\pgfpathmoveto{\pgfpoint{287.158020pt}{173.099991pt}}
\pgflineto{\pgfpoint{289.600037pt}{173.099991pt}}
\pgfusepath{stroke}
\pgfpathmoveto{\pgfpoint{47.442024pt}{222.000000pt}}
\pgflineto{\pgfpoint{45.000008pt}{222.000000pt}}
\pgfusepath{stroke}
\pgfpathmoveto{\pgfpoint{287.158020pt}{222.000000pt}}
\pgflineto{\pgfpoint{289.600037pt}{222.000000pt}}
\pgfusepath{stroke}
{
\pgftransformshift{\pgfpoint{40.008171pt}{26.399979pt}}
\pgfnode{rectangle}{east}{\fontsize{10}{0}\selectfont\textcolor[rgb]{0,0,0}{{-2e+08}}}{}{\pgfusepath{discard}}}
{
\pgftransformshift{\pgfpoint{40.008171pt}{75.299988pt}}
\pgfnode{rectangle}{east}{\fontsize{10}{0}\selectfont\textcolor[rgb]{0,0,0}{{-1e+08}}}{}{\pgfusepath{discard}}}
{
\pgftransformshift{\pgfpoint{40.008171pt}{124.199989pt}}
\pgfnode{rectangle}{east}{\fontsize{10}{0}\selectfont\textcolor[rgb]{0,0,0}{{0}}}{}{\pgfusepath{discard}}}
{
\pgftransformshift{\pgfpoint{40.008171pt}{173.099991pt}}
\pgfnode{rectangle}{east}{\fontsize{10}{0}\selectfont\textcolor[rgb]{0,0,0}{{1e+08}}}{}{\pgfusepath{discard}}}
{
\pgftransformshift{\pgfpoint{40.008171pt}{222.000000pt}}
\pgfnode{rectangle}{east}{\fontsize{10}{0}\selectfont\textcolor[rgb]{0,0,0}{{2e+08}}}{}{\pgfusepath{discard}}}
\color[rgb]{0.000000,0.000000,1.000000}
\pgfsetdash{}{0pt}
\pgfpathmoveto{\pgfpoint{196.935547pt}{119.434402pt}}
\pgflineto{\pgfpoint{190.935547pt}{119.434402pt}}
\pgfusepath{stroke}
\pgfpathmoveto{\pgfpoint{193.935547pt}{116.434402pt}}
\pgflineto{\pgfpoint{193.935547pt}{122.434402pt}}
\pgfusepath{stroke}
\pgfpathmoveto{\pgfpoint{199.886261pt}{119.242699pt}}
\pgflineto{\pgfpoint{193.886261pt}{119.242699pt}}
\pgfusepath{stroke}
\pgfpathmoveto{\pgfpoint{196.886261pt}{116.242699pt}}
\pgflineto{\pgfpoint{196.886261pt}{122.242699pt}}
\pgfusepath{stroke}
\pgfpathmoveto{\pgfpoint{183.982819pt}{117.223633pt}}
\pgflineto{\pgfpoint{177.982819pt}{117.223633pt}}
\pgfusepath{stroke}
\pgfpathmoveto{\pgfpoint{180.982819pt}{114.223633pt}}
\pgflineto{\pgfpoint{180.982819pt}{120.223633pt}}
\pgfusepath{stroke}
\pgfpathmoveto{\pgfpoint{179.547073pt}{115.611145pt}}
\pgflineto{\pgfpoint{173.547073pt}{115.611145pt}}
\pgfusepath{stroke}
\pgfpathmoveto{\pgfpoint{176.547073pt}{112.611145pt}}
\pgflineto{\pgfpoint{176.547073pt}{118.611145pt}}
\pgfusepath{stroke}
\pgfpathmoveto{\pgfpoint{190.868942pt}{106.227837pt}}
\pgflineto{\pgfpoint{184.868942pt}{106.227837pt}}
\pgfusepath{stroke}
\pgfpathmoveto{\pgfpoint{187.868942pt}{103.227837pt}}
\pgflineto{\pgfpoint{187.868942pt}{109.227837pt}}
\pgfusepath{stroke}
\pgfpathmoveto{\pgfpoint{201.759155pt}{109.677612pt}}
\pgflineto{\pgfpoint{195.759155pt}{109.677612pt}}
\pgfusepath{stroke}
\pgfpathmoveto{\pgfpoint{198.759155pt}{106.677612pt}}
\pgflineto{\pgfpoint{198.759155pt}{112.677612pt}}
\pgfusepath{stroke}
\pgfpathmoveto{\pgfpoint{183.758728pt}{108.592598pt}}
\pgflineto{\pgfpoint{177.758728pt}{108.592598pt}}
\pgfusepath{stroke}
\pgfpathmoveto{\pgfpoint{180.758728pt}{105.592598pt}}
\pgflineto{\pgfpoint{180.758728pt}{111.592606pt}}
\pgfusepath{stroke}
\pgfpathmoveto{\pgfpoint{112.574402pt}{75.370621pt}}
\pgflineto{\pgfpoint{106.574402pt}{75.370621pt}}
\pgfusepath{stroke}
\pgfpathmoveto{\pgfpoint{109.574402pt}{72.370621pt}}
\pgflineto{\pgfpoint{109.574402pt}{78.370621pt}}
\pgfusepath{stroke}
\pgfpathmoveto{\pgfpoint{174.193832pt}{101.355026pt}}
\pgflineto{\pgfpoint{168.193832pt}{101.355026pt}}
\pgfusepath{stroke}
\pgfpathmoveto{\pgfpoint{171.193832pt}{98.355026pt}}
\pgflineto{\pgfpoint{171.193832pt}{104.355026pt}}
\pgfusepath{stroke}
\pgfpathmoveto{\pgfpoint{182.152756pt}{112.984955pt}}
\pgflineto{\pgfpoint{176.152756pt}{112.984955pt}}
\pgfusepath{stroke}
\pgfpathmoveto{\pgfpoint{179.152756pt}{109.984955pt}}
\pgflineto{\pgfpoint{179.152756pt}{115.984955pt}}
\pgfusepath{stroke}
\pgfpathmoveto{\pgfpoint{168.108780pt}{110.069618pt}}
\pgflineto{\pgfpoint{162.108780pt}{110.069618pt}}
\pgfusepath{stroke}
\pgfpathmoveto{\pgfpoint{165.108780pt}{107.069618pt}}
\pgflineto{\pgfpoint{165.108780pt}{113.069618pt}}
\pgfusepath{stroke}
\pgfpathmoveto{\pgfpoint{201.415054pt}{114.212738pt}}
\pgflineto{\pgfpoint{195.415054pt}{114.212738pt}}
\pgfusepath{stroke}
\pgfpathmoveto{\pgfpoint{198.415054pt}{111.212738pt}}
\pgflineto{\pgfpoint{198.415054pt}{117.212738pt}}
\pgfusepath{stroke}
\pgfpathmoveto{\pgfpoint{192.300415pt}{121.396606pt}}
\pgflineto{\pgfpoint{186.300415pt}{121.396606pt}}
\pgfusepath{stroke}
\pgfpathmoveto{\pgfpoint{189.300415pt}{118.396606pt}}
\pgflineto{\pgfpoint{189.300415pt}{124.396606pt}}
\pgfusepath{stroke}
\pgfpathmoveto{\pgfpoint{193.598267pt}{124.001534pt}}
\pgflineto{\pgfpoint{187.598267pt}{124.001534pt}}
\pgfusepath{stroke}
\pgfpathmoveto{\pgfpoint{190.598267pt}{121.001534pt}}
\pgflineto{\pgfpoint{190.598267pt}{127.001534pt}}
\pgfusepath{stroke}
\pgfpathmoveto{\pgfpoint{180.524246pt}{109.700615pt}}
\pgflineto{\pgfpoint{174.524246pt}{109.700615pt}}
\pgfusepath{stroke}
\pgfpathmoveto{\pgfpoint{177.524246pt}{106.700615pt}}
\pgflineto{\pgfpoint{177.524246pt}{112.700615pt}}
\pgfusepath{stroke}
\pgfpathmoveto{\pgfpoint{176.954651pt}{103.460312pt}}
\pgflineto{\pgfpoint{170.954651pt}{103.460312pt}}
\pgfusepath{stroke}
\pgfpathmoveto{\pgfpoint{173.954651pt}{100.460312pt}}
\pgflineto{\pgfpoint{173.954651pt}{106.460312pt}}
\pgfusepath{stroke}
\pgfpathmoveto{\pgfpoint{185.662079pt}{110.839188pt}}
\pgflineto{\pgfpoint{179.662079pt}{110.839188pt}}
\pgfusepath{stroke}
\pgfpathmoveto{\pgfpoint{182.662079pt}{107.839188pt}}
\pgflineto{\pgfpoint{182.662079pt}{113.839188pt}}
\pgfusepath{stroke}
\pgfpathmoveto{\pgfpoint{148.840988pt}{87.041306pt}}
\pgflineto{\pgfpoint{142.840988pt}{87.041306pt}}
\pgfusepath{stroke}
\pgfpathmoveto{\pgfpoint{145.840988pt}{84.041306pt}}
\pgflineto{\pgfpoint{145.840988pt}{90.041306pt}}
\pgfusepath{stroke}
\pgfpathmoveto{\pgfpoint{149.264221pt}{74.474411pt}}
\pgflineto{\pgfpoint{143.264221pt}{74.474411pt}}
\pgfusepath{stroke}
\pgfpathmoveto{\pgfpoint{146.264221pt}{71.474411pt}}
\pgflineto{\pgfpoint{146.264221pt}{77.474411pt}}
\pgfusepath{stroke}
\pgfpathmoveto{\pgfpoint{208.496521pt}{104.231644pt}}
\pgflineto{\pgfpoint{202.496521pt}{104.231644pt}}
\pgfusepath{stroke}
\pgfpathmoveto{\pgfpoint{205.496521pt}{101.231644pt}}
\pgflineto{\pgfpoint{205.496521pt}{107.231644pt}}
\pgfusepath{stroke}
\pgfpathmoveto{\pgfpoint{171.510223pt}{118.414749pt}}
\pgflineto{\pgfpoint{165.510223pt}{118.414749pt}}
\pgfusepath{stroke}
\pgfpathmoveto{\pgfpoint{168.510223pt}{115.414749pt}}
\pgflineto{\pgfpoint{168.510223pt}{121.414749pt}}
\pgfusepath{stroke}
\pgfpathmoveto{\pgfpoint{179.501907pt}{118.791878pt}}
\pgflineto{\pgfpoint{173.501907pt}{118.791878pt}}
\pgfusepath{stroke}
\pgfpathmoveto{\pgfpoint{176.501907pt}{115.791878pt}}
\pgflineto{\pgfpoint{176.501907pt}{121.791878pt}}
\pgfusepath{stroke}
\pgfpathmoveto{\pgfpoint{184.015793pt}{119.659134pt}}
\pgflineto{\pgfpoint{178.015793pt}{119.659134pt}}
\pgfusepath{stroke}
\pgfpathmoveto{\pgfpoint{181.015793pt}{116.659134pt}}
\pgflineto{\pgfpoint{181.015793pt}{122.659142pt}}
\pgfusepath{stroke}
\pgfpathmoveto{\pgfpoint{170.695862pt}{117.468170pt}}
\pgflineto{\pgfpoint{164.695862pt}{117.468170pt}}
\pgfusepath{stroke}
\pgfpathmoveto{\pgfpoint{167.695862pt}{114.468163pt}}
\pgflineto{\pgfpoint{167.695862pt}{120.468170pt}}
\pgfusepath{stroke}
\pgfpathmoveto{\pgfpoint{196.609299pt}{118.015953pt}}
\pgflineto{\pgfpoint{190.609299pt}{118.015953pt}}
\pgfusepath{stroke}
\pgfpathmoveto{\pgfpoint{193.609299pt}{115.015953pt}}
\pgflineto{\pgfpoint{193.609299pt}{121.015953pt}}
\pgfusepath{stroke}
\pgfpathmoveto{\pgfpoint{197.507370pt}{115.068123pt}}
\pgflineto{\pgfpoint{191.507370pt}{115.068123pt}}
\pgfusepath{stroke}
\pgfpathmoveto{\pgfpoint{194.507370pt}{112.068123pt}}
\pgflineto{\pgfpoint{194.507370pt}{118.068123pt}}
\pgfusepath{stroke}
\pgfpathmoveto{\pgfpoint{193.406509pt}{114.448425pt}}
\pgflineto{\pgfpoint{187.406509pt}{114.448425pt}}
\pgfusepath{stroke}
\pgfpathmoveto{\pgfpoint{190.406509pt}{111.448425pt}}
\pgflineto{\pgfpoint{190.406509pt}{117.448425pt}}
\pgfusepath{stroke}
\pgfpathmoveto{\pgfpoint{194.718613pt}{114.180740pt}}
\pgflineto{\pgfpoint{188.718613pt}{114.180740pt}}
\pgfusepath{stroke}
\pgfpathmoveto{\pgfpoint{191.718613pt}{111.180733pt}}
\pgflineto{\pgfpoint{191.718613pt}{117.180740pt}}
\pgfusepath{stroke}
\pgfpathmoveto{\pgfpoint{202.550079pt}{120.974258pt}}
\pgflineto{\pgfpoint{196.550079pt}{120.974258pt}}
\pgfusepath{stroke}
\pgfpathmoveto{\pgfpoint{199.550079pt}{117.974258pt}}
\pgflineto{\pgfpoint{199.550079pt}{123.974258pt}}
\pgfusepath{stroke}
\pgfpathmoveto{\pgfpoint{194.683685pt}{124.179276pt}}
\pgflineto{\pgfpoint{188.683685pt}{124.179276pt}}
\pgfusepath{stroke}
\pgfpathmoveto{\pgfpoint{191.683685pt}{121.179276pt}}
\pgflineto{\pgfpoint{191.683685pt}{127.179276pt}}
\pgfusepath{stroke}
\pgfpathmoveto{\pgfpoint{205.879364pt}{112.132126pt}}
\pgflineto{\pgfpoint{199.879364pt}{112.132126pt}}
\pgfusepath{stroke}
\pgfpathmoveto{\pgfpoint{202.879364pt}{109.132126pt}}
\pgflineto{\pgfpoint{202.879364pt}{115.132133pt}}
\pgfusepath{stroke}
\pgfpathmoveto{\pgfpoint{206.979767pt}{90.041245pt}}
\pgflineto{\pgfpoint{200.979767pt}{90.041245pt}}
\pgfusepath{stroke}
\pgfpathmoveto{\pgfpoint{203.979767pt}{87.041245pt}}
\pgflineto{\pgfpoint{203.979767pt}{93.041245pt}}
\pgfusepath{stroke}
\pgfpathmoveto{\pgfpoint{196.589050pt}{123.302505pt}}
\pgflineto{\pgfpoint{190.589050pt}{123.302505pt}}
\pgfusepath{stroke}
\pgfpathmoveto{\pgfpoint{193.589050pt}{120.302505pt}}
\pgflineto{\pgfpoint{193.589050pt}{126.302505pt}}
\pgfusepath{stroke}
\pgfpathmoveto{\pgfpoint{209.073334pt}{97.586403pt}}
\pgflineto{\pgfpoint{203.073318pt}{97.586403pt}}
\pgfusepath{stroke}
\pgfpathmoveto{\pgfpoint{206.073318pt}{94.586403pt}}
\pgflineto{\pgfpoint{206.073318pt}{100.586403pt}}
\pgfusepath{stroke}
\pgfpathmoveto{\pgfpoint{195.115631pt}{118.254936pt}}
\pgflineto{\pgfpoint{189.115631pt}{118.254936pt}}
\pgfusepath{stroke}
\pgfpathmoveto{\pgfpoint{192.115631pt}{115.254936pt}}
\pgflineto{\pgfpoint{192.115631pt}{121.254936pt}}
\pgfusepath{stroke}
\pgfpathmoveto{\pgfpoint{201.029053pt}{98.080841pt}}
\pgflineto{\pgfpoint{195.029053pt}{98.080841pt}}
\pgfusepath{stroke}
\pgfpathmoveto{\pgfpoint{198.029053pt}{95.080841pt}}
\pgflineto{\pgfpoint{198.029053pt}{101.080841pt}}
\pgfusepath{stroke}
\pgfpathmoveto{\pgfpoint{197.477219pt}{106.478531pt}}
\pgflineto{\pgfpoint{191.477219pt}{106.478531pt}}
\pgfusepath{stroke}
\pgfpathmoveto{\pgfpoint{194.477219pt}{103.478531pt}}
\pgflineto{\pgfpoint{194.477219pt}{109.478531pt}}
\pgfusepath{stroke}
\pgfpathmoveto{\pgfpoint{148.010254pt}{79.851120pt}}
\pgflineto{\pgfpoint{142.010254pt}{79.851120pt}}
\pgfusepath{stroke}
\pgfpathmoveto{\pgfpoint{145.010254pt}{76.851120pt}}
\pgflineto{\pgfpoint{145.010254pt}{82.851120pt}}
\pgfusepath{stroke}
\pgfpathmoveto{\pgfpoint{175.602463pt}{102.839371pt}}
\pgflineto{\pgfpoint{169.602463pt}{102.839371pt}}
\pgfusepath{stroke}
\pgfpathmoveto{\pgfpoint{172.602463pt}{99.839371pt}}
\pgflineto{\pgfpoint{172.602463pt}{105.839371pt}}
\pgfusepath{stroke}
\pgfpathmoveto{\pgfpoint{168.293350pt}{102.053452pt}}
\pgflineto{\pgfpoint{162.293350pt}{102.053452pt}}
\pgfusepath{stroke}
\pgfpathmoveto{\pgfpoint{165.293350pt}{99.053444pt}}
\pgflineto{\pgfpoint{165.293350pt}{105.053452pt}}
\pgfusepath{stroke}
\pgfpathmoveto{\pgfpoint{188.499710pt}{114.862000pt}}
\pgflineto{\pgfpoint{182.499710pt}{114.862000pt}}
\pgfusepath{stroke}
\pgfpathmoveto{\pgfpoint{185.499710pt}{111.862000pt}}
\pgflineto{\pgfpoint{185.499710pt}{117.862007pt}}
\pgfusepath{stroke}
\pgfpathmoveto{\pgfpoint{174.017639pt}{119.228752pt}}
\pgflineto{\pgfpoint{168.017639pt}{119.228752pt}}
\pgfusepath{stroke}
\pgfpathmoveto{\pgfpoint{171.017639pt}{116.228752pt}}
\pgflineto{\pgfpoint{171.017639pt}{122.228752pt}}
\pgfusepath{stroke}
\pgfpathmoveto{\pgfpoint{166.606094pt}{107.932907pt}}
\pgflineto{\pgfpoint{160.606094pt}{107.932907pt}}
\pgfusepath{stroke}
\pgfpathmoveto{\pgfpoint{163.606094pt}{104.932907pt}}
\pgflineto{\pgfpoint{163.606094pt}{110.932907pt}}
\pgfusepath{stroke}
\pgfpathmoveto{\pgfpoint{178.223022pt}{102.634354pt}}
\pgflineto{\pgfpoint{172.223022pt}{102.634354pt}}
\pgfusepath{stroke}
\pgfpathmoveto{\pgfpoint{175.223022pt}{99.634346pt}}
\pgflineto{\pgfpoint{175.223022pt}{105.634354pt}}
\pgfusepath{stroke}
\pgfpathmoveto{\pgfpoint{214.268692pt}{114.793839pt}}
\pgflineto{\pgfpoint{208.268692pt}{114.793839pt}}
\pgfusepath{stroke}
\pgfpathmoveto{\pgfpoint{211.268692pt}{111.793839pt}}
\pgflineto{\pgfpoint{211.268692pt}{117.793839pt}}
\pgfusepath{stroke}
\pgfpathmoveto{\pgfpoint{222.330811pt}{111.535378pt}}
\pgflineto{\pgfpoint{216.330811pt}{111.535378pt}}
\pgfusepath{stroke}
\pgfpathmoveto{\pgfpoint{219.330811pt}{108.535378pt}}
\pgflineto{\pgfpoint{219.330811pt}{114.535378pt}}
\pgfusepath{stroke}
\pgfpathmoveto{\pgfpoint{212.852081pt}{120.428650pt}}
\pgflineto{\pgfpoint{206.852081pt}{120.428650pt}}
\pgfusepath{stroke}
\pgfpathmoveto{\pgfpoint{209.852081pt}{117.428650pt}}
\pgflineto{\pgfpoint{209.852081pt}{123.428650pt}}
\pgfusepath{stroke}
\pgfpathmoveto{\pgfpoint{201.270599pt}{119.242432pt}}
\pgflineto{\pgfpoint{195.270584pt}{119.242432pt}}
\pgfusepath{stroke}
\pgfpathmoveto{\pgfpoint{198.270584pt}{116.242432pt}}
\pgflineto{\pgfpoint{198.270584pt}{122.242432pt}}
\pgfusepath{stroke}
\pgfpathmoveto{\pgfpoint{170.743149pt}{91.735825pt}}
\pgflineto{\pgfpoint{164.743149pt}{91.735825pt}}
\pgfusepath{stroke}
\pgfpathmoveto{\pgfpoint{167.743149pt}{88.735825pt}}
\pgflineto{\pgfpoint{167.743149pt}{94.735825pt}}
\pgfusepath{stroke}
\pgfpathmoveto{\pgfpoint{208.540649pt}{102.282532pt}}
\pgflineto{\pgfpoint{202.540634pt}{102.282532pt}}
\pgfusepath{stroke}
\pgfpathmoveto{\pgfpoint{205.540634pt}{99.282532pt}}
\pgflineto{\pgfpoint{205.540634pt}{105.282532pt}}
\pgfusepath{stroke}
\pgfpathmoveto{\pgfpoint{199.218811pt}{68.839760pt}}
\pgflineto{\pgfpoint{193.218811pt}{68.839760pt}}
\pgfusepath{stroke}
\pgfpathmoveto{\pgfpoint{196.218811pt}{65.839760pt}}
\pgflineto{\pgfpoint{196.218811pt}{71.839760pt}}
\pgfusepath{stroke}
\pgfpathmoveto{\pgfpoint{187.196762pt}{116.726044pt}}
\pgflineto{\pgfpoint{181.196762pt}{116.726044pt}}
\pgfusepath{stroke}
\pgfpathmoveto{\pgfpoint{184.196762pt}{113.726044pt}}
\pgflineto{\pgfpoint{184.196762pt}{119.726044pt}}
\pgfusepath{stroke}
\pgfpathmoveto{\pgfpoint{147.139359pt}{71.754990pt}}
\pgflineto{\pgfpoint{141.139359pt}{71.754990pt}}
\pgfusepath{stroke}
\pgfpathmoveto{\pgfpoint{144.139359pt}{68.754990pt}}
\pgflineto{\pgfpoint{144.139359pt}{74.754990pt}}
\pgfusepath{stroke}
\pgfpathmoveto{\pgfpoint{215.177032pt}{136.581879pt}}
\pgflineto{\pgfpoint{209.177032pt}{136.581879pt}}
\pgfusepath{stroke}
\pgfpathmoveto{\pgfpoint{212.177032pt}{133.581879pt}}
\pgflineto{\pgfpoint{212.177032pt}{139.581879pt}}
\pgfusepath{stroke}
\pgfpathmoveto{\pgfpoint{184.478500pt}{114.593369pt}}
\pgflineto{\pgfpoint{178.478500pt}{114.593369pt}}
\pgfusepath{stroke}
\pgfpathmoveto{\pgfpoint{181.478500pt}{111.593369pt}}
\pgflineto{\pgfpoint{181.478500pt}{117.593369pt}}
\pgfusepath{stroke}
\pgfpathmoveto{\pgfpoint{190.108887pt}{122.127441pt}}
\pgflineto{\pgfpoint{184.108887pt}{122.127441pt}}
\pgfusepath{stroke}
\pgfpathmoveto{\pgfpoint{187.108887pt}{119.127441pt}}
\pgflineto{\pgfpoint{187.108887pt}{125.127441pt}}
\pgfusepath{stroke}
\pgfpathmoveto{\pgfpoint{179.439209pt}{116.649216pt}}
\pgflineto{\pgfpoint{173.439209pt}{116.649216pt}}
\pgfusepath{stroke}
\pgfpathmoveto{\pgfpoint{176.439209pt}{113.649216pt}}
\pgflineto{\pgfpoint{176.439209pt}{119.649216pt}}
\pgfusepath{stroke}
\pgfpathmoveto{\pgfpoint{104.465897pt}{107.626564pt}}
\pgflineto{\pgfpoint{98.465889pt}{107.626564pt}}
\pgfusepath{stroke}
\pgfpathmoveto{\pgfpoint{101.465897pt}{104.626564pt}}
\pgflineto{\pgfpoint{101.465897pt}{110.626564pt}}
\pgfusepath{stroke}
\pgfpathmoveto{\pgfpoint{177.118210pt}{119.713600pt}}
\pgflineto{\pgfpoint{171.118210pt}{119.713600pt}}
\pgfusepath{stroke}
\pgfpathmoveto{\pgfpoint{174.118210pt}{116.713600pt}}
\pgflineto{\pgfpoint{174.118210pt}{122.713600pt}}
\pgfusepath{stroke}
\pgfpathmoveto{\pgfpoint{164.368271pt}{113.844521pt}}
\pgflineto{\pgfpoint{158.368271pt}{113.844521pt}}
\pgfusepath{stroke}
\pgfpathmoveto{\pgfpoint{161.368271pt}{110.844521pt}}
\pgflineto{\pgfpoint{161.368271pt}{116.844521pt}}
\pgfusepath{stroke}
\pgfpathmoveto{\pgfpoint{174.302505pt}{114.449341pt}}
\pgflineto{\pgfpoint{168.302505pt}{114.449341pt}}
\pgfusepath{stroke}
\pgfpathmoveto{\pgfpoint{171.302505pt}{111.449341pt}}
\pgflineto{\pgfpoint{171.302505pt}{117.449341pt}}
\pgfusepath{stroke}
\pgfpathmoveto{\pgfpoint{188.850342pt}{120.117348pt}}
\pgflineto{\pgfpoint{182.850342pt}{120.117348pt}}
\pgfusepath{stroke}
\pgfpathmoveto{\pgfpoint{185.850342pt}{117.117348pt}}
\pgflineto{\pgfpoint{185.850342pt}{123.117348pt}}
\pgfusepath{stroke}
\pgfpathmoveto{\pgfpoint{189.115555pt}{120.904907pt}}
\pgflineto{\pgfpoint{183.115555pt}{120.904907pt}}
\pgfusepath{stroke}
\pgfpathmoveto{\pgfpoint{186.115555pt}{117.904907pt}}
\pgflineto{\pgfpoint{186.115555pt}{123.904907pt}}
\pgfusepath{stroke}
\pgfpathmoveto{\pgfpoint{164.011871pt}{105.423592pt}}
\pgflineto{\pgfpoint{158.011871pt}{105.423592pt}}
\pgfusepath{stroke}
\pgfpathmoveto{\pgfpoint{161.011871pt}{102.423584pt}}
\pgflineto{\pgfpoint{161.011871pt}{108.423592pt}}
\pgfusepath{stroke}
\pgfpathmoveto{\pgfpoint{123.379929pt}{92.500244pt}}
\pgflineto{\pgfpoint{117.379929pt}{92.500244pt}}
\pgfusepath{stroke}
\pgfpathmoveto{\pgfpoint{120.379936pt}{89.500244pt}}
\pgflineto{\pgfpoint{120.379936pt}{95.500244pt}}
\pgfusepath{stroke}
\pgfpathmoveto{\pgfpoint{224.852325pt}{104.413704pt}}
\pgflineto{\pgfpoint{218.852325pt}{104.413704pt}}
\pgfusepath{stroke}
\pgfpathmoveto{\pgfpoint{221.852325pt}{101.413704pt}}
\pgflineto{\pgfpoint{221.852325pt}{107.413704pt}}
\pgfusepath{stroke}
\pgfpathmoveto{\pgfpoint{158.911072pt}{77.804001pt}}
\pgflineto{\pgfpoint{152.911072pt}{77.804001pt}}
\pgfusepath{stroke}
\pgfpathmoveto{\pgfpoint{155.911072pt}{74.804001pt}}
\pgflineto{\pgfpoint{155.911072pt}{80.804001pt}}
\pgfusepath{stroke}
\pgfpathmoveto{\pgfpoint{200.542542pt}{112.245041pt}}
\pgflineto{\pgfpoint{194.542542pt}{112.245041pt}}
\pgfusepath{stroke}
\pgfpathmoveto{\pgfpoint{197.542542pt}{109.245041pt}}
\pgflineto{\pgfpoint{197.542542pt}{115.245041pt}}
\pgfusepath{stroke}
\pgfpathmoveto{\pgfpoint{202.097809pt}{102.256493pt}}
\pgflineto{\pgfpoint{196.097809pt}{102.256493pt}}
\pgfusepath{stroke}
\pgfpathmoveto{\pgfpoint{199.097809pt}{99.256493pt}}
\pgflineto{\pgfpoint{199.097809pt}{105.256493pt}}
\pgfusepath{stroke}
\pgfpathmoveto{\pgfpoint{209.810730pt}{94.916466pt}}
\pgflineto{\pgfpoint{203.810730pt}{94.916466pt}}
\pgfusepath{stroke}
\pgfpathmoveto{\pgfpoint{206.810730pt}{91.916466pt}}
\pgflineto{\pgfpoint{206.810730pt}{97.916466pt}}
\pgfusepath{stroke}
\pgfpathmoveto{\pgfpoint{188.358261pt}{110.721474pt}}
\pgflineto{\pgfpoint{182.358261pt}{110.721474pt}}
\pgfusepath{stroke}
\pgfpathmoveto{\pgfpoint{185.358261pt}{107.721474pt}}
\pgflineto{\pgfpoint{185.358261pt}{113.721474pt}}
\pgfusepath{stroke}
\pgfpathmoveto{\pgfpoint{202.170609pt}{103.766815pt}}
\pgflineto{\pgfpoint{196.170609pt}{103.766815pt}}
\pgfusepath{stroke}
\pgfpathmoveto{\pgfpoint{199.170609pt}{100.766815pt}}
\pgflineto{\pgfpoint{199.170609pt}{106.766815pt}}
\pgfusepath{stroke}
\pgfpathmoveto{\pgfpoint{193.927582pt}{120.353355pt}}
\pgflineto{\pgfpoint{187.927582pt}{120.353355pt}}
\pgfusepath{stroke}
\pgfpathmoveto{\pgfpoint{190.927582pt}{117.353355pt}}
\pgflineto{\pgfpoint{190.927582pt}{123.353355pt}}
\pgfusepath{stroke}
\pgfpathmoveto{\pgfpoint{214.340698pt}{107.736900pt}}
\pgflineto{\pgfpoint{208.340698pt}{107.736900pt}}
\pgfusepath{stroke}
\pgfpathmoveto{\pgfpoint{211.340698pt}{104.736893pt}}
\pgflineto{\pgfpoint{211.340698pt}{110.736900pt}}
\pgfusepath{stroke}
\pgfpathmoveto{\pgfpoint{192.854523pt}{116.522400pt}}
\pgflineto{\pgfpoint{186.854523pt}{116.522400pt}}
\pgfusepath{stroke}
\pgfpathmoveto{\pgfpoint{189.854523pt}{113.522400pt}}
\pgflineto{\pgfpoint{189.854523pt}{119.522400pt}}
\pgfusepath{stroke}
\pgfpathmoveto{\pgfpoint{180.888123pt}{119.808556pt}}
\pgflineto{\pgfpoint{174.888123pt}{119.808556pt}}
\pgfusepath{stroke}
\pgfpathmoveto{\pgfpoint{177.888123pt}{116.808556pt}}
\pgflineto{\pgfpoint{177.888123pt}{122.808556pt}}
\pgfusepath{stroke}
\pgfpathmoveto{\pgfpoint{176.316513pt}{120.338478pt}}
\pgflineto{\pgfpoint{170.316513pt}{120.338478pt}}
\pgfusepath{stroke}
\pgfpathmoveto{\pgfpoint{173.316513pt}{117.338478pt}}
\pgflineto{\pgfpoint{173.316513pt}{123.338478pt}}
\pgfusepath{stroke}
\pgfpathmoveto{\pgfpoint{183.214355pt}{122.327942pt}}
\pgflineto{\pgfpoint{177.214355pt}{122.327942pt}}
\pgfusepath{stroke}
\pgfpathmoveto{\pgfpoint{180.214355pt}{119.327942pt}}
\pgflineto{\pgfpoint{180.214355pt}{125.327942pt}}
\pgfusepath{stroke}
\pgfpathmoveto{\pgfpoint{191.410797pt}{122.499756pt}}
\pgflineto{\pgfpoint{185.410797pt}{122.499756pt}}
\pgfusepath{stroke}
\pgfpathmoveto{\pgfpoint{188.410797pt}{119.499756pt}}
\pgflineto{\pgfpoint{188.410797pt}{125.499756pt}}
\pgfusepath{stroke}
\pgfpathmoveto{\pgfpoint{197.348083pt}{116.337822pt}}
\pgflineto{\pgfpoint{191.348083pt}{116.337822pt}}
\pgfusepath{stroke}
\pgfpathmoveto{\pgfpoint{194.348083pt}{113.337822pt}}
\pgflineto{\pgfpoint{194.348083pt}{119.337822pt}}
\pgfusepath{stroke}
\pgfpathmoveto{\pgfpoint{150.088013pt}{108.014786pt}}
\pgflineto{\pgfpoint{144.088013pt}{108.014786pt}}
\pgfusepath{stroke}
\pgfpathmoveto{\pgfpoint{147.088013pt}{105.014786pt}}
\pgflineto{\pgfpoint{147.088013pt}{111.014786pt}}
\pgfusepath{stroke}
\pgfpathmoveto{\pgfpoint{165.265259pt}{92.156952pt}}
\pgflineto{\pgfpoint{159.265259pt}{92.156952pt}}
\pgfusepath{stroke}
\pgfpathmoveto{\pgfpoint{162.265259pt}{89.156952pt}}
\pgflineto{\pgfpoint{162.265259pt}{95.156952pt}}
\pgfusepath{stroke}
\pgfpathmoveto{\pgfpoint{164.344528pt}{89.086136pt}}
\pgflineto{\pgfpoint{158.344528pt}{89.086136pt}}
\pgfusepath{stroke}
\pgfpathmoveto{\pgfpoint{161.344528pt}{86.086136pt}}
\pgflineto{\pgfpoint{161.344528pt}{92.086136pt}}
\pgfusepath{stroke}
\pgfpathmoveto{\pgfpoint{144.495544pt}{104.760262pt}}
\pgflineto{\pgfpoint{138.495544pt}{104.760262pt}}
\pgfusepath{stroke}
\pgfpathmoveto{\pgfpoint{141.495544pt}{101.760262pt}}
\pgflineto{\pgfpoint{141.495544pt}{107.760262pt}}
\pgfusepath{stroke}
\pgfpathmoveto{\pgfpoint{152.836807pt}{126.138474pt}}
\pgflineto{\pgfpoint{146.836807pt}{126.138474pt}}
\pgfusepath{stroke}
\pgfpathmoveto{\pgfpoint{149.836807pt}{123.138474pt}}
\pgflineto{\pgfpoint{149.836807pt}{129.138474pt}}
\pgfusepath{stroke}
\pgfpathmoveto{\pgfpoint{200.675446pt}{115.899910pt}}
\pgflineto{\pgfpoint{194.675446pt}{115.899910pt}}
\pgfusepath{stroke}
\pgfpathmoveto{\pgfpoint{197.675446pt}{112.899910pt}}
\pgflineto{\pgfpoint{197.675446pt}{118.899910pt}}
\pgfusepath{stroke}
\pgfpathmoveto{\pgfpoint{190.927689pt}{96.198410pt}}
\pgflineto{\pgfpoint{184.927689pt}{96.198410pt}}
\pgfusepath{stroke}
\pgfpathmoveto{\pgfpoint{187.927689pt}{93.198410pt}}
\pgflineto{\pgfpoint{187.927689pt}{99.198410pt}}
\pgfusepath{stroke}
\pgfpathmoveto{\pgfpoint{203.814697pt}{103.487259pt}}
\pgflineto{\pgfpoint{197.814697pt}{103.487259pt}}
\pgfusepath{stroke}
\pgfpathmoveto{\pgfpoint{200.814697pt}{100.487259pt}}
\pgflineto{\pgfpoint{200.814697pt}{106.487267pt}}
\pgfusepath{stroke}
\pgfpathmoveto{\pgfpoint{200.864105pt}{116.337234pt}}
\pgflineto{\pgfpoint{194.864105pt}{116.337234pt}}
\pgfusepath{stroke}
\pgfpathmoveto{\pgfpoint{197.864105pt}{113.337234pt}}
\pgflineto{\pgfpoint{197.864105pt}{119.337234pt}}
\pgfusepath{stroke}
\pgfpathmoveto{\pgfpoint{195.630478pt}{117.957169pt}}
\pgflineto{\pgfpoint{189.630478pt}{117.957169pt}}
\pgfusepath{stroke}
\pgfpathmoveto{\pgfpoint{192.630478pt}{114.957169pt}}
\pgflineto{\pgfpoint{192.630478pt}{120.957176pt}}
\pgfusepath{stroke}
\pgfpathmoveto{\pgfpoint{164.992645pt}{119.094711pt}}
\pgflineto{\pgfpoint{158.992645pt}{119.094711pt}}
\pgfusepath{stroke}
\pgfpathmoveto{\pgfpoint{161.992645pt}{116.094711pt}}
\pgflineto{\pgfpoint{161.992645pt}{122.094719pt}}
\pgfusepath{stroke}
\pgfpathmoveto{\pgfpoint{184.135727pt}{114.890152pt}}
\pgflineto{\pgfpoint{178.135727pt}{114.890152pt}}
\pgfusepath{stroke}
\pgfpathmoveto{\pgfpoint{181.135727pt}{111.890152pt}}
\pgflineto{\pgfpoint{181.135727pt}{117.890160pt}}
\pgfusepath{stroke}
\pgfpathmoveto{\pgfpoint{183.212982pt}{107.354736pt}}
\pgflineto{\pgfpoint{177.212982pt}{107.354736pt}}
\pgfusepath{stroke}
\pgfpathmoveto{\pgfpoint{180.212982pt}{104.354736pt}}
\pgflineto{\pgfpoint{180.212982pt}{110.354736pt}}
\pgfusepath{stroke}
\pgfpathmoveto{\pgfpoint{205.889694pt}{104.733597pt}}
\pgflineto{\pgfpoint{199.889694pt}{104.733597pt}}
\pgfusepath{stroke}
\pgfpathmoveto{\pgfpoint{202.889694pt}{101.733597pt}}
\pgflineto{\pgfpoint{202.889694pt}{107.733597pt}}
\pgfusepath{stroke}
\pgfpathmoveto{\pgfpoint{67.810738pt}{29.410484pt}}
\pgflineto{\pgfpoint{61.810730pt}{29.410484pt}}
\pgfusepath{stroke}
\pgfpathmoveto{\pgfpoint{64.810730pt}{26.410492pt}}
\pgflineto{\pgfpoint{64.810730pt}{32.410484pt}}
\pgfusepath{stroke}
\pgfpathmoveto{\pgfpoint{138.959503pt}{82.218102pt}}
\pgflineto{\pgfpoint{132.959503pt}{82.218102pt}}
\pgfusepath{stroke}
\pgfpathmoveto{\pgfpoint{135.959503pt}{79.218102pt}}
\pgflineto{\pgfpoint{135.959503pt}{85.218109pt}}
\pgfusepath{stroke}
\pgfpathmoveto{\pgfpoint{195.715820pt}{105.990662pt}}
\pgflineto{\pgfpoint{189.715820pt}{105.990662pt}}
\pgfusepath{stroke}
\pgfpathmoveto{\pgfpoint{192.715820pt}{102.990662pt}}
\pgflineto{\pgfpoint{192.715820pt}{108.990662pt}}
\pgfusepath{stroke}
\pgfpathmoveto{\pgfpoint{210.786179pt}{108.818954pt}}
\pgflineto{\pgfpoint{204.786179pt}{108.818954pt}}
\pgfusepath{stroke}
\pgfpathmoveto{\pgfpoint{207.786179pt}{105.818947pt}}
\pgflineto{\pgfpoint{207.786179pt}{111.818954pt}}
\pgfusepath{stroke}
\pgfpathmoveto{\pgfpoint{204.065964pt}{84.708786pt}}
\pgflineto{\pgfpoint{198.065964pt}{84.708786pt}}
\pgfusepath{stroke}
\pgfpathmoveto{\pgfpoint{201.065964pt}{81.708786pt}}
\pgflineto{\pgfpoint{201.065964pt}{87.708786pt}}
\pgfusepath{stroke}
\pgfpathmoveto{\pgfpoint{194.247559pt}{84.369720pt}}
\pgflineto{\pgfpoint{188.247559pt}{84.369720pt}}
\pgfusepath{stroke}
\pgfpathmoveto{\pgfpoint{191.247559pt}{81.369720pt}}
\pgflineto{\pgfpoint{191.247559pt}{87.369720pt}}
\pgfusepath{stroke}
\pgfpathmoveto{\pgfpoint{235.681610pt}{74.712463pt}}
\pgflineto{\pgfpoint{229.681595pt}{74.712463pt}}
\pgfusepath{stroke}
\pgfpathmoveto{\pgfpoint{232.681595pt}{71.712463pt}}
\pgflineto{\pgfpoint{232.681595pt}{77.712463pt}}
\pgfusepath{stroke}
\pgfpathmoveto{\pgfpoint{208.190735pt}{79.811440pt}}
\pgflineto{\pgfpoint{202.190735pt}{79.811440pt}}
\pgfusepath{stroke}
\pgfpathmoveto{\pgfpoint{205.190735pt}{76.811440pt}}
\pgflineto{\pgfpoint{205.190735pt}{82.811440pt}}
\pgfusepath{stroke}
\pgfpathmoveto{\pgfpoint{208.186905pt}{79.900948pt}}
\pgflineto{\pgfpoint{202.186890pt}{79.900948pt}}
\pgfusepath{stroke}
\pgfpathmoveto{\pgfpoint{205.186905pt}{76.900948pt}}
\pgflineto{\pgfpoint{205.186905pt}{82.900948pt}}
\pgfusepath{stroke}
\pgfpathmoveto{\pgfpoint{194.421799pt}{101.387360pt}}
\pgflineto{\pgfpoint{188.421799pt}{101.387360pt}}
\pgfusepath{stroke}
\pgfpathmoveto{\pgfpoint{191.421799pt}{98.387360pt}}
\pgflineto{\pgfpoint{191.421799pt}{104.387360pt}}
\pgfusepath{stroke}
\pgfpathmoveto{\pgfpoint{185.854950pt}{114.317383pt}}
\pgflineto{\pgfpoint{179.854950pt}{114.317383pt}}
\pgfusepath{stroke}
\pgfpathmoveto{\pgfpoint{182.854950pt}{111.317375pt}}
\pgflineto{\pgfpoint{182.854950pt}{117.317383pt}}
\pgfusepath{stroke}
\pgfpathmoveto{\pgfpoint{169.837845pt}{103.355423pt}}
\pgflineto{\pgfpoint{163.837845pt}{103.355423pt}}
\pgfusepath{stroke}
\pgfpathmoveto{\pgfpoint{166.837845pt}{100.355423pt}}
\pgflineto{\pgfpoint{166.837845pt}{106.355431pt}}
\pgfusepath{stroke}
\pgfpathmoveto{\pgfpoint{177.191116pt}{103.091187pt}}
\pgflineto{\pgfpoint{171.191116pt}{103.091187pt}}
\pgfusepath{stroke}
\pgfpathmoveto{\pgfpoint{174.191116pt}{100.091187pt}}
\pgflineto{\pgfpoint{174.191116pt}{106.091187pt}}
\pgfusepath{stroke}
\pgfpathmoveto{\pgfpoint{158.701706pt}{103.395920pt}}
\pgflineto{\pgfpoint{152.701706pt}{103.395920pt}}
\pgfusepath{stroke}
\pgfpathmoveto{\pgfpoint{155.701706pt}{100.395920pt}}
\pgflineto{\pgfpoint{155.701706pt}{106.395927pt}}
\pgfusepath{stroke}
\pgfpathmoveto{\pgfpoint{158.003036pt}{119.001534pt}}
\pgflineto{\pgfpoint{152.003036pt}{119.001534pt}}
\pgfusepath{stroke}
\pgfpathmoveto{\pgfpoint{155.003036pt}{116.001534pt}}
\pgflineto{\pgfpoint{155.003036pt}{122.001534pt}}
\pgfusepath{stroke}
\pgfpathmoveto{\pgfpoint{198.435318pt}{121.251968pt}}
\pgflineto{\pgfpoint{192.435318pt}{121.251968pt}}
\pgfusepath{stroke}
\pgfpathmoveto{\pgfpoint{195.435318pt}{118.251968pt}}
\pgflineto{\pgfpoint{195.435318pt}{124.251968pt}}
\pgfusepath{stroke}
\pgfpathmoveto{\pgfpoint{200.801666pt}{114.776733pt}}
\pgflineto{\pgfpoint{194.801666pt}{114.776733pt}}
\pgfusepath{stroke}
\pgfpathmoveto{\pgfpoint{197.801666pt}{111.776733pt}}
\pgflineto{\pgfpoint{197.801666pt}{117.776733pt}}
\pgfusepath{stroke}
\pgfpathmoveto{\pgfpoint{190.116608pt}{115.944687pt}}
\pgflineto{\pgfpoint{184.116608pt}{115.944687pt}}
\pgfusepath{stroke}
\pgfpathmoveto{\pgfpoint{187.116608pt}{112.944687pt}}
\pgflineto{\pgfpoint{187.116608pt}{118.944687pt}}
\pgfusepath{stroke}
\pgfpathmoveto{\pgfpoint{186.450836pt}{117.508553pt}}
\pgflineto{\pgfpoint{180.450836pt}{117.508553pt}}
\pgfusepath{stroke}
\pgfpathmoveto{\pgfpoint{183.450836pt}{114.508553pt}}
\pgflineto{\pgfpoint{183.450836pt}{120.508553pt}}
\pgfusepath{stroke}
\pgfpathmoveto{\pgfpoint{182.945129pt}{94.769684pt}}
\pgflineto{\pgfpoint{176.945129pt}{94.769684pt}}
\pgfusepath{stroke}
\pgfpathmoveto{\pgfpoint{179.945129pt}{91.769684pt}}
\pgflineto{\pgfpoint{179.945129pt}{97.769684pt}}
\pgfusepath{stroke}
\pgfpathmoveto{\pgfpoint{217.117340pt}{103.659065pt}}
\pgflineto{\pgfpoint{211.117340pt}{103.659065pt}}
\pgfusepath{stroke}
\pgfpathmoveto{\pgfpoint{214.117340pt}{100.659065pt}}
\pgflineto{\pgfpoint{214.117340pt}{106.659065pt}}
\pgfusepath{stroke}
\pgfpathmoveto{\pgfpoint{238.411362pt}{84.948723pt}}
\pgflineto{\pgfpoint{232.411346pt}{84.948723pt}}
\pgfusepath{stroke}
\pgfpathmoveto{\pgfpoint{235.411362pt}{81.948715pt}}
\pgflineto{\pgfpoint{235.411362pt}{87.948723pt}}
\pgfusepath{stroke}
\pgfpathmoveto{\pgfpoint{179.200348pt}{89.153976pt}}
\pgflineto{\pgfpoint{173.200348pt}{89.153976pt}}
\pgfusepath{stroke}
\pgfpathmoveto{\pgfpoint{176.200348pt}{86.153976pt}}
\pgflineto{\pgfpoint{176.200348pt}{92.153984pt}}
\pgfusepath{stroke}
\pgfpathmoveto{\pgfpoint{235.074829pt}{94.344696pt}}
\pgflineto{\pgfpoint{229.074829pt}{94.344696pt}}
\pgfusepath{stroke}
\pgfpathmoveto{\pgfpoint{232.074829pt}{91.344696pt}}
\pgflineto{\pgfpoint{232.074829pt}{97.344696pt}}
\pgfusepath{stroke}
\pgfpathmoveto{\pgfpoint{237.887268pt}{67.848312pt}}
\pgflineto{\pgfpoint{231.887253pt}{67.848312pt}}
\pgfusepath{stroke}
\pgfpathmoveto{\pgfpoint{234.887253pt}{64.848312pt}}
\pgflineto{\pgfpoint{234.887253pt}{70.848312pt}}
\pgfusepath{stroke}
\pgfpathmoveto{\pgfpoint{181.916992pt}{116.796455pt}}
\pgflineto{\pgfpoint{175.916992pt}{116.796455pt}}
\pgfusepath{stroke}
\pgfpathmoveto{\pgfpoint{178.916992pt}{113.796455pt}}
\pgflineto{\pgfpoint{178.916992pt}{119.796455pt}}
\pgfusepath{stroke}
\pgfpathmoveto{\pgfpoint{158.279373pt}{108.222610pt}}
\pgflineto{\pgfpoint{152.279373pt}{108.222610pt}}
\pgfusepath{stroke}
\pgfpathmoveto{\pgfpoint{155.279373pt}{105.222610pt}}
\pgflineto{\pgfpoint{155.279373pt}{111.222610pt}}
\pgfusepath{stroke}
\pgfpathmoveto{\pgfpoint{258.962585pt}{110.360901pt}}
\pgflineto{\pgfpoint{252.962585pt}{110.360901pt}}
\pgfusepath{stroke}
\pgfpathmoveto{\pgfpoint{255.962585pt}{107.360901pt}}
\pgflineto{\pgfpoint{255.962585pt}{113.360901pt}}
\pgfusepath{stroke}
\pgfpathmoveto{\pgfpoint{173.935364pt}{104.089211pt}}
\pgflineto{\pgfpoint{167.935364pt}{104.089211pt}}
\pgfusepath{stroke}
\pgfpathmoveto{\pgfpoint{170.935364pt}{101.089203pt}}
\pgflineto{\pgfpoint{170.935364pt}{107.089211pt}}
\pgfusepath{stroke}
\pgfpathmoveto{\pgfpoint{155.325623pt}{92.199280pt}}
\pgflineto{\pgfpoint{149.325623pt}{92.199280pt}}
\pgfusepath{stroke}
\pgfpathmoveto{\pgfpoint{152.325623pt}{89.199280pt}}
\pgflineto{\pgfpoint{152.325623pt}{95.199280pt}}
\pgfusepath{stroke}
\pgfpathmoveto{\pgfpoint{202.332703pt}{106.899551pt}}
\pgflineto{\pgfpoint{196.332703pt}{106.899551pt}}
\pgfusepath{stroke}
\pgfpathmoveto{\pgfpoint{199.332703pt}{103.899551pt}}
\pgflineto{\pgfpoint{199.332703pt}{109.899551pt}}
\pgfusepath{stroke}
\pgfpathmoveto{\pgfpoint{188.291809pt}{106.586739pt}}
\pgflineto{\pgfpoint{182.291809pt}{106.586739pt}}
\pgfusepath{stroke}
\pgfpathmoveto{\pgfpoint{185.291809pt}{103.586739pt}}
\pgflineto{\pgfpoint{185.291809pt}{109.586739pt}}
\pgfusepath{stroke}
\pgfpathmoveto{\pgfpoint{170.988876pt}{117.481316pt}}
\pgflineto{\pgfpoint{164.988876pt}{117.481316pt}}
\pgfusepath{stroke}
\pgfpathmoveto{\pgfpoint{167.988876pt}{114.481316pt}}
\pgflineto{\pgfpoint{167.988876pt}{120.481316pt}}
\pgfusepath{stroke}
\pgfpathmoveto{\pgfpoint{165.308563pt}{111.498383pt}}
\pgflineto{\pgfpoint{159.308563pt}{111.498383pt}}
\pgfusepath{stroke}
\pgfpathmoveto{\pgfpoint{162.308563pt}{108.498383pt}}
\pgflineto{\pgfpoint{162.308563pt}{114.498383pt}}
\pgfusepath{stroke}
\pgfpathmoveto{\pgfpoint{196.641235pt}{106.520065pt}}
\pgflineto{\pgfpoint{190.641235pt}{106.520065pt}}
\pgfusepath{stroke}
\pgfpathmoveto{\pgfpoint{193.641235pt}{103.520065pt}}
\pgflineto{\pgfpoint{193.641235pt}{109.520065pt}}
\pgfusepath{stroke}
\pgfpathmoveto{\pgfpoint{198.887161pt}{112.126419pt}}
\pgflineto{\pgfpoint{192.887161pt}{112.126419pt}}
\pgfusepath{stroke}
\pgfpathmoveto{\pgfpoint{195.887161pt}{109.126419pt}}
\pgflineto{\pgfpoint{195.887161pt}{115.126419pt}}
\pgfusepath{stroke}
\pgfpathmoveto{\pgfpoint{190.139114pt}{118.018402pt}}
\pgflineto{\pgfpoint{184.139114pt}{118.018402pt}}
\pgfusepath{stroke}
\pgfpathmoveto{\pgfpoint{187.139114pt}{115.018394pt}}
\pgflineto{\pgfpoint{187.139114pt}{121.018402pt}}
\pgfusepath{stroke}
\pgfpathmoveto{\pgfpoint{180.520355pt}{106.133606pt}}
\pgflineto{\pgfpoint{174.520355pt}{106.133606pt}}
\pgfusepath{stroke}
\pgfpathmoveto{\pgfpoint{177.520355pt}{103.133606pt}}
\pgflineto{\pgfpoint{177.520355pt}{109.133606pt}}
\pgfusepath{stroke}
\pgfpathmoveto{\pgfpoint{166.703690pt}{93.097610pt}}
\pgflineto{\pgfpoint{160.703690pt}{93.097610pt}}
\pgfusepath{stroke}
\pgfpathmoveto{\pgfpoint{163.703690pt}{90.097610pt}}
\pgflineto{\pgfpoint{163.703690pt}{96.097618pt}}
\pgfusepath{stroke}
\pgfpathmoveto{\pgfpoint{172.526749pt}{104.289711pt}}
\pgflineto{\pgfpoint{166.526749pt}{104.289711pt}}
\pgfusepath{stroke}
\pgfpathmoveto{\pgfpoint{169.526749pt}{101.289703pt}}
\pgflineto{\pgfpoint{169.526749pt}{107.289711pt}}
\pgfusepath{stroke}
\pgfpathmoveto{\pgfpoint{187.477814pt}{117.596985pt}}
\pgflineto{\pgfpoint{181.477814pt}{117.596985pt}}
\pgfusepath{stroke}
\pgfpathmoveto{\pgfpoint{184.477814pt}{114.596985pt}}
\pgflineto{\pgfpoint{184.477814pt}{120.596985pt}}
\pgfusepath{stroke}
\pgfpathmoveto{\pgfpoint{181.620728pt}{109.305641pt}}
\pgflineto{\pgfpoint{175.620728pt}{109.305641pt}}
\pgfusepath{stroke}
\pgfpathmoveto{\pgfpoint{178.620728pt}{106.305641pt}}
\pgflineto{\pgfpoint{178.620728pt}{112.305641pt}}
\pgfusepath{stroke}
\pgfpathmoveto{\pgfpoint{200.743759pt}{116.577286pt}}
\pgflineto{\pgfpoint{194.743759pt}{116.577286pt}}
\pgfusepath{stroke}
\pgfpathmoveto{\pgfpoint{197.743759pt}{113.577286pt}}
\pgflineto{\pgfpoint{197.743759pt}{119.577286pt}}
\pgfusepath{stroke}
\pgfpathmoveto{\pgfpoint{171.920624pt}{108.920357pt}}
\pgflineto{\pgfpoint{165.920624pt}{108.920357pt}}
\pgfusepath{stroke}
\pgfpathmoveto{\pgfpoint{168.920624pt}{105.920357pt}}
\pgflineto{\pgfpoint{168.920624pt}{111.920364pt}}
\pgfusepath{stroke}
\pgfpathmoveto{\pgfpoint{135.909760pt}{73.728447pt}}
\pgflineto{\pgfpoint{129.909760pt}{73.728447pt}}
\pgfusepath{stroke}
\pgfpathmoveto{\pgfpoint{132.909760pt}{70.728447pt}}
\pgflineto{\pgfpoint{132.909760pt}{76.728447pt}}
\pgfusepath{stroke}
\pgfpathmoveto{\pgfpoint{120.742088pt}{87.429214pt}}
\pgflineto{\pgfpoint{114.742088pt}{87.429214pt}}
\pgfusepath{stroke}
\pgfpathmoveto{\pgfpoint{117.742088pt}{84.429214pt}}
\pgflineto{\pgfpoint{117.742088pt}{90.429214pt}}
\pgfusepath{stroke}
\pgfpathmoveto{\pgfpoint{184.925278pt}{115.256699pt}}
\pgflineto{\pgfpoint{178.925278pt}{115.256699pt}}
\pgfusepath{stroke}
\pgfpathmoveto{\pgfpoint{181.925278pt}{112.256699pt}}
\pgflineto{\pgfpoint{181.925278pt}{118.256699pt}}
\pgfusepath{stroke}
\color[rgb]{0.000000,0.500000,0.000000}
\pgfpathmoveto{\pgfpoint{184.936493pt}{137.227509pt}}
\pgflineto{\pgfpoint{178.936493pt}{143.227509pt}}
\pgfusepath{stroke}
\pgfpathmoveto{\pgfpoint{184.936493pt}{143.227509pt}}
\pgflineto{\pgfpoint{178.936493pt}{137.227509pt}}
\pgfusepath{stroke}
\pgfpathmoveto{\pgfpoint{117.511063pt}{203.316452pt}}
\pgflineto{\pgfpoint{111.511063pt}{209.316437pt}}
\pgfusepath{stroke}
\pgfpathmoveto{\pgfpoint{117.511063pt}{209.316437pt}}
\pgflineto{\pgfpoint{111.511063pt}{203.316452pt}}
\pgfusepath{stroke}
\pgfpathmoveto{\pgfpoint{188.206314pt}{125.640877pt}}
\pgflineto{\pgfpoint{182.206314pt}{131.640869pt}}
\pgfusepath{stroke}
\pgfpathmoveto{\pgfpoint{188.206314pt}{131.640869pt}}
\pgflineto{\pgfpoint{182.206314pt}{125.640877pt}}
\pgfusepath{stroke}
\pgfpathmoveto{\pgfpoint{111.670929pt}{135.618835pt}}
\pgflineto{\pgfpoint{105.670929pt}{141.618851pt}}
\pgfusepath{stroke}
\pgfpathmoveto{\pgfpoint{111.670929pt}{141.618851pt}}
\pgflineto{\pgfpoint{105.670929pt}{135.618835pt}}
\pgfusepath{stroke}
\pgfpathmoveto{\pgfpoint{133.251724pt}{139.476685pt}}
\pgflineto{\pgfpoint{127.251724pt}{145.476685pt}}
\pgfusepath{stroke}
\pgfpathmoveto{\pgfpoint{133.251724pt}{145.476685pt}}
\pgflineto{\pgfpoint{127.251724pt}{139.476685pt}}
\pgfusepath{stroke}
\pgfpathmoveto{\pgfpoint{66.771400pt}{212.555969pt}}
\pgflineto{\pgfpoint{60.771400pt}{218.555969pt}}
\pgfusepath{stroke}
\pgfpathmoveto{\pgfpoint{66.771400pt}{218.555969pt}}
\pgflineto{\pgfpoint{60.771400pt}{212.555969pt}}
\pgfusepath{stroke}
\pgfpathmoveto{\pgfpoint{185.577332pt}{146.044998pt}}
\pgflineto{\pgfpoint{179.577332pt}{152.044998pt}}
\pgfusepath{stroke}
\pgfpathmoveto{\pgfpoint{185.577332pt}{152.044998pt}}
\pgflineto{\pgfpoint{179.577332pt}{146.044998pt}}
\pgfusepath{stroke}
\pgfpathmoveto{\pgfpoint{163.424515pt}{129.375381pt}}
\pgflineto{\pgfpoint{157.424515pt}{135.375381pt}}
\pgfusepath{stroke}
\pgfpathmoveto{\pgfpoint{163.424515pt}{135.375381pt}}
\pgflineto{\pgfpoint{157.424515pt}{129.375381pt}}
\pgfusepath{stroke}
\pgfpathmoveto{\pgfpoint{164.779160pt}{174.718796pt}}
\pgflineto{\pgfpoint{158.779160pt}{180.718796pt}}
\pgfusepath{stroke}
\pgfpathmoveto{\pgfpoint{164.779160pt}{180.718796pt}}
\pgflineto{\pgfpoint{158.779160pt}{174.718796pt}}
\pgfusepath{stroke}
\pgfpathmoveto{\pgfpoint{131.375275pt}{150.253448pt}}
\pgflineto{\pgfpoint{125.375275pt}{156.253448pt}}
\pgfusepath{stroke}
\pgfpathmoveto{\pgfpoint{131.375275pt}{156.253448pt}}
\pgflineto{\pgfpoint{125.375275pt}{150.253448pt}}
\pgfusepath{stroke}
\pgfpathmoveto{\pgfpoint{64.678047pt}{165.434067pt}}
\pgflineto{\pgfpoint{58.678047pt}{171.434067pt}}
\pgfusepath{stroke}
\pgfpathmoveto{\pgfpoint{64.678047pt}{171.434067pt}}
\pgflineto{\pgfpoint{58.678047pt}{165.434067pt}}
\pgfusepath{stroke}
\pgfpathmoveto{\pgfpoint{168.128555pt}{128.687103pt}}
\pgflineto{\pgfpoint{162.128555pt}{134.687103pt}}
\pgfusepath{stroke}
\pgfpathmoveto{\pgfpoint{168.128555pt}{134.687103pt}}
\pgflineto{\pgfpoint{162.128555pt}{128.687103pt}}
\pgfusepath{stroke}
\pgfpathmoveto{\pgfpoint{205.642883pt}{147.756851pt}}
\pgflineto{\pgfpoint{199.642883pt}{153.756851pt}}
\pgfusepath{stroke}
\pgfpathmoveto{\pgfpoint{205.642883pt}{153.756851pt}}
\pgflineto{\pgfpoint{199.642883pt}{147.756851pt}}
\pgfusepath{stroke}
\pgfpathmoveto{\pgfpoint{156.901917pt}{135.317474pt}}
\pgflineto{\pgfpoint{150.901917pt}{141.317474pt}}
\pgfusepath{stroke}
\pgfpathmoveto{\pgfpoint{156.901917pt}{141.317474pt}}
\pgflineto{\pgfpoint{150.901917pt}{135.317474pt}}
\pgfusepath{stroke}
\pgfpathmoveto{\pgfpoint{178.512466pt}{125.103767pt}}
\pgflineto{\pgfpoint{172.512466pt}{131.103775pt}}
\pgfusepath{stroke}
\pgfpathmoveto{\pgfpoint{178.512466pt}{131.103775pt}}
\pgflineto{\pgfpoint{172.512466pt}{125.103767pt}}
\pgfusepath{stroke}
\pgfpathmoveto{\pgfpoint{182.324051pt}{109.428757pt}}
\pgflineto{\pgfpoint{176.324051pt}{115.428764pt}}
\pgfusepath{stroke}
\pgfpathmoveto{\pgfpoint{182.324051pt}{115.428764pt}}
\pgflineto{\pgfpoint{176.324051pt}{109.428757pt}}
\pgfusepath{stroke}
\pgfpathmoveto{\pgfpoint{183.976471pt}{163.479218pt}}
\pgflineto{\pgfpoint{177.976471pt}{169.479218pt}}
\pgfusepath{stroke}
\pgfpathmoveto{\pgfpoint{183.976471pt}{169.479218pt}}
\pgflineto{\pgfpoint{177.976471pt}{163.479218pt}}
\pgfusepath{stroke}
\pgfpathmoveto{\pgfpoint{186.273865pt}{123.065292pt}}
\pgflineto{\pgfpoint{180.273865pt}{129.065292pt}}
\pgfusepath{stroke}
\pgfpathmoveto{\pgfpoint{186.273865pt}{129.065292pt}}
\pgflineto{\pgfpoint{180.273865pt}{123.065292pt}}
\pgfusepath{stroke}
\pgfpathmoveto{\pgfpoint{158.661362pt}{148.370621pt}}
\pgflineto{\pgfpoint{152.661362pt}{154.370621pt}}
\pgfusepath{stroke}
\pgfpathmoveto{\pgfpoint{158.661362pt}{154.370621pt}}
\pgflineto{\pgfpoint{152.661362pt}{148.370621pt}}
\pgfusepath{stroke}
\pgfpathmoveto{\pgfpoint{137.910004pt}{126.041534pt}}
\pgflineto{\pgfpoint{131.910004pt}{132.041534pt}}
\pgfusepath{stroke}
\pgfpathmoveto{\pgfpoint{137.910004pt}{132.041534pt}}
\pgflineto{\pgfpoint{131.910004pt}{126.041534pt}}
\pgfusepath{stroke}
\pgfpathmoveto{\pgfpoint{175.946411pt}{127.328346pt}}
\pgflineto{\pgfpoint{169.946411pt}{133.328354pt}}
\pgfusepath{stroke}
\pgfpathmoveto{\pgfpoint{175.946411pt}{133.328354pt}}
\pgflineto{\pgfpoint{169.946411pt}{127.328346pt}}
\pgfusepath{stroke}
\pgfpathmoveto{\pgfpoint{172.102951pt}{131.214188pt}}
\pgflineto{\pgfpoint{166.102951pt}{137.214188pt}}
\pgfusepath{stroke}
\pgfpathmoveto{\pgfpoint{172.102951pt}{137.214188pt}}
\pgflineto{\pgfpoint{166.102951pt}{131.214188pt}}
\pgfusepath{stroke}
\pgfpathmoveto{\pgfpoint{154.296158pt}{144.527710pt}}
\pgflineto{\pgfpoint{148.296158pt}{150.527710pt}}
\pgfusepath{stroke}
\pgfpathmoveto{\pgfpoint{154.296158pt}{150.527710pt}}
\pgflineto{\pgfpoint{148.296158pt}{144.527710pt}}
\pgfusepath{stroke}
\pgfpathmoveto{\pgfpoint{159.834061pt}{132.893829pt}}
\pgflineto{\pgfpoint{153.834061pt}{138.893829pt}}
\pgfusepath{stroke}
\pgfpathmoveto{\pgfpoint{159.834061pt}{138.893829pt}}
\pgflineto{\pgfpoint{153.834061pt}{132.893829pt}}
\pgfusepath{stroke}
\pgfpathmoveto{\pgfpoint{177.900040pt}{134.565033pt}}
\pgflineto{\pgfpoint{171.900040pt}{140.565033pt}}
\pgfusepath{stroke}
\pgfpathmoveto{\pgfpoint{177.900040pt}{140.565033pt}}
\pgflineto{\pgfpoint{171.900040pt}{134.565033pt}}
\pgfusepath{stroke}
\pgfpathmoveto{\pgfpoint{158.810394pt}{146.112701pt}}
\pgflineto{\pgfpoint{152.810394pt}{152.112701pt}}
\pgfusepath{stroke}
\pgfpathmoveto{\pgfpoint{158.810394pt}{152.112701pt}}
\pgflineto{\pgfpoint{152.810394pt}{146.112701pt}}
\pgfusepath{stroke}
\pgfpathmoveto{\pgfpoint{165.751907pt}{148.133469pt}}
\pgflineto{\pgfpoint{159.751907pt}{154.133469pt}}
\pgfusepath{stroke}
\pgfpathmoveto{\pgfpoint{165.751907pt}{154.133469pt}}
\pgflineto{\pgfpoint{159.751907pt}{148.133469pt}}
\pgfusepath{stroke}
\pgfpathmoveto{\pgfpoint{188.722488pt}{128.864990pt}}
\pgflineto{\pgfpoint{182.722488pt}{134.864990pt}}
\pgfusepath{stroke}
\pgfpathmoveto{\pgfpoint{188.722488pt}{134.864990pt}}
\pgflineto{\pgfpoint{182.722488pt}{128.864990pt}}
\pgfusepath{stroke}
\pgfpathmoveto{\pgfpoint{163.783356pt}{136.138657pt}}
\pgflineto{\pgfpoint{157.783356pt}{142.138657pt}}
\pgfusepath{stroke}
\pgfpathmoveto{\pgfpoint{163.783356pt}{142.138657pt}}
\pgflineto{\pgfpoint{157.783356pt}{136.138657pt}}
\pgfusepath{stroke}
\pgfpathmoveto{\pgfpoint{156.204987pt}{139.583450pt}}
\pgflineto{\pgfpoint{150.204987pt}{145.583450pt}}
\pgfusepath{stroke}
\pgfpathmoveto{\pgfpoint{156.204987pt}{145.583450pt}}
\pgflineto{\pgfpoint{150.204987pt}{139.583450pt}}
\pgfusepath{stroke}
\pgfpathmoveto{\pgfpoint{108.090050pt}{167.233994pt}}
\pgflineto{\pgfpoint{102.090042pt}{173.233994pt}}
\pgfusepath{stroke}
\pgfpathmoveto{\pgfpoint{108.090050pt}{173.233994pt}}
\pgflineto{\pgfpoint{102.090042pt}{167.233994pt}}
\pgfusepath{stroke}
\pgfpathmoveto{\pgfpoint{103.367920pt}{194.184723pt}}
\pgflineto{\pgfpoint{97.367920pt}{200.184723pt}}
\pgfusepath{stroke}
\pgfpathmoveto{\pgfpoint{103.367920pt}{200.184723pt}}
\pgflineto{\pgfpoint{97.367920pt}{194.184723pt}}
\pgfusepath{stroke}
\pgfpathmoveto{\pgfpoint{158.083984pt}{140.284790pt}}
\pgflineto{\pgfpoint{152.083984pt}{146.284790pt}}
\pgfusepath{stroke}
\pgfpathmoveto{\pgfpoint{158.083984pt}{146.284790pt}}
\pgflineto{\pgfpoint{152.083984pt}{140.284790pt}}
\pgfusepath{stroke}
\pgfpathmoveto{\pgfpoint{165.972244pt}{160.728363pt}}
\pgflineto{\pgfpoint{159.972244pt}{166.728363pt}}
\pgfusepath{stroke}
\pgfpathmoveto{\pgfpoint{165.972244pt}{166.728363pt}}
\pgflineto{\pgfpoint{159.972244pt}{160.728363pt}}
\pgfusepath{stroke}
\pgfpathmoveto{\pgfpoint{173.461487pt}{128.505020pt}}
\pgflineto{\pgfpoint{167.461487pt}{134.505020pt}}
\pgfusepath{stroke}
\pgfpathmoveto{\pgfpoint{173.461487pt}{134.505020pt}}
\pgflineto{\pgfpoint{167.461487pt}{128.505020pt}}
\pgfusepath{stroke}
\pgfpathmoveto{\pgfpoint{169.788940pt}{130.577332pt}}
\pgflineto{\pgfpoint{163.788940pt}{136.577332pt}}
\pgfusepath{stroke}
\pgfpathmoveto{\pgfpoint{169.788940pt}{136.577332pt}}
\pgflineto{\pgfpoint{163.788940pt}{130.577332pt}}
\pgfusepath{stroke}
\pgfpathmoveto{\pgfpoint{182.993423pt}{124.435440pt}}
\pgflineto{\pgfpoint{176.993423pt}{130.435440pt}}
\pgfusepath{stroke}
\pgfpathmoveto{\pgfpoint{182.993423pt}{130.435440pt}}
\pgflineto{\pgfpoint{176.993423pt}{124.435440pt}}
\pgfusepath{stroke}
\pgfpathmoveto{\pgfpoint{162.180969pt}{139.215729pt}}
\pgflineto{\pgfpoint{156.180969pt}{145.215729pt}}
\pgfusepath{stroke}
\pgfpathmoveto{\pgfpoint{162.180969pt}{145.215729pt}}
\pgflineto{\pgfpoint{156.180969pt}{139.215729pt}}
\pgfusepath{stroke}
\pgfpathmoveto{\pgfpoint{170.219406pt}{131.559265pt}}
\pgflineto{\pgfpoint{164.219406pt}{137.559265pt}}
\pgfusepath{stroke}
\pgfpathmoveto{\pgfpoint{170.219406pt}{137.559265pt}}
\pgflineto{\pgfpoint{164.219406pt}{131.559265pt}}
\pgfusepath{stroke}
\pgfpathmoveto{\pgfpoint{167.697754pt}{131.977951pt}}
\pgflineto{\pgfpoint{161.697754pt}{137.977951pt}}
\pgfusepath{stroke}
\pgfpathmoveto{\pgfpoint{167.697754pt}{137.977951pt}}
\pgflineto{\pgfpoint{161.697754pt}{131.977951pt}}
\pgfusepath{stroke}
\pgfpathmoveto{\pgfpoint{131.257980pt}{152.243408pt}}
\pgflineto{\pgfpoint{125.257980pt}{158.243423pt}}
\pgfusepath{stroke}
\pgfpathmoveto{\pgfpoint{131.257980pt}{158.243423pt}}
\pgflineto{\pgfpoint{125.257980pt}{152.243408pt}}
\pgfusepath{stroke}
\pgfpathmoveto{\pgfpoint{184.625519pt}{119.561653pt}}
\pgflineto{\pgfpoint{178.625519pt}{125.561653pt}}
\pgfusepath{stroke}
\pgfpathmoveto{\pgfpoint{184.625519pt}{125.561653pt}}
\pgflineto{\pgfpoint{178.625519pt}{119.561653pt}}
\pgfusepath{stroke}
\pgfpathmoveto{\pgfpoint{177.656387pt}{108.470901pt}}
\pgflineto{\pgfpoint{171.656387pt}{114.470901pt}}
\pgfusepath{stroke}
\pgfpathmoveto{\pgfpoint{177.656387pt}{114.470901pt}}
\pgflineto{\pgfpoint{171.656387pt}{108.470901pt}}
\pgfusepath{stroke}
\pgfpathmoveto{\pgfpoint{167.816437pt}{119.507477pt}}
\pgflineto{\pgfpoint{161.816437pt}{125.507477pt}}
\pgfusepath{stroke}
\pgfpathmoveto{\pgfpoint{167.816437pt}{125.507477pt}}
\pgflineto{\pgfpoint{161.816437pt}{119.507477pt}}
\pgfusepath{stroke}
\pgfpathmoveto{\pgfpoint{165.924042pt}{123.079765pt}}
\pgflineto{\pgfpoint{159.924042pt}{129.079773pt}}
\pgfusepath{stroke}
\pgfpathmoveto{\pgfpoint{165.924042pt}{129.079773pt}}
\pgflineto{\pgfpoint{159.924042pt}{123.079765pt}}
\pgfusepath{stroke}
\pgfpathmoveto{\pgfpoint{182.292389pt}{126.585098pt}}
\pgflineto{\pgfpoint{176.292389pt}{132.585098pt}}
\pgfusepath{stroke}
\pgfpathmoveto{\pgfpoint{182.292389pt}{132.585098pt}}
\pgflineto{\pgfpoint{176.292389pt}{126.585098pt}}
\pgfusepath{stroke}
\pgfpathmoveto{\pgfpoint{97.353935pt}{165.778976pt}}
\pgflineto{\pgfpoint{91.353928pt}{171.778976pt}}
\pgfusepath{stroke}
\pgfpathmoveto{\pgfpoint{97.353935pt}{171.778976pt}}
\pgflineto{\pgfpoint{91.353928pt}{165.778976pt}}
\pgfusepath{stroke}
\pgfpathmoveto{\pgfpoint{148.885895pt}{141.106918pt}}
\pgflineto{\pgfpoint{142.885895pt}{147.106918pt}}
\pgfusepath{stroke}
\pgfpathmoveto{\pgfpoint{148.885895pt}{147.106918pt}}
\pgflineto{\pgfpoint{142.885895pt}{141.106918pt}}
\pgfusepath{stroke}
\pgfpathmoveto{\pgfpoint{118.961487pt}{144.851547pt}}
\pgflineto{\pgfpoint{112.961487pt}{150.851547pt}}
\pgfusepath{stroke}
\pgfpathmoveto{\pgfpoint{118.961487pt}{150.851547pt}}
\pgflineto{\pgfpoint{112.961487pt}{144.851547pt}}
\pgfusepath{stroke}
\pgfpathmoveto{\pgfpoint{177.716965pt}{129.267120pt}}
\pgflineto{\pgfpoint{171.716965pt}{135.267120pt}}
\pgfusepath{stroke}
\pgfpathmoveto{\pgfpoint{177.716965pt}{135.267120pt}}
\pgflineto{\pgfpoint{171.716965pt}{129.267120pt}}
\pgfusepath{stroke}
\pgfpathmoveto{\pgfpoint{170.739990pt}{134.313614pt}}
\pgflineto{\pgfpoint{164.739990pt}{140.313614pt}}
\pgfusepath{stroke}
\pgfpathmoveto{\pgfpoint{170.739990pt}{140.313614pt}}
\pgflineto{\pgfpoint{164.739990pt}{134.313614pt}}
\pgfusepath{stroke}
\pgfpathmoveto{\pgfpoint{175.963943pt}{158.481049pt}}
\pgflineto{\pgfpoint{169.963943pt}{164.481049pt}}
\pgfusepath{stroke}
\pgfpathmoveto{\pgfpoint{175.963943pt}{164.481049pt}}
\pgflineto{\pgfpoint{169.963943pt}{158.481049pt}}
\pgfusepath{stroke}
\pgfpathmoveto{\pgfpoint{147.413361pt}{133.222565pt}}
\pgflineto{\pgfpoint{141.413361pt}{139.222565pt}}
\pgfusepath{stroke}
\pgfpathmoveto{\pgfpoint{147.413361pt}{139.222565pt}}
\pgflineto{\pgfpoint{141.413361pt}{133.222565pt}}
\pgfusepath{stroke}
\pgfpathmoveto{\pgfpoint{79.925964pt}{149.746185pt}}
\pgflineto{\pgfpoint{73.925964pt}{155.746185pt}}
\pgfusepath{stroke}
\pgfpathmoveto{\pgfpoint{79.925964pt}{155.746185pt}}
\pgflineto{\pgfpoint{73.925964pt}{149.746185pt}}
\pgfusepath{stroke}
\pgfpathmoveto{\pgfpoint{73.745819pt}{168.351334pt}}
\pgflineto{\pgfpoint{67.745819pt}{174.351334pt}}
\pgfusepath{stroke}
\pgfpathmoveto{\pgfpoint{73.745819pt}{174.351334pt}}
\pgflineto{\pgfpoint{67.745819pt}{168.351334pt}}
\pgfusepath{stroke}
\pgfpathmoveto{\pgfpoint{172.578934pt}{127.022873pt}}
\pgflineto{\pgfpoint{166.578934pt}{133.022873pt}}
\pgfusepath{stroke}
\pgfpathmoveto{\pgfpoint{172.578934pt}{133.022873pt}}
\pgflineto{\pgfpoint{166.578934pt}{127.022873pt}}
\pgfusepath{stroke}
\pgfpathmoveto{\pgfpoint{165.846512pt}{137.737061pt}}
\pgflineto{\pgfpoint{159.846512pt}{143.737061pt}}
\pgfusepath{stroke}
\pgfpathmoveto{\pgfpoint{165.846512pt}{143.737061pt}}
\pgflineto{\pgfpoint{159.846512pt}{137.737061pt}}
\pgfusepath{stroke}
\pgfpathmoveto{\pgfpoint{150.903778pt}{157.222382pt}}
\pgflineto{\pgfpoint{144.903778pt}{163.222382pt}}
\pgfusepath{stroke}
\pgfpathmoveto{\pgfpoint{150.903778pt}{163.222382pt}}
\pgflineto{\pgfpoint{144.903778pt}{157.222382pt}}
\pgfusepath{stroke}
\pgfpathmoveto{\pgfpoint{111.097565pt}{150.775970pt}}
\pgflineto{\pgfpoint{105.097565pt}{156.775970pt}}
\pgfusepath{stroke}
\pgfpathmoveto{\pgfpoint{111.097565pt}{156.775970pt}}
\pgflineto{\pgfpoint{105.097565pt}{150.775970pt}}
\pgfusepath{stroke}
\pgfpathmoveto{\pgfpoint{141.830536pt}{141.786835pt}}
\pgflineto{\pgfpoint{135.830536pt}{147.786835pt}}
\pgfusepath{stroke}
\pgfpathmoveto{\pgfpoint{141.830536pt}{147.786835pt}}
\pgflineto{\pgfpoint{135.830536pt}{141.786835pt}}
\pgfusepath{stroke}
\color[rgb]{1.000000,0.000000,0.000000}
\pgfpathmoveto{\pgfpoint{203.971069pt}{127.326637pt}}
\pgflineto{\pgfpoint{204.544022pt}{129.089996pt}}
\pgfusepath{stroke}
\pgfpathmoveto{\pgfpoint{202.471069pt}{126.236824pt}}
\pgflineto{\pgfpoint{203.971069pt}{127.326637pt}}
\pgfusepath{stroke}
\pgfpathmoveto{\pgfpoint{200.616974pt}{126.236824pt}}
\pgflineto{\pgfpoint{202.471069pt}{126.236824pt}}
\pgfusepath{stroke}
\pgfpathmoveto{\pgfpoint{199.116974pt}{127.326637pt}}
\pgflineto{\pgfpoint{200.616974pt}{126.236824pt}}
\pgfusepath{stroke}
\pgfpathmoveto{\pgfpoint{198.544022pt}{129.089996pt}}
\pgflineto{\pgfpoint{199.116974pt}{127.326637pt}}
\pgfusepath{stroke}
\pgfpathmoveto{\pgfpoint{199.116974pt}{130.853348pt}}
\pgflineto{\pgfpoint{198.544022pt}{129.089996pt}}
\pgfusepath{stroke}
\pgfpathmoveto{\pgfpoint{200.616974pt}{131.943161pt}}
\pgflineto{\pgfpoint{199.116974pt}{130.853348pt}}
\pgfusepath{stroke}
\pgfpathmoveto{\pgfpoint{202.471069pt}{131.943161pt}}
\pgflineto{\pgfpoint{200.616974pt}{131.943161pt}}
\pgfusepath{stroke}
\pgfpathmoveto{\pgfpoint{203.971069pt}{130.853348pt}}
\pgflineto{\pgfpoint{202.471069pt}{131.943161pt}}
\pgfusepath{stroke}
\pgfpathmoveto{\pgfpoint{204.544022pt}{129.089996pt}}
\pgflineto{\pgfpoint{203.971069pt}{130.853348pt}}
\pgfusepath{stroke}
\color[rgb]{0.000000,0.000000,0.000000}
\pgfsetdash{{16pt}{0pt}}{0pt}
\pgfpathmoveto{\pgfpoint{288.074158pt}{197.039612pt}}
\pgflineto{\pgfpoint{260.624542pt}{197.039612pt}}
\pgfusepath{stroke}
\pgfpathmoveto{\pgfpoint{288.074158pt}{220.474182pt}}
\pgflineto{\pgfpoint{260.624542pt}{220.474182pt}}
\pgfusepath{stroke}
\pgfpathmoveto{\pgfpoint{260.624542pt}{220.474182pt}}
\pgflineto{\pgfpoint{260.624542pt}{197.039612pt}}
\pgfusepath{stroke}
\pgfpathmoveto{\pgfpoint{288.074158pt}{220.474182pt}}
\pgflineto{\pgfpoint{288.074158pt}{197.039612pt}}
\pgfusepath{stroke}
{
\pgftransformshift{\pgfpoint{275.119781pt}{216.568420pt}}
\pgfnode{rectangle}{west}{\fontsize{10}{0}\selectfont\textcolor[rgb]{0,0,0}{{BUT}}}{}{\pgfusepath{discard}}}
{
\pgftransformshift{\pgfpoint{275.119781pt}{208.756897pt}}
\pgfnode{rectangle}{west}{\fontsize{10}{0}\selectfont\textcolor[rgb]{0,0,0}{{VVJ}}}{}{\pgfusepath{discard}}}
{
\pgftransformshift{\pgfpoint{275.119781pt}{200.945374pt}}
\pgfnode{rectangle}{west}{\fontsize{10}{0}\selectfont\textcolor[rgb]{0,0,0}{{?}}}{}{\pgfusepath{discard}}}
\color[rgb]{0.000000,0.000000,1.000000}
\pgfsetdash{}{0pt}
\pgfpathmoveto{\pgfpoint{270.872192pt}{216.568420pt}}
\pgflineto{\pgfpoint{264.872192pt}{216.568420pt}}
\pgfusepath{stroke}
\pgfpathmoveto{\pgfpoint{267.872192pt}{213.568420pt}}
\pgflineto{\pgfpoint{267.872192pt}{219.568420pt}}
\pgfusepath{stroke}
\color[rgb]{0.000000,0.500000,0.000000}
\pgfpathmoveto{\pgfpoint{270.872192pt}{205.756897pt}}
\pgflineto{\pgfpoint{264.872192pt}{211.756897pt}}
\pgfusepath{stroke}
\pgfpathmoveto{\pgfpoint{270.872192pt}{211.756897pt}}
\pgflineto{\pgfpoint{264.872192pt}{205.756897pt}}
\pgfusepath{stroke}
\color[rgb]{1.000000,0.000000,0.000000}
\pgfpathmoveto{\pgfpoint{270.299255pt}{199.182007pt}}
\pgflineto{\pgfpoint{270.872192pt}{200.945374pt}}
\pgfusepath{stroke}
\pgfpathmoveto{\pgfpoint{268.799255pt}{198.092194pt}}
\pgflineto{\pgfpoint{270.299255pt}{199.182007pt}}
\pgfusepath{stroke}
\pgfpathmoveto{\pgfpoint{266.945160pt}{198.092194pt}}
\pgflineto{\pgfpoint{268.799255pt}{198.092194pt}}
\pgfusepath{stroke}
\pgfpathmoveto{\pgfpoint{265.445129pt}{199.182007pt}}
\pgflineto{\pgfpoint{266.945160pt}{198.092194pt}}
\pgfusepath{stroke}
\pgfpathmoveto{\pgfpoint{264.872192pt}{200.945374pt}}
\pgflineto{\pgfpoint{265.445129pt}{199.182007pt}}
\pgfusepath{stroke}
\pgfpathmoveto{\pgfpoint{265.445129pt}{202.708710pt}}
\pgflineto{\pgfpoint{264.872192pt}{200.945374pt}}
\pgfusepath{stroke}
\pgfpathmoveto{\pgfpoint{266.945160pt}{203.798523pt}}
\pgflineto{\pgfpoint{265.445129pt}{202.708710pt}}
\pgfusepath{stroke}
\pgfpathmoveto{\pgfpoint{268.799255pt}{203.798523pt}}
\pgflineto{\pgfpoint{266.945160pt}{203.798523pt}}
\pgfusepath{stroke}
\pgfpathmoveto{\pgfpoint{270.299255pt}{202.708710pt}}
\pgflineto{\pgfpoint{268.799255pt}{203.798523pt}}
\pgfusepath{stroke}
\pgfpathmoveto{\pgfpoint{270.872192pt}{200.945374pt}}
\pgflineto{\pgfpoint{270.299255pt}{202.708710pt}}
\pgfusepath{stroke}
\end{pgfpicture}

  \only<presentation>{
    \only<2>{\Large \alert{What if it a \textbf{completely different strain}?}}
  }
\end{frame}

\begin{frame}
  \frametitle{Hands on with Python console}
  

\end{frame}

\begin{frame}
  \frametitle{What have we learned}
  \begin{block}{Machine Learning background}
    \begin{itemize}
    \item Many \alert{technical} problem types.
    \item \alert{Scientific} applications.
    \item \alert{Societal} issues.
    \item Machine learning as science.
    \end{itemize}
  \end{block}

  \begin{block}{Methodology}
    \begin{itemize}
    \item \alert{Classification} Problems: \only<3->{$k$-nearest neighbour algorithm.}
    \item Elementary \alert{variable selection}.
    \item \alert{Verifying} your \alert{conclusions}: \only<3->{Validation/Test sets.}
    \item Verifying your \alert{assumptions}: \only<4->{(How?)}
    \end{itemize}
  \end{block}
\end{frame}

\section{Decision problems}

\only<article>{
All machine learning problems are essentially decision problems. This essentially means replacing some human decisions with machine decisions. One of the simplest decision problems is classification, where you want an algorithm to decide the correct class of some data, but even within this simple framework there is a multitude of decisions to be made. The first is how to frame the classification problem the first place. The second is how to collect, process and annotate the data. The third is choosing the type of classification model to use. The fourth is how to use the collected data to find an optimal classifier within the selected type. After all this has been done, there is the problem of classifying new data. In this course, we will take a holistic view of the problem, and consider each problem in turn, starting from the lowest level and working our way up.}

\section{Hierarchies of decision making problems}
\begin{frame}
  \tableofcontents[currentsection]
\end{frame}
\subsection{Simple decision problems}
\begin{frame}
  \frametitle{Preferences}

  \begin{example}
    \begin{block}{Food}
      \begin{itemize}
      \item[A] McDonald's cheeseburger
        \item[B] Surstromming
        \item[C] Oatmeal
        \end{itemize}
      \end{block}
      \begin{block}{Money}
        \begin{itemize}
        \item[A] 10,000,000 SEK
        \item[B] 10,000,000 USD
        \item[C] 10,000,000 BTC
        \end{itemize}
      \end{block}
      \begin{block}{Entertainment}
        \begin{itemize}
        \item[A] Ticket to Liseberg
        \item[B] Ticket to Rebstar
        \item[C] Ticket to Nutcracker
        \end{itemize}
      \end{block}
  \end{example}

  \only<article>{The simplest decision problem involves selecting one item from a set of choices.}  
  \begin{itemize}
  \item Each choice is called a \alert{reward} $r \in \CR$.
  \item There is a \alert{utility function} $U : \CR \to \Reals$, assigning values to reward.
  \item We (weakly) prefer $A$ to $B$ iff $U(A) \geq U(B)$.
  \end{itemize}

  \begin{exercise}
    From your individual preferences, derive a \alert{common utility function} that reflects everybody's preferences in the class.
  \end{exercise}
\end{frame}


\begin{frame}
  \frametitle{Uncertain rewards}
  \only<article>{However, in real life, there are many cases where we can only choose between uncertain outcomes. The simplest example are lottery tickets, where rewards are essentially random. However, in many cases the rewards are not really random, but simply uncertain. In those cases it is useful to represent our uncertainty with probabilities as well, even though there is nothing really random.}
  \begin{example}%ro: rather an exercise?
    You are going to work, and it might rain.
    What do you do?
    \begin{itemize}
    \item $\decision_1$: Take the umbrella.
    \item $\decision_2$: Risk it!
    \item $\outcome_1$: rain
    \item $\outcome_2$: dry
    \end{itemize}
    \begin{table}
      \centering
      \begin{tabular}{c|c|c}
        $\Rew(\outcome,\decision)$ & $\decision_1$ & $\decision_2$ \\ %ro: U has only one argument.
        \hline
        $\outcome_1$ & dry, carrying umbrella & wet\\
        $\outcome_2$ & dry, carrying umbrella & dry\\
        \hline
        \hline
        $U[\Rew(\outcome,\decision)]$ & $\decision_1$ & $\decision_2$ \\
        \hline
        $\outcome_1$ & 0 & -10\\
        $\outcome_2$ & 0 & 1
      \end{tabular}
      \caption{Rewards and utilities.}
      \label{tab:rain-utility-function}
    \end{table}

    \begin{itemize}
    \item<2-> $\max_\decision \min_\outcome U = 0$
    \item<3-> $\min_\outcome \max_\decision  U = 0$
    \end{itemize}
  \end{example}
\end{frame}



\begin{frame}
  \frametitle{Expected utility}
  \[
    \E (U \mid a) = \sum_r U[\Rew(\outcome, \decision)] \Pr(\outcome \mid \decision)
  \]
  \begin{example}%ro: rather an exercise?
    You are going to work, and it might rain. The forecast said that
    the probability of rain $(\outcome_1)$ was $20\%$. What do you do?
    \begin{itemize}
    \item $\decision_1$: Take the umbrella.
    \item $\decision_2$: Risk it!
    \end{itemize}
    \begin{table}
      \centering
      \begin{tabular}{c|c|c}
        $\Rew(\outcome,\decision)$ & $\decision_1$ & $\decision_2$ \\ %ro: U has only one argument.
        \hline
        $\outcome_1$ & dry, carrying umbrella & wet\\
        $\outcome_2$ & dry, carrying umbrella & dry\\
        \hline
        \hline
        $U[\Rew(\outcome,\decision)]$ & $\decision_1$ & $\decision_2$ \\
        \hline
        $\outcome_1$ & 0 & -10\\
        $\outcome_2$ & 0 & 1\\
        \hline
        \hline
        $\E_P(U \mid \decision)$ & 0 &  -1.2 \\ 
      \end{tabular}
      \caption{Rewards, utilities, expected utility for $20\%$ probability of rain.}
      \label{tab:rain-utility-function}
    \end{table}
  \end{example}
\end{frame}

\begin{frame}
  \frametitle{Preferences among random outcomes}
  \begin{example}
    Would you rather \ldots
    \begin{itemize}
    \item[A] Have 100 EUR now?
    \item[B] Flip a coin, and get 200 EUR if it comes heads?
    \end{itemize}    
  \end{example}
  \uncover<2->{
    \begin{block}{The expected utility hypothesis}
      Rational decision makers prefer choice $A$ to $B$ if
      \[
        \E(U | A) \geq \E(U | B),
      \]
      where the expected utility is
      \[
        \E(U | A) = \sum_r U(r) \Pr(r | A).
      \]
    \end{block}
    In the above example, $r \in \{0, 100, 200\}$ and $U(r)$ is
    increasing, and the coin is fair.
  }
  \begin{itemize}
  \item<3-> If $U$ is convex, we prefer B.
  \item<4-> If $U$ is concave, we prefer A.
  \item<5-> If $U$ is linear, we don't care.
  \end{itemize}
\end{frame}




\subsection{Decision rules}
\only<presentation>{
  \begin{frame}
    \tableofcontents[currentsection,currentsubsection]
  \end{frame}
}
\only<article>{We now move from simple decisions to decisions that
  depend on some observation. This is most easily embodied through the
  problem of classification. }
\begin{frame}
  \frametitle{Deciding a class given a model}
  \only<article>{In the simplest classification problem, we observe some features $x_t$ and want to make a guess $\decision_t$ about the true class label $y_t$. Assuming we have some probabilistic model $P(y_t \mid x_t)$, we want to define a decision rule $\pol(\decision_t \mid x_t)$ that is optimal, in the sense that it maximises expected utility for $P$.}
  \begin{itemize}
  \item Features $x_t \in \CX$.
  \item Label $y_t \in \CY$.
  \item Decisions $\decision_t \in \CA$.
  \item Decision rule $\pol(\decision_t \mid x_t)$ assigns probabilities to actions.
  \end{itemize}
  
  \begin{block}{Standard classification problem}
    \only<article>{In the simplest case, the set of decisions we make are the same as the set of classes}
    \[
    \CA = \CY, \qquad
    U(\decision, y) = \ind{\decision = y}
    \]
  \end{block}

  \begin{exercise}
    If we have a model $P(y_t \mid x_t)$, and a suitable $U$, what is the optimal decision to make?
  \end{exercise}
  \only<presentation>{
    \uncover<2->{
      \[
      \decision_t \in \argmax_{\decision \in \Decisions} \sum_y P(y_t = y \mid x_t) \Util(\decision, y)
      \]
    }
    \uncover<3>{
      For standard classification,
      \[
      \decision_t \in \argmax_{\decision \in \Decisions} P(y_t = \decision \mid x_t)
      \]
    }
  }
\end{frame}


\begin{frame}
  \frametitle{Deciding the class given a model family}
  \begin{itemize}
  \item Training data $S = (x_1, y_1, \ldots, x_n, y_n)$
  \item Models $\cset{P_\outcome}{\outcome \in \Outcome}$
  \item Prior $\bel$ on $\Outcome$.
  \end{itemize}
  \[
    \bel(\outcome \mid S)
    = \frac{P_\outcome(y_1, \ldots, y_n \mid x_1, \ldots, x_n) \bel(\outcome)}
    {\sum_{\outcome' \in \Outcome} P_{\outcome'}(y_1, \ldots, y_n \mid x_1, \ldots, x_n) \bel(\outcome')}
  \]
  We can then calculate the posterior marginal marginal label probability
  \[
    P_{\bel \mid S}(y_t \mid x_t) \defn
    P_{\bel}(y_t \mid x_t, S) = 
    \sum_{\outcome \in \Outcome} P_\outcome(y_t \mid x_t) \bel(\omega \mid S).
  \]
  We can then construct the following simple decision rule:
  \[
    \decision_t \in \argmax_{\decision \in \CY}\sum_{\outcome \in \Outcome} P_\outcome(y_t \mid x_t) \bel(\omega \mid S),
  \]
  otherwise known as the \alert{Bayes rule}.
\end{frame}

\section{Bayes decisions}
\begin{frame}
  \frametitle{Bayes decision rules}
  Consider the case where outcomes are independent of decisions:
  \[
    \Util (P, \decision) \defn \sum_{\outcome}  \Util (\outcome, \decision) P(\outcome)
  \]
  This corresponds e.g. to the case where $P(\omega)$ is the belief about an unknown world.
  \begin{definition}[Bayes utility]
    \label{def:bayes-utility}
    The maximising decision for $P$ has an expected utility equal to:
    \begin{equation}
      \BUtil(P) \defn \sup_{\decision \in \Decisions} \Util (P, \decision).
      \label{eq:bayes-utility}
    \end{equation}
  \end{definition}
\end{frame}


\only<article>{
  One of the simplest decision problems is classification. At the simplest level, this is the problem of observing some data point $x_t \in \CX$ and making a decision about what class $\CY$ it belongs to. Typically, a fixed classifier is defined as a decision rule $\pi(a | x)$ making decisions $a \in \CA$, where the decision space includes the class labels, so that if we observe some point $x_t$ and choose $a_t = 1$, we essentially declare that $y_t = 1$.

  Typically, we wish to have a classification policy that minimises classification error.
}
\begin{frame}
  \begin{definition}[Classification error of a fixed decision rule]
    
  \end{definition}
\end{frame}


\bibliographystyle{apalike}
\bibliography{../../bib/mine,../../bib/misc}

\end{document}

%%% Local Variables: 
%%% mode: latex
%%% TeX-master: t
%%% End: 
