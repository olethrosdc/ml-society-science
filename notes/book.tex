%% This is a shortened version of the notes
\documentclass[a4paper,twoside]{book}
\usepackage{geometry}
\usepackage[notheorems]{beamerarticle}

\makeindex

\mode<presentation>{
  % \useinnertheme{rectangles}
  %\useoutertheme{infolines}
  % \usecolortheme{crane}
  % \usecolortheme{rose}
}
%% Pre-amble - commonly defined macros.

%% Packages
\usepackage{amsmath}
\usepackage{amsfonts}
\usepackage{amssymb}
\usepackage{amsbsy}
\usepackage{isomath}
\usepackage{amsthm}
\usepackage{dsfont}
%%\usepackage{theorem}
\usepackage{algorithm}
\usepackage{algorithmicx}
\usepackage{algpseudocode}
\usepackage{mathrsfs}
\usepackage{epsfig}
\usepackage{subcaption}
\usepackage{makeidx} 
\usepackage{colortbl}
\usepackage{enumerate}
\usepackage{multirow}
\usepackage{listings}
\usepackage{pgfplots}
\newlength\fheight
\newlength\fwidth
\only<presentation>{
\setlength\fheight{0.5\columnwidth}
\setlength\fwidth{0.5\columnwidth}
}
\only<article>{
\setlength\fheight{0.25\textwidth}
\setlength\fwidth{0.25\textwidth}
}
\usepackage[sort&compress,comma,super]{natbib}
\def\newblock{} % To avoid a compilation error about a function \newblock undefined
\usepackage{hyperref}

\setbeamertemplate{theorems}[numbered] 
\mode<presentation>{
\theoremstyle{plain}
\newtheorem{assumption}{Assumption}
\theoremstyle{definition}
\newtheorem{exercise}{Exercise}
\theoremstyle{remark}
\newtheorem{remark}{Remark}
}

\numberwithin{equation}{section} 
\mode<article>{
\theoremstyle{plain}
%    \newtheorem{assumption}{Assumption}[section]
\newtheorem{lemma}{Lemma}[section]
\newtheorem{theorem}{Theorem}[section]
\newtheorem{corollary}{Corollary}[section]
\theoremstyle{definition}
\newtheorem{definition}{Definition}[section]
\theoremstyle{remark}
\newtheorem{remark}{Remark}[section]

%\theoremstyle{plain} \newtheorem{remark}{Remark}[section]
%\theoremstyle{plain} \newtheorem{definition}{Definition}[section]
\theoremstyle{plain} \newtheorem{assumption}{Assumption}[section]

%%% Examples %%%%
\newtheoremstyle{example}  % Name
{1em}       % Space above 
{1em}       % Space below
{\small}      % Body font
{}          % Indent amount 
{\scshape}  % Theorem head font
{.}         % Punctuation after theorem head
{.5em}      % Space after theorem head
{}          % Theorem head spec
\theoremstyle{example}
\newtheorem{example}{Example}
\newtheorem{exercise}{Exercise}

\usepackage{framed}
\renewenvironment{block}[1]
{\framed \par \textbf{#1} \newline}
{\par \endframed}

\renewenvironment{exampleblock}[1]
{\framed \par \textit{#1} \newline \bigskip}
{\par \endframed}

\renewenvironment{alertblock}[1]
{\framed \par \textit{\textbf{#1}} \newline \hrule \bigskip}
{\par \endframed}
}


%\theoremstyle{plain} \newtheorem{conjecture}{Conjecture}[section]
%\theoremstyle{plain} \newtheorem{theorem}{Theorem}[section]
%\theoremstyle{plain} \newtheorem{proposition}{Proposition}[section]
%\theoremstyle{plain} \newtheorem{lemma}{Lemma}[section]
%\theoremstyle{plain} \newtheorem{corollary}{Corollary}[section]


%\newenvironment{proof}[1][Proof]{\begin{trivlist}
%\item[\hskip \labelsep {\bfseries #1}]}{\end{trivlist}}
%\newcommand{\qed}{\nobreak \ifvmode \relax \else
%      \ifdim\lastskip<1.5em \hskip-\lastskip
%      \hskip1.5em plus0em minus0.5em \fi \nobreak
%      \vrule height0.5em width0.5em depth0.25em\fi}

\newcommand \indexmargin[1] {\marginpar{\emph{#1}}\index{#1}}
\newcommand \marginref[2] {\marginpar{\emph{#1}}\emph{#1}\index{#2}}
\newcommand \emindex[1] {\emph{#1}\marginpar{\emph{#1}}\index{#1}}


\newcommand \E {\mathop{\mbox{\ensuremath{\mathbb{E}}}}\nolimits}
\newcommand \hE {\hat{\mathop{\mbox{\ensuremath{\mathbb{E}}}}\nolimits}}
\renewcommand \Pr {\mathop{\mbox{\ensuremath{\mathbb{P}}}}\nolimits}
\newcommand \given {\mathrel{|}}
\newcommand \gvn {|}
\newcommand \eq {{=}}


%% Special characters
\newcommand\Reals {{\mathbb{R}}}
\newcommand\Naturals {{\mathbb{N}}} 
\newcommand\Simplex {\mathbold{\Delta}}

\newcommand \FB {{\mathfrak{B}}}
\newcommand \FD {{\mathfrak{D}}}
\newcommand \FF {{\mathfrak{F}}}
\newcommand \FM {{\mathfrak{M}}}
\newcommand \FK {{\mathfrak{K}}}
\newcommand \FJ {{\mathfrak{J}}}
\newcommand \FL {{\mathfrak{L}}}
\newcommand \FO {{\mathfrak{O}}}
\newcommand \FS {{\mathfrak{S}}}
\newcommand \FT {{\mathfrak{T}}}
\newcommand \FP {{\mathfrak{P}}}
\newcommand \FR {{\mathfrak{R}}}


\newcommand \CA {{\mathcal{A}}}
\newcommand \CB {{\mathcal{B}}}
\newcommand \CC {{\mathcal{C}}}
\newcommand \CD {{\mathcal{D}}}
\newcommand \CE {{\mathcal{E}}}
\newcommand \CF {{\mathcal{F}}}
\newcommand \CG {{\mathcal{G}}}
\newcommand \CH {{\mathcal{H}}}
\newcommand \CJ {{\mathcal{J}}}
\newcommand \CL {{\mathcal{L}}}
\newcommand \CM {{\mathcal{M}}}
\newcommand \CN {{\mathcal{N}}}
\newcommand \CO {{\mathcal{O}}}
\newcommand \CP {{\mathcal{P}}}
\newcommand \CQ {{\mathcal{Q}}}
\newcommand \CR {{\mathcal{R}}}
\newcommand \CS {{\mathcal{S}}}
\newcommand \CT {{\mathcal{T}}}
\newcommand \CU {{\mathcal{U}}}
\newcommand \CV {{\mathcal{V}}}
\newcommand \CW {{\mathcal{W}}}
\newcommand \CX {{\mathcal{X}}}
\newcommand \CY {{\mathcal{Y}}}
\newcommand \CZ {{\mathcal{Z}}}

\newcommand \BA {{\mathbb{A}}}
\newcommand \BI {{\mathbb{I}}}
\newcommand \BS {{\mathbb{S}}}

\newcommand \bx {{\mathbf{x}}}
\newcommand \by {{\mathbf{y}}}
\newcommand \bu {{\mathbf{u}}}
\newcommand \bw {{\mathbf{w}}}
\newcommand \ba {{\mathbf{a}}}
\newcommand \bz {{\mathbf{z}}}
\newcommand \bat {{\mathbf{a}_t}}
\newcommand \bh {{\mathbf{h}}}
\newcommand \bo {{\mathbf{o}}}
\newcommand \bp {{\mathbf{p}}}
\newcommand \bs {{\mathbf{s}}}
\newcommand \br {{\mathbf{r}}}

\newcommand \SA {\mathscr{A}}
\newcommand \SB {\mathscr{B}}
\newcommand \SC {\mathscr{C}}
\newcommand \SF {\mathscr{F}}
\newcommand \SG {\mathscr{G}}
\newcommand \SH {\mathscr{H}}
\newcommand \SJ {\mathscr{J}}
\newcommand \SL {\mathscr{L}}
\newcommand \SP {\mathscr{P}}
\newcommand \SR {\mathscr{R}}
%%\newcommand \SS {\mathscr{S}}
\newcommand \ST {\mathscr{T}}
\newcommand \SU {\mathscr{U}}
\newcommand \SV {\mathscr{V}}
\newcommand \SW {\mathscr{W}}

\newcommand \hM {\widehat{M}}

\newcommand \KL[2] {\mathbb{D}\left( #1 \| #2 \right)}


%\newcommand \p {\partial}

\newcommand \then{\Rightarrow}
\newcommand \defn {\mathrel{\triangleq}}
%\newcommand \StateSet {{\CQ}}


%% Commands

\newcommand \argmax{\mathop{\rm arg\,max}}
\newcommand \argmin{\mathop{\rm arg\,min}}
\newcommand \dtan{\mathop{\rm dtan}}
\newcommand \sgn{\mathop{\rm sgn}}
\newcommand \trace{\mathop{\rm tr}}

\newcommand \onenorm[1]{\left\|#1\right\|_1}
\newcommand \pnorm[2]{\left\|#1\right\|_{#2}}
\newcommand \inftynorm[1]{\left\right\|#1\|_\infty}
\newcommand \norm[1]{\left\|#1\right\|}

%%\newcommand \defn {\triangleq}
%%\newcommand \defn {\equiv}
%%\newcommand \defn {\coloneq}
%%\newcommand \defn {\stackrel{\text{\tiny def}}{=}}
%%\newcommand \defn {\stackrel{\text{def}}{\hbox{\equalsfill}}}

\DeclareMathAlphabet{\mathpzc}{OT1}{pzc}{m}{it}

\newcommand \Normal {\mathop{\mathpzc{N}}\nolimits}
\newcommand \Poisson {\mathop{\mathpzc{Poisson}}\nolimits}
\newcommand \Multinomial {\mathop{\mathpzc{Multinomial}}\nolimits}
\newcommand \Dirichlet {\mathop{\mathpzc{Dirichlet}}\nolimits}
\newcommand \Student {\mathop{\mathpzc{Student}}\nolimits}
\newcommand \Bernoulli {\mathop{\mathpzc{Bernoulli}}\nolimits}
\newcommand \BetaDist   {\mathop{\mathpzc{Beta}}\nolimits}
\newcommand \Singular   {\mathop{\mathpzc{D}}\nolimits}
\newcommand \GammaDist {\mathop{\mathpzc{Gamma}}\nolimits}
\newcommand \Softmax{\mathop{\mathpzc{Softmax}}\nolimits}
\newcommand \Exp{\mathop{\mathpzc{Exp}}\nolimits}
\newcommand \Uniform{\mathop{\mathpzc{Unif}}\nolimits}
\newcommand \Laplace {\mathop{\mathpzc{Laplace}}\nolimits}

\newcommand \Param {\Theta}
\newcommand \param {\theta}
\newcommand \vparam {\vectorsym{\theta}}
\newcommand \mparam {\matrixsym{\Theta}}
\newcommand \Hyperparam {\Phi}
\newcommand \hyperparam {\phi}
\newcommand \family {\mathcal{F}}
\newcommand{\ie}{\emph{i.e.}\xspace}
\newcommand{\eg}{\emph{e.g.}\xspace}
\newcommand{\etal}{\emph{et al.}\xspace}
\newcommand{\constg}{}
\newcommand{\Bel}{\Xi}
\newcommand \Bay {\ensuremath{\mathscr{B}}}
\newcommand \Adv {\ensuremath{\mathscr{A}}}


\newcommand \Borel[1] {\FF(#1)}
\newcommand \Probs[1] {\FM(#1)}


\newcommand \pol {\pi}
\newcommand \Pol {\Pi}
\newcommand \mdp {\mu}
\newcommand \MDP {\CM}
\newcommand \meanMDP {{\bar{\mdp}_\xi}}

\newcommand {\msqr} {\vrule height0.33cm width0.44cm}
\newcommand {\bsqr} {\vrule height0.55cm width0.66cm}

\newcommand\ind[1]{\mathop{\mbox{\ensuremath{\mathbb{I}}}}\left\{#1\right\}}
\newcommand\Ind{\mbox{\bf{I}}}

\newcommand\dd{\,\mathrm{d}}

\newcommand \seq[2]{#1^{#2}}
\newcommand \pseq[3]{#1_{#2}^{#3}}
\newcommand \sam[2]{#1^{(#2)}}
\newcommand \transpose[1] {#1^\top}
\newcommand\set[1] {\left\{#1\right\}}
\newcommand\tuple[1] {\left\langle #1\right\rangle}
\newcommand\cset[2] {\left\{#1 ~\middle|~ #2\right\}}
\newcommand \ceil[1]{\left\lceil #1 \right\rceil}





\newcommand{\indep}{\mathrel{\text{\scalebox{1.07}{$\perp\mkern-10mu\perp$}}}}


\newcommand \eqlike {\eqsim}
\newcommand \gtlike {\succ}
\newcommand \ltlike {\prec}
\newcommand \gelike {\succsim}
\newcommand \lelike {\precsim}

\newcommand \eqpref {\eqsim^*}
\newcommand \gtpref {\succ^*}
\newcommand \ltpref {\prec^*}
\newcommand \gepref {\succsim^*}
\newcommand \lepref {\precsim^*}

\newcommand \util {U}
\newcommand \BUtil {U^*}
\newcommand \MUtil {\matrixsym{U}}
\newcommand \risk {\sigma}
\newcommand \Brisk {\sigma^*}
\newcommand \Loss {\ell}
\newcommand \Regret {L}
\newcommand \regret {\ell}
\newcommand \Reward {\SR}
\newcommand \reward {r}
\newcommand \vreward {\vectorsym{r}}
\newcommand \Rew {\rho}
\newcommand \outcome {\omega}
\newcommand \Outcome {\Omega}
\newcommand \act {a}
\newcommand \Act {\CA}
\newcommand \decision {a}
\newcommand \Decision {\mathcal{A}}
\newcommand \dec {\delta}
\newcommand \Dec {\mathscr{D}}


\newcommand {\MH} {\matrixsym{H}}

\newcommand \alg {\lambda}
\newcommand \Alg {\Lambda}
\newcommand \KNN {\textsc{k-NN}}

\newcommand \model {\mu}
\newcommand \MAP {\model_{\textrm{MAP}}}
\newcommand \Model {\CM}
\newcommand \Datasets {\CD}
\newcommand \Data {D}
\newcommand \Training {D_T}
\newcommand \Holdout {D_H}
\newcommand \Testing {D^*}
\newcommand \error {\epsilon}
\newcommand \obs {x}
\newcommand \Obs {\CX}
\newcommand \Att {\CA}
\newcommand \att {a}
\newcommand \attv {v}
\newcommand \Attv {\CV}
\newcommand \cls {y}
\newcommand \Cls {\CY}
\newcommand \Entropy {\mathbb{H}}
\newcommand \Gain {\mathbb{G}}


\newcommand \IDThree {\texttt{ID3}}

\newcommand \nactions {A}
\newcommand \nclasses {C}
\newcommand \nstates{S}
\newcommand \nobservations {N}
\newcommand \ndata{T}

\newcommand \figwidth {0.6\textwidth}
\newcommand \figheight {0.4\textwidth}

\newcommand \eye {\matrixsym{I}}
\newcommand \MA {\matrixsym{A}}
\newcommand \MX {\matrixsym{X}}
\newcommand \MY {\matrixsym{Y}}
\newcommand \MB {\matrixsym{B}}
\newcommand \MV {\matrixsym{V}}
\newcommand \MW {\matrixsym{W}}
\newcommand \MP {\matrixsym{P}}
\newcommand \vg {\vectorsym{\gamma}}
\newcommand \vp {\vectorsym{p}}
\newcommand \vs {\vectorsym{s}}
\newcommand \vx {\vectorsym{x}}
\newcommand \vr {\vectorsym{r}}
\newcommand \vm {\vectorsym{m}}
\newcommand \vb {\vectorsym{b}}
\newcommand \vt {\vectorsym{\theta}}

\newcommand \pn[1] {\vx_{[#1]}}

\newcommand \basis {f}
\newcommand \bel {\xi}
\newcommand \hyper {\omega}
\newcommand \mbel {\bel^D}
\newcommand \pbel {\bel^C}

\newcommand \pmean {\matrixsym{M}}
\newcommand \pcov {\matrixsym{C}}
\newcommand \pwish {\matrixsym{W}}
\newcommand \porder {n}

\newcommand \Syx {\matrixsym{\Sigma}_{yx}}
\newcommand \Sxx {\matrixsym{\Sigma}_{xx}}
\newcommand \Syy {\matrixsym{\Sigma}_{yy}}
\newcommand \Symx {\matrixsym{\Sigma}_{y\mid x}}

\newcommand \trans {\matrixsym{P}}
\newcommand \ident {\matrixsym{I}}

\newcommand \noise {\vectorsym{\varepsilon}}

\newcommand \pt {p_t}


\newcommand \CSet {G}
\newcommand \Parent[1] {\mathfrak{P}(#1)}
\newcommand \Children[1] {\mathfrak{C}(#1)}
\newcommand \Ancestors[1] {\mathfrak{A}(#1)}
\newcommand \Descendants[1] {\mathfrak{D}(#1)}
\newcommand \metric[2] {\nu(#1, #2)}
\newcommand \zooming {\zeta}
\newcommand \depth[1] {d(#1)}

\newcommand \sensitivity[1] {\mathbb{L}\left(#1\right)}
\newcommand \disc {\gamma}
\newcommand \Value {V}
\newcommand \val {\vectorsym{v}}
\newcommand \Vals {\mathcal{V}}
\newcommand \qval {\vectorsym{q}}
\newcommand \Qvals {\mathcal{Q}}
\newcommand \blm {\mathscr{L}}
\newcommand \tdm {\mathscr{D}}
\newcommand \pim {\mathscr{B}}



\newcommand \dist[2]{D\left(#1 ~\middle\|~ #2\right)}

\newcommand \Ae {A_\epsilon^\hist}

\newcommand \lrdist[2]{d_{lr}(#1, #2)}
\newcommand \xdistChar{\rho}
\newcommand \xdist[2]{\xdistChar(#1, #2)}
\newcommand \pdist[2]{\kappa(#1, #2)}
\newcommand{\constScale}{\omega}
\newcommand{\constScaleB}{\kappa}

\newcommand \fields[1]{\sigma(#1)}

\newcommand \hist {h}

\newcommand \abs[1] {\left|#1\right|}

\newcommand{\errorband}[5][]{ % x column, y column, error column, optional argument for setting style of the area plot
\pgfplotstableread[col sep=comma, skip first n=2]{#2}\datatable
% Lower bound (invisible plot)
\addplot [draw=none, stack plots=y, forget plot] table [
x={#3},
y expr=\thisrow{#4}-\thisrow{#5}
] {\datatable};

% Stack twice the error, draw as area plot
\addplot [draw=none, fill=gray!40, stack plots=y, area legend, #1] table [
x={#3},
y expr=2*\thisrow{#5}
] {\datatable} \closedcycle;

% Reset stack using invisible plot
\addplot [forget plot, stack plots=y,draw=none] table [x={#3}, y expr=-(\thisrow{#4}+\thisrow{#5})] {\datatable};
}


%%% macros to make things smalller
% For comparison, the existing overlap macros:
% \def\llap#1{\hbox to 0pt{\hss#1}}
% \def\rlap#1{\hbox to 0pt{#1\hss}}
\def\clap#1{\hbox to 0pt{\hss#1\hss}}
\def\mathllap{\mathpalette\mathllapinternal}
\def\mathrlap{\mathpalette\mathrlapinternal}
\def\mathclap{\mathpalette\mathclapinternal}
\def\mathllapinternal#1#2{%
\llap{$\mathsurround=0pt#1{#2}$}}
\def\mathrlapinternal#1#2{%
\rlap{$\mathsurround=0pt#1{#2}$}}
\def\mathclapinternal#1#2{%
\clap{$\mathsurround=0pt#1{#2}$}}


\usepackage{tikz}

%\usetikzlibrary{external}
%\tikzexternalize[prefix=tikz/]
\usepackage{gnuplot-lua-tikz}


\usetikzlibrary{automata}
\usetikzlibrary{topaths}
\usetikzlibrary{shapes}
\usetikzlibrary{arrows}
\usetikzlibrary{decorations.markings}
\usetikzlibrary{intersections}
\usetikzlibrary{backgrounds}


\tikzstyle{utility}=[diamond,draw=black,draw=blue!50,fill=blue!10,inner sep=0mm, minimum size=8mm]
\tikzstyle{select}=[rectangle,draw=black,draw=blue!50,fill=blue!10,inner sep=0mm, minimum size=6mm]
\tikzstyle{hidden}=[dashed,draw=black,fill=red!10]
\tikzstyle{RV}=[circle,draw=black,draw=blue!50,fill=blue!10,inner sep=0mm, minimum size=6mm]
\tikzstyle{place}=[circle,draw=black,draw=blue!50,fill=blue!20,inner sep=0mm, minimum size=9mm]
\tikzstyle{select}=[rectangle,draw=black,draw=blue!50,fill=blue!20,inner sep=0mm, minimum size=6mm]
\tikzstyle{transition}=[rectangle,draw=black!50,fill=black!20,thick]
\tikzstyle{observed}=[circle,draw=black,draw=blue!50,fill=blue!10,inner sep=0mm, minimum size=6mm]
\tikzstyle{someset}=[circle,draw=black,minimum size=8mm]

\tikzstyle{known}=[rectangle,draw=green!50,fill=green!20,thick]
\tikzstyle{queried}=[rectangle,draw=blue!50,fill=blue!20,thick]
%\tikzstyle{transition}=[rectangle,draw=black!50,fill=black!20,thick]

\tikzstyle{thickarrow}=[->, >=latex, line width=15pt, green!50]
\tikzstyle{medarrow}=[->, >=latex,  line width=5pt]
\tikzstyle{arrow}=[->,>=triangle 60]

\tikzset{every picture/.style={
    line width=1
  }
}

\definecolor{dark-green}{rgb}{0,0.5,0}




\begin{document}


%\includeonly{causality}
\title{Machine learning in science and society}
\subtitle{From automated science to beneficial artificial intelligence}
\author[C. Dimitrakakis]{Christos Dimitrakakis}


\maketitle

\chapter*{Preface}

This book is a technical introduction to important aspects of machine learning in science and society. It covers the most fundamental concepts in privacy, fairness, causality, reproducibility and experiment design. While I have tried to make the book as rigorous as possible, in the interest of brevity some technical details are omitted. 

However, even though the book is not addressed to a technical audience, every concept is necessarily connected to precise technical definition. This includes the bare minimum amount of theoretical development for a thorough understanding of the main concepts. Throughout the book, we use Bayesian decision theory and graphical models to offer a unifying perspective on those issues.


Throughout the book, we will use the following blocks of text for particular types of material.
\begin{exampleblock}{An example.}
Such blocks contain illustrations and examples. 
\end{exampleblock}

\begin{alertblock}{An important note.}
Important caveats or properties of algorithms will be given in such blocks.
\end{alertblock}

\begin{theoryblock}{Theoretical note.}
Discussions of a more theoretical nature, or notes that require some additional thinking.
\end{theoryblock}

\begin{exerciseblock}{Exercise.}
  These exercises are meant to be done individually, either while reading the text, or in class.
\end{exerciseblock}

\begin{groupactivity}{Group activity.}
  These activities are meant to be done in class.
\end{groupactivity}

The first chapter offers an introduction to the field of machine learning in general and to fairness, privacy and reproducibility in particular. Chapter~\ref{ch:privacy} discusses privacy. Chapter~\ref{ch:fairness} talks about fairness.

I whole-heartedly recommend the book ``The Ethical Algorithm'' as a non-technical companion to this book and ``The foundations of differential privacy'' as a more thorough overview of differential privacy.



\tableofcontents


\chapter{Simple decision problems}

\only<article>{This chapter deals with simple decision problems, whereby a decision maker (DM) makes a simple choice among many. In some of this problems the DM has to make a decision after first observing some side-information. Then the DM uses a \emph{decision rule} to assign a probability to each possible decision for each possible side-information. However, designing the decision rule is not trivial , as it relies on previously collected data. A higher-level decision includes choosing the decision rule itself. The problems of classification and regression fall within this framework. While most steps in the process can be automated and formalised, a lot of decisions are actual design choices made by humans. This creates scope for errors and misinterpretation of results.

In this chapter, we shall formalise all these simple decision problems from the point of view of statistical decision theory. The first question is, given a real world application, what type of decision problem does it map to? Then, what kind of machine learning algorithms can we use to solve it? What are the underlying assumptions and how valid are our conclusions? Does application of the method have a societal impact? In particular, is there scope for privacy violations, reinforcing racial biases or creating other negative externalities?
}
\section{Introduction to machine learning}
 

\only<article>{
 What are the central problems in machine learning?

 Problems in machine learning are similar to problems in science.
 Scientists must plan experiments intelligently and collect data.
 The must be able to use the data to verify a different hypothesis.
 More generally, they must be able to make decisions under
 uncertainty (Without uncertainty, there would be no need to gather more data).
 Similar problems appear in more mundane tasks, like learning to drive a car.
}
\only<presentation>{
 \begin{frame}
   \frametitle{Scientific applications}
   \centering
   \begin{columns}
     \begin{column}{0.5\textwidth}
       \centering
       \includegraphics[width=0.8\columnwidth]{../figures/climate.jpg}\\
       \includegraphics[width=\columnwidth]{../figures/networks-2.jpg}
     \end{column}
     \begin{column}{0.5\textwidth}
       \includegraphics[width=\columnwidth]{../figures/dark_matter.jpg}
       \\
       \includegraphics[width=\columnwidth]{../figures/protein.jpg}
     \end{column}
   \end{columns}
  \only<2>{
    \begin{tikzpicture}[remember picture,overlay]
      \draw[fill=black,opacity=0.75] 
      (current page.north east) rectangle (current page.south west);
      \node at (current page.center) {
        {\Huge \alert{Interpretability, Reproducibility}}
      };
    \end{tikzpicture}}
\end{frame}
}

\only<article>{
 For that reason, science is a very natural application area for
 machine learning.  We can model the effects of climate change and
 how to mitigate it; discover structure in social networks; map
 the existence of dark matter in the universe by intelligently
 shifting through weak gravitational lens data, and not only study
 the mechanisms of protein folding, but discover methods to
 synthesize new drugs.

 We must be careful, however. In many cases we need to be able to
 interpret what our model tells us. We also must make sure that
 the any results we obtain are reproducible. This is something
 that we shall emphasize in this course.
}

\only<presentation>{
\begin{frame}
  \frametitle{Pervasive ``intelligent'' systems}
  \begin{columns}
    \begin{column}{0.3\textwidth}
      \centering
      \includegraphics[width=\textwidth]{../figures/echo-home.jpg}
      \\
      Home assistants

      \vspace{\fill}

      \bigskip

      \includegraphics[width=\textwidth]{../figures/tesla.jpg}
      \\
      Autonomous vehicles
    \end{column}
    \begin{column}{0.3\textwidth}
      \centering 
      \includegraphics[width=\textwidth]{../figures/web-ads.png}
      \\
      Web advertising

      \vspace{\fill}

      \bigskip

      \includegraphics[width=\textwidth]{../figures/uber-here-maps.jpg}
      \\
      Ridesharing
    \end{column}
    \begin{column}{0.3\textwidth}
      \centering 
      \\
      \includegraphics[width=\textwidth,clip = true, trim=0 0 0 42.5cm]{../figures/lending.pdf}
      \\
      Lending

      \vspace{\fill}

      \bigskip

      \includegraphics[width=\textwidth]{../figures/algorithms-public.jpg}
      \\
      Public policy
    \end{column}
  \end{columns}
  \only<2>{
    \begin{tikzpicture}[remember picture,overlay]
      \draw[fill=black,opacity=0.75] 
      (current page.north east) rectangle (current page.south west);
      \node at (current page.center) {
        {\Huge \alert{Privacy, Fairness, Safety}}
      };
    \end{tikzpicture}}
\end{frame}
}

\only<article>{
 While machine learning models in science are typically carefully
 handcrafted by scientists and experts in machine learning and
 statistics, this is not typically the case in everyday
 applications. Nevertheless, well-known or home-grown machine
 learning models are being deployed across the application
 spectrum. This involve home assistants that try and get you want,
 web advertising, which tries to find new things for you to want,
 lending, which tries to optimally lend you money so that you buy
 what you didn't need before. We also have autonomous vehicles,
 which take you were you want to go, and ridesharing services,
 which do the same thing, but use humans instead. Finally, there
 are many applications in public policy, such as crime prevention,
 justice, and disease control which use machine learning.  In all
 those cases, we have to worry about a great many things that are
 outside the scope of the machine learning problems itself. These
 are (a) privacy: you don't want your data used in ways that you have
 not consented to (b) fairness: you don't want minorities to be
 disadvantaged and (c) safety: you don't want your car to crash.
}

\subsection{Data analysis}

\only<article>{
  To make the above more concrete, let's have a look at a number of problems. 
}

\only<presentation>{
\begin{frame}
  \centering
  \Huge{What can machine learning do?}
\end{frame}
}
\begin{frame}
  \frametitle{Can machines learn from data?}
  \begin{center}
    \only<1>{\includegraphics[width=0.8\textwidth]{../figures/text-cloud}
      \\

      {\large An unsupervised learning problem: topic modelling}
    }
    \only<2>{\includegraphics[width=0.8\textwidth]{../figures/Face-Recognition}
      \\

      {\large A supervised learning problem: object recognition}
    }
  \end{center}
\end{frame}


\only<article>{
You can use machine learning just to analyse, or find structure in
the data. This is generally called unsupervised learning. One such
example is topic modelling, where you let the algorithm find topics
from a corpus of text.  These days machines are used to learn from
in many applications.  These include speech recognition, facial
authentication, weather prediction, etc. In general, in these
problems we are given a \emph{labelled} dataset with, say, example
images from each class. Unfortunately this does not scale very
well, because obtaining labels is expensive.

This is partially how science works, because what we need to do
is to find a general rule of nature from data. Starting from some
hypothesis and some data, we reach a conclusion. However, many
times we may need to actively experiment to obtain more data,
perhaps because we found that our model is wrong.
}



\begin{frame}
  \frametitle{Can machines learn from their mistakes?}
  \begin{center}
    \includegraphics[width=0.7\textwidth]{../figures/rl_interaction}
  \end{center}
  \begin{block}{Reinforcement learning}
    Take actions $a_1, \ldots, a_t$, so as to maximise utility
    $U = \sum_{t=1}^T r_t$
  \end{block}
\end{frame}


\only<article>{
So, what happens when we make a mistake? Can we somehow recognise
it? Humans and other animals can actually learn from their
mistakes. Consider the proverbial rat in the maze. At some
intervals, the experimenter places some cheese in there, and the
rat must do a series of actions to obtain it, such as navigating
the maze and pulling some levers. It doesn't know how to get to
the cheese easily, but it slowly learns the layout of the maze
through observation, and in the end, through trial-and-error it
is able to get to the cheese very efficiently.

We can formalise this as a reinforcement learning problem, where
the rat takes a series of actions; at each step it also obtains a
reward, let's say equal to 0 when it has no cheese, and 1 when it
eats cheese. Then we can declare that the rat's utility is the sum
of all rewards over time, i.e. the total amount of cheese it can
eat before it dies. The rat needs to explore the environment in order to be able to
get to the cheese. 

An example in robotics is trying to teach a
robot to flip pancakes. One easy thing we can try is to show the robot
how to do it, and then let it just copy the demonstrated
movement. However, this doesn't work! The robot needs to explore
variations of the movement, until it manages to successfully flip
pancakes. Again, we can formulate this as a reinforcement learning
problem, with a reward that is high whenever the pancake's position is
flipped, and on the pan; and low everywhere else. Then the robot can
learn to perform this behaviour through trial and error. It's
important to note that in this example, merely demonstration is not
enough. Neither is reinforcement learning enough. The same thing is
true for the recent success of AlphaGo in beating a master human:
apart from planning, they used both demonstration data and self-play,
so that it could learn through trial and error.  }

\begin{frame}
  \frametitle{Can machines make complex plans?}
  \begin{center}
    \includegraphics[width=0.8\textwidth]{../figures/619px-FloorGoban}
  \end{center}
\end{frame}


\only<article>{
 I suppose the first question is whether machines can plan
 ahead. Indeed, even for large problems, such as Go, machines can
 now perform at least as well as top-rated humans. How is this
 achieved?
}

\begin{frame}
  \frametitle{Machines can make complex plans!}
  \begin{center}
    \includegraphics[width=0.8\textwidth]{../figures/Tic-tac-toe-game-tree}
  \end{center}
\end{frame}


\only<article>{
The basic construction is the planning tree. This is an enumeration
of all possible future events. If a complete enumeration is
impossible, a partial tree is constructed. However this requires
evaluating non-terminal game positions. In the old times, this was
done with heuristics, but now this is data-driven, both through the
use of expert databases, and through self-play and reinforcement
learning.
}


\subsection{Experiment design}

\only<presentation>{
  \begin{frame}
    \centering
    \Huge{The scientific process as machine learning}
  \end{frame}
  \begin{frame}
    \centering
    \includegraphics[width=\textwidth]{../figures/Las_Vegas_slot_machines}
  \end{frame}
}


\only<article>{
An example that typifies trial and error learning are bandit
problems. Imagine that you are in a Casino and you wish to
maximise the amount of money you make during the night. There are
a lot of machines to play. If you knew which one was the best,
then you'd just play it all night long. However, you must also
spend time trying out different machines, in order to get an
estimate of how much money each one gives out. The trade off
between trying out different machines and playing the one you
currently think is best is called the exploration-exploitation
trade-off and it appears in many problems of experiment design for
science.
}


\begin{frame}
  \frametitle{Adam, the robot scientist}
  \centering
  \includegraphics[width=0.8\textwidth]{../figures/robot-scientist}
\end{frame}


\only<article>{
 Let's say we want to build a robot scientist and tell it to
 discover a cure for cancer. What does the scientist do and how can the robot replicate it??
}



\begin{frame}
  \frametitle{Drug discovery}
  \centering
  \includegraphics[width=\columnwidth]{../figures/drug-discovery-000}
\end{frame}


\only<article>{
Simplifying the problem a bit, consider that you have a large
number of drug candidates for cancer and you wish to discover
those that are active against it. The ideas is that you select
some of them, then screen them, to sort them into active and
inactive. However, there are too many drugs to screen, so the
process is interactive. At each cycle, we select some drugs to
screen, classify them, and then use this information to select
more drugs to screen. This cycle, consequently has two parts:
1. Selecting some drugs given our current knowledge.
2. Updating our knowledge given new evidence.
}


\begin{frame}
  \frametitle{Drawing conclusions from results}
  \centering
  \begin{tikzpicture}[line width=2pt]
    \node at (0,0) (bt) {hypothesis};
    \node[select] at (0,2) (at) {experiment};
    \node[utility] at (3,-2) (rt) {result};
    \draw[blue,->] (at) -- (rt);
    \node at (4,0) (bt2) {conclusion};
    \draw[red,->] (at) -- (bt2);
    \draw[red,->] (bt) -- (bt2);
    \draw[red,->] (rt) -- (bt2);
  \end{tikzpicture}
\end{frame}

\only<article>{
  In general, we would like to have some method which can draw
  conclusions from results. This involves starting with a
  hypothesis, performing an experiment to verify or refute it,
  obtain some experimental result; and then concluding for or
  against the hypothesis. Here the arrows show dependencies
  between these variables. So what do we mean by "hypothesis" in this case?
}

\subsection{Bayesian inference.}
\begin{frame}
  \frametitle{Tycho Brahe's minute eye measurements}
  \begin{columns}
    \begin{column}{0.5\textwidth}
      \includegraphics[width=0.5\textwidth]{../figures/circular-orbits}
    \end{column}
    \begin{column}{0.5\textwidth}
      \includegraphics[width=0.5\textwidth]{../figures/tycho-observations}
    \end{column}
  \end{columns}
  \begin{itemize}
  \item Hypothesis: Circular orbits
  \item Conclusion: \alert{Specific} circular orbits
  \end{itemize}
\end{frame}


\only<article>{
  Let's take the example of planetary orbits. Here Tycho famously
  spent 20 years experimentally measuring the location of Mars. He
  had a hypothesis: that planetary orbits were circular, but he
  didn't know which were the right orbits. When he tried to fit his data to this hypothesis, he concluded a specific circular orbit for Mars \ldots around Earth.
}


\begin{frame}
  \frametitle{Johannes Kepler's alternative hypothesis}
  \begin{columns}
    \begin{column}{0.5\textwidth}
      \includegraphics[width=0.5\textwidth]{../figures/orbits}
    \end{column}
    \begin{column}{0.5\textwidth}
      \includegraphics[width=0.5\textwidth]{../figures/tycho-observations}
    \end{column}
  \end{columns}
  \begin{itemize}
  \item Hypothesis: Circular \alert{or} elliptic orbits
  \item Conclusion: Specific \alert{elliptic} orbits
  \end{itemize}
\end{frame}


\only<article>{
Kepler had a more general hypothesis: that orbits could be
circular or elliptic, and he actually accepted that the planets
orbited the sun. This led him to the broadly correct model of all
planets being in elliptical orbits around the sun. However, the
actual verification that all things do not revolve around earth,
requires different experiments.
}


\begin{frame}
  \frametitle{200 years later, Gauss formalised this statistically}
  \begin{columns}
    \begin{column}{0.5\textwidth}
      \includegraphics[width=\columnwidth]{../figures/gauss-diagram}
    \end{column}
    \begin{column}{0.5\textwidth}
      \includegraphics[width=\columnwidth]{../figures/SeptemberTable}
    \end{column}
  \end{columns}
\end{frame}


\only<article>{
Later on, Gauss collected even more experimental data to calculate the orbit of Ceres. He did this using one of the first formal statistical methods; this allowed him to avoid cheating (like Kepler did, to accentuate his finding that orbits were elliptical).
}


\begin{frame}
  \frametitle{A warning: The dead salmon mirage}
  \includegraphics[width=\textwidth]{../figures/fmri-salmon}
\end{frame}


\only<article>{
It is quite easy to draw the wrong conclusions from applying
machine learning / statistics to your data. For example, it was
fashionable to perform fMRI studies in humans to see whether some
neurons have a particular functional role. There were even
articles saying that "we found the neurons encoding for Angelina
Jolie". So some scientists tried to replicate those results. They
took a dead salmon, and put it an fMRI scanner. They checked its
brain activity when it was shown images of happy or sad
people. Perhaps surprisingly, they found an area of the brain that
was correlated with the pictures - so it seemed, as though the
dead salmon could distinguish photos of happy people from sad
ones. However, this was all due to a misapplication of
statistics. In this course, we will try and teach you to avoid
such mistakes.
}


\begin{frame}
  \frametitle{Planning future experiments}
  \centering
  \begin{tikzpicture}[line width=2pt]
    \node at (0,0) (bt) {hypothesis};
    \node[select] at (0,2) (at) {experiment};
    \node[utility] at (3,-2) (rt) {result};
    \draw[blue,->] (at) -- (rt);
    \node at (4,0) (bt2) {conclusion};
    \draw[red,->] (at) -- (bt2);
    \draw[red,->] (bt) -- (bt2);
    \draw[red,->] (rt) -- (bt2);
  \end{tikzpicture}
\end{frame}

\only<article>{
I mentioned before that we must decide what experiment to do. This is indeed difficult, especially in setting such as drug discovery where the number of experiments is huge.  However, conceptually, there is a simple and elegant solution to this problem.
}


\begin{frame}
  \frametitle{Planning experiments is like Tic-Tac-Toe}
  \begin{center}
    \includegraphics[width=\textwidth]{../figures/Tic-tac-toe-game-tree}
  \end{center}
\end{frame}


\only<article>{
  The basic idea is to think of experiment design as a game between the scientist and Nature. At every step, the scientist plays an X to  denote an experiment. Then Nature responds with an Observation. The main difference from a game is that Nature is (probably) not adversarial. We can also generalise this idea to problems in robotics, etc.
}

\only<presentation>{
\begin{frame}
  \frametitle{Eve, another robot scientist}
  \centering \movie{\includegraphics[width=\textwidth]{../figures/eve.jpg}}{Eve-video.mp4}
  Discovered a malaria drug
\end{frame}
}
\only<article>{
 These kinds of techniques, coming from the reinforcement learning literature have been successfully used at the university of Manchester to create a robot, called Eve, that recently (re)-discovered a malaria drug.
}

\subsection{Course overview}

\begin{frame}
  \frametitle{Machine learning in practice}
  \begin{block}{Avoiding pitfalls}
    \begin{itemize}
    \item Choosing hypotheses.
    \item Correctly interpreting conclusions.
    \item Using a good testing methodology.
    \end{itemize}
  \end{block}
  \begin{block}{Machine learning in society}
    \begin{itemize}
    \item<alert@2> Privacy \uncover<2->{--- Medical data.}
    \item<alert@3> Fairness \uncover<3->{--- Credit risk.}
    \item<alert@4> Safety \uncover<4->{--- Autonomous vehicles.}
    \end{itemize}
  \end{block}
\end{frame}

\only<article>{
One of the things we want to do in this course is teach you to
avoid common pitfalls.

Now I want to get into a different track. So far everything has
been about pure research, but now machine learning is pervasive:
Our phones, cars, watches, bathrooms, kettles are connected to the
internet and send a continuous stream of data to companies. In
addition, many companies and government actors use machine
learning algorithms to make or support decisions. This creates a
number of problems in privacy, fairness and safety.
}


\begin{frame}
  \frametitle{Technical topics}
  
  \begin{block}{Machine learning problems}
    \begin{itemize}
    \item Unsupervised learning.
    \item Supervised learning.
    \item Reinforcement learning.
    \end{itemize}
  \end{block}

  \begin{block}{Machine learning tools}
    \begin{itemize}
    \item Stochastic optimisation and neural networks.
    \item Probabilistic inference and Bayesian networks.
    \item Markov decision processes.
    \end{itemize}
  \end{block}
\end{frame}

\begin{frame}
  \frametitle{Course structure}
  \begin{block}{Module structure}
    \begin{itemize}
    \item \alert{Activity}-based, hands-on.
    \item Background reading \alert{before} class.
    \item Mini-lecture with \alert{Q/A} at each class.
    \item Small \alert{group project} in second half of class.
    \end{itemize}
  \end{block}

  \begin{block}{Modules}
    \begin{itemize}
    \item Medical diagnostics.
    \item Speech recognition.
    \item Recommendation systems.
    \item Helicopter flight.
    \end{itemize}
  \end{block}
\end{frame}


\section{Nearest neighbours}
\begin{frame}
  \frametitle{Discriminating between diseases}
  % Title: glps_renderer figure
% Creator: GL2PS 1.3.8, (C) 1999-2012 C. Geuzaine
% For: Octave
% CreationDate: Fri Jun 16 12:38:10 2017
\begin{pgfpicture}
\pgfsetlinewidth{0.01pt}
\color[rgb]{1.000000,1.000000,1.000000}
\pgfpathmoveto{\pgfpoint{41.600006pt}{205.577454pt}}
\pgflineto{\pgfpoint{289.600037pt}{140.777435pt}}
\pgflineto{\pgfpoint{41.600006pt}{140.777435pt}}
\pgfpathclose
\pgfusepath{fill,stroke}
\pgfpathmoveto{\pgfpoint{41.600006pt}{205.577454pt}}
\pgflineto{\pgfpoint{289.600037pt}{205.577454pt}}
\pgflineto{\pgfpoint{289.600037pt}{140.777435pt}}
\pgfpathclose
\pgfusepath{fill,stroke}
\pgfpathmoveto{\pgfpoint{41.600006pt}{91.199989pt}}
\pgflineto{\pgfpoint{289.600037pt}{26.399979pt}}
\pgflineto{\pgfpoint{41.600006pt}{26.399979pt}}
\pgfpathclose
\pgfusepath{fill,stroke}
\pgfpathmoveto{\pgfpoint{41.600006pt}{91.199989pt}}
\pgflineto{\pgfpoint{289.600037pt}{91.199989pt}}
\pgflineto{\pgfpoint{289.600037pt}{26.399979pt}}
\pgfpathclose
\pgfusepath{fill,stroke}
\color[rgb]{1.000000,0.000000,0.000000}
\pgfpathmoveto{\pgfpoint{287.608032pt}{140.777435pt}}
\pgflineto{\pgfpoint{288.604004pt}{140.777435pt}}
\pgflineto{\pgfpoint{288.604004pt}{142.317886pt}}
\pgfpathclose
\pgfusepath{fill,stroke}
\pgfpathmoveto{\pgfpoint{289.600037pt}{141.234161pt}}
\pgflineto{\pgfpoint{288.604004pt}{142.317886pt}}
\pgflineto{\pgfpoint{288.604004pt}{140.777435pt}}
\pgfpathclose
\pgfusepath{fill,stroke}
\pgfpathmoveto{\pgfpoint{289.600037pt}{140.777435pt}}
\pgflineto{\pgfpoint{289.600037pt}{141.234161pt}}
\pgflineto{\pgfpoint{288.604004pt}{140.777435pt}}
\pgfpathclose
\pgfusepath{fill,stroke}
\pgfpathmoveto{\pgfpoint{283.624115pt}{140.777435pt}}
\pgflineto{\pgfpoint{284.620087pt}{140.777435pt}}
\pgflineto{\pgfpoint{284.620087pt}{140.819778pt}}
\pgfpathclose
\pgfusepath{fill,stroke}
\pgfpathmoveto{\pgfpoint{285.616089pt}{141.090790pt}}
\pgflineto{\pgfpoint{284.620087pt}{140.819778pt}}
\pgflineto{\pgfpoint{284.620087pt}{140.777435pt}}
\pgfpathclose
\pgfusepath{fill,stroke}
\pgfpathmoveto{\pgfpoint{285.616089pt}{140.777435pt}}
\pgflineto{\pgfpoint{285.616089pt}{141.090790pt}}
\pgflineto{\pgfpoint{284.620087pt}{140.777435pt}}
\pgfpathclose
\pgfusepath{fill,stroke}
\pgfpathmoveto{\pgfpoint{286.612061pt}{140.777435pt}}
\pgflineto{\pgfpoint{285.616089pt}{141.090790pt}}
\pgflineto{\pgfpoint{285.616089pt}{140.777435pt}}
\pgfpathclose
\pgfusepath{fill,stroke}
\pgfpathmoveto{\pgfpoint{275.656250pt}{140.777435pt}}
\pgflineto{\pgfpoint{276.652222pt}{140.777435pt}}
\pgflineto{\pgfpoint{276.652222pt}{141.883286pt}}
\pgfpathclose
\pgfusepath{fill,stroke}
\pgfpathmoveto{\pgfpoint{277.648193pt}{145.700470pt}}
\pgflineto{\pgfpoint{276.652222pt}{141.883286pt}}
\pgflineto{\pgfpoint{276.652222pt}{140.777435pt}}
\pgfpathclose
\pgfusepath{fill,stroke}
\pgfpathmoveto{\pgfpoint{277.648193pt}{140.777435pt}}
\pgflineto{\pgfpoint{277.648193pt}{145.700470pt}}
\pgflineto{\pgfpoint{276.652222pt}{140.777435pt}}
\pgfpathclose
\pgfusepath{fill,stroke}
\pgfpathmoveto{\pgfpoint{278.644196pt}{141.350510pt}}
\pgflineto{\pgfpoint{277.648193pt}{145.700470pt}}
\pgflineto{\pgfpoint{277.648193pt}{140.777435pt}}
\pgfpathclose
\pgfusepath{fill,stroke}
\pgfpathmoveto{\pgfpoint{278.644196pt}{140.777435pt}}
\pgflineto{\pgfpoint{278.644196pt}{141.350510pt}}
\pgflineto{\pgfpoint{277.648193pt}{140.777435pt}}
\pgfpathclose
\pgfusepath{fill,stroke}
\pgfpathmoveto{\pgfpoint{279.640167pt}{141.685425pt}}
\pgflineto{\pgfpoint{278.644196pt}{141.350510pt}}
\pgflineto{\pgfpoint{278.644196pt}{140.777435pt}}
\pgfpathclose
\pgfusepath{fill,stroke}
\pgfpathmoveto{\pgfpoint{279.640167pt}{140.777435pt}}
\pgflineto{\pgfpoint{279.640167pt}{141.685425pt}}
\pgflineto{\pgfpoint{278.644196pt}{140.777435pt}}
\pgfpathclose
\pgfusepath{fill,stroke}
\pgfpathmoveto{\pgfpoint{280.636139pt}{140.839111pt}}
\pgflineto{\pgfpoint{279.640167pt}{141.685425pt}}
\pgflineto{\pgfpoint{279.640167pt}{140.777435pt}}
\pgfpathclose
\pgfusepath{fill,stroke}
\pgfpathmoveto{\pgfpoint{280.636139pt}{140.777435pt}}
\pgflineto{\pgfpoint{280.636139pt}{140.839111pt}}
\pgflineto{\pgfpoint{279.640167pt}{140.777435pt}}
\pgfpathclose
\pgfusepath{fill,stroke}
\pgfpathmoveto{\pgfpoint{281.632141pt}{140.783401pt}}
\pgflineto{\pgfpoint{280.636139pt}{140.839111pt}}
\pgflineto{\pgfpoint{280.636139pt}{140.777435pt}}
\pgfpathclose
\pgfusepath{fill,stroke}
\pgfpathmoveto{\pgfpoint{281.632141pt}{140.777435pt}}
\pgflineto{\pgfpoint{281.632141pt}{140.783401pt}}
\pgflineto{\pgfpoint{280.636139pt}{140.777435pt}}
\pgfpathclose
\pgfusepath{fill,stroke}
\pgfpathmoveto{\pgfpoint{282.628113pt}{141.247986pt}}
\pgflineto{\pgfpoint{281.632141pt}{140.783401pt}}
\pgflineto{\pgfpoint{281.632141pt}{140.777435pt}}
\pgfpathclose
\pgfusepath{fill,stroke}
\pgfpathmoveto{\pgfpoint{282.628113pt}{140.777435pt}}
\pgflineto{\pgfpoint{282.628113pt}{141.247986pt}}
\pgflineto{\pgfpoint{281.632141pt}{140.777435pt}}
\pgfpathclose
\pgfusepath{fill,stroke}
\pgfpathmoveto{\pgfpoint{283.624115pt}{140.777435pt}}
\pgflineto{\pgfpoint{282.628113pt}{141.247986pt}}
\pgflineto{\pgfpoint{282.628113pt}{140.777435pt}}
\pgfpathclose
\pgfusepath{fill,stroke}
\pgfpathmoveto{\pgfpoint{262.708435pt}{140.777435pt}}
\pgflineto{\pgfpoint{263.704407pt}{140.777435pt}}
\pgflineto{\pgfpoint{263.704407pt}{149.672318pt}}
\pgfpathclose
\pgfusepath{fill,stroke}
\pgfpathmoveto{\pgfpoint{264.700409pt}{150.470245pt}}
\pgflineto{\pgfpoint{263.704407pt}{149.672318pt}}
\pgflineto{\pgfpoint{263.704407pt}{140.777435pt}}
\pgfpathclose
\pgfusepath{fill,stroke}
\pgfpathmoveto{\pgfpoint{264.700409pt}{140.777435pt}}
\pgflineto{\pgfpoint{264.700409pt}{150.470245pt}}
\pgflineto{\pgfpoint{263.704407pt}{140.777435pt}}
\pgfpathclose
\pgfusepath{fill,stroke}
\pgfpathmoveto{\pgfpoint{265.696411pt}{146.772659pt}}
\pgflineto{\pgfpoint{264.700409pt}{150.470245pt}}
\pgflineto{\pgfpoint{264.700409pt}{140.777435pt}}
\pgfpathclose
\pgfusepath{fill,stroke}
\pgfpathmoveto{\pgfpoint{265.696411pt}{140.777435pt}}
\pgflineto{\pgfpoint{265.696411pt}{146.772659pt}}
\pgflineto{\pgfpoint{264.700409pt}{140.777435pt}}
\pgfpathclose
\pgfusepath{fill,stroke}
\pgfpathmoveto{\pgfpoint{266.692383pt}{141.012543pt}}
\pgflineto{\pgfpoint{265.696411pt}{146.772659pt}}
\pgflineto{\pgfpoint{265.696411pt}{140.777435pt}}
\pgfpathclose
\pgfusepath{fill,stroke}
\pgfpathmoveto{\pgfpoint{266.692383pt}{140.777435pt}}
\pgflineto{\pgfpoint{266.692383pt}{141.012543pt}}
\pgflineto{\pgfpoint{265.696411pt}{140.777435pt}}
\pgfpathclose
\pgfusepath{fill,stroke}
\pgfpathmoveto{\pgfpoint{267.688354pt}{140.947601pt}}
\pgflineto{\pgfpoint{266.692383pt}{141.012543pt}}
\pgflineto{\pgfpoint{266.692383pt}{140.777435pt}}
\pgfpathclose
\pgfusepath{fill,stroke}
\pgfpathmoveto{\pgfpoint{267.688354pt}{140.777435pt}}
\pgflineto{\pgfpoint{267.688354pt}{140.947601pt}}
\pgflineto{\pgfpoint{266.692383pt}{140.777435pt}}
\pgfpathclose
\pgfusepath{fill,stroke}
\pgfpathmoveto{\pgfpoint{268.684326pt}{165.780823pt}}
\pgflineto{\pgfpoint{267.688354pt}{140.947601pt}}
\pgflineto{\pgfpoint{267.688354pt}{140.777435pt}}
\pgfpathclose
\pgfusepath{fill,stroke}
\pgfpathmoveto{\pgfpoint{268.684326pt}{140.777435pt}}
\pgflineto{\pgfpoint{268.684326pt}{165.780823pt}}
\pgflineto{\pgfpoint{267.688354pt}{140.777435pt}}
\pgfpathclose
\pgfusepath{fill,stroke}
\pgfpathmoveto{\pgfpoint{269.680328pt}{142.178589pt}}
\pgflineto{\pgfpoint{268.684326pt}{165.780823pt}}
\pgflineto{\pgfpoint{268.684326pt}{140.777435pt}}
\pgfpathclose
\pgfusepath{fill,stroke}
\pgfpathmoveto{\pgfpoint{269.680328pt}{140.777435pt}}
\pgflineto{\pgfpoint{269.680328pt}{142.178589pt}}
\pgflineto{\pgfpoint{268.684326pt}{140.777435pt}}
\pgfpathclose
\pgfusepath{fill,stroke}
\pgfpathmoveto{\pgfpoint{270.676331pt}{140.852936pt}}
\pgflineto{\pgfpoint{269.680328pt}{142.178589pt}}
\pgflineto{\pgfpoint{269.680328pt}{140.777435pt}}
\pgfpathclose
\pgfusepath{fill,stroke}
\pgfpathmoveto{\pgfpoint{270.676331pt}{140.777435pt}}
\pgflineto{\pgfpoint{270.676331pt}{140.852936pt}}
\pgflineto{\pgfpoint{269.680328pt}{140.777435pt}}
\pgfpathclose
\pgfusepath{fill,stroke}
\pgfpathmoveto{\pgfpoint{271.672302pt}{149.853119pt}}
\pgflineto{\pgfpoint{270.676331pt}{140.852936pt}}
\pgflineto{\pgfpoint{270.676331pt}{140.777435pt}}
\pgfpathclose
\pgfusepath{fill,stroke}
\pgfpathmoveto{\pgfpoint{271.672302pt}{140.777435pt}}
\pgflineto{\pgfpoint{271.672302pt}{149.853119pt}}
\pgflineto{\pgfpoint{270.676331pt}{140.777435pt}}
\pgfpathclose
\pgfusepath{fill,stroke}
\pgfpathmoveto{\pgfpoint{272.668274pt}{167.198486pt}}
\pgflineto{\pgfpoint{271.672302pt}{149.853119pt}}
\pgflineto{\pgfpoint{271.672302pt}{140.777435pt}}
\pgfpathclose
\pgfusepath{fill,stroke}
\pgfpathmoveto{\pgfpoint{272.668274pt}{140.777435pt}}
\pgflineto{\pgfpoint{272.668274pt}{167.198486pt}}
\pgflineto{\pgfpoint{271.672302pt}{140.777435pt}}
\pgfpathclose
\pgfusepath{fill,stroke}
\pgfpathmoveto{\pgfpoint{273.664276pt}{140.777435pt}}
\pgflineto{\pgfpoint{272.668274pt}{167.198486pt}}
\pgflineto{\pgfpoint{272.668274pt}{140.777435pt}}
\pgfpathclose
\pgfusepath{fill,stroke}
\pgfpathmoveto{\pgfpoint{255.736542pt}{140.777435pt}}
\pgflineto{\pgfpoint{256.732544pt}{140.777435pt}}
\pgflineto{\pgfpoint{256.732544pt}{160.055664pt}}
\pgfpathclose
\pgfusepath{fill,stroke}
\pgfpathmoveto{\pgfpoint{257.728516pt}{145.276688pt}}
\pgflineto{\pgfpoint{256.732544pt}{160.055664pt}}
\pgflineto{\pgfpoint{256.732544pt}{140.777435pt}}
\pgfpathclose
\pgfusepath{fill,stroke}
\pgfpathmoveto{\pgfpoint{257.728516pt}{140.777435pt}}
\pgflineto{\pgfpoint{257.728516pt}{145.276688pt}}
\pgflineto{\pgfpoint{256.732544pt}{140.777435pt}}
\pgfpathclose
\pgfusepath{fill,stroke}
\pgfpathmoveto{\pgfpoint{258.724518pt}{145.143265pt}}
\pgflineto{\pgfpoint{257.728516pt}{145.276688pt}}
\pgflineto{\pgfpoint{257.728516pt}{140.777435pt}}
\pgfpathclose
\pgfusepath{fill,stroke}
\pgfpathmoveto{\pgfpoint{258.724518pt}{140.777435pt}}
\pgflineto{\pgfpoint{258.724518pt}{145.143265pt}}
\pgflineto{\pgfpoint{257.728516pt}{140.777435pt}}
\pgfpathclose
\pgfusepath{fill,stroke}
\pgfpathmoveto{\pgfpoint{259.720490pt}{141.107025pt}}
\pgflineto{\pgfpoint{258.724518pt}{145.143265pt}}
\pgflineto{\pgfpoint{258.724518pt}{140.777435pt}}
\pgfpathclose
\pgfusepath{fill,stroke}
\pgfpathmoveto{\pgfpoint{259.720490pt}{140.777435pt}}
\pgflineto{\pgfpoint{259.720490pt}{141.107025pt}}
\pgflineto{\pgfpoint{258.724518pt}{140.777435pt}}
\pgfpathclose
\pgfusepath{fill,stroke}
\pgfpathmoveto{\pgfpoint{260.716492pt}{140.785751pt}}
\pgflineto{\pgfpoint{259.720490pt}{141.107025pt}}
\pgflineto{\pgfpoint{259.720490pt}{140.777435pt}}
\pgfpathclose
\pgfusepath{fill,stroke}
\pgfpathmoveto{\pgfpoint{260.716492pt}{140.777435pt}}
\pgflineto{\pgfpoint{260.716492pt}{140.785751pt}}
\pgflineto{\pgfpoint{259.720490pt}{140.777435pt}}
\pgfpathclose
\pgfusepath{fill,stroke}
\pgfpathmoveto{\pgfpoint{261.712463pt}{140.881470pt}}
\pgflineto{\pgfpoint{260.716492pt}{140.785751pt}}
\pgflineto{\pgfpoint{260.716492pt}{140.777435pt}}
\pgfpathclose
\pgfusepath{fill,stroke}
\pgfpathmoveto{\pgfpoint{261.712463pt}{140.777435pt}}
\pgflineto{\pgfpoint{261.712463pt}{140.881470pt}}
\pgflineto{\pgfpoint{260.716492pt}{140.777435pt}}
\pgfpathclose
\pgfusepath{fill,stroke}
\pgfpathmoveto{\pgfpoint{262.708435pt}{140.777435pt}}
\pgflineto{\pgfpoint{261.712463pt}{140.881470pt}}
\pgflineto{\pgfpoint{261.712463pt}{140.777435pt}}
\pgfpathclose
\pgfusepath{fill,stroke}
\pgfpathmoveto{\pgfpoint{252.748611pt}{140.777435pt}}
\pgflineto{\pgfpoint{253.744598pt}{205.642242pt}}
\pgflineto{\pgfpoint{253.234100pt}{205.642242pt}}
\pgfpathclose
\pgfusepath{fill,stroke}
\pgfpathmoveto{\pgfpoint{252.748611pt}{140.777435pt}}
\pgflineto{\pgfpoint{253.744598pt}{140.777435pt}}
\pgflineto{\pgfpoint{253.744598pt}{205.642242pt}}
\pgfpathclose
\pgfusepath{fill,stroke}
\pgfpathmoveto{\pgfpoint{254.325424pt}{205.642242pt}}
\pgflineto{\pgfpoint{253.744598pt}{205.642242pt}}
\pgflineto{\pgfpoint{253.744598pt}{140.777435pt}}
\pgfpathclose
\pgfusepath{fill,stroke}
\pgfpathmoveto{\pgfpoint{254.740570pt}{140.777435pt}}
\pgflineto{\pgfpoint{254.325424pt}{205.642242pt}}
\pgflineto{\pgfpoint{253.744598pt}{140.777435pt}}
\pgfpathclose
\pgfusepath{fill,stroke}
\pgfpathmoveto{\pgfpoint{254.740570pt}{140.777435pt}}
\pgflineto{\pgfpoint{254.740570pt}{205.642242pt}}
\pgflineto{\pgfpoint{254.325424pt}{205.642242pt}}
\pgfpathclose
\pgfusepath{fill,stroke}
\pgfpathmoveto{\pgfpoint{255.736542pt}{140.777435pt}}
\pgflineto{\pgfpoint{254.740570pt}{205.642242pt}}
\pgflineto{\pgfpoint{254.740570pt}{140.777435pt}}
\pgfpathclose
\pgfusepath{fill,stroke}
\pgfpathmoveto{\pgfpoint{255.736542pt}{140.777435pt}}
\pgflineto{\pgfpoint{255.155716pt}{205.642242pt}}
\pgflineto{\pgfpoint{254.740570pt}{205.642242pt}}
\pgfpathclose
\pgfusepath{fill,stroke}
\pgfpathmoveto{\pgfpoint{244.780731pt}{140.777435pt}}
\pgflineto{\pgfpoint{245.776718pt}{140.777435pt}}
\pgflineto{\pgfpoint{245.776718pt}{140.921082pt}}
\pgfpathclose
\pgfusepath{fill,stroke}
\pgfpathmoveto{\pgfpoint{246.231064pt}{205.642242pt}}
\pgflineto{\pgfpoint{245.776718pt}{140.921082pt}}
\pgflineto{\pgfpoint{245.776718pt}{140.777435pt}}
\pgfpathclose
\pgfusepath{fill,stroke}
\pgfpathmoveto{\pgfpoint{246.231064pt}{205.642242pt}}
\pgflineto{\pgfpoint{246.230515pt}{205.642242pt}}
\pgflineto{\pgfpoint{245.776718pt}{140.921082pt}}
\pgfpathclose
\pgfusepath{fill,stroke}
\pgfpathmoveto{\pgfpoint{246.772705pt}{140.777435pt}}
\pgflineto{\pgfpoint{246.231064pt}{205.642242pt}}
\pgflineto{\pgfpoint{245.776718pt}{140.777435pt}}
\pgfpathclose
\pgfusepath{fill,stroke}
\pgfpathmoveto{\pgfpoint{246.772705pt}{140.777435pt}}
\pgflineto{\pgfpoint{246.772705pt}{205.642242pt}}
\pgflineto{\pgfpoint{246.231064pt}{205.642242pt}}
\pgfpathclose
\pgfusepath{fill,stroke}
\pgfpathmoveto{\pgfpoint{247.768677pt}{144.035980pt}}
\pgflineto{\pgfpoint{246.772705pt}{205.642242pt}}
\pgflineto{\pgfpoint{246.772705pt}{140.777435pt}}
\pgfpathclose
\pgfusepath{fill,stroke}
\pgfpathmoveto{\pgfpoint{247.768677pt}{144.035980pt}}
\pgflineto{\pgfpoint{247.327042pt}{205.642242pt}}
\pgflineto{\pgfpoint{246.772705pt}{205.642242pt}}
\pgfpathclose
\pgfusepath{fill,stroke}
\pgfpathmoveto{\pgfpoint{247.768677pt}{140.777435pt}}
\pgflineto{\pgfpoint{247.768677pt}{144.035980pt}}
\pgflineto{\pgfpoint{246.772705pt}{140.777435pt}}
\pgfpathclose
\pgfusepath{fill,stroke}
\pgfpathmoveto{\pgfpoint{248.764679pt}{140.987183pt}}
\pgflineto{\pgfpoint{247.768677pt}{144.035980pt}}
\pgflineto{\pgfpoint{247.768677pt}{140.777435pt}}
\pgfpathclose
\pgfusepath{fill,stroke}
\pgfpathmoveto{\pgfpoint{248.764679pt}{140.777435pt}}
\pgflineto{\pgfpoint{248.764679pt}{140.987183pt}}
\pgflineto{\pgfpoint{247.768677pt}{140.777435pt}}
\pgfpathclose
\pgfusepath{fill,stroke}
\pgfpathmoveto{\pgfpoint{249.760651pt}{140.861908pt}}
\pgflineto{\pgfpoint{248.764679pt}{140.987183pt}}
\pgflineto{\pgfpoint{248.764679pt}{140.777435pt}}
\pgfpathclose
\pgfusepath{fill,stroke}
\pgfpathmoveto{\pgfpoint{249.760651pt}{140.777435pt}}
\pgflineto{\pgfpoint{249.760651pt}{140.861908pt}}
\pgflineto{\pgfpoint{248.764679pt}{140.777435pt}}
\pgfpathclose
\pgfusepath{fill,stroke}
\pgfpathmoveto{\pgfpoint{250.756638pt}{143.997910pt}}
\pgflineto{\pgfpoint{249.760651pt}{140.861908pt}}
\pgflineto{\pgfpoint{249.760651pt}{140.777435pt}}
\pgfpathclose
\pgfusepath{fill,stroke}
\pgfpathmoveto{\pgfpoint{250.756638pt}{140.777435pt}}
\pgflineto{\pgfpoint{250.756638pt}{143.997910pt}}
\pgflineto{\pgfpoint{249.760651pt}{140.777435pt}}
\pgfpathclose
\pgfusepath{fill,stroke}
\pgfpathmoveto{\pgfpoint{251.752625pt}{140.777435pt}}
\pgflineto{\pgfpoint{250.756638pt}{143.997910pt}}
\pgflineto{\pgfpoint{250.756638pt}{140.777435pt}}
\pgfpathclose
\pgfusepath{fill,stroke}
\pgfpathmoveto{\pgfpoint{241.792786pt}{140.777435pt}}
\pgflineto{\pgfpoint{242.788757pt}{140.777435pt}}
\pgflineto{\pgfpoint{242.788757pt}{141.050537pt}}
\pgfpathclose
\pgfusepath{fill,stroke}
\pgfpathmoveto{\pgfpoint{243.784744pt}{140.826157pt}}
\pgflineto{\pgfpoint{242.788757pt}{141.050537pt}}
\pgflineto{\pgfpoint{242.788757pt}{140.777435pt}}
\pgfpathclose
\pgfusepath{fill,stroke}
\pgfpathmoveto{\pgfpoint{243.784744pt}{140.777435pt}}
\pgflineto{\pgfpoint{243.784744pt}{140.826157pt}}
\pgflineto{\pgfpoint{242.788757pt}{140.777435pt}}
\pgfpathclose
\pgfusepath{fill,stroke}
\pgfpathmoveto{\pgfpoint{244.780731pt}{140.777435pt}}
\pgflineto{\pgfpoint{243.784744pt}{140.826157pt}}
\pgflineto{\pgfpoint{243.784744pt}{140.777435pt}}
\pgfpathclose
\pgfusepath{fill,stroke}
\pgfpathmoveto{\pgfpoint{239.800812pt}{140.777435pt}}
\pgflineto{\pgfpoint{240.796814pt}{140.777435pt}}
\pgflineto{\pgfpoint{240.796814pt}{141.635849pt}}
\pgfpathclose
\pgfusepath{fill,stroke}
\pgfpathmoveto{\pgfpoint{241.792786pt}{140.777435pt}}
\pgflineto{\pgfpoint{240.796814pt}{141.635849pt}}
\pgflineto{\pgfpoint{240.796814pt}{140.777435pt}}
\pgfpathclose
\pgfusepath{fill,stroke}
\pgfpathmoveto{\pgfpoint{232.828934pt}{140.777435pt}}
\pgflineto{\pgfpoint{233.824921pt}{140.777435pt}}
\pgflineto{\pgfpoint{233.824921pt}{143.663483pt}}
\pgfpathclose
\pgfusepath{fill,stroke}
\pgfpathmoveto{\pgfpoint{234.820892pt}{144.424850pt}}
\pgflineto{\pgfpoint{233.824921pt}{143.663483pt}}
\pgflineto{\pgfpoint{233.824921pt}{140.777435pt}}
\pgfpathclose
\pgfusepath{fill,stroke}
\pgfpathmoveto{\pgfpoint{234.820892pt}{140.777435pt}}
\pgflineto{\pgfpoint{234.820892pt}{144.424850pt}}
\pgflineto{\pgfpoint{233.824921pt}{140.777435pt}}
\pgfpathclose
\pgfusepath{fill,stroke}
\pgfpathmoveto{\pgfpoint{235.816864pt}{141.070068pt}}
\pgflineto{\pgfpoint{234.820892pt}{144.424850pt}}
\pgflineto{\pgfpoint{234.820892pt}{140.777435pt}}
\pgfpathclose
\pgfusepath{fill,stroke}
\pgfpathmoveto{\pgfpoint{235.816864pt}{140.777435pt}}
\pgflineto{\pgfpoint{235.816864pt}{141.070068pt}}
\pgflineto{\pgfpoint{234.820892pt}{140.777435pt}}
\pgfpathclose
\pgfusepath{fill,stroke}
\pgfpathmoveto{\pgfpoint{236.812866pt}{141.387177pt}}
\pgflineto{\pgfpoint{235.816864pt}{141.070068pt}}
\pgflineto{\pgfpoint{235.816864pt}{140.777435pt}}
\pgfpathclose
\pgfusepath{fill,stroke}
\pgfpathmoveto{\pgfpoint{236.812866pt}{140.777435pt}}
\pgflineto{\pgfpoint{236.812866pt}{141.387177pt}}
\pgflineto{\pgfpoint{235.816864pt}{140.777435pt}}
\pgfpathclose
\pgfusepath{fill,stroke}
\pgfpathmoveto{\pgfpoint{237.808838pt}{141.605896pt}}
\pgflineto{\pgfpoint{236.812866pt}{141.387177pt}}
\pgflineto{\pgfpoint{236.812866pt}{140.777435pt}}
\pgfpathclose
\pgfusepath{fill,stroke}
\pgfpathmoveto{\pgfpoint{237.808838pt}{140.777435pt}}
\pgflineto{\pgfpoint{237.808838pt}{141.605896pt}}
\pgflineto{\pgfpoint{236.812866pt}{140.777435pt}}
\pgfpathclose
\pgfusepath{fill,stroke}
\pgfpathmoveto{\pgfpoint{238.804825pt}{141.892487pt}}
\pgflineto{\pgfpoint{237.808838pt}{141.605896pt}}
\pgflineto{\pgfpoint{237.808838pt}{140.777435pt}}
\pgfpathclose
\pgfusepath{fill,stroke}
\pgfpathmoveto{\pgfpoint{238.804825pt}{140.777435pt}}
\pgflineto{\pgfpoint{238.804825pt}{141.892487pt}}
\pgflineto{\pgfpoint{237.808838pt}{140.777435pt}}
\pgfpathclose
\pgfusepath{fill,stroke}
\pgfpathmoveto{\pgfpoint{239.800812pt}{140.777435pt}}
\pgflineto{\pgfpoint{238.804825pt}{141.892487pt}}
\pgflineto{\pgfpoint{238.804825pt}{140.777435pt}}
\pgfpathclose
\pgfusepath{fill,stroke}
\pgfpathmoveto{\pgfpoint{226.853027pt}{140.777435pt}}
\pgflineto{\pgfpoint{227.849014pt}{140.777435pt}}
\pgflineto{\pgfpoint{227.849014pt}{141.269379pt}}
\pgfpathclose
\pgfusepath{fill,stroke}
\pgfpathmoveto{\pgfpoint{228.845001pt}{143.487183pt}}
\pgflineto{\pgfpoint{227.849014pt}{141.269379pt}}
\pgflineto{\pgfpoint{227.849014pt}{140.777435pt}}
\pgfpathclose
\pgfusepath{fill,stroke}
\pgfpathmoveto{\pgfpoint{228.845001pt}{140.777435pt}}
\pgflineto{\pgfpoint{228.845001pt}{143.487183pt}}
\pgflineto{\pgfpoint{227.849014pt}{140.777435pt}}
\pgfpathclose
\pgfusepath{fill,stroke}
\pgfpathmoveto{\pgfpoint{229.840973pt}{143.864670pt}}
\pgflineto{\pgfpoint{228.845001pt}{143.487183pt}}
\pgflineto{\pgfpoint{228.845001pt}{140.777435pt}}
\pgfpathclose
\pgfusepath{fill,stroke}
\pgfpathmoveto{\pgfpoint{229.840973pt}{140.777435pt}}
\pgflineto{\pgfpoint{229.840973pt}{143.864670pt}}
\pgflineto{\pgfpoint{228.845001pt}{140.777435pt}}
\pgfpathclose
\pgfusepath{fill,stroke}
\pgfpathmoveto{\pgfpoint{230.836945pt}{141.428329pt}}
\pgflineto{\pgfpoint{229.840973pt}{143.864670pt}}
\pgflineto{\pgfpoint{229.840973pt}{140.777435pt}}
\pgfpathclose
\pgfusepath{fill,stroke}
\pgfpathmoveto{\pgfpoint{230.836945pt}{140.777435pt}}
\pgflineto{\pgfpoint{230.836945pt}{141.428329pt}}
\pgflineto{\pgfpoint{229.840973pt}{140.777435pt}}
\pgfpathclose
\pgfusepath{fill,stroke}
\pgfpathmoveto{\pgfpoint{231.832932pt}{162.511902pt}}
\pgflineto{\pgfpoint{230.836945pt}{141.428329pt}}
\pgflineto{\pgfpoint{230.836945pt}{140.777435pt}}
\pgfpathclose
\pgfusepath{fill,stroke}
\pgfpathmoveto{\pgfpoint{231.832932pt}{140.777435pt}}
\pgflineto{\pgfpoint{231.832932pt}{162.511902pt}}
\pgflineto{\pgfpoint{230.836945pt}{140.777435pt}}
\pgfpathclose
\pgfusepath{fill,stroke}
\pgfpathmoveto{\pgfpoint{232.828934pt}{140.777435pt}}
\pgflineto{\pgfpoint{231.832932pt}{162.511902pt}}
\pgflineto{\pgfpoint{231.832932pt}{140.777435pt}}
\pgfpathclose
\pgfusepath{fill,stroke}
\pgfpathmoveto{\pgfpoint{222.869080pt}{140.777435pt}}
\pgflineto{\pgfpoint{223.865082pt}{140.777435pt}}
\pgflineto{\pgfpoint{223.865082pt}{142.648438pt}}
\pgfpathclose
\pgfusepath{fill,stroke}
\pgfpathmoveto{\pgfpoint{224.861053pt}{140.814316pt}}
\pgflineto{\pgfpoint{223.865082pt}{142.648438pt}}
\pgflineto{\pgfpoint{223.865082pt}{140.777435pt}}
\pgfpathclose
\pgfusepath{fill,stroke}
\pgfpathmoveto{\pgfpoint{224.861053pt}{140.777435pt}}
\pgflineto{\pgfpoint{224.861053pt}{140.814316pt}}
\pgflineto{\pgfpoint{223.865082pt}{140.777435pt}}
\pgfpathclose
\pgfusepath{fill,stroke}
\pgfpathmoveto{\pgfpoint{225.857040pt}{142.893295pt}}
\pgflineto{\pgfpoint{224.861053pt}{140.814316pt}}
\pgflineto{\pgfpoint{224.861053pt}{140.777435pt}}
\pgfpathclose
\pgfusepath{fill,stroke}
\pgfpathmoveto{\pgfpoint{225.857040pt}{140.777435pt}}
\pgflineto{\pgfpoint{225.857040pt}{142.893295pt}}
\pgflineto{\pgfpoint{224.861053pt}{140.777435pt}}
\pgfpathclose
\pgfusepath{fill,stroke}
\pgfpathmoveto{\pgfpoint{226.853027pt}{140.777435pt}}
\pgflineto{\pgfpoint{225.857040pt}{142.893295pt}}
\pgflineto{\pgfpoint{225.857040pt}{140.777435pt}}
\pgfpathclose
\pgfusepath{fill,stroke}
\pgfpathmoveto{\pgfpoint{218.885147pt}{140.777435pt}}
\pgflineto{\pgfpoint{219.881134pt}{140.777435pt}}
\pgflineto{\pgfpoint{219.881134pt}{141.240646pt}}
\pgfpathclose
\pgfusepath{fill,stroke}
\pgfpathmoveto{\pgfpoint{220.877121pt}{140.777435pt}}
\pgflineto{\pgfpoint{219.881134pt}{141.240646pt}}
\pgflineto{\pgfpoint{219.881134pt}{140.777435pt}}
\pgfpathclose
\pgfusepath{fill,stroke}
\pgfpathmoveto{\pgfpoint{216.893188pt}{140.777435pt}}
\pgflineto{\pgfpoint{217.889160pt}{140.777435pt}}
\pgflineto{\pgfpoint{217.889160pt}{144.868698pt}}
\pgfpathclose
\pgfusepath{fill,stroke}
\pgfpathmoveto{\pgfpoint{218.885147pt}{140.777435pt}}
\pgflineto{\pgfpoint{217.889160pt}{144.868698pt}}
\pgflineto{\pgfpoint{217.889160pt}{140.777435pt}}
\pgfpathclose
\pgfusepath{fill,stroke}
\pgfpathmoveto{\pgfpoint{207.929337pt}{140.777435pt}}
\pgflineto{\pgfpoint{208.925323pt}{140.777435pt}}
\pgflineto{\pgfpoint{208.925323pt}{168.977936pt}}
\pgfpathclose
\pgfusepath{fill,stroke}
\pgfpathmoveto{\pgfpoint{209.716339pt}{205.642242pt}}
\pgflineto{\pgfpoint{208.925323pt}{168.977936pt}}
\pgflineto{\pgfpoint{208.925323pt}{140.777435pt}}
\pgfpathclose
\pgfusepath{fill,stroke}
\pgfpathmoveto{\pgfpoint{209.716339pt}{205.642242pt}}
\pgflineto{\pgfpoint{209.608246pt}{205.642242pt}}
\pgflineto{\pgfpoint{208.925323pt}{168.977936pt}}
\pgfpathclose
\pgfusepath{fill,stroke}
\pgfpathmoveto{\pgfpoint{209.921295pt}{140.777435pt}}
\pgflineto{\pgfpoint{209.716339pt}{205.642242pt}}
\pgflineto{\pgfpoint{208.925323pt}{140.777435pt}}
\pgfpathclose
\pgfusepath{fill,stroke}
\pgfpathmoveto{\pgfpoint{209.921295pt}{140.777435pt}}
\pgflineto{\pgfpoint{209.921295pt}{205.642242pt}}
\pgflineto{\pgfpoint{209.716339pt}{205.642242pt}}
\pgfpathclose
\pgfusepath{fill,stroke}
\pgfpathmoveto{\pgfpoint{210.917267pt}{156.156219pt}}
\pgflineto{\pgfpoint{209.921295pt}{205.642242pt}}
\pgflineto{\pgfpoint{209.921295pt}{140.777435pt}}
\pgfpathclose
\pgfusepath{fill,stroke}
\pgfpathmoveto{\pgfpoint{210.917267pt}{156.156219pt}}
\pgflineto{\pgfpoint{210.173798pt}{205.642242pt}}
\pgflineto{\pgfpoint{209.921295pt}{205.642242pt}}
\pgfpathclose
\pgfusepath{fill,stroke}
\pgfpathmoveto{\pgfpoint{210.917267pt}{140.777435pt}}
\pgflineto{\pgfpoint{210.917267pt}{156.156219pt}}
\pgflineto{\pgfpoint{209.921295pt}{140.777435pt}}
\pgfpathclose
\pgfusepath{fill,stroke}
\pgfpathmoveto{\pgfpoint{211.913269pt}{142.768951pt}}
\pgflineto{\pgfpoint{210.917267pt}{156.156219pt}}
\pgflineto{\pgfpoint{210.917267pt}{140.777435pt}}
\pgfpathclose
\pgfusepath{fill,stroke}
\pgfpathmoveto{\pgfpoint{211.913269pt}{140.777435pt}}
\pgflineto{\pgfpoint{211.913269pt}{142.768951pt}}
\pgflineto{\pgfpoint{210.917267pt}{140.777435pt}}
\pgfpathclose
\pgfusepath{fill,stroke}
\pgfpathmoveto{\pgfpoint{212.909241pt}{140.796997pt}}
\pgflineto{\pgfpoint{211.913269pt}{142.768951pt}}
\pgflineto{\pgfpoint{211.913269pt}{140.777435pt}}
\pgfpathclose
\pgfusepath{fill,stroke}
\pgfpathmoveto{\pgfpoint{212.909241pt}{140.777435pt}}
\pgflineto{\pgfpoint{212.909241pt}{140.796997pt}}
\pgflineto{\pgfpoint{211.913269pt}{140.777435pt}}
\pgfpathclose
\pgfusepath{fill,stroke}
\pgfpathmoveto{\pgfpoint{213.905228pt}{147.912155pt}}
\pgflineto{\pgfpoint{212.909241pt}{140.796997pt}}
\pgflineto{\pgfpoint{212.909241pt}{140.777435pt}}
\pgfpathclose
\pgfusepath{fill,stroke}
\pgfpathmoveto{\pgfpoint{213.905228pt}{140.777435pt}}
\pgflineto{\pgfpoint{213.905228pt}{147.912155pt}}
\pgflineto{\pgfpoint{212.909241pt}{140.777435pt}}
\pgfpathclose
\pgfusepath{fill,stroke}
\pgfpathmoveto{\pgfpoint{214.901215pt}{141.340469pt}}
\pgflineto{\pgfpoint{213.905228pt}{147.912155pt}}
\pgflineto{\pgfpoint{213.905228pt}{140.777435pt}}
\pgfpathclose
\pgfusepath{fill,stroke}
\pgfpathmoveto{\pgfpoint{214.901215pt}{140.777435pt}}
\pgflineto{\pgfpoint{214.901215pt}{141.340469pt}}
\pgflineto{\pgfpoint{213.905228pt}{140.777435pt}}
\pgfpathclose
\pgfusepath{fill,stroke}
\pgfpathmoveto{\pgfpoint{215.897217pt}{146.916656pt}}
\pgflineto{\pgfpoint{214.901215pt}{141.340469pt}}
\pgflineto{\pgfpoint{214.901215pt}{140.777435pt}}
\pgfpathclose
\pgfusepath{fill,stroke}
\pgfpathmoveto{\pgfpoint{215.897217pt}{140.777435pt}}
\pgflineto{\pgfpoint{215.897217pt}{146.916656pt}}
\pgflineto{\pgfpoint{214.901215pt}{140.777435pt}}
\pgfpathclose
\pgfusepath{fill,stroke}
\pgfpathmoveto{\pgfpoint{216.893188pt}{140.777435pt}}
\pgflineto{\pgfpoint{215.897217pt}{146.916656pt}}
\pgflineto{\pgfpoint{215.897217pt}{140.777435pt}}
\pgfpathclose
\pgfusepath{fill,stroke}
\pgfpathmoveto{\pgfpoint{204.941376pt}{140.777435pt}}
\pgflineto{\pgfpoint{205.937347pt}{140.777435pt}}
\pgflineto{\pgfpoint{205.937347pt}{144.822937pt}}
\pgfpathclose
\pgfusepath{fill,stroke}
\pgfpathmoveto{\pgfpoint{206.933334pt}{140.777435pt}}
\pgflineto{\pgfpoint{205.937347pt}{144.822937pt}}
\pgflineto{\pgfpoint{205.937347pt}{140.777435pt}}
\pgfpathclose
\pgfusepath{fill,stroke}
\pgfpathmoveto{\pgfpoint{199.961456pt}{140.777435pt}}
\pgflineto{\pgfpoint{200.957443pt}{140.777435pt}}
\pgflineto{\pgfpoint{200.957443pt}{141.105011pt}}
\pgfpathclose
\pgfusepath{fill,stroke}
\pgfpathmoveto{\pgfpoint{201.953430pt}{140.916779pt}}
\pgflineto{\pgfpoint{200.957443pt}{141.105011pt}}
\pgflineto{\pgfpoint{200.957443pt}{140.777435pt}}
\pgfpathclose
\pgfusepath{fill,stroke}
\pgfpathmoveto{\pgfpoint{201.953430pt}{140.777435pt}}
\pgflineto{\pgfpoint{201.953430pt}{140.916779pt}}
\pgflineto{\pgfpoint{200.957443pt}{140.777435pt}}
\pgfpathclose
\pgfusepath{fill,stroke}
\pgfpathmoveto{\pgfpoint{202.949402pt}{140.777435pt}}
\pgflineto{\pgfpoint{201.953430pt}{140.916779pt}}
\pgflineto{\pgfpoint{201.953430pt}{140.777435pt}}
\pgfpathclose
\pgfusepath{fill,stroke}
\pgfpathmoveto{\pgfpoint{188.009659pt}{140.777435pt}}
\pgflineto{\pgfpoint{189.005630pt}{140.777435pt}}
\pgflineto{\pgfpoint{189.005630pt}{140.805771pt}}
\pgfpathclose
\pgfusepath{fill,stroke}
\pgfpathmoveto{\pgfpoint{190.001617pt}{149.564423pt}}
\pgflineto{\pgfpoint{189.005630pt}{140.805771pt}}
\pgflineto{\pgfpoint{189.005630pt}{140.777435pt}}
\pgfpathclose
\pgfusepath{fill,stroke}
\pgfpathmoveto{\pgfpoint{190.001617pt}{140.777435pt}}
\pgflineto{\pgfpoint{190.001617pt}{149.564423pt}}
\pgflineto{\pgfpoint{189.005630pt}{140.777435pt}}
\pgfpathclose
\pgfusepath{fill,stroke}
\pgfpathmoveto{\pgfpoint{190.997604pt}{140.863358pt}}
\pgflineto{\pgfpoint{190.001617pt}{149.564423pt}}
\pgflineto{\pgfpoint{190.001617pt}{140.777435pt}}
\pgfpathclose
\pgfusepath{fill,stroke}
\pgfpathmoveto{\pgfpoint{190.997604pt}{140.777435pt}}
\pgflineto{\pgfpoint{190.997604pt}{140.863358pt}}
\pgflineto{\pgfpoint{190.001617pt}{140.777435pt}}
\pgfpathclose
\pgfusepath{fill,stroke}
\pgfpathmoveto{\pgfpoint{191.993591pt}{142.184387pt}}
\pgflineto{\pgfpoint{190.997604pt}{140.863358pt}}
\pgflineto{\pgfpoint{190.997604pt}{140.777435pt}}
\pgfpathclose
\pgfusepath{fill,stroke}
\pgfpathmoveto{\pgfpoint{191.993591pt}{140.777435pt}}
\pgflineto{\pgfpoint{191.993591pt}{142.184387pt}}
\pgflineto{\pgfpoint{190.997604pt}{140.777435pt}}
\pgfpathclose
\pgfusepath{fill,stroke}
\pgfpathmoveto{\pgfpoint{192.989563pt}{195.151352pt}}
\pgflineto{\pgfpoint{191.993591pt}{142.184387pt}}
\pgflineto{\pgfpoint{191.993591pt}{140.777435pt}}
\pgfpathclose
\pgfusepath{fill,stroke}
\pgfpathmoveto{\pgfpoint{192.989563pt}{140.777435pt}}
\pgflineto{\pgfpoint{192.989563pt}{195.151352pt}}
\pgflineto{\pgfpoint{191.993591pt}{140.777435pt}}
\pgfpathclose
\pgfusepath{fill,stroke}
\pgfpathmoveto{\pgfpoint{193.985565pt}{141.529373pt}}
\pgflineto{\pgfpoint{192.989563pt}{195.151352pt}}
\pgflineto{\pgfpoint{192.989563pt}{140.777435pt}}
\pgfpathclose
\pgfusepath{fill,stroke}
\pgfpathmoveto{\pgfpoint{193.985565pt}{140.777435pt}}
\pgflineto{\pgfpoint{193.985565pt}{141.529373pt}}
\pgflineto{\pgfpoint{192.989563pt}{140.777435pt}}
\pgfpathclose
\pgfusepath{fill,stroke}
\pgfpathmoveto{\pgfpoint{194.981537pt}{140.831909pt}}
\pgflineto{\pgfpoint{193.985565pt}{141.529373pt}}
\pgflineto{\pgfpoint{193.985565pt}{140.777435pt}}
\pgfpathclose
\pgfusepath{fill,stroke}
\pgfpathmoveto{\pgfpoint{194.981537pt}{140.777435pt}}
\pgflineto{\pgfpoint{194.981537pt}{140.831909pt}}
\pgflineto{\pgfpoint{193.985565pt}{140.777435pt}}
\pgfpathclose
\pgfusepath{fill,stroke}
\pgfpathmoveto{\pgfpoint{195.977524pt}{149.854721pt}}
\pgflineto{\pgfpoint{194.981537pt}{140.831909pt}}
\pgflineto{\pgfpoint{194.981537pt}{140.777435pt}}
\pgfpathclose
\pgfusepath{fill,stroke}
\pgfpathmoveto{\pgfpoint{195.977524pt}{140.777435pt}}
\pgflineto{\pgfpoint{195.977524pt}{149.854721pt}}
\pgflineto{\pgfpoint{194.981537pt}{140.777435pt}}
\pgfpathclose
\pgfusepath{fill,stroke}
\pgfpathmoveto{\pgfpoint{196.973511pt}{144.710114pt}}
\pgflineto{\pgfpoint{195.977524pt}{149.854721pt}}
\pgflineto{\pgfpoint{195.977524pt}{140.777435pt}}
\pgfpathclose
\pgfusepath{fill,stroke}
\pgfpathmoveto{\pgfpoint{196.973511pt}{140.777435pt}}
\pgflineto{\pgfpoint{196.973511pt}{144.710114pt}}
\pgflineto{\pgfpoint{195.977524pt}{140.777435pt}}
\pgfpathclose
\pgfusepath{fill,stroke}
\pgfpathmoveto{\pgfpoint{197.969498pt}{144.257416pt}}
\pgflineto{\pgfpoint{196.973511pt}{144.710114pt}}
\pgflineto{\pgfpoint{196.973511pt}{140.777435pt}}
\pgfpathclose
\pgfusepath{fill,stroke}
\pgfpathmoveto{\pgfpoint{197.969498pt}{140.777435pt}}
\pgflineto{\pgfpoint{197.969498pt}{144.257416pt}}
\pgflineto{\pgfpoint{196.973511pt}{140.777435pt}}
\pgfpathclose
\pgfusepath{fill,stroke}
\pgfpathmoveto{\pgfpoint{198.965469pt}{197.820709pt}}
\pgflineto{\pgfpoint{197.969498pt}{144.257416pt}}
\pgflineto{\pgfpoint{197.969498pt}{140.777435pt}}
\pgfpathclose
\pgfusepath{fill,stroke}
\pgfpathmoveto{\pgfpoint{198.965469pt}{140.777435pt}}
\pgflineto{\pgfpoint{198.965469pt}{197.820709pt}}
\pgflineto{\pgfpoint{197.969498pt}{140.777435pt}}
\pgfpathclose
\pgfusepath{fill,stroke}
\pgfpathmoveto{\pgfpoint{199.961456pt}{140.777435pt}}
\pgflineto{\pgfpoint{198.965469pt}{197.820709pt}}
\pgflineto{\pgfpoint{198.965469pt}{140.777435pt}}
\pgfpathclose
\pgfusepath{fill,stroke}
\pgfpathmoveto{\pgfpoint{186.017685pt}{140.777435pt}}
\pgflineto{\pgfpoint{187.013672pt}{140.777435pt}}
\pgflineto{\pgfpoint{187.013672pt}{141.098480pt}}
\pgfpathclose
\pgfusepath{fill,stroke}
\pgfpathmoveto{\pgfpoint{188.009659pt}{140.777435pt}}
\pgflineto{\pgfpoint{187.013672pt}{141.098480pt}}
\pgflineto{\pgfpoint{187.013672pt}{140.777435pt}}
\pgfpathclose
\pgfusepath{fill,stroke}
\pgfpathmoveto{\pgfpoint{182.033752pt}{140.777435pt}}
\pgflineto{\pgfpoint{183.029724pt}{140.777435pt}}
\pgflineto{\pgfpoint{183.029724pt}{142.399628pt}}
\pgfpathclose
\pgfusepath{fill,stroke}
\pgfpathmoveto{\pgfpoint{184.025711pt}{140.777435pt}}
\pgflineto{\pgfpoint{183.029724pt}{142.399628pt}}
\pgflineto{\pgfpoint{183.029724pt}{140.777435pt}}
\pgfpathclose
\pgfusepath{fill,stroke}
\pgfpathmoveto{\pgfpoint{176.057846pt}{140.777435pt}}
\pgflineto{\pgfpoint{177.053818pt}{140.777435pt}}
\pgflineto{\pgfpoint{177.053818pt}{144.404053pt}}
\pgfpathclose
\pgfusepath{fill,stroke}
\pgfpathmoveto{\pgfpoint{178.049805pt}{144.353317pt}}
\pgflineto{\pgfpoint{177.053818pt}{144.404053pt}}
\pgflineto{\pgfpoint{177.053818pt}{140.777435pt}}
\pgfpathclose
\pgfusepath{fill,stroke}
\pgfpathmoveto{\pgfpoint{178.049805pt}{140.777435pt}}
\pgflineto{\pgfpoint{178.049805pt}{144.353317pt}}
\pgflineto{\pgfpoint{177.053818pt}{140.777435pt}}
\pgfpathclose
\pgfusepath{fill,stroke}
\pgfpathmoveto{\pgfpoint{179.045792pt}{140.963043pt}}
\pgflineto{\pgfpoint{178.049805pt}{144.353317pt}}
\pgflineto{\pgfpoint{178.049805pt}{140.777435pt}}
\pgfpathclose
\pgfusepath{fill,stroke}
\pgfpathmoveto{\pgfpoint{179.045792pt}{140.777435pt}}
\pgflineto{\pgfpoint{179.045792pt}{140.963043pt}}
\pgflineto{\pgfpoint{178.049805pt}{140.777435pt}}
\pgfpathclose
\pgfusepath{fill,stroke}
\pgfpathmoveto{\pgfpoint{180.041779pt}{140.930664pt}}
\pgflineto{\pgfpoint{179.045792pt}{140.963043pt}}
\pgflineto{\pgfpoint{179.045792pt}{140.777435pt}}
\pgfpathclose
\pgfusepath{fill,stroke}
\pgfpathmoveto{\pgfpoint{180.041779pt}{140.777435pt}}
\pgflineto{\pgfpoint{180.041779pt}{140.930664pt}}
\pgflineto{\pgfpoint{179.045792pt}{140.777435pt}}
\pgfpathclose
\pgfusepath{fill,stroke}
\pgfpathmoveto{\pgfpoint{181.037766pt}{141.619095pt}}
\pgflineto{\pgfpoint{180.041779pt}{140.930664pt}}
\pgflineto{\pgfpoint{180.041779pt}{140.777435pt}}
\pgfpathclose
\pgfusepath{fill,stroke}
\pgfpathmoveto{\pgfpoint{181.037766pt}{140.777435pt}}
\pgflineto{\pgfpoint{181.037766pt}{141.619095pt}}
\pgflineto{\pgfpoint{180.041779pt}{140.777435pt}}
\pgfpathclose
\pgfusepath{fill,stroke}
\pgfpathmoveto{\pgfpoint{182.033752pt}{140.777435pt}}
\pgflineto{\pgfpoint{181.037766pt}{141.619095pt}}
\pgflineto{\pgfpoint{181.037766pt}{140.777435pt}}
\pgfpathclose
\pgfusepath{fill,stroke}
\pgfpathmoveto{\pgfpoint{169.085953pt}{140.777435pt}}
\pgflineto{\pgfpoint{170.081940pt}{140.777435pt}}
\pgflineto{\pgfpoint{170.081940pt}{146.720169pt}}
\pgfpathclose
\pgfusepath{fill,stroke}
\pgfpathmoveto{\pgfpoint{171.077911pt}{145.686310pt}}
\pgflineto{\pgfpoint{170.081940pt}{146.720169pt}}
\pgflineto{\pgfpoint{170.081940pt}{140.777435pt}}
\pgfpathclose
\pgfusepath{fill,stroke}
\pgfpathmoveto{\pgfpoint{171.077911pt}{140.777435pt}}
\pgflineto{\pgfpoint{171.077911pt}{145.686310pt}}
\pgflineto{\pgfpoint{170.081940pt}{140.777435pt}}
\pgfpathclose
\pgfusepath{fill,stroke}
\pgfpathmoveto{\pgfpoint{172.073914pt}{149.736984pt}}
\pgflineto{\pgfpoint{171.077911pt}{145.686310pt}}
\pgflineto{\pgfpoint{171.077911pt}{140.777435pt}}
\pgfpathclose
\pgfusepath{fill,stroke}
\pgfpathmoveto{\pgfpoint{172.073914pt}{140.777435pt}}
\pgflineto{\pgfpoint{172.073914pt}{149.736984pt}}
\pgflineto{\pgfpoint{171.077911pt}{140.777435pt}}
\pgfpathclose
\pgfusepath{fill,stroke}
\pgfpathmoveto{\pgfpoint{173.069885pt}{148.531342pt}}
\pgflineto{\pgfpoint{172.073914pt}{149.736984pt}}
\pgflineto{\pgfpoint{172.073914pt}{140.777435pt}}
\pgfpathclose
\pgfusepath{fill,stroke}
\pgfpathmoveto{\pgfpoint{173.069885pt}{140.777435pt}}
\pgflineto{\pgfpoint{173.069885pt}{148.531342pt}}
\pgflineto{\pgfpoint{172.073914pt}{140.777435pt}}
\pgfpathclose
\pgfusepath{fill,stroke}
\pgfpathmoveto{\pgfpoint{174.065872pt}{141.515808pt}}
\pgflineto{\pgfpoint{173.069885pt}{148.531342pt}}
\pgflineto{\pgfpoint{173.069885pt}{140.777435pt}}
\pgfpathclose
\pgfusepath{fill,stroke}
\pgfpathmoveto{\pgfpoint{174.065872pt}{140.777435pt}}
\pgflineto{\pgfpoint{174.065872pt}{141.515808pt}}
\pgflineto{\pgfpoint{173.069885pt}{140.777435pt}}
\pgfpathclose
\pgfusepath{fill,stroke}
\pgfpathmoveto{\pgfpoint{175.061859pt}{141.323135pt}}
\pgflineto{\pgfpoint{174.065872pt}{141.515808pt}}
\pgflineto{\pgfpoint{174.065872pt}{140.777435pt}}
\pgfpathclose
\pgfusepath{fill,stroke}
\pgfpathmoveto{\pgfpoint{175.061859pt}{140.777435pt}}
\pgflineto{\pgfpoint{175.061859pt}{141.323135pt}}
\pgflineto{\pgfpoint{174.065872pt}{140.777435pt}}
\pgfpathclose
\pgfusepath{fill,stroke}
\pgfpathmoveto{\pgfpoint{176.057846pt}{140.777435pt}}
\pgflineto{\pgfpoint{175.061859pt}{141.323135pt}}
\pgflineto{\pgfpoint{175.061859pt}{140.777435pt}}
\pgfpathclose
\pgfusepath{fill,stroke}
\pgfpathmoveto{\pgfpoint{166.098007pt}{140.777435pt}}
\pgflineto{\pgfpoint{167.093994pt}{140.777435pt}}
\pgflineto{\pgfpoint{167.093994pt}{141.434799pt}}
\pgfpathclose
\pgfusepath{fill,stroke}
\pgfpathmoveto{\pgfpoint{167.417542pt}{205.642242pt}}
\pgflineto{\pgfpoint{167.093994pt}{141.434799pt}}
\pgflineto{\pgfpoint{167.093994pt}{140.777435pt}}
\pgfpathclose
\pgfusepath{fill,stroke}
\pgfpathmoveto{\pgfpoint{167.417542pt}{205.642242pt}}
\pgflineto{\pgfpoint{167.415314pt}{205.642242pt}}
\pgflineto{\pgfpoint{167.093994pt}{141.434799pt}}
\pgfpathclose
\pgfusepath{fill,stroke}
\pgfpathmoveto{\pgfpoint{168.089966pt}{140.777435pt}}
\pgflineto{\pgfpoint{167.417542pt}{205.642242pt}}
\pgflineto{\pgfpoint{167.093994pt}{140.777435pt}}
\pgfpathclose
\pgfusepath{fill,stroke}
\pgfpathmoveto{\pgfpoint{168.089966pt}{140.777435pt}}
\pgflineto{\pgfpoint{168.089966pt}{205.642242pt}}
\pgflineto{\pgfpoint{167.417542pt}{205.642242pt}}
\pgfpathclose
\pgfusepath{fill,stroke}
\pgfpathmoveto{\pgfpoint{169.085953pt}{140.777435pt}}
\pgflineto{\pgfpoint{168.089966pt}{205.642242pt}}
\pgflineto{\pgfpoint{168.089966pt}{140.777435pt}}
\pgfpathclose
\pgfusepath{fill,stroke}
\pgfpathmoveto{\pgfpoint{169.085953pt}{140.777435pt}}
\pgflineto{\pgfpoint{168.762405pt}{205.642242pt}}
\pgflineto{\pgfpoint{168.089966pt}{205.642242pt}}
\pgfpathclose
\pgfusepath{fill,stroke}
\pgfpathmoveto{\pgfpoint{163.110062pt}{140.777435pt}}
\pgflineto{\pgfpoint{164.106033pt}{140.777435pt}}
\pgflineto{\pgfpoint{164.106033pt}{194.127029pt}}
\pgfpathclose
\pgfusepath{fill,stroke}
\pgfpathmoveto{\pgfpoint{165.102020pt}{140.828079pt}}
\pgflineto{\pgfpoint{164.106033pt}{194.127029pt}}
\pgflineto{\pgfpoint{164.106033pt}{140.777435pt}}
\pgfpathclose
\pgfusepath{fill,stroke}
\pgfpathmoveto{\pgfpoint{165.102020pt}{140.777435pt}}
\pgflineto{\pgfpoint{165.102020pt}{140.828079pt}}
\pgflineto{\pgfpoint{164.106033pt}{140.777435pt}}
\pgfpathclose
\pgfusepath{fill,stroke}
\pgfpathmoveto{\pgfpoint{166.098007pt}{140.777435pt}}
\pgflineto{\pgfpoint{165.102020pt}{140.828079pt}}
\pgflineto{\pgfpoint{165.102020pt}{140.777435pt}}
\pgfpathclose
\pgfusepath{fill,stroke}
\pgfpathmoveto{\pgfpoint{159.126114pt}{140.777435pt}}
\pgflineto{\pgfpoint{160.122101pt}{140.777435pt}}
\pgflineto{\pgfpoint{160.122101pt}{158.193100pt}}
\pgfpathclose
\pgfusepath{fill,stroke}
\pgfpathmoveto{\pgfpoint{161.118088pt}{140.795273pt}}
\pgflineto{\pgfpoint{160.122101pt}{158.193100pt}}
\pgflineto{\pgfpoint{160.122101pt}{140.777435pt}}
\pgfpathclose
\pgfusepath{fill,stroke}
\pgfpathmoveto{\pgfpoint{161.118088pt}{140.777435pt}}
\pgflineto{\pgfpoint{161.118088pt}{140.795273pt}}
\pgflineto{\pgfpoint{160.122101pt}{140.777435pt}}
\pgfpathclose
\pgfusepath{fill,stroke}
\pgfpathmoveto{\pgfpoint{162.114075pt}{141.404877pt}}
\pgflineto{\pgfpoint{161.118088pt}{140.795273pt}}
\pgflineto{\pgfpoint{161.118088pt}{140.777435pt}}
\pgfpathclose
\pgfusepath{fill,stroke}
\pgfpathmoveto{\pgfpoint{162.114075pt}{140.777435pt}}
\pgflineto{\pgfpoint{162.114075pt}{141.404877pt}}
\pgflineto{\pgfpoint{161.118088pt}{140.777435pt}}
\pgfpathclose
\pgfusepath{fill,stroke}
\pgfpathmoveto{\pgfpoint{163.110062pt}{140.777435pt}}
\pgflineto{\pgfpoint{162.114075pt}{141.404877pt}}
\pgflineto{\pgfpoint{162.114075pt}{140.777435pt}}
\pgfpathclose
\pgfusepath{fill,stroke}
\pgfpathmoveto{\pgfpoint{156.138168pt}{140.777435pt}}
\pgflineto{\pgfpoint{157.134155pt}{140.777435pt}}
\pgflineto{\pgfpoint{157.134155pt}{140.926575pt}}
\pgfpathclose
\pgfusepath{fill,stroke}
\pgfpathmoveto{\pgfpoint{158.130127pt}{140.979492pt}}
\pgflineto{\pgfpoint{157.134155pt}{140.926575pt}}
\pgflineto{\pgfpoint{157.134155pt}{140.777435pt}}
\pgfpathclose
\pgfusepath{fill,stroke}
\pgfpathmoveto{\pgfpoint{158.130127pt}{140.777435pt}}
\pgflineto{\pgfpoint{158.130127pt}{140.979492pt}}
\pgflineto{\pgfpoint{157.134155pt}{140.777435pt}}
\pgfpathclose
\pgfusepath{fill,stroke}
\pgfpathmoveto{\pgfpoint{159.126114pt}{140.777435pt}}
\pgflineto{\pgfpoint{158.130127pt}{140.979492pt}}
\pgflineto{\pgfpoint{158.130127pt}{140.777435pt}}
\pgfpathclose
\pgfusepath{fill,stroke}
\pgfpathmoveto{\pgfpoint{154.146194pt}{140.777435pt}}
\pgflineto{\pgfpoint{155.142181pt}{140.777435pt}}
\pgflineto{\pgfpoint{155.142181pt}{195.815277pt}}
\pgfpathclose
\pgfusepath{fill,stroke}
\pgfpathmoveto{\pgfpoint{156.138168pt}{140.777435pt}}
\pgflineto{\pgfpoint{155.142181pt}{195.815277pt}}
\pgflineto{\pgfpoint{155.142181pt}{140.777435pt}}
\pgfpathclose
\pgfusepath{fill,stroke}
\pgfpathmoveto{\pgfpoint{150.162262pt}{140.777435pt}}
\pgflineto{\pgfpoint{151.158249pt}{140.777435pt}}
\pgflineto{\pgfpoint{151.158249pt}{141.561523pt}}
\pgfpathclose
\pgfusepath{fill,stroke}
\pgfpathmoveto{\pgfpoint{152.154221pt}{144.148651pt}}
\pgflineto{\pgfpoint{151.158249pt}{141.561523pt}}
\pgflineto{\pgfpoint{151.158249pt}{140.777435pt}}
\pgfpathclose
\pgfusepath{fill,stroke}
\pgfpathmoveto{\pgfpoint{152.154221pt}{140.777435pt}}
\pgflineto{\pgfpoint{152.154221pt}{144.148651pt}}
\pgflineto{\pgfpoint{151.158249pt}{140.777435pt}}
\pgfpathclose
\pgfusepath{fill,stroke}
\pgfpathmoveto{\pgfpoint{153.150208pt}{143.520859pt}}
\pgflineto{\pgfpoint{152.154221pt}{144.148651pt}}
\pgflineto{\pgfpoint{152.154221pt}{140.777435pt}}
\pgfpathclose
\pgfusepath{fill,stroke}
\pgfpathmoveto{\pgfpoint{153.150208pt}{140.777435pt}}
\pgflineto{\pgfpoint{153.150208pt}{143.520859pt}}
\pgflineto{\pgfpoint{152.154221pt}{140.777435pt}}
\pgfpathclose
\pgfusepath{fill,stroke}
\pgfpathmoveto{\pgfpoint{154.146194pt}{140.777435pt}}
\pgflineto{\pgfpoint{153.150208pt}{143.520859pt}}
\pgflineto{\pgfpoint{153.150208pt}{140.777435pt}}
\pgfpathclose
\pgfusepath{fill,stroke}
\pgfpathmoveto{\pgfpoint{144.186356pt}{140.777435pt}}
\pgflineto{\pgfpoint{145.182343pt}{140.777435pt}}
\pgflineto{\pgfpoint{145.182343pt}{141.507751pt}}
\pgfpathclose
\pgfusepath{fill,stroke}
\pgfpathmoveto{\pgfpoint{146.178314pt}{141.062988pt}}
\pgflineto{\pgfpoint{145.182343pt}{141.507751pt}}
\pgflineto{\pgfpoint{145.182343pt}{140.777435pt}}
\pgfpathclose
\pgfusepath{fill,stroke}
\pgfpathmoveto{\pgfpoint{146.178314pt}{140.777435pt}}
\pgflineto{\pgfpoint{146.178314pt}{141.062988pt}}
\pgflineto{\pgfpoint{145.182343pt}{140.777435pt}}
\pgfpathclose
\pgfusepath{fill,stroke}
\pgfpathmoveto{\pgfpoint{147.174316pt}{143.409790pt}}
\pgflineto{\pgfpoint{146.178314pt}{141.062988pt}}
\pgflineto{\pgfpoint{146.178314pt}{140.777435pt}}
\pgfpathclose
\pgfusepath{fill,stroke}
\pgfpathmoveto{\pgfpoint{147.174316pt}{140.777435pt}}
\pgflineto{\pgfpoint{147.174316pt}{143.409790pt}}
\pgflineto{\pgfpoint{146.178314pt}{140.777435pt}}
\pgfpathclose
\pgfusepath{fill,stroke}
\pgfpathmoveto{\pgfpoint{148.170288pt}{140.777435pt}}
\pgflineto{\pgfpoint{147.174316pt}{143.409790pt}}
\pgflineto{\pgfpoint{147.174316pt}{140.777435pt}}
\pgfpathclose
\pgfusepath{fill,stroke}
\pgfpathmoveto{\pgfpoint{142.194382pt}{140.777435pt}}
\pgflineto{\pgfpoint{143.190369pt}{140.777435pt}}
\pgflineto{\pgfpoint{143.190369pt}{141.972321pt}}
\pgfpathclose
\pgfusepath{fill,stroke}
\pgfpathmoveto{\pgfpoint{144.186356pt}{140.777435pt}}
\pgflineto{\pgfpoint{143.190369pt}{141.972321pt}}
\pgflineto{\pgfpoint{143.190369pt}{140.777435pt}}
\pgfpathclose
\pgfusepath{fill,stroke}
\pgfpathmoveto{\pgfpoint{136.218475pt}{140.777435pt}}
\pgflineto{\pgfpoint{137.214478pt}{140.777435pt}}
\pgflineto{\pgfpoint{137.214478pt}{144.493912pt}}
\pgfpathclose
\pgfusepath{fill,stroke}
\pgfpathmoveto{\pgfpoint{138.210449pt}{140.824677pt}}
\pgflineto{\pgfpoint{137.214478pt}{144.493912pt}}
\pgflineto{\pgfpoint{137.214478pt}{140.777435pt}}
\pgfpathclose
\pgfusepath{fill,stroke}
\pgfpathmoveto{\pgfpoint{138.210449pt}{140.777435pt}}
\pgflineto{\pgfpoint{138.210449pt}{140.824677pt}}
\pgflineto{\pgfpoint{137.214478pt}{140.777435pt}}
\pgfpathclose
\pgfusepath{fill,stroke}
\pgfpathmoveto{\pgfpoint{139.206436pt}{155.555008pt}}
\pgflineto{\pgfpoint{138.210449pt}{140.824677pt}}
\pgflineto{\pgfpoint{138.210449pt}{140.777435pt}}
\pgfpathclose
\pgfusepath{fill,stroke}
\pgfpathmoveto{\pgfpoint{139.206436pt}{140.777435pt}}
\pgflineto{\pgfpoint{139.206436pt}{155.555008pt}}
\pgflineto{\pgfpoint{138.210449pt}{140.777435pt}}
\pgfpathclose
\pgfusepath{fill,stroke}
\pgfpathmoveto{\pgfpoint{140.202423pt}{141.880554pt}}
\pgflineto{\pgfpoint{139.206436pt}{155.555008pt}}
\pgflineto{\pgfpoint{139.206436pt}{140.777435pt}}
\pgfpathclose
\pgfusepath{fill,stroke}
\pgfpathmoveto{\pgfpoint{140.202423pt}{140.777435pt}}
\pgflineto{\pgfpoint{140.202423pt}{141.880554pt}}
\pgflineto{\pgfpoint{139.206436pt}{140.777435pt}}
\pgfpathclose
\pgfusepath{fill,stroke}
\pgfpathmoveto{\pgfpoint{141.198410pt}{141.096649pt}}
\pgflineto{\pgfpoint{140.202423pt}{141.880554pt}}
\pgflineto{\pgfpoint{140.202423pt}{140.777435pt}}
\pgfpathclose
\pgfusepath{fill,stroke}
\pgfpathmoveto{\pgfpoint{141.198410pt}{140.777435pt}}
\pgflineto{\pgfpoint{141.198410pt}{141.096649pt}}
\pgflineto{\pgfpoint{140.202423pt}{140.777435pt}}
\pgfpathclose
\pgfusepath{fill,stroke}
\pgfpathmoveto{\pgfpoint{142.194382pt}{140.777435pt}}
\pgflineto{\pgfpoint{141.198410pt}{141.096649pt}}
\pgflineto{\pgfpoint{141.198410pt}{140.777435pt}}
\pgfpathclose
\pgfusepath{fill,stroke}
\pgfpathmoveto{\pgfpoint{134.226517pt}{140.777435pt}}
\pgflineto{\pgfpoint{135.222504pt}{140.777435pt}}
\pgflineto{\pgfpoint{135.222504pt}{141.478424pt}}
\pgfpathclose
\pgfusepath{fill,stroke}
\pgfpathmoveto{\pgfpoint{136.218475pt}{140.777435pt}}
\pgflineto{\pgfpoint{135.222504pt}{141.478424pt}}
\pgflineto{\pgfpoint{135.222504pt}{140.777435pt}}
\pgfpathclose
\pgfusepath{fill,stroke}
\pgfpathmoveto{\pgfpoint{128.250610pt}{140.777435pt}}
\pgflineto{\pgfpoint{129.246597pt}{140.777435pt}}
\pgflineto{\pgfpoint{129.246597pt}{146.889862pt}}
\pgfpathclose
\pgfusepath{fill,stroke}
\pgfpathmoveto{\pgfpoint{130.242584pt}{141.305695pt}}
\pgflineto{\pgfpoint{129.246597pt}{146.889862pt}}
\pgflineto{\pgfpoint{129.246597pt}{140.777435pt}}
\pgfpathclose
\pgfusepath{fill,stroke}
\pgfpathmoveto{\pgfpoint{130.242584pt}{140.777435pt}}
\pgflineto{\pgfpoint{130.242584pt}{141.305695pt}}
\pgflineto{\pgfpoint{129.246597pt}{140.777435pt}}
\pgfpathclose
\pgfusepath{fill,stroke}
\pgfpathmoveto{\pgfpoint{131.238571pt}{142.924377pt}}
\pgflineto{\pgfpoint{130.242584pt}{141.305695pt}}
\pgflineto{\pgfpoint{130.242584pt}{140.777435pt}}
\pgfpathclose
\pgfusepath{fill,stroke}
\pgfpathmoveto{\pgfpoint{131.238571pt}{140.777435pt}}
\pgflineto{\pgfpoint{131.238571pt}{142.924377pt}}
\pgflineto{\pgfpoint{130.242584pt}{140.777435pt}}
\pgfpathclose
\pgfusepath{fill,stroke}
\pgfpathmoveto{\pgfpoint{132.234558pt}{140.784973pt}}
\pgflineto{\pgfpoint{131.238571pt}{142.924377pt}}
\pgflineto{\pgfpoint{131.238571pt}{140.777435pt}}
\pgfpathclose
\pgfusepath{fill,stroke}
\pgfpathmoveto{\pgfpoint{132.234558pt}{140.777435pt}}
\pgflineto{\pgfpoint{132.234558pt}{140.784973pt}}
\pgflineto{\pgfpoint{131.238571pt}{140.777435pt}}
\pgfpathclose
\pgfusepath{fill,stroke}
\pgfpathmoveto{\pgfpoint{133.230530pt}{140.777435pt}}
\pgflineto{\pgfpoint{132.234558pt}{140.784973pt}}
\pgflineto{\pgfpoint{132.234558pt}{140.777435pt}}
\pgfpathclose
\pgfusepath{fill,stroke}
\pgfpathmoveto{\pgfpoint{125.262665pt}{140.777435pt}}
\pgflineto{\pgfpoint{126.258652pt}{140.777435pt}}
\pgflineto{\pgfpoint{126.258652pt}{141.187393pt}}
\pgfpathclose
\pgfusepath{fill,stroke}
\pgfpathmoveto{\pgfpoint{127.254631pt}{158.624695pt}}
\pgflineto{\pgfpoint{126.258652pt}{141.187393pt}}
\pgflineto{\pgfpoint{126.258652pt}{140.777435pt}}
\pgfpathclose
\pgfusepath{fill,stroke}
\pgfpathmoveto{\pgfpoint{127.254631pt}{140.777435pt}}
\pgflineto{\pgfpoint{127.254631pt}{158.624695pt}}
\pgflineto{\pgfpoint{126.258652pt}{140.777435pt}}
\pgfpathclose
\pgfusepath{fill,stroke}
\pgfpathmoveto{\pgfpoint{128.250610pt}{140.777435pt}}
\pgflineto{\pgfpoint{127.254631pt}{158.624695pt}}
\pgflineto{\pgfpoint{127.254631pt}{140.777435pt}}
\pgfpathclose
\pgfusepath{fill,stroke}
\pgfpathmoveto{\pgfpoint{122.274712pt}{140.777435pt}}
\pgflineto{\pgfpoint{123.270691pt}{140.777435pt}}
\pgflineto{\pgfpoint{123.270691pt}{141.045074pt}}
\pgfpathclose
\pgfusepath{fill,stroke}
\pgfpathmoveto{\pgfpoint{124.266678pt}{140.777435pt}}
\pgflineto{\pgfpoint{123.270691pt}{141.045074pt}}
\pgflineto{\pgfpoint{123.270691pt}{140.777435pt}}
\pgfpathclose
\pgfusepath{fill,stroke}
\pgfpathmoveto{\pgfpoint{120.282745pt}{140.777435pt}}
\pgflineto{\pgfpoint{121.278725pt}{140.777435pt}}
\pgflineto{\pgfpoint{121.278725pt}{141.992584pt}}
\pgfpathclose
\pgfusepath{fill,stroke}
\pgfpathmoveto{\pgfpoint{122.274712pt}{140.777435pt}}
\pgflineto{\pgfpoint{121.278725pt}{141.992584pt}}
\pgflineto{\pgfpoint{121.278725pt}{140.777435pt}}
\pgfpathclose
\pgfusepath{fill,stroke}
\pgfpathmoveto{\pgfpoint{115.302826pt}{140.777435pt}}
\pgflineto{\pgfpoint{116.298813pt}{140.777435pt}}
\pgflineto{\pgfpoint{116.298813pt}{141.607086pt}}
\pgfpathclose
\pgfusepath{fill,stroke}
\pgfpathmoveto{\pgfpoint{117.294792pt}{143.559708pt}}
\pgflineto{\pgfpoint{116.298813pt}{141.607086pt}}
\pgflineto{\pgfpoint{116.298813pt}{140.777435pt}}
\pgfpathclose
\pgfusepath{fill,stroke}
\pgfpathmoveto{\pgfpoint{117.294792pt}{140.777435pt}}
\pgflineto{\pgfpoint{117.294792pt}{143.559708pt}}
\pgflineto{\pgfpoint{116.298813pt}{140.777435pt}}
\pgfpathclose
\pgfusepath{fill,stroke}
\pgfpathmoveto{\pgfpoint{118.290779pt}{140.848618pt}}
\pgflineto{\pgfpoint{117.294792pt}{143.559708pt}}
\pgflineto{\pgfpoint{117.294792pt}{140.777435pt}}
\pgfpathclose
\pgfusepath{fill,stroke}
\pgfpathmoveto{\pgfpoint{118.290779pt}{140.777435pt}}
\pgflineto{\pgfpoint{118.290779pt}{140.848618pt}}
\pgflineto{\pgfpoint{117.294792pt}{140.777435pt}}
\pgfpathclose
\pgfusepath{fill,stroke}
\pgfpathmoveto{\pgfpoint{119.286758pt}{140.997116pt}}
\pgflineto{\pgfpoint{118.290779pt}{140.848618pt}}
\pgflineto{\pgfpoint{118.290779pt}{140.777435pt}}
\pgfpathclose
\pgfusepath{fill,stroke}
\pgfpathmoveto{\pgfpoint{119.286758pt}{140.777435pt}}
\pgflineto{\pgfpoint{119.286758pt}{140.997116pt}}
\pgflineto{\pgfpoint{118.290779pt}{140.777435pt}}
\pgfpathclose
\pgfusepath{fill,stroke}
\pgfpathmoveto{\pgfpoint{120.282745pt}{140.777435pt}}
\pgflineto{\pgfpoint{119.286758pt}{140.997116pt}}
\pgflineto{\pgfpoint{119.286758pt}{140.777435pt}}
\pgfpathclose
\pgfusepath{fill,stroke}
\pgfpathmoveto{\pgfpoint{111.318893pt}{140.777435pt}}
\pgflineto{\pgfpoint{112.314873pt}{140.777435pt}}
\pgflineto{\pgfpoint{112.314873pt}{142.841614pt}}
\pgfpathclose
\pgfusepath{fill,stroke}
\pgfpathmoveto{\pgfpoint{113.310852pt}{141.335541pt}}
\pgflineto{\pgfpoint{112.314873pt}{142.841614pt}}
\pgflineto{\pgfpoint{112.314873pt}{140.777435pt}}
\pgfpathclose
\pgfusepath{fill,stroke}
\pgfpathmoveto{\pgfpoint{113.310852pt}{140.777435pt}}
\pgflineto{\pgfpoint{113.310852pt}{141.335541pt}}
\pgflineto{\pgfpoint{112.314873pt}{140.777435pt}}
\pgfpathclose
\pgfusepath{fill,stroke}
\pgfpathmoveto{\pgfpoint{114.306839pt}{143.259003pt}}
\pgflineto{\pgfpoint{113.310852pt}{141.335541pt}}
\pgflineto{\pgfpoint{113.310852pt}{140.777435pt}}
\pgfpathclose
\pgfusepath{fill,stroke}
\pgfpathmoveto{\pgfpoint{114.306839pt}{140.777435pt}}
\pgflineto{\pgfpoint{114.306839pt}{143.259003pt}}
\pgflineto{\pgfpoint{113.310852pt}{140.777435pt}}
\pgfpathclose
\pgfusepath{fill,stroke}
\pgfpathmoveto{\pgfpoint{115.302826pt}{140.777435pt}}
\pgflineto{\pgfpoint{114.306839pt}{143.259003pt}}
\pgflineto{\pgfpoint{114.306839pt}{140.777435pt}}
\pgfpathclose
\pgfusepath{fill,stroke}
\pgfpathmoveto{\pgfpoint{107.334953pt}{140.777435pt}}
\pgflineto{\pgfpoint{108.330933pt}{140.777435pt}}
\pgflineto{\pgfpoint{108.330933pt}{156.427597pt}}
\pgfpathclose
\pgfusepath{fill,stroke}
\pgfpathmoveto{\pgfpoint{109.326920pt}{140.854156pt}}
\pgflineto{\pgfpoint{108.330933pt}{156.427597pt}}
\pgflineto{\pgfpoint{108.330933pt}{140.777435pt}}
\pgfpathclose
\pgfusepath{fill,stroke}
\pgfpathmoveto{\pgfpoint{109.326920pt}{140.777435pt}}
\pgflineto{\pgfpoint{109.326920pt}{140.854156pt}}
\pgflineto{\pgfpoint{108.330933pt}{140.777435pt}}
\pgfpathclose
\pgfusepath{fill,stroke}
\pgfpathmoveto{\pgfpoint{110.322906pt}{143.840607pt}}
\pgflineto{\pgfpoint{109.326920pt}{140.854156pt}}
\pgflineto{\pgfpoint{109.326920pt}{140.777435pt}}
\pgfpathclose
\pgfusepath{fill,stroke}
\pgfpathmoveto{\pgfpoint{110.322906pt}{140.777435pt}}
\pgflineto{\pgfpoint{110.322906pt}{143.840607pt}}
\pgflineto{\pgfpoint{109.326920pt}{140.777435pt}}
\pgfpathclose
\pgfusepath{fill,stroke}
\pgfpathmoveto{\pgfpoint{111.318893pt}{140.777435pt}}
\pgflineto{\pgfpoint{110.322906pt}{143.840607pt}}
\pgflineto{\pgfpoint{110.322906pt}{140.777435pt}}
\pgfpathclose
\pgfusepath{fill,stroke}
\pgfpathmoveto{\pgfpoint{103.351013pt}{140.777435pt}}
\pgflineto{\pgfpoint{104.347000pt}{140.777435pt}}
\pgflineto{\pgfpoint{104.347000pt}{152.852020pt}}
\pgfpathclose
\pgfusepath{fill,stroke}
\pgfpathmoveto{\pgfpoint{105.342987pt}{140.777435pt}}
\pgflineto{\pgfpoint{104.347000pt}{152.852020pt}}
\pgflineto{\pgfpoint{104.347000pt}{140.777435pt}}
\pgfpathclose
\pgfusepath{fill,stroke}
\pgfpathmoveto{\pgfpoint{94.387161pt}{140.777435pt}}
\pgflineto{\pgfpoint{95.383141pt}{140.777435pt}}
\pgflineto{\pgfpoint{95.383141pt}{141.040619pt}}
\pgfpathclose
\pgfusepath{fill,stroke}
\pgfpathmoveto{\pgfpoint{96.379128pt}{141.052612pt}}
\pgflineto{\pgfpoint{95.383141pt}{141.040619pt}}
\pgflineto{\pgfpoint{95.383141pt}{140.777435pt}}
\pgfpathclose
\pgfusepath{fill,stroke}
\pgfpathmoveto{\pgfpoint{96.379128pt}{140.777435pt}}
\pgflineto{\pgfpoint{96.379128pt}{141.052612pt}}
\pgflineto{\pgfpoint{95.383141pt}{140.777435pt}}
\pgfpathclose
\pgfusepath{fill,stroke}
\pgfpathmoveto{\pgfpoint{97.375107pt}{142.021271pt}}
\pgflineto{\pgfpoint{96.379128pt}{141.052612pt}}
\pgflineto{\pgfpoint{96.379128pt}{140.777435pt}}
\pgfpathclose
\pgfusepath{fill,stroke}
\pgfpathmoveto{\pgfpoint{97.375107pt}{140.777435pt}}
\pgflineto{\pgfpoint{97.375107pt}{142.021271pt}}
\pgflineto{\pgfpoint{96.379128pt}{140.777435pt}}
\pgfpathclose
\pgfusepath{fill,stroke}
\pgfpathmoveto{\pgfpoint{98.371094pt}{187.727631pt}}
\pgflineto{\pgfpoint{97.375107pt}{142.021271pt}}
\pgflineto{\pgfpoint{97.375107pt}{140.777435pt}}
\pgfpathclose
\pgfusepath{fill,stroke}
\pgfpathmoveto{\pgfpoint{98.371094pt}{140.777435pt}}
\pgflineto{\pgfpoint{98.371094pt}{187.727631pt}}
\pgflineto{\pgfpoint{97.375107pt}{140.777435pt}}
\pgfpathclose
\pgfusepath{fill,stroke}
\pgfpathmoveto{\pgfpoint{99.367081pt}{143.174622pt}}
\pgflineto{\pgfpoint{98.371094pt}{187.727631pt}}
\pgflineto{\pgfpoint{98.371094pt}{140.777435pt}}
\pgfpathclose
\pgfusepath{fill,stroke}
\pgfpathmoveto{\pgfpoint{99.367081pt}{140.777435pt}}
\pgflineto{\pgfpoint{99.367081pt}{143.174622pt}}
\pgflineto{\pgfpoint{98.371094pt}{140.777435pt}}
\pgfpathclose
\pgfusepath{fill,stroke}
\pgfpathmoveto{\pgfpoint{100.363068pt}{142.326782pt}}
\pgflineto{\pgfpoint{99.367081pt}{143.174622pt}}
\pgflineto{\pgfpoint{99.367081pt}{140.777435pt}}
\pgfpathclose
\pgfusepath{fill,stroke}
\pgfpathmoveto{\pgfpoint{100.363068pt}{140.777435pt}}
\pgflineto{\pgfpoint{100.363068pt}{142.326782pt}}
\pgflineto{\pgfpoint{99.367081pt}{140.777435pt}}
\pgfpathclose
\pgfusepath{fill,stroke}
\pgfpathmoveto{\pgfpoint{101.359047pt}{141.473938pt}}
\pgflineto{\pgfpoint{100.363068pt}{142.326782pt}}
\pgflineto{\pgfpoint{100.363068pt}{140.777435pt}}
\pgfpathclose
\pgfusepath{fill,stroke}
\pgfpathmoveto{\pgfpoint{101.359047pt}{140.777435pt}}
\pgflineto{\pgfpoint{101.359047pt}{141.473938pt}}
\pgflineto{\pgfpoint{100.363068pt}{140.777435pt}}
\pgfpathclose
\pgfusepath{fill,stroke}
\pgfpathmoveto{\pgfpoint{102.355034pt}{153.173050pt}}
\pgflineto{\pgfpoint{101.359047pt}{141.473938pt}}
\pgflineto{\pgfpoint{101.359047pt}{140.777435pt}}
\pgfpathclose
\pgfusepath{fill,stroke}
\pgfpathmoveto{\pgfpoint{102.355034pt}{140.777435pt}}
\pgflineto{\pgfpoint{102.355034pt}{153.173050pt}}
\pgflineto{\pgfpoint{101.359047pt}{140.777435pt}}
\pgfpathclose
\pgfusepath{fill,stroke}
\pgfpathmoveto{\pgfpoint{103.351013pt}{140.777435pt}}
\pgflineto{\pgfpoint{102.355034pt}{153.173050pt}}
\pgflineto{\pgfpoint{102.355034pt}{140.777435pt}}
\pgfpathclose
\pgfusepath{fill,stroke}
\pgfpathmoveto{\pgfpoint{91.399208pt}{140.777435pt}}
\pgflineto{\pgfpoint{92.395187pt}{140.777435pt}}
\pgflineto{\pgfpoint{92.395187pt}{140.789795pt}}
\pgfpathclose
\pgfusepath{fill,stroke}
\pgfpathmoveto{\pgfpoint{93.391174pt}{141.209564pt}}
\pgflineto{\pgfpoint{92.395187pt}{140.789795pt}}
\pgflineto{\pgfpoint{92.395187pt}{140.777435pt}}
\pgfpathclose
\pgfusepath{fill,stroke}
\pgfpathmoveto{\pgfpoint{93.391174pt}{140.777435pt}}
\pgflineto{\pgfpoint{93.391174pt}{141.209564pt}}
\pgflineto{\pgfpoint{92.395187pt}{140.777435pt}}
\pgfpathclose
\pgfusepath{fill,stroke}
\pgfpathmoveto{\pgfpoint{94.387161pt}{140.777435pt}}
\pgflineto{\pgfpoint{93.391174pt}{141.209564pt}}
\pgflineto{\pgfpoint{93.391174pt}{140.777435pt}}
\pgfpathclose
\pgfusepath{fill,stroke}
\pgfpathmoveto{\pgfpoint{86.419289pt}{140.777435pt}}
\pgflineto{\pgfpoint{87.415276pt}{140.777435pt}}
\pgflineto{\pgfpoint{87.415276pt}{152.008636pt}}
\pgfpathclose
\pgfusepath{fill,stroke}
\pgfpathmoveto{\pgfpoint{88.411255pt}{163.097900pt}}
\pgflineto{\pgfpoint{87.415276pt}{152.008636pt}}
\pgflineto{\pgfpoint{87.415276pt}{140.777435pt}}
\pgfpathclose
\pgfusepath{fill,stroke}
\pgfpathmoveto{\pgfpoint{88.411255pt}{140.777435pt}}
\pgflineto{\pgfpoint{88.411255pt}{163.097900pt}}
\pgflineto{\pgfpoint{87.415276pt}{140.777435pt}}
\pgfpathclose
\pgfusepath{fill,stroke}
\pgfpathmoveto{\pgfpoint{89.407242pt}{147.608887pt}}
\pgflineto{\pgfpoint{88.411255pt}{163.097900pt}}
\pgflineto{\pgfpoint{88.411255pt}{140.777435pt}}
\pgfpathclose
\pgfusepath{fill,stroke}
\pgfpathmoveto{\pgfpoint{89.407242pt}{140.777435pt}}
\pgflineto{\pgfpoint{89.407242pt}{147.608887pt}}
\pgflineto{\pgfpoint{88.411255pt}{140.777435pt}}
\pgfpathclose
\pgfusepath{fill,stroke}
\pgfpathmoveto{\pgfpoint{90.403221pt}{189.355881pt}}
\pgflineto{\pgfpoint{89.407242pt}{147.608887pt}}
\pgflineto{\pgfpoint{89.407242pt}{140.777435pt}}
\pgfpathclose
\pgfusepath{fill,stroke}
\pgfpathmoveto{\pgfpoint{90.403221pt}{140.777435pt}}
\pgflineto{\pgfpoint{90.403221pt}{189.355881pt}}
\pgflineto{\pgfpoint{89.407242pt}{140.777435pt}}
\pgfpathclose
\pgfusepath{fill,stroke}
\pgfpathmoveto{\pgfpoint{91.399208pt}{140.777435pt}}
\pgflineto{\pgfpoint{90.403221pt}{189.355881pt}}
\pgflineto{\pgfpoint{90.403221pt}{140.777435pt}}
\pgfpathclose
\pgfusepath{fill,stroke}
\pgfpathmoveto{\pgfpoint{78.451424pt}{140.777435pt}}
\pgflineto{\pgfpoint{79.447403pt}{140.777435pt}}
\pgflineto{\pgfpoint{79.447403pt}{164.721649pt}}
\pgfpathclose
\pgfusepath{fill,stroke}
\pgfpathmoveto{\pgfpoint{80.443390pt}{141.021835pt}}
\pgflineto{\pgfpoint{79.447403pt}{164.721649pt}}
\pgflineto{\pgfpoint{79.447403pt}{140.777435pt}}
\pgfpathclose
\pgfusepath{fill,stroke}
\pgfpathmoveto{\pgfpoint{80.443390pt}{140.777435pt}}
\pgflineto{\pgfpoint{80.443390pt}{141.021835pt}}
\pgflineto{\pgfpoint{79.447403pt}{140.777435pt}}
\pgfpathclose
\pgfusepath{fill,stroke}
\pgfpathmoveto{\pgfpoint{81.439369pt}{141.797729pt}}
\pgflineto{\pgfpoint{80.443390pt}{141.021835pt}}
\pgflineto{\pgfpoint{80.443390pt}{140.777435pt}}
\pgfpathclose
\pgfusepath{fill,stroke}
\pgfpathmoveto{\pgfpoint{81.439369pt}{140.777435pt}}
\pgflineto{\pgfpoint{81.439369pt}{141.797729pt}}
\pgflineto{\pgfpoint{80.443390pt}{140.777435pt}}
\pgfpathclose
\pgfusepath{fill,stroke}
\pgfpathmoveto{\pgfpoint{82.435356pt}{141.784378pt}}
\pgflineto{\pgfpoint{81.439369pt}{141.797729pt}}
\pgflineto{\pgfpoint{81.439369pt}{140.777435pt}}
\pgfpathclose
\pgfusepath{fill,stroke}
\pgfpathmoveto{\pgfpoint{82.435356pt}{140.777435pt}}
\pgflineto{\pgfpoint{82.435356pt}{141.784378pt}}
\pgflineto{\pgfpoint{81.439369pt}{140.777435pt}}
\pgfpathclose
\pgfusepath{fill,stroke}
\pgfpathmoveto{\pgfpoint{83.431335pt}{147.373245pt}}
\pgflineto{\pgfpoint{82.435356pt}{141.784378pt}}
\pgflineto{\pgfpoint{82.435356pt}{140.777435pt}}
\pgfpathclose
\pgfusepath{fill,stroke}
\pgfpathmoveto{\pgfpoint{83.431335pt}{140.777435pt}}
\pgflineto{\pgfpoint{83.431335pt}{147.373245pt}}
\pgflineto{\pgfpoint{82.435356pt}{140.777435pt}}
\pgfpathclose
\pgfusepath{fill,stroke}
\pgfpathmoveto{\pgfpoint{84.427322pt}{140.853500pt}}
\pgflineto{\pgfpoint{83.431335pt}{147.373245pt}}
\pgflineto{\pgfpoint{83.431335pt}{140.777435pt}}
\pgfpathclose
\pgfusepath{fill,stroke}
\pgfpathmoveto{\pgfpoint{84.427322pt}{140.777435pt}}
\pgflineto{\pgfpoint{84.427322pt}{140.853500pt}}
\pgflineto{\pgfpoint{83.431335pt}{140.777435pt}}
\pgfpathclose
\pgfusepath{fill,stroke}
\pgfpathmoveto{\pgfpoint{85.423309pt}{140.987396pt}}
\pgflineto{\pgfpoint{84.427322pt}{140.853500pt}}
\pgflineto{\pgfpoint{84.427322pt}{140.777435pt}}
\pgfpathclose
\pgfusepath{fill,stroke}
\pgfpathmoveto{\pgfpoint{85.423309pt}{140.777435pt}}
\pgflineto{\pgfpoint{85.423309pt}{140.987396pt}}
\pgflineto{\pgfpoint{84.427322pt}{140.777435pt}}
\pgfpathclose
\pgfusepath{fill,stroke}
\pgfpathmoveto{\pgfpoint{86.419289pt}{140.777435pt}}
\pgflineto{\pgfpoint{85.423309pt}{140.987396pt}}
\pgflineto{\pgfpoint{85.423309pt}{140.777435pt}}
\pgfpathclose
\pgfusepath{fill,stroke}
\pgfpathmoveto{\pgfpoint{73.471497pt}{140.777435pt}}
\pgflineto{\pgfpoint{74.467484pt}{140.777435pt}}
\pgflineto{\pgfpoint{74.467484pt}{141.606003pt}}
\pgfpathclose
\pgfusepath{fill,stroke}
\pgfpathmoveto{\pgfpoint{75.463470pt}{145.536850pt}}
\pgflineto{\pgfpoint{74.467484pt}{141.606003pt}}
\pgflineto{\pgfpoint{74.467484pt}{140.777435pt}}
\pgfpathclose
\pgfusepath{fill,stroke}
\pgfpathmoveto{\pgfpoint{75.463470pt}{140.777435pt}}
\pgflineto{\pgfpoint{75.463470pt}{145.536850pt}}
\pgflineto{\pgfpoint{74.467484pt}{140.777435pt}}
\pgfpathclose
\pgfusepath{fill,stroke}
\pgfpathmoveto{\pgfpoint{76.459442pt}{143.382019pt}}
\pgflineto{\pgfpoint{75.463470pt}{145.536850pt}}
\pgflineto{\pgfpoint{75.463470pt}{140.777435pt}}
\pgfpathclose
\pgfusepath{fill,stroke}
\pgfpathmoveto{\pgfpoint{76.459442pt}{140.777435pt}}
\pgflineto{\pgfpoint{76.459442pt}{143.382019pt}}
\pgflineto{\pgfpoint{75.463470pt}{140.777435pt}}
\pgfpathclose
\pgfusepath{fill,stroke}
\pgfpathmoveto{\pgfpoint{77.455437pt}{142.594910pt}}
\pgflineto{\pgfpoint{76.459442pt}{143.382019pt}}
\pgflineto{\pgfpoint{76.459442pt}{140.777435pt}}
\pgfpathclose
\pgfusepath{fill,stroke}
\pgfpathmoveto{\pgfpoint{77.455437pt}{140.777435pt}}
\pgflineto{\pgfpoint{77.455437pt}{142.594910pt}}
\pgflineto{\pgfpoint{76.459442pt}{140.777435pt}}
\pgfpathclose
\pgfusepath{fill,stroke}
\pgfpathmoveto{\pgfpoint{78.451424pt}{140.777435pt}}
\pgflineto{\pgfpoint{77.455437pt}{142.594910pt}}
\pgflineto{\pgfpoint{77.455437pt}{140.777435pt}}
\pgfpathclose
\pgfusepath{fill,stroke}
\pgfpathmoveto{\pgfpoint{68.491577pt}{140.777435pt}}
\pgflineto{\pgfpoint{69.487564pt}{140.777435pt}}
\pgflineto{\pgfpoint{69.487564pt}{149.515793pt}}
\pgfpathclose
\pgfusepath{fill,stroke}
\pgfpathmoveto{\pgfpoint{70.483551pt}{141.832703pt}}
\pgflineto{\pgfpoint{69.487564pt}{149.515793pt}}
\pgflineto{\pgfpoint{69.487564pt}{140.777435pt}}
\pgfpathclose
\pgfusepath{fill,stroke}
\pgfpathmoveto{\pgfpoint{70.483551pt}{140.777435pt}}
\pgflineto{\pgfpoint{70.483551pt}{141.832703pt}}
\pgflineto{\pgfpoint{69.487564pt}{140.777435pt}}
\pgfpathclose
\pgfusepath{fill,stroke}
\pgfpathmoveto{\pgfpoint{71.479530pt}{190.848999pt}}
\pgflineto{\pgfpoint{70.483551pt}{141.832703pt}}
\pgflineto{\pgfpoint{70.483551pt}{140.777435pt}}
\pgfpathclose
\pgfusepath{fill,stroke}
\pgfpathmoveto{\pgfpoint{71.479530pt}{140.777435pt}}
\pgflineto{\pgfpoint{71.479530pt}{190.848999pt}}
\pgflineto{\pgfpoint{70.483551pt}{140.777435pt}}
\pgfpathclose
\pgfusepath{fill,stroke}
\pgfpathmoveto{\pgfpoint{72.475510pt}{148.664902pt}}
\pgflineto{\pgfpoint{71.479530pt}{190.848999pt}}
\pgflineto{\pgfpoint{71.479530pt}{140.777435pt}}
\pgfpathclose
\pgfusepath{fill,stroke}
\pgfpathmoveto{\pgfpoint{72.475510pt}{140.777435pt}}
\pgflineto{\pgfpoint{72.475510pt}{148.664902pt}}
\pgflineto{\pgfpoint{71.479530pt}{140.777435pt}}
\pgfpathclose
\pgfusepath{fill,stroke}
\pgfpathmoveto{\pgfpoint{73.471497pt}{140.777435pt}}
\pgflineto{\pgfpoint{72.475510pt}{148.664902pt}}
\pgflineto{\pgfpoint{72.475510pt}{140.777435pt}}
\pgfpathclose
\pgfusepath{fill,stroke}
\pgfpathmoveto{\pgfpoint{50.563873pt}{140.777435pt}}
\pgflineto{\pgfpoint{51.559845pt}{140.777435pt}}
\pgflineto{\pgfpoint{51.559845pt}{146.765289pt}}
\pgfpathclose
\pgfusepath{fill,stroke}
\pgfpathmoveto{\pgfpoint{52.555840pt}{142.530624pt}}
\pgflineto{\pgfpoint{51.559845pt}{146.765289pt}}
\pgflineto{\pgfpoint{51.559845pt}{140.777435pt}}
\pgfpathclose
\pgfusepath{fill,stroke}
\pgfpathmoveto{\pgfpoint{52.555840pt}{140.777435pt}}
\pgflineto{\pgfpoint{52.555840pt}{142.530624pt}}
\pgflineto{\pgfpoint{51.559845pt}{140.777435pt}}
\pgfpathclose
\pgfusepath{fill,stroke}
\pgfpathmoveto{\pgfpoint{53.551819pt}{146.137848pt}}
\pgflineto{\pgfpoint{52.555840pt}{142.530624pt}}
\pgflineto{\pgfpoint{52.555840pt}{140.777435pt}}
\pgfpathclose
\pgfusepath{fill,stroke}
\pgfpathmoveto{\pgfpoint{53.551819pt}{140.777435pt}}
\pgflineto{\pgfpoint{53.551819pt}{146.137848pt}}
\pgflineto{\pgfpoint{52.555840pt}{140.777435pt}}
\pgfpathclose
\pgfusepath{fill,stroke}
\pgfpathmoveto{\pgfpoint{54.547806pt}{142.199554pt}}
\pgflineto{\pgfpoint{53.551819pt}{146.137848pt}}
\pgflineto{\pgfpoint{53.551819pt}{140.777435pt}}
\pgfpathclose
\pgfusepath{fill,stroke}
\pgfpathmoveto{\pgfpoint{54.547806pt}{140.777435pt}}
\pgflineto{\pgfpoint{54.547806pt}{142.199554pt}}
\pgflineto{\pgfpoint{53.551819pt}{140.777435pt}}
\pgfpathclose
\pgfusepath{fill,stroke}
\pgfpathmoveto{\pgfpoint{55.543785pt}{196.136078pt}}
\pgflineto{\pgfpoint{54.547806pt}{142.199554pt}}
\pgflineto{\pgfpoint{54.547806pt}{140.777435pt}}
\pgfpathclose
\pgfusepath{fill,stroke}
\pgfpathmoveto{\pgfpoint{55.543785pt}{140.777435pt}}
\pgflineto{\pgfpoint{55.543785pt}{196.136078pt}}
\pgflineto{\pgfpoint{54.547806pt}{140.777435pt}}
\pgfpathclose
\pgfusepath{fill,stroke}
\pgfpathmoveto{\pgfpoint{56.539772pt}{168.281158pt}}
\pgflineto{\pgfpoint{55.543785pt}{196.136078pt}}
\pgflineto{\pgfpoint{55.543785pt}{140.777435pt}}
\pgfpathclose
\pgfusepath{fill,stroke}
\pgfpathmoveto{\pgfpoint{56.539772pt}{140.777435pt}}
\pgflineto{\pgfpoint{56.539772pt}{168.281158pt}}
\pgflineto{\pgfpoint{55.543785pt}{140.777435pt}}
\pgfpathclose
\pgfusepath{fill,stroke}
\pgfpathmoveto{\pgfpoint{57.535751pt}{147.868973pt}}
\pgflineto{\pgfpoint{56.539772pt}{168.281158pt}}
\pgflineto{\pgfpoint{56.539772pt}{140.777435pt}}
\pgfpathclose
\pgfusepath{fill,stroke}
\pgfpathmoveto{\pgfpoint{57.535751pt}{140.777435pt}}
\pgflineto{\pgfpoint{57.535751pt}{147.868973pt}}
\pgflineto{\pgfpoint{56.539772pt}{140.777435pt}}
\pgfpathclose
\pgfusepath{fill,stroke}
\pgfpathmoveto{\pgfpoint{58.531738pt}{141.160339pt}}
\pgflineto{\pgfpoint{57.535751pt}{147.868973pt}}
\pgflineto{\pgfpoint{57.535751pt}{140.777435pt}}
\pgfpathclose
\pgfusepath{fill,stroke}
\pgfpathmoveto{\pgfpoint{58.531738pt}{140.777435pt}}
\pgflineto{\pgfpoint{58.531738pt}{141.160339pt}}
\pgflineto{\pgfpoint{57.535751pt}{140.777435pt}}
\pgfpathclose
\pgfusepath{fill,stroke}
\pgfpathmoveto{\pgfpoint{59.527725pt}{141.894562pt}}
\pgflineto{\pgfpoint{58.531738pt}{141.160339pt}}
\pgflineto{\pgfpoint{58.531738pt}{140.777435pt}}
\pgfpathclose
\pgfusepath{fill,stroke}
\pgfpathmoveto{\pgfpoint{59.527725pt}{140.777435pt}}
\pgflineto{\pgfpoint{59.527725pt}{141.894562pt}}
\pgflineto{\pgfpoint{58.531738pt}{140.777435pt}}
\pgfpathclose
\pgfusepath{fill,stroke}
\pgfpathmoveto{\pgfpoint{60.523712pt}{142.042740pt}}
\pgflineto{\pgfpoint{59.527725pt}{141.894562pt}}
\pgflineto{\pgfpoint{59.527725pt}{140.777435pt}}
\pgfpathclose
\pgfusepath{fill,stroke}
\pgfpathmoveto{\pgfpoint{60.523712pt}{140.777435pt}}
\pgflineto{\pgfpoint{60.523712pt}{142.042740pt}}
\pgflineto{\pgfpoint{59.527725pt}{140.777435pt}}
\pgfpathclose
\pgfusepath{fill,stroke}
\pgfpathmoveto{\pgfpoint{61.519691pt}{141.419281pt}}
\pgflineto{\pgfpoint{60.523712pt}{142.042740pt}}
\pgflineto{\pgfpoint{60.523712pt}{140.777435pt}}
\pgfpathclose
\pgfusepath{fill,stroke}
\pgfpathmoveto{\pgfpoint{61.519691pt}{140.777435pt}}
\pgflineto{\pgfpoint{61.519691pt}{141.419281pt}}
\pgflineto{\pgfpoint{60.523712pt}{140.777435pt}}
\pgfpathclose
\pgfusepath{fill,stroke}
\pgfpathmoveto{\pgfpoint{62.515678pt}{202.358673pt}}
\pgflineto{\pgfpoint{61.519691pt}{141.419281pt}}
\pgflineto{\pgfpoint{61.519691pt}{140.777435pt}}
\pgfpathclose
\pgfusepath{fill,stroke}
\pgfpathmoveto{\pgfpoint{62.515678pt}{140.777435pt}}
\pgflineto{\pgfpoint{62.515678pt}{202.358673pt}}
\pgflineto{\pgfpoint{61.519691pt}{140.777435pt}}
\pgfpathclose
\pgfusepath{fill,stroke}
\pgfpathmoveto{\pgfpoint{63.511658pt}{143.879868pt}}
\pgflineto{\pgfpoint{62.515678pt}{202.358673pt}}
\pgflineto{\pgfpoint{62.515678pt}{140.777435pt}}
\pgfpathclose
\pgfusepath{fill,stroke}
\pgfpathmoveto{\pgfpoint{63.511658pt}{140.777435pt}}
\pgflineto{\pgfpoint{63.511658pt}{143.879868pt}}
\pgflineto{\pgfpoint{62.515678pt}{140.777435pt}}
\pgfpathclose
\pgfusepath{fill,stroke}
\pgfpathmoveto{\pgfpoint{64.507637pt}{144.233887pt}}
\pgflineto{\pgfpoint{63.511658pt}{143.879868pt}}
\pgflineto{\pgfpoint{63.511658pt}{140.777435pt}}
\pgfpathclose
\pgfusepath{fill,stroke}
\pgfpathmoveto{\pgfpoint{64.507637pt}{140.777435pt}}
\pgflineto{\pgfpoint{64.507637pt}{144.233887pt}}
\pgflineto{\pgfpoint{63.511658pt}{140.777435pt}}
\pgfpathclose
\pgfusepath{fill,stroke}
\pgfpathmoveto{\pgfpoint{65.503624pt}{142.070892pt}}
\pgflineto{\pgfpoint{64.507637pt}{144.233887pt}}
\pgflineto{\pgfpoint{64.507637pt}{140.777435pt}}
\pgfpathclose
\pgfusepath{fill,stroke}
\pgfpathmoveto{\pgfpoint{65.503624pt}{140.777435pt}}
\pgflineto{\pgfpoint{65.503624pt}{142.070892pt}}
\pgflineto{\pgfpoint{64.507637pt}{140.777435pt}}
\pgfpathclose
\pgfusepath{fill,stroke}
\pgfpathmoveto{\pgfpoint{66.499619pt}{141.913147pt}}
\pgflineto{\pgfpoint{65.503624pt}{142.070892pt}}
\pgflineto{\pgfpoint{65.503624pt}{140.777435pt}}
\pgfpathclose
\pgfusepath{fill,stroke}
\pgfpathmoveto{\pgfpoint{66.499619pt}{140.777435pt}}
\pgflineto{\pgfpoint{66.499619pt}{141.913147pt}}
\pgflineto{\pgfpoint{65.503624pt}{140.777435pt}}
\pgfpathclose
\pgfusepath{fill,stroke}
\pgfpathmoveto{\pgfpoint{67.495590pt}{140.777435pt}}
\pgflineto{\pgfpoint{66.499619pt}{141.913147pt}}
\pgflineto{\pgfpoint{66.499619pt}{140.777435pt}}
\pgfpathclose
\pgfusepath{fill,stroke}
\pgfpathmoveto{\pgfpoint{46.579933pt}{140.777435pt}}
\pgflineto{\pgfpoint{47.575912pt}{140.777435pt}}
\pgflineto{\pgfpoint{47.575912pt}{141.646210pt}}
\pgfpathclose
\pgfusepath{fill,stroke}
\pgfpathmoveto{\pgfpoint{48.571899pt}{141.530609pt}}
\pgflineto{\pgfpoint{47.575912pt}{141.646210pt}}
\pgflineto{\pgfpoint{47.575912pt}{140.777435pt}}
\pgfpathclose
\pgfusepath{fill,stroke}
\pgfpathmoveto{\pgfpoint{48.571899pt}{140.777435pt}}
\pgflineto{\pgfpoint{48.571899pt}{141.530609pt}}
\pgflineto{\pgfpoint{47.575912pt}{140.777435pt}}
\pgfpathclose
\pgfusepath{fill,stroke}
\pgfpathmoveto{\pgfpoint{49.567879pt}{143.458160pt}}
\pgflineto{\pgfpoint{48.571899pt}{141.530609pt}}
\pgflineto{\pgfpoint{48.571899pt}{140.777435pt}}
\pgfpathclose
\pgfusepath{fill,stroke}
\pgfpathmoveto{\pgfpoint{49.567879pt}{140.777435pt}}
\pgflineto{\pgfpoint{49.567879pt}{143.458160pt}}
\pgflineto{\pgfpoint{48.571899pt}{140.777435pt}}
\pgfpathclose
\pgfusepath{fill,stroke}
\pgfpathmoveto{\pgfpoint{50.563873pt}{140.777435pt}}
\pgflineto{\pgfpoint{49.567879pt}{143.458160pt}}
\pgflineto{\pgfpoint{49.567879pt}{140.777435pt}}
\pgfpathclose
\pgfusepath{fill,stroke}
\pgfpathmoveto{\pgfpoint{44.587967pt}{140.777435pt}}
\pgflineto{\pgfpoint{45.583946pt}{140.777435pt}}
\pgflineto{\pgfpoint{45.583946pt}{148.946091pt}}
\pgfpathclose
\pgfusepath{fill,stroke}
\pgfpathmoveto{\pgfpoint{46.579933pt}{140.777435pt}}
\pgflineto{\pgfpoint{45.583946pt}{148.946091pt}}
\pgflineto{\pgfpoint{45.583946pt}{140.777435pt}}
\pgfpathclose
\pgfusepath{fill,stroke}
\pgfpathmoveto{\pgfpoint{42.595993pt}{140.777435pt}}
\pgflineto{\pgfpoint{43.591980pt}{140.777435pt}}
\pgflineto{\pgfpoint{43.591980pt}{186.324295pt}}
\pgfpathclose
\pgfusepath{fill,stroke}
\pgfpathmoveto{\pgfpoint{44.587967pt}{140.777435pt}}
\pgflineto{\pgfpoint{43.591980pt}{186.324295pt}}
\pgflineto{\pgfpoint{43.591980pt}{140.777435pt}}
\pgfpathclose
\pgfusepath{fill,stroke}
\color[rgb]{0.000000,1.000000,0.000000}
\pgfpathmoveto{\pgfpoint{284.620087pt}{26.399979pt}}
\pgflineto{\pgfpoint{285.616089pt}{26.399979pt}}
\pgflineto{\pgfpoint{285.616089pt}{26.474419pt}}
\pgfpathclose
\pgfusepath{fill,stroke}
\pgfpathmoveto{\pgfpoint{286.612061pt}{26.399979pt}}
\pgflineto{\pgfpoint{285.616089pt}{26.474419pt}}
\pgflineto{\pgfpoint{285.616089pt}{26.399979pt}}
\pgfpathclose
\pgfusepath{fill,stroke}
\pgfpathmoveto{\pgfpoint{281.632141pt}{26.399979pt}}
\pgflineto{\pgfpoint{282.628113pt}{26.399979pt}}
\pgflineto{\pgfpoint{282.628113pt}{29.464996pt}}
\pgfpathclose
\pgfusepath{fill,stroke}
\pgfpathmoveto{\pgfpoint{283.624115pt}{28.196754pt}}
\pgflineto{\pgfpoint{282.628113pt}{29.464996pt}}
\pgflineto{\pgfpoint{282.628113pt}{26.399979pt}}
\pgfpathclose
\pgfusepath{fill,stroke}
\pgfpathmoveto{\pgfpoint{283.624115pt}{26.399979pt}}
\pgflineto{\pgfpoint{283.624115pt}{28.196754pt}}
\pgflineto{\pgfpoint{282.628113pt}{26.399979pt}}
\pgfpathclose
\pgfusepath{fill,stroke}
\pgfpathmoveto{\pgfpoint{284.620087pt}{26.399979pt}}
\pgflineto{\pgfpoint{283.624115pt}{28.196754pt}}
\pgflineto{\pgfpoint{283.624115pt}{26.399979pt}}
\pgfpathclose
\pgfusepath{fill,stroke}
\pgfpathmoveto{\pgfpoint{277.648193pt}{26.399979pt}}
\pgflineto{\pgfpoint{278.644196pt}{26.399979pt}}
\pgflineto{\pgfpoint{278.644196pt}{30.203644pt}}
\pgfpathclose
\pgfusepath{fill,stroke}
\pgfpathmoveto{\pgfpoint{279.640167pt}{35.907738pt}}
\pgflineto{\pgfpoint{278.644196pt}{30.203644pt}}
\pgflineto{\pgfpoint{278.644196pt}{26.399979pt}}
\pgfpathclose
\pgfusepath{fill,stroke}
\pgfpathmoveto{\pgfpoint{279.640167pt}{26.399979pt}}
\pgflineto{\pgfpoint{279.640167pt}{35.907738pt}}
\pgflineto{\pgfpoint{278.644196pt}{26.399979pt}}
\pgfpathclose
\pgfusepath{fill,stroke}
\pgfpathmoveto{\pgfpoint{280.636139pt}{26.399979pt}}
\pgflineto{\pgfpoint{279.640167pt}{35.907738pt}}
\pgflineto{\pgfpoint{279.640167pt}{26.399979pt}}
\pgfpathclose
\pgfusepath{fill,stroke}
\pgfpathmoveto{\pgfpoint{267.688354pt}{26.399979pt}}
\pgflineto{\pgfpoint{268.684326pt}{26.399979pt}}
\pgflineto{\pgfpoint{268.684326pt}{26.647873pt}}
\pgfpathclose
\pgfusepath{fill,stroke}
\pgfpathmoveto{\pgfpoint{269.680328pt}{26.596115pt}}
\pgflineto{\pgfpoint{268.684326pt}{26.647873pt}}
\pgflineto{\pgfpoint{268.684326pt}{26.399979pt}}
\pgfpathclose
\pgfusepath{fill,stroke}
\pgfpathmoveto{\pgfpoint{269.680328pt}{26.399979pt}}
\pgflineto{\pgfpoint{269.680328pt}{26.596115pt}}
\pgflineto{\pgfpoint{268.684326pt}{26.399979pt}}
\pgfpathclose
\pgfusepath{fill,stroke}
\pgfpathmoveto{\pgfpoint{270.676331pt}{27.976303pt}}
\pgflineto{\pgfpoint{269.680328pt}{26.596115pt}}
\pgflineto{\pgfpoint{269.680328pt}{26.399979pt}}
\pgfpathclose
\pgfusepath{fill,stroke}
\pgfpathmoveto{\pgfpoint{270.676331pt}{26.399979pt}}
\pgflineto{\pgfpoint{270.676331pt}{27.976303pt}}
\pgflineto{\pgfpoint{269.680328pt}{26.399979pt}}
\pgfpathclose
\pgfusepath{fill,stroke}
\pgfpathmoveto{\pgfpoint{271.672302pt}{42.660454pt}}
\pgflineto{\pgfpoint{270.676331pt}{27.976303pt}}
\pgflineto{\pgfpoint{270.676331pt}{26.399979pt}}
\pgfpathclose
\pgfusepath{fill,stroke}
\pgfpathmoveto{\pgfpoint{271.672302pt}{26.399979pt}}
\pgflineto{\pgfpoint{271.672302pt}{42.660454pt}}
\pgflineto{\pgfpoint{270.676331pt}{26.399979pt}}
\pgfpathclose
\pgfusepath{fill,stroke}
\pgfpathmoveto{\pgfpoint{272.668274pt}{26.867470pt}}
\pgflineto{\pgfpoint{271.672302pt}{42.660454pt}}
\pgflineto{\pgfpoint{271.672302pt}{26.399979pt}}
\pgfpathclose
\pgfusepath{fill,stroke}
\pgfpathmoveto{\pgfpoint{272.668274pt}{26.399979pt}}
\pgflineto{\pgfpoint{272.668274pt}{26.867470pt}}
\pgflineto{\pgfpoint{271.672302pt}{26.399979pt}}
\pgfpathclose
\pgfusepath{fill,stroke}
\pgfpathmoveto{\pgfpoint{273.664276pt}{26.399979pt}}
\pgflineto{\pgfpoint{272.668274pt}{26.867470pt}}
\pgflineto{\pgfpoint{272.668274pt}{26.399979pt}}
\pgfpathclose
\pgfusepath{fill,stroke}
\pgfpathmoveto{\pgfpoint{265.696411pt}{26.399979pt}}
\pgflineto{\pgfpoint{266.692383pt}{26.399979pt}}
\pgflineto{\pgfpoint{266.692383pt}{26.914734pt}}
\pgfpathclose
\pgfusepath{fill,stroke}
\pgfpathmoveto{\pgfpoint{267.688354pt}{26.399979pt}}
\pgflineto{\pgfpoint{266.692383pt}{26.914734pt}}
\pgflineto{\pgfpoint{266.692383pt}{26.399979pt}}
\pgfpathclose
\pgfusepath{fill,stroke}
\pgfpathmoveto{\pgfpoint{263.704407pt}{26.399979pt}}
\pgflineto{\pgfpoint{264.700409pt}{26.399979pt}}
\pgflineto{\pgfpoint{264.700409pt}{26.932861pt}}
\pgfpathclose
\pgfusepath{fill,stroke}
\pgfpathmoveto{\pgfpoint{265.696411pt}{26.399979pt}}
\pgflineto{\pgfpoint{264.700409pt}{26.932861pt}}
\pgflineto{\pgfpoint{264.700409pt}{26.399979pt}}
\pgfpathclose
\pgfusepath{fill,stroke}
\pgfpathmoveto{\pgfpoint{260.716492pt}{26.399979pt}}
\pgflineto{\pgfpoint{261.712463pt}{26.399979pt}}
\pgflineto{\pgfpoint{261.712463pt}{27.286751pt}}
\pgfpathclose
\pgfusepath{fill,stroke}
\pgfpathmoveto{\pgfpoint{262.708435pt}{26.399979pt}}
\pgflineto{\pgfpoint{261.712463pt}{27.286751pt}}
\pgflineto{\pgfpoint{261.712463pt}{26.399979pt}}
\pgfpathclose
\pgfusepath{fill,stroke}
\pgfpathmoveto{\pgfpoint{257.728516pt}{26.399979pt}}
\pgflineto{\pgfpoint{258.724518pt}{26.399979pt}}
\pgflineto{\pgfpoint{258.724518pt}{26.886185pt}}
\pgfpathclose
\pgfusepath{fill,stroke}
\pgfpathmoveto{\pgfpoint{259.720490pt}{26.399979pt}}
\pgflineto{\pgfpoint{258.724518pt}{26.886185pt}}
\pgflineto{\pgfpoint{258.724518pt}{26.399979pt}}
\pgfpathclose
\pgfusepath{fill,stroke}
\pgfpathmoveto{\pgfpoint{253.744598pt}{26.399979pt}}
\pgflineto{\pgfpoint{254.740570pt}{26.399979pt}}
\pgflineto{\pgfpoint{254.740570pt}{27.508163pt}}
\pgfpathclose
\pgfusepath{fill,stroke}
\pgfpathmoveto{\pgfpoint{255.736542pt}{26.399979pt}}
\pgflineto{\pgfpoint{254.740570pt}{27.508163pt}}
\pgflineto{\pgfpoint{254.740570pt}{26.399979pt}}
\pgfpathclose
\pgfusepath{fill,stroke}
\pgfpathmoveto{\pgfpoint{248.764679pt}{26.399979pt}}
\pgflineto{\pgfpoint{249.760651pt}{26.399979pt}}
\pgflineto{\pgfpoint{249.760651pt}{26.474113pt}}
\pgfpathclose
\pgfusepath{fill,stroke}
\pgfpathmoveto{\pgfpoint{250.756638pt}{26.708969pt}}
\pgflineto{\pgfpoint{249.760651pt}{26.474113pt}}
\pgflineto{\pgfpoint{249.760651pt}{26.399979pt}}
\pgfpathclose
\pgfusepath{fill,stroke}
\pgfpathmoveto{\pgfpoint{250.756638pt}{26.399979pt}}
\pgflineto{\pgfpoint{250.756638pt}{26.708969pt}}
\pgflineto{\pgfpoint{249.760651pt}{26.399979pt}}
\pgfpathclose
\pgfusepath{fill,stroke}
\pgfpathmoveto{\pgfpoint{251.752625pt}{26.399979pt}}
\pgflineto{\pgfpoint{250.756638pt}{26.708969pt}}
\pgflineto{\pgfpoint{250.756638pt}{26.399979pt}}
\pgfpathclose
\pgfusepath{fill,stroke}
\pgfpathmoveto{\pgfpoint{244.780731pt}{26.399979pt}}
\pgflineto{\pgfpoint{245.776718pt}{26.399979pt}}
\pgflineto{\pgfpoint{245.776718pt}{26.477646pt}}
\pgfpathclose
\pgfusepath{fill,stroke}
\pgfpathmoveto{\pgfpoint{246.772705pt}{26.550812pt}}
\pgflineto{\pgfpoint{245.776718pt}{26.477646pt}}
\pgflineto{\pgfpoint{245.776718pt}{26.399979pt}}
\pgfpathclose
\pgfusepath{fill,stroke}
\pgfpathmoveto{\pgfpoint{246.772705pt}{26.399979pt}}
\pgflineto{\pgfpoint{246.772705pt}{26.550812pt}}
\pgflineto{\pgfpoint{245.776718pt}{26.399979pt}}
\pgfpathclose
\pgfusepath{fill,stroke}
\pgfpathmoveto{\pgfpoint{247.768677pt}{41.200562pt}}
\pgflineto{\pgfpoint{246.772705pt}{26.550812pt}}
\pgflineto{\pgfpoint{246.772705pt}{26.399979pt}}
\pgfpathclose
\pgfusepath{fill,stroke}
\pgfpathmoveto{\pgfpoint{247.768677pt}{26.399979pt}}
\pgflineto{\pgfpoint{247.768677pt}{41.200562pt}}
\pgflineto{\pgfpoint{246.772705pt}{26.399979pt}}
\pgfpathclose
\pgfusepath{fill,stroke}
\pgfpathmoveto{\pgfpoint{248.764679pt}{26.399979pt}}
\pgflineto{\pgfpoint{247.768677pt}{41.200562pt}}
\pgflineto{\pgfpoint{247.768677pt}{26.399979pt}}
\pgfpathclose
\pgfusepath{fill,stroke}
\pgfpathmoveto{\pgfpoint{241.792786pt}{26.399979pt}}
\pgflineto{\pgfpoint{242.788757pt}{26.399979pt}}
\pgflineto{\pgfpoint{242.788757pt}{26.819893pt}}
\pgfpathclose
\pgfusepath{fill,stroke}
\pgfpathmoveto{\pgfpoint{243.784744pt}{27.221687pt}}
\pgflineto{\pgfpoint{242.788757pt}{26.819893pt}}
\pgflineto{\pgfpoint{242.788757pt}{26.399979pt}}
\pgfpathclose
\pgfusepath{fill,stroke}
\pgfpathmoveto{\pgfpoint{243.784744pt}{26.399979pt}}
\pgflineto{\pgfpoint{243.784744pt}{27.221687pt}}
\pgflineto{\pgfpoint{242.788757pt}{26.399979pt}}
\pgfpathclose
\pgfusepath{fill,stroke}
\pgfpathmoveto{\pgfpoint{244.780731pt}{26.399979pt}}
\pgflineto{\pgfpoint{243.784744pt}{27.221687pt}}
\pgflineto{\pgfpoint{243.784744pt}{26.399979pt}}
\pgfpathclose
\pgfusepath{fill,stroke}
\pgfpathmoveto{\pgfpoint{235.816864pt}{26.399979pt}}
\pgflineto{\pgfpoint{236.812866pt}{26.399979pt}}
\pgflineto{\pgfpoint{236.812866pt}{35.730797pt}}
\pgfpathclose
\pgfusepath{fill,stroke}
\pgfpathmoveto{\pgfpoint{237.808838pt}{26.516998pt}}
\pgflineto{\pgfpoint{236.812866pt}{35.730797pt}}
\pgflineto{\pgfpoint{236.812866pt}{26.399979pt}}
\pgfpathclose
\pgfusepath{fill,stroke}
\pgfpathmoveto{\pgfpoint{237.808838pt}{26.399979pt}}
\pgflineto{\pgfpoint{237.808838pt}{26.516998pt}}
\pgflineto{\pgfpoint{236.812866pt}{26.399979pt}}
\pgfpathclose
\pgfusepath{fill,stroke}
\pgfpathmoveto{\pgfpoint{238.804825pt}{26.399979pt}}
\pgflineto{\pgfpoint{237.808838pt}{26.516998pt}}
\pgflineto{\pgfpoint{237.808838pt}{26.399979pt}}
\pgfpathclose
\pgfusepath{fill,stroke}
\pgfpathmoveto{\pgfpoint{227.849014pt}{26.399979pt}}
\pgflineto{\pgfpoint{228.845001pt}{26.399979pt}}
\pgflineto{\pgfpoint{228.845001pt}{27.012459pt}}
\pgfpathclose
\pgfusepath{fill,stroke}
\pgfpathmoveto{\pgfpoint{229.840973pt}{26.558662pt}}
\pgflineto{\pgfpoint{228.845001pt}{27.012459pt}}
\pgflineto{\pgfpoint{228.845001pt}{26.399979pt}}
\pgfpathclose
\pgfusepath{fill,stroke}
\pgfpathmoveto{\pgfpoint{229.840973pt}{26.399979pt}}
\pgflineto{\pgfpoint{229.840973pt}{26.558662pt}}
\pgflineto{\pgfpoint{228.845001pt}{26.399979pt}}
\pgfpathclose
\pgfusepath{fill,stroke}
\pgfpathmoveto{\pgfpoint{230.836945pt}{26.682472pt}}
\pgflineto{\pgfpoint{229.840973pt}{26.558662pt}}
\pgflineto{\pgfpoint{229.840973pt}{26.399979pt}}
\pgfpathclose
\pgfusepath{fill,stroke}
\pgfpathmoveto{\pgfpoint{230.836945pt}{26.399979pt}}
\pgflineto{\pgfpoint{230.836945pt}{26.682472pt}}
\pgflineto{\pgfpoint{229.840973pt}{26.399979pt}}
\pgfpathclose
\pgfusepath{fill,stroke}
\pgfpathmoveto{\pgfpoint{231.832932pt}{26.399979pt}}
\pgflineto{\pgfpoint{230.836945pt}{26.682472pt}}
\pgflineto{\pgfpoint{230.836945pt}{26.399979pt}}
\pgfpathclose
\pgfusepath{fill,stroke}
\pgfpathmoveto{\pgfpoint{224.861053pt}{26.399979pt}}
\pgflineto{\pgfpoint{225.857040pt}{26.399979pt}}
\pgflineto{\pgfpoint{225.857040pt}{28.125343pt}}
\pgfpathclose
\pgfusepath{fill,stroke}
\pgfpathmoveto{\pgfpoint{226.853027pt}{26.399979pt}}
\pgflineto{\pgfpoint{225.857040pt}{28.125343pt}}
\pgflineto{\pgfpoint{225.857040pt}{26.399979pt}}
\pgfpathclose
\pgfusepath{fill,stroke}
\pgfpathmoveto{\pgfpoint{216.893188pt}{26.399979pt}}
\pgflineto{\pgfpoint{217.889160pt}{26.399979pt}}
\pgflineto{\pgfpoint{217.889160pt}{66.077660pt}}
\pgfpathclose
\pgfusepath{fill,stroke}
\pgfpathmoveto{\pgfpoint{218.885147pt}{26.399979pt}}
\pgflineto{\pgfpoint{217.889160pt}{66.077660pt}}
\pgflineto{\pgfpoint{217.889160pt}{26.399979pt}}
\pgfpathclose
\pgfusepath{fill,stroke}
\pgfpathmoveto{\pgfpoint{213.905228pt}{26.399979pt}}
\pgflineto{\pgfpoint{214.901215pt}{26.399979pt}}
\pgflineto{\pgfpoint{214.901215pt}{27.407372pt}}
\pgfpathclose
\pgfusepath{fill,stroke}
\pgfpathmoveto{\pgfpoint{215.897217pt}{26.495361pt}}
\pgflineto{\pgfpoint{214.901215pt}{27.407372pt}}
\pgflineto{\pgfpoint{214.901215pt}{26.399979pt}}
\pgfpathclose
\pgfusepath{fill,stroke}
\pgfpathmoveto{\pgfpoint{215.897217pt}{26.399979pt}}
\pgflineto{\pgfpoint{215.897217pt}{26.495361pt}}
\pgflineto{\pgfpoint{214.901215pt}{26.399979pt}}
\pgfpathclose
\pgfusepath{fill,stroke}
\pgfpathmoveto{\pgfpoint{216.893188pt}{26.399979pt}}
\pgflineto{\pgfpoint{215.897217pt}{26.495361pt}}
\pgflineto{\pgfpoint{215.897217pt}{26.399979pt}}
\pgfpathclose
\pgfusepath{fill,stroke}
\pgfpathmoveto{\pgfpoint{209.921295pt}{26.399979pt}}
\pgflineto{\pgfpoint{210.917267pt}{26.399979pt}}
\pgflineto{\pgfpoint{210.917267pt}{56.628197pt}}
\pgfpathclose
\pgfusepath{fill,stroke}
\pgfpathmoveto{\pgfpoint{211.913269pt}{30.417183pt}}
\pgflineto{\pgfpoint{210.917267pt}{56.628197pt}}
\pgflineto{\pgfpoint{210.917267pt}{26.399979pt}}
\pgfpathclose
\pgfusepath{fill,stroke}
\pgfpathmoveto{\pgfpoint{211.913269pt}{26.399979pt}}
\pgflineto{\pgfpoint{211.913269pt}{30.417183pt}}
\pgflineto{\pgfpoint{210.917267pt}{26.399979pt}}
\pgfpathclose
\pgfusepath{fill,stroke}
\pgfpathmoveto{\pgfpoint{212.909241pt}{47.374290pt}}
\pgflineto{\pgfpoint{211.913269pt}{30.417183pt}}
\pgflineto{\pgfpoint{211.913269pt}{26.399979pt}}
\pgfpathclose
\pgfusepath{fill,stroke}
\pgfpathmoveto{\pgfpoint{212.909241pt}{26.399979pt}}
\pgflineto{\pgfpoint{212.909241pt}{47.374290pt}}
\pgflineto{\pgfpoint{211.913269pt}{26.399979pt}}
\pgfpathclose
\pgfusepath{fill,stroke}
\pgfpathmoveto{\pgfpoint{213.905228pt}{26.399979pt}}
\pgflineto{\pgfpoint{212.909241pt}{47.374290pt}}
\pgflineto{\pgfpoint{212.909241pt}{26.399979pt}}
\pgfpathclose
\pgfusepath{fill,stroke}
\pgfpathmoveto{\pgfpoint{207.929337pt}{26.399979pt}}
\pgflineto{\pgfpoint{208.925323pt}{26.399979pt}}
\pgflineto{\pgfpoint{208.925323pt}{29.200577pt}}
\pgfpathclose
\pgfusepath{fill,stroke}
\pgfpathmoveto{\pgfpoint{209.921295pt}{26.399979pt}}
\pgflineto{\pgfpoint{208.925323pt}{29.200577pt}}
\pgflineto{\pgfpoint{208.925323pt}{26.399979pt}}
\pgfpathclose
\pgfusepath{fill,stroke}
\pgfpathmoveto{\pgfpoint{204.941376pt}{26.399979pt}}
\pgflineto{\pgfpoint{205.937347pt}{26.399979pt}}
\pgflineto{\pgfpoint{205.937347pt}{29.413773pt}}
\pgfpathclose
\pgfusepath{fill,stroke}
\pgfpathmoveto{\pgfpoint{206.933334pt}{28.372978pt}}
\pgflineto{\pgfpoint{205.937347pt}{29.413773pt}}
\pgflineto{\pgfpoint{205.937347pt}{26.399979pt}}
\pgfpathclose
\pgfusepath{fill,stroke}
\pgfpathmoveto{\pgfpoint{206.933334pt}{26.399979pt}}
\pgflineto{\pgfpoint{206.933334pt}{28.372978pt}}
\pgflineto{\pgfpoint{205.937347pt}{26.399979pt}}
\pgfpathclose
\pgfusepath{fill,stroke}
\pgfpathmoveto{\pgfpoint{207.929337pt}{26.399979pt}}
\pgflineto{\pgfpoint{206.933334pt}{28.372978pt}}
\pgflineto{\pgfpoint{206.933334pt}{26.399979pt}}
\pgfpathclose
\pgfusepath{fill,stroke}
\pgfpathmoveto{\pgfpoint{202.949402pt}{26.399979pt}}
\pgflineto{\pgfpoint{203.945404pt}{26.399979pt}}
\pgflineto{\pgfpoint{203.945404pt}{26.673592pt}}
\pgfpathclose
\pgfusepath{fill,stroke}
\pgfpathmoveto{\pgfpoint{204.941376pt}{26.399979pt}}
\pgflineto{\pgfpoint{203.945404pt}{26.673592pt}}
\pgflineto{\pgfpoint{203.945404pt}{26.399979pt}}
\pgfpathclose
\pgfusepath{fill,stroke}
\pgfpathmoveto{\pgfpoint{196.973511pt}{26.399979pt}}
\pgflineto{\pgfpoint{197.969498pt}{26.399979pt}}
\pgflineto{\pgfpoint{197.969498pt}{27.127144pt}}
\pgfpathclose
\pgfusepath{fill,stroke}
\pgfpathmoveto{\pgfpoint{198.965469pt}{26.399979pt}}
\pgflineto{\pgfpoint{197.969498pt}{27.127144pt}}
\pgflineto{\pgfpoint{197.969498pt}{26.399979pt}}
\pgfpathclose
\pgfusepath{fill,stroke}
\pgfpathmoveto{\pgfpoint{193.985565pt}{26.399979pt}}
\pgflineto{\pgfpoint{194.981537pt}{26.399979pt}}
\pgflineto{\pgfpoint{194.981537pt}{28.872696pt}}
\pgfpathclose
\pgfusepath{fill,stroke}
\pgfpathmoveto{\pgfpoint{195.977524pt}{27.054726pt}}
\pgflineto{\pgfpoint{194.981537pt}{28.872696pt}}
\pgflineto{\pgfpoint{194.981537pt}{26.399979pt}}
\pgfpathclose
\pgfusepath{fill,stroke}
\pgfpathmoveto{\pgfpoint{195.977524pt}{26.399979pt}}
\pgflineto{\pgfpoint{195.977524pt}{27.054726pt}}
\pgflineto{\pgfpoint{194.981537pt}{26.399979pt}}
\pgfpathclose
\pgfusepath{fill,stroke}
\pgfpathmoveto{\pgfpoint{196.973511pt}{26.399979pt}}
\pgflineto{\pgfpoint{195.977524pt}{27.054726pt}}
\pgflineto{\pgfpoint{195.977524pt}{26.399979pt}}
\pgfpathclose
\pgfusepath{fill,stroke}
\pgfpathmoveto{\pgfpoint{190.997604pt}{26.399979pt}}
\pgflineto{\pgfpoint{191.993591pt}{26.399979pt}}
\pgflineto{\pgfpoint{191.993591pt}{31.034332pt}}
\pgfpathclose
\pgfusepath{fill,stroke}
\pgfpathmoveto{\pgfpoint{192.989563pt}{26.399979pt}}
\pgflineto{\pgfpoint{191.993591pt}{31.034332pt}}
\pgflineto{\pgfpoint{191.993591pt}{26.399979pt}}
\pgfpathclose
\pgfusepath{fill,stroke}
\pgfpathmoveto{\pgfpoint{189.005630pt}{26.399979pt}}
\pgflineto{\pgfpoint{190.001617pt}{26.399979pt}}
\pgflineto{\pgfpoint{190.001617pt}{34.817444pt}}
\pgfpathclose
\pgfusepath{fill,stroke}
\pgfpathmoveto{\pgfpoint{190.997604pt}{26.399979pt}}
\pgflineto{\pgfpoint{190.001617pt}{34.817444pt}}
\pgflineto{\pgfpoint{190.001617pt}{26.399979pt}}
\pgfpathclose
\pgfusepath{fill,stroke}
\pgfpathmoveto{\pgfpoint{182.033752pt}{26.399979pt}}
\pgflineto{\pgfpoint{183.029724pt}{26.399979pt}}
\pgflineto{\pgfpoint{183.029724pt}{26.463455pt}}
\pgfpathclose
\pgfusepath{fill,stroke}
\pgfpathmoveto{\pgfpoint{184.025711pt}{26.399979pt}}
\pgflineto{\pgfpoint{183.029724pt}{26.463455pt}}
\pgflineto{\pgfpoint{183.029724pt}{26.399979pt}}
\pgfpathclose
\pgfusepath{fill,stroke}
\pgfpathmoveto{\pgfpoint{177.053818pt}{26.399979pt}}
\pgflineto{\pgfpoint{178.049805pt}{91.264786pt}}
\pgflineto{\pgfpoint{177.581543pt}{91.264786pt}}
\pgfpathclose
\pgfusepath{fill,stroke}
\pgfpathmoveto{\pgfpoint{177.053818pt}{26.399979pt}}
\pgflineto{\pgfpoint{178.049805pt}{26.399979pt}}
\pgflineto{\pgfpoint{178.049805pt}{91.264786pt}}
\pgfpathclose
\pgfusepath{fill,stroke}
\pgfpathmoveto{\pgfpoint{179.045792pt}{26.504753pt}}
\pgflineto{\pgfpoint{178.049805pt}{91.264786pt}}
\pgflineto{\pgfpoint{178.049805pt}{26.399979pt}}
\pgfpathclose
\pgfusepath{fill,stroke}
\pgfpathmoveto{\pgfpoint{179.045792pt}{26.504753pt}}
\pgflineto{\pgfpoint{178.518478pt}{91.264793pt}}
\pgflineto{\pgfpoint{178.049805pt}{91.264786pt}}
\pgfpathclose
\pgfusepath{fill,stroke}
\pgfpathmoveto{\pgfpoint{179.045792pt}{26.399979pt}}
\pgflineto{\pgfpoint{179.045792pt}{26.504753pt}}
\pgflineto{\pgfpoint{178.049805pt}{26.399979pt}}
\pgfpathclose
\pgfusepath{fill,stroke}
\pgfpathmoveto{\pgfpoint{180.041779pt}{30.030716pt}}
\pgflineto{\pgfpoint{179.045792pt}{26.504753pt}}
\pgflineto{\pgfpoint{179.045792pt}{26.399979pt}}
\pgfpathclose
\pgfusepath{fill,stroke}
\pgfpathmoveto{\pgfpoint{180.041779pt}{26.399979pt}}
\pgflineto{\pgfpoint{180.041779pt}{30.030716pt}}
\pgflineto{\pgfpoint{179.045792pt}{26.399979pt}}
\pgfpathclose
\pgfusepath{fill,stroke}
\pgfpathmoveto{\pgfpoint{181.037766pt}{26.399979pt}}
\pgflineto{\pgfpoint{180.041779pt}{30.030716pt}}
\pgflineto{\pgfpoint{180.041779pt}{26.399979pt}}
\pgfpathclose
\pgfusepath{fill,stroke}
\pgfpathmoveto{\pgfpoint{169.085953pt}{26.399979pt}}
\pgflineto{\pgfpoint{170.081940pt}{26.399979pt}}
\pgflineto{\pgfpoint{170.081940pt}{26.839508pt}}
\pgfpathclose
\pgfusepath{fill,stroke}
\pgfpathmoveto{\pgfpoint{171.077911pt}{26.399979pt}}
\pgflineto{\pgfpoint{170.081940pt}{26.839508pt}}
\pgflineto{\pgfpoint{170.081940pt}{26.399979pt}}
\pgfpathclose
\pgfusepath{fill,stroke}
\pgfpathmoveto{\pgfpoint{166.098007pt}{26.399979pt}}
\pgflineto{\pgfpoint{167.093994pt}{26.399979pt}}
\pgflineto{\pgfpoint{167.093994pt}{30.536880pt}}
\pgfpathclose
\pgfusepath{fill,stroke}
\pgfpathmoveto{\pgfpoint{168.089966pt}{26.447098pt}}
\pgflineto{\pgfpoint{167.093994pt}{30.536880pt}}
\pgflineto{\pgfpoint{167.093994pt}{26.399979pt}}
\pgfpathclose
\pgfusepath{fill,stroke}
\pgfpathmoveto{\pgfpoint{168.089966pt}{26.399979pt}}
\pgflineto{\pgfpoint{168.089966pt}{26.447098pt}}
\pgflineto{\pgfpoint{167.093994pt}{26.399979pt}}
\pgfpathclose
\pgfusepath{fill,stroke}
\pgfpathmoveto{\pgfpoint{169.085953pt}{26.399979pt}}
\pgflineto{\pgfpoint{168.089966pt}{26.447098pt}}
\pgflineto{\pgfpoint{168.089966pt}{26.399979pt}}
\pgfpathclose
\pgfusepath{fill,stroke}
\pgfpathmoveto{\pgfpoint{164.106033pt}{26.399979pt}}
\pgflineto{\pgfpoint{165.102020pt}{26.399979pt}}
\pgflineto{\pgfpoint{165.102020pt}{26.737396pt}}
\pgfpathclose
\pgfusepath{fill,stroke}
\pgfpathmoveto{\pgfpoint{166.098007pt}{26.399979pt}}
\pgflineto{\pgfpoint{165.102020pt}{26.737396pt}}
\pgflineto{\pgfpoint{165.102020pt}{26.399979pt}}
\pgfpathclose
\pgfusepath{fill,stroke}
\pgfpathmoveto{\pgfpoint{160.122101pt}{26.399979pt}}
\pgflineto{\pgfpoint{161.118088pt}{26.399979pt}}
\pgflineto{\pgfpoint{161.118088pt}{26.483330pt}}
\pgfpathclose
\pgfusepath{fill,stroke}
\pgfpathmoveto{\pgfpoint{162.114075pt}{26.604111pt}}
\pgflineto{\pgfpoint{161.118088pt}{26.483330pt}}
\pgflineto{\pgfpoint{161.118088pt}{26.399979pt}}
\pgfpathclose
\pgfusepath{fill,stroke}
\pgfpathmoveto{\pgfpoint{162.114075pt}{26.399979pt}}
\pgflineto{\pgfpoint{162.114075pt}{26.604111pt}}
\pgflineto{\pgfpoint{161.118088pt}{26.399979pt}}
\pgfpathclose
\pgfusepath{fill,stroke}
\pgfpathmoveto{\pgfpoint{163.110062pt}{28.794250pt}}
\pgflineto{\pgfpoint{162.114075pt}{26.604111pt}}
\pgflineto{\pgfpoint{162.114075pt}{26.399979pt}}
\pgfpathclose
\pgfusepath{fill,stroke}
\pgfpathmoveto{\pgfpoint{163.110062pt}{26.399979pt}}
\pgflineto{\pgfpoint{163.110062pt}{28.794250pt}}
\pgflineto{\pgfpoint{162.114075pt}{26.399979pt}}
\pgfpathclose
\pgfusepath{fill,stroke}
\pgfpathmoveto{\pgfpoint{164.106033pt}{26.399979pt}}
\pgflineto{\pgfpoint{163.110062pt}{28.794250pt}}
\pgflineto{\pgfpoint{163.110062pt}{26.399979pt}}
\pgfpathclose
\pgfusepath{fill,stroke}
\pgfpathmoveto{\pgfpoint{156.138168pt}{26.399979pt}}
\pgflineto{\pgfpoint{157.134155pt}{26.399979pt}}
\pgflineto{\pgfpoint{157.134155pt}{26.492256pt}}
\pgfpathclose
\pgfusepath{fill,stroke}
\pgfpathmoveto{\pgfpoint{158.130127pt}{26.480469pt}}
\pgflineto{\pgfpoint{157.134155pt}{26.492256pt}}
\pgflineto{\pgfpoint{157.134155pt}{26.399979pt}}
\pgfpathclose
\pgfusepath{fill,stroke}
\pgfpathmoveto{\pgfpoint{158.130127pt}{26.399979pt}}
\pgflineto{\pgfpoint{158.130127pt}{26.480469pt}}
\pgflineto{\pgfpoint{157.134155pt}{26.399979pt}}
\pgfpathclose
\pgfusepath{fill,stroke}
\pgfpathmoveto{\pgfpoint{159.126114pt}{26.399979pt}}
\pgflineto{\pgfpoint{158.130127pt}{26.480469pt}}
\pgflineto{\pgfpoint{158.130127pt}{26.399979pt}}
\pgfpathclose
\pgfusepath{fill,stroke}
\pgfpathmoveto{\pgfpoint{154.146194pt}{26.399979pt}}
\pgflineto{\pgfpoint{155.142181pt}{26.399979pt}}
\pgflineto{\pgfpoint{155.142181pt}{28.098244pt}}
\pgfpathclose
\pgfusepath{fill,stroke}
\pgfpathmoveto{\pgfpoint{156.138168pt}{26.399979pt}}
\pgflineto{\pgfpoint{155.142181pt}{28.098244pt}}
\pgflineto{\pgfpoint{155.142181pt}{26.399979pt}}
\pgfpathclose
\pgfusepath{fill,stroke}
\pgfpathmoveto{\pgfpoint{150.162262pt}{26.399979pt}}
\pgflineto{\pgfpoint{151.158249pt}{26.399979pt}}
\pgflineto{\pgfpoint{151.158249pt}{26.439781pt}}
\pgfpathclose
\pgfusepath{fill,stroke}
\pgfpathmoveto{\pgfpoint{152.154221pt}{26.910515pt}}
\pgflineto{\pgfpoint{151.158249pt}{26.439781pt}}
\pgflineto{\pgfpoint{151.158249pt}{26.399979pt}}
\pgfpathclose
\pgfusepath{fill,stroke}
\pgfpathmoveto{\pgfpoint{152.154221pt}{26.399979pt}}
\pgflineto{\pgfpoint{152.154221pt}{26.910515pt}}
\pgflineto{\pgfpoint{151.158249pt}{26.399979pt}}
\pgfpathclose
\pgfusepath{fill,stroke}
\pgfpathmoveto{\pgfpoint{153.150208pt}{26.399979pt}}
\pgflineto{\pgfpoint{152.154221pt}{26.910515pt}}
\pgflineto{\pgfpoint{152.154221pt}{26.399979pt}}
\pgfpathclose
\pgfusepath{fill,stroke}
\pgfpathmoveto{\pgfpoint{145.182343pt}{26.399979pt}}
\pgflineto{\pgfpoint{146.178314pt}{26.399979pt}}
\pgflineto{\pgfpoint{146.178314pt}{26.535568pt}}
\pgfpathclose
\pgfusepath{fill,stroke}
\pgfpathmoveto{\pgfpoint{147.174316pt}{26.399979pt}}
\pgflineto{\pgfpoint{146.178314pt}{26.535568pt}}
\pgflineto{\pgfpoint{146.178314pt}{26.399979pt}}
\pgfpathclose
\pgfusepath{fill,stroke}
\pgfpathmoveto{\pgfpoint{142.194382pt}{26.399979pt}}
\pgflineto{\pgfpoint{143.190369pt}{26.399979pt}}
\pgflineto{\pgfpoint{143.190369pt}{31.207901pt}}
\pgfpathclose
\pgfusepath{fill,stroke}
\pgfpathmoveto{\pgfpoint{144.186356pt}{26.399979pt}}
\pgflineto{\pgfpoint{143.190369pt}{31.207901pt}}
\pgflineto{\pgfpoint{143.190369pt}{26.399979pt}}
\pgfpathclose
\pgfusepath{fill,stroke}
\pgfpathmoveto{\pgfpoint{138.210449pt}{26.399979pt}}
\pgflineto{\pgfpoint{139.206436pt}{26.399979pt}}
\pgflineto{\pgfpoint{139.206436pt}{27.485779pt}}
\pgfpathclose
\pgfusepath{fill,stroke}
\pgfpathmoveto{\pgfpoint{140.202423pt}{27.782883pt}}
\pgflineto{\pgfpoint{139.206436pt}{27.485779pt}}
\pgflineto{\pgfpoint{139.206436pt}{26.399979pt}}
\pgfpathclose
\pgfusepath{fill,stroke}
\pgfpathmoveto{\pgfpoint{140.202423pt}{26.399979pt}}
\pgflineto{\pgfpoint{140.202423pt}{27.782883pt}}
\pgflineto{\pgfpoint{139.206436pt}{26.399979pt}}
\pgfpathclose
\pgfusepath{fill,stroke}
\pgfpathmoveto{\pgfpoint{141.198410pt}{26.399979pt}}
\pgflineto{\pgfpoint{140.202423pt}{27.782883pt}}
\pgflineto{\pgfpoint{140.202423pt}{26.399979pt}}
\pgfpathclose
\pgfusepath{fill,stroke}
\pgfpathmoveto{\pgfpoint{125.262665pt}{26.399979pt}}
\pgflineto{\pgfpoint{126.258652pt}{26.399979pt}}
\pgflineto{\pgfpoint{126.258652pt}{28.660576pt}}
\pgfpathclose
\pgfusepath{fill,stroke}
\pgfpathmoveto{\pgfpoint{127.254631pt}{29.838959pt}}
\pgflineto{\pgfpoint{126.258652pt}{28.660576pt}}
\pgflineto{\pgfpoint{126.258652pt}{26.399979pt}}
\pgfpathclose
\pgfusepath{fill,stroke}
\pgfpathmoveto{\pgfpoint{127.254631pt}{26.399979pt}}
\pgflineto{\pgfpoint{127.254631pt}{29.838959pt}}
\pgflineto{\pgfpoint{126.258652pt}{26.399979pt}}
\pgfpathclose
\pgfusepath{fill,stroke}
\pgfpathmoveto{\pgfpoint{128.250610pt}{26.614319pt}}
\pgflineto{\pgfpoint{127.254631pt}{29.838959pt}}
\pgflineto{\pgfpoint{127.254631pt}{26.399979pt}}
\pgfpathclose
\pgfusepath{fill,stroke}
\pgfpathmoveto{\pgfpoint{128.250610pt}{26.399979pt}}
\pgflineto{\pgfpoint{128.250610pt}{26.614319pt}}
\pgflineto{\pgfpoint{127.254631pt}{26.399979pt}}
\pgfpathclose
\pgfusepath{fill,stroke}
\pgfpathmoveto{\pgfpoint{129.246597pt}{26.451225pt}}
\pgflineto{\pgfpoint{128.250610pt}{26.614319pt}}
\pgflineto{\pgfpoint{128.250610pt}{26.399979pt}}
\pgfpathclose
\pgfusepath{fill,stroke}
\pgfpathmoveto{\pgfpoint{129.246597pt}{26.399979pt}}
\pgflineto{\pgfpoint{129.246597pt}{26.451225pt}}
\pgflineto{\pgfpoint{128.250610pt}{26.399979pt}}
\pgfpathclose
\pgfusepath{fill,stroke}
\pgfpathmoveto{\pgfpoint{130.242584pt}{26.430634pt}}
\pgflineto{\pgfpoint{129.246597pt}{26.451225pt}}
\pgflineto{\pgfpoint{129.246597pt}{26.399979pt}}
\pgfpathclose
\pgfusepath{fill,stroke}
\pgfpathmoveto{\pgfpoint{130.242584pt}{26.399979pt}}
\pgflineto{\pgfpoint{130.242584pt}{26.430634pt}}
\pgflineto{\pgfpoint{129.246597pt}{26.399979pt}}
\pgfpathclose
\pgfusepath{fill,stroke}
\pgfpathmoveto{\pgfpoint{131.238571pt}{27.958008pt}}
\pgflineto{\pgfpoint{130.242584pt}{26.430634pt}}
\pgflineto{\pgfpoint{130.242584pt}{26.399979pt}}
\pgfpathclose
\pgfusepath{fill,stroke}
\pgfpathmoveto{\pgfpoint{131.238571pt}{26.399979pt}}
\pgflineto{\pgfpoint{131.238571pt}{27.958008pt}}
\pgflineto{\pgfpoint{130.242584pt}{26.399979pt}}
\pgfpathclose
\pgfusepath{fill,stroke}
\pgfpathmoveto{\pgfpoint{132.234558pt}{27.682930pt}}
\pgflineto{\pgfpoint{131.238571pt}{27.958008pt}}
\pgflineto{\pgfpoint{131.238571pt}{26.399979pt}}
\pgfpathclose
\pgfusepath{fill,stroke}
\pgfpathmoveto{\pgfpoint{132.234558pt}{26.399979pt}}
\pgflineto{\pgfpoint{132.234558pt}{27.682930pt}}
\pgflineto{\pgfpoint{131.238571pt}{26.399979pt}}
\pgfpathclose
\pgfusepath{fill,stroke}
\pgfpathmoveto{\pgfpoint{133.230530pt}{26.399979pt}}
\pgflineto{\pgfpoint{132.234558pt}{27.682930pt}}
\pgflineto{\pgfpoint{132.234558pt}{26.399979pt}}
\pgfpathclose
\pgfusepath{fill,stroke}
\pgfpathmoveto{\pgfpoint{122.274712pt}{26.399979pt}}
\pgflineto{\pgfpoint{123.270691pt}{26.399979pt}}
\pgflineto{\pgfpoint{123.270691pt}{26.975601pt}}
\pgfpathclose
\pgfusepath{fill,stroke}
\pgfpathmoveto{\pgfpoint{124.266678pt}{26.399979pt}}
\pgflineto{\pgfpoint{123.270691pt}{26.975601pt}}
\pgflineto{\pgfpoint{123.270691pt}{26.399979pt}}
\pgfpathclose
\pgfusepath{fill,stroke}
\pgfpathmoveto{\pgfpoint{120.282745pt}{26.399979pt}}
\pgflineto{\pgfpoint{121.278725pt}{26.399979pt}}
\pgflineto{\pgfpoint{121.278725pt}{27.190407pt}}
\pgfpathclose
\pgfusepath{fill,stroke}
\pgfpathmoveto{\pgfpoint{122.274712pt}{26.399979pt}}
\pgflineto{\pgfpoint{121.278725pt}{27.190407pt}}
\pgflineto{\pgfpoint{121.278725pt}{26.399979pt}}
\pgfpathclose
\pgfusepath{fill,stroke}
\pgfpathmoveto{\pgfpoint{115.302826pt}{26.399979pt}}
\pgflineto{\pgfpoint{116.298813pt}{26.399979pt}}
\pgflineto{\pgfpoint{116.298813pt}{29.134193pt}}
\pgfpathclose
\pgfusepath{fill,stroke}
\pgfpathmoveto{\pgfpoint{117.294792pt}{26.815964pt}}
\pgflineto{\pgfpoint{116.298813pt}{29.134193pt}}
\pgflineto{\pgfpoint{116.298813pt}{26.399979pt}}
\pgfpathclose
\pgfusepath{fill,stroke}
\pgfpathmoveto{\pgfpoint{117.294792pt}{26.399979pt}}
\pgflineto{\pgfpoint{117.294792pt}{26.815964pt}}
\pgflineto{\pgfpoint{116.298813pt}{26.399979pt}}
\pgfpathclose
\pgfusepath{fill,stroke}
\pgfpathmoveto{\pgfpoint{117.967636pt}{91.264793pt}}
\pgflineto{\pgfpoint{117.294792pt}{26.815964pt}}
\pgflineto{\pgfpoint{117.294792pt}{26.399979pt}}
\pgfpathclose
\pgfusepath{fill,stroke}
\pgfpathmoveto{\pgfpoint{117.967636pt}{91.264793pt}}
\pgflineto{\pgfpoint{117.966232pt}{91.264786pt}}
\pgflineto{\pgfpoint{117.294792pt}{26.815964pt}}
\pgfpathclose
\pgfusepath{fill,stroke}
\pgfpathmoveto{\pgfpoint{118.290779pt}{26.399979pt}}
\pgflineto{\pgfpoint{117.967636pt}{91.264793pt}}
\pgflineto{\pgfpoint{117.294792pt}{26.399979pt}}
\pgfpathclose
\pgfusepath{fill,stroke}
\pgfpathmoveto{\pgfpoint{118.290779pt}{26.399979pt}}
\pgflineto{\pgfpoint{118.290779pt}{91.264793pt}}
\pgflineto{\pgfpoint{117.967636pt}{91.264793pt}}
\pgfpathclose
\pgfusepath{fill,stroke}
\pgfpathmoveto{\pgfpoint{119.286758pt}{26.399979pt}}
\pgflineto{\pgfpoint{118.290779pt}{91.264793pt}}
\pgflineto{\pgfpoint{118.290779pt}{26.399979pt}}
\pgfpathclose
\pgfusepath{fill,stroke}
\pgfpathmoveto{\pgfpoint{119.286758pt}{26.399979pt}}
\pgflineto{\pgfpoint{118.613922pt}{91.264793pt}}
\pgflineto{\pgfpoint{118.290779pt}{91.264793pt}}
\pgfpathclose
\pgfusepath{fill,stroke}
\pgfpathmoveto{\pgfpoint{111.318893pt}{26.399979pt}}
\pgflineto{\pgfpoint{112.314873pt}{26.399979pt}}
\pgflineto{\pgfpoint{112.314873pt}{37.362213pt}}
\pgfpathclose
\pgfusepath{fill,stroke}
\pgfpathmoveto{\pgfpoint{113.310852pt}{26.517036pt}}
\pgflineto{\pgfpoint{112.314873pt}{37.362213pt}}
\pgflineto{\pgfpoint{112.314873pt}{26.399979pt}}
\pgfpathclose
\pgfusepath{fill,stroke}
\pgfpathmoveto{\pgfpoint{113.310852pt}{26.399979pt}}
\pgflineto{\pgfpoint{113.310852pt}{26.517036pt}}
\pgflineto{\pgfpoint{112.314873pt}{26.399979pt}}
\pgfpathclose
\pgfusepath{fill,stroke}
\pgfpathmoveto{\pgfpoint{114.306839pt}{26.399979pt}}
\pgflineto{\pgfpoint{113.310852pt}{26.517036pt}}
\pgflineto{\pgfpoint{113.310852pt}{26.399979pt}}
\pgfpathclose
\pgfusepath{fill,stroke}
\pgfpathmoveto{\pgfpoint{107.334953pt}{26.399979pt}}
\pgflineto{\pgfpoint{108.330933pt}{26.399979pt}}
\pgflineto{\pgfpoint{108.330933pt}{27.150108pt}}
\pgfpathclose
\pgfusepath{fill,stroke}
\pgfpathmoveto{\pgfpoint{109.326920pt}{28.632256pt}}
\pgflineto{\pgfpoint{108.330933pt}{27.150108pt}}
\pgflineto{\pgfpoint{108.330933pt}{26.399979pt}}
\pgfpathclose
\pgfusepath{fill,stroke}
\pgfpathmoveto{\pgfpoint{109.326920pt}{26.399979pt}}
\pgflineto{\pgfpoint{109.326920pt}{28.632256pt}}
\pgflineto{\pgfpoint{108.330933pt}{26.399979pt}}
\pgfpathclose
\pgfusepath{fill,stroke}
\pgfpathmoveto{\pgfpoint{110.322906pt}{26.399979pt}}
\pgflineto{\pgfpoint{109.326920pt}{28.632256pt}}
\pgflineto{\pgfpoint{109.326920pt}{26.399979pt}}
\pgfpathclose
\pgfusepath{fill,stroke}
\pgfpathmoveto{\pgfpoint{103.351013pt}{26.399979pt}}
\pgflineto{\pgfpoint{104.347000pt}{26.399979pt}}
\pgflineto{\pgfpoint{104.347000pt}{35.689026pt}}
\pgfpathclose
\pgfusepath{fill,stroke}
\pgfpathmoveto{\pgfpoint{105.342987pt}{26.399979pt}}
\pgflineto{\pgfpoint{104.347000pt}{35.689026pt}}
\pgflineto{\pgfpoint{104.347000pt}{26.399979pt}}
\pgfpathclose
\pgfusepath{fill,stroke}
\pgfpathmoveto{\pgfpoint{101.359047pt}{26.399979pt}}
\pgflineto{\pgfpoint{102.355034pt}{26.399979pt}}
\pgflineto{\pgfpoint{102.355034pt}{26.959930pt}}
\pgfpathclose
\pgfusepath{fill,stroke}
\pgfpathmoveto{\pgfpoint{103.351013pt}{26.399979pt}}
\pgflineto{\pgfpoint{102.355034pt}{26.959930pt}}
\pgflineto{\pgfpoint{102.355034pt}{26.399979pt}}
\pgfpathclose
\pgfusepath{fill,stroke}
\pgfpathmoveto{\pgfpoint{97.375107pt}{26.399979pt}}
\pgflineto{\pgfpoint{98.371094pt}{26.399979pt}}
\pgflineto{\pgfpoint{98.371094pt}{47.782463pt}}
\pgfpathclose
\pgfusepath{fill,stroke}
\pgfpathmoveto{\pgfpoint{99.367081pt}{54.046951pt}}
\pgflineto{\pgfpoint{98.371094pt}{47.782463pt}}
\pgflineto{\pgfpoint{98.371094pt}{26.399979pt}}
\pgfpathclose
\pgfusepath{fill,stroke}
\pgfpathmoveto{\pgfpoint{99.367081pt}{26.399979pt}}
\pgflineto{\pgfpoint{99.367081pt}{54.046951pt}}
\pgflineto{\pgfpoint{98.371094pt}{26.399979pt}}
\pgfpathclose
\pgfusepath{fill,stroke}
\pgfpathmoveto{\pgfpoint{100.363068pt}{32.883598pt}}
\pgflineto{\pgfpoint{99.367081pt}{54.046951pt}}
\pgflineto{\pgfpoint{99.367081pt}{26.399979pt}}
\pgfpathclose
\pgfusepath{fill,stroke}
\pgfpathmoveto{\pgfpoint{100.363068pt}{26.399979pt}}
\pgflineto{\pgfpoint{100.363068pt}{32.883598pt}}
\pgflineto{\pgfpoint{99.367081pt}{26.399979pt}}
\pgfpathclose
\pgfusepath{fill,stroke}
\pgfpathmoveto{\pgfpoint{101.359047pt}{26.399979pt}}
\pgflineto{\pgfpoint{100.363068pt}{32.883598pt}}
\pgflineto{\pgfpoint{100.363068pt}{26.399979pt}}
\pgfpathclose
\pgfusepath{fill,stroke}
\pgfpathmoveto{\pgfpoint{91.399208pt}{26.399979pt}}
\pgflineto{\pgfpoint{92.395187pt}{26.399979pt}}
\pgflineto{\pgfpoint{92.395187pt}{26.889816pt}}
\pgfpathclose
\pgfusepath{fill,stroke}
\pgfpathmoveto{\pgfpoint{93.391174pt}{27.006203pt}}
\pgflineto{\pgfpoint{92.395187pt}{26.889816pt}}
\pgflineto{\pgfpoint{92.395187pt}{26.399979pt}}
\pgfpathclose
\pgfusepath{fill,stroke}
\pgfpathmoveto{\pgfpoint{93.391174pt}{26.399979pt}}
\pgflineto{\pgfpoint{93.391174pt}{27.006203pt}}
\pgflineto{\pgfpoint{92.395187pt}{26.399979pt}}
\pgfpathclose
\pgfusepath{fill,stroke}
\pgfpathmoveto{\pgfpoint{94.387161pt}{26.399979pt}}
\pgflineto{\pgfpoint{93.391174pt}{27.006203pt}}
\pgflineto{\pgfpoint{93.391174pt}{26.399979pt}}
\pgfpathclose
\pgfusepath{fill,stroke}
\pgfpathmoveto{\pgfpoint{89.407242pt}{26.399979pt}}
\pgflineto{\pgfpoint{90.403221pt}{26.399979pt}}
\pgflineto{\pgfpoint{90.403221pt}{26.454811pt}}
\pgfpathclose
\pgfusepath{fill,stroke}
\pgfpathmoveto{\pgfpoint{91.399208pt}{26.399979pt}}
\pgflineto{\pgfpoint{90.403221pt}{26.454811pt}}
\pgflineto{\pgfpoint{90.403221pt}{26.399979pt}}
\pgfpathclose
\pgfusepath{fill,stroke}
\pgfpathmoveto{\pgfpoint{87.415276pt}{26.399979pt}}
\pgflineto{\pgfpoint{88.411255pt}{26.399979pt}}
\pgflineto{\pgfpoint{88.411255pt}{31.944191pt}}
\pgfpathclose
\pgfusepath{fill,stroke}
\pgfpathmoveto{\pgfpoint{89.407242pt}{26.399979pt}}
\pgflineto{\pgfpoint{88.411255pt}{31.944191pt}}
\pgflineto{\pgfpoint{88.411255pt}{26.399979pt}}
\pgfpathclose
\pgfusepath{fill,stroke}
\pgfpathmoveto{\pgfpoint{85.423309pt}{26.399979pt}}
\pgflineto{\pgfpoint{86.419289pt}{26.399979pt}}
\pgflineto{\pgfpoint{86.419289pt}{26.871178pt}}
\pgfpathclose
\pgfusepath{fill,stroke}
\pgfpathmoveto{\pgfpoint{87.415276pt}{26.399979pt}}
\pgflineto{\pgfpoint{86.419289pt}{26.871178pt}}
\pgflineto{\pgfpoint{86.419289pt}{26.399979pt}}
\pgfpathclose
\pgfusepath{fill,stroke}
\pgfpathmoveto{\pgfpoint{79.447403pt}{26.399979pt}}
\pgflineto{\pgfpoint{80.443390pt}{26.399979pt}}
\pgflineto{\pgfpoint{80.443390pt}{28.913742pt}}
\pgfpathclose
\pgfusepath{fill,stroke}
\pgfpathmoveto{\pgfpoint{81.439369pt}{28.404305pt}}
\pgflineto{\pgfpoint{80.443390pt}{28.913742pt}}
\pgflineto{\pgfpoint{80.443390pt}{26.399979pt}}
\pgfpathclose
\pgfusepath{fill,stroke}
\pgfpathmoveto{\pgfpoint{81.439369pt}{26.399979pt}}
\pgflineto{\pgfpoint{81.439369pt}{28.404305pt}}
\pgflineto{\pgfpoint{80.443390pt}{26.399979pt}}
\pgfpathclose
\pgfusepath{fill,stroke}
\pgfpathmoveto{\pgfpoint{82.435356pt}{28.310295pt}}
\pgflineto{\pgfpoint{81.439369pt}{28.404305pt}}
\pgflineto{\pgfpoint{81.439369pt}{26.399979pt}}
\pgfpathclose
\pgfusepath{fill,stroke}
\pgfpathmoveto{\pgfpoint{82.435356pt}{26.399979pt}}
\pgflineto{\pgfpoint{82.435356pt}{28.310295pt}}
\pgflineto{\pgfpoint{81.439369pt}{26.399979pt}}
\pgfpathclose
\pgfusepath{fill,stroke}
\pgfpathmoveto{\pgfpoint{83.431335pt}{26.399979pt}}
\pgflineto{\pgfpoint{82.435356pt}{28.310295pt}}
\pgflineto{\pgfpoint{82.435356pt}{26.399979pt}}
\pgfpathclose
\pgfusepath{fill,stroke}
\pgfpathmoveto{\pgfpoint{73.471497pt}{26.399979pt}}
\pgflineto{\pgfpoint{74.467484pt}{26.399979pt}}
\pgflineto{\pgfpoint{74.467484pt}{26.868034pt}}
\pgfpathclose
\pgfusepath{fill,stroke}
\pgfpathmoveto{\pgfpoint{75.463470pt}{26.470932pt}}
\pgflineto{\pgfpoint{74.467484pt}{26.868034pt}}
\pgflineto{\pgfpoint{74.467484pt}{26.399979pt}}
\pgfpathclose
\pgfusepath{fill,stroke}
\pgfpathmoveto{\pgfpoint{75.463470pt}{26.399979pt}}
\pgflineto{\pgfpoint{75.463470pt}{26.470932pt}}
\pgflineto{\pgfpoint{74.467484pt}{26.399979pt}}
\pgfpathclose
\pgfusepath{fill,stroke}
\pgfpathmoveto{\pgfpoint{76.459442pt}{26.399979pt}}
\pgflineto{\pgfpoint{75.463470pt}{26.470932pt}}
\pgflineto{\pgfpoint{75.463470pt}{26.399979pt}}
\pgfpathclose
\pgfusepath{fill,stroke}
\pgfpathmoveto{\pgfpoint{70.483551pt}{26.399979pt}}
\pgflineto{\pgfpoint{71.479530pt}{26.399979pt}}
\pgflineto{\pgfpoint{71.479530pt}{52.145065pt}}
\pgfpathclose
\pgfusepath{fill,stroke}
\pgfpathmoveto{\pgfpoint{72.475510pt}{31.143661pt}}
\pgflineto{\pgfpoint{71.479530pt}{52.145065pt}}
\pgflineto{\pgfpoint{71.479530pt}{26.399979pt}}
\pgfpathclose
\pgfusepath{fill,stroke}
\pgfpathmoveto{\pgfpoint{72.475510pt}{26.399979pt}}
\pgflineto{\pgfpoint{72.475510pt}{31.143661pt}}
\pgflineto{\pgfpoint{71.479530pt}{26.399979pt}}
\pgfpathclose
\pgfusepath{fill,stroke}
\pgfpathmoveto{\pgfpoint{73.471497pt}{26.399979pt}}
\pgflineto{\pgfpoint{72.475510pt}{31.143661pt}}
\pgflineto{\pgfpoint{72.475510pt}{26.399979pt}}
\pgfpathclose
\pgfusepath{fill,stroke}
\pgfpathmoveto{\pgfpoint{67.495590pt}{26.399979pt}}
\pgflineto{\pgfpoint{68.491577pt}{26.399979pt}}
\pgflineto{\pgfpoint{68.491577pt}{26.687355pt}}
\pgfpathclose
\pgfusepath{fill,stroke}
\pgfpathmoveto{\pgfpoint{69.487564pt}{27.948082pt}}
\pgflineto{\pgfpoint{68.491577pt}{26.687355pt}}
\pgflineto{\pgfpoint{68.491577pt}{26.399979pt}}
\pgfpathclose
\pgfusepath{fill,stroke}
\pgfpathmoveto{\pgfpoint{69.487564pt}{26.399979pt}}
\pgflineto{\pgfpoint{69.487564pt}{27.948082pt}}
\pgflineto{\pgfpoint{68.491577pt}{26.399979pt}}
\pgfpathclose
\pgfusepath{fill,stroke}
\pgfpathmoveto{\pgfpoint{70.483551pt}{26.399979pt}}
\pgflineto{\pgfpoint{69.487564pt}{27.948082pt}}
\pgflineto{\pgfpoint{69.487564pt}{26.399979pt}}
\pgfpathclose
\pgfusepath{fill,stroke}
\pgfpathmoveto{\pgfpoint{63.511658pt}{26.399979pt}}
\pgflineto{\pgfpoint{64.507637pt}{26.399979pt}}
\pgflineto{\pgfpoint{64.507637pt}{26.602051pt}}
\pgfpathclose
\pgfusepath{fill,stroke}
\pgfpathmoveto{\pgfpoint{65.503624pt}{27.212898pt}}
\pgflineto{\pgfpoint{64.507637pt}{26.602051pt}}
\pgflineto{\pgfpoint{64.507637pt}{26.399979pt}}
\pgfpathclose
\pgfusepath{fill,stroke}
\pgfpathmoveto{\pgfpoint{65.503624pt}{26.399979pt}}
\pgflineto{\pgfpoint{65.503624pt}{27.212898pt}}
\pgflineto{\pgfpoint{64.507637pt}{26.399979pt}}
\pgfpathclose
\pgfusepath{fill,stroke}
\pgfpathmoveto{\pgfpoint{66.499619pt}{29.321190pt}}
\pgflineto{\pgfpoint{65.503624pt}{27.212898pt}}
\pgflineto{\pgfpoint{65.503624pt}{26.399979pt}}
\pgfpathclose
\pgfusepath{fill,stroke}
\pgfpathmoveto{\pgfpoint{66.499619pt}{26.399979pt}}
\pgflineto{\pgfpoint{66.499619pt}{29.321190pt}}
\pgflineto{\pgfpoint{65.503624pt}{26.399979pt}}
\pgfpathclose
\pgfusepath{fill,stroke}
\pgfpathmoveto{\pgfpoint{67.495590pt}{26.399979pt}}
\pgflineto{\pgfpoint{66.499619pt}{29.321190pt}}
\pgflineto{\pgfpoint{66.499619pt}{26.399979pt}}
\pgfpathclose
\pgfusepath{fill,stroke}
\pgfpathmoveto{\pgfpoint{61.519691pt}{26.399979pt}}
\pgflineto{\pgfpoint{62.515678pt}{26.399979pt}}
\pgflineto{\pgfpoint{62.515678pt}{26.488380pt}}
\pgfpathclose
\pgfusepath{fill,stroke}
\pgfpathmoveto{\pgfpoint{63.511658pt}{26.399979pt}}
\pgflineto{\pgfpoint{62.515678pt}{26.488380pt}}
\pgflineto{\pgfpoint{62.515678pt}{26.399979pt}}
\pgfpathclose
\pgfusepath{fill,stroke}
\pgfpathmoveto{\pgfpoint{57.535751pt}{26.399979pt}}
\pgflineto{\pgfpoint{58.531738pt}{26.399979pt}}
\pgflineto{\pgfpoint{58.531738pt}{27.217796pt}}
\pgfpathclose
\pgfusepath{fill,stroke}
\pgfpathmoveto{\pgfpoint{59.527725pt}{26.723701pt}}
\pgflineto{\pgfpoint{58.531738pt}{27.217796pt}}
\pgflineto{\pgfpoint{58.531738pt}{26.399979pt}}
\pgfpathclose
\pgfusepath{fill,stroke}
\pgfpathmoveto{\pgfpoint{59.527725pt}{26.399979pt}}
\pgflineto{\pgfpoint{59.527725pt}{26.723701pt}}
\pgflineto{\pgfpoint{58.531738pt}{26.399979pt}}
\pgfpathclose
\pgfusepath{fill,stroke}
\pgfpathmoveto{\pgfpoint{60.523712pt}{26.399979pt}}
\pgflineto{\pgfpoint{59.527725pt}{26.723701pt}}
\pgflineto{\pgfpoint{59.527725pt}{26.399979pt}}
\pgfpathclose
\pgfusepath{fill,stroke}
\pgfpathmoveto{\pgfpoint{54.547806pt}{26.399979pt}}
\pgflineto{\pgfpoint{55.543785pt}{26.399979pt}}
\pgflineto{\pgfpoint{55.543785pt}{26.577354pt}}
\pgfpathclose
\pgfusepath{fill,stroke}
\pgfpathmoveto{\pgfpoint{56.539772pt}{65.803238pt}}
\pgflineto{\pgfpoint{55.543785pt}{26.577354pt}}
\pgflineto{\pgfpoint{55.543785pt}{26.399979pt}}
\pgfpathclose
\pgfusepath{fill,stroke}
\pgfpathmoveto{\pgfpoint{56.539772pt}{26.399979pt}}
\pgflineto{\pgfpoint{56.539772pt}{65.803238pt}}
\pgflineto{\pgfpoint{55.543785pt}{26.399979pt}}
\pgfpathclose
\pgfusepath{fill,stroke}
\pgfpathmoveto{\pgfpoint{57.535751pt}{26.399979pt}}
\pgflineto{\pgfpoint{56.539772pt}{65.803238pt}}
\pgflineto{\pgfpoint{56.539772pt}{26.399979pt}}
\pgfpathclose
\pgfusepath{fill,stroke}
\pgfpathmoveto{\pgfpoint{52.555840pt}{26.399979pt}}
\pgflineto{\pgfpoint{53.551819pt}{26.399979pt}}
\pgflineto{\pgfpoint{53.551819pt}{32.692734pt}}
\pgfpathclose
\pgfusepath{fill,stroke}
\pgfpathmoveto{\pgfpoint{54.547806pt}{26.399979pt}}
\pgflineto{\pgfpoint{53.551819pt}{32.692734pt}}
\pgflineto{\pgfpoint{53.551819pt}{26.399979pt}}
\pgfpathclose
\pgfusepath{fill,stroke}
\pgfpathmoveto{\pgfpoint{46.579933pt}{26.399979pt}}
\pgflineto{\pgfpoint{47.575912pt}{26.399979pt}}
\pgflineto{\pgfpoint{47.575912pt}{27.559311pt}}
\pgfpathclose
\pgfusepath{fill,stroke}
\pgfpathmoveto{\pgfpoint{48.571899pt}{26.399979pt}}
\pgflineto{\pgfpoint{47.575912pt}{27.559311pt}}
\pgflineto{\pgfpoint{47.575912pt}{26.399979pt}}
\pgfpathclose
\pgfusepath{fill,stroke}
\pgfpathmoveto{\pgfpoint{42.595993pt}{26.399979pt}}
\pgflineto{\pgfpoint{43.591980pt}{26.399979pt}}
\pgflineto{\pgfpoint{43.591980pt}{33.198105pt}}
\pgfpathclose
\pgfusepath{fill,stroke}
\pgfpathmoveto{\pgfpoint{44.587967pt}{26.399979pt}}
\pgflineto{\pgfpoint{43.591980pt}{33.198105pt}}
\pgflineto{\pgfpoint{43.591980pt}{26.399979pt}}
\pgfpathclose
\pgfusepath{fill,stroke}
\color[rgb]{0.000000,0.000000,0.000000}
\pgfsetlinewidth{0.500000pt}
\pgfsetdash{{16pt}{0pt}}{0pt}
\pgfpathmoveto{\pgfpoint{289.600037pt}{140.777435pt}}
\pgflineto{\pgfpoint{41.600006pt}{140.777435pt}}
\pgfusepath{stroke}
\pgfpathmoveto{\pgfpoint{289.600037pt}{205.577454pt}}
\pgflineto{\pgfpoint{41.600006pt}{205.577454pt}}
\pgfusepath{stroke}
\pgfpathmoveto{\pgfpoint{41.600006pt}{205.577454pt}}
\pgflineto{\pgfpoint{41.600006pt}{140.777435pt}}
\pgfusepath{stroke}
\pgfpathmoveto{\pgfpoint{289.600037pt}{205.577454pt}}
\pgflineto{\pgfpoint{289.600037pt}{140.777435pt}}
\pgfusepath{stroke}
\pgfpathmoveto{\pgfpoint{90.403221pt}{143.249802pt}}
\pgflineto{\pgfpoint{90.403221pt}{140.777435pt}}
\pgfusepath{stroke}
\pgfpathmoveto{\pgfpoint{90.403221pt}{203.105072pt}}
\pgflineto{\pgfpoint{90.403221pt}{205.577454pt}}
\pgfusepath{stroke}
\pgfpathmoveto{\pgfpoint{140.202423pt}{143.249802pt}}
\pgflineto{\pgfpoint{140.202423pt}{140.777435pt}}
\pgfusepath{stroke}
\pgfpathmoveto{\pgfpoint{140.202423pt}{203.105072pt}}
\pgflineto{\pgfpoint{140.202423pt}{205.577454pt}}
\pgfusepath{stroke}
\pgfpathmoveto{\pgfpoint{190.001617pt}{143.249802pt}}
\pgflineto{\pgfpoint{190.001617pt}{140.777435pt}}
\pgfusepath{stroke}
\pgfpathmoveto{\pgfpoint{190.001617pt}{203.105072pt}}
\pgflineto{\pgfpoint{190.001617pt}{205.577454pt}}
\pgfusepath{stroke}
\pgfpathmoveto{\pgfpoint{239.800812pt}{143.249802pt}}
\pgflineto{\pgfpoint{239.800812pt}{140.777435pt}}
\pgfusepath{stroke}
\pgfpathmoveto{\pgfpoint{239.800812pt}{203.105072pt}}
\pgflineto{\pgfpoint{239.800812pt}{205.577454pt}}
\pgfusepath{stroke}
\pgfpathmoveto{\pgfpoint{289.600037pt}{143.249802pt}}
\pgflineto{\pgfpoint{289.600037pt}{140.777435pt}}
\pgfusepath{stroke}
\pgfpathmoveto{\pgfpoint{289.600037pt}{203.105072pt}}
\pgflineto{\pgfpoint{289.600037pt}{205.577454pt}}
\pgfusepath{stroke}
{
\pgftransformshift{\pgfpoint{90.403229pt}{135.792816pt}}
\pgfnode{rectangle}{north}{\fontsize{10}{0}\selectfont\textcolor[rgb]{0,0,0}{{50}}}{}{\pgfusepath{discard}}}
{
\pgftransformshift{\pgfpoint{140.202423pt}{135.792816pt}}
\pgfnode{rectangle}{north}{\fontsize{10}{0}\selectfont\textcolor[rgb]{0,0,0}{{100}}}{}{\pgfusepath{discard}}}
{
\pgftransformshift{\pgfpoint{190.001617pt}{135.792816pt}}
\pgfnode{rectangle}{north}{\fontsize{10}{0}\selectfont\textcolor[rgb]{0,0,0}{{150}}}{}{\pgfusepath{discard}}}
{
\pgftransformshift{\pgfpoint{239.800812pt}{135.792816pt}}
\pgfnode{rectangle}{north}{\fontsize{10}{0}\selectfont\textcolor[rgb]{0,0,0}{{200}}}{}{\pgfusepath{discard}}}
{
\pgftransformshift{\pgfpoint{289.600037pt}{135.792816pt}}
\pgfnode{rectangle}{north}{\fontsize{10}{0}\selectfont\textcolor[rgb]{0,0,0}{{250}}}{}{\pgfusepath{discard}}}
\pgfpathmoveto{\pgfpoint{44.080009pt}{140.777435pt}}
\pgflineto{\pgfpoint{41.600006pt}{140.777435pt}}
\pgfusepath{stroke}
\pgfpathmoveto{\pgfpoint{287.119995pt}{140.777435pt}}
\pgflineto{\pgfpoint{289.600037pt}{140.777435pt}}
\pgfusepath{stroke}
\pgfpathmoveto{\pgfpoint{44.080009pt}{153.737442pt}}
\pgflineto{\pgfpoint{41.600006pt}{153.737442pt}}
\pgfusepath{stroke}
\pgfpathmoveto{\pgfpoint{287.119995pt}{153.737442pt}}
\pgflineto{\pgfpoint{289.600037pt}{153.737442pt}}
\pgfusepath{stroke}
\pgfpathmoveto{\pgfpoint{44.080009pt}{166.697433pt}}
\pgflineto{\pgfpoint{41.600006pt}{166.697433pt}}
\pgfusepath{stroke}
\pgfpathmoveto{\pgfpoint{287.119995pt}{166.697433pt}}
\pgflineto{\pgfpoint{289.600037pt}{166.697433pt}}
\pgfusepath{stroke}
\pgfpathmoveto{\pgfpoint{44.080009pt}{179.657440pt}}
\pgflineto{\pgfpoint{41.600006pt}{179.657440pt}}
\pgfusepath{stroke}
\pgfpathmoveto{\pgfpoint{287.119995pt}{179.657440pt}}
\pgflineto{\pgfpoint{289.600037pt}{179.657440pt}}
\pgfusepath{stroke}
\pgfpathmoveto{\pgfpoint{44.080009pt}{192.617432pt}}
\pgflineto{\pgfpoint{41.600006pt}{192.617432pt}}
\pgfusepath{stroke}
\pgfpathmoveto{\pgfpoint{287.119995pt}{192.617432pt}}
\pgflineto{\pgfpoint{289.600037pt}{192.617432pt}}
\pgfusepath{stroke}
\pgfpathmoveto{\pgfpoint{44.080009pt}{205.577454pt}}
\pgflineto{\pgfpoint{41.600006pt}{205.577454pt}}
\pgfusepath{stroke}
\pgfpathmoveto{\pgfpoint{287.119995pt}{205.577454pt}}
\pgflineto{\pgfpoint{289.600037pt}{205.577454pt}}
\pgfusepath{stroke}
{
\pgftransformshift{\pgfpoint{36.600006pt}{140.777435pt}}
\pgfnode{rectangle}{east}{\fontsize{10}{0}\selectfont\textcolor[rgb]{0,0,0}{{0}}}{}{\pgfusepath{discard}}}
{
\pgftransformshift{\pgfpoint{36.600006pt}{153.737442pt}}
\pgfnode{rectangle}{east}{\fontsize{10}{0}\selectfont\textcolor[rgb]{0,0,0}{{2e+06}}}{}{\pgfusepath{discard}}}
{
\pgftransformshift{\pgfpoint{36.600006pt}{166.697433pt}}
\pgfnode{rectangle}{east}{\fontsize{10}{0}\selectfont\textcolor[rgb]{0,0,0}{{4e+06}}}{}{\pgfusepath{discard}}}
{
\pgftransformshift{\pgfpoint{36.600006pt}{179.657440pt}}
\pgfnode{rectangle}{east}{\fontsize{10}{0}\selectfont\textcolor[rgb]{0,0,0}{{6e+06}}}{}{\pgfusepath{discard}}}
{
\pgftransformshift{\pgfpoint{36.600006pt}{192.617432pt}}
\pgfnode{rectangle}{east}{\fontsize{10}{0}\selectfont\textcolor[rgb]{0,0,0}{{8e+06}}}{}{\pgfusepath{discard}}}
{
\pgftransformshift{\pgfpoint{36.600006pt}{205.577454pt}}
\pgfnode{rectangle}{east}{\fontsize{10}{0}\selectfont\textcolor[rgb]{0,0,0}{{1e+07}}}{}{\pgfusepath{discard}}}
\pgfsetlinewidth{0.000100pt}
\pgfsetdash{}{0pt}
\pgfpathmoveto{\pgfpoint{43.591980pt}{140.777435pt}}
\pgflineto{\pgfpoint{44.587967pt}{140.777435pt}}
\pgfusepath{stroke}
\pgfpathmoveto{\pgfpoint{42.595993pt}{140.777435pt}}
\pgflineto{\pgfpoint{43.591980pt}{140.777435pt}}
\pgfusepath{stroke}
\pgfpathmoveto{\pgfpoint{43.591980pt}{186.324295pt}}
\pgflineto{\pgfpoint{42.595993pt}{140.777435pt}}
\pgfusepath{stroke}
\pgfpathmoveto{\pgfpoint{44.587967pt}{140.777435pt}}
\pgflineto{\pgfpoint{43.591980pt}{186.324295pt}}
\pgfusepath{stroke}
\pgfpathmoveto{\pgfpoint{45.583946pt}{140.777435pt}}
\pgflineto{\pgfpoint{46.579933pt}{140.777435pt}}
\pgfusepath{stroke}
\pgfpathmoveto{\pgfpoint{44.587967pt}{140.777435pt}}
\pgflineto{\pgfpoint{45.583946pt}{140.777435pt}}
\pgfusepath{stroke}
\pgfpathmoveto{\pgfpoint{45.583946pt}{148.946091pt}}
\pgflineto{\pgfpoint{44.587967pt}{140.777435pt}}
\pgfusepath{stroke}
\pgfpathmoveto{\pgfpoint{46.579933pt}{140.777435pt}}
\pgflineto{\pgfpoint{45.583946pt}{148.946091pt}}
\pgfusepath{stroke}
\pgfpathmoveto{\pgfpoint{49.567879pt}{140.777435pt}}
\pgflineto{\pgfpoint{50.563873pt}{140.777435pt}}
\pgfusepath{stroke}
\pgfpathmoveto{\pgfpoint{48.571899pt}{140.777435pt}}
\pgflineto{\pgfpoint{49.567879pt}{140.777435pt}}
\pgfusepath{stroke}
\pgfpathmoveto{\pgfpoint{47.575912pt}{140.777435pt}}
\pgflineto{\pgfpoint{48.571899pt}{140.777435pt}}
\pgfusepath{stroke}
\pgfpathmoveto{\pgfpoint{46.579933pt}{140.777435pt}}
\pgflineto{\pgfpoint{47.575912pt}{140.777435pt}}
\pgfusepath{stroke}
\pgfpathmoveto{\pgfpoint{47.575912pt}{141.646210pt}}
\pgflineto{\pgfpoint{46.579933pt}{140.777435pt}}
\pgfusepath{stroke}
\pgfpathmoveto{\pgfpoint{48.571899pt}{141.530609pt}}
\pgflineto{\pgfpoint{47.575912pt}{141.646210pt}}
\pgfusepath{stroke}
\pgfpathmoveto{\pgfpoint{49.567879pt}{143.458160pt}}
\pgflineto{\pgfpoint{48.571899pt}{141.530609pt}}
\pgfusepath{stroke}
\pgfpathmoveto{\pgfpoint{50.563873pt}{140.777435pt}}
\pgflineto{\pgfpoint{49.567879pt}{143.458160pt}}
\pgfusepath{stroke}
\pgfpathmoveto{\pgfpoint{66.499619pt}{140.777435pt}}
\pgflineto{\pgfpoint{67.495590pt}{140.777435pt}}
\pgfusepath{stroke}
\pgfpathmoveto{\pgfpoint{65.503624pt}{140.777435pt}}
\pgflineto{\pgfpoint{66.499619pt}{140.777435pt}}
\pgfusepath{stroke}
\pgfpathmoveto{\pgfpoint{64.507637pt}{140.777435pt}}
\pgflineto{\pgfpoint{65.503624pt}{140.777435pt}}
\pgfusepath{stroke}
\pgfpathmoveto{\pgfpoint{63.511658pt}{140.777435pt}}
\pgflineto{\pgfpoint{64.507637pt}{140.777435pt}}
\pgfusepath{stroke}
\pgfpathmoveto{\pgfpoint{62.515678pt}{140.777435pt}}
\pgflineto{\pgfpoint{63.511658pt}{140.777435pt}}
\pgfusepath{stroke}
\pgfpathmoveto{\pgfpoint{61.519691pt}{140.777435pt}}
\pgflineto{\pgfpoint{62.515678pt}{140.777435pt}}
\pgfusepath{stroke}
\pgfpathmoveto{\pgfpoint{60.523712pt}{140.777435pt}}
\pgflineto{\pgfpoint{61.519691pt}{140.777435pt}}
\pgfusepath{stroke}
\pgfpathmoveto{\pgfpoint{59.527725pt}{140.777435pt}}
\pgflineto{\pgfpoint{60.523712pt}{140.777435pt}}
\pgfusepath{stroke}
\pgfpathmoveto{\pgfpoint{58.531738pt}{140.777435pt}}
\pgflineto{\pgfpoint{59.527725pt}{140.777435pt}}
\pgfusepath{stroke}
\pgfpathmoveto{\pgfpoint{57.535751pt}{140.777435pt}}
\pgflineto{\pgfpoint{58.531738pt}{140.777435pt}}
\pgfusepath{stroke}
\pgfpathmoveto{\pgfpoint{56.539772pt}{140.777435pt}}
\pgflineto{\pgfpoint{57.535751pt}{140.777435pt}}
\pgfusepath{stroke}
\pgfpathmoveto{\pgfpoint{55.543785pt}{140.777435pt}}
\pgflineto{\pgfpoint{56.539772pt}{140.777435pt}}
\pgfusepath{stroke}
\pgfpathmoveto{\pgfpoint{54.547806pt}{140.777435pt}}
\pgflineto{\pgfpoint{55.543785pt}{140.777435pt}}
\pgfusepath{stroke}
\pgfpathmoveto{\pgfpoint{53.551819pt}{140.777435pt}}
\pgflineto{\pgfpoint{54.547806pt}{140.777435pt}}
\pgfusepath{stroke}
\pgfpathmoveto{\pgfpoint{52.555840pt}{140.777435pt}}
\pgflineto{\pgfpoint{53.551819pt}{140.777435pt}}
\pgfusepath{stroke}
\pgfpathmoveto{\pgfpoint{51.559845pt}{140.777435pt}}
\pgflineto{\pgfpoint{52.555840pt}{140.777435pt}}
\pgfusepath{stroke}
\pgfpathmoveto{\pgfpoint{50.563873pt}{140.777435pt}}
\pgflineto{\pgfpoint{51.559845pt}{140.777435pt}}
\pgfusepath{stroke}
\pgfpathmoveto{\pgfpoint{51.559845pt}{146.765289pt}}
\pgflineto{\pgfpoint{50.563873pt}{140.777435pt}}
\pgfusepath{stroke}
\pgfpathmoveto{\pgfpoint{52.555840pt}{142.530624pt}}
\pgflineto{\pgfpoint{51.559845pt}{146.765289pt}}
\pgfusepath{stroke}
\pgfpathmoveto{\pgfpoint{53.551819pt}{146.137848pt}}
\pgflineto{\pgfpoint{52.555840pt}{142.530624pt}}
\pgfusepath{stroke}
\pgfpathmoveto{\pgfpoint{54.547806pt}{142.199554pt}}
\pgflineto{\pgfpoint{53.551819pt}{146.137848pt}}
\pgfusepath{stroke}
\pgfpathmoveto{\pgfpoint{55.543785pt}{196.136078pt}}
\pgflineto{\pgfpoint{54.547806pt}{142.199554pt}}
\pgfusepath{stroke}
\pgfpathmoveto{\pgfpoint{56.539772pt}{168.281158pt}}
\pgflineto{\pgfpoint{55.543785pt}{196.136078pt}}
\pgfusepath{stroke}
\pgfpathmoveto{\pgfpoint{57.535751pt}{147.868973pt}}
\pgflineto{\pgfpoint{56.539772pt}{168.281158pt}}
\pgfusepath{stroke}
\pgfpathmoveto{\pgfpoint{58.531738pt}{141.160339pt}}
\pgflineto{\pgfpoint{57.535751pt}{147.868973pt}}
\pgfusepath{stroke}
\pgfpathmoveto{\pgfpoint{59.527725pt}{141.894562pt}}
\pgflineto{\pgfpoint{58.531738pt}{141.160339pt}}
\pgfusepath{stroke}
\pgfpathmoveto{\pgfpoint{60.523712pt}{142.042740pt}}
\pgflineto{\pgfpoint{59.527725pt}{141.894562pt}}
\pgfusepath{stroke}
\pgfpathmoveto{\pgfpoint{61.519691pt}{141.419281pt}}
\pgflineto{\pgfpoint{60.523712pt}{142.042740pt}}
\pgfusepath{stroke}
\pgfpathmoveto{\pgfpoint{62.515678pt}{202.358673pt}}
\pgflineto{\pgfpoint{61.519691pt}{141.419281pt}}
\pgfusepath{stroke}
\pgfpathmoveto{\pgfpoint{63.511658pt}{143.879868pt}}
\pgflineto{\pgfpoint{62.515678pt}{202.358673pt}}
\pgfusepath{stroke}
\pgfpathmoveto{\pgfpoint{64.507637pt}{144.233887pt}}
\pgflineto{\pgfpoint{63.511658pt}{143.879868pt}}
\pgfusepath{stroke}
\pgfpathmoveto{\pgfpoint{65.503624pt}{142.070892pt}}
\pgflineto{\pgfpoint{64.507637pt}{144.233887pt}}
\pgfusepath{stroke}
\pgfpathmoveto{\pgfpoint{66.499619pt}{141.913147pt}}
\pgflineto{\pgfpoint{65.503624pt}{142.070892pt}}
\pgfusepath{stroke}
\pgfpathmoveto{\pgfpoint{67.495590pt}{140.777435pt}}
\pgflineto{\pgfpoint{66.499619pt}{141.913147pt}}
\pgfusepath{stroke}
\pgfpathmoveto{\pgfpoint{72.475510pt}{140.777435pt}}
\pgflineto{\pgfpoint{73.471497pt}{140.777435pt}}
\pgfusepath{stroke}
\pgfpathmoveto{\pgfpoint{71.479530pt}{140.777435pt}}
\pgflineto{\pgfpoint{72.475510pt}{140.777435pt}}
\pgfusepath{stroke}
\pgfpathmoveto{\pgfpoint{70.483551pt}{140.777435pt}}
\pgflineto{\pgfpoint{71.479530pt}{140.777435pt}}
\pgfusepath{stroke}
\pgfpathmoveto{\pgfpoint{69.487564pt}{140.777435pt}}
\pgflineto{\pgfpoint{70.483551pt}{140.777435pt}}
\pgfusepath{stroke}
\pgfpathmoveto{\pgfpoint{68.491577pt}{140.777435pt}}
\pgflineto{\pgfpoint{69.487564pt}{140.777435pt}}
\pgfusepath{stroke}
\pgfpathmoveto{\pgfpoint{69.487564pt}{149.515793pt}}
\pgflineto{\pgfpoint{68.491577pt}{140.777435pt}}
\pgfusepath{stroke}
\pgfpathmoveto{\pgfpoint{70.483551pt}{141.832703pt}}
\pgflineto{\pgfpoint{69.487564pt}{149.515793pt}}
\pgfusepath{stroke}
\pgfpathmoveto{\pgfpoint{71.479530pt}{190.848999pt}}
\pgflineto{\pgfpoint{70.483551pt}{141.832703pt}}
\pgfusepath{stroke}
\pgfpathmoveto{\pgfpoint{72.475510pt}{148.664902pt}}
\pgflineto{\pgfpoint{71.479530pt}{190.848999pt}}
\pgfusepath{stroke}
\pgfpathmoveto{\pgfpoint{73.471497pt}{140.777435pt}}
\pgflineto{\pgfpoint{72.475510pt}{148.664902pt}}
\pgfusepath{stroke}
\pgfpathmoveto{\pgfpoint{77.455437pt}{140.777435pt}}
\pgflineto{\pgfpoint{78.451424pt}{140.777435pt}}
\pgfusepath{stroke}
\pgfpathmoveto{\pgfpoint{76.459442pt}{140.777435pt}}
\pgflineto{\pgfpoint{77.455437pt}{140.777435pt}}
\pgfusepath{stroke}
\pgfpathmoveto{\pgfpoint{75.463470pt}{140.777435pt}}
\pgflineto{\pgfpoint{76.459442pt}{140.777435pt}}
\pgfusepath{stroke}
\pgfpathmoveto{\pgfpoint{74.467484pt}{140.777435pt}}
\pgflineto{\pgfpoint{75.463470pt}{140.777435pt}}
\pgfusepath{stroke}
\pgfpathmoveto{\pgfpoint{73.471497pt}{140.777435pt}}
\pgflineto{\pgfpoint{74.467484pt}{140.777435pt}}
\pgfusepath{stroke}
\pgfpathmoveto{\pgfpoint{74.467484pt}{141.606003pt}}
\pgflineto{\pgfpoint{73.471497pt}{140.777435pt}}
\pgfusepath{stroke}
\pgfpathmoveto{\pgfpoint{75.463470pt}{145.536850pt}}
\pgflineto{\pgfpoint{74.467484pt}{141.606003pt}}
\pgfusepath{stroke}
\pgfpathmoveto{\pgfpoint{76.459442pt}{143.382019pt}}
\pgflineto{\pgfpoint{75.463470pt}{145.536850pt}}
\pgfusepath{stroke}
\pgfpathmoveto{\pgfpoint{77.455437pt}{142.594910pt}}
\pgflineto{\pgfpoint{76.459442pt}{143.382019pt}}
\pgfusepath{stroke}
\pgfpathmoveto{\pgfpoint{78.451424pt}{140.777435pt}}
\pgflineto{\pgfpoint{77.455437pt}{142.594910pt}}
\pgfusepath{stroke}
\pgfpathmoveto{\pgfpoint{85.423309pt}{140.777435pt}}
\pgflineto{\pgfpoint{86.419289pt}{140.777435pt}}
\pgfusepath{stroke}
\pgfpathmoveto{\pgfpoint{84.427322pt}{140.777435pt}}
\pgflineto{\pgfpoint{85.423309pt}{140.777435pt}}
\pgfusepath{stroke}
\pgfpathmoveto{\pgfpoint{83.431335pt}{140.777435pt}}
\pgflineto{\pgfpoint{84.427322pt}{140.777435pt}}
\pgfusepath{stroke}
\pgfpathmoveto{\pgfpoint{82.435356pt}{140.777435pt}}
\pgflineto{\pgfpoint{83.431335pt}{140.777435pt}}
\pgfusepath{stroke}
\pgfpathmoveto{\pgfpoint{81.439369pt}{140.777435pt}}
\pgflineto{\pgfpoint{82.435356pt}{140.777435pt}}
\pgfusepath{stroke}
\pgfpathmoveto{\pgfpoint{80.443390pt}{140.777435pt}}
\pgflineto{\pgfpoint{81.439369pt}{140.777435pt}}
\pgfusepath{stroke}
\pgfpathmoveto{\pgfpoint{79.447403pt}{140.777435pt}}
\pgflineto{\pgfpoint{80.443390pt}{140.777435pt}}
\pgfusepath{stroke}
\pgfpathmoveto{\pgfpoint{78.451424pt}{140.777435pt}}
\pgflineto{\pgfpoint{79.447403pt}{140.777435pt}}
\pgfusepath{stroke}
\pgfpathmoveto{\pgfpoint{79.447403pt}{164.721649pt}}
\pgflineto{\pgfpoint{78.451424pt}{140.777435pt}}
\pgfusepath{stroke}
\pgfpathmoveto{\pgfpoint{80.443390pt}{141.021835pt}}
\pgflineto{\pgfpoint{79.447403pt}{164.721649pt}}
\pgfusepath{stroke}
\pgfpathmoveto{\pgfpoint{81.439369pt}{141.797729pt}}
\pgflineto{\pgfpoint{80.443390pt}{141.021835pt}}
\pgfusepath{stroke}
\pgfpathmoveto{\pgfpoint{82.435356pt}{141.784378pt}}
\pgflineto{\pgfpoint{81.439369pt}{141.797729pt}}
\pgfusepath{stroke}
\pgfpathmoveto{\pgfpoint{83.431335pt}{147.373245pt}}
\pgflineto{\pgfpoint{82.435356pt}{141.784378pt}}
\pgfusepath{stroke}
\pgfpathmoveto{\pgfpoint{84.427322pt}{140.853500pt}}
\pgflineto{\pgfpoint{83.431335pt}{147.373245pt}}
\pgfusepath{stroke}
\pgfpathmoveto{\pgfpoint{85.423309pt}{140.987396pt}}
\pgflineto{\pgfpoint{84.427322pt}{140.853500pt}}
\pgfusepath{stroke}
\pgfpathmoveto{\pgfpoint{86.419289pt}{140.777435pt}}
\pgflineto{\pgfpoint{85.423309pt}{140.987396pt}}
\pgfusepath{stroke}
\pgfpathmoveto{\pgfpoint{90.403221pt}{140.777435pt}}
\pgflineto{\pgfpoint{91.399208pt}{140.777435pt}}
\pgfusepath{stroke}
\pgfpathmoveto{\pgfpoint{89.407242pt}{140.777435pt}}
\pgflineto{\pgfpoint{90.403221pt}{140.777435pt}}
\pgfusepath{stroke}
\pgfpathmoveto{\pgfpoint{88.411255pt}{140.777435pt}}
\pgflineto{\pgfpoint{89.407242pt}{140.777435pt}}
\pgfusepath{stroke}
\pgfpathmoveto{\pgfpoint{87.415276pt}{140.777435pt}}
\pgflineto{\pgfpoint{88.411255pt}{140.777435pt}}
\pgfusepath{stroke}
\pgfpathmoveto{\pgfpoint{86.419289pt}{140.777435pt}}
\pgflineto{\pgfpoint{87.415276pt}{140.777435pt}}
\pgfusepath{stroke}
\pgfpathmoveto{\pgfpoint{87.415276pt}{152.008636pt}}
\pgflineto{\pgfpoint{86.419289pt}{140.777435pt}}
\pgfusepath{stroke}
\pgfpathmoveto{\pgfpoint{88.411255pt}{163.097900pt}}
\pgflineto{\pgfpoint{87.415276pt}{152.008636pt}}
\pgfusepath{stroke}
\pgfpathmoveto{\pgfpoint{89.407242pt}{147.608887pt}}
\pgflineto{\pgfpoint{88.411255pt}{163.097900pt}}
\pgfusepath{stroke}
\pgfpathmoveto{\pgfpoint{90.403221pt}{189.355881pt}}
\pgflineto{\pgfpoint{89.407242pt}{147.608887pt}}
\pgfusepath{stroke}
\pgfpathmoveto{\pgfpoint{91.399208pt}{140.777435pt}}
\pgflineto{\pgfpoint{90.403221pt}{189.355881pt}}
\pgfusepath{stroke}
\pgfpathmoveto{\pgfpoint{93.391174pt}{140.777435pt}}
\pgflineto{\pgfpoint{94.387161pt}{140.777435pt}}
\pgfusepath{stroke}
\pgfpathmoveto{\pgfpoint{92.395187pt}{140.777435pt}}
\pgflineto{\pgfpoint{93.391174pt}{140.777435pt}}
\pgfusepath{stroke}
\pgfpathmoveto{\pgfpoint{91.399208pt}{140.777435pt}}
\pgflineto{\pgfpoint{92.395187pt}{140.777435pt}}
\pgfusepath{stroke}
\pgfpathmoveto{\pgfpoint{92.395187pt}{140.789795pt}}
\pgflineto{\pgfpoint{91.399208pt}{140.777435pt}}
\pgfusepath{stroke}
\pgfpathmoveto{\pgfpoint{93.391174pt}{141.209564pt}}
\pgflineto{\pgfpoint{92.395187pt}{140.789795pt}}
\pgfusepath{stroke}
\pgfpathmoveto{\pgfpoint{94.387161pt}{140.777435pt}}
\pgflineto{\pgfpoint{93.391174pt}{141.209564pt}}
\pgfusepath{stroke}
\pgfpathmoveto{\pgfpoint{102.355034pt}{140.777435pt}}
\pgflineto{\pgfpoint{103.351013pt}{140.777435pt}}
\pgfusepath{stroke}
\pgfpathmoveto{\pgfpoint{101.359047pt}{140.777435pt}}
\pgflineto{\pgfpoint{102.355034pt}{140.777435pt}}
\pgfusepath{stroke}
\pgfpathmoveto{\pgfpoint{100.363068pt}{140.777435pt}}
\pgflineto{\pgfpoint{101.359047pt}{140.777435pt}}
\pgfusepath{stroke}
\pgfpathmoveto{\pgfpoint{99.367081pt}{140.777435pt}}
\pgflineto{\pgfpoint{100.363068pt}{140.777435pt}}
\pgfusepath{stroke}
\pgfpathmoveto{\pgfpoint{98.371094pt}{140.777435pt}}
\pgflineto{\pgfpoint{99.367081pt}{140.777435pt}}
\pgfusepath{stroke}
\pgfpathmoveto{\pgfpoint{97.375107pt}{140.777435pt}}
\pgflineto{\pgfpoint{98.371094pt}{140.777435pt}}
\pgfusepath{stroke}
\pgfpathmoveto{\pgfpoint{96.379128pt}{140.777435pt}}
\pgflineto{\pgfpoint{97.375107pt}{140.777435pt}}
\pgfusepath{stroke}
\pgfpathmoveto{\pgfpoint{95.383141pt}{140.777435pt}}
\pgflineto{\pgfpoint{96.379128pt}{140.777435pt}}
\pgfusepath{stroke}
\pgfpathmoveto{\pgfpoint{94.387161pt}{140.777435pt}}
\pgflineto{\pgfpoint{95.383141pt}{140.777435pt}}
\pgfusepath{stroke}
\pgfpathmoveto{\pgfpoint{95.383141pt}{141.040619pt}}
\pgflineto{\pgfpoint{94.387161pt}{140.777435pt}}
\pgfusepath{stroke}
\pgfpathmoveto{\pgfpoint{96.379128pt}{141.052612pt}}
\pgflineto{\pgfpoint{95.383141pt}{141.040619pt}}
\pgfusepath{stroke}
\pgfpathmoveto{\pgfpoint{97.375107pt}{142.021271pt}}
\pgflineto{\pgfpoint{96.379128pt}{141.052612pt}}
\pgfusepath{stroke}
\pgfpathmoveto{\pgfpoint{98.371094pt}{187.727631pt}}
\pgflineto{\pgfpoint{97.375107pt}{142.021271pt}}
\pgfusepath{stroke}
\pgfpathmoveto{\pgfpoint{99.367081pt}{143.174622pt}}
\pgflineto{\pgfpoint{98.371094pt}{187.727631pt}}
\pgfusepath{stroke}
\pgfpathmoveto{\pgfpoint{100.363068pt}{142.326782pt}}
\pgflineto{\pgfpoint{99.367081pt}{143.174622pt}}
\pgfusepath{stroke}
\pgfpathmoveto{\pgfpoint{101.359047pt}{141.473938pt}}
\pgflineto{\pgfpoint{100.363068pt}{142.326782pt}}
\pgfusepath{stroke}
\pgfpathmoveto{\pgfpoint{102.355034pt}{153.173050pt}}
\pgflineto{\pgfpoint{101.359047pt}{141.473938pt}}
\pgfusepath{stroke}
\pgfpathmoveto{\pgfpoint{103.351013pt}{140.777435pt}}
\pgflineto{\pgfpoint{102.355034pt}{153.173050pt}}
\pgfusepath{stroke}
\pgfpathmoveto{\pgfpoint{104.347000pt}{140.777435pt}}
\pgflineto{\pgfpoint{105.342987pt}{140.777435pt}}
\pgfusepath{stroke}
\pgfpathmoveto{\pgfpoint{103.351013pt}{140.777435pt}}
\pgflineto{\pgfpoint{104.347000pt}{140.777435pt}}
\pgfusepath{stroke}
\pgfpathmoveto{\pgfpoint{104.347000pt}{152.852020pt}}
\pgflineto{\pgfpoint{103.351013pt}{140.777435pt}}
\pgfusepath{stroke}
\pgfpathmoveto{\pgfpoint{105.342987pt}{140.777435pt}}
\pgflineto{\pgfpoint{104.347000pt}{152.852020pt}}
\pgfusepath{stroke}
\pgfpathmoveto{\pgfpoint{110.322906pt}{140.777435pt}}
\pgflineto{\pgfpoint{111.318893pt}{140.777435pt}}
\pgfusepath{stroke}
\pgfpathmoveto{\pgfpoint{109.326920pt}{140.777435pt}}
\pgflineto{\pgfpoint{110.322906pt}{140.777435pt}}
\pgfusepath{stroke}
\pgfpathmoveto{\pgfpoint{108.330933pt}{140.777435pt}}
\pgflineto{\pgfpoint{109.326920pt}{140.777435pt}}
\pgfusepath{stroke}
\pgfpathmoveto{\pgfpoint{107.334953pt}{140.777435pt}}
\pgflineto{\pgfpoint{108.330933pt}{140.777435pt}}
\pgfusepath{stroke}
\pgfpathmoveto{\pgfpoint{108.330933pt}{156.427597pt}}
\pgflineto{\pgfpoint{107.334953pt}{140.777435pt}}
\pgfusepath{stroke}
\pgfpathmoveto{\pgfpoint{109.326920pt}{140.854156pt}}
\pgflineto{\pgfpoint{108.330933pt}{156.427597pt}}
\pgfusepath{stroke}
\pgfpathmoveto{\pgfpoint{110.322906pt}{143.840607pt}}
\pgflineto{\pgfpoint{109.326920pt}{140.854156pt}}
\pgfusepath{stroke}
\pgfpathmoveto{\pgfpoint{111.318893pt}{140.777435pt}}
\pgflineto{\pgfpoint{110.322906pt}{143.840607pt}}
\pgfusepath{stroke}
\pgfpathmoveto{\pgfpoint{114.306839pt}{140.777435pt}}
\pgflineto{\pgfpoint{115.302826pt}{140.777435pt}}
\pgfusepath{stroke}
\pgfpathmoveto{\pgfpoint{113.310852pt}{140.777435pt}}
\pgflineto{\pgfpoint{114.306839pt}{140.777435pt}}
\pgfusepath{stroke}
\pgfpathmoveto{\pgfpoint{112.314873pt}{140.777435pt}}
\pgflineto{\pgfpoint{113.310852pt}{140.777435pt}}
\pgfusepath{stroke}
\pgfpathmoveto{\pgfpoint{111.318893pt}{140.777435pt}}
\pgflineto{\pgfpoint{112.314873pt}{140.777435pt}}
\pgfusepath{stroke}
\pgfpathmoveto{\pgfpoint{112.314873pt}{142.841614pt}}
\pgflineto{\pgfpoint{111.318893pt}{140.777435pt}}
\pgfusepath{stroke}
\pgfpathmoveto{\pgfpoint{113.310852pt}{141.335541pt}}
\pgflineto{\pgfpoint{112.314873pt}{142.841614pt}}
\pgfusepath{stroke}
\pgfpathmoveto{\pgfpoint{114.306839pt}{143.259003pt}}
\pgflineto{\pgfpoint{113.310852pt}{141.335541pt}}
\pgfusepath{stroke}
\pgfpathmoveto{\pgfpoint{115.302826pt}{140.777435pt}}
\pgflineto{\pgfpoint{114.306839pt}{143.259003pt}}
\pgfusepath{stroke}
\pgfpathmoveto{\pgfpoint{119.286758pt}{140.777435pt}}
\pgflineto{\pgfpoint{120.282745pt}{140.777435pt}}
\pgfusepath{stroke}
\pgfpathmoveto{\pgfpoint{118.290779pt}{140.777435pt}}
\pgflineto{\pgfpoint{119.286758pt}{140.777435pt}}
\pgfusepath{stroke}
\pgfpathmoveto{\pgfpoint{117.294792pt}{140.777435pt}}
\pgflineto{\pgfpoint{118.290779pt}{140.777435pt}}
\pgfusepath{stroke}
\pgfpathmoveto{\pgfpoint{116.298813pt}{140.777435pt}}
\pgflineto{\pgfpoint{117.294792pt}{140.777435pt}}
\pgfusepath{stroke}
\pgfpathmoveto{\pgfpoint{115.302826pt}{140.777435pt}}
\pgflineto{\pgfpoint{116.298813pt}{140.777435pt}}
\pgfusepath{stroke}
\pgfpathmoveto{\pgfpoint{116.298813pt}{141.607086pt}}
\pgflineto{\pgfpoint{115.302826pt}{140.777435pt}}
\pgfusepath{stroke}
\pgfpathmoveto{\pgfpoint{117.294792pt}{143.559708pt}}
\pgflineto{\pgfpoint{116.298813pt}{141.607086pt}}
\pgfusepath{stroke}
\pgfpathmoveto{\pgfpoint{118.290779pt}{140.848618pt}}
\pgflineto{\pgfpoint{117.294792pt}{143.559708pt}}
\pgfusepath{stroke}
\pgfpathmoveto{\pgfpoint{119.286758pt}{140.997116pt}}
\pgflineto{\pgfpoint{118.290779pt}{140.848618pt}}
\pgfusepath{stroke}
\pgfpathmoveto{\pgfpoint{120.282745pt}{140.777435pt}}
\pgflineto{\pgfpoint{119.286758pt}{140.997116pt}}
\pgfusepath{stroke}
\pgfpathmoveto{\pgfpoint{121.278725pt}{140.777435pt}}
\pgflineto{\pgfpoint{122.274712pt}{140.777435pt}}
\pgfusepath{stroke}
\pgfpathmoveto{\pgfpoint{120.282745pt}{140.777435pt}}
\pgflineto{\pgfpoint{121.278725pt}{140.777435pt}}
\pgfusepath{stroke}
\pgfpathmoveto{\pgfpoint{121.278725pt}{141.992584pt}}
\pgflineto{\pgfpoint{120.282745pt}{140.777435pt}}
\pgfusepath{stroke}
\pgfpathmoveto{\pgfpoint{122.274712pt}{140.777435pt}}
\pgflineto{\pgfpoint{121.278725pt}{141.992584pt}}
\pgfusepath{stroke}
\pgfpathmoveto{\pgfpoint{123.270691pt}{140.777435pt}}
\pgflineto{\pgfpoint{124.266678pt}{140.777435pt}}
\pgfusepath{stroke}
\pgfpathmoveto{\pgfpoint{122.274712pt}{140.777435pt}}
\pgflineto{\pgfpoint{123.270691pt}{140.777435pt}}
\pgfusepath{stroke}
\pgfpathmoveto{\pgfpoint{123.270691pt}{141.045074pt}}
\pgflineto{\pgfpoint{122.274712pt}{140.777435pt}}
\pgfusepath{stroke}
\pgfpathmoveto{\pgfpoint{124.266678pt}{140.777435pt}}
\pgflineto{\pgfpoint{123.270691pt}{141.045074pt}}
\pgfusepath{stroke}
\pgfpathmoveto{\pgfpoint{127.254631pt}{140.777435pt}}
\pgflineto{\pgfpoint{128.250610pt}{140.777435pt}}
\pgfusepath{stroke}
\pgfpathmoveto{\pgfpoint{126.258652pt}{140.777435pt}}
\pgflineto{\pgfpoint{127.254631pt}{140.777435pt}}
\pgfusepath{stroke}
\pgfpathmoveto{\pgfpoint{125.262665pt}{140.777435pt}}
\pgflineto{\pgfpoint{126.258652pt}{140.777435pt}}
\pgfusepath{stroke}
\pgfpathmoveto{\pgfpoint{126.258652pt}{141.187393pt}}
\pgflineto{\pgfpoint{125.262665pt}{140.777435pt}}
\pgfusepath{stroke}
\pgfpathmoveto{\pgfpoint{127.254631pt}{158.624695pt}}
\pgflineto{\pgfpoint{126.258652pt}{141.187393pt}}
\pgfusepath{stroke}
\pgfpathmoveto{\pgfpoint{128.250610pt}{140.777435pt}}
\pgflineto{\pgfpoint{127.254631pt}{158.624695pt}}
\pgfusepath{stroke}
\pgfpathmoveto{\pgfpoint{132.234558pt}{140.777435pt}}
\pgflineto{\pgfpoint{133.230530pt}{140.777435pt}}
\pgfusepath{stroke}
\pgfpathmoveto{\pgfpoint{131.238571pt}{140.777435pt}}
\pgflineto{\pgfpoint{132.234558pt}{140.777435pt}}
\pgfusepath{stroke}
\pgfpathmoveto{\pgfpoint{130.242584pt}{140.777435pt}}
\pgflineto{\pgfpoint{131.238571pt}{140.777435pt}}
\pgfusepath{stroke}
\pgfpathmoveto{\pgfpoint{129.246597pt}{140.777435pt}}
\pgflineto{\pgfpoint{130.242584pt}{140.777435pt}}
\pgfusepath{stroke}
\pgfpathmoveto{\pgfpoint{128.250610pt}{140.777435pt}}
\pgflineto{\pgfpoint{129.246597pt}{140.777435pt}}
\pgfusepath{stroke}
\pgfpathmoveto{\pgfpoint{129.246597pt}{146.889862pt}}
\pgflineto{\pgfpoint{128.250610pt}{140.777435pt}}
\pgfusepath{stroke}
\pgfpathmoveto{\pgfpoint{130.242584pt}{141.305695pt}}
\pgflineto{\pgfpoint{129.246597pt}{146.889862pt}}
\pgfusepath{stroke}
\pgfpathmoveto{\pgfpoint{131.238571pt}{142.924377pt}}
\pgflineto{\pgfpoint{130.242584pt}{141.305695pt}}
\pgfusepath{stroke}
\pgfpathmoveto{\pgfpoint{132.234558pt}{140.784973pt}}
\pgflineto{\pgfpoint{131.238571pt}{142.924377pt}}
\pgfusepath{stroke}
\pgfpathmoveto{\pgfpoint{133.230530pt}{140.777435pt}}
\pgflineto{\pgfpoint{132.234558pt}{140.784973pt}}
\pgfusepath{stroke}
\pgfpathmoveto{\pgfpoint{135.222504pt}{140.777435pt}}
\pgflineto{\pgfpoint{136.218475pt}{140.777435pt}}
\pgfusepath{stroke}
\pgfpathmoveto{\pgfpoint{134.226517pt}{140.777435pt}}
\pgflineto{\pgfpoint{135.222504pt}{140.777435pt}}
\pgfusepath{stroke}
\pgfpathmoveto{\pgfpoint{135.222504pt}{141.478424pt}}
\pgflineto{\pgfpoint{134.226517pt}{140.777435pt}}
\pgfusepath{stroke}
\pgfpathmoveto{\pgfpoint{136.218475pt}{140.777435pt}}
\pgflineto{\pgfpoint{135.222504pt}{141.478424pt}}
\pgfusepath{stroke}
\pgfpathmoveto{\pgfpoint{141.198410pt}{140.777435pt}}
\pgflineto{\pgfpoint{142.194382pt}{140.777435pt}}
\pgfusepath{stroke}
\pgfpathmoveto{\pgfpoint{140.202423pt}{140.777435pt}}
\pgflineto{\pgfpoint{141.198410pt}{140.777435pt}}
\pgfusepath{stroke}
\pgfpathmoveto{\pgfpoint{139.206436pt}{140.777435pt}}
\pgflineto{\pgfpoint{140.202423pt}{140.777435pt}}
\pgfusepath{stroke}
\pgfpathmoveto{\pgfpoint{138.210449pt}{140.777435pt}}
\pgflineto{\pgfpoint{139.206436pt}{140.777435pt}}
\pgfusepath{stroke}
\pgfpathmoveto{\pgfpoint{137.214478pt}{140.777435pt}}
\pgflineto{\pgfpoint{138.210449pt}{140.777435pt}}
\pgfusepath{stroke}
\pgfpathmoveto{\pgfpoint{136.218475pt}{140.777435pt}}
\pgflineto{\pgfpoint{137.214478pt}{140.777435pt}}
\pgfusepath{stroke}
\pgfpathmoveto{\pgfpoint{137.214478pt}{144.493912pt}}
\pgflineto{\pgfpoint{136.218475pt}{140.777435pt}}
\pgfusepath{stroke}
\pgfpathmoveto{\pgfpoint{138.210449pt}{140.824677pt}}
\pgflineto{\pgfpoint{137.214478pt}{144.493912pt}}
\pgfusepath{stroke}
\pgfpathmoveto{\pgfpoint{139.206436pt}{155.555008pt}}
\pgflineto{\pgfpoint{138.210449pt}{140.824677pt}}
\pgfusepath{stroke}
\pgfpathmoveto{\pgfpoint{140.202423pt}{141.880554pt}}
\pgflineto{\pgfpoint{139.206436pt}{155.555008pt}}
\pgfusepath{stroke}
\pgfpathmoveto{\pgfpoint{141.198410pt}{141.096649pt}}
\pgflineto{\pgfpoint{140.202423pt}{141.880554pt}}
\pgfusepath{stroke}
\pgfpathmoveto{\pgfpoint{142.194382pt}{140.777435pt}}
\pgflineto{\pgfpoint{141.198410pt}{141.096649pt}}
\pgfusepath{stroke}
\pgfpathmoveto{\pgfpoint{143.190369pt}{140.777435pt}}
\pgflineto{\pgfpoint{144.186356pt}{140.777435pt}}
\pgfusepath{stroke}
\pgfpathmoveto{\pgfpoint{142.194382pt}{140.777435pt}}
\pgflineto{\pgfpoint{143.190369pt}{140.777435pt}}
\pgfusepath{stroke}
\pgfpathmoveto{\pgfpoint{143.190369pt}{141.972321pt}}
\pgflineto{\pgfpoint{142.194382pt}{140.777435pt}}
\pgfusepath{stroke}
\pgfpathmoveto{\pgfpoint{144.186356pt}{140.777435pt}}
\pgflineto{\pgfpoint{143.190369pt}{141.972321pt}}
\pgfusepath{stroke}
\pgfpathmoveto{\pgfpoint{147.174316pt}{140.777435pt}}
\pgflineto{\pgfpoint{148.170288pt}{140.777435pt}}
\pgfusepath{stroke}
\pgfpathmoveto{\pgfpoint{146.178314pt}{140.777435pt}}
\pgflineto{\pgfpoint{147.174316pt}{140.777435pt}}
\pgfusepath{stroke}
\pgfpathmoveto{\pgfpoint{145.182343pt}{140.777435pt}}
\pgflineto{\pgfpoint{146.178314pt}{140.777435pt}}
\pgfusepath{stroke}
\pgfpathmoveto{\pgfpoint{144.186356pt}{140.777435pt}}
\pgflineto{\pgfpoint{145.182343pt}{140.777435pt}}
\pgfusepath{stroke}
\pgfpathmoveto{\pgfpoint{145.182343pt}{141.507751pt}}
\pgflineto{\pgfpoint{144.186356pt}{140.777435pt}}
\pgfusepath{stroke}
\pgfpathmoveto{\pgfpoint{146.178314pt}{141.062988pt}}
\pgflineto{\pgfpoint{145.182343pt}{141.507751pt}}
\pgfusepath{stroke}
\pgfpathmoveto{\pgfpoint{147.174316pt}{143.409790pt}}
\pgflineto{\pgfpoint{146.178314pt}{141.062988pt}}
\pgfusepath{stroke}
\pgfpathmoveto{\pgfpoint{148.170288pt}{140.777435pt}}
\pgflineto{\pgfpoint{147.174316pt}{143.409790pt}}
\pgfusepath{stroke}
\pgfpathmoveto{\pgfpoint{153.150208pt}{140.777435pt}}
\pgflineto{\pgfpoint{154.146194pt}{140.777435pt}}
\pgfusepath{stroke}
\pgfpathmoveto{\pgfpoint{152.154221pt}{140.777435pt}}
\pgflineto{\pgfpoint{153.150208pt}{140.777435pt}}
\pgfusepath{stroke}
\pgfpathmoveto{\pgfpoint{151.158249pt}{140.777435pt}}
\pgflineto{\pgfpoint{152.154221pt}{140.777435pt}}
\pgfusepath{stroke}
\pgfpathmoveto{\pgfpoint{150.162262pt}{140.777435pt}}
\pgflineto{\pgfpoint{151.158249pt}{140.777435pt}}
\pgfusepath{stroke}
\pgfpathmoveto{\pgfpoint{151.158249pt}{141.561523pt}}
\pgflineto{\pgfpoint{150.162262pt}{140.777435pt}}
\pgfusepath{stroke}
\pgfpathmoveto{\pgfpoint{152.154221pt}{144.148651pt}}
\pgflineto{\pgfpoint{151.158249pt}{141.561523pt}}
\pgfusepath{stroke}
\pgfpathmoveto{\pgfpoint{153.150208pt}{143.520859pt}}
\pgflineto{\pgfpoint{152.154221pt}{144.148651pt}}
\pgfusepath{stroke}
\pgfpathmoveto{\pgfpoint{154.146194pt}{140.777435pt}}
\pgflineto{\pgfpoint{153.150208pt}{143.520859pt}}
\pgfusepath{stroke}
\pgfpathmoveto{\pgfpoint{155.142181pt}{140.777435pt}}
\pgflineto{\pgfpoint{156.138168pt}{140.777435pt}}
\pgfusepath{stroke}
\pgfpathmoveto{\pgfpoint{154.146194pt}{140.777435pt}}
\pgflineto{\pgfpoint{155.142181pt}{140.777435pt}}
\pgfusepath{stroke}
\pgfpathmoveto{\pgfpoint{155.142181pt}{195.815277pt}}
\pgflineto{\pgfpoint{154.146194pt}{140.777435pt}}
\pgfusepath{stroke}
\pgfpathmoveto{\pgfpoint{156.138168pt}{140.777435pt}}
\pgflineto{\pgfpoint{155.142181pt}{195.815277pt}}
\pgfusepath{stroke}
\pgfpathmoveto{\pgfpoint{158.130127pt}{140.777435pt}}
\pgflineto{\pgfpoint{159.126114pt}{140.777435pt}}
\pgfusepath{stroke}
\pgfpathmoveto{\pgfpoint{157.134155pt}{140.777435pt}}
\pgflineto{\pgfpoint{158.130127pt}{140.777435pt}}
\pgfusepath{stroke}
\pgfpathmoveto{\pgfpoint{156.138168pt}{140.777435pt}}
\pgflineto{\pgfpoint{157.134155pt}{140.777435pt}}
\pgfusepath{stroke}
\pgfpathmoveto{\pgfpoint{157.134155pt}{140.926575pt}}
\pgflineto{\pgfpoint{156.138168pt}{140.777435pt}}
\pgfusepath{stroke}
\pgfpathmoveto{\pgfpoint{158.130127pt}{140.979492pt}}
\pgflineto{\pgfpoint{157.134155pt}{140.926575pt}}
\pgfusepath{stroke}
\pgfpathmoveto{\pgfpoint{159.126114pt}{140.777435pt}}
\pgflineto{\pgfpoint{158.130127pt}{140.979492pt}}
\pgfusepath{stroke}
\pgfpathmoveto{\pgfpoint{162.114075pt}{140.777435pt}}
\pgflineto{\pgfpoint{163.110062pt}{140.777435pt}}
\pgfusepath{stroke}
\pgfpathmoveto{\pgfpoint{161.118088pt}{140.777435pt}}
\pgflineto{\pgfpoint{162.114075pt}{140.777435pt}}
\pgfusepath{stroke}
\pgfpathmoveto{\pgfpoint{160.122101pt}{140.777435pt}}
\pgflineto{\pgfpoint{161.118088pt}{140.777435pt}}
\pgfusepath{stroke}
\pgfpathmoveto{\pgfpoint{159.126114pt}{140.777435pt}}
\pgflineto{\pgfpoint{160.122101pt}{140.777435pt}}
\pgfusepath{stroke}
\pgfpathmoveto{\pgfpoint{160.122101pt}{158.193100pt}}
\pgflineto{\pgfpoint{159.126114pt}{140.777435pt}}
\pgfusepath{stroke}
\pgfpathmoveto{\pgfpoint{161.118088pt}{140.795273pt}}
\pgflineto{\pgfpoint{160.122101pt}{158.193100pt}}
\pgfusepath{stroke}
\pgfpathmoveto{\pgfpoint{162.114075pt}{141.404877pt}}
\pgflineto{\pgfpoint{161.118088pt}{140.795273pt}}
\pgfusepath{stroke}
\pgfpathmoveto{\pgfpoint{163.110062pt}{140.777435pt}}
\pgflineto{\pgfpoint{162.114075pt}{141.404877pt}}
\pgfusepath{stroke}
\pgfpathmoveto{\pgfpoint{165.102020pt}{140.777435pt}}
\pgflineto{\pgfpoint{166.098007pt}{140.777435pt}}
\pgfusepath{stroke}
\pgfpathmoveto{\pgfpoint{164.106033pt}{140.777435pt}}
\pgflineto{\pgfpoint{165.102020pt}{140.777435pt}}
\pgfusepath{stroke}
\pgfpathmoveto{\pgfpoint{163.110062pt}{140.777435pt}}
\pgflineto{\pgfpoint{164.106033pt}{140.777435pt}}
\pgfusepath{stroke}
\pgfpathmoveto{\pgfpoint{164.106033pt}{194.127029pt}}
\pgflineto{\pgfpoint{163.110062pt}{140.777435pt}}
\pgfusepath{stroke}
\pgfpathmoveto{\pgfpoint{165.102020pt}{140.828079pt}}
\pgflineto{\pgfpoint{164.106033pt}{194.127029pt}}
\pgfusepath{stroke}
\pgfpathmoveto{\pgfpoint{166.098007pt}{140.777435pt}}
\pgflineto{\pgfpoint{165.102020pt}{140.828079pt}}
\pgfusepath{stroke}
\pgfpathmoveto{\pgfpoint{168.089966pt}{140.777435pt}}
\pgflineto{\pgfpoint{169.085953pt}{140.777435pt}}
\pgfusepath{stroke}
\pgfpathmoveto{\pgfpoint{167.093994pt}{140.777435pt}}
\pgflineto{\pgfpoint{168.089966pt}{140.777435pt}}
\pgfusepath{stroke}
\pgfpathmoveto{\pgfpoint{166.098007pt}{140.777435pt}}
\pgflineto{\pgfpoint{167.093994pt}{140.777435pt}}
\pgfusepath{stroke}
\pgfpathmoveto{\pgfpoint{167.093994pt}{141.434799pt}}
\pgflineto{\pgfpoint{166.098007pt}{140.777435pt}}
\pgfusepath{stroke}
\pgfpathmoveto{\pgfpoint{167.415314pt}{205.642242pt}}
\pgflineto{\pgfpoint{167.093994pt}{141.434799pt}}
\pgfusepath{stroke}
\pgfpathmoveto{\pgfpoint{169.085953pt}{140.777435pt}}
\pgflineto{\pgfpoint{168.762405pt}{205.642242pt}}
\pgfusepath{stroke}
\pgfpathmoveto{\pgfpoint{175.061859pt}{140.777435pt}}
\pgflineto{\pgfpoint{176.057846pt}{140.777435pt}}
\pgfusepath{stroke}
\pgfpathmoveto{\pgfpoint{174.065872pt}{140.777435pt}}
\pgflineto{\pgfpoint{175.061859pt}{140.777435pt}}
\pgfusepath{stroke}
\pgfpathmoveto{\pgfpoint{173.069885pt}{140.777435pt}}
\pgflineto{\pgfpoint{174.065872pt}{140.777435pt}}
\pgfusepath{stroke}
\pgfpathmoveto{\pgfpoint{172.073914pt}{140.777435pt}}
\pgflineto{\pgfpoint{173.069885pt}{140.777435pt}}
\pgfusepath{stroke}
\pgfpathmoveto{\pgfpoint{171.077911pt}{140.777435pt}}
\pgflineto{\pgfpoint{172.073914pt}{140.777435pt}}
\pgfusepath{stroke}
\pgfpathmoveto{\pgfpoint{170.081940pt}{140.777435pt}}
\pgflineto{\pgfpoint{171.077911pt}{140.777435pt}}
\pgfusepath{stroke}
\pgfpathmoveto{\pgfpoint{169.085953pt}{140.777435pt}}
\pgflineto{\pgfpoint{170.081940pt}{140.777435pt}}
\pgfusepath{stroke}
\pgfpathmoveto{\pgfpoint{170.081940pt}{146.720169pt}}
\pgflineto{\pgfpoint{169.085953pt}{140.777435pt}}
\pgfusepath{stroke}
\pgfpathmoveto{\pgfpoint{171.077911pt}{145.686310pt}}
\pgflineto{\pgfpoint{170.081940pt}{146.720169pt}}
\pgfusepath{stroke}
\pgfpathmoveto{\pgfpoint{172.073914pt}{149.736984pt}}
\pgflineto{\pgfpoint{171.077911pt}{145.686310pt}}
\pgfusepath{stroke}
\pgfpathmoveto{\pgfpoint{173.069885pt}{148.531342pt}}
\pgflineto{\pgfpoint{172.073914pt}{149.736984pt}}
\pgfusepath{stroke}
\pgfpathmoveto{\pgfpoint{174.065872pt}{141.515808pt}}
\pgflineto{\pgfpoint{173.069885pt}{148.531342pt}}
\pgfusepath{stroke}
\pgfpathmoveto{\pgfpoint{175.061859pt}{141.323135pt}}
\pgflineto{\pgfpoint{174.065872pt}{141.515808pt}}
\pgfusepath{stroke}
\pgfpathmoveto{\pgfpoint{176.057846pt}{140.777435pt}}
\pgflineto{\pgfpoint{175.061859pt}{141.323135pt}}
\pgfusepath{stroke}
\pgfpathmoveto{\pgfpoint{181.037766pt}{140.777435pt}}
\pgflineto{\pgfpoint{182.033752pt}{140.777435pt}}
\pgfusepath{stroke}
\pgfpathmoveto{\pgfpoint{180.041779pt}{140.777435pt}}
\pgflineto{\pgfpoint{181.037766pt}{140.777435pt}}
\pgfusepath{stroke}
\pgfpathmoveto{\pgfpoint{179.045792pt}{140.777435pt}}
\pgflineto{\pgfpoint{180.041779pt}{140.777435pt}}
\pgfusepath{stroke}
\pgfpathmoveto{\pgfpoint{178.049805pt}{140.777435pt}}
\pgflineto{\pgfpoint{179.045792pt}{140.777435pt}}
\pgfusepath{stroke}
\pgfpathmoveto{\pgfpoint{177.053818pt}{140.777435pt}}
\pgflineto{\pgfpoint{178.049805pt}{140.777435pt}}
\pgfusepath{stroke}
\pgfpathmoveto{\pgfpoint{176.057846pt}{140.777435pt}}
\pgflineto{\pgfpoint{177.053818pt}{140.777435pt}}
\pgfusepath{stroke}
\pgfpathmoveto{\pgfpoint{177.053818pt}{144.404053pt}}
\pgflineto{\pgfpoint{176.057846pt}{140.777435pt}}
\pgfusepath{stroke}
\pgfpathmoveto{\pgfpoint{178.049805pt}{144.353317pt}}
\pgflineto{\pgfpoint{177.053818pt}{144.404053pt}}
\pgfusepath{stroke}
\pgfpathmoveto{\pgfpoint{179.045792pt}{140.963043pt}}
\pgflineto{\pgfpoint{178.049805pt}{144.353317pt}}
\pgfusepath{stroke}
\pgfpathmoveto{\pgfpoint{180.041779pt}{140.930664pt}}
\pgflineto{\pgfpoint{179.045792pt}{140.963043pt}}
\pgfusepath{stroke}
\pgfpathmoveto{\pgfpoint{181.037766pt}{141.619095pt}}
\pgflineto{\pgfpoint{180.041779pt}{140.930664pt}}
\pgfusepath{stroke}
\pgfpathmoveto{\pgfpoint{182.033752pt}{140.777435pt}}
\pgflineto{\pgfpoint{181.037766pt}{141.619095pt}}
\pgfusepath{stroke}
\pgfpathmoveto{\pgfpoint{183.029724pt}{140.777435pt}}
\pgflineto{\pgfpoint{184.025711pt}{140.777435pt}}
\pgfusepath{stroke}
\pgfpathmoveto{\pgfpoint{182.033752pt}{140.777435pt}}
\pgflineto{\pgfpoint{183.029724pt}{140.777435pt}}
\pgfusepath{stroke}
\pgfpathmoveto{\pgfpoint{183.029724pt}{142.399628pt}}
\pgflineto{\pgfpoint{182.033752pt}{140.777435pt}}
\pgfusepath{stroke}
\pgfpathmoveto{\pgfpoint{184.025711pt}{140.777435pt}}
\pgflineto{\pgfpoint{183.029724pt}{142.399628pt}}
\pgfusepath{stroke}
\pgfpathmoveto{\pgfpoint{187.013672pt}{140.777435pt}}
\pgflineto{\pgfpoint{188.009659pt}{140.777435pt}}
\pgfusepath{stroke}
\pgfpathmoveto{\pgfpoint{186.017685pt}{140.777435pt}}
\pgflineto{\pgfpoint{187.013672pt}{140.777435pt}}
\pgfusepath{stroke}
\pgfpathmoveto{\pgfpoint{187.013672pt}{141.098480pt}}
\pgflineto{\pgfpoint{186.017685pt}{140.777435pt}}
\pgfusepath{stroke}
\pgfpathmoveto{\pgfpoint{188.009659pt}{140.777435pt}}
\pgflineto{\pgfpoint{187.013672pt}{141.098480pt}}
\pgfusepath{stroke}
\pgfpathmoveto{\pgfpoint{198.965469pt}{140.777435pt}}
\pgflineto{\pgfpoint{199.961456pt}{140.777435pt}}
\pgfusepath{stroke}
\pgfpathmoveto{\pgfpoint{197.969498pt}{140.777435pt}}
\pgflineto{\pgfpoint{198.965469pt}{140.777435pt}}
\pgfusepath{stroke}
\pgfpathmoveto{\pgfpoint{196.973511pt}{140.777435pt}}
\pgflineto{\pgfpoint{197.969498pt}{140.777435pt}}
\pgfusepath{stroke}
\pgfpathmoveto{\pgfpoint{195.977524pt}{140.777435pt}}
\pgflineto{\pgfpoint{196.973511pt}{140.777435pt}}
\pgfusepath{stroke}
\pgfpathmoveto{\pgfpoint{194.981537pt}{140.777435pt}}
\pgflineto{\pgfpoint{195.977524pt}{140.777435pt}}
\pgfusepath{stroke}
\pgfpathmoveto{\pgfpoint{193.985565pt}{140.777435pt}}
\pgflineto{\pgfpoint{194.981537pt}{140.777435pt}}
\pgfusepath{stroke}
\pgfpathmoveto{\pgfpoint{192.989563pt}{140.777435pt}}
\pgflineto{\pgfpoint{193.985565pt}{140.777435pt}}
\pgfusepath{stroke}
\pgfpathmoveto{\pgfpoint{191.993591pt}{140.777435pt}}
\pgflineto{\pgfpoint{192.989563pt}{140.777435pt}}
\pgfusepath{stroke}
\pgfpathmoveto{\pgfpoint{190.997604pt}{140.777435pt}}
\pgflineto{\pgfpoint{191.993591pt}{140.777435pt}}
\pgfusepath{stroke}
\pgfpathmoveto{\pgfpoint{190.001617pt}{140.777435pt}}
\pgflineto{\pgfpoint{190.997604pt}{140.777435pt}}
\pgfusepath{stroke}
\pgfpathmoveto{\pgfpoint{189.005630pt}{140.777435pt}}
\pgflineto{\pgfpoint{190.001617pt}{140.777435pt}}
\pgfusepath{stroke}
\pgfpathmoveto{\pgfpoint{188.009659pt}{140.777435pt}}
\pgflineto{\pgfpoint{189.005630pt}{140.777435pt}}
\pgfusepath{stroke}
\pgfpathmoveto{\pgfpoint{189.005630pt}{140.805771pt}}
\pgflineto{\pgfpoint{188.009659pt}{140.777435pt}}
\pgfusepath{stroke}
\pgfpathmoveto{\pgfpoint{190.001617pt}{149.564423pt}}
\pgflineto{\pgfpoint{189.005630pt}{140.805771pt}}
\pgfusepath{stroke}
\pgfpathmoveto{\pgfpoint{190.997604pt}{140.863358pt}}
\pgflineto{\pgfpoint{190.001617pt}{149.564423pt}}
\pgfusepath{stroke}
\pgfpathmoveto{\pgfpoint{191.993591pt}{142.184387pt}}
\pgflineto{\pgfpoint{190.997604pt}{140.863358pt}}
\pgfusepath{stroke}
\pgfpathmoveto{\pgfpoint{192.989563pt}{195.151352pt}}
\pgflineto{\pgfpoint{191.993591pt}{142.184387pt}}
\pgfusepath{stroke}
\pgfpathmoveto{\pgfpoint{193.985565pt}{141.529373pt}}
\pgflineto{\pgfpoint{192.989563pt}{195.151352pt}}
\pgfusepath{stroke}
\pgfpathmoveto{\pgfpoint{194.981537pt}{140.831909pt}}
\pgflineto{\pgfpoint{193.985565pt}{141.529373pt}}
\pgfusepath{stroke}
\pgfpathmoveto{\pgfpoint{195.977524pt}{149.854721pt}}
\pgflineto{\pgfpoint{194.981537pt}{140.831909pt}}
\pgfusepath{stroke}
\pgfpathmoveto{\pgfpoint{196.973511pt}{144.710114pt}}
\pgflineto{\pgfpoint{195.977524pt}{149.854721pt}}
\pgfusepath{stroke}
\pgfpathmoveto{\pgfpoint{197.969498pt}{144.257416pt}}
\pgflineto{\pgfpoint{196.973511pt}{144.710114pt}}
\pgfusepath{stroke}
\pgfpathmoveto{\pgfpoint{198.965469pt}{197.820709pt}}
\pgflineto{\pgfpoint{197.969498pt}{144.257416pt}}
\pgfusepath{stroke}
\pgfpathmoveto{\pgfpoint{199.961456pt}{140.777435pt}}
\pgflineto{\pgfpoint{198.965469pt}{197.820709pt}}
\pgfusepath{stroke}
\pgfpathmoveto{\pgfpoint{201.953430pt}{140.777435pt}}
\pgflineto{\pgfpoint{202.949402pt}{140.777435pt}}
\pgfusepath{stroke}
\pgfpathmoveto{\pgfpoint{200.957443pt}{140.777435pt}}
\pgflineto{\pgfpoint{201.953430pt}{140.777435pt}}
\pgfusepath{stroke}
\pgfpathmoveto{\pgfpoint{199.961456pt}{140.777435pt}}
\pgflineto{\pgfpoint{200.957443pt}{140.777435pt}}
\pgfusepath{stroke}
\pgfpathmoveto{\pgfpoint{200.957443pt}{141.105011pt}}
\pgflineto{\pgfpoint{199.961456pt}{140.777435pt}}
\pgfusepath{stroke}
\pgfpathmoveto{\pgfpoint{201.953430pt}{140.916779pt}}
\pgflineto{\pgfpoint{200.957443pt}{141.105011pt}}
\pgfusepath{stroke}
\pgfpathmoveto{\pgfpoint{202.949402pt}{140.777435pt}}
\pgflineto{\pgfpoint{201.953430pt}{140.916779pt}}
\pgfusepath{stroke}
\pgfpathmoveto{\pgfpoint{205.937347pt}{140.777435pt}}
\pgflineto{\pgfpoint{206.933334pt}{140.777435pt}}
\pgfusepath{stroke}
\pgfpathmoveto{\pgfpoint{204.941376pt}{140.777435pt}}
\pgflineto{\pgfpoint{205.937347pt}{140.777435pt}}
\pgfusepath{stroke}
\pgfpathmoveto{\pgfpoint{205.937347pt}{144.822937pt}}
\pgflineto{\pgfpoint{204.941376pt}{140.777435pt}}
\pgfusepath{stroke}
\pgfpathmoveto{\pgfpoint{206.933334pt}{140.777435pt}}
\pgflineto{\pgfpoint{205.937347pt}{144.822937pt}}
\pgfusepath{stroke}
\pgfpathmoveto{\pgfpoint{215.897217pt}{140.777435pt}}
\pgflineto{\pgfpoint{216.893188pt}{140.777435pt}}
\pgfusepath{stroke}
\pgfpathmoveto{\pgfpoint{214.901215pt}{140.777435pt}}
\pgflineto{\pgfpoint{215.897217pt}{140.777435pt}}
\pgfusepath{stroke}
\pgfpathmoveto{\pgfpoint{213.905228pt}{140.777435pt}}
\pgflineto{\pgfpoint{214.901215pt}{140.777435pt}}
\pgfusepath{stroke}
\pgfpathmoveto{\pgfpoint{212.909241pt}{140.777435pt}}
\pgflineto{\pgfpoint{213.905228pt}{140.777435pt}}
\pgfusepath{stroke}
\pgfpathmoveto{\pgfpoint{211.913269pt}{140.777435pt}}
\pgflineto{\pgfpoint{212.909241pt}{140.777435pt}}
\pgfusepath{stroke}
\pgfpathmoveto{\pgfpoint{210.917267pt}{140.777435pt}}
\pgflineto{\pgfpoint{211.913269pt}{140.777435pt}}
\pgfusepath{stroke}
\pgfpathmoveto{\pgfpoint{209.921295pt}{140.777435pt}}
\pgflineto{\pgfpoint{210.917267pt}{140.777435pt}}
\pgfusepath{stroke}
\pgfpathmoveto{\pgfpoint{208.925323pt}{140.777435pt}}
\pgflineto{\pgfpoint{209.921295pt}{140.777435pt}}
\pgfusepath{stroke}
\pgfpathmoveto{\pgfpoint{207.929337pt}{140.777435pt}}
\pgflineto{\pgfpoint{208.925323pt}{140.777435pt}}
\pgfusepath{stroke}
\pgfpathmoveto{\pgfpoint{208.925323pt}{168.977936pt}}
\pgflineto{\pgfpoint{207.929337pt}{140.777435pt}}
\pgfusepath{stroke}
\pgfpathmoveto{\pgfpoint{209.608246pt}{205.642242pt}}
\pgflineto{\pgfpoint{208.925323pt}{168.977936pt}}
\pgfusepath{stroke}
\pgfpathmoveto{\pgfpoint{210.917267pt}{156.156219pt}}
\pgflineto{\pgfpoint{210.173798pt}{205.642242pt}}
\pgfusepath{stroke}
\pgfpathmoveto{\pgfpoint{211.913269pt}{142.768951pt}}
\pgflineto{\pgfpoint{210.917267pt}{156.156219pt}}
\pgfusepath{stroke}
\pgfpathmoveto{\pgfpoint{212.909241pt}{140.796997pt}}
\pgflineto{\pgfpoint{211.913269pt}{142.768951pt}}
\pgfusepath{stroke}
\pgfpathmoveto{\pgfpoint{213.905228pt}{147.912155pt}}
\pgflineto{\pgfpoint{212.909241pt}{140.796997pt}}
\pgfusepath{stroke}
\pgfpathmoveto{\pgfpoint{214.901215pt}{141.340469pt}}
\pgflineto{\pgfpoint{213.905228pt}{147.912155pt}}
\pgfusepath{stroke}
\pgfpathmoveto{\pgfpoint{215.897217pt}{146.916656pt}}
\pgflineto{\pgfpoint{214.901215pt}{141.340469pt}}
\pgfusepath{stroke}
\pgfpathmoveto{\pgfpoint{216.893188pt}{140.777435pt}}
\pgflineto{\pgfpoint{215.897217pt}{146.916656pt}}
\pgfusepath{stroke}
\pgfpathmoveto{\pgfpoint{217.889160pt}{140.777435pt}}
\pgflineto{\pgfpoint{218.885147pt}{140.777435pt}}
\pgfusepath{stroke}
\pgfpathmoveto{\pgfpoint{216.893188pt}{140.777435pt}}
\pgflineto{\pgfpoint{217.889160pt}{140.777435pt}}
\pgfusepath{stroke}
\pgfpathmoveto{\pgfpoint{217.889160pt}{144.868698pt}}
\pgflineto{\pgfpoint{216.893188pt}{140.777435pt}}
\pgfusepath{stroke}
\pgfpathmoveto{\pgfpoint{218.885147pt}{140.777435pt}}
\pgflineto{\pgfpoint{217.889160pt}{144.868698pt}}
\pgfusepath{stroke}
\pgfpathmoveto{\pgfpoint{219.881134pt}{140.777435pt}}
\pgflineto{\pgfpoint{220.877121pt}{140.777435pt}}
\pgfusepath{stroke}
\pgfpathmoveto{\pgfpoint{218.885147pt}{140.777435pt}}
\pgflineto{\pgfpoint{219.881134pt}{140.777435pt}}
\pgfusepath{stroke}
\pgfpathmoveto{\pgfpoint{219.881134pt}{141.240646pt}}
\pgflineto{\pgfpoint{218.885147pt}{140.777435pt}}
\pgfusepath{stroke}
\pgfpathmoveto{\pgfpoint{220.877121pt}{140.777435pt}}
\pgflineto{\pgfpoint{219.881134pt}{141.240646pt}}
\pgfusepath{stroke}
\pgfpathmoveto{\pgfpoint{225.857040pt}{140.777435pt}}
\pgflineto{\pgfpoint{226.853027pt}{140.777435pt}}
\pgfusepath{stroke}
\pgfpathmoveto{\pgfpoint{224.861053pt}{140.777435pt}}
\pgflineto{\pgfpoint{225.857040pt}{140.777435pt}}
\pgfusepath{stroke}
\pgfpathmoveto{\pgfpoint{223.865082pt}{140.777435pt}}
\pgflineto{\pgfpoint{224.861053pt}{140.777435pt}}
\pgfusepath{stroke}
\pgfpathmoveto{\pgfpoint{222.869080pt}{140.777435pt}}
\pgflineto{\pgfpoint{223.865082pt}{140.777435pt}}
\pgfusepath{stroke}
\pgfpathmoveto{\pgfpoint{223.865082pt}{142.648438pt}}
\pgflineto{\pgfpoint{222.869080pt}{140.777435pt}}
\pgfusepath{stroke}
\pgfpathmoveto{\pgfpoint{224.861053pt}{140.814316pt}}
\pgflineto{\pgfpoint{223.865082pt}{142.648438pt}}
\pgfusepath{stroke}
\pgfpathmoveto{\pgfpoint{225.857040pt}{142.893295pt}}
\pgflineto{\pgfpoint{224.861053pt}{140.814316pt}}
\pgfusepath{stroke}
\pgfpathmoveto{\pgfpoint{226.853027pt}{140.777435pt}}
\pgflineto{\pgfpoint{225.857040pt}{142.893295pt}}
\pgfusepath{stroke}
\pgfpathmoveto{\pgfpoint{231.832932pt}{140.777435pt}}
\pgflineto{\pgfpoint{232.828934pt}{140.777435pt}}
\pgfusepath{stroke}
\pgfpathmoveto{\pgfpoint{230.836945pt}{140.777435pt}}
\pgflineto{\pgfpoint{231.832932pt}{140.777435pt}}
\pgfusepath{stroke}
\pgfpathmoveto{\pgfpoint{229.840973pt}{140.777435pt}}
\pgflineto{\pgfpoint{230.836945pt}{140.777435pt}}
\pgfusepath{stroke}
\pgfpathmoveto{\pgfpoint{228.845001pt}{140.777435pt}}
\pgflineto{\pgfpoint{229.840973pt}{140.777435pt}}
\pgfusepath{stroke}
\pgfpathmoveto{\pgfpoint{227.849014pt}{140.777435pt}}
\pgflineto{\pgfpoint{228.845001pt}{140.777435pt}}
\pgfusepath{stroke}
\pgfpathmoveto{\pgfpoint{226.853027pt}{140.777435pt}}
\pgflineto{\pgfpoint{227.849014pt}{140.777435pt}}
\pgfusepath{stroke}
\pgfpathmoveto{\pgfpoint{227.849014pt}{141.269379pt}}
\pgflineto{\pgfpoint{226.853027pt}{140.777435pt}}
\pgfusepath{stroke}
\pgfpathmoveto{\pgfpoint{228.845001pt}{143.487183pt}}
\pgflineto{\pgfpoint{227.849014pt}{141.269379pt}}
\pgfusepath{stroke}
\pgfpathmoveto{\pgfpoint{229.840973pt}{143.864670pt}}
\pgflineto{\pgfpoint{228.845001pt}{143.487183pt}}
\pgfusepath{stroke}
\pgfpathmoveto{\pgfpoint{230.836945pt}{141.428329pt}}
\pgflineto{\pgfpoint{229.840973pt}{143.864670pt}}
\pgfusepath{stroke}
\pgfpathmoveto{\pgfpoint{231.832932pt}{162.511902pt}}
\pgflineto{\pgfpoint{230.836945pt}{141.428329pt}}
\pgfusepath{stroke}
\pgfpathmoveto{\pgfpoint{232.828934pt}{140.777435pt}}
\pgflineto{\pgfpoint{231.832932pt}{162.511902pt}}
\pgfusepath{stroke}
\pgfpathmoveto{\pgfpoint{238.804825pt}{140.777435pt}}
\pgflineto{\pgfpoint{239.800812pt}{140.777435pt}}
\pgfusepath{stroke}
\pgfpathmoveto{\pgfpoint{237.808838pt}{140.777435pt}}
\pgflineto{\pgfpoint{238.804825pt}{140.777435pt}}
\pgfusepath{stroke}
\pgfpathmoveto{\pgfpoint{236.812866pt}{140.777435pt}}
\pgflineto{\pgfpoint{237.808838pt}{140.777435pt}}
\pgfusepath{stroke}
\pgfpathmoveto{\pgfpoint{235.816864pt}{140.777435pt}}
\pgflineto{\pgfpoint{236.812866pt}{140.777435pt}}
\pgfusepath{stroke}
\pgfpathmoveto{\pgfpoint{234.820892pt}{140.777435pt}}
\pgflineto{\pgfpoint{235.816864pt}{140.777435pt}}
\pgfusepath{stroke}
\pgfpathmoveto{\pgfpoint{233.824921pt}{140.777435pt}}
\pgflineto{\pgfpoint{234.820892pt}{140.777435pt}}
\pgfusepath{stroke}
\pgfpathmoveto{\pgfpoint{232.828934pt}{140.777435pt}}
\pgflineto{\pgfpoint{233.824921pt}{140.777435pt}}
\pgfusepath{stroke}
\pgfpathmoveto{\pgfpoint{233.824921pt}{143.663483pt}}
\pgflineto{\pgfpoint{232.828934pt}{140.777435pt}}
\pgfusepath{stroke}
\pgfpathmoveto{\pgfpoint{234.820892pt}{144.424850pt}}
\pgflineto{\pgfpoint{233.824921pt}{143.663483pt}}
\pgfusepath{stroke}
\pgfpathmoveto{\pgfpoint{235.816864pt}{141.070068pt}}
\pgflineto{\pgfpoint{234.820892pt}{144.424850pt}}
\pgfusepath{stroke}
\pgfpathmoveto{\pgfpoint{236.812866pt}{141.387177pt}}
\pgflineto{\pgfpoint{235.816864pt}{141.070068pt}}
\pgfusepath{stroke}
\pgfpathmoveto{\pgfpoint{237.808838pt}{141.605896pt}}
\pgflineto{\pgfpoint{236.812866pt}{141.387177pt}}
\pgfusepath{stroke}
\pgfpathmoveto{\pgfpoint{238.804825pt}{141.892487pt}}
\pgflineto{\pgfpoint{237.808838pt}{141.605896pt}}
\pgfusepath{stroke}
\pgfpathmoveto{\pgfpoint{239.800812pt}{140.777435pt}}
\pgflineto{\pgfpoint{238.804825pt}{141.892487pt}}
\pgfusepath{stroke}
\pgfpathmoveto{\pgfpoint{240.796814pt}{140.777435pt}}
\pgflineto{\pgfpoint{241.792786pt}{140.777435pt}}
\pgfusepath{stroke}
\pgfpathmoveto{\pgfpoint{239.800812pt}{140.777435pt}}
\pgflineto{\pgfpoint{240.796814pt}{140.777435pt}}
\pgfusepath{stroke}
\pgfpathmoveto{\pgfpoint{240.796814pt}{141.635849pt}}
\pgflineto{\pgfpoint{239.800812pt}{140.777435pt}}
\pgfusepath{stroke}
\pgfpathmoveto{\pgfpoint{241.792786pt}{140.777435pt}}
\pgflineto{\pgfpoint{240.796814pt}{141.635849pt}}
\pgfusepath{stroke}
\pgfpathmoveto{\pgfpoint{243.784744pt}{140.777435pt}}
\pgflineto{\pgfpoint{244.780731pt}{140.777435pt}}
\pgfusepath{stroke}
\pgfpathmoveto{\pgfpoint{242.788757pt}{140.777435pt}}
\pgflineto{\pgfpoint{243.784744pt}{140.777435pt}}
\pgfusepath{stroke}
\pgfpathmoveto{\pgfpoint{241.792786pt}{140.777435pt}}
\pgflineto{\pgfpoint{242.788757pt}{140.777435pt}}
\pgfusepath{stroke}
\pgfpathmoveto{\pgfpoint{242.788757pt}{141.050537pt}}
\pgflineto{\pgfpoint{241.792786pt}{140.777435pt}}
\pgfusepath{stroke}
\pgfpathmoveto{\pgfpoint{243.784744pt}{140.826157pt}}
\pgflineto{\pgfpoint{242.788757pt}{141.050537pt}}
\pgfusepath{stroke}
\pgfpathmoveto{\pgfpoint{244.780731pt}{140.777435pt}}
\pgflineto{\pgfpoint{243.784744pt}{140.826157pt}}
\pgfusepath{stroke}
\pgfpathmoveto{\pgfpoint{250.756638pt}{140.777435pt}}
\pgflineto{\pgfpoint{251.752625pt}{140.777435pt}}
\pgfusepath{stroke}
\pgfpathmoveto{\pgfpoint{249.760651pt}{140.777435pt}}
\pgflineto{\pgfpoint{250.756638pt}{140.777435pt}}
\pgfusepath{stroke}
\pgfpathmoveto{\pgfpoint{248.764679pt}{140.777435pt}}
\pgflineto{\pgfpoint{249.760651pt}{140.777435pt}}
\pgfusepath{stroke}
\pgfpathmoveto{\pgfpoint{247.768677pt}{140.777435pt}}
\pgflineto{\pgfpoint{248.764679pt}{140.777435pt}}
\pgfusepath{stroke}
\pgfpathmoveto{\pgfpoint{246.772705pt}{140.777435pt}}
\pgflineto{\pgfpoint{247.768677pt}{140.777435pt}}
\pgfusepath{stroke}
\pgfpathmoveto{\pgfpoint{245.776718pt}{140.777435pt}}
\pgflineto{\pgfpoint{246.772705pt}{140.777435pt}}
\pgfusepath{stroke}
\pgfpathmoveto{\pgfpoint{244.780731pt}{140.777435pt}}
\pgflineto{\pgfpoint{245.776718pt}{140.777435pt}}
\pgfusepath{stroke}
\pgfpathmoveto{\pgfpoint{245.776718pt}{140.921082pt}}
\pgflineto{\pgfpoint{244.780731pt}{140.777435pt}}
\pgfusepath{stroke}
\pgfpathmoveto{\pgfpoint{246.230515pt}{205.642242pt}}
\pgflineto{\pgfpoint{245.776718pt}{140.921082pt}}
\pgfusepath{stroke}
\pgfpathmoveto{\pgfpoint{247.768677pt}{144.035980pt}}
\pgflineto{\pgfpoint{247.327026pt}{205.642242pt}}
\pgfusepath{stroke}
\pgfpathmoveto{\pgfpoint{248.764679pt}{140.987183pt}}
\pgflineto{\pgfpoint{247.768677pt}{144.035980pt}}
\pgfusepath{stroke}
\pgfpathmoveto{\pgfpoint{249.760651pt}{140.861908pt}}
\pgflineto{\pgfpoint{248.764679pt}{140.987183pt}}
\pgfusepath{stroke}
\pgfpathmoveto{\pgfpoint{250.756638pt}{143.997910pt}}
\pgflineto{\pgfpoint{249.760651pt}{140.861908pt}}
\pgfusepath{stroke}
\pgfpathmoveto{\pgfpoint{251.752625pt}{140.777435pt}}
\pgflineto{\pgfpoint{250.756638pt}{143.997910pt}}
\pgfusepath{stroke}
\pgfpathmoveto{\pgfpoint{254.740570pt}{140.777435pt}}
\pgflineto{\pgfpoint{255.736542pt}{140.777435pt}}
\pgfusepath{stroke}
\pgfpathmoveto{\pgfpoint{253.744598pt}{140.777435pt}}
\pgflineto{\pgfpoint{254.740570pt}{140.777435pt}}
\pgfusepath{stroke}
\pgfpathmoveto{\pgfpoint{252.748611pt}{140.777435pt}}
\pgflineto{\pgfpoint{253.744598pt}{140.777435pt}}
\pgfusepath{stroke}
\pgfpathmoveto{\pgfpoint{253.234100pt}{205.642242pt}}
\pgflineto{\pgfpoint{252.748611pt}{140.777435pt}}
\pgfusepath{stroke}
\pgfpathmoveto{\pgfpoint{255.736542pt}{140.777435pt}}
\pgflineto{\pgfpoint{255.155716pt}{205.642242pt}}
\pgfusepath{stroke}
\pgfpathmoveto{\pgfpoint{261.712463pt}{140.777435pt}}
\pgflineto{\pgfpoint{262.708435pt}{140.777435pt}}
\pgfusepath{stroke}
\pgfpathmoveto{\pgfpoint{260.716492pt}{140.777435pt}}
\pgflineto{\pgfpoint{261.712463pt}{140.777435pt}}
\pgfusepath{stroke}
\pgfpathmoveto{\pgfpoint{259.720490pt}{140.777435pt}}
\pgflineto{\pgfpoint{260.716492pt}{140.777435pt}}
\pgfusepath{stroke}
\pgfpathmoveto{\pgfpoint{258.724518pt}{140.777435pt}}
\pgflineto{\pgfpoint{259.720490pt}{140.777435pt}}
\pgfusepath{stroke}
\pgfpathmoveto{\pgfpoint{257.728516pt}{140.777435pt}}
\pgflineto{\pgfpoint{258.724518pt}{140.777435pt}}
\pgfusepath{stroke}
\pgfpathmoveto{\pgfpoint{256.732544pt}{140.777435pt}}
\pgflineto{\pgfpoint{257.728516pt}{140.777435pt}}
\pgfusepath{stroke}
\pgfpathmoveto{\pgfpoint{255.736542pt}{140.777435pt}}
\pgflineto{\pgfpoint{256.732544pt}{140.777435pt}}
\pgfusepath{stroke}
\pgfpathmoveto{\pgfpoint{256.732544pt}{160.055664pt}}
\pgflineto{\pgfpoint{255.736542pt}{140.777435pt}}
\pgfusepath{stroke}
\pgfpathmoveto{\pgfpoint{257.728516pt}{145.276688pt}}
\pgflineto{\pgfpoint{256.732544pt}{160.055664pt}}
\pgfusepath{stroke}
\pgfpathmoveto{\pgfpoint{258.724518pt}{145.143265pt}}
\pgflineto{\pgfpoint{257.728516pt}{145.276688pt}}
\pgfusepath{stroke}
\pgfpathmoveto{\pgfpoint{259.720490pt}{141.107025pt}}
\pgflineto{\pgfpoint{258.724518pt}{145.143265pt}}
\pgfusepath{stroke}
\pgfpathmoveto{\pgfpoint{260.716492pt}{140.785751pt}}
\pgflineto{\pgfpoint{259.720490pt}{141.107025pt}}
\pgfusepath{stroke}
\pgfpathmoveto{\pgfpoint{261.712463pt}{140.881470pt}}
\pgflineto{\pgfpoint{260.716492pt}{140.785751pt}}
\pgfusepath{stroke}
\pgfpathmoveto{\pgfpoint{262.708435pt}{140.777435pt}}
\pgflineto{\pgfpoint{261.712463pt}{140.881470pt}}
\pgfusepath{stroke}
\pgfpathmoveto{\pgfpoint{272.668274pt}{140.777435pt}}
\pgflineto{\pgfpoint{273.664276pt}{140.777435pt}}
\pgfusepath{stroke}
\pgfpathmoveto{\pgfpoint{271.672302pt}{140.777435pt}}
\pgflineto{\pgfpoint{272.668274pt}{140.777435pt}}
\pgfusepath{stroke}
\pgfpathmoveto{\pgfpoint{270.676331pt}{140.777435pt}}
\pgflineto{\pgfpoint{271.672302pt}{140.777435pt}}
\pgfusepath{stroke}
\pgfpathmoveto{\pgfpoint{269.680328pt}{140.777435pt}}
\pgflineto{\pgfpoint{270.676331pt}{140.777435pt}}
\pgfusepath{stroke}
\pgfpathmoveto{\pgfpoint{268.684326pt}{140.777435pt}}
\pgflineto{\pgfpoint{269.680328pt}{140.777435pt}}
\pgfusepath{stroke}
\pgfpathmoveto{\pgfpoint{267.688354pt}{140.777435pt}}
\pgflineto{\pgfpoint{268.684326pt}{140.777435pt}}
\pgfusepath{stroke}
\pgfpathmoveto{\pgfpoint{266.692383pt}{140.777435pt}}
\pgflineto{\pgfpoint{267.688354pt}{140.777435pt}}
\pgfusepath{stroke}
\pgfpathmoveto{\pgfpoint{265.696411pt}{140.777435pt}}
\pgflineto{\pgfpoint{266.692383pt}{140.777435pt}}
\pgfusepath{stroke}
\pgfpathmoveto{\pgfpoint{264.700409pt}{140.777435pt}}
\pgflineto{\pgfpoint{265.696411pt}{140.777435pt}}
\pgfusepath{stroke}
\pgfpathmoveto{\pgfpoint{263.704407pt}{140.777435pt}}
\pgflineto{\pgfpoint{264.700409pt}{140.777435pt}}
\pgfusepath{stroke}
\pgfpathmoveto{\pgfpoint{262.708435pt}{140.777435pt}}
\pgflineto{\pgfpoint{263.704407pt}{140.777435pt}}
\pgfusepath{stroke}
\pgfpathmoveto{\pgfpoint{263.704407pt}{149.672318pt}}
\pgflineto{\pgfpoint{262.708435pt}{140.777435pt}}
\pgfusepath{stroke}
\pgfpathmoveto{\pgfpoint{264.700409pt}{150.470245pt}}
\pgflineto{\pgfpoint{263.704407pt}{149.672318pt}}
\pgfusepath{stroke}
\pgfpathmoveto{\pgfpoint{265.696411pt}{146.772659pt}}
\pgflineto{\pgfpoint{264.700409pt}{150.470245pt}}
\pgfusepath{stroke}
\pgfpathmoveto{\pgfpoint{266.692383pt}{141.012543pt}}
\pgflineto{\pgfpoint{265.696411pt}{146.772659pt}}
\pgfusepath{stroke}
\pgfpathmoveto{\pgfpoint{267.688354pt}{140.947601pt}}
\pgflineto{\pgfpoint{266.692383pt}{141.012543pt}}
\pgfusepath{stroke}
\pgfpathmoveto{\pgfpoint{268.684326pt}{165.780823pt}}
\pgflineto{\pgfpoint{267.688354pt}{140.947601pt}}
\pgfusepath{stroke}
\pgfpathmoveto{\pgfpoint{269.680328pt}{142.178589pt}}
\pgflineto{\pgfpoint{268.684326pt}{165.780823pt}}
\pgfusepath{stroke}
\pgfpathmoveto{\pgfpoint{270.676331pt}{140.852936pt}}
\pgflineto{\pgfpoint{269.680328pt}{142.178589pt}}
\pgfusepath{stroke}
\pgfpathmoveto{\pgfpoint{271.672302pt}{149.853119pt}}
\pgflineto{\pgfpoint{270.676331pt}{140.852936pt}}
\pgfusepath{stroke}
\pgfpathmoveto{\pgfpoint{272.668274pt}{167.198486pt}}
\pgflineto{\pgfpoint{271.672302pt}{149.853119pt}}
\pgfusepath{stroke}
\pgfpathmoveto{\pgfpoint{273.664276pt}{140.777435pt}}
\pgflineto{\pgfpoint{272.668274pt}{167.198486pt}}
\pgfusepath{stroke}
\pgfpathmoveto{\pgfpoint{282.628113pt}{140.777435pt}}
\pgflineto{\pgfpoint{283.624115pt}{140.777435pt}}
\pgfusepath{stroke}
\pgfpathmoveto{\pgfpoint{281.632141pt}{140.777435pt}}
\pgflineto{\pgfpoint{282.628113pt}{140.777435pt}}
\pgfusepath{stroke}
\pgfpathmoveto{\pgfpoint{280.636139pt}{140.777435pt}}
\pgflineto{\pgfpoint{281.632141pt}{140.777435pt}}
\pgfusepath{stroke}
\pgfpathmoveto{\pgfpoint{279.640167pt}{140.777435pt}}
\pgflineto{\pgfpoint{280.636139pt}{140.777435pt}}
\pgfusepath{stroke}
\pgfpathmoveto{\pgfpoint{278.644196pt}{140.777435pt}}
\pgflineto{\pgfpoint{279.640167pt}{140.777435pt}}
\pgfusepath{stroke}
\pgfpathmoveto{\pgfpoint{277.648193pt}{140.777435pt}}
\pgflineto{\pgfpoint{278.644196pt}{140.777435pt}}
\pgfusepath{stroke}
\pgfpathmoveto{\pgfpoint{276.652222pt}{140.777435pt}}
\pgflineto{\pgfpoint{277.648193pt}{140.777435pt}}
\pgfusepath{stroke}
\pgfpathmoveto{\pgfpoint{275.656250pt}{140.777435pt}}
\pgflineto{\pgfpoint{276.652222pt}{140.777435pt}}
\pgfusepath{stroke}
\pgfpathmoveto{\pgfpoint{276.652222pt}{141.883286pt}}
\pgflineto{\pgfpoint{275.656250pt}{140.777435pt}}
\pgfusepath{stroke}
\pgfpathmoveto{\pgfpoint{277.648193pt}{145.700470pt}}
\pgflineto{\pgfpoint{276.652222pt}{141.883286pt}}
\pgfusepath{stroke}
\pgfpathmoveto{\pgfpoint{278.644196pt}{141.350510pt}}
\pgflineto{\pgfpoint{277.648193pt}{145.700470pt}}
\pgfusepath{stroke}
\pgfpathmoveto{\pgfpoint{279.640167pt}{141.685425pt}}
\pgflineto{\pgfpoint{278.644196pt}{141.350510pt}}
\pgfusepath{stroke}
\pgfpathmoveto{\pgfpoint{280.636139pt}{140.839111pt}}
\pgflineto{\pgfpoint{279.640167pt}{141.685425pt}}
\pgfusepath{stroke}
\pgfpathmoveto{\pgfpoint{281.632141pt}{140.783401pt}}
\pgflineto{\pgfpoint{280.636139pt}{140.839111pt}}
\pgfusepath{stroke}
\pgfpathmoveto{\pgfpoint{282.628113pt}{141.247986pt}}
\pgflineto{\pgfpoint{281.632141pt}{140.783401pt}}
\pgfusepath{stroke}
\pgfpathmoveto{\pgfpoint{283.624115pt}{140.777435pt}}
\pgflineto{\pgfpoint{282.628113pt}{141.247986pt}}
\pgfusepath{stroke}
\pgfpathmoveto{\pgfpoint{285.616089pt}{140.777435pt}}
\pgflineto{\pgfpoint{286.612061pt}{140.777435pt}}
\pgfusepath{stroke}
\pgfpathmoveto{\pgfpoint{284.620087pt}{140.777435pt}}
\pgflineto{\pgfpoint{285.616089pt}{140.777435pt}}
\pgfusepath{stroke}
\pgfpathmoveto{\pgfpoint{283.624115pt}{140.777435pt}}
\pgflineto{\pgfpoint{284.620087pt}{140.777435pt}}
\pgfusepath{stroke}
\pgfpathmoveto{\pgfpoint{284.620087pt}{140.819778pt}}
\pgflineto{\pgfpoint{283.624115pt}{140.777435pt}}
\pgfusepath{stroke}
\pgfpathmoveto{\pgfpoint{285.616089pt}{141.090790pt}}
\pgflineto{\pgfpoint{284.620087pt}{140.819778pt}}
\pgfusepath{stroke}
\pgfpathmoveto{\pgfpoint{286.612061pt}{140.777435pt}}
\pgflineto{\pgfpoint{285.616089pt}{141.090790pt}}
\pgfusepath{stroke}
\pgfpathmoveto{\pgfpoint{288.604004pt}{140.777435pt}}
\pgflineto{\pgfpoint{289.600037pt}{140.777435pt}}
\pgfusepath{stroke}
\pgfpathmoveto{\pgfpoint{287.608032pt}{140.777435pt}}
\pgflineto{\pgfpoint{288.604004pt}{140.777435pt}}
\pgfusepath{stroke}
\pgfpathmoveto{\pgfpoint{288.604004pt}{142.317886pt}}
\pgflineto{\pgfpoint{287.608032pt}{140.777435pt}}
\pgfusepath{stroke}
\pgfpathmoveto{\pgfpoint{289.600037pt}{141.234161pt}}
\pgflineto{\pgfpoint{288.604004pt}{142.317886pt}}
\pgfusepath{stroke}
\pgfpathmoveto{\pgfpoint{289.600037pt}{140.777435pt}}
\pgflineto{\pgfpoint{289.600037pt}{141.234161pt}}
\pgfusepath{stroke}
\color[rgb]{0.000000,0.000000,1.000000}
\pgfsetlinewidth{2.000000pt}
\pgfpathmoveto{\pgfpoint{42.595993pt}{140.777435pt}}
\pgflineto{\pgfpoint{41.600006pt}{140.777435pt}}
\pgfusepath{stroke}
\pgfpathmoveto{\pgfpoint{43.591980pt}{142.291229pt}}
\pgflineto{\pgfpoint{42.595993pt}{140.777435pt}}
\pgfusepath{stroke}
\pgfpathmoveto{\pgfpoint{44.587967pt}{140.777435pt}}
\pgflineto{\pgfpoint{43.591980pt}{142.291229pt}}
\pgfusepath{stroke}
\pgfpathmoveto{\pgfpoint{45.583946pt}{140.910431pt}}
\pgflineto{\pgfpoint{44.587967pt}{140.777435pt}}
\pgfusepath{stroke}
\pgfpathmoveto{\pgfpoint{46.579933pt}{140.777435pt}}
\pgflineto{\pgfpoint{45.583946pt}{140.910431pt}}
\pgfusepath{stroke}
\pgfpathmoveto{\pgfpoint{47.575912pt}{140.789795pt}}
\pgflineto{\pgfpoint{46.579933pt}{140.777435pt}}
\pgfusepath{stroke}
\pgfpathmoveto{\pgfpoint{48.571899pt}{140.788818pt}}
\pgflineto{\pgfpoint{47.575912pt}{140.789795pt}}
\pgfusepath{stroke}
\pgfpathmoveto{\pgfpoint{49.567879pt}{140.835007pt}}
\pgflineto{\pgfpoint{48.571899pt}{140.788818pt}}
\pgfusepath{stroke}
\pgfpathmoveto{\pgfpoint{50.563873pt}{140.777435pt}}
\pgflineto{\pgfpoint{49.567879pt}{140.835007pt}}
\pgfusepath{stroke}
\pgfpathmoveto{\pgfpoint{51.559845pt}{140.826508pt}}
\pgflineto{\pgfpoint{50.563873pt}{140.777435pt}}
\pgfusepath{stroke}
\pgfpathmoveto{\pgfpoint{52.555840pt}{140.814529pt}}
\pgflineto{\pgfpoint{51.559845pt}{140.826508pt}}
\pgfusepath{stroke}
\pgfpathmoveto{\pgfpoint{53.551819pt}{141.029465pt}}
\pgflineto{\pgfpoint{52.555840pt}{140.814529pt}}
\pgfusepath{stroke}
\pgfpathmoveto{\pgfpoint{54.547806pt}{140.800766pt}}
\pgflineto{\pgfpoint{53.551819pt}{141.029465pt}}
\pgfusepath{stroke}
\pgfpathmoveto{\pgfpoint{55.543785pt}{141.743744pt}}
\pgflineto{\pgfpoint{54.547806pt}{140.800766pt}}
\pgfusepath{stroke}
\pgfpathmoveto{\pgfpoint{56.539772pt}{143.031479pt}}
\pgflineto{\pgfpoint{55.543785pt}{141.743744pt}}
\pgfusepath{stroke}
\pgfpathmoveto{\pgfpoint{57.535751pt}{140.964340pt}}
\pgflineto{\pgfpoint{56.539772pt}{143.031479pt}}
\pgfusepath{stroke}
\pgfpathmoveto{\pgfpoint{58.531738pt}{140.780151pt}}
\pgflineto{\pgfpoint{57.535751pt}{140.964340pt}}
\pgfusepath{stroke}
\pgfpathmoveto{\pgfpoint{59.527725pt}{140.855103pt}}
\pgflineto{\pgfpoint{58.531738pt}{140.780151pt}}
\pgfusepath{stroke}
\pgfpathmoveto{\pgfpoint{60.523712pt}{140.787338pt}}
\pgflineto{\pgfpoint{59.527725pt}{140.855103pt}}
\pgfusepath{stroke}
\pgfpathmoveto{\pgfpoint{61.519691pt}{140.782700pt}}
\pgflineto{\pgfpoint{60.523712pt}{140.787338pt}}
\pgfusepath{stroke}
\pgfpathmoveto{\pgfpoint{62.515678pt}{148.013794pt}}
\pgflineto{\pgfpoint{61.519691pt}{140.782700pt}}
\pgfusepath{stroke}
\pgfpathmoveto{\pgfpoint{63.511658pt}{140.808426pt}}
\pgflineto{\pgfpoint{62.515678pt}{148.013794pt}}
\pgfusepath{stroke}
\pgfpathmoveto{\pgfpoint{64.507637pt}{140.848022pt}}
\pgflineto{\pgfpoint{63.511658pt}{140.808426pt}}
\pgfusepath{stroke}
\pgfpathmoveto{\pgfpoint{65.503624pt}{140.813904pt}}
\pgflineto{\pgfpoint{64.507637pt}{140.848022pt}}
\pgfusepath{stroke}
\pgfpathmoveto{\pgfpoint{66.499619pt}{140.822861pt}}
\pgflineto{\pgfpoint{65.503624pt}{140.813904pt}}
\pgfusepath{stroke}
\pgfpathmoveto{\pgfpoint{67.495590pt}{140.777435pt}}
\pgflineto{\pgfpoint{66.499619pt}{140.822861pt}}
\pgfusepath{stroke}
\pgfpathmoveto{\pgfpoint{68.491577pt}{140.777435pt}}
\pgflineto{\pgfpoint{67.495590pt}{140.777435pt}}
\pgfusepath{stroke}
\pgfpathmoveto{\pgfpoint{69.487564pt}{140.957489pt}}
\pgflineto{\pgfpoint{68.491577pt}{140.777435pt}}
\pgfusepath{stroke}
\pgfpathmoveto{\pgfpoint{70.483551pt}{140.814529pt}}
\pgflineto{\pgfpoint{69.487564pt}{140.957489pt}}
\pgfusepath{stroke}
\pgfpathmoveto{\pgfpoint{71.479530pt}{141.474945pt}}
\pgflineto{\pgfpoint{70.483551pt}{140.814529pt}}
\pgfusepath{stroke}
\pgfpathmoveto{\pgfpoint{72.475510pt}{140.931519pt}}
\pgflineto{\pgfpoint{71.479530pt}{141.474945pt}}
\pgfusepath{stroke}
\pgfpathmoveto{\pgfpoint{73.471497pt}{140.777435pt}}
\pgflineto{\pgfpoint{72.475510pt}{140.931519pt}}
\pgfusepath{stroke}
\pgfpathmoveto{\pgfpoint{74.467484pt}{140.789795pt}}
\pgflineto{\pgfpoint{73.471497pt}{140.777435pt}}
\pgfusepath{stroke}
\pgfpathmoveto{\pgfpoint{75.463470pt}{140.988373pt}}
\pgflineto{\pgfpoint{74.467484pt}{140.789795pt}}
\pgfusepath{stroke}
\pgfpathmoveto{\pgfpoint{76.459442pt}{140.912750pt}}
\pgflineto{\pgfpoint{75.463470pt}{140.988373pt}}
\pgfusepath{stroke}
\pgfpathmoveto{\pgfpoint{77.455437pt}{140.810455pt}}
\pgflineto{\pgfpoint{76.459442pt}{140.912750pt}}
\pgfusepath{stroke}
\pgfpathmoveto{\pgfpoint{78.451424pt}{140.777435pt}}
\pgflineto{\pgfpoint{77.455437pt}{140.810455pt}}
\pgfusepath{stroke}
\pgfpathmoveto{\pgfpoint{79.447403pt}{142.779175pt}}
\pgflineto{\pgfpoint{78.451424pt}{140.777435pt}}
\pgfusepath{stroke}
\pgfpathmoveto{\pgfpoint{80.443390pt}{140.781250pt}}
\pgflineto{\pgfpoint{79.447403pt}{142.779175pt}}
\pgfusepath{stroke}
\pgfpathmoveto{\pgfpoint{81.439369pt}{140.786606pt}}
\pgflineto{\pgfpoint{80.443390pt}{140.781250pt}}
\pgfusepath{stroke}
\pgfpathmoveto{\pgfpoint{82.435356pt}{140.785400pt}}
\pgflineto{\pgfpoint{81.439369pt}{140.786606pt}}
\pgfusepath{stroke}
\pgfpathmoveto{\pgfpoint{83.431335pt}{141.342422pt}}
\pgflineto{\pgfpoint{82.435356pt}{140.785400pt}}
\pgfusepath{stroke}
\pgfpathmoveto{\pgfpoint{84.427322pt}{140.777969pt}}
\pgflineto{\pgfpoint{83.431335pt}{141.342422pt}}
\pgfusepath{stroke}
\pgfpathmoveto{\pgfpoint{85.423309pt}{140.779861pt}}
\pgflineto{\pgfpoint{84.427322pt}{140.777969pt}}
\pgfusepath{stroke}
\pgfpathmoveto{\pgfpoint{86.419289pt}{140.777435pt}}
\pgflineto{\pgfpoint{85.423309pt}{140.779861pt}}
\pgfusepath{stroke}
\pgfpathmoveto{\pgfpoint{87.415276pt}{141.062363pt}}
\pgflineto{\pgfpoint{86.419289pt}{140.777435pt}}
\pgfusepath{stroke}
\pgfpathmoveto{\pgfpoint{88.411255pt}{141.826538pt}}
\pgflineto{\pgfpoint{87.415276pt}{141.062363pt}}
\pgfusepath{stroke}
\pgfpathmoveto{\pgfpoint{89.407242pt}{141.208725pt}}
\pgflineto{\pgfpoint{88.411255pt}{141.826538pt}}
\pgfusepath{stroke}
\pgfpathmoveto{\pgfpoint{90.403221pt}{141.961151pt}}
\pgflineto{\pgfpoint{89.407242pt}{141.208725pt}}
\pgfusepath{stroke}
\pgfpathmoveto{\pgfpoint{91.399208pt}{140.777435pt}}
\pgflineto{\pgfpoint{90.403221pt}{141.961151pt}}
\pgfusepath{stroke}
\pgfpathmoveto{\pgfpoint{92.395187pt}{140.777527pt}}
\pgflineto{\pgfpoint{91.399208pt}{140.777435pt}}
\pgfusepath{stroke}
\pgfpathmoveto{\pgfpoint{93.391174pt}{140.785782pt}}
\pgflineto{\pgfpoint{92.395187pt}{140.777527pt}}
\pgfusepath{stroke}
\pgfpathmoveto{\pgfpoint{94.387161pt}{140.777435pt}}
\pgflineto{\pgfpoint{93.391174pt}{140.785782pt}}
\pgfusepath{stroke}
\pgfpathmoveto{\pgfpoint{95.383141pt}{140.780762pt}}
\pgflineto{\pgfpoint{94.387161pt}{140.777435pt}}
\pgfusepath{stroke}
\pgfpathmoveto{\pgfpoint{96.379128pt}{140.781006pt}}
\pgflineto{\pgfpoint{95.383141pt}{140.780762pt}}
\pgfusepath{stroke}
\pgfpathmoveto{\pgfpoint{97.375107pt}{140.790512pt}}
\pgflineto{\pgfpoint{96.379128pt}{140.781006pt}}
\pgfusepath{stroke}
\pgfpathmoveto{\pgfpoint{98.371094pt}{141.396454pt}}
\pgflineto{\pgfpoint{97.375107pt}{140.790512pt}}
\pgfusepath{stroke}
\pgfpathmoveto{\pgfpoint{99.367081pt}{140.828232pt}}
\pgflineto{\pgfpoint{98.371094pt}{141.396454pt}}
\pgfusepath{stroke}
\pgfpathmoveto{\pgfpoint{100.363068pt}{140.788437pt}}
\pgflineto{\pgfpoint{99.367081pt}{140.828232pt}}
\pgfusepath{stroke}
\pgfpathmoveto{\pgfpoint{101.359047pt}{140.782562pt}}
\pgflineto{\pgfpoint{100.363068pt}{140.788437pt}}
\pgfusepath{stroke}
\pgfpathmoveto{\pgfpoint{102.355034pt}{140.893219pt}}
\pgflineto{\pgfpoint{101.359047pt}{140.782562pt}}
\pgfusepath{stroke}
\pgfpathmoveto{\pgfpoint{103.351013pt}{140.777435pt}}
\pgflineto{\pgfpoint{102.355034pt}{140.893219pt}}
\pgfusepath{stroke}
\pgfpathmoveto{\pgfpoint{104.347000pt}{141.194229pt}}
\pgflineto{\pgfpoint{103.351013pt}{140.777435pt}}
\pgfusepath{stroke}
\pgfpathmoveto{\pgfpoint{105.342987pt}{140.777435pt}}
\pgflineto{\pgfpoint{104.347000pt}{141.194229pt}}
\pgfusepath{stroke}
\pgfpathmoveto{\pgfpoint{106.338966pt}{140.777435pt}}
\pgflineto{\pgfpoint{105.342987pt}{140.777435pt}}
\pgfusepath{stroke}
\pgfpathmoveto{\pgfpoint{107.334953pt}{140.777435pt}}
\pgflineto{\pgfpoint{106.338966pt}{140.777435pt}}
\pgfusepath{stroke}
\pgfpathmoveto{\pgfpoint{108.330933pt}{141.164261pt}}
\pgflineto{\pgfpoint{107.334953pt}{140.777435pt}}
\pgfusepath{stroke}
\pgfpathmoveto{\pgfpoint{109.326920pt}{140.777985pt}}
\pgflineto{\pgfpoint{108.330933pt}{141.164261pt}}
\pgfusepath{stroke}
\pgfpathmoveto{\pgfpoint{110.322906pt}{140.811081pt}}
\pgflineto{\pgfpoint{109.326920pt}{140.777985pt}}
\pgfusepath{stroke}
\pgfpathmoveto{\pgfpoint{111.318893pt}{140.777435pt}}
\pgflineto{\pgfpoint{110.322906pt}{140.811081pt}}
\pgfusepath{stroke}
\pgfpathmoveto{\pgfpoint{112.314873pt}{140.798782pt}}
\pgflineto{\pgfpoint{111.318893pt}{140.777435pt}}
\pgfusepath{stroke}
\pgfpathmoveto{\pgfpoint{113.310852pt}{140.789963pt}}
\pgflineto{\pgfpoint{112.314873pt}{140.798782pt}}
\pgfusepath{stroke}
\pgfpathmoveto{\pgfpoint{114.306839pt}{140.825577pt}}
\pgflineto{\pgfpoint{113.310852pt}{140.789963pt}}
\pgfusepath{stroke}
\pgfpathmoveto{\pgfpoint{115.302826pt}{140.777435pt}}
\pgflineto{\pgfpoint{114.306839pt}{140.825577pt}}
\pgfusepath{stroke}
\pgfpathmoveto{\pgfpoint{116.298813pt}{140.784241pt}}
\pgflineto{\pgfpoint{115.302826pt}{140.777435pt}}
\pgfusepath{stroke}
\pgfpathmoveto{\pgfpoint{117.294792pt}{140.853699pt}}
\pgflineto{\pgfpoint{116.298813pt}{140.784241pt}}
\pgfusepath{stroke}
\pgfpathmoveto{\pgfpoint{118.290779pt}{140.778458pt}}
\pgflineto{\pgfpoint{117.294792pt}{140.853699pt}}
\pgfusepath{stroke}
\pgfpathmoveto{\pgfpoint{119.286758pt}{140.778992pt}}
\pgflineto{\pgfpoint{118.290779pt}{140.778458pt}}
\pgfusepath{stroke}
\pgfpathmoveto{\pgfpoint{120.282745pt}{140.777435pt}}
\pgflineto{\pgfpoint{119.286758pt}{140.778992pt}}
\pgfusepath{stroke}
\pgfpathmoveto{\pgfpoint{121.278725pt}{140.821732pt}}
\pgflineto{\pgfpoint{120.282745pt}{140.777435pt}}
\pgfusepath{stroke}
\pgfpathmoveto{\pgfpoint{122.274712pt}{140.777435pt}}
\pgflineto{\pgfpoint{121.278725pt}{140.821732pt}}
\pgfusepath{stroke}
\pgfpathmoveto{\pgfpoint{123.270691pt}{140.781097pt}}
\pgflineto{\pgfpoint{122.274712pt}{140.777435pt}}
\pgfusepath{stroke}
\pgfpathmoveto{\pgfpoint{124.266678pt}{140.777435pt}}
\pgflineto{\pgfpoint{123.270691pt}{140.781097pt}}
\pgfusepath{stroke}
\pgfpathmoveto{\pgfpoint{125.262665pt}{140.777435pt}}
\pgflineto{\pgfpoint{124.266678pt}{140.777435pt}}
\pgfusepath{stroke}
\pgfpathmoveto{\pgfpoint{126.258652pt}{140.787552pt}}
\pgflineto{\pgfpoint{125.262665pt}{140.777435pt}}
\pgfusepath{stroke}
\pgfpathmoveto{\pgfpoint{127.254631pt}{142.211929pt}}
\pgflineto{\pgfpoint{126.258652pt}{140.787552pt}}
\pgfusepath{stroke}
\pgfpathmoveto{\pgfpoint{128.250610pt}{140.777435pt}}
\pgflineto{\pgfpoint{127.254631pt}{142.211929pt}}
\pgfusepath{stroke}
\pgfpathmoveto{\pgfpoint{129.246597pt}{140.893631pt}}
\pgflineto{\pgfpoint{128.250610pt}{140.777435pt}}
\pgfusepath{stroke}
\pgfpathmoveto{\pgfpoint{130.242584pt}{140.790909pt}}
\pgflineto{\pgfpoint{129.246597pt}{140.893631pt}}
\pgfusepath{stroke}
\pgfpathmoveto{\pgfpoint{131.238571pt}{140.817093pt}}
\pgflineto{\pgfpoint{130.242584pt}{140.790909pt}}
\pgfusepath{stroke}
\pgfpathmoveto{\pgfpoint{132.234558pt}{140.777496pt}}
\pgflineto{\pgfpoint{131.238571pt}{140.817093pt}}
\pgfusepath{stroke}
\pgfpathmoveto{\pgfpoint{133.230530pt}{140.777435pt}}
\pgflineto{\pgfpoint{132.234558pt}{140.777496pt}}
\pgfusepath{stroke}
\pgfpathmoveto{\pgfpoint{134.226517pt}{140.777435pt}}
\pgflineto{\pgfpoint{133.230530pt}{140.777435pt}}
\pgfusepath{stroke}
\pgfpathmoveto{\pgfpoint{135.222504pt}{140.782410pt}}
\pgflineto{\pgfpoint{134.226517pt}{140.777435pt}}
\pgfusepath{stroke}
\pgfpathmoveto{\pgfpoint{136.218475pt}{140.777435pt}}
\pgflineto{\pgfpoint{135.222504pt}{140.782410pt}}
\pgfusepath{stroke}
\pgfpathmoveto{\pgfpoint{137.214478pt}{140.868332pt}}
\pgflineto{\pgfpoint{136.218475pt}{140.777435pt}}
\pgfusepath{stroke}
\pgfpathmoveto{\pgfpoint{138.210449pt}{140.777908pt}}
\pgflineto{\pgfpoint{137.214478pt}{140.868332pt}}
\pgfusepath{stroke}
\pgfpathmoveto{\pgfpoint{139.206436pt}{141.098373pt}}
\pgflineto{\pgfpoint{138.210449pt}{140.777908pt}}
\pgfusepath{stroke}
\pgfpathmoveto{\pgfpoint{140.202423pt}{140.810928pt}}
\pgflineto{\pgfpoint{139.206436pt}{141.098373pt}}
\pgfusepath{stroke}
\pgfpathmoveto{\pgfpoint{141.198410pt}{140.779694pt}}
\pgflineto{\pgfpoint{140.202423pt}{140.810928pt}}
\pgfusepath{stroke}
\pgfpathmoveto{\pgfpoint{142.194382pt}{140.777435pt}}
\pgflineto{\pgfpoint{141.198410pt}{140.779694pt}}
\pgfusepath{stroke}
\pgfpathmoveto{\pgfpoint{143.190369pt}{140.818817pt}}
\pgflineto{\pgfpoint{142.194382pt}{140.777435pt}}
\pgfusepath{stroke}
\pgfpathmoveto{\pgfpoint{144.186356pt}{140.777435pt}}
\pgflineto{\pgfpoint{143.190369pt}{140.818817pt}}
\pgfusepath{stroke}
\pgfpathmoveto{\pgfpoint{145.182343pt}{140.783417pt}}
\pgflineto{\pgfpoint{144.186356pt}{140.777435pt}}
\pgfusepath{stroke}
\pgfpathmoveto{\pgfpoint{146.178314pt}{140.779510pt}}
\pgflineto{\pgfpoint{145.182343pt}{140.783417pt}}
\pgfusepath{stroke}
\pgfpathmoveto{\pgfpoint{147.174316pt}{140.817200pt}}
\pgflineto{\pgfpoint{146.178314pt}{140.779510pt}}
\pgfusepath{stroke}
\pgfpathmoveto{\pgfpoint{148.170288pt}{140.777435pt}}
\pgflineto{\pgfpoint{147.174316pt}{140.817200pt}}
\pgfusepath{stroke}
\pgfpathmoveto{\pgfpoint{149.166275pt}{140.777435pt}}
\pgflineto{\pgfpoint{148.170288pt}{140.777435pt}}
\pgfusepath{stroke}
\pgfpathmoveto{\pgfpoint{150.162262pt}{140.777435pt}}
\pgflineto{\pgfpoint{149.166275pt}{140.777435pt}}
\pgfusepath{stroke}
\pgfpathmoveto{\pgfpoint{151.158249pt}{140.784195pt}}
\pgflineto{\pgfpoint{150.162262pt}{140.777435pt}}
\pgfusepath{stroke}
\pgfpathmoveto{\pgfpoint{152.154221pt}{140.816559pt}}
\pgflineto{\pgfpoint{151.158249pt}{140.784195pt}}
\pgfusepath{stroke}
\pgfpathmoveto{\pgfpoint{153.150208pt}{140.826141pt}}
\pgflineto{\pgfpoint{152.154221pt}{140.816559pt}}
\pgfusepath{stroke}
\pgfpathmoveto{\pgfpoint{154.146194pt}{140.777435pt}}
\pgflineto{\pgfpoint{153.150208pt}{140.826141pt}}
\pgfusepath{stroke}
\pgfpathmoveto{\pgfpoint{155.142181pt}{145.339951pt}}
\pgflineto{\pgfpoint{154.146194pt}{140.777435pt}}
\pgfusepath{stroke}
\pgfpathmoveto{\pgfpoint{156.138168pt}{140.777435pt}}
\pgflineto{\pgfpoint{155.142181pt}{145.339951pt}}
\pgfusepath{stroke}
\pgfpathmoveto{\pgfpoint{157.134155pt}{140.779984pt}}
\pgflineto{\pgfpoint{156.138168pt}{140.777435pt}}
\pgfusepath{stroke}
\pgfpathmoveto{\pgfpoint{158.130127pt}{140.780472pt}}
\pgflineto{\pgfpoint{157.134155pt}{140.779984pt}}
\pgfusepath{stroke}
\pgfpathmoveto{\pgfpoint{159.126114pt}{140.777435pt}}
\pgflineto{\pgfpoint{158.130127pt}{140.780472pt}}
\pgfusepath{stroke}
\pgfpathmoveto{\pgfpoint{160.122101pt}{140.912704pt}}
\pgflineto{\pgfpoint{159.126114pt}{140.777435pt}}
\pgfusepath{stroke}
\pgfpathmoveto{\pgfpoint{161.118088pt}{140.777557pt}}
\pgflineto{\pgfpoint{160.122101pt}{140.912704pt}}
\pgfusepath{stroke}
\pgfpathmoveto{\pgfpoint{162.114075pt}{140.788925pt}}
\pgflineto{\pgfpoint{161.118088pt}{140.777557pt}}
\pgfusepath{stroke}
\pgfpathmoveto{\pgfpoint{163.110062pt}{140.777435pt}}
\pgflineto{\pgfpoint{162.114075pt}{140.788925pt}}
\pgfusepath{stroke}
\pgfpathmoveto{\pgfpoint{164.106033pt}{141.622849pt}}
\pgflineto{\pgfpoint{163.110062pt}{140.777435pt}}
\pgfusepath{stroke}
\pgfpathmoveto{\pgfpoint{165.102020pt}{140.777802pt}}
\pgflineto{\pgfpoint{164.106033pt}{141.622849pt}}
\pgfusepath{stroke}
\pgfpathmoveto{\pgfpoint{166.098007pt}{140.777435pt}}
\pgflineto{\pgfpoint{165.102020pt}{140.777802pt}}
\pgfusepath{stroke}
\pgfpathmoveto{\pgfpoint{167.093994pt}{140.785172pt}}
\pgflineto{\pgfpoint{166.098007pt}{140.777435pt}}
\pgfusepath{stroke}
\pgfpathmoveto{\pgfpoint{168.089966pt}{165.120728pt}}
\pgflineto{\pgfpoint{167.093994pt}{140.785172pt}}
\pgfusepath{stroke}
\pgfpathmoveto{\pgfpoint{169.085953pt}{140.777435pt}}
\pgflineto{\pgfpoint{168.089966pt}{165.120728pt}}
\pgfusepath{stroke}
\pgfpathmoveto{\pgfpoint{170.081940pt}{141.018890pt}}
\pgflineto{\pgfpoint{169.085953pt}{140.777435pt}}
\pgfusepath{stroke}
\pgfpathmoveto{\pgfpoint{171.077911pt}{140.882141pt}}
\pgflineto{\pgfpoint{170.081940pt}{141.018890pt}}
\pgfusepath{stroke}
\pgfpathmoveto{\pgfpoint{172.073914pt}{141.041473pt}}
\pgflineto{\pgfpoint{171.077911pt}{140.882141pt}}
\pgfusepath{stroke}
\pgfpathmoveto{\pgfpoint{173.069885pt}{140.914948pt}}
\pgflineto{\pgfpoint{172.073914pt}{141.041473pt}}
\pgfusepath{stroke}
\pgfpathmoveto{\pgfpoint{174.065872pt}{140.782669pt}}
\pgflineto{\pgfpoint{173.069885pt}{140.914948pt}}
\pgfusepath{stroke}
\pgfpathmoveto{\pgfpoint{175.061859pt}{140.783112pt}}
\pgflineto{\pgfpoint{174.065872pt}{140.782669pt}}
\pgfusepath{stroke}
\pgfpathmoveto{\pgfpoint{176.057846pt}{140.777435pt}}
\pgflineto{\pgfpoint{175.061859pt}{140.783112pt}}
\pgfusepath{stroke}
\pgfpathmoveto{\pgfpoint{177.053818pt}{140.905045pt}}
\pgflineto{\pgfpoint{176.057846pt}{140.777435pt}}
\pgfusepath{stroke}
\pgfpathmoveto{\pgfpoint{178.049805pt}{140.866837pt}}
\pgflineto{\pgfpoint{177.053818pt}{140.905045pt}}
\pgfusepath{stroke}
\pgfpathmoveto{\pgfpoint{179.045792pt}{140.781509pt}}
\pgflineto{\pgfpoint{178.049805pt}{140.866837pt}}
\pgfusepath{stroke}
\pgfpathmoveto{\pgfpoint{180.041779pt}{140.779846pt}}
\pgflineto{\pgfpoint{179.045792pt}{140.781509pt}}
\pgfusepath{stroke}
\pgfpathmoveto{\pgfpoint{181.037766pt}{140.783401pt}}
\pgflineto{\pgfpoint{180.041779pt}{140.779846pt}}
\pgfusepath{stroke}
\pgfpathmoveto{\pgfpoint{182.033752pt}{140.777435pt}}
\pgflineto{\pgfpoint{181.037766pt}{140.783401pt}}
\pgfusepath{stroke}
\pgfpathmoveto{\pgfpoint{183.029724pt}{140.817596pt}}
\pgflineto{\pgfpoint{182.033752pt}{140.777435pt}}
\pgfusepath{stroke}
\pgfpathmoveto{\pgfpoint{184.025711pt}{140.777435pt}}
\pgflineto{\pgfpoint{183.029724pt}{140.817596pt}}
\pgfusepath{stroke}
\pgfpathmoveto{\pgfpoint{185.021698pt}{140.777435pt}}
\pgflineto{\pgfpoint{184.025711pt}{140.777435pt}}
\pgfusepath{stroke}
\pgfpathmoveto{\pgfpoint{186.017685pt}{140.777435pt}}
\pgflineto{\pgfpoint{185.021698pt}{140.777435pt}}
\pgfusepath{stroke}
\pgfpathmoveto{\pgfpoint{187.013672pt}{140.781067pt}}
\pgflineto{\pgfpoint{186.017685pt}{140.777435pt}}
\pgfusepath{stroke}
\pgfpathmoveto{\pgfpoint{188.009659pt}{140.777435pt}}
\pgflineto{\pgfpoint{187.013672pt}{140.781067pt}}
\pgfusepath{stroke}
\pgfpathmoveto{\pgfpoint{189.005630pt}{140.777634pt}}
\pgflineto{\pgfpoint{188.009659pt}{140.777435pt}}
\pgfusepath{stroke}
\pgfpathmoveto{\pgfpoint{190.001617pt}{141.032303pt}}
\pgflineto{\pgfpoint{189.005630pt}{140.777634pt}}
\pgfusepath{stroke}
\pgfpathmoveto{\pgfpoint{190.997604pt}{140.778046pt}}
\pgflineto{\pgfpoint{190.001617pt}{141.032303pt}}
\pgfusepath{stroke}
\pgfpathmoveto{\pgfpoint{191.993591pt}{140.805786pt}}
\pgflineto{\pgfpoint{190.997604pt}{140.778046pt}}
\pgfusepath{stroke}
\pgfpathmoveto{\pgfpoint{192.989563pt}{141.541718pt}}
\pgflineto{\pgfpoint{191.993591pt}{140.805786pt}}
\pgfusepath{stroke}
\pgfpathmoveto{\pgfpoint{193.985565pt}{140.790375pt}}
\pgflineto{\pgfpoint{192.989563pt}{141.541718pt}}
\pgfusepath{stroke}
\pgfpathmoveto{\pgfpoint{194.981537pt}{140.778259pt}}
\pgflineto{\pgfpoint{193.985565pt}{140.790375pt}}
\pgfusepath{stroke}
\pgfpathmoveto{\pgfpoint{195.977524pt}{141.314301pt}}
\pgflineto{\pgfpoint{194.981537pt}{140.778259pt}}
\pgfusepath{stroke}
\pgfpathmoveto{\pgfpoint{196.973511pt}{140.944916pt}}
\pgflineto{\pgfpoint{195.977524pt}{141.314301pt}}
\pgfusepath{stroke}
\pgfpathmoveto{\pgfpoint{197.969498pt}{140.809601pt}}
\pgflineto{\pgfpoint{196.973511pt}{140.944916pt}}
\pgfusepath{stroke}
\pgfpathmoveto{\pgfpoint{198.965469pt}{149.605469pt}}
\pgflineto{\pgfpoint{197.969498pt}{140.809601pt}}
\pgfusepath{stroke}
\pgfpathmoveto{\pgfpoint{199.961456pt}{140.777435pt}}
\pgflineto{\pgfpoint{198.965469pt}{149.605469pt}}
\pgfusepath{stroke}
\pgfpathmoveto{\pgfpoint{200.957443pt}{140.786819pt}}
\pgflineto{\pgfpoint{199.961456pt}{140.777435pt}}
\pgfusepath{stroke}
\pgfpathmoveto{\pgfpoint{201.953430pt}{140.778427pt}}
\pgflineto{\pgfpoint{200.957443pt}{140.786819pt}}
\pgfusepath{stroke}
\pgfpathmoveto{\pgfpoint{202.949402pt}{140.777435pt}}
\pgflineto{\pgfpoint{201.953430pt}{140.778427pt}}
\pgfusepath{stroke}
\pgfpathmoveto{\pgfpoint{203.945404pt}{140.777435pt}}
\pgflineto{\pgfpoint{202.949402pt}{140.777435pt}}
\pgfusepath{stroke}
\pgfpathmoveto{\pgfpoint{204.941376pt}{140.777435pt}}
\pgflineto{\pgfpoint{203.945404pt}{140.777435pt}}
\pgfusepath{stroke}
\pgfpathmoveto{\pgfpoint{205.937347pt}{140.854156pt}}
\pgflineto{\pgfpoint{204.941376pt}{140.777435pt}}
\pgfusepath{stroke}
\pgfpathmoveto{\pgfpoint{206.933334pt}{140.777435pt}}
\pgflineto{\pgfpoint{205.937347pt}{140.854156pt}}
\pgfusepath{stroke}
\pgfpathmoveto{\pgfpoint{207.929337pt}{140.777435pt}}
\pgflineto{\pgfpoint{206.933334pt}{140.777435pt}}
\pgfusepath{stroke}
\pgfpathmoveto{\pgfpoint{208.925323pt}{141.508636pt}}
\pgflineto{\pgfpoint{207.929337pt}{140.777435pt}}
\pgfusepath{stroke}
\pgfpathmoveto{\pgfpoint{209.921295pt}{143.205780pt}}
\pgflineto{\pgfpoint{208.925323pt}{141.508636pt}}
\pgfusepath{stroke}
\pgfpathmoveto{\pgfpoint{210.917267pt}{141.750488pt}}
\pgflineto{\pgfpoint{209.921295pt}{143.205780pt}}
\pgfusepath{stroke}
\pgfpathmoveto{\pgfpoint{211.913269pt}{140.869766pt}}
\pgflineto{\pgfpoint{210.917267pt}{141.750488pt}}
\pgfusepath{stroke}
\pgfpathmoveto{\pgfpoint{212.909241pt}{140.777573pt}}
\pgflineto{\pgfpoint{211.913269pt}{140.869766pt}}
\pgfusepath{stroke}
\pgfpathmoveto{\pgfpoint{213.905228pt}{140.910294pt}}
\pgflineto{\pgfpoint{212.909241pt}{140.777573pt}}
\pgfusepath{stroke}
\pgfpathmoveto{\pgfpoint{214.901215pt}{140.781433pt}}
\pgflineto{\pgfpoint{213.905228pt}{140.910294pt}}
\pgfusepath{stroke}
\pgfpathmoveto{\pgfpoint{215.897217pt}{141.024170pt}}
\pgflineto{\pgfpoint{214.901215pt}{140.781433pt}}
\pgfusepath{stroke}
\pgfpathmoveto{\pgfpoint{216.893188pt}{140.777435pt}}
\pgflineto{\pgfpoint{215.897217pt}{141.024170pt}}
\pgfusepath{stroke}
\pgfpathmoveto{\pgfpoint{217.889160pt}{140.945007pt}}
\pgflineto{\pgfpoint{216.893188pt}{140.777435pt}}
\pgfusepath{stroke}
\pgfpathmoveto{\pgfpoint{218.885147pt}{140.777435pt}}
\pgflineto{\pgfpoint{217.889160pt}{140.945007pt}}
\pgfusepath{stroke}
\pgfpathmoveto{\pgfpoint{219.881134pt}{140.780746pt}}
\pgflineto{\pgfpoint{218.885147pt}{140.777435pt}}
\pgfusepath{stroke}
\pgfpathmoveto{\pgfpoint{220.877121pt}{140.777435pt}}
\pgflineto{\pgfpoint{219.881134pt}{140.780746pt}}
\pgfusepath{stroke}
\pgfpathmoveto{\pgfpoint{221.873108pt}{140.777435pt}}
\pgflineto{\pgfpoint{220.877121pt}{140.777435pt}}
\pgfusepath{stroke}
\pgfpathmoveto{\pgfpoint{222.869080pt}{140.777435pt}}
\pgflineto{\pgfpoint{221.873108pt}{140.777435pt}}
\pgfusepath{stroke}
\pgfpathmoveto{\pgfpoint{223.865082pt}{140.800522pt}}
\pgflineto{\pgfpoint{222.869080pt}{140.777435pt}}
\pgfusepath{stroke}
\pgfpathmoveto{\pgfpoint{224.861053pt}{140.777695pt}}
\pgflineto{\pgfpoint{223.865082pt}{140.800522pt}}
\pgfusepath{stroke}
\pgfpathmoveto{\pgfpoint{225.857040pt}{140.916656pt}}
\pgflineto{\pgfpoint{224.861053pt}{140.777695pt}}
\pgfusepath{stroke}
\pgfpathmoveto{\pgfpoint{226.853027pt}{140.777435pt}}
\pgflineto{\pgfpoint{225.857040pt}{140.916656pt}}
\pgfusepath{stroke}
\pgfpathmoveto{\pgfpoint{227.849014pt}{140.782928pt}}
\pgflineto{\pgfpoint{226.853027pt}{140.777435pt}}
\pgfusepath{stroke}
\pgfpathmoveto{\pgfpoint{228.845001pt}{140.833771pt}}
\pgflineto{\pgfpoint{227.849014pt}{140.782928pt}}
\pgfusepath{stroke}
\pgfpathmoveto{\pgfpoint{229.840973pt}{140.894608pt}}
\pgflineto{\pgfpoint{228.845001pt}{140.833771pt}}
\pgfusepath{stroke}
\pgfpathmoveto{\pgfpoint{230.836945pt}{140.789825pt}}
\pgflineto{\pgfpoint{229.840973pt}{140.894608pt}}
\pgfusepath{stroke}
\pgfpathmoveto{\pgfpoint{231.832932pt}{142.041382pt}}
\pgflineto{\pgfpoint{230.836945pt}{140.789825pt}}
\pgfusepath{stroke}
\pgfpathmoveto{\pgfpoint{232.828934pt}{140.777435pt}}
\pgflineto{\pgfpoint{231.832932pt}{142.041382pt}}
\pgfusepath{stroke}
\pgfpathmoveto{\pgfpoint{233.824921pt}{140.823318pt}}
\pgflineto{\pgfpoint{232.828934pt}{140.777435pt}}
\pgfusepath{stroke}
\pgfpathmoveto{\pgfpoint{234.820892pt}{140.922363pt}}
\pgflineto{\pgfpoint{233.824921pt}{140.823318pt}}
\pgfusepath{stroke}
\pgfpathmoveto{\pgfpoint{235.816864pt}{140.779510pt}}
\pgflineto{\pgfpoint{234.820892pt}{140.922363pt}}
\pgfusepath{stroke}
\pgfpathmoveto{\pgfpoint{236.812866pt}{140.790619pt}}
\pgflineto{\pgfpoint{235.816864pt}{140.779510pt}}
\pgfusepath{stroke}
\pgfpathmoveto{\pgfpoint{237.808838pt}{140.799713pt}}
\pgflineto{\pgfpoint{236.812866pt}{140.790619pt}}
\pgfusepath{stroke}
\pgfpathmoveto{\pgfpoint{238.804825pt}{140.785339pt}}
\pgflineto{\pgfpoint{237.808838pt}{140.799713pt}}
\pgfusepath{stroke}
\pgfpathmoveto{\pgfpoint{239.800812pt}{140.777435pt}}
\pgflineto{\pgfpoint{238.804825pt}{140.785339pt}}
\pgfusepath{stroke}
\pgfpathmoveto{\pgfpoint{240.796814pt}{140.796509pt}}
\pgflineto{\pgfpoint{239.800812pt}{140.777435pt}}
\pgfusepath{stroke}
\pgfpathmoveto{\pgfpoint{241.792786pt}{140.777435pt}}
\pgflineto{\pgfpoint{240.796814pt}{140.796509pt}}
\pgfusepath{stroke}
\pgfpathmoveto{\pgfpoint{242.788757pt}{140.780701pt}}
\pgflineto{\pgfpoint{241.792786pt}{140.777435pt}}
\pgfusepath{stroke}
\pgfpathmoveto{\pgfpoint{243.784744pt}{140.777786pt}}
\pgflineto{\pgfpoint{242.788757pt}{140.780701pt}}
\pgfusepath{stroke}
\pgfpathmoveto{\pgfpoint{244.780731pt}{140.777435pt}}
\pgflineto{\pgfpoint{243.784744pt}{140.777786pt}}
\pgfusepath{stroke}
\pgfpathmoveto{\pgfpoint{245.776718pt}{140.778488pt}}
\pgflineto{\pgfpoint{244.780731pt}{140.777435pt}}
\pgfusepath{stroke}
\pgfpathmoveto{\pgfpoint{246.772705pt}{143.280365pt}}
\pgflineto{\pgfpoint{245.776718pt}{140.778488pt}}
\pgfusepath{stroke}
\pgfpathmoveto{\pgfpoint{247.768677pt}{140.830551pt}}
\pgflineto{\pgfpoint{246.772705pt}{143.280365pt}}
\pgfusepath{stroke}
\pgfpathmoveto{\pgfpoint{248.764679pt}{140.781235pt}}
\pgflineto{\pgfpoint{247.768677pt}{140.830551pt}}
\pgfusepath{stroke}
\pgfpathmoveto{\pgfpoint{249.760651pt}{140.778198pt}}
\pgflineto{\pgfpoint{248.764679pt}{140.781235pt}}
\pgfusepath{stroke}
\pgfpathmoveto{\pgfpoint{250.756638pt}{140.823486pt}}
\pgflineto{\pgfpoint{249.760651pt}{140.778198pt}}
\pgfusepath{stroke}
\pgfpathmoveto{\pgfpoint{251.752625pt}{140.777435pt}}
\pgflineto{\pgfpoint{250.756638pt}{140.823486pt}}
\pgfusepath{stroke}
\pgfpathmoveto{\pgfpoint{252.748611pt}{140.777435pt}}
\pgflineto{\pgfpoint{251.752625pt}{140.777435pt}}
\pgfusepath{stroke}
\pgfpathmoveto{\pgfpoint{253.744598pt}{142.920593pt}}
\pgflineto{\pgfpoint{252.748611pt}{140.777435pt}}
\pgfusepath{stroke}
\pgfpathmoveto{\pgfpoint{254.740570pt}{142.519669pt}}
\pgflineto{\pgfpoint{253.744598pt}{142.920593pt}}
\pgfusepath{stroke}
\pgfpathmoveto{\pgfpoint{255.736542pt}{140.777435pt}}
\pgflineto{\pgfpoint{254.740570pt}{142.519669pt}}
\pgfusepath{stroke}
\pgfpathmoveto{\pgfpoint{256.732544pt}{143.110718pt}}
\pgflineto{\pgfpoint{255.736542pt}{140.777435pt}}
\pgfusepath{stroke}
\pgfpathmoveto{\pgfpoint{257.728516pt}{140.907501pt}}
\pgflineto{\pgfpoint{256.732544pt}{143.110718pt}}
\pgfusepath{stroke}
\pgfpathmoveto{\pgfpoint{258.724518pt}{140.840500pt}}
\pgflineto{\pgfpoint{257.728516pt}{140.907501pt}}
\pgfusepath{stroke}
\pgfpathmoveto{\pgfpoint{259.720490pt}{140.779770pt}}
\pgflineto{\pgfpoint{258.724518pt}{140.840500pt}}
\pgfusepath{stroke}
\pgfpathmoveto{\pgfpoint{260.716492pt}{140.777496pt}}
\pgflineto{\pgfpoint{259.720490pt}{140.779770pt}}
\pgfusepath{stroke}
\pgfpathmoveto{\pgfpoint{261.712463pt}{140.779022pt}}
\pgflineto{\pgfpoint{260.716492pt}{140.777496pt}}
\pgfusepath{stroke}
\pgfpathmoveto{\pgfpoint{262.708435pt}{140.777435pt}}
\pgflineto{\pgfpoint{261.712463pt}{140.779022pt}}
\pgfusepath{stroke}
\pgfpathmoveto{\pgfpoint{263.704407pt}{140.927124pt}}
\pgflineto{\pgfpoint{262.708435pt}{140.777435pt}}
\pgfusepath{stroke}
\pgfpathmoveto{\pgfpoint{264.700409pt}{141.153885pt}}
\pgflineto{\pgfpoint{263.704407pt}{140.927124pt}}
\pgfusepath{stroke}
\pgfpathmoveto{\pgfpoint{265.696411pt}{140.951752pt}}
\pgflineto{\pgfpoint{264.700409pt}{141.153885pt}}
\pgfusepath{stroke}
\pgfpathmoveto{\pgfpoint{266.692383pt}{140.785080pt}}
\pgflineto{\pgfpoint{265.696411pt}{140.951752pt}}
\pgfusepath{stroke}
\pgfpathmoveto{\pgfpoint{267.688354pt}{140.779419pt}}
\pgflineto{\pgfpoint{266.692383pt}{140.785080pt}}
\pgfusepath{stroke}
\pgfpathmoveto{\pgfpoint{268.684326pt}{140.976303pt}}
\pgflineto{\pgfpoint{267.688354pt}{140.779419pt}}
\pgfusepath{stroke}
\pgfpathmoveto{\pgfpoint{269.680328pt}{140.824692pt}}
\pgflineto{\pgfpoint{268.684326pt}{140.976303pt}}
\pgfusepath{stroke}
\pgfpathmoveto{\pgfpoint{270.676331pt}{140.778473pt}}
\pgflineto{\pgfpoint{269.680328pt}{140.824692pt}}
\pgfusepath{stroke}
\pgfpathmoveto{\pgfpoint{271.672302pt}{141.361237pt}}
\pgflineto{\pgfpoint{270.676331pt}{140.778473pt}}
\pgfusepath{stroke}
\pgfpathmoveto{\pgfpoint{272.668274pt}{141.409683pt}}
\pgflineto{\pgfpoint{271.672302pt}{141.361237pt}}
\pgfusepath{stroke}
\pgfpathmoveto{\pgfpoint{273.664276pt}{140.777435pt}}
\pgflineto{\pgfpoint{272.668274pt}{141.409683pt}}
\pgfusepath{stroke}
\pgfpathmoveto{\pgfpoint{274.660248pt}{140.777435pt}}
\pgflineto{\pgfpoint{273.664276pt}{140.777435pt}}
\pgfusepath{stroke}
\pgfpathmoveto{\pgfpoint{275.656250pt}{140.777435pt}}
\pgflineto{\pgfpoint{274.660248pt}{140.777435pt}}
\pgfusepath{stroke}
\pgfpathmoveto{\pgfpoint{276.652222pt}{140.792587pt}}
\pgflineto{\pgfpoint{275.656250pt}{140.777435pt}}
\pgfusepath{stroke}
\pgfpathmoveto{\pgfpoint{277.648193pt}{140.928909pt}}
\pgflineto{\pgfpoint{276.652222pt}{140.792587pt}}
\pgfusepath{stroke}
\pgfpathmoveto{\pgfpoint{278.644196pt}{140.783813pt}}
\pgflineto{\pgfpoint{277.648193pt}{140.928909pt}}
\pgfusepath{stroke}
\pgfpathmoveto{\pgfpoint{279.640167pt}{140.790070pt}}
\pgflineto{\pgfpoint{278.644196pt}{140.783813pt}}
\pgfusepath{stroke}
\pgfpathmoveto{\pgfpoint{280.636139pt}{140.777878pt}}
\pgflineto{\pgfpoint{279.640167pt}{140.790070pt}}
\pgfusepath{stroke}
\pgfpathmoveto{\pgfpoint{281.632141pt}{140.777481pt}}
\pgflineto{\pgfpoint{280.636139pt}{140.777878pt}}
\pgfusepath{stroke}
\pgfpathmoveto{\pgfpoint{282.628113pt}{140.782150pt}}
\pgflineto{\pgfpoint{281.632141pt}{140.777481pt}}
\pgfusepath{stroke}
\pgfpathmoveto{\pgfpoint{283.624115pt}{140.777435pt}}
\pgflineto{\pgfpoint{282.628113pt}{140.782150pt}}
\pgfusepath{stroke}
\pgfpathmoveto{\pgfpoint{284.620087pt}{140.777740pt}}
\pgflineto{\pgfpoint{283.624115pt}{140.777435pt}}
\pgfusepath{stroke}
\pgfpathmoveto{\pgfpoint{285.616089pt}{140.784164pt}}
\pgflineto{\pgfpoint{284.620087pt}{140.777740pt}}
\pgfusepath{stroke}
\pgfpathmoveto{\pgfpoint{286.612061pt}{140.777435pt}}
\pgflineto{\pgfpoint{285.616089pt}{140.784164pt}}
\pgfusepath{stroke}
\pgfpathmoveto{\pgfpoint{287.608032pt}{140.777435pt}}
\pgflineto{\pgfpoint{286.612061pt}{140.777435pt}}
\pgfusepath{stroke}
\pgfpathmoveto{\pgfpoint{288.604004pt}{140.798889pt}}
\pgflineto{\pgfpoint{287.608032pt}{140.777435pt}}
\pgfusepath{stroke}
\pgfpathmoveto{\pgfpoint{289.600037pt}{140.790405pt}}
\pgflineto{\pgfpoint{288.604004pt}{140.798889pt}}
\pgfusepath{stroke}
\color[rgb]{0.000000,0.000000,0.000000}
\pgfsetlinewidth{0.500000pt}
\pgfsetdash{{16pt}{0pt}}{0pt}
\pgfpathmoveto{\pgfpoint{289.600037pt}{26.399979pt}}
\pgflineto{\pgfpoint{41.600006pt}{26.399979pt}}
\pgfusepath{stroke}
\pgfpathmoveto{\pgfpoint{289.600037pt}{91.199989pt}}
\pgflineto{\pgfpoint{41.600006pt}{91.199989pt}}
\pgfusepath{stroke}
\pgfpathmoveto{\pgfpoint{41.600006pt}{91.199989pt}}
\pgflineto{\pgfpoint{41.600006pt}{26.399979pt}}
\pgfusepath{stroke}
\pgfpathmoveto{\pgfpoint{289.600037pt}{91.199989pt}}
\pgflineto{\pgfpoint{289.600037pt}{26.399979pt}}
\pgfusepath{stroke}
\pgfpathmoveto{\pgfpoint{90.403221pt}{28.872353pt}}
\pgflineto{\pgfpoint{90.403221pt}{26.399979pt}}
\pgfusepath{stroke}
\pgfpathmoveto{\pgfpoint{90.403221pt}{88.727623pt}}
\pgflineto{\pgfpoint{90.403221pt}{91.199989pt}}
\pgfusepath{stroke}
\pgfpathmoveto{\pgfpoint{140.202423pt}{28.872353pt}}
\pgflineto{\pgfpoint{140.202423pt}{26.399979pt}}
\pgfusepath{stroke}
\pgfpathmoveto{\pgfpoint{140.202423pt}{88.727623pt}}
\pgflineto{\pgfpoint{140.202423pt}{91.199989pt}}
\pgfusepath{stroke}
\pgfpathmoveto{\pgfpoint{190.001617pt}{28.872353pt}}
\pgflineto{\pgfpoint{190.001617pt}{26.399979pt}}
\pgfusepath{stroke}
\pgfpathmoveto{\pgfpoint{190.001617pt}{88.727623pt}}
\pgflineto{\pgfpoint{190.001617pt}{91.199989pt}}
\pgfusepath{stroke}
\pgfpathmoveto{\pgfpoint{239.800812pt}{28.872353pt}}
\pgflineto{\pgfpoint{239.800812pt}{26.399979pt}}
\pgfusepath{stroke}
\pgfpathmoveto{\pgfpoint{239.800812pt}{88.727623pt}}
\pgflineto{\pgfpoint{239.800812pt}{91.199989pt}}
\pgfusepath{stroke}
\pgfpathmoveto{\pgfpoint{289.600037pt}{28.872353pt}}
\pgflineto{\pgfpoint{289.600037pt}{26.399979pt}}
\pgfusepath{stroke}
\pgfpathmoveto{\pgfpoint{289.600037pt}{88.727623pt}}
\pgflineto{\pgfpoint{289.600037pt}{91.199989pt}}
\pgfusepath{stroke}
{
\pgftransformshift{\pgfpoint{90.403229pt}{21.415367pt}}
\pgfnode{rectangle}{north}{\fontsize{10}{0}\selectfont\textcolor[rgb]{0,0,0}{{50}}}{}{\pgfusepath{discard}}}
{
\pgftransformshift{\pgfpoint{140.202423pt}{21.415367pt}}
\pgfnode{rectangle}{north}{\fontsize{10}{0}\selectfont\textcolor[rgb]{0,0,0}{{100}}}{}{\pgfusepath{discard}}}
{
\pgftransformshift{\pgfpoint{190.001617pt}{21.415367pt}}
\pgfnode{rectangle}{north}{\fontsize{10}{0}\selectfont\textcolor[rgb]{0,0,0}{{150}}}{}{\pgfusepath{discard}}}
{
\pgftransformshift{\pgfpoint{239.800812pt}{21.415367pt}}
\pgfnode{rectangle}{north}{\fontsize{10}{0}\selectfont\textcolor[rgb]{0,0,0}{{200}}}{}{\pgfusepath{discard}}}
{
\pgftransformshift{\pgfpoint{289.600037pt}{21.415367pt}}
\pgfnode{rectangle}{north}{\fontsize{10}{0}\selectfont\textcolor[rgb]{0,0,0}{{250}}}{}{\pgfusepath{discard}}}
\pgfpathmoveto{\pgfpoint{44.080009pt}{26.399979pt}}
\pgflineto{\pgfpoint{41.600006pt}{26.399979pt}}
\pgfusepath{stroke}
\pgfpathmoveto{\pgfpoint{287.119995pt}{26.399979pt}}
\pgflineto{\pgfpoint{289.600037pt}{26.399979pt}}
\pgfusepath{stroke}
\pgfpathmoveto{\pgfpoint{44.080009pt}{39.359985pt}}
\pgflineto{\pgfpoint{41.600006pt}{39.359985pt}}
\pgfusepath{stroke}
\pgfpathmoveto{\pgfpoint{287.119995pt}{39.359985pt}}
\pgflineto{\pgfpoint{289.600037pt}{39.359985pt}}
\pgfusepath{stroke}
\pgfpathmoveto{\pgfpoint{44.080009pt}{52.319984pt}}
\pgflineto{\pgfpoint{41.600006pt}{52.319984pt}}
\pgfusepath{stroke}
\pgfpathmoveto{\pgfpoint{287.119995pt}{52.319984pt}}
\pgflineto{\pgfpoint{289.600037pt}{52.319984pt}}
\pgfusepath{stroke}
\pgfpathmoveto{\pgfpoint{44.080009pt}{65.279991pt}}
\pgflineto{\pgfpoint{41.600006pt}{65.279991pt}}
\pgfusepath{stroke}
\pgfpathmoveto{\pgfpoint{287.119995pt}{65.279991pt}}
\pgflineto{\pgfpoint{289.600037pt}{65.279991pt}}
\pgfusepath{stroke}
\pgfpathmoveto{\pgfpoint{44.080009pt}{78.239990pt}}
\pgflineto{\pgfpoint{41.600006pt}{78.239990pt}}
\pgfusepath{stroke}
\pgfpathmoveto{\pgfpoint{287.119995pt}{78.239990pt}}
\pgflineto{\pgfpoint{289.600037pt}{78.239990pt}}
\pgfusepath{stroke}
\pgfpathmoveto{\pgfpoint{44.080009pt}{91.199989pt}}
\pgflineto{\pgfpoint{41.600006pt}{91.199989pt}}
\pgfusepath{stroke}
\pgfpathmoveto{\pgfpoint{287.119995pt}{91.199989pt}}
\pgflineto{\pgfpoint{289.600037pt}{91.199989pt}}
\pgfusepath{stroke}
{
\pgftransformshift{\pgfpoint{36.600006pt}{26.399979pt}}
\pgfnode{rectangle}{east}{\fontsize{10}{0}\selectfont\textcolor[rgb]{0,0,0}{{0}}}{}{\pgfusepath{discard}}}
{
\pgftransformshift{\pgfpoint{36.600006pt}{39.359978pt}}
\pgfnode{rectangle}{east}{\fontsize{10}{0}\selectfont\textcolor[rgb]{0,0,0}{{2e+06}}}{}{\pgfusepath{discard}}}
{
\pgftransformshift{\pgfpoint{36.600006pt}{52.319984pt}}
\pgfnode{rectangle}{east}{\fontsize{10}{0}\selectfont\textcolor[rgb]{0,0,0}{{4e+06}}}{}{\pgfusepath{discard}}}
{
\pgftransformshift{\pgfpoint{36.600006pt}{65.279991pt}}
\pgfnode{rectangle}{east}{\fontsize{10}{0}\selectfont\textcolor[rgb]{0,0,0}{{6e+06}}}{}{\pgfusepath{discard}}}
{
\pgftransformshift{\pgfpoint{36.600006pt}{78.239990pt}}
\pgfnode{rectangle}{east}{\fontsize{10}{0}\selectfont\textcolor[rgb]{0,0,0}{{8e+06}}}{}{\pgfusepath{discard}}}
{
\pgftransformshift{\pgfpoint{36.600006pt}{91.199989pt}}
\pgfnode{rectangle}{east}{\fontsize{10}{0}\selectfont\textcolor[rgb]{0,0,0}{{1e+07}}}{}{\pgfusepath{discard}}}
\pgfsetlinewidth{0.000100pt}
\pgfsetdash{}{0pt}
\pgfpathmoveto{\pgfpoint{43.591980pt}{26.399979pt}}
\pgflineto{\pgfpoint{44.587967pt}{26.399979pt}}
\pgfusepath{stroke}
\pgfpathmoveto{\pgfpoint{42.595993pt}{26.399979pt}}
\pgflineto{\pgfpoint{43.591980pt}{26.399979pt}}
\pgfusepath{stroke}
\pgfpathmoveto{\pgfpoint{43.591980pt}{33.198105pt}}
\pgflineto{\pgfpoint{42.595993pt}{26.399979pt}}
\pgfusepath{stroke}
\pgfpathmoveto{\pgfpoint{44.587967pt}{26.399979pt}}
\pgflineto{\pgfpoint{43.591980pt}{33.198105pt}}
\pgfusepath{stroke}
\pgfpathmoveto{\pgfpoint{47.575912pt}{26.399979pt}}
\pgflineto{\pgfpoint{48.571899pt}{26.399979pt}}
\pgfusepath{stroke}
\pgfpathmoveto{\pgfpoint{46.579933pt}{26.399979pt}}
\pgflineto{\pgfpoint{47.575912pt}{26.399979pt}}
\pgfusepath{stroke}
\pgfpathmoveto{\pgfpoint{47.575912pt}{27.559311pt}}
\pgflineto{\pgfpoint{46.579933pt}{26.399979pt}}
\pgfusepath{stroke}
\pgfpathmoveto{\pgfpoint{48.571899pt}{26.399979pt}}
\pgflineto{\pgfpoint{47.575912pt}{27.559311pt}}
\pgfusepath{stroke}
\pgfpathmoveto{\pgfpoint{53.551819pt}{26.399979pt}}
\pgflineto{\pgfpoint{54.547806pt}{26.399979pt}}
\pgfusepath{stroke}
\pgfpathmoveto{\pgfpoint{52.555840pt}{26.399979pt}}
\pgflineto{\pgfpoint{53.551819pt}{26.399979pt}}
\pgfusepath{stroke}
\pgfpathmoveto{\pgfpoint{53.551819pt}{32.692734pt}}
\pgflineto{\pgfpoint{52.555840pt}{26.399979pt}}
\pgfusepath{stroke}
\pgfpathmoveto{\pgfpoint{54.547806pt}{26.399979pt}}
\pgflineto{\pgfpoint{53.551819pt}{32.692734pt}}
\pgfusepath{stroke}
\pgfpathmoveto{\pgfpoint{56.539772pt}{26.399979pt}}
\pgflineto{\pgfpoint{57.535751pt}{26.399979pt}}
\pgfusepath{stroke}
\pgfpathmoveto{\pgfpoint{55.543785pt}{26.399979pt}}
\pgflineto{\pgfpoint{56.539772pt}{26.399979pt}}
\pgfusepath{stroke}
\pgfpathmoveto{\pgfpoint{54.547806pt}{26.399979pt}}
\pgflineto{\pgfpoint{55.543785pt}{26.399979pt}}
\pgfusepath{stroke}
\pgfpathmoveto{\pgfpoint{55.543785pt}{26.577354pt}}
\pgflineto{\pgfpoint{54.547806pt}{26.399979pt}}
\pgfusepath{stroke}
\pgfpathmoveto{\pgfpoint{56.539772pt}{65.803238pt}}
\pgflineto{\pgfpoint{55.543785pt}{26.577354pt}}
\pgfusepath{stroke}
\pgfpathmoveto{\pgfpoint{57.535751pt}{26.399979pt}}
\pgflineto{\pgfpoint{56.539772pt}{65.803238pt}}
\pgfusepath{stroke}
\pgfpathmoveto{\pgfpoint{59.527725pt}{26.399979pt}}
\pgflineto{\pgfpoint{60.523712pt}{26.399979pt}}
\pgfusepath{stroke}
\pgfpathmoveto{\pgfpoint{58.531738pt}{26.399979pt}}
\pgflineto{\pgfpoint{59.527725pt}{26.399979pt}}
\pgfusepath{stroke}
\pgfpathmoveto{\pgfpoint{57.535751pt}{26.399979pt}}
\pgflineto{\pgfpoint{58.531738pt}{26.399979pt}}
\pgfusepath{stroke}
\pgfpathmoveto{\pgfpoint{58.531738pt}{27.217796pt}}
\pgflineto{\pgfpoint{57.535751pt}{26.399979pt}}
\pgfusepath{stroke}
\pgfpathmoveto{\pgfpoint{59.527725pt}{26.723701pt}}
\pgflineto{\pgfpoint{58.531738pt}{27.217796pt}}
\pgfusepath{stroke}
\pgfpathmoveto{\pgfpoint{60.523712pt}{26.399979pt}}
\pgflineto{\pgfpoint{59.527725pt}{26.723701pt}}
\pgfusepath{stroke}
\pgfpathmoveto{\pgfpoint{62.515678pt}{26.399979pt}}
\pgflineto{\pgfpoint{63.511658pt}{26.399979pt}}
\pgfusepath{stroke}
\pgfpathmoveto{\pgfpoint{61.519691pt}{26.399979pt}}
\pgflineto{\pgfpoint{62.515678pt}{26.399979pt}}
\pgfusepath{stroke}
\pgfpathmoveto{\pgfpoint{62.515678pt}{26.488380pt}}
\pgflineto{\pgfpoint{61.519691pt}{26.399979pt}}
\pgfusepath{stroke}
\pgfpathmoveto{\pgfpoint{63.511658pt}{26.399979pt}}
\pgflineto{\pgfpoint{62.515678pt}{26.488380pt}}
\pgfusepath{stroke}
\pgfpathmoveto{\pgfpoint{66.499619pt}{26.399979pt}}
\pgflineto{\pgfpoint{67.495590pt}{26.399979pt}}
\pgfusepath{stroke}
\pgfpathmoveto{\pgfpoint{65.503624pt}{26.399979pt}}
\pgflineto{\pgfpoint{66.499619pt}{26.399979pt}}
\pgfusepath{stroke}
\pgfpathmoveto{\pgfpoint{64.507637pt}{26.399979pt}}
\pgflineto{\pgfpoint{65.503624pt}{26.399979pt}}
\pgfusepath{stroke}
\pgfpathmoveto{\pgfpoint{63.511658pt}{26.399979pt}}
\pgflineto{\pgfpoint{64.507637pt}{26.399979pt}}
\pgfusepath{stroke}
\pgfpathmoveto{\pgfpoint{64.507637pt}{26.602051pt}}
\pgflineto{\pgfpoint{63.511658pt}{26.399979pt}}
\pgfusepath{stroke}
\pgfpathmoveto{\pgfpoint{65.503624pt}{27.212898pt}}
\pgflineto{\pgfpoint{64.507637pt}{26.602051pt}}
\pgfusepath{stroke}
\pgfpathmoveto{\pgfpoint{66.499619pt}{29.321190pt}}
\pgflineto{\pgfpoint{65.503624pt}{27.212898pt}}
\pgfusepath{stroke}
\pgfpathmoveto{\pgfpoint{67.495590pt}{26.399979pt}}
\pgflineto{\pgfpoint{66.499619pt}{29.321190pt}}
\pgfusepath{stroke}
\pgfpathmoveto{\pgfpoint{69.487564pt}{26.399979pt}}
\pgflineto{\pgfpoint{70.483551pt}{26.399979pt}}
\pgfusepath{stroke}
\pgfpathmoveto{\pgfpoint{68.491577pt}{26.399979pt}}
\pgflineto{\pgfpoint{69.487564pt}{26.399979pt}}
\pgfusepath{stroke}
\pgfpathmoveto{\pgfpoint{67.495590pt}{26.399979pt}}
\pgflineto{\pgfpoint{68.491577pt}{26.399979pt}}
\pgfusepath{stroke}
\pgfpathmoveto{\pgfpoint{68.491577pt}{26.687355pt}}
\pgflineto{\pgfpoint{67.495590pt}{26.399979pt}}
\pgfusepath{stroke}
\pgfpathmoveto{\pgfpoint{69.487564pt}{27.948082pt}}
\pgflineto{\pgfpoint{68.491577pt}{26.687355pt}}
\pgfusepath{stroke}
\pgfpathmoveto{\pgfpoint{70.483551pt}{26.399979pt}}
\pgflineto{\pgfpoint{69.487564pt}{27.948082pt}}
\pgfusepath{stroke}
\pgfpathmoveto{\pgfpoint{72.475510pt}{26.399979pt}}
\pgflineto{\pgfpoint{73.471497pt}{26.399979pt}}
\pgfusepath{stroke}
\pgfpathmoveto{\pgfpoint{71.479530pt}{26.399979pt}}
\pgflineto{\pgfpoint{72.475510pt}{26.399979pt}}
\pgfusepath{stroke}
\pgfpathmoveto{\pgfpoint{70.483551pt}{26.399979pt}}
\pgflineto{\pgfpoint{71.479530pt}{26.399979pt}}
\pgfusepath{stroke}
\pgfpathmoveto{\pgfpoint{71.479530pt}{52.145065pt}}
\pgflineto{\pgfpoint{70.483551pt}{26.399979pt}}
\pgfusepath{stroke}
\pgfpathmoveto{\pgfpoint{72.475510pt}{31.143661pt}}
\pgflineto{\pgfpoint{71.479530pt}{52.145065pt}}
\pgfusepath{stroke}
\pgfpathmoveto{\pgfpoint{73.471497pt}{26.399979pt}}
\pgflineto{\pgfpoint{72.475510pt}{31.143661pt}}
\pgfusepath{stroke}
\pgfpathmoveto{\pgfpoint{75.463470pt}{26.399979pt}}
\pgflineto{\pgfpoint{76.459442pt}{26.399979pt}}
\pgfusepath{stroke}
\pgfpathmoveto{\pgfpoint{74.467484pt}{26.399979pt}}
\pgflineto{\pgfpoint{75.463470pt}{26.399979pt}}
\pgfusepath{stroke}
\pgfpathmoveto{\pgfpoint{73.471497pt}{26.399979pt}}
\pgflineto{\pgfpoint{74.467484pt}{26.399979pt}}
\pgfusepath{stroke}
\pgfpathmoveto{\pgfpoint{74.467484pt}{26.868034pt}}
\pgflineto{\pgfpoint{73.471497pt}{26.399979pt}}
\pgfusepath{stroke}
\pgfpathmoveto{\pgfpoint{75.463470pt}{26.470932pt}}
\pgflineto{\pgfpoint{74.467484pt}{26.868034pt}}
\pgfusepath{stroke}
\pgfpathmoveto{\pgfpoint{76.459442pt}{26.399979pt}}
\pgflineto{\pgfpoint{75.463470pt}{26.470932pt}}
\pgfusepath{stroke}
\pgfpathmoveto{\pgfpoint{82.435356pt}{26.399979pt}}
\pgflineto{\pgfpoint{83.431335pt}{26.399979pt}}
\pgfusepath{stroke}
\pgfpathmoveto{\pgfpoint{81.439369pt}{26.399979pt}}
\pgflineto{\pgfpoint{82.435356pt}{26.399979pt}}
\pgfusepath{stroke}
\pgfpathmoveto{\pgfpoint{80.443390pt}{26.399979pt}}
\pgflineto{\pgfpoint{81.439369pt}{26.399979pt}}
\pgfusepath{stroke}
\pgfpathmoveto{\pgfpoint{79.447403pt}{26.399979pt}}
\pgflineto{\pgfpoint{80.443390pt}{26.399979pt}}
\pgfusepath{stroke}
\pgfpathmoveto{\pgfpoint{80.443390pt}{28.913742pt}}
\pgflineto{\pgfpoint{79.447403pt}{26.399979pt}}
\pgfusepath{stroke}
\pgfpathmoveto{\pgfpoint{81.439369pt}{28.404305pt}}
\pgflineto{\pgfpoint{80.443390pt}{28.913742pt}}
\pgfusepath{stroke}
\pgfpathmoveto{\pgfpoint{82.435356pt}{28.310295pt}}
\pgflineto{\pgfpoint{81.439369pt}{28.404305pt}}
\pgfusepath{stroke}
\pgfpathmoveto{\pgfpoint{83.431335pt}{26.399979pt}}
\pgflineto{\pgfpoint{82.435356pt}{28.310295pt}}
\pgfusepath{stroke}
\pgfpathmoveto{\pgfpoint{86.419289pt}{26.399979pt}}
\pgflineto{\pgfpoint{87.415276pt}{26.399979pt}}
\pgfusepath{stroke}
\pgfpathmoveto{\pgfpoint{85.423309pt}{26.399979pt}}
\pgflineto{\pgfpoint{86.419289pt}{26.399979pt}}
\pgfusepath{stroke}
\pgfpathmoveto{\pgfpoint{86.419289pt}{26.871178pt}}
\pgflineto{\pgfpoint{85.423309pt}{26.399979pt}}
\pgfusepath{stroke}
\pgfpathmoveto{\pgfpoint{87.415276pt}{26.399979pt}}
\pgflineto{\pgfpoint{86.419289pt}{26.871178pt}}
\pgfusepath{stroke}
\pgfpathmoveto{\pgfpoint{88.411255pt}{26.399979pt}}
\pgflineto{\pgfpoint{89.407242pt}{26.399979pt}}
\pgfusepath{stroke}
\pgfpathmoveto{\pgfpoint{87.415276pt}{26.399979pt}}
\pgflineto{\pgfpoint{88.411255pt}{26.399979pt}}
\pgfusepath{stroke}
\pgfpathmoveto{\pgfpoint{88.411255pt}{31.944191pt}}
\pgflineto{\pgfpoint{87.415276pt}{26.399979pt}}
\pgfusepath{stroke}
\pgfpathmoveto{\pgfpoint{89.407242pt}{26.399979pt}}
\pgflineto{\pgfpoint{88.411255pt}{31.944191pt}}
\pgfusepath{stroke}
\pgfpathmoveto{\pgfpoint{90.403221pt}{26.399979pt}}
\pgflineto{\pgfpoint{91.399208pt}{26.399979pt}}
\pgfusepath{stroke}
\pgfpathmoveto{\pgfpoint{89.407242pt}{26.399979pt}}
\pgflineto{\pgfpoint{90.403221pt}{26.399979pt}}
\pgfusepath{stroke}
\pgfpathmoveto{\pgfpoint{90.403221pt}{26.454811pt}}
\pgflineto{\pgfpoint{89.407242pt}{26.399979pt}}
\pgfusepath{stroke}
\pgfpathmoveto{\pgfpoint{91.399208pt}{26.399979pt}}
\pgflineto{\pgfpoint{90.403221pt}{26.454811pt}}
\pgfusepath{stroke}
\pgfpathmoveto{\pgfpoint{93.391174pt}{26.399979pt}}
\pgflineto{\pgfpoint{94.387161pt}{26.399979pt}}
\pgfusepath{stroke}
\pgfpathmoveto{\pgfpoint{92.395187pt}{26.399979pt}}
\pgflineto{\pgfpoint{93.391174pt}{26.399979pt}}
\pgfusepath{stroke}
\pgfpathmoveto{\pgfpoint{91.399208pt}{26.399979pt}}
\pgflineto{\pgfpoint{92.395187pt}{26.399979pt}}
\pgfusepath{stroke}
\pgfpathmoveto{\pgfpoint{92.395187pt}{26.889816pt}}
\pgflineto{\pgfpoint{91.399208pt}{26.399979pt}}
\pgfusepath{stroke}
\pgfpathmoveto{\pgfpoint{93.391174pt}{27.006203pt}}
\pgflineto{\pgfpoint{92.395187pt}{26.889816pt}}
\pgfusepath{stroke}
\pgfpathmoveto{\pgfpoint{94.387161pt}{26.399979pt}}
\pgflineto{\pgfpoint{93.391174pt}{27.006203pt}}
\pgfusepath{stroke}
\pgfpathmoveto{\pgfpoint{100.363068pt}{26.399979pt}}
\pgflineto{\pgfpoint{101.359047pt}{26.399979pt}}
\pgfusepath{stroke}
\pgfpathmoveto{\pgfpoint{99.367081pt}{26.399979pt}}
\pgflineto{\pgfpoint{100.363068pt}{26.399979pt}}
\pgfusepath{stroke}
\pgfpathmoveto{\pgfpoint{98.371094pt}{26.399979pt}}
\pgflineto{\pgfpoint{99.367081pt}{26.399979pt}}
\pgfusepath{stroke}
\pgfpathmoveto{\pgfpoint{97.375107pt}{26.399979pt}}
\pgflineto{\pgfpoint{98.371094pt}{26.399979pt}}
\pgfusepath{stroke}
\pgfpathmoveto{\pgfpoint{98.371094pt}{47.782463pt}}
\pgflineto{\pgfpoint{97.375107pt}{26.399979pt}}
\pgfusepath{stroke}
\pgfpathmoveto{\pgfpoint{99.367081pt}{54.046951pt}}
\pgflineto{\pgfpoint{98.371094pt}{47.782463pt}}
\pgfusepath{stroke}
\pgfpathmoveto{\pgfpoint{100.363068pt}{32.883598pt}}
\pgflineto{\pgfpoint{99.367081pt}{54.046951pt}}
\pgfusepath{stroke}
\pgfpathmoveto{\pgfpoint{101.359047pt}{26.399979pt}}
\pgflineto{\pgfpoint{100.363068pt}{32.883598pt}}
\pgfusepath{stroke}
\pgfpathmoveto{\pgfpoint{102.355034pt}{26.399979pt}}
\pgflineto{\pgfpoint{103.351013pt}{26.399979pt}}
\pgfusepath{stroke}
\pgfpathmoveto{\pgfpoint{101.359047pt}{26.399979pt}}
\pgflineto{\pgfpoint{102.355034pt}{26.399979pt}}
\pgfusepath{stroke}
\pgfpathmoveto{\pgfpoint{102.355034pt}{26.959930pt}}
\pgflineto{\pgfpoint{101.359047pt}{26.399979pt}}
\pgfusepath{stroke}
\pgfpathmoveto{\pgfpoint{103.351013pt}{26.399979pt}}
\pgflineto{\pgfpoint{102.355034pt}{26.959930pt}}
\pgfusepath{stroke}
\pgfpathmoveto{\pgfpoint{104.347000pt}{26.399979pt}}
\pgflineto{\pgfpoint{105.342987pt}{26.399979pt}}
\pgfusepath{stroke}
\pgfpathmoveto{\pgfpoint{103.351013pt}{26.399979pt}}
\pgflineto{\pgfpoint{104.347000pt}{26.399979pt}}
\pgfusepath{stroke}
\pgfpathmoveto{\pgfpoint{104.347000pt}{35.689026pt}}
\pgflineto{\pgfpoint{103.351013pt}{26.399979pt}}
\pgfusepath{stroke}
\pgfpathmoveto{\pgfpoint{105.342987pt}{26.399979pt}}
\pgflineto{\pgfpoint{104.347000pt}{35.689026pt}}
\pgfusepath{stroke}
\pgfpathmoveto{\pgfpoint{109.326920pt}{26.399979pt}}
\pgflineto{\pgfpoint{110.322906pt}{26.399979pt}}
\pgfusepath{stroke}
\pgfpathmoveto{\pgfpoint{108.330933pt}{26.399979pt}}
\pgflineto{\pgfpoint{109.326920pt}{26.399979pt}}
\pgfusepath{stroke}
\pgfpathmoveto{\pgfpoint{107.334953pt}{26.399979pt}}
\pgflineto{\pgfpoint{108.330933pt}{26.399979pt}}
\pgfusepath{stroke}
\pgfpathmoveto{\pgfpoint{108.330933pt}{27.150108pt}}
\pgflineto{\pgfpoint{107.334953pt}{26.399979pt}}
\pgfusepath{stroke}
\pgfpathmoveto{\pgfpoint{109.326920pt}{28.632256pt}}
\pgflineto{\pgfpoint{108.330933pt}{27.150108pt}}
\pgfusepath{stroke}
\pgfpathmoveto{\pgfpoint{110.322906pt}{26.399979pt}}
\pgflineto{\pgfpoint{109.326920pt}{28.632256pt}}
\pgfusepath{stroke}
\pgfpathmoveto{\pgfpoint{113.310852pt}{26.399979pt}}
\pgflineto{\pgfpoint{114.306839pt}{26.399979pt}}
\pgfusepath{stroke}
\pgfpathmoveto{\pgfpoint{112.314873pt}{26.399979pt}}
\pgflineto{\pgfpoint{113.310852pt}{26.399979pt}}
\pgfusepath{stroke}
\pgfpathmoveto{\pgfpoint{111.318893pt}{26.399979pt}}
\pgflineto{\pgfpoint{112.314873pt}{26.399979pt}}
\pgfusepath{stroke}
\pgfpathmoveto{\pgfpoint{112.314873pt}{37.362213pt}}
\pgflineto{\pgfpoint{111.318893pt}{26.399979pt}}
\pgfusepath{stroke}
\pgfpathmoveto{\pgfpoint{113.310852pt}{26.517036pt}}
\pgflineto{\pgfpoint{112.314873pt}{37.362213pt}}
\pgfusepath{stroke}
\pgfpathmoveto{\pgfpoint{114.306839pt}{26.399979pt}}
\pgflineto{\pgfpoint{113.310852pt}{26.517036pt}}
\pgfusepath{stroke}
\pgfpathmoveto{\pgfpoint{118.290779pt}{26.399979pt}}
\pgflineto{\pgfpoint{119.286758pt}{26.399979pt}}
\pgfusepath{stroke}
\pgfpathmoveto{\pgfpoint{117.294792pt}{26.399979pt}}
\pgflineto{\pgfpoint{118.290779pt}{26.399979pt}}
\pgfusepath{stroke}
\pgfpathmoveto{\pgfpoint{116.298813pt}{26.399979pt}}
\pgflineto{\pgfpoint{117.294792pt}{26.399979pt}}
\pgfusepath{stroke}
\pgfpathmoveto{\pgfpoint{115.302826pt}{26.399979pt}}
\pgflineto{\pgfpoint{116.298813pt}{26.399979pt}}
\pgfusepath{stroke}
\pgfpathmoveto{\pgfpoint{116.298813pt}{29.134193pt}}
\pgflineto{\pgfpoint{115.302826pt}{26.399979pt}}
\pgfusepath{stroke}
\pgfpathmoveto{\pgfpoint{117.294792pt}{26.815964pt}}
\pgflineto{\pgfpoint{116.298813pt}{29.134193pt}}
\pgfusepath{stroke}
\pgfpathmoveto{\pgfpoint{117.966232pt}{91.264786pt}}
\pgflineto{\pgfpoint{117.294792pt}{26.815964pt}}
\pgfusepath{stroke}
\pgfpathmoveto{\pgfpoint{119.286758pt}{26.399979pt}}
\pgflineto{\pgfpoint{118.613922pt}{91.264793pt}}
\pgfusepath{stroke}
\pgfpathmoveto{\pgfpoint{121.278725pt}{26.399979pt}}
\pgflineto{\pgfpoint{122.274712pt}{26.399979pt}}
\pgfusepath{stroke}
\pgfpathmoveto{\pgfpoint{120.282745pt}{26.399979pt}}
\pgflineto{\pgfpoint{121.278725pt}{26.399979pt}}
\pgfusepath{stroke}
\pgfpathmoveto{\pgfpoint{121.278725pt}{27.190407pt}}
\pgflineto{\pgfpoint{120.282745pt}{26.399979pt}}
\pgfusepath{stroke}
\pgfpathmoveto{\pgfpoint{122.274712pt}{26.399979pt}}
\pgflineto{\pgfpoint{121.278725pt}{27.190407pt}}
\pgfusepath{stroke}
\pgfpathmoveto{\pgfpoint{123.270691pt}{26.399979pt}}
\pgflineto{\pgfpoint{124.266678pt}{26.399979pt}}
\pgfusepath{stroke}
\pgfpathmoveto{\pgfpoint{122.274712pt}{26.399979pt}}
\pgflineto{\pgfpoint{123.270691pt}{26.399979pt}}
\pgfusepath{stroke}
\pgfpathmoveto{\pgfpoint{123.270691pt}{26.975601pt}}
\pgflineto{\pgfpoint{122.274712pt}{26.399979pt}}
\pgfusepath{stroke}
\pgfpathmoveto{\pgfpoint{124.266678pt}{26.399979pt}}
\pgflineto{\pgfpoint{123.270691pt}{26.975601pt}}
\pgfusepath{stroke}
\pgfpathmoveto{\pgfpoint{132.234558pt}{26.399979pt}}
\pgflineto{\pgfpoint{133.230530pt}{26.399979pt}}
\pgfusepath{stroke}
\pgfpathmoveto{\pgfpoint{131.238571pt}{26.399979pt}}
\pgflineto{\pgfpoint{132.234558pt}{26.399979pt}}
\pgfusepath{stroke}
\pgfpathmoveto{\pgfpoint{130.242584pt}{26.399979pt}}
\pgflineto{\pgfpoint{131.238571pt}{26.399979pt}}
\pgfusepath{stroke}
\pgfpathmoveto{\pgfpoint{129.246597pt}{26.399979pt}}
\pgflineto{\pgfpoint{130.242584pt}{26.399979pt}}
\pgfusepath{stroke}
\pgfpathmoveto{\pgfpoint{128.250610pt}{26.399979pt}}
\pgflineto{\pgfpoint{129.246597pt}{26.399979pt}}
\pgfusepath{stroke}
\pgfpathmoveto{\pgfpoint{127.254631pt}{26.399979pt}}
\pgflineto{\pgfpoint{128.250610pt}{26.399979pt}}
\pgfusepath{stroke}
\pgfpathmoveto{\pgfpoint{126.258652pt}{26.399979pt}}
\pgflineto{\pgfpoint{127.254631pt}{26.399979pt}}
\pgfusepath{stroke}
\pgfpathmoveto{\pgfpoint{125.262665pt}{26.399979pt}}
\pgflineto{\pgfpoint{126.258652pt}{26.399979pt}}
\pgfusepath{stroke}
\pgfpathmoveto{\pgfpoint{126.258652pt}{28.660576pt}}
\pgflineto{\pgfpoint{125.262665pt}{26.399979pt}}
\pgfusepath{stroke}
\pgfpathmoveto{\pgfpoint{127.254631pt}{29.838959pt}}
\pgflineto{\pgfpoint{126.258652pt}{28.660576pt}}
\pgfusepath{stroke}
\pgfpathmoveto{\pgfpoint{128.250610pt}{26.614319pt}}
\pgflineto{\pgfpoint{127.254631pt}{29.838959pt}}
\pgfusepath{stroke}
\pgfpathmoveto{\pgfpoint{129.246597pt}{26.451225pt}}
\pgflineto{\pgfpoint{128.250610pt}{26.614319pt}}
\pgfusepath{stroke}
\pgfpathmoveto{\pgfpoint{130.242584pt}{26.430634pt}}
\pgflineto{\pgfpoint{129.246597pt}{26.451225pt}}
\pgfusepath{stroke}
\pgfpathmoveto{\pgfpoint{131.238571pt}{27.958008pt}}
\pgflineto{\pgfpoint{130.242584pt}{26.430634pt}}
\pgfusepath{stroke}
\pgfpathmoveto{\pgfpoint{132.234558pt}{27.682930pt}}
\pgflineto{\pgfpoint{131.238571pt}{27.958008pt}}
\pgfusepath{stroke}
\pgfpathmoveto{\pgfpoint{133.230530pt}{26.399979pt}}
\pgflineto{\pgfpoint{132.234558pt}{27.682930pt}}
\pgfusepath{stroke}
\pgfpathmoveto{\pgfpoint{140.202423pt}{26.399979pt}}
\pgflineto{\pgfpoint{141.198410pt}{26.399979pt}}
\pgfusepath{stroke}
\pgfpathmoveto{\pgfpoint{139.206436pt}{26.399979pt}}
\pgflineto{\pgfpoint{140.202423pt}{26.399979pt}}
\pgfusepath{stroke}
\pgfpathmoveto{\pgfpoint{138.210449pt}{26.399979pt}}
\pgflineto{\pgfpoint{139.206436pt}{26.399979pt}}
\pgfusepath{stroke}
\pgfpathmoveto{\pgfpoint{139.206436pt}{27.485779pt}}
\pgflineto{\pgfpoint{138.210449pt}{26.399979pt}}
\pgfusepath{stroke}
\pgfpathmoveto{\pgfpoint{140.202423pt}{27.782883pt}}
\pgflineto{\pgfpoint{139.206436pt}{27.485779pt}}
\pgfusepath{stroke}
\pgfpathmoveto{\pgfpoint{141.198410pt}{26.399979pt}}
\pgflineto{\pgfpoint{140.202423pt}{27.782883pt}}
\pgfusepath{stroke}
\pgfpathmoveto{\pgfpoint{143.190369pt}{26.399979pt}}
\pgflineto{\pgfpoint{144.186356pt}{26.399979pt}}
\pgfusepath{stroke}
\pgfpathmoveto{\pgfpoint{142.194382pt}{26.399979pt}}
\pgflineto{\pgfpoint{143.190369pt}{26.399979pt}}
\pgfusepath{stroke}
\pgfpathmoveto{\pgfpoint{143.190369pt}{31.207901pt}}
\pgflineto{\pgfpoint{142.194382pt}{26.399979pt}}
\pgfusepath{stroke}
\pgfpathmoveto{\pgfpoint{144.186356pt}{26.399979pt}}
\pgflineto{\pgfpoint{143.190369pt}{31.207901pt}}
\pgfusepath{stroke}
\pgfpathmoveto{\pgfpoint{146.178314pt}{26.399979pt}}
\pgflineto{\pgfpoint{147.174316pt}{26.399979pt}}
\pgfusepath{stroke}
\pgfpathmoveto{\pgfpoint{145.182343pt}{26.399979pt}}
\pgflineto{\pgfpoint{146.178314pt}{26.399979pt}}
\pgfusepath{stroke}
\pgfpathmoveto{\pgfpoint{146.178314pt}{26.535568pt}}
\pgflineto{\pgfpoint{145.182343pt}{26.399979pt}}
\pgfusepath{stroke}
\pgfpathmoveto{\pgfpoint{147.174316pt}{26.399979pt}}
\pgflineto{\pgfpoint{146.178314pt}{26.535568pt}}
\pgfusepath{stroke}
\pgfpathmoveto{\pgfpoint{152.154221pt}{26.399979pt}}
\pgflineto{\pgfpoint{153.150208pt}{26.399979pt}}
\pgfusepath{stroke}
\pgfpathmoveto{\pgfpoint{151.158249pt}{26.399979pt}}
\pgflineto{\pgfpoint{152.154221pt}{26.399979pt}}
\pgfusepath{stroke}
\pgfpathmoveto{\pgfpoint{150.162262pt}{26.399979pt}}
\pgflineto{\pgfpoint{151.158249pt}{26.399979pt}}
\pgfusepath{stroke}
\pgfpathmoveto{\pgfpoint{151.158249pt}{26.439781pt}}
\pgflineto{\pgfpoint{150.162262pt}{26.399979pt}}
\pgfusepath{stroke}
\pgfpathmoveto{\pgfpoint{152.154221pt}{26.910515pt}}
\pgflineto{\pgfpoint{151.158249pt}{26.439781pt}}
\pgfusepath{stroke}
\pgfpathmoveto{\pgfpoint{153.150208pt}{26.399979pt}}
\pgflineto{\pgfpoint{152.154221pt}{26.910515pt}}
\pgfusepath{stroke}
\pgfpathmoveto{\pgfpoint{155.142181pt}{26.399979pt}}
\pgflineto{\pgfpoint{156.138168pt}{26.399979pt}}
\pgfusepath{stroke}
\pgfpathmoveto{\pgfpoint{154.146194pt}{26.399979pt}}
\pgflineto{\pgfpoint{155.142181pt}{26.399979pt}}
\pgfusepath{stroke}
\pgfpathmoveto{\pgfpoint{155.142181pt}{28.098244pt}}
\pgflineto{\pgfpoint{154.146194pt}{26.399979pt}}
\pgfusepath{stroke}
\pgfpathmoveto{\pgfpoint{156.138168pt}{26.399979pt}}
\pgflineto{\pgfpoint{155.142181pt}{28.098244pt}}
\pgfusepath{stroke}
\pgfpathmoveto{\pgfpoint{158.130127pt}{26.399979pt}}
\pgflineto{\pgfpoint{159.126114pt}{26.399979pt}}
\pgfusepath{stroke}
\pgfpathmoveto{\pgfpoint{157.134155pt}{26.399979pt}}
\pgflineto{\pgfpoint{158.130127pt}{26.399979pt}}
\pgfusepath{stroke}
\pgfpathmoveto{\pgfpoint{156.138168pt}{26.399979pt}}
\pgflineto{\pgfpoint{157.134155pt}{26.399979pt}}
\pgfusepath{stroke}
\pgfpathmoveto{\pgfpoint{157.134155pt}{26.492256pt}}
\pgflineto{\pgfpoint{156.138168pt}{26.399979pt}}
\pgfusepath{stroke}
\pgfpathmoveto{\pgfpoint{158.130127pt}{26.480469pt}}
\pgflineto{\pgfpoint{157.134155pt}{26.492256pt}}
\pgfusepath{stroke}
\pgfpathmoveto{\pgfpoint{159.126114pt}{26.399979pt}}
\pgflineto{\pgfpoint{158.130127pt}{26.480469pt}}
\pgfusepath{stroke}
\pgfpathmoveto{\pgfpoint{163.110062pt}{26.399979pt}}
\pgflineto{\pgfpoint{164.106033pt}{26.399979pt}}
\pgfusepath{stroke}
\pgfpathmoveto{\pgfpoint{162.114075pt}{26.399979pt}}
\pgflineto{\pgfpoint{163.110062pt}{26.399979pt}}
\pgfusepath{stroke}
\pgfpathmoveto{\pgfpoint{161.118088pt}{26.399979pt}}
\pgflineto{\pgfpoint{162.114075pt}{26.399979pt}}
\pgfusepath{stroke}
\pgfpathmoveto{\pgfpoint{160.122101pt}{26.399979pt}}
\pgflineto{\pgfpoint{161.118088pt}{26.399979pt}}
\pgfusepath{stroke}
\pgfpathmoveto{\pgfpoint{161.118088pt}{26.483330pt}}
\pgflineto{\pgfpoint{160.122101pt}{26.399979pt}}
\pgfusepath{stroke}
\pgfpathmoveto{\pgfpoint{162.114075pt}{26.604111pt}}
\pgflineto{\pgfpoint{161.118088pt}{26.483330pt}}
\pgfusepath{stroke}
\pgfpathmoveto{\pgfpoint{163.110062pt}{28.794250pt}}
\pgflineto{\pgfpoint{162.114075pt}{26.604111pt}}
\pgfusepath{stroke}
\pgfpathmoveto{\pgfpoint{164.106033pt}{26.399979pt}}
\pgflineto{\pgfpoint{163.110062pt}{28.794250pt}}
\pgfusepath{stroke}
\pgfpathmoveto{\pgfpoint{165.102020pt}{26.399979pt}}
\pgflineto{\pgfpoint{166.098007pt}{26.399979pt}}
\pgfusepath{stroke}
\pgfpathmoveto{\pgfpoint{164.106033pt}{26.399979pt}}
\pgflineto{\pgfpoint{165.102020pt}{26.399979pt}}
\pgfusepath{stroke}
\pgfpathmoveto{\pgfpoint{165.102020pt}{26.737396pt}}
\pgflineto{\pgfpoint{164.106033pt}{26.399979pt}}
\pgfusepath{stroke}
\pgfpathmoveto{\pgfpoint{166.098007pt}{26.399979pt}}
\pgflineto{\pgfpoint{165.102020pt}{26.737396pt}}
\pgfusepath{stroke}
\pgfpathmoveto{\pgfpoint{168.089966pt}{26.399979pt}}
\pgflineto{\pgfpoint{169.085953pt}{26.399979pt}}
\pgfusepath{stroke}
\pgfpathmoveto{\pgfpoint{167.093994pt}{26.399979pt}}
\pgflineto{\pgfpoint{168.089966pt}{26.399979pt}}
\pgfusepath{stroke}
\pgfpathmoveto{\pgfpoint{166.098007pt}{26.399979pt}}
\pgflineto{\pgfpoint{167.093994pt}{26.399979pt}}
\pgfusepath{stroke}
\pgfpathmoveto{\pgfpoint{167.093994pt}{30.536880pt}}
\pgflineto{\pgfpoint{166.098007pt}{26.399979pt}}
\pgfusepath{stroke}
\pgfpathmoveto{\pgfpoint{168.089966pt}{26.447098pt}}
\pgflineto{\pgfpoint{167.093994pt}{30.536880pt}}
\pgfusepath{stroke}
\pgfpathmoveto{\pgfpoint{169.085953pt}{26.399979pt}}
\pgflineto{\pgfpoint{168.089966pt}{26.447098pt}}
\pgfusepath{stroke}
\pgfpathmoveto{\pgfpoint{170.081940pt}{26.399979pt}}
\pgflineto{\pgfpoint{171.077911pt}{26.399979pt}}
\pgfusepath{stroke}
\pgfpathmoveto{\pgfpoint{169.085953pt}{26.399979pt}}
\pgflineto{\pgfpoint{170.081940pt}{26.399979pt}}
\pgfusepath{stroke}
\pgfpathmoveto{\pgfpoint{170.081940pt}{26.839508pt}}
\pgflineto{\pgfpoint{169.085953pt}{26.399979pt}}
\pgfusepath{stroke}
\pgfpathmoveto{\pgfpoint{171.077911pt}{26.399979pt}}
\pgflineto{\pgfpoint{170.081940pt}{26.839508pt}}
\pgfusepath{stroke}
\pgfpathmoveto{\pgfpoint{180.041779pt}{26.399979pt}}
\pgflineto{\pgfpoint{181.037766pt}{26.399979pt}}
\pgfusepath{stroke}
\pgfpathmoveto{\pgfpoint{179.045792pt}{26.399979pt}}
\pgflineto{\pgfpoint{180.041779pt}{26.399979pt}}
\pgfusepath{stroke}
\pgfpathmoveto{\pgfpoint{178.049805pt}{26.399979pt}}
\pgflineto{\pgfpoint{179.045792pt}{26.399979pt}}
\pgfusepath{stroke}
\pgfpathmoveto{\pgfpoint{177.053818pt}{26.399979pt}}
\pgflineto{\pgfpoint{178.049805pt}{26.399979pt}}
\pgfusepath{stroke}
\pgfpathmoveto{\pgfpoint{177.581543pt}{91.264786pt}}
\pgflineto{\pgfpoint{177.053818pt}{26.399979pt}}
\pgfusepath{stroke}
\pgfpathmoveto{\pgfpoint{179.045792pt}{26.504753pt}}
\pgflineto{\pgfpoint{178.518478pt}{91.264793pt}}
\pgfusepath{stroke}
\pgfpathmoveto{\pgfpoint{180.041779pt}{30.030716pt}}
\pgflineto{\pgfpoint{179.045792pt}{26.504753pt}}
\pgfusepath{stroke}
\pgfpathmoveto{\pgfpoint{181.037766pt}{26.399979pt}}
\pgflineto{\pgfpoint{180.041779pt}{30.030716pt}}
\pgfusepath{stroke}
\pgfpathmoveto{\pgfpoint{183.029724pt}{26.399979pt}}
\pgflineto{\pgfpoint{184.025711pt}{26.399979pt}}
\pgfusepath{stroke}
\pgfpathmoveto{\pgfpoint{182.033752pt}{26.399979pt}}
\pgflineto{\pgfpoint{183.029724pt}{26.399979pt}}
\pgfusepath{stroke}
\pgfpathmoveto{\pgfpoint{183.029724pt}{26.463455pt}}
\pgflineto{\pgfpoint{182.033752pt}{26.399979pt}}
\pgfusepath{stroke}
\pgfpathmoveto{\pgfpoint{184.025711pt}{26.399979pt}}
\pgflineto{\pgfpoint{183.029724pt}{26.463455pt}}
\pgfusepath{stroke}
\pgfpathmoveto{\pgfpoint{190.001617pt}{26.399979pt}}
\pgflineto{\pgfpoint{190.997604pt}{26.399979pt}}
\pgfusepath{stroke}
\pgfpathmoveto{\pgfpoint{189.005630pt}{26.399979pt}}
\pgflineto{\pgfpoint{190.001617pt}{26.399979pt}}
\pgfusepath{stroke}
\pgfpathmoveto{\pgfpoint{190.001617pt}{34.817444pt}}
\pgflineto{\pgfpoint{189.005630pt}{26.399979pt}}
\pgfusepath{stroke}
\pgfpathmoveto{\pgfpoint{190.997604pt}{26.399979pt}}
\pgflineto{\pgfpoint{190.001617pt}{34.817444pt}}
\pgfusepath{stroke}
\pgfpathmoveto{\pgfpoint{191.993591pt}{26.399979pt}}
\pgflineto{\pgfpoint{192.989563pt}{26.399979pt}}
\pgfusepath{stroke}
\pgfpathmoveto{\pgfpoint{190.997604pt}{26.399979pt}}
\pgflineto{\pgfpoint{191.993591pt}{26.399979pt}}
\pgfusepath{stroke}
\pgfpathmoveto{\pgfpoint{191.993591pt}{31.034332pt}}
\pgflineto{\pgfpoint{190.997604pt}{26.399979pt}}
\pgfusepath{stroke}
\pgfpathmoveto{\pgfpoint{192.989563pt}{26.399979pt}}
\pgflineto{\pgfpoint{191.993591pt}{31.034332pt}}
\pgfusepath{stroke}
\pgfpathmoveto{\pgfpoint{195.977524pt}{26.399979pt}}
\pgflineto{\pgfpoint{196.973511pt}{26.399979pt}}
\pgfusepath{stroke}
\pgfpathmoveto{\pgfpoint{194.981537pt}{26.399979pt}}
\pgflineto{\pgfpoint{195.977524pt}{26.399979pt}}
\pgfusepath{stroke}
\pgfpathmoveto{\pgfpoint{193.985565pt}{26.399979pt}}
\pgflineto{\pgfpoint{194.981537pt}{26.399979pt}}
\pgfusepath{stroke}
\pgfpathmoveto{\pgfpoint{194.981537pt}{28.872696pt}}
\pgflineto{\pgfpoint{193.985565pt}{26.399979pt}}
\pgfusepath{stroke}
\pgfpathmoveto{\pgfpoint{195.977524pt}{27.054726pt}}
\pgflineto{\pgfpoint{194.981537pt}{28.872696pt}}
\pgfusepath{stroke}
\pgfpathmoveto{\pgfpoint{196.973511pt}{26.399979pt}}
\pgflineto{\pgfpoint{195.977524pt}{27.054726pt}}
\pgfusepath{stroke}
\pgfpathmoveto{\pgfpoint{197.969498pt}{26.399979pt}}
\pgflineto{\pgfpoint{198.965469pt}{26.399979pt}}
\pgfusepath{stroke}
\pgfpathmoveto{\pgfpoint{196.973511pt}{26.399979pt}}
\pgflineto{\pgfpoint{197.969498pt}{26.399979pt}}
\pgfusepath{stroke}
\pgfpathmoveto{\pgfpoint{197.969498pt}{27.127144pt}}
\pgflineto{\pgfpoint{196.973511pt}{26.399979pt}}
\pgfusepath{stroke}
\pgfpathmoveto{\pgfpoint{198.965469pt}{26.399979pt}}
\pgflineto{\pgfpoint{197.969498pt}{27.127144pt}}
\pgfusepath{stroke}
\pgfpathmoveto{\pgfpoint{203.945404pt}{26.399979pt}}
\pgflineto{\pgfpoint{204.941376pt}{26.399979pt}}
\pgfusepath{stroke}
\pgfpathmoveto{\pgfpoint{202.949402pt}{26.399979pt}}
\pgflineto{\pgfpoint{203.945404pt}{26.399979pt}}
\pgfusepath{stroke}
\pgfpathmoveto{\pgfpoint{203.945404pt}{26.673592pt}}
\pgflineto{\pgfpoint{202.949402pt}{26.399979pt}}
\pgfusepath{stroke}
\pgfpathmoveto{\pgfpoint{204.941376pt}{26.399979pt}}
\pgflineto{\pgfpoint{203.945404pt}{26.673592pt}}
\pgfusepath{stroke}
\pgfpathmoveto{\pgfpoint{206.933334pt}{26.399979pt}}
\pgflineto{\pgfpoint{207.929337pt}{26.399979pt}}
\pgfusepath{stroke}
\pgfpathmoveto{\pgfpoint{205.937347pt}{26.399979pt}}
\pgflineto{\pgfpoint{206.933334pt}{26.399979pt}}
\pgfusepath{stroke}
\pgfpathmoveto{\pgfpoint{204.941376pt}{26.399979pt}}
\pgflineto{\pgfpoint{205.937347pt}{26.399979pt}}
\pgfusepath{stroke}
\pgfpathmoveto{\pgfpoint{205.937347pt}{29.413773pt}}
\pgflineto{\pgfpoint{204.941376pt}{26.399979pt}}
\pgfusepath{stroke}
\pgfpathmoveto{\pgfpoint{206.933334pt}{28.372978pt}}
\pgflineto{\pgfpoint{205.937347pt}{29.413773pt}}
\pgfusepath{stroke}
\pgfpathmoveto{\pgfpoint{207.929337pt}{26.399979pt}}
\pgflineto{\pgfpoint{206.933334pt}{28.372978pt}}
\pgfusepath{stroke}
\pgfpathmoveto{\pgfpoint{208.925323pt}{26.399979pt}}
\pgflineto{\pgfpoint{209.921295pt}{26.399979pt}}
\pgfusepath{stroke}
\pgfpathmoveto{\pgfpoint{207.929337pt}{26.399979pt}}
\pgflineto{\pgfpoint{208.925323pt}{26.399979pt}}
\pgfusepath{stroke}
\pgfpathmoveto{\pgfpoint{208.925323pt}{29.200577pt}}
\pgflineto{\pgfpoint{207.929337pt}{26.399979pt}}
\pgfusepath{stroke}
\pgfpathmoveto{\pgfpoint{209.921295pt}{26.399979pt}}
\pgflineto{\pgfpoint{208.925323pt}{29.200577pt}}
\pgfusepath{stroke}
\pgfpathmoveto{\pgfpoint{212.909241pt}{26.399979pt}}
\pgflineto{\pgfpoint{213.905228pt}{26.399979pt}}
\pgfusepath{stroke}
\pgfpathmoveto{\pgfpoint{211.913269pt}{26.399979pt}}
\pgflineto{\pgfpoint{212.909241pt}{26.399979pt}}
\pgfusepath{stroke}
\pgfpathmoveto{\pgfpoint{210.917267pt}{26.399979pt}}
\pgflineto{\pgfpoint{211.913269pt}{26.399979pt}}
\pgfusepath{stroke}
\pgfpathmoveto{\pgfpoint{209.921295pt}{26.399979pt}}
\pgflineto{\pgfpoint{210.917267pt}{26.399979pt}}
\pgfusepath{stroke}
\pgfpathmoveto{\pgfpoint{210.917267pt}{56.628197pt}}
\pgflineto{\pgfpoint{209.921295pt}{26.399979pt}}
\pgfusepath{stroke}
\pgfpathmoveto{\pgfpoint{211.913269pt}{30.417183pt}}
\pgflineto{\pgfpoint{210.917267pt}{56.628197pt}}
\pgfusepath{stroke}
\pgfpathmoveto{\pgfpoint{212.909241pt}{47.374290pt}}
\pgflineto{\pgfpoint{211.913269pt}{30.417183pt}}
\pgfusepath{stroke}
\pgfpathmoveto{\pgfpoint{213.905228pt}{26.399979pt}}
\pgflineto{\pgfpoint{212.909241pt}{47.374290pt}}
\pgfusepath{stroke}
\pgfpathmoveto{\pgfpoint{215.897217pt}{26.399979pt}}
\pgflineto{\pgfpoint{216.893188pt}{26.399979pt}}
\pgfusepath{stroke}
\pgfpathmoveto{\pgfpoint{214.901215pt}{26.399979pt}}
\pgflineto{\pgfpoint{215.897217pt}{26.399979pt}}
\pgfusepath{stroke}
\pgfpathmoveto{\pgfpoint{213.905228pt}{26.399979pt}}
\pgflineto{\pgfpoint{214.901215pt}{26.399979pt}}
\pgfusepath{stroke}
\pgfpathmoveto{\pgfpoint{214.901215pt}{27.407372pt}}
\pgflineto{\pgfpoint{213.905228pt}{26.399979pt}}
\pgfusepath{stroke}
\pgfpathmoveto{\pgfpoint{215.897217pt}{26.495361pt}}
\pgflineto{\pgfpoint{214.901215pt}{27.407372pt}}
\pgfusepath{stroke}
\pgfpathmoveto{\pgfpoint{216.893188pt}{26.399979pt}}
\pgflineto{\pgfpoint{215.897217pt}{26.495361pt}}
\pgfusepath{stroke}
\pgfpathmoveto{\pgfpoint{217.889160pt}{26.399979pt}}
\pgflineto{\pgfpoint{218.885147pt}{26.399979pt}}
\pgfusepath{stroke}
\pgfpathmoveto{\pgfpoint{216.893188pt}{26.399979pt}}
\pgflineto{\pgfpoint{217.889160pt}{26.399979pt}}
\pgfusepath{stroke}
\pgfpathmoveto{\pgfpoint{217.889160pt}{66.077660pt}}
\pgflineto{\pgfpoint{216.893188pt}{26.399979pt}}
\pgfusepath{stroke}
\pgfpathmoveto{\pgfpoint{218.885147pt}{26.399979pt}}
\pgflineto{\pgfpoint{217.889160pt}{66.077660pt}}
\pgfusepath{stroke}
\pgfpathmoveto{\pgfpoint{225.857040pt}{26.399979pt}}
\pgflineto{\pgfpoint{226.853027pt}{26.399979pt}}
\pgfusepath{stroke}
\pgfpathmoveto{\pgfpoint{224.861053pt}{26.399979pt}}
\pgflineto{\pgfpoint{225.857040pt}{26.399979pt}}
\pgfusepath{stroke}
\pgfpathmoveto{\pgfpoint{225.857040pt}{28.125343pt}}
\pgflineto{\pgfpoint{224.861053pt}{26.399979pt}}
\pgfusepath{stroke}
\pgfpathmoveto{\pgfpoint{226.853027pt}{26.399979pt}}
\pgflineto{\pgfpoint{225.857040pt}{28.125343pt}}
\pgfusepath{stroke}
\pgfpathmoveto{\pgfpoint{230.836945pt}{26.399979pt}}
\pgflineto{\pgfpoint{231.832932pt}{26.399979pt}}
\pgfusepath{stroke}
\pgfpathmoveto{\pgfpoint{229.840973pt}{26.399979pt}}
\pgflineto{\pgfpoint{230.836945pt}{26.399979pt}}
\pgfusepath{stroke}
\pgfpathmoveto{\pgfpoint{228.845001pt}{26.399979pt}}
\pgflineto{\pgfpoint{229.840973pt}{26.399979pt}}
\pgfusepath{stroke}
\pgfpathmoveto{\pgfpoint{227.849014pt}{26.399979pt}}
\pgflineto{\pgfpoint{228.845001pt}{26.399979pt}}
\pgfusepath{stroke}
\pgfpathmoveto{\pgfpoint{228.845001pt}{27.012459pt}}
\pgflineto{\pgfpoint{227.849014pt}{26.399979pt}}
\pgfusepath{stroke}
\pgfpathmoveto{\pgfpoint{229.840973pt}{26.558662pt}}
\pgflineto{\pgfpoint{228.845001pt}{27.012459pt}}
\pgfusepath{stroke}
\pgfpathmoveto{\pgfpoint{230.836945pt}{26.682472pt}}
\pgflineto{\pgfpoint{229.840973pt}{26.558662pt}}
\pgfusepath{stroke}
\pgfpathmoveto{\pgfpoint{231.832932pt}{26.399979pt}}
\pgflineto{\pgfpoint{230.836945pt}{26.682472pt}}
\pgfusepath{stroke}
\pgfpathmoveto{\pgfpoint{237.808838pt}{26.399979pt}}
\pgflineto{\pgfpoint{238.804825pt}{26.399979pt}}
\pgfusepath{stroke}
\pgfpathmoveto{\pgfpoint{236.812866pt}{26.399979pt}}
\pgflineto{\pgfpoint{237.808838pt}{26.399979pt}}
\pgfusepath{stroke}
\pgfpathmoveto{\pgfpoint{235.816864pt}{26.399979pt}}
\pgflineto{\pgfpoint{236.812866pt}{26.399979pt}}
\pgfusepath{stroke}
\pgfpathmoveto{\pgfpoint{236.812866pt}{35.730797pt}}
\pgflineto{\pgfpoint{235.816864pt}{26.399979pt}}
\pgfusepath{stroke}
\pgfpathmoveto{\pgfpoint{237.808838pt}{26.516998pt}}
\pgflineto{\pgfpoint{236.812866pt}{35.730797pt}}
\pgfusepath{stroke}
\pgfpathmoveto{\pgfpoint{238.804825pt}{26.399979pt}}
\pgflineto{\pgfpoint{237.808838pt}{26.516998pt}}
\pgfusepath{stroke}
\pgfpathmoveto{\pgfpoint{243.784744pt}{26.399979pt}}
\pgflineto{\pgfpoint{244.780731pt}{26.399979pt}}
\pgfusepath{stroke}
\pgfpathmoveto{\pgfpoint{242.788757pt}{26.399979pt}}
\pgflineto{\pgfpoint{243.784744pt}{26.399979pt}}
\pgfusepath{stroke}
\pgfpathmoveto{\pgfpoint{241.792786pt}{26.399979pt}}
\pgflineto{\pgfpoint{242.788757pt}{26.399979pt}}
\pgfusepath{stroke}
\pgfpathmoveto{\pgfpoint{242.788757pt}{26.819893pt}}
\pgflineto{\pgfpoint{241.792786pt}{26.399979pt}}
\pgfusepath{stroke}
\pgfpathmoveto{\pgfpoint{243.784744pt}{27.221687pt}}
\pgflineto{\pgfpoint{242.788757pt}{26.819893pt}}
\pgfusepath{stroke}
\pgfpathmoveto{\pgfpoint{244.780731pt}{26.399979pt}}
\pgflineto{\pgfpoint{243.784744pt}{27.221687pt}}
\pgfusepath{stroke}
\pgfpathmoveto{\pgfpoint{247.768677pt}{26.399979pt}}
\pgflineto{\pgfpoint{248.764679pt}{26.399979pt}}
\pgfusepath{stroke}
\pgfpathmoveto{\pgfpoint{246.772705pt}{26.399979pt}}
\pgflineto{\pgfpoint{247.768677pt}{26.399979pt}}
\pgfusepath{stroke}
\pgfpathmoveto{\pgfpoint{245.776718pt}{26.399979pt}}
\pgflineto{\pgfpoint{246.772705pt}{26.399979pt}}
\pgfusepath{stroke}
\pgfpathmoveto{\pgfpoint{244.780731pt}{26.399979pt}}
\pgflineto{\pgfpoint{245.776718pt}{26.399979pt}}
\pgfusepath{stroke}
\pgfpathmoveto{\pgfpoint{245.776718pt}{26.477646pt}}
\pgflineto{\pgfpoint{244.780731pt}{26.399979pt}}
\pgfusepath{stroke}
\pgfpathmoveto{\pgfpoint{246.772705pt}{26.550812pt}}
\pgflineto{\pgfpoint{245.776718pt}{26.477646pt}}
\pgfusepath{stroke}
\pgfpathmoveto{\pgfpoint{247.768677pt}{41.200562pt}}
\pgflineto{\pgfpoint{246.772705pt}{26.550812pt}}
\pgfusepath{stroke}
\pgfpathmoveto{\pgfpoint{248.764679pt}{26.399979pt}}
\pgflineto{\pgfpoint{247.768677pt}{41.200562pt}}
\pgfusepath{stroke}
\pgfpathmoveto{\pgfpoint{250.756638pt}{26.399979pt}}
\pgflineto{\pgfpoint{251.752625pt}{26.399979pt}}
\pgfusepath{stroke}
\pgfpathmoveto{\pgfpoint{249.760651pt}{26.399979pt}}
\pgflineto{\pgfpoint{250.756638pt}{26.399979pt}}
\pgfusepath{stroke}
\pgfpathmoveto{\pgfpoint{248.764679pt}{26.399979pt}}
\pgflineto{\pgfpoint{249.760651pt}{26.399979pt}}
\pgfusepath{stroke}
\pgfpathmoveto{\pgfpoint{249.760651pt}{26.474113pt}}
\pgflineto{\pgfpoint{248.764679pt}{26.399979pt}}
\pgfusepath{stroke}
\pgfpathmoveto{\pgfpoint{250.756638pt}{26.708969pt}}
\pgflineto{\pgfpoint{249.760651pt}{26.474113pt}}
\pgfusepath{stroke}
\pgfpathmoveto{\pgfpoint{251.752625pt}{26.399979pt}}
\pgflineto{\pgfpoint{250.756638pt}{26.708969pt}}
\pgfusepath{stroke}
\pgfpathmoveto{\pgfpoint{254.740570pt}{26.399979pt}}
\pgflineto{\pgfpoint{255.736542pt}{26.399979pt}}
\pgfusepath{stroke}
\pgfpathmoveto{\pgfpoint{253.744598pt}{26.399979pt}}
\pgflineto{\pgfpoint{254.740570pt}{26.399979pt}}
\pgfusepath{stroke}
\pgfpathmoveto{\pgfpoint{254.740570pt}{27.508163pt}}
\pgflineto{\pgfpoint{253.744598pt}{26.399979pt}}
\pgfusepath{stroke}
\pgfpathmoveto{\pgfpoint{255.736542pt}{26.399979pt}}
\pgflineto{\pgfpoint{254.740570pt}{27.508163pt}}
\pgfusepath{stroke}
\pgfpathmoveto{\pgfpoint{258.724518pt}{26.399979pt}}
\pgflineto{\pgfpoint{259.720490pt}{26.399979pt}}
\pgfusepath{stroke}
\pgfpathmoveto{\pgfpoint{257.728516pt}{26.399979pt}}
\pgflineto{\pgfpoint{258.724518pt}{26.399979pt}}
\pgfusepath{stroke}
\pgfpathmoveto{\pgfpoint{258.724518pt}{26.886185pt}}
\pgflineto{\pgfpoint{257.728516pt}{26.399979pt}}
\pgfusepath{stroke}
\pgfpathmoveto{\pgfpoint{259.720490pt}{26.399979pt}}
\pgflineto{\pgfpoint{258.724518pt}{26.886185pt}}
\pgfusepath{stroke}
\pgfpathmoveto{\pgfpoint{261.712463pt}{26.399979pt}}
\pgflineto{\pgfpoint{262.708435pt}{26.399979pt}}
\pgfusepath{stroke}
\pgfpathmoveto{\pgfpoint{260.716492pt}{26.399979pt}}
\pgflineto{\pgfpoint{261.712463pt}{26.399979pt}}
\pgfusepath{stroke}
\pgfpathmoveto{\pgfpoint{261.712463pt}{27.286751pt}}
\pgflineto{\pgfpoint{260.716492pt}{26.399979pt}}
\pgfusepath{stroke}
\pgfpathmoveto{\pgfpoint{262.708435pt}{26.399979pt}}
\pgflineto{\pgfpoint{261.712463pt}{27.286751pt}}
\pgfusepath{stroke}
\pgfpathmoveto{\pgfpoint{264.700409pt}{26.399979pt}}
\pgflineto{\pgfpoint{265.696411pt}{26.399979pt}}
\pgfusepath{stroke}
\pgfpathmoveto{\pgfpoint{263.704407pt}{26.399979pt}}
\pgflineto{\pgfpoint{264.700409pt}{26.399979pt}}
\pgfusepath{stroke}
\pgfpathmoveto{\pgfpoint{264.700409pt}{26.932861pt}}
\pgflineto{\pgfpoint{263.704407pt}{26.399979pt}}
\pgfusepath{stroke}
\pgfpathmoveto{\pgfpoint{265.696411pt}{26.399979pt}}
\pgflineto{\pgfpoint{264.700409pt}{26.932861pt}}
\pgfusepath{stroke}
\pgfpathmoveto{\pgfpoint{266.692383pt}{26.399979pt}}
\pgflineto{\pgfpoint{267.688354pt}{26.399979pt}}
\pgfusepath{stroke}
\pgfpathmoveto{\pgfpoint{265.696411pt}{26.399979pt}}
\pgflineto{\pgfpoint{266.692383pt}{26.399979pt}}
\pgfusepath{stroke}
\pgfpathmoveto{\pgfpoint{266.692383pt}{26.914734pt}}
\pgflineto{\pgfpoint{265.696411pt}{26.399979pt}}
\pgfusepath{stroke}
\pgfpathmoveto{\pgfpoint{267.688354pt}{26.399979pt}}
\pgflineto{\pgfpoint{266.692383pt}{26.914734pt}}
\pgfusepath{stroke}
\pgfpathmoveto{\pgfpoint{272.668274pt}{26.399979pt}}
\pgflineto{\pgfpoint{273.664276pt}{26.399979pt}}
\pgfusepath{stroke}
\pgfpathmoveto{\pgfpoint{271.672302pt}{26.399979pt}}
\pgflineto{\pgfpoint{272.668274pt}{26.399979pt}}
\pgfusepath{stroke}
\pgfpathmoveto{\pgfpoint{270.676331pt}{26.399979pt}}
\pgflineto{\pgfpoint{271.672302pt}{26.399979pt}}
\pgfusepath{stroke}
\pgfpathmoveto{\pgfpoint{269.680328pt}{26.399979pt}}
\pgflineto{\pgfpoint{270.676331pt}{26.399979pt}}
\pgfusepath{stroke}
\pgfpathmoveto{\pgfpoint{268.684326pt}{26.399979pt}}
\pgflineto{\pgfpoint{269.680328pt}{26.399979pt}}
\pgfusepath{stroke}
\pgfpathmoveto{\pgfpoint{267.688354pt}{26.399979pt}}
\pgflineto{\pgfpoint{268.684326pt}{26.399979pt}}
\pgfusepath{stroke}
\pgfpathmoveto{\pgfpoint{268.684326pt}{26.647873pt}}
\pgflineto{\pgfpoint{267.688354pt}{26.399979pt}}
\pgfusepath{stroke}
\pgfpathmoveto{\pgfpoint{269.680328pt}{26.596115pt}}
\pgflineto{\pgfpoint{268.684326pt}{26.647873pt}}
\pgfusepath{stroke}
\pgfpathmoveto{\pgfpoint{270.676331pt}{27.976303pt}}
\pgflineto{\pgfpoint{269.680328pt}{26.596115pt}}
\pgfusepath{stroke}
\pgfpathmoveto{\pgfpoint{271.672302pt}{42.660454pt}}
\pgflineto{\pgfpoint{270.676331pt}{27.976303pt}}
\pgfusepath{stroke}
\pgfpathmoveto{\pgfpoint{272.668274pt}{26.867470pt}}
\pgflineto{\pgfpoint{271.672302pt}{42.660454pt}}
\pgfusepath{stroke}
\pgfpathmoveto{\pgfpoint{273.664276pt}{26.399979pt}}
\pgflineto{\pgfpoint{272.668274pt}{26.867470pt}}
\pgfusepath{stroke}
\pgfpathmoveto{\pgfpoint{279.640167pt}{26.399979pt}}
\pgflineto{\pgfpoint{280.636139pt}{26.399979pt}}
\pgfusepath{stroke}
\pgfpathmoveto{\pgfpoint{278.644196pt}{26.399979pt}}
\pgflineto{\pgfpoint{279.640167pt}{26.399979pt}}
\pgfusepath{stroke}
\pgfpathmoveto{\pgfpoint{277.648193pt}{26.399979pt}}
\pgflineto{\pgfpoint{278.644196pt}{26.399979pt}}
\pgfusepath{stroke}
\pgfpathmoveto{\pgfpoint{278.644196pt}{30.203644pt}}
\pgflineto{\pgfpoint{277.648193pt}{26.399979pt}}
\pgfusepath{stroke}
\pgfpathmoveto{\pgfpoint{279.640167pt}{35.907738pt}}
\pgflineto{\pgfpoint{278.644196pt}{30.203644pt}}
\pgfusepath{stroke}
\pgfpathmoveto{\pgfpoint{280.636139pt}{26.399979pt}}
\pgflineto{\pgfpoint{279.640167pt}{35.907738pt}}
\pgfusepath{stroke}
\pgfpathmoveto{\pgfpoint{283.624115pt}{26.399979pt}}
\pgflineto{\pgfpoint{284.620087pt}{26.399979pt}}
\pgfusepath{stroke}
\pgfpathmoveto{\pgfpoint{282.628113pt}{26.399979pt}}
\pgflineto{\pgfpoint{283.624115pt}{26.399979pt}}
\pgfusepath{stroke}
\pgfpathmoveto{\pgfpoint{281.632141pt}{26.399979pt}}
\pgflineto{\pgfpoint{282.628113pt}{26.399979pt}}
\pgfusepath{stroke}
\pgfpathmoveto{\pgfpoint{282.628113pt}{29.464996pt}}
\pgflineto{\pgfpoint{281.632141pt}{26.399979pt}}
\pgfusepath{stroke}
\pgfpathmoveto{\pgfpoint{283.624115pt}{28.196754pt}}
\pgflineto{\pgfpoint{282.628113pt}{29.464996pt}}
\pgfusepath{stroke}
\pgfpathmoveto{\pgfpoint{284.620087pt}{26.399979pt}}
\pgflineto{\pgfpoint{283.624115pt}{28.196754pt}}
\pgfusepath{stroke}
\pgfpathmoveto{\pgfpoint{285.616089pt}{26.399979pt}}
\pgflineto{\pgfpoint{286.612061pt}{26.399979pt}}
\pgfusepath{stroke}
\pgfpathmoveto{\pgfpoint{284.620087pt}{26.399979pt}}
\pgflineto{\pgfpoint{285.616089pt}{26.399979pt}}
\pgfusepath{stroke}
\pgfpathmoveto{\pgfpoint{285.616089pt}{26.474419pt}}
\pgflineto{\pgfpoint{284.620087pt}{26.399979pt}}
\pgfusepath{stroke}
\pgfpathmoveto{\pgfpoint{286.612061pt}{26.399979pt}}
\pgflineto{\pgfpoint{285.616089pt}{26.474419pt}}
\pgfusepath{stroke}
\color[rgb]{0.000000,0.000000,1.000000}
\pgfsetlinewidth{2.000000pt}
\pgfpathmoveto{\pgfpoint{42.595993pt}{26.399979pt}}
\pgflineto{\pgfpoint{41.600006pt}{26.399979pt}}
\pgfusepath{stroke}
\pgfpathmoveto{\pgfpoint{43.591980pt}{26.721756pt}}
\pgflineto{\pgfpoint{42.595993pt}{26.399979pt}}
\pgfusepath{stroke}
\pgfpathmoveto{\pgfpoint{44.587967pt}{26.399979pt}}
\pgflineto{\pgfpoint{43.591980pt}{26.721756pt}}
\pgfusepath{stroke}
\pgfpathmoveto{\pgfpoint{45.583946pt}{26.399979pt}}
\pgflineto{\pgfpoint{44.587967pt}{26.399979pt}}
\pgfusepath{stroke}
\pgfpathmoveto{\pgfpoint{46.579933pt}{26.399979pt}}
\pgflineto{\pgfpoint{45.583946pt}{26.399979pt}}
\pgfusepath{stroke}
\pgfpathmoveto{\pgfpoint{47.575912pt}{26.440331pt}}
\pgflineto{\pgfpoint{46.579933pt}{26.399979pt}}
\pgfusepath{stroke}
\pgfpathmoveto{\pgfpoint{48.571899pt}{26.399979pt}}
\pgflineto{\pgfpoint{47.575912pt}{26.440331pt}}
\pgfusepath{stroke}
\pgfpathmoveto{\pgfpoint{49.567879pt}{26.399979pt}}
\pgflineto{\pgfpoint{48.571899pt}{26.399979pt}}
\pgfusepath{stroke}
\pgfpathmoveto{\pgfpoint{50.563873pt}{26.399979pt}}
\pgflineto{\pgfpoint{49.567879pt}{26.399979pt}}
\pgfusepath{stroke}
\pgfpathmoveto{\pgfpoint{51.559845pt}{26.399979pt}}
\pgflineto{\pgfpoint{50.563873pt}{26.399979pt}}
\pgfusepath{stroke}
\pgfpathmoveto{\pgfpoint{52.555840pt}{26.399979pt}}
\pgflineto{\pgfpoint{51.559845pt}{26.399979pt}}
\pgfusepath{stroke}
\pgfpathmoveto{\pgfpoint{53.551819pt}{27.127853pt}}
\pgflineto{\pgfpoint{52.555840pt}{26.399979pt}}
\pgfusepath{stroke}
\pgfpathmoveto{\pgfpoint{54.547806pt}{26.399979pt}}
\pgflineto{\pgfpoint{53.551819pt}{27.127853pt}}
\pgfusepath{stroke}
\pgfpathmoveto{\pgfpoint{55.543785pt}{26.408897pt}}
\pgflineto{\pgfpoint{54.547806pt}{26.399979pt}}
\pgfusepath{stroke}
\pgfpathmoveto{\pgfpoint{56.539772pt}{28.370888pt}}
\pgflineto{\pgfpoint{55.543785pt}{26.408897pt}}
\pgfusepath{stroke}
\pgfpathmoveto{\pgfpoint{57.535751pt}{26.399979pt}}
\pgflineto{\pgfpoint{56.539772pt}{28.370888pt}}
\pgfusepath{stroke}
\pgfpathmoveto{\pgfpoint{58.531738pt}{26.430054pt}}
\pgflineto{\pgfpoint{57.535751pt}{26.399979pt}}
\pgfusepath{stroke}
\pgfpathmoveto{\pgfpoint{59.527725pt}{26.413170pt}}
\pgflineto{\pgfpoint{58.531738pt}{26.430054pt}}
\pgfusepath{stroke}
\pgfpathmoveto{\pgfpoint{60.523712pt}{26.399979pt}}
\pgflineto{\pgfpoint{59.527725pt}{26.413170pt}}
\pgfusepath{stroke}
\pgfpathmoveto{\pgfpoint{61.519691pt}{26.399979pt}}
\pgflineto{\pgfpoint{60.523712pt}{26.399979pt}}
\pgfusepath{stroke}
\pgfpathmoveto{\pgfpoint{62.515678pt}{26.401459pt}}
\pgflineto{\pgfpoint{61.519691pt}{26.399979pt}}
\pgfusepath{stroke}
\pgfpathmoveto{\pgfpoint{63.511658pt}{26.399979pt}}
\pgflineto{\pgfpoint{62.515678pt}{26.401459pt}}
\pgfusepath{stroke}
\pgfpathmoveto{\pgfpoint{64.507637pt}{26.409256pt}}
\pgflineto{\pgfpoint{63.511658pt}{26.399979pt}}
\pgfusepath{stroke}
\pgfpathmoveto{\pgfpoint{65.503624pt}{26.424644pt}}
\pgflineto{\pgfpoint{64.507637pt}{26.409256pt}}
\pgfusepath{stroke}
\pgfpathmoveto{\pgfpoint{66.499619pt}{26.614471pt}}
\pgflineto{\pgfpoint{65.503624pt}{26.424644pt}}
\pgfusepath{stroke}
\pgfpathmoveto{\pgfpoint{67.495590pt}{26.399979pt}}
\pgflineto{\pgfpoint{66.499619pt}{26.614471pt}}
\pgfusepath{stroke}
\pgfpathmoveto{\pgfpoint{68.491577pt}{26.404778pt}}
\pgflineto{\pgfpoint{67.495590pt}{26.399979pt}}
\pgfusepath{stroke}
\pgfpathmoveto{\pgfpoint{69.487564pt}{26.463905pt}}
\pgflineto{\pgfpoint{68.491577pt}{26.404778pt}}
\pgfusepath{stroke}
\pgfpathmoveto{\pgfpoint{70.483551pt}{26.399979pt}}
\pgflineto{\pgfpoint{69.487564pt}{26.463905pt}}
\pgfusepath{stroke}
\pgfpathmoveto{\pgfpoint{71.479530pt}{29.584541pt}}
\pgflineto{\pgfpoint{70.483551pt}{26.399979pt}}
\pgfusepath{stroke}
\pgfpathmoveto{\pgfpoint{72.475510pt}{26.503563pt}}
\pgflineto{\pgfpoint{71.479530pt}{29.584541pt}}
\pgfusepath{stroke}
\pgfpathmoveto{\pgfpoint{73.471497pt}{26.399979pt}}
\pgflineto{\pgfpoint{72.475510pt}{26.503563pt}}
\pgfusepath{stroke}
\pgfpathmoveto{\pgfpoint{74.467484pt}{26.431870pt}}
\pgflineto{\pgfpoint{73.471497pt}{26.399979pt}}
\pgfusepath{stroke}
\pgfpathmoveto{\pgfpoint{75.463470pt}{26.402107pt}}
\pgflineto{\pgfpoint{74.467484pt}{26.431870pt}}
\pgfusepath{stroke}
\pgfpathmoveto{\pgfpoint{76.459442pt}{26.399979pt}}
\pgflineto{\pgfpoint{75.463470pt}{26.402107pt}}
\pgfusepath{stroke}
\pgfpathmoveto{\pgfpoint{77.455437pt}{26.399979pt}}
\pgflineto{\pgfpoint{76.459442pt}{26.399979pt}}
\pgfusepath{stroke}
\pgfpathmoveto{\pgfpoint{78.451424pt}{26.399979pt}}
\pgflineto{\pgfpoint{77.455437pt}{26.399979pt}}
\pgfusepath{stroke}
\pgfpathmoveto{\pgfpoint{79.447403pt}{26.399979pt}}
\pgflineto{\pgfpoint{78.451424pt}{26.399979pt}}
\pgfusepath{stroke}
\pgfpathmoveto{\pgfpoint{80.443390pt}{26.744049pt}}
\pgflineto{\pgfpoint{79.447403pt}{26.399979pt}}
\pgfusepath{stroke}
\pgfpathmoveto{\pgfpoint{81.439369pt}{26.646423pt}}
\pgflineto{\pgfpoint{80.443390pt}{26.744049pt}}
\pgfusepath{stroke}
\pgfpathmoveto{\pgfpoint{82.435356pt}{26.442352pt}}
\pgflineto{\pgfpoint{81.439369pt}{26.646423pt}}
\pgfusepath{stroke}
\pgfpathmoveto{\pgfpoint{83.431335pt}{26.399979pt}}
\pgflineto{\pgfpoint{82.435356pt}{26.442352pt}}
\pgfusepath{stroke}
\pgfpathmoveto{\pgfpoint{84.427322pt}{26.399979pt}}
\pgflineto{\pgfpoint{83.431335pt}{26.399979pt}}
\pgfusepath{stroke}
\pgfpathmoveto{\pgfpoint{85.423309pt}{26.399979pt}}
\pgflineto{\pgfpoint{84.427322pt}{26.399979pt}}
\pgfusepath{stroke}
\pgfpathmoveto{\pgfpoint{86.419289pt}{26.407837pt}}
\pgflineto{\pgfpoint{85.423309pt}{26.399979pt}}
\pgfusepath{stroke}
\pgfpathmoveto{\pgfpoint{87.415276pt}{26.399979pt}}
\pgflineto{\pgfpoint{86.419289pt}{26.407837pt}}
\pgfusepath{stroke}
\pgfpathmoveto{\pgfpoint{88.411255pt}{26.617722pt}}
\pgflineto{\pgfpoint{87.415276pt}{26.399979pt}}
\pgfusepath{stroke}
\pgfpathmoveto{\pgfpoint{89.407242pt}{26.399979pt}}
\pgflineto{\pgfpoint{88.411255pt}{26.617722pt}}
\pgfusepath{stroke}
\pgfpathmoveto{\pgfpoint{90.403221pt}{26.400894pt}}
\pgflineto{\pgfpoint{89.407242pt}{26.399979pt}}
\pgfusepath{stroke}
\pgfpathmoveto{\pgfpoint{91.399208pt}{26.399979pt}}
\pgflineto{\pgfpoint{90.403221pt}{26.400894pt}}
\pgfusepath{stroke}
\pgfpathmoveto{\pgfpoint{92.395187pt}{26.413033pt}}
\pgflineto{\pgfpoint{91.399208pt}{26.399979pt}}
\pgfusepath{stroke}
\pgfpathmoveto{\pgfpoint{93.391174pt}{26.413887pt}}
\pgflineto{\pgfpoint{92.395187pt}{26.413033pt}}
\pgfusepath{stroke}
\pgfpathmoveto{\pgfpoint{94.387161pt}{26.399979pt}}
\pgflineto{\pgfpoint{93.391174pt}{26.413887pt}}
\pgfusepath{stroke}
\pgfpathmoveto{\pgfpoint{95.383141pt}{26.399979pt}}
\pgflineto{\pgfpoint{94.387161pt}{26.399979pt}}
\pgfusepath{stroke}
\pgfpathmoveto{\pgfpoint{96.379128pt}{26.399979pt}}
\pgflineto{\pgfpoint{95.383141pt}{26.399979pt}}
\pgfusepath{stroke}
\pgfpathmoveto{\pgfpoint{97.375107pt}{26.399979pt}}
\pgflineto{\pgfpoint{96.379128pt}{26.399979pt}}
\pgfusepath{stroke}
\pgfpathmoveto{\pgfpoint{98.371094pt}{27.362320pt}}
\pgflineto{\pgfpoint{97.375107pt}{26.399979pt}}
\pgfusepath{stroke}
\pgfpathmoveto{\pgfpoint{99.367081pt}{30.703705pt}}
\pgflineto{\pgfpoint{98.371094pt}{27.362320pt}}
\pgfusepath{stroke}
\pgfpathmoveto{\pgfpoint{100.363068pt}{26.755974pt}}
\pgflineto{\pgfpoint{99.367081pt}{30.703705pt}}
\pgfusepath{stroke}
\pgfpathmoveto{\pgfpoint{101.359047pt}{26.399979pt}}
\pgflineto{\pgfpoint{100.363068pt}{26.755974pt}}
\pgfusepath{stroke}
\pgfpathmoveto{\pgfpoint{102.355034pt}{26.436134pt}}
\pgflineto{\pgfpoint{101.359047pt}{26.399979pt}}
\pgfusepath{stroke}
\pgfpathmoveto{\pgfpoint{103.351013pt}{26.399979pt}}
\pgflineto{\pgfpoint{102.355034pt}{26.436134pt}}
\pgfusepath{stroke}
\pgfpathmoveto{\pgfpoint{104.347000pt}{27.051666pt}}
\pgflineto{\pgfpoint{103.351013pt}{26.399979pt}}
\pgfusepath{stroke}
\pgfpathmoveto{\pgfpoint{105.342987pt}{26.399979pt}}
\pgflineto{\pgfpoint{104.347000pt}{27.051666pt}}
\pgfusepath{stroke}
\pgfpathmoveto{\pgfpoint{106.338966pt}{26.399979pt}}
\pgflineto{\pgfpoint{105.342987pt}{26.399979pt}}
\pgfusepath{stroke}
\pgfpathmoveto{\pgfpoint{107.334953pt}{26.399979pt}}
\pgflineto{\pgfpoint{106.338966pt}{26.399979pt}}
\pgfusepath{stroke}
\pgfpathmoveto{\pgfpoint{108.330933pt}{26.421722pt}}
\pgflineto{\pgfpoint{107.334953pt}{26.399979pt}}
\pgfusepath{stroke}
\pgfpathmoveto{\pgfpoint{109.326920pt}{26.454926pt}}
\pgflineto{\pgfpoint{108.330933pt}{26.421722pt}}
\pgfusepath{stroke}
\pgfpathmoveto{\pgfpoint{110.322906pt}{26.399979pt}}
\pgflineto{\pgfpoint{109.326920pt}{26.454926pt}}
\pgfusepath{stroke}
\pgfpathmoveto{\pgfpoint{111.318893pt}{26.399979pt}}
\pgflineto{\pgfpoint{110.322906pt}{26.399979pt}}
\pgfusepath{stroke}
\pgfpathmoveto{\pgfpoint{112.314873pt}{27.245140pt}}
\pgflineto{\pgfpoint{111.318893pt}{26.399979pt}}
\pgfusepath{stroke}
\pgfpathmoveto{\pgfpoint{113.310852pt}{26.404144pt}}
\pgflineto{\pgfpoint{112.314873pt}{27.245140pt}}
\pgfusepath{stroke}
\pgfpathmoveto{\pgfpoint{114.306839pt}{26.399979pt}}
\pgflineto{\pgfpoint{113.310852pt}{26.404144pt}}
\pgfusepath{stroke}
\pgfpathmoveto{\pgfpoint{115.302826pt}{26.399979pt}}
\pgflineto{\pgfpoint{114.306839pt}{26.399979pt}}
\pgfusepath{stroke}
\pgfpathmoveto{\pgfpoint{116.298813pt}{26.625999pt}}
\pgflineto{\pgfpoint{115.302826pt}{26.399979pt}}
\pgfusepath{stroke}
\pgfpathmoveto{\pgfpoint{117.294792pt}{26.408501pt}}
\pgflineto{\pgfpoint{116.298813pt}{26.625999pt}}
\pgfusepath{stroke}
\pgfpathmoveto{\pgfpoint{118.290779pt}{41.526093pt}}
\pgflineto{\pgfpoint{117.294792pt}{26.408501pt}}
\pgfusepath{stroke}
\pgfpathmoveto{\pgfpoint{119.286758pt}{26.399979pt}}
\pgflineto{\pgfpoint{118.290779pt}{41.526093pt}}
\pgfusepath{stroke}
\pgfpathmoveto{\pgfpoint{120.282745pt}{26.399979pt}}
\pgflineto{\pgfpoint{119.286758pt}{26.399979pt}}
\pgfusepath{stroke}
\pgfpathmoveto{\pgfpoint{121.278725pt}{26.441711pt}}
\pgflineto{\pgfpoint{120.282745pt}{26.399979pt}}
\pgfusepath{stroke}
\pgfpathmoveto{\pgfpoint{122.274712pt}{26.399979pt}}
\pgflineto{\pgfpoint{121.278725pt}{26.441711pt}}
\pgfusepath{stroke}
\pgfpathmoveto{\pgfpoint{123.270691pt}{26.409576pt}}
\pgflineto{\pgfpoint{122.274712pt}{26.399979pt}}
\pgfusepath{stroke}
\pgfpathmoveto{\pgfpoint{124.266678pt}{26.399979pt}}
\pgflineto{\pgfpoint{123.270691pt}{26.409576pt}}
\pgfusepath{stroke}
\pgfpathmoveto{\pgfpoint{125.262665pt}{26.399979pt}}
\pgflineto{\pgfpoint{124.266678pt}{26.399979pt}}
\pgfusepath{stroke}
\pgfpathmoveto{\pgfpoint{126.258652pt}{26.571793pt}}
\pgflineto{\pgfpoint{125.262665pt}{26.399979pt}}
\pgfusepath{stroke}
\pgfpathmoveto{\pgfpoint{127.254631pt}{26.565819pt}}
\pgflineto{\pgfpoint{126.258652pt}{26.571793pt}}
\pgfusepath{stroke}
\pgfpathmoveto{\pgfpoint{128.250610pt}{26.405869pt}}
\pgflineto{\pgfpoint{127.254631pt}{26.565819pt}}
\pgfusepath{stroke}
\pgfpathmoveto{\pgfpoint{129.246597pt}{26.400833pt}}
\pgflineto{\pgfpoint{128.250610pt}{26.405869pt}}
\pgfusepath{stroke}
\pgfpathmoveto{\pgfpoint{130.242584pt}{26.400490pt}}
\pgflineto{\pgfpoint{129.246597pt}{26.400833pt}}
\pgfusepath{stroke}
\pgfpathmoveto{\pgfpoint{131.238571pt}{26.425949pt}}
\pgflineto{\pgfpoint{130.242584pt}{26.400490pt}}
\pgfusepath{stroke}
\pgfpathmoveto{\pgfpoint{132.234558pt}{26.445862pt}}
\pgflineto{\pgfpoint{131.238571pt}{26.425949pt}}
\pgfusepath{stroke}
\pgfpathmoveto{\pgfpoint{133.230530pt}{26.399979pt}}
\pgflineto{\pgfpoint{132.234558pt}{26.445862pt}}
\pgfusepath{stroke}
\pgfpathmoveto{\pgfpoint{134.226517pt}{26.399979pt}}
\pgflineto{\pgfpoint{133.230530pt}{26.399979pt}}
\pgfusepath{stroke}
\pgfpathmoveto{\pgfpoint{135.222504pt}{26.399979pt}}
\pgflineto{\pgfpoint{134.226517pt}{26.399979pt}}
\pgfusepath{stroke}
\pgfpathmoveto{\pgfpoint{136.218475pt}{26.399979pt}}
\pgflineto{\pgfpoint{135.222504pt}{26.399979pt}}
\pgfusepath{stroke}
\pgfpathmoveto{\pgfpoint{137.214478pt}{26.399979pt}}
\pgflineto{\pgfpoint{136.218475pt}{26.399979pt}}
\pgfusepath{stroke}
\pgfpathmoveto{\pgfpoint{138.210449pt}{26.399979pt}}
\pgflineto{\pgfpoint{137.214478pt}{26.399979pt}}
\pgfusepath{stroke}
\pgfpathmoveto{\pgfpoint{139.206436pt}{26.449463pt}}
\pgflineto{\pgfpoint{138.210449pt}{26.399979pt}}
\pgfusepath{stroke}
\pgfpathmoveto{\pgfpoint{140.202423pt}{26.439690pt}}
\pgflineto{\pgfpoint{139.206436pt}{26.449463pt}}
\pgfusepath{stroke}
\pgfpathmoveto{\pgfpoint{141.198410pt}{26.399979pt}}
\pgflineto{\pgfpoint{140.202423pt}{26.439690pt}}
\pgfusepath{stroke}
\pgfpathmoveto{\pgfpoint{142.194382pt}{26.399979pt}}
\pgflineto{\pgfpoint{141.198410pt}{26.399979pt}}
\pgfusepath{stroke}
\pgfpathmoveto{\pgfpoint{143.190369pt}{26.747536pt}}
\pgflineto{\pgfpoint{142.194382pt}{26.399979pt}}
\pgfusepath{stroke}
\pgfpathmoveto{\pgfpoint{144.186356pt}{26.399979pt}}
\pgflineto{\pgfpoint{143.190369pt}{26.747536pt}}
\pgfusepath{stroke}
\pgfpathmoveto{\pgfpoint{145.182343pt}{26.399979pt}}
\pgflineto{\pgfpoint{144.186356pt}{26.399979pt}}
\pgfusepath{stroke}
\pgfpathmoveto{\pgfpoint{146.178314pt}{26.402245pt}}
\pgflineto{\pgfpoint{145.182343pt}{26.399979pt}}
\pgfusepath{stroke}
\pgfpathmoveto{\pgfpoint{147.174316pt}{26.399979pt}}
\pgflineto{\pgfpoint{146.178314pt}{26.402245pt}}
\pgfusepath{stroke}
\pgfpathmoveto{\pgfpoint{148.170288pt}{26.399979pt}}
\pgflineto{\pgfpoint{147.174316pt}{26.399979pt}}
\pgfusepath{stroke}
\pgfpathmoveto{\pgfpoint{149.166275pt}{26.399979pt}}
\pgflineto{\pgfpoint{148.170288pt}{26.399979pt}}
\pgfusepath{stroke}
\pgfpathmoveto{\pgfpoint{150.162262pt}{26.399979pt}}
\pgflineto{\pgfpoint{149.166275pt}{26.399979pt}}
\pgfusepath{stroke}
\pgfpathmoveto{\pgfpoint{151.158249pt}{26.400650pt}}
\pgflineto{\pgfpoint{150.162262pt}{26.399979pt}}
\pgfusepath{stroke}
\pgfpathmoveto{\pgfpoint{152.154221pt}{26.413498pt}}
\pgflineto{\pgfpoint{151.158249pt}{26.400650pt}}
\pgfusepath{stroke}
\pgfpathmoveto{\pgfpoint{153.150208pt}{26.399979pt}}
\pgflineto{\pgfpoint{152.154221pt}{26.413498pt}}
\pgfusepath{stroke}
\pgfpathmoveto{\pgfpoint{154.146194pt}{26.399979pt}}
\pgflineto{\pgfpoint{153.150208pt}{26.399979pt}}
\pgfusepath{stroke}
\pgfpathmoveto{\pgfpoint{155.142181pt}{26.428284pt}}
\pgflineto{\pgfpoint{154.146194pt}{26.399979pt}}
\pgfusepath{stroke}
\pgfpathmoveto{\pgfpoint{156.138168pt}{26.399979pt}}
\pgflineto{\pgfpoint{155.142181pt}{26.428284pt}}
\pgfusepath{stroke}
\pgfpathmoveto{\pgfpoint{157.134155pt}{26.402534pt}}
\pgflineto{\pgfpoint{156.138168pt}{26.399979pt}}
\pgfusepath{stroke}
\pgfpathmoveto{\pgfpoint{158.130127pt}{26.404060pt}}
\pgflineto{\pgfpoint{157.134155pt}{26.402534pt}}
\pgfusepath{stroke}
\pgfpathmoveto{\pgfpoint{159.126114pt}{26.399979pt}}
\pgflineto{\pgfpoint{158.130127pt}{26.404060pt}}
\pgfusepath{stroke}
\pgfpathmoveto{\pgfpoint{160.122101pt}{26.399979pt}}
\pgflineto{\pgfpoint{159.126114pt}{26.399979pt}}
\pgfusepath{stroke}
\pgfpathmoveto{\pgfpoint{161.118088pt}{26.401367pt}}
\pgflineto{\pgfpoint{160.122101pt}{26.399979pt}}
\pgfusepath{stroke}
\pgfpathmoveto{\pgfpoint{162.114075pt}{26.404663pt}}
\pgflineto{\pgfpoint{161.118088pt}{26.401367pt}}
\pgfusepath{stroke}
\pgfpathmoveto{\pgfpoint{163.110062pt}{26.439888pt}}
\pgflineto{\pgfpoint{162.114075pt}{26.404663pt}}
\pgfusepath{stroke}
\pgfpathmoveto{\pgfpoint{164.106033pt}{26.399979pt}}
\pgflineto{\pgfpoint{163.110062pt}{26.439888pt}}
\pgfusepath{stroke}
\pgfpathmoveto{\pgfpoint{165.102020pt}{26.405602pt}}
\pgflineto{\pgfpoint{164.106033pt}{26.399979pt}}
\pgfusepath{stroke}
\pgfpathmoveto{\pgfpoint{166.098007pt}{26.399979pt}}
\pgflineto{\pgfpoint{165.102020pt}{26.405602pt}}
\pgfusepath{stroke}
\pgfpathmoveto{\pgfpoint{167.093994pt}{26.491440pt}}
\pgflineto{\pgfpoint{166.098007pt}{26.399979pt}}
\pgfusepath{stroke}
\pgfpathmoveto{\pgfpoint{168.089966pt}{26.401459pt}}
\pgflineto{\pgfpoint{167.093994pt}{26.491440pt}}
\pgfusepath{stroke}
\pgfpathmoveto{\pgfpoint{169.085953pt}{26.399979pt}}
\pgflineto{\pgfpoint{168.089966pt}{26.401459pt}}
\pgfusepath{stroke}
\pgfpathmoveto{\pgfpoint{170.081940pt}{26.417023pt}}
\pgflineto{\pgfpoint{169.085953pt}{26.399979pt}}
\pgfusepath{stroke}
\pgfpathmoveto{\pgfpoint{171.077911pt}{26.399979pt}}
\pgflineto{\pgfpoint{170.081940pt}{26.417023pt}}
\pgfusepath{stroke}
\pgfpathmoveto{\pgfpoint{172.073914pt}{26.399979pt}}
\pgflineto{\pgfpoint{171.077911pt}{26.399979pt}}
\pgfusepath{stroke}
\pgfpathmoveto{\pgfpoint{173.069885pt}{26.399979pt}}
\pgflineto{\pgfpoint{172.073914pt}{26.399979pt}}
\pgfusepath{stroke}
\pgfpathmoveto{\pgfpoint{174.065872pt}{26.399979pt}}
\pgflineto{\pgfpoint{173.069885pt}{26.399979pt}}
\pgfusepath{stroke}
\pgfpathmoveto{\pgfpoint{175.061859pt}{26.399979pt}}
\pgflineto{\pgfpoint{174.065872pt}{26.399979pt}}
\pgfusepath{stroke}
\pgfpathmoveto{\pgfpoint{176.057846pt}{26.399979pt}}
\pgflineto{\pgfpoint{175.061859pt}{26.399979pt}}
\pgfusepath{stroke}
\pgfpathmoveto{\pgfpoint{177.053818pt}{26.399979pt}}
\pgflineto{\pgfpoint{176.057846pt}{26.399979pt}}
\pgfusepath{stroke}
\pgfpathmoveto{\pgfpoint{178.049805pt}{38.817703pt}}
\pgflineto{\pgfpoint{177.053818pt}{26.399979pt}}
\pgfusepath{stroke}
\pgfpathmoveto{\pgfpoint{179.045792pt}{26.402115pt}}
\pgflineto{\pgfpoint{178.049805pt}{38.817703pt}}
\pgfusepath{stroke}
\pgfpathmoveto{\pgfpoint{180.041779pt}{26.478966pt}}
\pgflineto{\pgfpoint{179.045792pt}{26.402115pt}}
\pgfusepath{stroke}
\pgfpathmoveto{\pgfpoint{181.037766pt}{26.399979pt}}
\pgflineto{\pgfpoint{180.041779pt}{26.478966pt}}
\pgfusepath{stroke}
\pgfpathmoveto{\pgfpoint{182.033752pt}{26.399979pt}}
\pgflineto{\pgfpoint{181.037766pt}{26.399979pt}}
\pgfusepath{stroke}
\pgfpathmoveto{\pgfpoint{183.029724pt}{26.401039pt}}
\pgflineto{\pgfpoint{182.033752pt}{26.399979pt}}
\pgfusepath{stroke}
\pgfpathmoveto{\pgfpoint{184.025711pt}{26.399979pt}}
\pgflineto{\pgfpoint{183.029724pt}{26.401039pt}}
\pgfusepath{stroke}
\pgfpathmoveto{\pgfpoint{185.021698pt}{26.399979pt}}
\pgflineto{\pgfpoint{184.025711pt}{26.399979pt}}
\pgfusepath{stroke}
\pgfpathmoveto{\pgfpoint{186.017685pt}{26.399979pt}}
\pgflineto{\pgfpoint{185.021698pt}{26.399979pt}}
\pgfusepath{stroke}
\pgfpathmoveto{\pgfpoint{187.013672pt}{26.399979pt}}
\pgflineto{\pgfpoint{186.017685pt}{26.399979pt}}
\pgfusepath{stroke}
\pgfpathmoveto{\pgfpoint{188.009659pt}{26.399979pt}}
\pgflineto{\pgfpoint{187.013672pt}{26.399979pt}}
\pgfusepath{stroke}
\pgfpathmoveto{\pgfpoint{189.005630pt}{26.399979pt}}
\pgflineto{\pgfpoint{188.009659pt}{26.399979pt}}
\pgfusepath{stroke}
\pgfpathmoveto{\pgfpoint{190.001617pt}{27.040497pt}}
\pgflineto{\pgfpoint{189.005630pt}{26.399979pt}}
\pgfusepath{stroke}
\pgfpathmoveto{\pgfpoint{190.997604pt}{26.399979pt}}
\pgflineto{\pgfpoint{190.001617pt}{27.040497pt}}
\pgfusepath{stroke}
\pgfpathmoveto{\pgfpoint{191.993591pt}{26.562309pt}}
\pgflineto{\pgfpoint{190.997604pt}{26.399979pt}}
\pgfusepath{stroke}
\pgfpathmoveto{\pgfpoint{192.989563pt}{26.399979pt}}
\pgflineto{\pgfpoint{191.993591pt}{26.562309pt}}
\pgfusepath{stroke}
\pgfpathmoveto{\pgfpoint{193.985565pt}{26.399979pt}}
\pgflineto{\pgfpoint{192.989563pt}{26.399979pt}}
\pgfusepath{stroke}
\pgfpathmoveto{\pgfpoint{194.981537pt}{26.700401pt}}
\pgflineto{\pgfpoint{193.985565pt}{26.399979pt}}
\pgfusepath{stroke}
\pgfpathmoveto{\pgfpoint{195.977524pt}{26.438347pt}}
\pgflineto{\pgfpoint{194.981537pt}{26.700401pt}}
\pgfusepath{stroke}
\pgfpathmoveto{\pgfpoint{196.973511pt}{26.399979pt}}
\pgflineto{\pgfpoint{195.977524pt}{26.438347pt}}
\pgfusepath{stroke}
\pgfpathmoveto{\pgfpoint{197.969498pt}{26.425056pt}}
\pgflineto{\pgfpoint{196.973511pt}{26.399979pt}}
\pgfusepath{stroke}
\pgfpathmoveto{\pgfpoint{198.965469pt}{26.399979pt}}
\pgflineto{\pgfpoint{197.969498pt}{26.425056pt}}
\pgfusepath{stroke}
\pgfpathmoveto{\pgfpoint{199.961456pt}{26.399979pt}}
\pgflineto{\pgfpoint{198.965469pt}{26.399979pt}}
\pgfusepath{stroke}
\pgfpathmoveto{\pgfpoint{200.957443pt}{26.399979pt}}
\pgflineto{\pgfpoint{199.961456pt}{26.399979pt}}
\pgfusepath{stroke}
\pgfpathmoveto{\pgfpoint{201.953430pt}{26.399979pt}}
\pgflineto{\pgfpoint{200.957443pt}{26.399979pt}}
\pgfusepath{stroke}
\pgfpathmoveto{\pgfpoint{202.949402pt}{26.399979pt}}
\pgflineto{\pgfpoint{201.953430pt}{26.399979pt}}
\pgfusepath{stroke}
\pgfpathmoveto{\pgfpoint{203.945404pt}{26.404549pt}}
\pgflineto{\pgfpoint{202.949402pt}{26.399979pt}}
\pgfusepath{stroke}
\pgfpathmoveto{\pgfpoint{204.941376pt}{26.399979pt}}
\pgflineto{\pgfpoint{203.945404pt}{26.404549pt}}
\pgfusepath{stroke}
\pgfpathmoveto{\pgfpoint{205.937347pt}{26.498993pt}}
\pgflineto{\pgfpoint{204.941376pt}{26.399979pt}}
\pgfusepath{stroke}
\pgfpathmoveto{\pgfpoint{206.933334pt}{26.624886pt}}
\pgflineto{\pgfpoint{205.937347pt}{26.498993pt}}
\pgfusepath{stroke}
\pgfpathmoveto{\pgfpoint{207.929337pt}{26.399979pt}}
\pgflineto{\pgfpoint{206.933334pt}{26.624886pt}}
\pgfusepath{stroke}
\pgfpathmoveto{\pgfpoint{208.925323pt}{26.499222pt}}
\pgflineto{\pgfpoint{207.929337pt}{26.399979pt}}
\pgfusepath{stroke}
\pgfpathmoveto{\pgfpoint{209.921295pt}{26.399979pt}}
\pgflineto{\pgfpoint{208.925323pt}{26.499222pt}}
\pgfusepath{stroke}
\pgfpathmoveto{\pgfpoint{210.917267pt}{28.758682pt}}
\pgflineto{\pgfpoint{209.921295pt}{26.399979pt}}
\pgfusepath{stroke}
\pgfpathmoveto{\pgfpoint{211.913269pt}{26.580238pt}}
\pgflineto{\pgfpoint{210.917267pt}{28.758682pt}}
\pgfusepath{stroke}
\pgfpathmoveto{\pgfpoint{212.909241pt}{27.484299pt}}
\pgflineto{\pgfpoint{211.913269pt}{26.580238pt}}
\pgfusepath{stroke}
\pgfpathmoveto{\pgfpoint{213.905228pt}{26.399979pt}}
\pgflineto{\pgfpoint{212.909241pt}{27.484299pt}}
\pgfusepath{stroke}
\pgfpathmoveto{\pgfpoint{214.901215pt}{26.416771pt}}
\pgflineto{\pgfpoint{213.905228pt}{26.399979pt}}
\pgfusepath{stroke}
\pgfpathmoveto{\pgfpoint{215.897217pt}{26.401573pt}}
\pgflineto{\pgfpoint{214.901215pt}{26.416771pt}}
\pgfusepath{stroke}
\pgfpathmoveto{\pgfpoint{216.893188pt}{26.399979pt}}
\pgflineto{\pgfpoint{215.897217pt}{26.401573pt}}
\pgfusepath{stroke}
\pgfpathmoveto{\pgfpoint{217.889160pt}{30.564590pt}}
\pgflineto{\pgfpoint{216.893188pt}{26.399979pt}}
\pgfusepath{stroke}
\pgfpathmoveto{\pgfpoint{218.885147pt}{26.399979pt}}
\pgflineto{\pgfpoint{217.889160pt}{30.564590pt}}
\pgfusepath{stroke}
\pgfpathmoveto{\pgfpoint{219.881134pt}{26.399979pt}}
\pgflineto{\pgfpoint{218.885147pt}{26.399979pt}}
\pgfusepath{stroke}
\pgfpathmoveto{\pgfpoint{220.877121pt}{26.399979pt}}
\pgflineto{\pgfpoint{219.881134pt}{26.399979pt}}
\pgfusepath{stroke}
\pgfpathmoveto{\pgfpoint{221.873108pt}{26.399979pt}}
\pgflineto{\pgfpoint{220.877121pt}{26.399979pt}}
\pgfusepath{stroke}
\pgfpathmoveto{\pgfpoint{222.869080pt}{26.399979pt}}
\pgflineto{\pgfpoint{221.873108pt}{26.399979pt}}
\pgfusepath{stroke}
\pgfpathmoveto{\pgfpoint{223.865082pt}{26.399979pt}}
\pgflineto{\pgfpoint{222.869080pt}{26.399979pt}}
\pgfusepath{stroke}
\pgfpathmoveto{\pgfpoint{224.861053pt}{26.399979pt}}
\pgflineto{\pgfpoint{223.865082pt}{26.399979pt}}
\pgfusepath{stroke}
\pgfpathmoveto{\pgfpoint{225.857040pt}{26.482925pt}}
\pgflineto{\pgfpoint{224.861053pt}{26.399979pt}}
\pgfusepath{stroke}
\pgfpathmoveto{\pgfpoint{226.853027pt}{26.399979pt}}
\pgflineto{\pgfpoint{225.857040pt}{26.482925pt}}
\pgfusepath{stroke}
\pgfpathmoveto{\pgfpoint{227.849014pt}{26.399979pt}}
\pgflineto{\pgfpoint{226.853027pt}{26.399979pt}}
\pgfusepath{stroke}
\pgfpathmoveto{\pgfpoint{228.845001pt}{26.419968pt}}
\pgflineto{\pgfpoint{227.849014pt}{26.399979pt}}
\pgfusepath{stroke}
\pgfpathmoveto{\pgfpoint{229.840973pt}{26.402626pt}}
\pgflineto{\pgfpoint{228.845001pt}{26.419968pt}}
\pgfusepath{stroke}
\pgfpathmoveto{\pgfpoint{230.836945pt}{26.404686pt}}
\pgflineto{\pgfpoint{229.840973pt}{26.402626pt}}
\pgfusepath{stroke}
\pgfpathmoveto{\pgfpoint{231.832932pt}{26.399979pt}}
\pgflineto{\pgfpoint{230.836945pt}{26.404686pt}}
\pgfusepath{stroke}
\pgfpathmoveto{\pgfpoint{232.828934pt}{26.399979pt}}
\pgflineto{\pgfpoint{231.832932pt}{26.399979pt}}
\pgfusepath{stroke}
\pgfpathmoveto{\pgfpoint{233.824921pt}{26.399979pt}}
\pgflineto{\pgfpoint{232.828934pt}{26.399979pt}}
\pgfusepath{stroke}
\pgfpathmoveto{\pgfpoint{234.820892pt}{26.399979pt}}
\pgflineto{\pgfpoint{233.824921pt}{26.399979pt}}
\pgfusepath{stroke}
\pgfpathmoveto{\pgfpoint{235.816864pt}{26.399979pt}}
\pgflineto{\pgfpoint{234.820892pt}{26.399979pt}}
\pgfusepath{stroke}
\pgfpathmoveto{\pgfpoint{236.812866pt}{27.336357pt}}
\pgflineto{\pgfpoint{235.816864pt}{26.399979pt}}
\pgfusepath{stroke}
\pgfpathmoveto{\pgfpoint{237.808838pt}{26.403709pt}}
\pgflineto{\pgfpoint{236.812866pt}{27.336357pt}}
\pgfusepath{stroke}
\pgfpathmoveto{\pgfpoint{238.804825pt}{26.399979pt}}
\pgflineto{\pgfpoint{237.808838pt}{26.403709pt}}
\pgfusepath{stroke}
\pgfpathmoveto{\pgfpoint{239.800812pt}{26.399979pt}}
\pgflineto{\pgfpoint{238.804825pt}{26.399979pt}}
\pgfusepath{stroke}
\pgfpathmoveto{\pgfpoint{240.796814pt}{26.399979pt}}
\pgflineto{\pgfpoint{239.800812pt}{26.399979pt}}
\pgfusepath{stroke}
\pgfpathmoveto{\pgfpoint{241.792786pt}{26.399979pt}}
\pgflineto{\pgfpoint{240.796814pt}{26.399979pt}}
\pgfusepath{stroke}
\pgfpathmoveto{\pgfpoint{242.788757pt}{26.408813pt}}
\pgflineto{\pgfpoint{241.792786pt}{26.399979pt}}
\pgfusepath{stroke}
\pgfpathmoveto{\pgfpoint{243.784744pt}{26.426201pt}}
\pgflineto{\pgfpoint{242.788757pt}{26.408813pt}}
\pgfusepath{stroke}
\pgfpathmoveto{\pgfpoint{244.780731pt}{26.399979pt}}
\pgflineto{\pgfpoint{243.784744pt}{26.426201pt}}
\pgfusepath{stroke}
\pgfpathmoveto{\pgfpoint{245.776718pt}{26.401276pt}}
\pgflineto{\pgfpoint{244.780731pt}{26.399979pt}}
\pgfusepath{stroke}
\pgfpathmoveto{\pgfpoint{246.772705pt}{26.412910pt}}
\pgflineto{\pgfpoint{245.776718pt}{26.401276pt}}
\pgfusepath{stroke}
\pgfpathmoveto{\pgfpoint{247.768677pt}{28.808250pt}}
\pgflineto{\pgfpoint{246.772705pt}{26.412910pt}}
\pgfusepath{stroke}
\pgfpathmoveto{\pgfpoint{248.764679pt}{26.399979pt}}
\pgflineto{\pgfpoint{247.768677pt}{28.808250pt}}
\pgfusepath{stroke}
\pgfpathmoveto{\pgfpoint{249.760651pt}{26.401222pt}}
\pgflineto{\pgfpoint{248.764679pt}{26.399979pt}}
\pgfusepath{stroke}
\pgfpathmoveto{\pgfpoint{250.756638pt}{26.407082pt}}
\pgflineto{\pgfpoint{249.760651pt}{26.401222pt}}
\pgfusepath{stroke}
\pgfpathmoveto{\pgfpoint{251.752625pt}{26.399979pt}}
\pgflineto{\pgfpoint{250.756638pt}{26.407082pt}}
\pgfusepath{stroke}
\pgfpathmoveto{\pgfpoint{252.748611pt}{26.399979pt}}
\pgflineto{\pgfpoint{251.752625pt}{26.399979pt}}
\pgfusepath{stroke}
\pgfpathmoveto{\pgfpoint{253.744598pt}{26.399979pt}}
\pgflineto{\pgfpoint{252.748611pt}{26.399979pt}}
\pgfusepath{stroke}
\pgfpathmoveto{\pgfpoint{254.740570pt}{26.418449pt}}
\pgflineto{\pgfpoint{253.744598pt}{26.399979pt}}
\pgfusepath{stroke}
\pgfpathmoveto{\pgfpoint{255.736542pt}{26.399979pt}}
\pgflineto{\pgfpoint{254.740570pt}{26.418449pt}}
\pgfusepath{stroke}
\pgfpathmoveto{\pgfpoint{256.732544pt}{26.399979pt}}
\pgflineto{\pgfpoint{255.736542pt}{26.399979pt}}
\pgfusepath{stroke}
\pgfpathmoveto{\pgfpoint{257.728516pt}{26.399979pt}}
\pgflineto{\pgfpoint{256.732544pt}{26.399979pt}}
\pgfusepath{stroke}
\pgfpathmoveto{\pgfpoint{258.724518pt}{26.417419pt}}
\pgflineto{\pgfpoint{257.728516pt}{26.399979pt}}
\pgfusepath{stroke}
\pgfpathmoveto{\pgfpoint{259.720490pt}{26.399979pt}}
\pgflineto{\pgfpoint{258.724518pt}{26.417419pt}}
\pgfusepath{stroke}
\pgfpathmoveto{\pgfpoint{260.716492pt}{26.399979pt}}
\pgflineto{\pgfpoint{259.720490pt}{26.399979pt}}
\pgfusepath{stroke}
\pgfpathmoveto{\pgfpoint{261.712463pt}{26.456230pt}}
\pgflineto{\pgfpoint{260.716492pt}{26.399979pt}}
\pgfusepath{stroke}
\pgfpathmoveto{\pgfpoint{262.708435pt}{26.399979pt}}
\pgflineto{\pgfpoint{261.712463pt}{26.456230pt}}
\pgfusepath{stroke}
\pgfpathmoveto{\pgfpoint{263.704407pt}{26.399979pt}}
\pgflineto{\pgfpoint{262.708435pt}{26.399979pt}}
\pgfusepath{stroke}
\pgfpathmoveto{\pgfpoint{264.700409pt}{26.422783pt}}
\pgflineto{\pgfpoint{263.704407pt}{26.399979pt}}
\pgfusepath{stroke}
\pgfpathmoveto{\pgfpoint{265.696411pt}{26.399979pt}}
\pgflineto{\pgfpoint{264.700409pt}{26.422783pt}}
\pgfusepath{stroke}
\pgfpathmoveto{\pgfpoint{266.692383pt}{26.434059pt}}
\pgflineto{\pgfpoint{265.696411pt}{26.399979pt}}
\pgfusepath{stroke}
\pgfpathmoveto{\pgfpoint{267.688354pt}{26.399979pt}}
\pgflineto{\pgfpoint{266.692383pt}{26.434059pt}}
\pgfusepath{stroke}
\pgfpathmoveto{\pgfpoint{268.684326pt}{26.407372pt}}
\pgflineto{\pgfpoint{267.688354pt}{26.399979pt}}
\pgfusepath{stroke}
\pgfpathmoveto{\pgfpoint{269.680328pt}{26.407188pt}}
\pgflineto{\pgfpoint{268.684326pt}{26.407372pt}}
\pgfusepath{stroke}
\pgfpathmoveto{\pgfpoint{270.676331pt}{26.458199pt}}
\pgflineto{\pgfpoint{269.680328pt}{26.407188pt}}
\pgfusepath{stroke}
\pgfpathmoveto{\pgfpoint{271.672302pt}{28.359955pt}}
\pgflineto{\pgfpoint{270.676331pt}{26.458199pt}}
\pgfusepath{stroke}
\pgfpathmoveto{\pgfpoint{272.668274pt}{26.415123pt}}
\pgflineto{\pgfpoint{271.672302pt}{28.359955pt}}
\pgfusepath{stroke}
\pgfpathmoveto{\pgfpoint{273.664276pt}{26.399979pt}}
\pgflineto{\pgfpoint{272.668274pt}{26.415123pt}}
\pgfusepath{stroke}
\pgfpathmoveto{\pgfpoint{274.660248pt}{26.399979pt}}
\pgflineto{\pgfpoint{273.664276pt}{26.399979pt}}
\pgfusepath{stroke}
\pgfpathmoveto{\pgfpoint{275.656250pt}{26.399979pt}}
\pgflineto{\pgfpoint{274.660248pt}{26.399979pt}}
\pgfusepath{stroke}
\pgfpathmoveto{\pgfpoint{276.652222pt}{26.399979pt}}
\pgflineto{\pgfpoint{275.656250pt}{26.399979pt}}
\pgfusepath{stroke}
\pgfpathmoveto{\pgfpoint{277.648193pt}{26.399979pt}}
\pgflineto{\pgfpoint{276.652222pt}{26.399979pt}}
\pgfusepath{stroke}
\pgfpathmoveto{\pgfpoint{278.644196pt}{26.623550pt}}
\pgflineto{\pgfpoint{277.648193pt}{26.399979pt}}
\pgfusepath{stroke}
\pgfpathmoveto{\pgfpoint{279.640167pt}{27.888870pt}}
\pgflineto{\pgfpoint{278.644196pt}{26.623550pt}}
\pgfusepath{stroke}
\pgfpathmoveto{\pgfpoint{280.636139pt}{26.399979pt}}
\pgflineto{\pgfpoint{279.640167pt}{27.888870pt}}
\pgfusepath{stroke}
\pgfpathmoveto{\pgfpoint{281.632141pt}{26.399979pt}}
\pgflineto{\pgfpoint{280.636139pt}{26.399979pt}}
\pgfusepath{stroke}
\pgfpathmoveto{\pgfpoint{282.628113pt}{26.524590pt}}
\pgflineto{\pgfpoint{281.632141pt}{26.399979pt}}
\pgfusepath{stroke}
\pgfpathmoveto{\pgfpoint{283.624115pt}{26.456955pt}}
\pgflineto{\pgfpoint{282.628113pt}{26.524590pt}}
\pgfusepath{stroke}
\pgfpathmoveto{\pgfpoint{284.620087pt}{26.399979pt}}
\pgflineto{\pgfpoint{283.624115pt}{26.456955pt}}
\pgfusepath{stroke}
\pgfpathmoveto{\pgfpoint{285.616089pt}{26.402275pt}}
\pgflineto{\pgfpoint{284.620087pt}{26.399979pt}}
\pgfusepath{stroke}
\pgfpathmoveto{\pgfpoint{286.612061pt}{26.399979pt}}
\pgflineto{\pgfpoint{285.616089pt}{26.402275pt}}
\pgfusepath{stroke}
\pgfpathmoveto{\pgfpoint{287.608032pt}{26.399979pt}}
\pgflineto{\pgfpoint{286.612061pt}{26.399979pt}}
\pgfusepath{stroke}
\pgfpathmoveto{\pgfpoint{288.604004pt}{26.399979pt}}
\pgflineto{\pgfpoint{287.608032pt}{26.399979pt}}
\pgfusepath{stroke}
\pgfpathmoveto{\pgfpoint{289.600037pt}{26.399979pt}}
\pgflineto{\pgfpoint{288.604004pt}{26.399979pt}}
\pgfusepath{stroke}
{
\pgftransformshift{\pgfpoint{165.600006pt}{215.577454pt}}
\pgfnode{rectangle}{south}{\fontsize{10}{0}\selectfont\textcolor[rgb]{0,0,0}{{Spectral statistics VVX strain}}}{}{\pgfusepath{discard}}}
{
\pgftransformshift{\pgfpoint{165.600006pt}{101.199989pt}}
\pgfnode{rectangle}{south}{\fontsize{10}{0}\selectfont\textcolor[rgb]{0,0,0}{{Spectral statistics for BUT }}}{}{\pgfusepath{discard}}}
\end{pgfpicture}

\end{frame}

\only<article>{
  Let's tackle the problem of discriminating between different
disease vectors. Ideally, we'd like to have a simple test that
tells us what ails us. One kind of test is mass spectrometry. This
graph shows spectrometry results for two types of bacteria. There
is plenty of variation within each type, both due to measurement
error and due to changes in the bacterial strains. Here, we plot
the average and maximum energies measured for about 100 different
examples from each strain.
}

\begin{frame}
  \frametitle{Nearest neighbour: the hidden secret of machine learning}
  \input{../figures/separation1.tikz}
\end{frame}

\only<article>{
  Now, is it possible to identify an unknown strain based on this
data? Actually, this is possible. Sometimes, very simple algorithms
work very well. One of the simplest one involves just measuring the
distance between the decsription of a new unknown strain and known
ones. In this visualisation, I projected the 1300-dimensional data
into a 2-dimensional space. Here you can clearly see that it is
possible to separate the two strains. In order to classify a new
point, you just need to see whether it's closer to the train VVT or
BUT.
}

\begin{frame}
\frametitle{Comparing spectral data}
  \only<1>{\input{../figures/difference1.tikz}}
  \only<2>{% Title: glps_renderer figure
% Creator: GL2PS 1.3.8, (C) 1999-2012 C. Geuzaine
% For: Octave
% CreationDate: Fri Jun 16 12:38:10 2017
\begin{pgfpicture}
\pgfsetlinewidth{0.01pt}
\color[rgb]{1.000000,1.000000,1.000000}
\pgfpathmoveto{\pgfpoint{41.600006pt}{222.000000pt}}
\pgflineto{\pgfpoint{289.600037pt}{26.399979pt}}
\pgflineto{\pgfpoint{41.600006pt}{26.399979pt}}
\pgfpathclose
\pgfusepath{fill,stroke}
\pgfpathmoveto{\pgfpoint{41.600006pt}{222.000000pt}}
\pgflineto{\pgfpoint{289.600037pt}{222.000000pt}}
\pgflineto{\pgfpoint{289.600037pt}{26.399979pt}}
\pgfpathclose
\pgfusepath{fill,stroke}
\color[rgb]{1.000000,0.000000,0.000000}
\pgfpathmoveto{\pgfpoint{46.560013pt}{171.656006pt}}
\pgflineto{\pgfpoint{46.560013pt}{65.485382pt}}
\pgflineto{\pgfpoint{51.272934pt}{60.450993pt}}
\pgfpathclose
\pgfusepath{fill,stroke}
\pgfpathmoveto{\pgfpoint{51.272934pt}{60.450993pt}}
\pgflineto{\pgfpoint{51.520004pt}{54.621078pt}}
\pgflineto{\pgfpoint{51.520004pt}{60.187065pt}}
\pgfpathclose
\pgfusepath{fill,stroke}
\pgfpathmoveto{\pgfpoint{52.729080pt}{55.915833pt}}
\pgflineto{\pgfpoint{51.520004pt}{60.187065pt}}
\pgflineto{\pgfpoint{51.520004pt}{54.621078pt}}
\pgfpathclose
\pgfusepath{fill,stroke}
\pgfpathmoveto{\pgfpoint{52.729080pt}{55.915833pt}}
\pgflineto{\pgfpoint{56.480011pt}{42.665169pt}}
\pgflineto{\pgfpoint{56.480011pt}{59.932564pt}}
\pgfpathclose
\pgfusepath{fill,stroke}
\pgfpathmoveto{\pgfpoint{61.440010pt}{105.284721pt}}
\pgflineto{\pgfpoint{56.480011pt}{59.932564pt}}
\pgflineto{\pgfpoint{56.480011pt}{42.665169pt}}
\pgfpathclose
\pgfusepath{fill,stroke}
\pgfpathmoveto{\pgfpoint{61.440010pt}{46.590111pt}}
\pgflineto{\pgfpoint{61.440010pt}{105.284721pt}}
\pgflineto{\pgfpoint{56.480011pt}{42.665169pt}}
\pgfpathclose
\pgfusepath{fill,stroke}
\pgfpathmoveto{\pgfpoint{66.400009pt}{163.107224pt}}
\pgflineto{\pgfpoint{61.440010pt}{105.284721pt}}
\pgflineto{\pgfpoint{61.440010pt}{46.590111pt}}
\pgfpathclose
\pgfusepath{fill,stroke}
\pgfpathmoveto{\pgfpoint{66.400009pt}{63.127323pt}}
\pgflineto{\pgfpoint{66.400009pt}{163.107224pt}}
\pgflineto{\pgfpoint{61.440010pt}{46.590111pt}}
\pgfpathclose
\pgfusepath{fill,stroke}
\pgfpathmoveto{\pgfpoint{69.454124pt}{89.661858pt}}
\pgflineto{\pgfpoint{66.400009pt}{163.107224pt}}
\pgflineto{\pgfpoint{66.400009pt}{63.127323pt}}
\pgfpathclose
\pgfusepath{fill,stroke}
\pgfpathmoveto{\pgfpoint{69.454124pt}{89.661858pt}}
\pgflineto{\pgfpoint{71.360008pt}{43.829178pt}}
\pgflineto{\pgfpoint{71.360008pt}{106.220398pt}}
\pgfpathclose
\pgfusepath{fill,stroke}
\pgfpathmoveto{\pgfpoint{73.884773pt}{93.204117pt}}
\pgflineto{\pgfpoint{71.360008pt}{106.220398pt}}
\pgflineto{\pgfpoint{71.360008pt}{43.829178pt}}
\pgfpathclose
\pgfusepath{fill,stroke}
\pgfpathmoveto{\pgfpoint{73.884773pt}{93.204117pt}}
\pgflineto{\pgfpoint{76.320007pt}{80.649338pt}}
\pgflineto{\pgfpoint{76.320007pt}{140.828415pt}}
\pgfpathclose
\pgfusepath{fill,stroke}
\pgfpathmoveto{\pgfpoint{80.181015pt}{65.915634pt}}
\pgflineto{\pgfpoint{76.320007pt}{140.828415pt}}
\pgflineto{\pgfpoint{76.320007pt}{80.649338pt}}
\pgfpathclose
\pgfusepath{fill,stroke}
\pgfpathmoveto{\pgfpoint{80.181015pt}{65.915634pt}}
\pgflineto{\pgfpoint{81.280014pt}{44.592613pt}}
\pgflineto{\pgfpoint{81.280014pt}{61.721870pt}}
\pgfpathclose
\pgfusepath{fill,stroke}
\pgfpathmoveto{\pgfpoint{85.898277pt}{60.412163pt}}
\pgflineto{\pgfpoint{81.280014pt}{61.721870pt}}
\pgflineto{\pgfpoint{81.280014pt}{44.592613pt}}
\pgfpathclose
\pgfusepath{fill,stroke}
\pgfpathmoveto{\pgfpoint{85.898277pt}{60.412163pt}}
\pgflineto{\pgfpoint{86.240013pt}{60.315250pt}}
\pgflineto{\pgfpoint{86.240013pt}{61.582748pt}}
\pgfpathclose
\pgfusepath{fill,stroke}
\pgfpathmoveto{\pgfpoint{91.200012pt}{82.405769pt}}
\pgflineto{\pgfpoint{86.240013pt}{61.582748pt}}
\pgflineto{\pgfpoint{86.240013pt}{60.315250pt}}
\pgfpathclose
\pgfusepath{fill,stroke}
\pgfpathmoveto{\pgfpoint{91.200012pt}{67.963776pt}}
\pgflineto{\pgfpoint{91.200012pt}{82.405769pt}}
\pgflineto{\pgfpoint{86.240013pt}{60.315250pt}}
\pgfpathclose
\pgfusepath{fill,stroke}
\pgfpathmoveto{\pgfpoint{96.160011pt}{121.447739pt}}
\pgflineto{\pgfpoint{91.200012pt}{82.405769pt}}
\pgflineto{\pgfpoint{91.200012pt}{67.963776pt}}
\pgfpathclose
\pgfusepath{fill,stroke}
\pgfpathmoveto{\pgfpoint{96.160011pt}{39.422333pt}}
\pgflineto{\pgfpoint{96.160011pt}{121.447739pt}}
\pgflineto{\pgfpoint{91.200012pt}{67.963776pt}}
\pgfpathclose
\pgfusepath{fill,stroke}
\pgfpathmoveto{\pgfpoint{101.120010pt}{45.170807pt}}
\pgflineto{\pgfpoint{96.160011pt}{121.447739pt}}
\pgflineto{\pgfpoint{96.160011pt}{39.422333pt}}
\pgfpathclose
\pgfusepath{fill,stroke}
\pgfpathmoveto{\pgfpoint{101.120010pt}{36.805199pt}}
\pgflineto{\pgfpoint{101.120010pt}{45.170807pt}}
\pgflineto{\pgfpoint{96.160011pt}{39.422333pt}}
\pgfpathclose
\pgfusepath{fill,stroke}
\pgfpathmoveto{\pgfpoint{106.080017pt}{93.045525pt}}
\pgflineto{\pgfpoint{101.120010pt}{45.170807pt}}
\pgflineto{\pgfpoint{101.120010pt}{36.805199pt}}
\pgfpathclose
\pgfusepath{fill,stroke}
\pgfpathmoveto{\pgfpoint{106.080017pt}{57.417770pt}}
\pgflineto{\pgfpoint{106.080017pt}{93.045525pt}}
\pgflineto{\pgfpoint{101.120010pt}{36.805199pt}}
\pgfpathclose
\pgfusepath{fill,stroke}
\pgfpathmoveto{\pgfpoint{111.040009pt}{89.065750pt}}
\pgflineto{\pgfpoint{106.080017pt}{93.045525pt}}
\pgflineto{\pgfpoint{106.080017pt}{57.417770pt}}
\pgfpathclose
\pgfusepath{fill,stroke}
\pgfpathmoveto{\pgfpoint{111.040009pt}{48.803902pt}}
\pgflineto{\pgfpoint{111.040009pt}{89.065750pt}}
\pgflineto{\pgfpoint{106.080017pt}{57.417770pt}}
\pgfpathclose
\pgfusepath{fill,stroke}
\pgfpathmoveto{\pgfpoint{116.000015pt}{43.683624pt}}
\pgflineto{\pgfpoint{111.040009pt}{89.065750pt}}
\pgflineto{\pgfpoint{111.040009pt}{48.803902pt}}
\pgfpathclose
\pgfusepath{fill,stroke}
\pgfpathmoveto{\pgfpoint{116.000015pt}{40.934975pt}}
\pgflineto{\pgfpoint{116.000015pt}{43.683624pt}}
\pgflineto{\pgfpoint{111.040009pt}{48.803902pt}}
\pgfpathclose
\pgfusepath{fill,stroke}
\pgfpathmoveto{\pgfpoint{120.960007pt}{149.515289pt}}
\pgflineto{\pgfpoint{116.000015pt}{43.683624pt}}
\pgflineto{\pgfpoint{116.000015pt}{40.934975pt}}
\pgfpathclose
\pgfusepath{fill,stroke}
\pgfpathmoveto{\pgfpoint{120.960007pt}{115.830338pt}}
\pgflineto{\pgfpoint{120.960007pt}{149.515289pt}}
\pgflineto{\pgfpoint{116.000015pt}{40.934975pt}}
\pgfpathclose
\pgfusepath{fill,stroke}
\pgfpathmoveto{\pgfpoint{125.920013pt}{72.071335pt}}
\pgflineto{\pgfpoint{120.960007pt}{149.515289pt}}
\pgflineto{\pgfpoint{120.960007pt}{115.830338pt}}
\pgfpathclose
\pgfusepath{fill,stroke}
\pgfpathmoveto{\pgfpoint{125.920013pt}{66.937149pt}}
\pgflineto{\pgfpoint{125.920013pt}{72.071335pt}}
\pgflineto{\pgfpoint{120.960007pt}{115.830338pt}}
\pgfpathclose
\pgfusepath{fill,stroke}
\pgfpathmoveto{\pgfpoint{127.352821pt}{64.771484pt}}
\pgflineto{\pgfpoint{125.920013pt}{72.071335pt}}
\pgflineto{\pgfpoint{125.920013pt}{66.937149pt}}
\pgfpathclose
\pgfusepath{fill,stroke}
\pgfpathmoveto{\pgfpoint{127.352821pt}{64.771484pt}}
\pgflineto{\pgfpoint{130.880005pt}{46.801132pt}}
\pgflineto{\pgfpoint{130.880005pt}{59.440186pt}}
\pgfpathclose
\pgfusepath{fill,stroke}
\pgfpathmoveto{\pgfpoint{132.276794pt}{53.129593pt}}
\pgflineto{\pgfpoint{130.880005pt}{59.440186pt}}
\pgflineto{\pgfpoint{130.880005pt}{46.801132pt}}
\pgfpathclose
\pgfusepath{fill,stroke}
\pgfpathmoveto{\pgfpoint{132.276794pt}{53.129593pt}}
\pgflineto{\pgfpoint{135.840012pt}{37.031158pt}}
\pgflineto{\pgfpoint{135.840012pt}{69.273598pt}}
\pgfpathclose
\pgfusepath{fill,stroke}
\pgfpathmoveto{\pgfpoint{138.230530pt}{66.341614pt}}
\pgflineto{\pgfpoint{135.840012pt}{69.273598pt}}
\pgflineto{\pgfpoint{135.840012pt}{37.031158pt}}
\pgfpathclose
\pgfusepath{fill,stroke}
\pgfpathmoveto{\pgfpoint{138.230530pt}{66.341614pt}}
\pgflineto{\pgfpoint{140.800003pt}{63.190159pt}}
\pgflineto{\pgfpoint{140.800003pt}{97.846222pt}}
\pgfpathclose
\pgfusepath{fill,stroke}
\pgfpathmoveto{\pgfpoint{145.760010pt}{49.291016pt}}
\pgflineto{\pgfpoint{140.800003pt}{97.846222pt}}
\pgflineto{\pgfpoint{140.800003pt}{63.190159pt}}
\pgfpathclose
\pgfusepath{fill,stroke}
\pgfpathmoveto{\pgfpoint{145.760010pt}{45.087975pt}}
\pgflineto{\pgfpoint{145.760010pt}{49.291016pt}}
\pgflineto{\pgfpoint{140.800003pt}{63.190159pt}}
\pgfpathclose
\pgfusepath{fill,stroke}
\pgfpathmoveto{\pgfpoint{148.486938pt}{53.263680pt}}
\pgflineto{\pgfpoint{145.760010pt}{49.291016pt}}
\pgflineto{\pgfpoint{145.760010pt}{45.087975pt}}
\pgfpathclose
\pgfusepath{fill,stroke}
\pgfpathmoveto{\pgfpoint{148.486938pt}{53.263680pt}}
\pgflineto{\pgfpoint{150.720016pt}{56.516876pt}}
\pgflineto{\pgfpoint{150.720016pt}{59.958729pt}}
\pgfpathclose
\pgfusepath{fill,stroke}
\pgfpathmoveto{\pgfpoint{155.680023pt}{145.320908pt}}
\pgflineto{\pgfpoint{150.720016pt}{59.958729pt}}
\pgflineto{\pgfpoint{150.720016pt}{56.516876pt}}
\pgfpathclose
\pgfusepath{fill,stroke}
\pgfpathmoveto{\pgfpoint{155.680023pt}{39.056114pt}}
\pgflineto{\pgfpoint{155.680023pt}{145.320908pt}}
\pgflineto{\pgfpoint{150.720016pt}{56.516876pt}}
\pgfpathclose
\pgfusepath{fill,stroke}
\pgfpathmoveto{\pgfpoint{158.851837pt}{88.183441pt}}
\pgflineto{\pgfpoint{155.680023pt}{145.320908pt}}
\pgflineto{\pgfpoint{155.680023pt}{39.056114pt}}
\pgfpathclose
\pgfusepath{fill,stroke}
\pgfpathmoveto{\pgfpoint{158.851837pt}{88.183441pt}}
\pgflineto{\pgfpoint{160.640015pt}{55.971481pt}}
\pgflineto{\pgfpoint{160.640015pt}{115.879601pt}}
\pgfpathclose
\pgfusepath{fill,stroke}
\pgfpathmoveto{\pgfpoint{165.274719pt}{64.387100pt}}
\pgflineto{\pgfpoint{160.640015pt}{115.879601pt}}
\pgflineto{\pgfpoint{160.640015pt}{55.971481pt}}
\pgfpathclose
\pgfusepath{fill,stroke}
\pgfpathmoveto{\pgfpoint{165.274719pt}{64.387100pt}}
\pgflineto{\pgfpoint{165.600006pt}{60.773132pt}}
\pgflineto{\pgfpoint{165.600006pt}{64.977737pt}}
\pgfpathclose
\pgfusepath{fill,stroke}
\pgfpathmoveto{\pgfpoint{166.011597pt}{63.599285pt}}
\pgflineto{\pgfpoint{165.600006pt}{64.977737pt}}
\pgflineto{\pgfpoint{165.600006pt}{60.773132pt}}
\pgfpathclose
\pgfusepath{fill,stroke}
\pgfpathmoveto{\pgfpoint{166.011597pt}{63.599285pt}}
\pgflineto{\pgfpoint{170.560013pt}{48.365959pt}}
\pgflineto{\pgfpoint{170.560013pt}{94.831261pt}}
\pgfpathclose
\pgfusepath{fill,stroke}
\pgfpathmoveto{\pgfpoint{174.223160pt}{57.045738pt}}
\pgflineto{\pgfpoint{170.560013pt}{94.831261pt}}
\pgflineto{\pgfpoint{170.560013pt}{48.365959pt}}
\pgfpathclose
\pgfusepath{fill,stroke}
\pgfpathmoveto{\pgfpoint{174.223160pt}{57.045738pt}}
\pgflineto{\pgfpoint{175.520004pt}{43.668625pt}}
\pgflineto{\pgfpoint{175.520004pt}{60.118607pt}}
\pgfpathclose
\pgfusepath{fill,stroke}
\pgfpathmoveto{\pgfpoint{180.480011pt}{75.116745pt}}
\pgflineto{\pgfpoint{175.520004pt}{60.118607pt}}
\pgflineto{\pgfpoint{175.520004pt}{43.668625pt}}
\pgfpathclose
\pgfusepath{fill,stroke}
\pgfpathmoveto{\pgfpoint{180.480011pt}{40.850761pt}}
\pgflineto{\pgfpoint{180.480011pt}{75.116745pt}}
\pgflineto{\pgfpoint{175.520004pt}{43.668625pt}}
\pgfpathclose
\pgfusepath{fill,stroke}
\pgfpathmoveto{\pgfpoint{182.813766pt}{65.866241pt}}
\pgflineto{\pgfpoint{180.480011pt}{75.116745pt}}
\pgflineto{\pgfpoint{180.480011pt}{40.850761pt}}
\pgfpathclose
\pgfusepath{fill,stroke}
\pgfpathmoveto{\pgfpoint{182.813766pt}{65.866241pt}}
\pgflineto{\pgfpoint{185.440018pt}{55.456352pt}}
\pgflineto{\pgfpoint{185.440018pt}{94.016953pt}}
\pgfpathclose
\pgfusepath{fill,stroke}
\pgfpathmoveto{\pgfpoint{190.400024pt}{74.316742pt}}
\pgflineto{\pgfpoint{185.440018pt}{94.016953pt}}
\pgflineto{\pgfpoint{185.440018pt}{55.456352pt}}
\pgfpathclose
\pgfusepath{fill,stroke}
\pgfpathmoveto{\pgfpoint{190.400024pt}{40.646240pt}}
\pgflineto{\pgfpoint{190.400024pt}{74.316742pt}}
\pgflineto{\pgfpoint{185.440018pt}{55.456352pt}}
\pgfpathclose
\pgfusepath{fill,stroke}
\pgfpathmoveto{\pgfpoint{194.088348pt}{48.193687pt}}
\pgflineto{\pgfpoint{190.400024pt}{74.316742pt}}
\pgflineto{\pgfpoint{190.400024pt}{40.646240pt}}
\pgfpathclose
\pgfusepath{fill,stroke}
\pgfpathmoveto{\pgfpoint{194.088348pt}{48.193687pt}}
\pgflineto{\pgfpoint{195.360016pt}{39.186867pt}}
\pgflineto{\pgfpoint{195.360016pt}{50.795929pt}}
\pgfpathclose
\pgfusepath{fill,stroke}
\pgfpathmoveto{\pgfpoint{200.320007pt}{162.896408pt}}
\pgflineto{\pgfpoint{195.360016pt}{50.795929pt}}
\pgflineto{\pgfpoint{195.360016pt}{39.186867pt}}
\pgfpathclose
\pgfusepath{fill,stroke}
\pgfpathmoveto{\pgfpoint{200.320007pt}{111.419617pt}}
\pgflineto{\pgfpoint{200.320007pt}{162.896408pt}}
\pgflineto{\pgfpoint{195.360016pt}{39.186867pt}}
\pgfpathclose
\pgfusepath{fill,stroke}
\pgfpathmoveto{\pgfpoint{205.279999pt}{99.158791pt}}
\pgflineto{\pgfpoint{200.320007pt}{162.896408pt}}
\pgflineto{\pgfpoint{200.320007pt}{111.419617pt}}
\pgfpathclose
\pgfusepath{fill,stroke}
\pgfpathmoveto{\pgfpoint{205.279999pt}{77.342728pt}}
\pgflineto{\pgfpoint{205.279999pt}{99.158791pt}}
\pgflineto{\pgfpoint{200.320007pt}{111.419617pt}}
\pgfpathclose
\pgfusepath{fill,stroke}
\pgfpathmoveto{\pgfpoint{210.147217pt}{64.919937pt}}
\pgflineto{\pgfpoint{205.279999pt}{99.158791pt}}
\pgflineto{\pgfpoint{205.279999pt}{77.342728pt}}
\pgfpathclose
\pgfusepath{fill,stroke}
\pgfpathmoveto{\pgfpoint{210.147217pt}{64.919937pt}}
\pgflineto{\pgfpoint{210.240021pt}{64.267227pt}}
\pgflineto{\pgfpoint{210.240021pt}{64.683121pt}}
\pgfpathclose
\pgfusepath{fill,stroke}
\pgfpathmoveto{\pgfpoint{215.200012pt}{68.944382pt}}
\pgflineto{\pgfpoint{210.240021pt}{64.683121pt}}
\pgflineto{\pgfpoint{210.240021pt}{64.267227pt}}
\pgfpathclose
\pgfusepath{fill,stroke}
\pgfpathmoveto{\pgfpoint{215.200012pt}{36.115891pt}}
\pgflineto{\pgfpoint{215.200012pt}{68.944382pt}}
\pgflineto{\pgfpoint{210.240021pt}{64.267227pt}}
\pgfpathclose
\pgfusepath{fill,stroke}
\pgfpathmoveto{\pgfpoint{220.160019pt}{56.577995pt}}
\pgflineto{\pgfpoint{215.200012pt}{68.944382pt}}
\pgflineto{\pgfpoint{215.200012pt}{36.115891pt}}
\pgfpathclose
\pgfusepath{fill,stroke}
\pgfpathmoveto{\pgfpoint{220.160019pt}{48.298035pt}}
\pgflineto{\pgfpoint{220.160019pt}{56.577995pt}}
\pgflineto{\pgfpoint{215.200012pt}{36.115891pt}}
\pgfpathclose
\pgfusepath{fill,stroke}
\pgfpathmoveto{\pgfpoint{225.120026pt}{63.517673pt}}
\pgflineto{\pgfpoint{220.160019pt}{56.577995pt}}
\pgflineto{\pgfpoint{220.160019pt}{48.298035pt}}
\pgfpathclose
\pgfusepath{fill,stroke}
\pgfpathmoveto{\pgfpoint{225.120026pt}{51.340904pt}}
\pgflineto{\pgfpoint{225.120026pt}{63.517673pt}}
\pgflineto{\pgfpoint{220.160019pt}{48.298035pt}}
\pgfpathclose
\pgfusepath{fill,stroke}
\pgfpathmoveto{\pgfpoint{228.633514pt}{65.886925pt}}
\pgflineto{\pgfpoint{225.120026pt}{63.517673pt}}
\pgflineto{\pgfpoint{225.120026pt}{51.340904pt}}
\pgfpathclose
\pgfusepath{fill,stroke}
\pgfpathmoveto{\pgfpoint{228.633514pt}{65.886925pt}}
\pgflineto{\pgfpoint{230.080017pt}{66.862328pt}}
\pgflineto{\pgfpoint{230.080017pt}{71.875427pt}}
\pgfpathclose
\pgfusepath{fill,stroke}
\pgfpathmoveto{\pgfpoint{231.822998pt}{62.078987pt}}
\pgflineto{\pgfpoint{230.080017pt}{71.875427pt}}
\pgflineto{\pgfpoint{230.080017pt}{66.862328pt}}
\pgfpathclose
\pgfusepath{fill,stroke}
\pgfpathmoveto{\pgfpoint{231.822998pt}{62.078987pt}}
\pgflineto{\pgfpoint{235.040024pt}{43.997490pt}}
\pgflineto{\pgfpoint{235.040024pt}{53.250259pt}}
\pgfpathclose
\pgfusepath{fill,stroke}
\pgfpathmoveto{\pgfpoint{240.000000pt}{75.645462pt}}
\pgflineto{\pgfpoint{235.040024pt}{53.250259pt}}
\pgflineto{\pgfpoint{235.040024pt}{43.997490pt}}
\pgfpathclose
\pgfusepath{fill,stroke}
\pgfpathmoveto{\pgfpoint{240.000000pt}{61.037575pt}}
\pgflineto{\pgfpoint{240.000000pt}{75.645462pt}}
\pgflineto{\pgfpoint{235.040024pt}{43.997490pt}}
\pgfpathclose
\pgfusepath{fill,stroke}
\pgfpathmoveto{\pgfpoint{244.506226pt}{42.584846pt}}
\pgflineto{\pgfpoint{240.000000pt}{75.645462pt}}
\pgflineto{\pgfpoint{240.000000pt}{61.037575pt}}
\pgfpathclose
\pgfusepath{fill,stroke}
\pgfpathmoveto{\pgfpoint{244.506226pt}{42.584846pt}}
\pgflineto{\pgfpoint{244.960022pt}{39.255608pt}}
\pgflineto{\pgfpoint{244.960022pt}{40.726631pt}}
\pgfpathclose
\pgfusepath{fill,stroke}
\pgfpathmoveto{\pgfpoint{249.920013pt}{116.421631pt}}
\pgflineto{\pgfpoint{244.960022pt}{40.726631pt}}
\pgflineto{\pgfpoint{244.960022pt}{39.255608pt}}
\pgfpathclose
\pgfusepath{fill,stroke}
\pgfpathmoveto{\pgfpoint{249.920013pt}{106.094025pt}}
\pgflineto{\pgfpoint{249.920013pt}{116.421631pt}}
\pgflineto{\pgfpoint{244.960022pt}{39.255608pt}}
\pgfpathclose
\pgfusepath{fill,stroke}
\pgfpathmoveto{\pgfpoint{254.880020pt}{215.171143pt}}
\pgflineto{\pgfpoint{249.920013pt}{116.421631pt}}
\pgflineto{\pgfpoint{249.920013pt}{106.094025pt}}
\pgfpathclose
\pgfusepath{fill,stroke}
\pgfpathmoveto{\pgfpoint{254.880020pt}{70.739304pt}}
\pgflineto{\pgfpoint{254.880020pt}{215.171143pt}}
\pgflineto{\pgfpoint{249.920013pt}{106.094025pt}}
\pgfpathclose
\pgfusepath{fill,stroke}
\pgfpathmoveto{\pgfpoint{259.840027pt}{141.531891pt}}
\pgflineto{\pgfpoint{254.880020pt}{215.171143pt}}
\pgflineto{\pgfpoint{254.880020pt}{70.739304pt}}
\pgfpathclose
\pgfusepath{fill,stroke}
\pgfpathmoveto{\pgfpoint{259.840027pt}{73.925293pt}}
\pgflineto{\pgfpoint{259.840027pt}{141.531891pt}}
\pgflineto{\pgfpoint{254.880020pt}{70.739304pt}}
\pgfpathclose
\pgfusepath{fill,stroke}
\pgfpathmoveto{\pgfpoint{264.800018pt}{190.998230pt}}
\pgflineto{\pgfpoint{259.840027pt}{141.531891pt}}
\pgflineto{\pgfpoint{259.840027pt}{73.925293pt}}
\pgfpathclose
\pgfusepath{fill,stroke}
\pgfpathmoveto{\pgfpoint{264.800018pt}{34.848969pt}}
\pgflineto{\pgfpoint{264.800018pt}{190.998230pt}}
\pgflineto{\pgfpoint{259.840027pt}{73.925293pt}}
\pgfpathclose
\pgfusepath{fill,stroke}
\pgfpathmoveto{\pgfpoint{269.487213pt}{69.737061pt}}
\pgflineto{\pgfpoint{264.800018pt}{190.998230pt}}
\pgflineto{\pgfpoint{264.800018pt}{34.848969pt}}
\pgfpathclose
\pgfusepath{fill,stroke}
\pgfpathmoveto{\pgfpoint{269.487213pt}{69.737061pt}}
\pgflineto{\pgfpoint{269.760010pt}{62.679287pt}}
\pgflineto{\pgfpoint{269.760010pt}{71.767647pt}}
\pgfpathclose
\pgfusepath{fill,stroke}
\pgfpathmoveto{\pgfpoint{274.720001pt}{76.811203pt}}
\pgflineto{\pgfpoint{269.760010pt}{71.767647pt}}
\pgflineto{\pgfpoint{269.760010pt}{62.679287pt}}
\pgfpathclose
\pgfusepath{fill,stroke}
\pgfpathmoveto{\pgfpoint{274.720001pt}{37.509773pt}}
\pgflineto{\pgfpoint{274.720001pt}{76.811203pt}}
\pgflineto{\pgfpoint{269.760010pt}{62.679287pt}}
\pgfpathclose
\pgfusepath{fill,stroke}
\pgfpathmoveto{\pgfpoint{279.680023pt}{69.222656pt}}
\pgflineto{\pgfpoint{274.720001pt}{76.811203pt}}
\pgflineto{\pgfpoint{274.720001pt}{37.509773pt}}
\pgfpathclose
\pgfusepath{fill,stroke}
\pgfpathmoveto{\pgfpoint{279.680023pt}{54.383209pt}}
\pgflineto{\pgfpoint{279.680023pt}{69.222656pt}}
\pgflineto{\pgfpoint{274.720001pt}{37.509773pt}}
\pgfpathclose
\pgfusepath{fill,stroke}
\pgfpathmoveto{\pgfpoint{280.898163pt}{61.260727pt}}
\pgflineto{\pgfpoint{279.680023pt}{69.222656pt}}
\pgflineto{\pgfpoint{279.680023pt}{54.383209pt}}
\pgfpathclose
\pgfusepath{fill,stroke}
\pgfpathmoveto{\pgfpoint{280.898163pt}{61.260727pt}}
\pgflineto{\pgfpoint{284.640015pt}{36.803474pt}}
\pgflineto{\pgfpoint{284.640015pt}{82.386887pt}}
\pgfpathclose
\pgfusepath{fill,stroke}
\pgfpathmoveto{\pgfpoint{288.703094pt}{64.960159pt}}
\pgflineto{\pgfpoint{284.640015pt}{82.386887pt}}
\pgflineto{\pgfpoint{284.640015pt}{36.803474pt}}
\pgfpathclose
\pgfusepath{fill,stroke}
\pgfpathmoveto{\pgfpoint{289.600037pt}{71.175858pt}}
\pgflineto{\pgfpoint{288.703094pt}{64.960159pt}}
\pgflineto{\pgfpoint{289.600037pt}{61.113129pt}}
\pgfpathclose
\pgfusepath{fill,stroke}
\color[rgb]{0.000000,0.000000,0.000000}
\pgfsetlinewidth{0.500000pt}
\pgfsetdash{{16pt}{0pt}}{0pt}
\pgfpathmoveto{\pgfpoint{289.600037pt}{26.399979pt}}
\pgflineto{\pgfpoint{41.600006pt}{26.399979pt}}
\pgfusepath{stroke}
\pgfpathmoveto{\pgfpoint{289.600037pt}{222.000000pt}}
\pgflineto{\pgfpoint{41.600006pt}{222.000000pt}}
\pgfusepath{stroke}
\pgfpathmoveto{\pgfpoint{41.600006pt}{222.000000pt}}
\pgflineto{\pgfpoint{41.600006pt}{26.399979pt}}
\pgfusepath{stroke}
\pgfpathmoveto{\pgfpoint{289.600037pt}{222.000000pt}}
\pgflineto{\pgfpoint{289.600037pt}{26.399979pt}}
\pgfusepath{stroke}
\pgfpathmoveto{\pgfpoint{41.600006pt}{28.874924pt}}
\pgflineto{\pgfpoint{41.600006pt}{26.399979pt}}
\pgfusepath{stroke}
\pgfpathmoveto{\pgfpoint{41.600006pt}{219.525085pt}}
\pgflineto{\pgfpoint{41.600006pt}{222.000000pt}}
\pgfusepath{stroke}
\pgfpathmoveto{\pgfpoint{91.200012pt}{28.874924pt}}
\pgflineto{\pgfpoint{91.200012pt}{26.399979pt}}
\pgfusepath{stroke}
\pgfpathmoveto{\pgfpoint{91.200012pt}{219.525085pt}}
\pgflineto{\pgfpoint{91.200012pt}{222.000000pt}}
\pgfusepath{stroke}
\pgfpathmoveto{\pgfpoint{140.800003pt}{28.874924pt}}
\pgflineto{\pgfpoint{140.800003pt}{26.399979pt}}
\pgfusepath{stroke}
\pgfpathmoveto{\pgfpoint{140.800003pt}{219.525085pt}}
\pgflineto{\pgfpoint{140.800003pt}{222.000000pt}}
\pgfusepath{stroke}
\pgfpathmoveto{\pgfpoint{190.400024pt}{28.874924pt}}
\pgflineto{\pgfpoint{190.400024pt}{26.399979pt}}
\pgfusepath{stroke}
\pgfpathmoveto{\pgfpoint{190.400024pt}{219.525085pt}}
\pgflineto{\pgfpoint{190.400024pt}{222.000000pt}}
\pgfusepath{stroke}
\pgfpathmoveto{\pgfpoint{240.000000pt}{28.874924pt}}
\pgflineto{\pgfpoint{240.000000pt}{26.399979pt}}
\pgfusepath{stroke}
\pgfpathmoveto{\pgfpoint{240.000000pt}{219.525085pt}}
\pgflineto{\pgfpoint{240.000000pt}{222.000000pt}}
\pgfusepath{stroke}
\pgfpathmoveto{\pgfpoint{289.600037pt}{28.874924pt}}
\pgflineto{\pgfpoint{289.600037pt}{26.399979pt}}
\pgfusepath{stroke}
\pgfpathmoveto{\pgfpoint{289.600037pt}{219.525085pt}}
\pgflineto{\pgfpoint{289.600037pt}{222.000000pt}}
\pgfusepath{stroke}
{
\pgftransformshift{\pgfpoint{41.600006pt}{21.410187pt}}
\pgfnode{rectangle}{north}{\fontsize{10}{0}\selectfont\textcolor[rgb]{0,0,0}{{0}}}{}{\pgfusepath{discard}}}
{
\pgftransformshift{\pgfpoint{91.200012pt}{21.410187pt}}
\pgfnode{rectangle}{north}{\fontsize{10}{0}\selectfont\textcolor[rgb]{0,0,0}{{10}}}{}{\pgfusepath{discard}}}
{
\pgftransformshift{\pgfpoint{140.800003pt}{21.410187pt}}
\pgfnode{rectangle}{north}{\fontsize{10}{0}\selectfont\textcolor[rgb]{0,0,0}{{20}}}{}{\pgfusepath{discard}}}
{
\pgftransformshift{\pgfpoint{190.400009pt}{21.410187pt}}
\pgfnode{rectangle}{north}{\fontsize{10}{0}\selectfont\textcolor[rgb]{0,0,0}{{30}}}{}{\pgfusepath{discard}}}
{
\pgftransformshift{\pgfpoint{240.000000pt}{21.410187pt}}
\pgfnode{rectangle}{north}{\fontsize{10}{0}\selectfont\textcolor[rgb]{0,0,0}{{40}}}{}{\pgfusepath{discard}}}
{
\pgftransformshift{\pgfpoint{289.600037pt}{21.410187pt}}
\pgfnode{rectangle}{north}{\fontsize{10}{0}\selectfont\textcolor[rgb]{0,0,0}{{50}}}{}{\pgfusepath{discard}}}
\pgfpathmoveto{\pgfpoint{44.080009pt}{26.399979pt}}
\pgflineto{\pgfpoint{41.600006pt}{26.399979pt}}
\pgfusepath{stroke}
\pgfpathmoveto{\pgfpoint{287.120026pt}{26.399979pt}}
\pgflineto{\pgfpoint{289.600037pt}{26.399979pt}}
\pgfusepath{stroke}
\pgfpathmoveto{\pgfpoint{44.080009pt}{58.999985pt}}
\pgflineto{\pgfpoint{41.600006pt}{58.999985pt}}
\pgfusepath{stroke}
\pgfpathmoveto{\pgfpoint{287.120026pt}{58.999985pt}}
\pgflineto{\pgfpoint{289.600037pt}{58.999985pt}}
\pgfusepath{stroke}
\pgfpathmoveto{\pgfpoint{44.080009pt}{91.599991pt}}
\pgflineto{\pgfpoint{41.600006pt}{91.599991pt}}
\pgfusepath{stroke}
\pgfpathmoveto{\pgfpoint{287.120026pt}{91.599991pt}}
\pgflineto{\pgfpoint{289.600037pt}{91.599991pt}}
\pgfusepath{stroke}
\pgfpathmoveto{\pgfpoint{44.080009pt}{124.199997pt}}
\pgflineto{\pgfpoint{41.600006pt}{124.199997pt}}
\pgfusepath{stroke}
\pgfpathmoveto{\pgfpoint{287.120026pt}{124.199997pt}}
\pgflineto{\pgfpoint{289.600037pt}{124.199997pt}}
\pgfusepath{stroke}
\pgfpathmoveto{\pgfpoint{44.080009pt}{156.800003pt}}
\pgflineto{\pgfpoint{41.600006pt}{156.800003pt}}
\pgfusepath{stroke}
\pgfpathmoveto{\pgfpoint{287.120026pt}{156.800003pt}}
\pgflineto{\pgfpoint{289.600037pt}{156.800003pt}}
\pgfusepath{stroke}
\pgfpathmoveto{\pgfpoint{44.080009pt}{189.400009pt}}
\pgflineto{\pgfpoint{41.600006pt}{189.400009pt}}
\pgfusepath{stroke}
\pgfpathmoveto{\pgfpoint{287.120026pt}{189.400009pt}}
\pgflineto{\pgfpoint{289.600037pt}{189.400009pt}}
\pgfusepath{stroke}
\pgfpathmoveto{\pgfpoint{44.080009pt}{222.000000pt}}
\pgflineto{\pgfpoint{41.600006pt}{222.000000pt}}
\pgfusepath{stroke}
\pgfpathmoveto{\pgfpoint{287.120026pt}{222.000000pt}}
\pgflineto{\pgfpoint{289.600037pt}{222.000000pt}}
\pgfusepath{stroke}
{
\pgftransformshift{\pgfpoint{36.600006pt}{26.399979pt}}
\pgfnode{rectangle}{east}{\fontsize{10}{0}\selectfont\textcolor[rgb]{0,0,0}{{0}}}{}{\pgfusepath{discard}}}
{
\pgftransformshift{\pgfpoint{36.600006pt}{58.999985pt}}
\pgfnode{rectangle}{east}{\fontsize{10}{0}\selectfont\textcolor[rgb]{0,0,0}{{1}}}{}{\pgfusepath{discard}}}
{
\pgftransformshift{\pgfpoint{36.600006pt}{91.599991pt}}
\pgfnode{rectangle}{east}{\fontsize{10}{0}\selectfont\textcolor[rgb]{0,0,0}{{2}}}{}{\pgfusepath{discard}}}
{
\pgftransformshift{\pgfpoint{36.600006pt}{124.199989pt}}
\pgfnode{rectangle}{east}{\fontsize{10}{0}\selectfont\textcolor[rgb]{0,0,0}{{3}}}{}{\pgfusepath{discard}}}
{
\pgftransformshift{\pgfpoint{36.600006pt}{156.799988pt}}
\pgfnode{rectangle}{east}{\fontsize{10}{0}\selectfont\textcolor[rgb]{0,0,0}{{4}}}{}{\pgfusepath{discard}}}
{
\pgftransformshift{\pgfpoint{36.600006pt}{189.399994pt}}
\pgfnode{rectangle}{east}{\fontsize{10}{0}\selectfont\textcolor[rgb]{0,0,0}{{5}}}{}{\pgfusepath{discard}}}
{
\pgftransformshift{\pgfpoint{36.600006pt}{222.000000pt}}
\pgfnode{rectangle}{east}{\fontsize{10}{0}\selectfont\textcolor[rgb]{0,0,0}{{6}}}{}{\pgfusepath{discard}}}
\color[rgb]{0.000000,0.000000,1.000000}
\pgfsetdash{}{0pt}
\pgfpathmoveto{\pgfpoint{51.520004pt}{60.187065pt}}
\pgflineto{\pgfpoint{46.560013pt}{65.485382pt}}
\pgfusepath{stroke}
\pgfpathmoveto{\pgfpoint{56.480011pt}{42.665169pt}}
\pgflineto{\pgfpoint{51.520004pt}{60.187065pt}}
\pgfusepath{stroke}
\pgfpathmoveto{\pgfpoint{61.440010pt}{46.590111pt}}
\pgflineto{\pgfpoint{56.480011pt}{42.665169pt}}
\pgfusepath{stroke}
\pgfpathmoveto{\pgfpoint{66.400009pt}{63.127323pt}}
\pgflineto{\pgfpoint{61.440010pt}{46.590111pt}}
\pgfusepath{stroke}
\pgfpathmoveto{\pgfpoint{71.360008pt}{106.220398pt}}
\pgflineto{\pgfpoint{66.400009pt}{63.127323pt}}
\pgfusepath{stroke}
\pgfpathmoveto{\pgfpoint{76.320007pt}{80.649338pt}}
\pgflineto{\pgfpoint{71.360008pt}{106.220398pt}}
\pgfusepath{stroke}
\pgfpathmoveto{\pgfpoint{81.280014pt}{61.721870pt}}
\pgflineto{\pgfpoint{76.320007pt}{80.649338pt}}
\pgfusepath{stroke}
\pgfpathmoveto{\pgfpoint{86.240013pt}{60.315250pt}}
\pgflineto{\pgfpoint{81.280014pt}{61.721870pt}}
\pgfusepath{stroke}
\pgfpathmoveto{\pgfpoint{91.200012pt}{67.963776pt}}
\pgflineto{\pgfpoint{86.240013pt}{60.315250pt}}
\pgfusepath{stroke}
\pgfpathmoveto{\pgfpoint{96.160011pt}{39.422333pt}}
\pgflineto{\pgfpoint{91.200012pt}{67.963776pt}}
\pgfusepath{stroke}
\pgfpathmoveto{\pgfpoint{101.120010pt}{36.805199pt}}
\pgflineto{\pgfpoint{96.160011pt}{39.422333pt}}
\pgfusepath{stroke}
\pgfpathmoveto{\pgfpoint{106.080017pt}{57.417770pt}}
\pgflineto{\pgfpoint{101.120010pt}{36.805199pt}}
\pgfusepath{stroke}
\pgfpathmoveto{\pgfpoint{111.040009pt}{48.803902pt}}
\pgflineto{\pgfpoint{106.080017pt}{57.417770pt}}
\pgfusepath{stroke}
\pgfpathmoveto{\pgfpoint{116.000015pt}{40.934975pt}}
\pgflineto{\pgfpoint{111.040009pt}{48.803902pt}}
\pgfusepath{stroke}
\pgfpathmoveto{\pgfpoint{120.960007pt}{115.830338pt}}
\pgflineto{\pgfpoint{116.000015pt}{40.934975pt}}
\pgfusepath{stroke}
\pgfpathmoveto{\pgfpoint{125.920013pt}{66.937149pt}}
\pgflineto{\pgfpoint{120.960007pt}{115.830338pt}}
\pgfusepath{stroke}
\pgfpathmoveto{\pgfpoint{130.880005pt}{59.440186pt}}
\pgflineto{\pgfpoint{125.920013pt}{66.937149pt}}
\pgfusepath{stroke}
\pgfpathmoveto{\pgfpoint{135.840012pt}{37.031158pt}}
\pgflineto{\pgfpoint{130.880005pt}{59.440186pt}}
\pgfusepath{stroke}
\pgfpathmoveto{\pgfpoint{140.800003pt}{97.846222pt}}
\pgflineto{\pgfpoint{135.840012pt}{37.031158pt}}
\pgfusepath{stroke}
\pgfpathmoveto{\pgfpoint{145.760010pt}{49.291016pt}}
\pgflineto{\pgfpoint{140.800003pt}{97.846222pt}}
\pgfusepath{stroke}
\pgfpathmoveto{\pgfpoint{150.720016pt}{56.516876pt}}
\pgflineto{\pgfpoint{145.760010pt}{49.291016pt}}
\pgfusepath{stroke}
\pgfpathmoveto{\pgfpoint{155.680023pt}{39.056114pt}}
\pgflineto{\pgfpoint{150.720016pt}{56.516876pt}}
\pgfusepath{stroke}
\pgfpathmoveto{\pgfpoint{160.640015pt}{115.879601pt}}
\pgflineto{\pgfpoint{155.680023pt}{39.056114pt}}
\pgfusepath{stroke}
\pgfpathmoveto{\pgfpoint{165.600006pt}{60.773132pt}}
\pgflineto{\pgfpoint{160.640015pt}{115.879601pt}}
\pgfusepath{stroke}
\pgfpathmoveto{\pgfpoint{170.560013pt}{94.831261pt}}
\pgflineto{\pgfpoint{165.600006pt}{60.773132pt}}
\pgfusepath{stroke}
\pgfpathmoveto{\pgfpoint{175.520004pt}{43.668625pt}}
\pgflineto{\pgfpoint{170.560013pt}{94.831261pt}}
\pgfusepath{stroke}
\pgfpathmoveto{\pgfpoint{180.480011pt}{40.850761pt}}
\pgflineto{\pgfpoint{175.520004pt}{43.668625pt}}
\pgfusepath{stroke}
\pgfpathmoveto{\pgfpoint{185.440018pt}{94.016953pt}}
\pgflineto{\pgfpoint{180.480011pt}{40.850761pt}}
\pgfusepath{stroke}
\pgfpathmoveto{\pgfpoint{190.400024pt}{74.316742pt}}
\pgflineto{\pgfpoint{185.440018pt}{94.016953pt}}
\pgfusepath{stroke}
\pgfpathmoveto{\pgfpoint{195.360016pt}{39.186867pt}}
\pgflineto{\pgfpoint{190.400024pt}{74.316742pt}}
\pgfusepath{stroke}
\pgfpathmoveto{\pgfpoint{200.320007pt}{111.419617pt}}
\pgflineto{\pgfpoint{195.360016pt}{39.186867pt}}
\pgfusepath{stroke}
\pgfpathmoveto{\pgfpoint{205.279999pt}{77.342728pt}}
\pgflineto{\pgfpoint{200.320007pt}{111.419617pt}}
\pgfusepath{stroke}
\pgfpathmoveto{\pgfpoint{210.240021pt}{64.683121pt}}
\pgflineto{\pgfpoint{205.279999pt}{77.342728pt}}
\pgfusepath{stroke}
\pgfpathmoveto{\pgfpoint{215.200012pt}{68.944382pt}}
\pgflineto{\pgfpoint{210.240021pt}{64.683121pt}}
\pgfusepath{stroke}
\pgfpathmoveto{\pgfpoint{220.160019pt}{56.577995pt}}
\pgflineto{\pgfpoint{215.200012pt}{68.944382pt}}
\pgfusepath{stroke}
\pgfpathmoveto{\pgfpoint{225.120026pt}{63.517673pt}}
\pgflineto{\pgfpoint{220.160019pt}{56.577995pt}}
\pgfusepath{stroke}
\pgfpathmoveto{\pgfpoint{230.080017pt}{66.862328pt}}
\pgflineto{\pgfpoint{225.120026pt}{63.517673pt}}
\pgfusepath{stroke}
\pgfpathmoveto{\pgfpoint{235.040024pt}{53.250259pt}}
\pgflineto{\pgfpoint{230.080017pt}{66.862328pt}}
\pgfusepath{stroke}
\pgfpathmoveto{\pgfpoint{240.000000pt}{75.645462pt}}
\pgflineto{\pgfpoint{235.040024pt}{53.250259pt}}
\pgfusepath{stroke}
\pgfpathmoveto{\pgfpoint{244.960022pt}{39.255608pt}}
\pgflineto{\pgfpoint{240.000000pt}{75.645462pt}}
\pgfusepath{stroke}
\pgfpathmoveto{\pgfpoint{249.920013pt}{106.094025pt}}
\pgflineto{\pgfpoint{244.960022pt}{39.255608pt}}
\pgfusepath{stroke}
\pgfpathmoveto{\pgfpoint{254.880020pt}{70.739304pt}}
\pgflineto{\pgfpoint{249.920013pt}{106.094025pt}}
\pgfusepath{stroke}
\pgfpathmoveto{\pgfpoint{259.840027pt}{73.925293pt}}
\pgflineto{\pgfpoint{254.880020pt}{70.739304pt}}
\pgfusepath{stroke}
\pgfpathmoveto{\pgfpoint{264.800018pt}{34.848969pt}}
\pgflineto{\pgfpoint{259.840027pt}{73.925293pt}}
\pgfusepath{stroke}
\pgfpathmoveto{\pgfpoint{269.760010pt}{71.767647pt}}
\pgflineto{\pgfpoint{264.800018pt}{34.848969pt}}
\pgfusepath{stroke}
\pgfpathmoveto{\pgfpoint{274.720001pt}{76.811203pt}}
\pgflineto{\pgfpoint{269.760010pt}{71.767647pt}}
\pgfusepath{stroke}
\pgfpathmoveto{\pgfpoint{279.680023pt}{69.222656pt}}
\pgflineto{\pgfpoint{274.720001pt}{76.811203pt}}
\pgfusepath{stroke}
\pgfpathmoveto{\pgfpoint{284.640015pt}{36.803474pt}}
\pgflineto{\pgfpoint{279.680023pt}{69.222656pt}}
\pgfusepath{stroke}
\pgfpathmoveto{\pgfpoint{289.600037pt}{71.175858pt}}
\pgflineto{\pgfpoint{284.640015pt}{36.803474pt}}
\pgfusepath{stroke}
\color[rgb]{0.000000,0.500000,0.000000}
\pgfpathmoveto{\pgfpoint{51.520004pt}{54.621078pt}}
\pgflineto{\pgfpoint{46.560013pt}{171.656006pt}}
\pgfusepath{stroke}
\pgfpathmoveto{\pgfpoint{56.480011pt}{59.932564pt}}
\pgflineto{\pgfpoint{51.520004pt}{54.621078pt}}
\pgfusepath{stroke}
\pgfpathmoveto{\pgfpoint{61.440010pt}{105.284721pt}}
\pgflineto{\pgfpoint{56.480011pt}{59.932564pt}}
\pgfusepath{stroke}
\pgfpathmoveto{\pgfpoint{66.400009pt}{163.107224pt}}
\pgflineto{\pgfpoint{61.440010pt}{105.284721pt}}
\pgfusepath{stroke}
\pgfpathmoveto{\pgfpoint{71.360008pt}{43.829178pt}}
\pgflineto{\pgfpoint{66.400009pt}{163.107224pt}}
\pgfusepath{stroke}
\pgfpathmoveto{\pgfpoint{76.320007pt}{140.828415pt}}
\pgflineto{\pgfpoint{71.360008pt}{43.829178pt}}
\pgfusepath{stroke}
\pgfpathmoveto{\pgfpoint{81.280014pt}{44.592613pt}}
\pgflineto{\pgfpoint{76.320007pt}{140.828415pt}}
\pgfusepath{stroke}
\pgfpathmoveto{\pgfpoint{86.240013pt}{61.582748pt}}
\pgflineto{\pgfpoint{81.280014pt}{44.592613pt}}
\pgfusepath{stroke}
\pgfpathmoveto{\pgfpoint{91.200012pt}{82.405769pt}}
\pgflineto{\pgfpoint{86.240013pt}{61.582748pt}}
\pgfusepath{stroke}
\pgfpathmoveto{\pgfpoint{96.160011pt}{121.447739pt}}
\pgflineto{\pgfpoint{91.200012pt}{82.405769pt}}
\pgfusepath{stroke}
\pgfpathmoveto{\pgfpoint{101.120010pt}{45.170807pt}}
\pgflineto{\pgfpoint{96.160011pt}{121.447739pt}}
\pgfusepath{stroke}
\pgfpathmoveto{\pgfpoint{106.080017pt}{93.045525pt}}
\pgflineto{\pgfpoint{101.120010pt}{45.170807pt}}
\pgfusepath{stroke}
\pgfpathmoveto{\pgfpoint{111.040009pt}{89.065750pt}}
\pgflineto{\pgfpoint{106.080017pt}{93.045525pt}}
\pgfusepath{stroke}
\pgfpathmoveto{\pgfpoint{116.000015pt}{43.683624pt}}
\pgflineto{\pgfpoint{111.040009pt}{89.065750pt}}
\pgfusepath{stroke}
\pgfpathmoveto{\pgfpoint{120.960007pt}{149.515289pt}}
\pgflineto{\pgfpoint{116.000015pt}{43.683624pt}}
\pgfusepath{stroke}
\pgfpathmoveto{\pgfpoint{125.920013pt}{72.071335pt}}
\pgflineto{\pgfpoint{120.960007pt}{149.515289pt}}
\pgfusepath{stroke}
\pgfpathmoveto{\pgfpoint{130.880005pt}{46.801132pt}}
\pgflineto{\pgfpoint{125.920013pt}{72.071335pt}}
\pgfusepath{stroke}
\pgfpathmoveto{\pgfpoint{135.840012pt}{69.273598pt}}
\pgflineto{\pgfpoint{130.880005pt}{46.801132pt}}
\pgfusepath{stroke}
\pgfpathmoveto{\pgfpoint{140.800003pt}{63.190159pt}}
\pgflineto{\pgfpoint{135.840012pt}{69.273598pt}}
\pgfusepath{stroke}
\pgfpathmoveto{\pgfpoint{145.760010pt}{45.087975pt}}
\pgflineto{\pgfpoint{140.800003pt}{63.190159pt}}
\pgfusepath{stroke}
\pgfpathmoveto{\pgfpoint{150.720016pt}{59.958729pt}}
\pgflineto{\pgfpoint{145.760010pt}{45.087975pt}}
\pgfusepath{stroke}
\pgfpathmoveto{\pgfpoint{155.680023pt}{145.320908pt}}
\pgflineto{\pgfpoint{150.720016pt}{59.958729pt}}
\pgfusepath{stroke}
\pgfpathmoveto{\pgfpoint{160.640015pt}{55.971481pt}}
\pgflineto{\pgfpoint{155.680023pt}{145.320908pt}}
\pgfusepath{stroke}
\pgfpathmoveto{\pgfpoint{165.600006pt}{64.977737pt}}
\pgflineto{\pgfpoint{160.640015pt}{55.971481pt}}
\pgfusepath{stroke}
\pgfpathmoveto{\pgfpoint{170.560013pt}{48.365959pt}}
\pgflineto{\pgfpoint{165.600006pt}{64.977737pt}}
\pgfusepath{stroke}
\pgfpathmoveto{\pgfpoint{175.520004pt}{60.118607pt}}
\pgflineto{\pgfpoint{170.560013pt}{48.365959pt}}
\pgfusepath{stroke}
\pgfpathmoveto{\pgfpoint{180.480011pt}{75.116745pt}}
\pgflineto{\pgfpoint{175.520004pt}{60.118607pt}}
\pgfusepath{stroke}
\pgfpathmoveto{\pgfpoint{185.440018pt}{55.456352pt}}
\pgflineto{\pgfpoint{180.480011pt}{75.116745pt}}
\pgfusepath{stroke}
\pgfpathmoveto{\pgfpoint{190.400024pt}{40.646240pt}}
\pgflineto{\pgfpoint{185.440018pt}{55.456352pt}}
\pgfusepath{stroke}
\pgfpathmoveto{\pgfpoint{195.360016pt}{50.795929pt}}
\pgflineto{\pgfpoint{190.400024pt}{40.646240pt}}
\pgfusepath{stroke}
\pgfpathmoveto{\pgfpoint{200.320007pt}{162.896408pt}}
\pgflineto{\pgfpoint{195.360016pt}{50.795929pt}}
\pgfusepath{stroke}
\pgfpathmoveto{\pgfpoint{205.279999pt}{99.158791pt}}
\pgflineto{\pgfpoint{200.320007pt}{162.896408pt}}
\pgfusepath{stroke}
\pgfpathmoveto{\pgfpoint{210.240021pt}{64.267227pt}}
\pgflineto{\pgfpoint{205.279999pt}{99.158791pt}}
\pgfusepath{stroke}
\pgfpathmoveto{\pgfpoint{215.200012pt}{36.115891pt}}
\pgflineto{\pgfpoint{210.240021pt}{64.267227pt}}
\pgfusepath{stroke}
\pgfpathmoveto{\pgfpoint{220.160019pt}{48.298035pt}}
\pgflineto{\pgfpoint{215.200012pt}{36.115891pt}}
\pgfusepath{stroke}
\pgfpathmoveto{\pgfpoint{225.120026pt}{51.340904pt}}
\pgflineto{\pgfpoint{220.160019pt}{48.298035pt}}
\pgfusepath{stroke}
\pgfpathmoveto{\pgfpoint{230.080017pt}{71.875427pt}}
\pgflineto{\pgfpoint{225.120026pt}{51.340904pt}}
\pgfusepath{stroke}
\pgfpathmoveto{\pgfpoint{235.040024pt}{43.997490pt}}
\pgflineto{\pgfpoint{230.080017pt}{71.875427pt}}
\pgfusepath{stroke}
\pgfpathmoveto{\pgfpoint{240.000000pt}{61.037575pt}}
\pgflineto{\pgfpoint{235.040024pt}{43.997490pt}}
\pgfusepath{stroke}
\pgfpathmoveto{\pgfpoint{244.960022pt}{40.726631pt}}
\pgflineto{\pgfpoint{240.000000pt}{61.037575pt}}
\pgfusepath{stroke}
\pgfpathmoveto{\pgfpoint{249.920013pt}{116.421631pt}}
\pgflineto{\pgfpoint{244.960022pt}{40.726631pt}}
\pgfusepath{stroke}
\pgfpathmoveto{\pgfpoint{254.880020pt}{215.171143pt}}
\pgflineto{\pgfpoint{249.920013pt}{116.421631pt}}
\pgfusepath{stroke}
\pgfpathmoveto{\pgfpoint{259.840027pt}{141.531891pt}}
\pgflineto{\pgfpoint{254.880020pt}{215.171143pt}}
\pgfusepath{stroke}
\pgfpathmoveto{\pgfpoint{264.800018pt}{190.998230pt}}
\pgflineto{\pgfpoint{259.840027pt}{141.531891pt}}
\pgfusepath{stroke}
\pgfpathmoveto{\pgfpoint{269.760010pt}{62.679287pt}}
\pgflineto{\pgfpoint{264.800018pt}{190.998230pt}}
\pgfusepath{stroke}
\pgfpathmoveto{\pgfpoint{274.720001pt}{37.509773pt}}
\pgflineto{\pgfpoint{269.760010pt}{62.679287pt}}
\pgfusepath{stroke}
\pgfpathmoveto{\pgfpoint{279.680023pt}{54.383209pt}}
\pgflineto{\pgfpoint{274.720001pt}{37.509773pt}}
\pgfusepath{stroke}
\pgfpathmoveto{\pgfpoint{284.640015pt}{82.386887pt}}
\pgflineto{\pgfpoint{279.680023pt}{54.383209pt}}
\pgfusepath{stroke}
\pgfpathmoveto{\pgfpoint{289.600037pt}{61.113129pt}}
\pgflineto{\pgfpoint{284.640015pt}{82.386887pt}}
\pgfusepath{stroke}
\color[rgb]{0.000000,0.000000,0.000000}
\pgfpathmoveto{\pgfpoint{288.703094pt}{64.960159pt}}
\pgflineto{\pgfpoint{289.600037pt}{61.113129pt}}
\pgfusepath{stroke}
\pgfpathmoveto{\pgfpoint{289.600037pt}{71.175858pt}}
\pgflineto{\pgfpoint{288.703094pt}{64.960159pt}}
\pgfusepath{stroke}
\pgfpathmoveto{\pgfpoint{289.600037pt}{61.113129pt}}
\pgflineto{\pgfpoint{289.600037pt}{71.175858pt}}
\pgfusepath{stroke}
\pgfpathmoveto{\pgfpoint{284.640015pt}{36.803474pt}}
\pgflineto{\pgfpoint{288.703094pt}{64.960159pt}}
\pgfusepath{stroke}
\pgfpathmoveto{\pgfpoint{280.898163pt}{61.260727pt}}
\pgflineto{\pgfpoint{284.640015pt}{36.803474pt}}
\pgfusepath{stroke}
\pgfpathmoveto{\pgfpoint{284.640015pt}{82.386887pt}}
\pgflineto{\pgfpoint{280.898163pt}{61.260727pt}}
\pgfusepath{stroke}
\pgfpathmoveto{\pgfpoint{288.703094pt}{64.960159pt}}
\pgflineto{\pgfpoint{284.640015pt}{82.386887pt}}
\pgfusepath{stroke}
\pgfpathmoveto{\pgfpoint{279.680023pt}{54.383209pt}}
\pgflineto{\pgfpoint{280.898163pt}{61.260727pt}}
\pgfusepath{stroke}
\pgfpathmoveto{\pgfpoint{274.720001pt}{37.509773pt}}
\pgflineto{\pgfpoint{279.680023pt}{54.383209pt}}
\pgfusepath{stroke}
\pgfpathmoveto{\pgfpoint{269.760010pt}{62.679287pt}}
\pgflineto{\pgfpoint{274.720001pt}{37.509773pt}}
\pgfusepath{stroke}
\pgfpathmoveto{\pgfpoint{269.487213pt}{69.737061pt}}
\pgflineto{\pgfpoint{269.760010pt}{62.679287pt}}
\pgfusepath{stroke}
\pgfpathmoveto{\pgfpoint{269.760010pt}{71.767647pt}}
\pgflineto{\pgfpoint{269.487213pt}{69.737061pt}}
\pgfusepath{stroke}
\pgfpathmoveto{\pgfpoint{274.720001pt}{76.811203pt}}
\pgflineto{\pgfpoint{269.760010pt}{71.767647pt}}
\pgfusepath{stroke}
\pgfpathmoveto{\pgfpoint{279.680023pt}{69.222656pt}}
\pgflineto{\pgfpoint{274.720001pt}{76.811203pt}}
\pgfusepath{stroke}
\pgfpathmoveto{\pgfpoint{280.898163pt}{61.260727pt}}
\pgflineto{\pgfpoint{279.680023pt}{69.222656pt}}
\pgfusepath{stroke}
\pgfpathmoveto{\pgfpoint{264.800018pt}{34.848969pt}}
\pgflineto{\pgfpoint{269.487213pt}{69.737061pt}}
\pgfusepath{stroke}
\pgfpathmoveto{\pgfpoint{259.840027pt}{73.925293pt}}
\pgflineto{\pgfpoint{264.800018pt}{34.848969pt}}
\pgfusepath{stroke}
\pgfpathmoveto{\pgfpoint{254.880020pt}{70.739304pt}}
\pgflineto{\pgfpoint{259.840027pt}{73.925293pt}}
\pgfusepath{stroke}
\pgfpathmoveto{\pgfpoint{249.920013pt}{106.094025pt}}
\pgflineto{\pgfpoint{254.880020pt}{70.739304pt}}
\pgfusepath{stroke}
\pgfpathmoveto{\pgfpoint{244.960022pt}{39.255608pt}}
\pgflineto{\pgfpoint{249.920013pt}{106.094025pt}}
\pgfusepath{stroke}
\pgfpathmoveto{\pgfpoint{244.506226pt}{42.584846pt}}
\pgflineto{\pgfpoint{244.960022pt}{39.255608pt}}
\pgfusepath{stroke}
\pgfpathmoveto{\pgfpoint{244.960022pt}{40.726631pt}}
\pgflineto{\pgfpoint{244.506226pt}{42.584846pt}}
\pgfusepath{stroke}
\pgfpathmoveto{\pgfpoint{249.920013pt}{116.421631pt}}
\pgflineto{\pgfpoint{244.960022pt}{40.726631pt}}
\pgfusepath{stroke}
\pgfpathmoveto{\pgfpoint{254.880020pt}{215.171143pt}}
\pgflineto{\pgfpoint{249.920013pt}{116.421631pt}}
\pgfusepath{stroke}
\pgfpathmoveto{\pgfpoint{259.840027pt}{141.531891pt}}
\pgflineto{\pgfpoint{254.880020pt}{215.171143pt}}
\pgfusepath{stroke}
\pgfpathmoveto{\pgfpoint{264.800018pt}{190.998230pt}}
\pgflineto{\pgfpoint{259.840027pt}{141.531891pt}}
\pgfusepath{stroke}
\pgfpathmoveto{\pgfpoint{269.487213pt}{69.737061pt}}
\pgflineto{\pgfpoint{264.800018pt}{190.998230pt}}
\pgfusepath{stroke}
\pgfpathmoveto{\pgfpoint{240.000000pt}{61.037575pt}}
\pgflineto{\pgfpoint{244.506226pt}{42.584846pt}}
\pgfusepath{stroke}
\pgfpathmoveto{\pgfpoint{235.040024pt}{43.997490pt}}
\pgflineto{\pgfpoint{240.000000pt}{61.037575pt}}
\pgfusepath{stroke}
\pgfpathmoveto{\pgfpoint{231.822998pt}{62.078987pt}}
\pgflineto{\pgfpoint{235.040024pt}{43.997490pt}}
\pgfusepath{stroke}
\pgfpathmoveto{\pgfpoint{235.040024pt}{53.250259pt}}
\pgflineto{\pgfpoint{231.822998pt}{62.078987pt}}
\pgfusepath{stroke}
\pgfpathmoveto{\pgfpoint{240.000000pt}{75.645462pt}}
\pgflineto{\pgfpoint{235.040024pt}{53.250259pt}}
\pgfusepath{stroke}
\pgfpathmoveto{\pgfpoint{244.506226pt}{42.584846pt}}
\pgflineto{\pgfpoint{240.000000pt}{75.645462pt}}
\pgfusepath{stroke}
\pgfpathmoveto{\pgfpoint{230.080017pt}{66.862328pt}}
\pgflineto{\pgfpoint{231.822998pt}{62.078987pt}}
\pgfusepath{stroke}
\pgfpathmoveto{\pgfpoint{228.633514pt}{65.886925pt}}
\pgflineto{\pgfpoint{230.080017pt}{66.862328pt}}
\pgfusepath{stroke}
\pgfpathmoveto{\pgfpoint{230.080017pt}{71.875427pt}}
\pgflineto{\pgfpoint{228.633514pt}{65.886925pt}}
\pgfusepath{stroke}
\pgfpathmoveto{\pgfpoint{231.822998pt}{62.078987pt}}
\pgflineto{\pgfpoint{230.080017pt}{71.875427pt}}
\pgfusepath{stroke}
\pgfpathmoveto{\pgfpoint{225.120026pt}{51.340904pt}}
\pgflineto{\pgfpoint{228.633514pt}{65.886925pt}}
\pgfusepath{stroke}
\pgfpathmoveto{\pgfpoint{220.160019pt}{48.298035pt}}
\pgflineto{\pgfpoint{225.120026pt}{51.340904pt}}
\pgfusepath{stroke}
\pgfpathmoveto{\pgfpoint{215.200012pt}{36.115891pt}}
\pgflineto{\pgfpoint{220.160019pt}{48.298035pt}}
\pgfusepath{stroke}
\pgfpathmoveto{\pgfpoint{210.240021pt}{64.267227pt}}
\pgflineto{\pgfpoint{215.200012pt}{36.115891pt}}
\pgfusepath{stroke}
\pgfpathmoveto{\pgfpoint{210.147217pt}{64.919937pt}}
\pgflineto{\pgfpoint{210.240021pt}{64.267227pt}}
\pgfusepath{stroke}
\pgfpathmoveto{\pgfpoint{210.240021pt}{64.683121pt}}
\pgflineto{\pgfpoint{210.147217pt}{64.919937pt}}
\pgfusepath{stroke}
\pgfpathmoveto{\pgfpoint{215.200012pt}{68.944382pt}}
\pgflineto{\pgfpoint{210.240021pt}{64.683121pt}}
\pgfusepath{stroke}
\pgfpathmoveto{\pgfpoint{220.160019pt}{56.577995pt}}
\pgflineto{\pgfpoint{215.200012pt}{68.944382pt}}
\pgfusepath{stroke}
\pgfpathmoveto{\pgfpoint{225.120026pt}{63.517673pt}}
\pgflineto{\pgfpoint{220.160019pt}{56.577995pt}}
\pgfusepath{stroke}
\pgfpathmoveto{\pgfpoint{228.633514pt}{65.886925pt}}
\pgflineto{\pgfpoint{225.120026pt}{63.517673pt}}
\pgfusepath{stroke}
\pgfpathmoveto{\pgfpoint{205.279999pt}{77.342728pt}}
\pgflineto{\pgfpoint{210.147217pt}{64.919937pt}}
\pgfusepath{stroke}
\pgfpathmoveto{\pgfpoint{200.320007pt}{111.419617pt}}
\pgflineto{\pgfpoint{205.279999pt}{77.342728pt}}
\pgfusepath{stroke}
\pgfpathmoveto{\pgfpoint{195.360016pt}{39.186867pt}}
\pgflineto{\pgfpoint{200.320007pt}{111.419617pt}}
\pgfusepath{stroke}
\pgfpathmoveto{\pgfpoint{194.088348pt}{48.193687pt}}
\pgflineto{\pgfpoint{195.360016pt}{39.186867pt}}
\pgfusepath{stroke}
\pgfpathmoveto{\pgfpoint{195.360016pt}{50.795929pt}}
\pgflineto{\pgfpoint{194.088348pt}{48.193687pt}}
\pgfusepath{stroke}
\pgfpathmoveto{\pgfpoint{200.320007pt}{162.896408pt}}
\pgflineto{\pgfpoint{195.360016pt}{50.795929pt}}
\pgfusepath{stroke}
\pgfpathmoveto{\pgfpoint{205.279999pt}{99.158791pt}}
\pgflineto{\pgfpoint{200.320007pt}{162.896408pt}}
\pgfusepath{stroke}
\pgfpathmoveto{\pgfpoint{210.147217pt}{64.919937pt}}
\pgflineto{\pgfpoint{205.279999pt}{99.158791pt}}
\pgfusepath{stroke}
\pgfpathmoveto{\pgfpoint{190.400024pt}{40.646240pt}}
\pgflineto{\pgfpoint{194.088348pt}{48.193687pt}}
\pgfusepath{stroke}
\pgfpathmoveto{\pgfpoint{185.440018pt}{55.456352pt}}
\pgflineto{\pgfpoint{190.400024pt}{40.646240pt}}
\pgfusepath{stroke}
\pgfpathmoveto{\pgfpoint{182.813766pt}{65.866241pt}}
\pgflineto{\pgfpoint{185.440018pt}{55.456352pt}}
\pgfusepath{stroke}
\pgfpathmoveto{\pgfpoint{185.440018pt}{94.016953pt}}
\pgflineto{\pgfpoint{182.813766pt}{65.866241pt}}
\pgfusepath{stroke}
\pgfpathmoveto{\pgfpoint{190.400024pt}{74.316742pt}}
\pgflineto{\pgfpoint{185.440018pt}{94.016953pt}}
\pgfusepath{stroke}
\pgfpathmoveto{\pgfpoint{194.088348pt}{48.193687pt}}
\pgflineto{\pgfpoint{190.400024pt}{74.316742pt}}
\pgfusepath{stroke}
\pgfpathmoveto{\pgfpoint{180.480011pt}{40.850761pt}}
\pgflineto{\pgfpoint{182.813766pt}{65.866241pt}}
\pgfusepath{stroke}
\pgfpathmoveto{\pgfpoint{175.520004pt}{43.668625pt}}
\pgflineto{\pgfpoint{180.480011pt}{40.850761pt}}
\pgfusepath{stroke}
\pgfpathmoveto{\pgfpoint{174.223160pt}{57.045738pt}}
\pgflineto{\pgfpoint{175.520004pt}{43.668625pt}}
\pgfusepath{stroke}
\pgfpathmoveto{\pgfpoint{175.520004pt}{60.118607pt}}
\pgflineto{\pgfpoint{174.223160pt}{57.045738pt}}
\pgfusepath{stroke}
\pgfpathmoveto{\pgfpoint{180.480011pt}{75.116745pt}}
\pgflineto{\pgfpoint{175.520004pt}{60.118607pt}}
\pgfusepath{stroke}
\pgfpathmoveto{\pgfpoint{182.813766pt}{65.866241pt}}
\pgflineto{\pgfpoint{180.480011pt}{75.116745pt}}
\pgfusepath{stroke}
\pgfpathmoveto{\pgfpoint{170.560013pt}{48.365959pt}}
\pgflineto{\pgfpoint{174.223160pt}{57.045738pt}}
\pgfusepath{stroke}
\pgfpathmoveto{\pgfpoint{166.011597pt}{63.599285pt}}
\pgflineto{\pgfpoint{170.560013pt}{48.365959pt}}
\pgfusepath{stroke}
\pgfpathmoveto{\pgfpoint{170.560013pt}{94.831261pt}}
\pgflineto{\pgfpoint{166.011597pt}{63.599285pt}}
\pgfusepath{stroke}
\pgfpathmoveto{\pgfpoint{174.223160pt}{57.045738pt}}
\pgflineto{\pgfpoint{170.560013pt}{94.831261pt}}
\pgfusepath{stroke}
\pgfpathmoveto{\pgfpoint{165.600006pt}{60.773132pt}}
\pgflineto{\pgfpoint{166.011597pt}{63.599285pt}}
\pgfusepath{stroke}
\pgfpathmoveto{\pgfpoint{165.274719pt}{64.387100pt}}
\pgflineto{\pgfpoint{165.600006pt}{60.773132pt}}
\pgfusepath{stroke}
\pgfpathmoveto{\pgfpoint{165.600006pt}{64.977737pt}}
\pgflineto{\pgfpoint{165.274719pt}{64.387100pt}}
\pgfusepath{stroke}
\pgfpathmoveto{\pgfpoint{166.011597pt}{63.599285pt}}
\pgflineto{\pgfpoint{165.600006pt}{64.977737pt}}
\pgfusepath{stroke}
\pgfpathmoveto{\pgfpoint{160.640015pt}{55.971481pt}}
\pgflineto{\pgfpoint{165.274719pt}{64.387100pt}}
\pgfusepath{stroke}
\pgfpathmoveto{\pgfpoint{158.851837pt}{88.183441pt}}
\pgflineto{\pgfpoint{160.640015pt}{55.971481pt}}
\pgfusepath{stroke}
\pgfpathmoveto{\pgfpoint{160.640015pt}{115.879601pt}}
\pgflineto{\pgfpoint{158.851837pt}{88.183441pt}}
\pgfusepath{stroke}
\pgfpathmoveto{\pgfpoint{165.274719pt}{64.387100pt}}
\pgflineto{\pgfpoint{160.640015pt}{115.879601pt}}
\pgfusepath{stroke}
\pgfpathmoveto{\pgfpoint{155.680023pt}{39.056114pt}}
\pgflineto{\pgfpoint{158.851837pt}{88.183441pt}}
\pgfusepath{stroke}
\pgfpathmoveto{\pgfpoint{150.720016pt}{56.516876pt}}
\pgflineto{\pgfpoint{155.680023pt}{39.056114pt}}
\pgfusepath{stroke}
\pgfpathmoveto{\pgfpoint{148.486938pt}{53.263680pt}}
\pgflineto{\pgfpoint{150.720016pt}{56.516876pt}}
\pgfusepath{stroke}
\pgfpathmoveto{\pgfpoint{150.720016pt}{59.958729pt}}
\pgflineto{\pgfpoint{148.486938pt}{53.263680pt}}
\pgfusepath{stroke}
\pgfpathmoveto{\pgfpoint{155.680023pt}{145.320908pt}}
\pgflineto{\pgfpoint{150.720016pt}{59.958729pt}}
\pgfusepath{stroke}
\pgfpathmoveto{\pgfpoint{158.851837pt}{88.183441pt}}
\pgflineto{\pgfpoint{155.680023pt}{145.320908pt}}
\pgfusepath{stroke}
\pgfpathmoveto{\pgfpoint{145.760010pt}{45.087975pt}}
\pgflineto{\pgfpoint{148.486938pt}{53.263680pt}}
\pgfusepath{stroke}
\pgfpathmoveto{\pgfpoint{140.800003pt}{63.190159pt}}
\pgflineto{\pgfpoint{145.760010pt}{45.087975pt}}
\pgfusepath{stroke}
\pgfpathmoveto{\pgfpoint{138.230530pt}{66.341614pt}}
\pgflineto{\pgfpoint{140.800003pt}{63.190159pt}}
\pgfusepath{stroke}
\pgfpathmoveto{\pgfpoint{140.800003pt}{97.846222pt}}
\pgflineto{\pgfpoint{138.230530pt}{66.341614pt}}
\pgfusepath{stroke}
\pgfpathmoveto{\pgfpoint{145.760010pt}{49.291016pt}}
\pgflineto{\pgfpoint{140.800003pt}{97.846222pt}}
\pgfusepath{stroke}
\pgfpathmoveto{\pgfpoint{148.486938pt}{53.263680pt}}
\pgflineto{\pgfpoint{145.760010pt}{49.291016pt}}
\pgfusepath{stroke}
\pgfpathmoveto{\pgfpoint{135.840012pt}{37.031158pt}}
\pgflineto{\pgfpoint{138.230530pt}{66.341614pt}}
\pgfusepath{stroke}
\pgfpathmoveto{\pgfpoint{132.276794pt}{53.129593pt}}
\pgflineto{\pgfpoint{135.840012pt}{37.031158pt}}
\pgfusepath{stroke}
\pgfpathmoveto{\pgfpoint{135.840012pt}{69.273598pt}}
\pgflineto{\pgfpoint{132.276794pt}{53.129593pt}}
\pgfusepath{stroke}
\pgfpathmoveto{\pgfpoint{138.230530pt}{66.341614pt}}
\pgflineto{\pgfpoint{135.840012pt}{69.273598pt}}
\pgfusepath{stroke}
\pgfpathmoveto{\pgfpoint{130.880005pt}{46.801132pt}}
\pgflineto{\pgfpoint{132.276794pt}{53.129593pt}}
\pgfusepath{stroke}
\pgfpathmoveto{\pgfpoint{127.352821pt}{64.771484pt}}
\pgflineto{\pgfpoint{130.880005pt}{46.801132pt}}
\pgfusepath{stroke}
\pgfpathmoveto{\pgfpoint{130.880005pt}{59.440186pt}}
\pgflineto{\pgfpoint{127.352821pt}{64.771484pt}}
\pgfusepath{stroke}
\pgfpathmoveto{\pgfpoint{132.276794pt}{53.129593pt}}
\pgflineto{\pgfpoint{130.880005pt}{59.440186pt}}
\pgfusepath{stroke}
\pgfpathmoveto{\pgfpoint{125.920013pt}{66.937149pt}}
\pgflineto{\pgfpoint{127.352821pt}{64.771484pt}}
\pgfusepath{stroke}
\pgfpathmoveto{\pgfpoint{120.960007pt}{115.830338pt}}
\pgflineto{\pgfpoint{125.920013pt}{66.937149pt}}
\pgfusepath{stroke}
\pgfpathmoveto{\pgfpoint{116.000015pt}{40.934975pt}}
\pgflineto{\pgfpoint{120.960007pt}{115.830338pt}}
\pgfusepath{stroke}
\pgfpathmoveto{\pgfpoint{111.040009pt}{48.803902pt}}
\pgflineto{\pgfpoint{116.000015pt}{40.934975pt}}
\pgfusepath{stroke}
\pgfpathmoveto{\pgfpoint{106.080017pt}{57.417770pt}}
\pgflineto{\pgfpoint{111.040009pt}{48.803902pt}}
\pgfusepath{stroke}
\pgfpathmoveto{\pgfpoint{101.120010pt}{36.805199pt}}
\pgflineto{\pgfpoint{106.080017pt}{57.417770pt}}
\pgfusepath{stroke}
\pgfpathmoveto{\pgfpoint{96.160011pt}{39.422333pt}}
\pgflineto{\pgfpoint{101.120010pt}{36.805199pt}}
\pgfusepath{stroke}
\pgfpathmoveto{\pgfpoint{91.200012pt}{67.963776pt}}
\pgflineto{\pgfpoint{96.160011pt}{39.422333pt}}
\pgfusepath{stroke}
\pgfpathmoveto{\pgfpoint{86.240013pt}{60.315250pt}}
\pgflineto{\pgfpoint{91.200012pt}{67.963776pt}}
\pgfusepath{stroke}
\pgfpathmoveto{\pgfpoint{85.898277pt}{60.412163pt}}
\pgflineto{\pgfpoint{86.240013pt}{60.315250pt}}
\pgfusepath{stroke}
\pgfpathmoveto{\pgfpoint{86.240013pt}{61.582748pt}}
\pgflineto{\pgfpoint{85.898277pt}{60.412163pt}}
\pgfusepath{stroke}
\pgfpathmoveto{\pgfpoint{91.200012pt}{82.405769pt}}
\pgflineto{\pgfpoint{86.240013pt}{61.582748pt}}
\pgfusepath{stroke}
\pgfpathmoveto{\pgfpoint{96.160011pt}{121.447739pt}}
\pgflineto{\pgfpoint{91.200012pt}{82.405769pt}}
\pgfusepath{stroke}
\pgfpathmoveto{\pgfpoint{101.120010pt}{45.170807pt}}
\pgflineto{\pgfpoint{96.160011pt}{121.447739pt}}
\pgfusepath{stroke}
\pgfpathmoveto{\pgfpoint{106.080017pt}{93.045525pt}}
\pgflineto{\pgfpoint{101.120010pt}{45.170807pt}}
\pgfusepath{stroke}
\pgfpathmoveto{\pgfpoint{111.040009pt}{89.065750pt}}
\pgflineto{\pgfpoint{106.080017pt}{93.045525pt}}
\pgfusepath{stroke}
\pgfpathmoveto{\pgfpoint{116.000015pt}{43.683624pt}}
\pgflineto{\pgfpoint{111.040009pt}{89.065750pt}}
\pgfusepath{stroke}
\pgfpathmoveto{\pgfpoint{120.960007pt}{149.515289pt}}
\pgflineto{\pgfpoint{116.000015pt}{43.683624pt}}
\pgfusepath{stroke}
\pgfpathmoveto{\pgfpoint{125.920013pt}{72.071335pt}}
\pgflineto{\pgfpoint{120.960007pt}{149.515289pt}}
\pgfusepath{stroke}
\pgfpathmoveto{\pgfpoint{127.352821pt}{64.771484pt}}
\pgflineto{\pgfpoint{125.920013pt}{72.071335pt}}
\pgfusepath{stroke}
\pgfpathmoveto{\pgfpoint{81.280014pt}{44.592613pt}}
\pgflineto{\pgfpoint{85.898277pt}{60.412163pt}}
\pgfusepath{stroke}
\pgfpathmoveto{\pgfpoint{80.181015pt}{65.915634pt}}
\pgflineto{\pgfpoint{81.280014pt}{44.592613pt}}
\pgfusepath{stroke}
\pgfpathmoveto{\pgfpoint{81.280014pt}{61.721870pt}}
\pgflineto{\pgfpoint{80.181015pt}{65.915634pt}}
\pgfusepath{stroke}
\pgfpathmoveto{\pgfpoint{85.898277pt}{60.412163pt}}
\pgflineto{\pgfpoint{81.280014pt}{61.721870pt}}
\pgfusepath{stroke}
\pgfpathmoveto{\pgfpoint{76.320007pt}{80.649338pt}}
\pgflineto{\pgfpoint{80.181015pt}{65.915634pt}}
\pgfusepath{stroke}
\pgfpathmoveto{\pgfpoint{73.884773pt}{93.204117pt}}
\pgflineto{\pgfpoint{76.320007pt}{80.649338pt}}
\pgfusepath{stroke}
\pgfpathmoveto{\pgfpoint{76.320007pt}{140.828415pt}}
\pgflineto{\pgfpoint{73.884773pt}{93.204117pt}}
\pgfusepath{stroke}
\pgfpathmoveto{\pgfpoint{80.181015pt}{65.915634pt}}
\pgflineto{\pgfpoint{76.320007pt}{140.828415pt}}
\pgfusepath{stroke}
\pgfpathmoveto{\pgfpoint{71.360008pt}{43.829178pt}}
\pgflineto{\pgfpoint{73.884773pt}{93.204117pt}}
\pgfusepath{stroke}
\pgfpathmoveto{\pgfpoint{69.454124pt}{89.661858pt}}
\pgflineto{\pgfpoint{71.360008pt}{43.829178pt}}
\pgfusepath{stroke}
\pgfpathmoveto{\pgfpoint{71.360008pt}{106.220398pt}}
\pgflineto{\pgfpoint{69.454124pt}{89.661858pt}}
\pgfusepath{stroke}
\pgfpathmoveto{\pgfpoint{73.884773pt}{93.204117pt}}
\pgflineto{\pgfpoint{71.360008pt}{106.220398pt}}
\pgfusepath{stroke}
\pgfpathmoveto{\pgfpoint{66.400009pt}{63.127323pt}}
\pgflineto{\pgfpoint{69.454124pt}{89.661858pt}}
\pgfusepath{stroke}
\pgfpathmoveto{\pgfpoint{61.440010pt}{46.590111pt}}
\pgflineto{\pgfpoint{66.400009pt}{63.127323pt}}
\pgfusepath{stroke}
\pgfpathmoveto{\pgfpoint{56.480011pt}{42.665169pt}}
\pgflineto{\pgfpoint{61.440010pt}{46.590111pt}}
\pgfusepath{stroke}
\pgfpathmoveto{\pgfpoint{52.729080pt}{55.915833pt}}
\pgflineto{\pgfpoint{56.480011pt}{42.665169pt}}
\pgfusepath{stroke}
\pgfpathmoveto{\pgfpoint{56.480011pt}{59.932564pt}}
\pgflineto{\pgfpoint{52.729080pt}{55.915833pt}}
\pgfusepath{stroke}
\pgfpathmoveto{\pgfpoint{61.440010pt}{105.284721pt}}
\pgflineto{\pgfpoint{56.480011pt}{59.932564pt}}
\pgfusepath{stroke}
\pgfpathmoveto{\pgfpoint{66.400009pt}{163.107224pt}}
\pgflineto{\pgfpoint{61.440010pt}{105.284721pt}}
\pgfusepath{stroke}
\pgfpathmoveto{\pgfpoint{69.454124pt}{89.661858pt}}
\pgflineto{\pgfpoint{66.400009pt}{163.107224pt}}
\pgfusepath{stroke}
\pgfpathmoveto{\pgfpoint{51.520004pt}{54.621078pt}}
\pgflineto{\pgfpoint{52.729080pt}{55.915833pt}}
\pgfusepath{stroke}
\pgfpathmoveto{\pgfpoint{51.272934pt}{60.450993pt}}
\pgflineto{\pgfpoint{51.520004pt}{54.621078pt}}
\pgfusepath{stroke}
\pgfpathmoveto{\pgfpoint{51.520004pt}{60.187065pt}}
\pgflineto{\pgfpoint{51.272934pt}{60.450993pt}}
\pgfusepath{stroke}
\pgfpathmoveto{\pgfpoint{52.729080pt}{55.915833pt}}
\pgflineto{\pgfpoint{51.520004pt}{60.187065pt}}
\pgfusepath{stroke}
\pgfpathmoveto{\pgfpoint{46.560013pt}{65.485382pt}}
\pgflineto{\pgfpoint{51.272934pt}{60.450993pt}}
\pgfusepath{stroke}
\pgfpathmoveto{\pgfpoint{46.560013pt}{171.656006pt}}
\pgflineto{\pgfpoint{46.560013pt}{65.485382pt}}
\pgfusepath{stroke}
\pgfpathmoveto{\pgfpoint{51.272934pt}{60.450993pt}}
\pgflineto{\pgfpoint{46.560013pt}{171.656006pt}}
\pgfusepath{stroke}
\end{pgfpicture}
}
\end{frame}
\only<article>{
The choice of distance in this kind of algorithm is important,
particularly for very high dimensions. For something like a
spectrogram, one idea is look at the total area of the difference
between two spectral lines. 
}
\begin{frame}
  \frametitle{Nearest neighbour: What type is the new bacterium?}
  % Title: glps_renderer figure
% Creator: GL2PS 1.3.8, (C) 1999-2012 C. Geuzaine
% For: Octave
% CreationDate: Fri Jun 16 12:49:21 2017
\begin{pgfpicture}
\pgfsetlinewidth{0.01pt}
\color[rgb]{1.000000,1.000000,1.000000}
\pgfpathmoveto{\pgfpoint{45.000008pt}{222.000000pt}}
\pgflineto{\pgfpoint{289.600037pt}{26.399979pt}}
\pgflineto{\pgfpoint{45.000008pt}{26.399979pt}}
\pgfpathclose
\pgfusepath{fill,stroke}
\pgfpathmoveto{\pgfpoint{45.000008pt}{222.000000pt}}
\pgflineto{\pgfpoint{289.600037pt}{222.000000pt}}
\pgflineto{\pgfpoint{289.600037pt}{26.399979pt}}
\pgfpathclose
\pgfusepath{fill,stroke}
\pgfpathmoveto{\pgfpoint{260.624542pt}{220.474182pt}}
\pgflineto{\pgfpoint{288.074158pt}{197.039612pt}}
\pgflineto{\pgfpoint{260.624542pt}{197.039612pt}}
\pgfpathclose
\pgfusepath{fill,stroke}
\pgfpathmoveto{\pgfpoint{260.624542pt}{220.474182pt}}
\pgflineto{\pgfpoint{288.074158pt}{220.474182pt}}
\pgflineto{\pgfpoint{288.074158pt}{197.039612pt}}
\pgfpathclose
\pgfusepath{fill,stroke}
\color[rgb]{0.000000,0.000000,0.000000}
\pgfsetlinewidth{0.500000pt}
\pgfsetdash{{16pt}{0pt}}{0pt}
\pgfpathmoveto{\pgfpoint{289.600037pt}{26.399979pt}}
\pgflineto{\pgfpoint{45.000008pt}{26.399979pt}}
\pgfusepath{stroke}
\pgfpathmoveto{\pgfpoint{289.600037pt}{222.000000pt}}
\pgflineto{\pgfpoint{45.000008pt}{222.000000pt}}
\pgfusepath{stroke}
\pgfpathmoveto{\pgfpoint{45.000008pt}{222.000000pt}}
\pgflineto{\pgfpoint{45.000008pt}{26.399979pt}}
\pgfusepath{stroke}
\pgfpathmoveto{\pgfpoint{289.600037pt}{222.000000pt}}
\pgflineto{\pgfpoint{289.600037pt}{26.399979pt}}
\pgfusepath{stroke}
\pgfpathmoveto{\pgfpoint{45.000008pt}{28.840996pt}}
\pgflineto{\pgfpoint{45.000008pt}{26.399979pt}}
\pgfusepath{stroke}
\pgfpathmoveto{\pgfpoint{45.000008pt}{219.558990pt}}
\pgflineto{\pgfpoint{45.000008pt}{222.000000pt}}
\pgfusepath{stroke}
\pgfpathmoveto{\pgfpoint{93.920013pt}{28.840996pt}}
\pgflineto{\pgfpoint{93.920013pt}{26.399979pt}}
\pgfusepath{stroke}
\pgfpathmoveto{\pgfpoint{93.920013pt}{219.558990pt}}
\pgflineto{\pgfpoint{93.920013pt}{222.000000pt}}
\pgfusepath{stroke}
\pgfpathmoveto{\pgfpoint{142.840012pt}{28.840996pt}}
\pgflineto{\pgfpoint{142.840012pt}{26.399979pt}}
\pgfusepath{stroke}
\pgfpathmoveto{\pgfpoint{142.840012pt}{219.558990pt}}
\pgflineto{\pgfpoint{142.840012pt}{222.000000pt}}
\pgfusepath{stroke}
\pgfpathmoveto{\pgfpoint{191.760010pt}{28.840996pt}}
\pgflineto{\pgfpoint{191.760010pt}{26.399979pt}}
\pgfusepath{stroke}
\pgfpathmoveto{\pgfpoint{191.760010pt}{219.558990pt}}
\pgflineto{\pgfpoint{191.760010pt}{222.000000pt}}
\pgfusepath{stroke}
\pgfpathmoveto{\pgfpoint{240.680023pt}{28.840996pt}}
\pgflineto{\pgfpoint{240.680023pt}{26.399979pt}}
\pgfusepath{stroke}
\pgfpathmoveto{\pgfpoint{240.680023pt}{219.558990pt}}
\pgflineto{\pgfpoint{240.680023pt}{222.000000pt}}
\pgfusepath{stroke}
\pgfpathmoveto{\pgfpoint{289.600037pt}{28.840996pt}}
\pgflineto{\pgfpoint{289.600037pt}{26.399979pt}}
\pgfusepath{stroke}
\pgfpathmoveto{\pgfpoint{289.600037pt}{219.558990pt}}
\pgflineto{\pgfpoint{289.600037pt}{222.000000pt}}
\pgfusepath{stroke}
{
\pgftransformshift{\pgfpoint{45.000015pt}{21.410187pt}}
\pgfnode{rectangle}{north}{\fontsize{10}{0}\selectfont\textcolor[rgb]{0,0,0}{{-1.5e+08}}}{}{\pgfusepath{discard}}}
{
\pgftransformshift{\pgfpoint{93.920013pt}{21.410187pt}}
\pgfnode{rectangle}{north}{\fontsize{10}{0}\selectfont\textcolor[rgb]{0,0,0}{{-1e+08}}}{}{\pgfusepath{discard}}}
{
\pgftransformshift{\pgfpoint{142.840012pt}{21.410187pt}}
\pgfnode{rectangle}{north}{\fontsize{10}{0}\selectfont\textcolor[rgb]{0,0,0}{{-5e+07}}}{}{\pgfusepath{discard}}}
{
\pgftransformshift{\pgfpoint{191.760010pt}{21.410187pt}}
\pgfnode{rectangle}{north}{\fontsize{10}{0}\selectfont\textcolor[rgb]{0,0,0}{{0}}}{}{\pgfusepath{discard}}}
{
\pgftransformshift{\pgfpoint{240.680008pt}{21.410187pt}}
\pgfnode{rectangle}{north}{\fontsize{10}{0}\selectfont\textcolor[rgb]{0,0,0}{{5e+07}}}{}{\pgfusepath{discard}}}
{
\pgftransformshift{\pgfpoint{289.600006pt}{21.410187pt}}
\pgfnode{rectangle}{north}{\fontsize{10}{0}\selectfont\textcolor[rgb]{0,0,0}{{1e+08}}}{}{\pgfusepath{discard}}}
\pgfpathmoveto{\pgfpoint{47.442024pt}{26.399979pt}}
\pgflineto{\pgfpoint{45.000008pt}{26.399979pt}}
\pgfusepath{stroke}
\pgfpathmoveto{\pgfpoint{287.158020pt}{26.399979pt}}
\pgflineto{\pgfpoint{289.600037pt}{26.399979pt}}
\pgfusepath{stroke}
\pgfpathmoveto{\pgfpoint{47.442024pt}{75.299988pt}}
\pgflineto{\pgfpoint{45.000008pt}{75.299988pt}}
\pgfusepath{stroke}
\pgfpathmoveto{\pgfpoint{287.158020pt}{75.299988pt}}
\pgflineto{\pgfpoint{289.600037pt}{75.299988pt}}
\pgfusepath{stroke}
\pgfpathmoveto{\pgfpoint{47.442024pt}{124.199989pt}}
\pgflineto{\pgfpoint{45.000008pt}{124.199989pt}}
\pgfusepath{stroke}
\pgfpathmoveto{\pgfpoint{287.158020pt}{124.199989pt}}
\pgflineto{\pgfpoint{289.600037pt}{124.199989pt}}
\pgfusepath{stroke}
\pgfpathmoveto{\pgfpoint{47.442024pt}{173.099991pt}}
\pgflineto{\pgfpoint{45.000008pt}{173.099991pt}}
\pgfusepath{stroke}
\pgfpathmoveto{\pgfpoint{287.158020pt}{173.099991pt}}
\pgflineto{\pgfpoint{289.600037pt}{173.099991pt}}
\pgfusepath{stroke}
\pgfpathmoveto{\pgfpoint{47.442024pt}{222.000000pt}}
\pgflineto{\pgfpoint{45.000008pt}{222.000000pt}}
\pgfusepath{stroke}
\pgfpathmoveto{\pgfpoint{287.158020pt}{222.000000pt}}
\pgflineto{\pgfpoint{289.600037pt}{222.000000pt}}
\pgfusepath{stroke}
{
\pgftransformshift{\pgfpoint{40.008171pt}{26.399979pt}}
\pgfnode{rectangle}{east}{\fontsize{10}{0}\selectfont\textcolor[rgb]{0,0,0}{{-2e+08}}}{}{\pgfusepath{discard}}}
{
\pgftransformshift{\pgfpoint{40.008171pt}{75.299988pt}}
\pgfnode{rectangle}{east}{\fontsize{10}{0}\selectfont\textcolor[rgb]{0,0,0}{{-1e+08}}}{}{\pgfusepath{discard}}}
{
\pgftransformshift{\pgfpoint{40.008171pt}{124.199989pt}}
\pgfnode{rectangle}{east}{\fontsize{10}{0}\selectfont\textcolor[rgb]{0,0,0}{{0}}}{}{\pgfusepath{discard}}}
{
\pgftransformshift{\pgfpoint{40.008171pt}{173.099991pt}}
\pgfnode{rectangle}{east}{\fontsize{10}{0}\selectfont\textcolor[rgb]{0,0,0}{{1e+08}}}{}{\pgfusepath{discard}}}
{
\pgftransformshift{\pgfpoint{40.008171pt}{222.000000pt}}
\pgfnode{rectangle}{east}{\fontsize{10}{0}\selectfont\textcolor[rgb]{0,0,0}{{2e+08}}}{}{\pgfusepath{discard}}}
\color[rgb]{0.000000,0.000000,1.000000}
\pgfsetdash{}{0pt}
\pgfpathmoveto{\pgfpoint{196.935547pt}{119.434402pt}}
\pgflineto{\pgfpoint{190.935547pt}{119.434402pt}}
\pgfusepath{stroke}
\pgfpathmoveto{\pgfpoint{193.935547pt}{116.434402pt}}
\pgflineto{\pgfpoint{193.935547pt}{122.434402pt}}
\pgfusepath{stroke}
\pgfpathmoveto{\pgfpoint{199.886261pt}{119.242699pt}}
\pgflineto{\pgfpoint{193.886261pt}{119.242699pt}}
\pgfusepath{stroke}
\pgfpathmoveto{\pgfpoint{196.886261pt}{116.242699pt}}
\pgflineto{\pgfpoint{196.886261pt}{122.242699pt}}
\pgfusepath{stroke}
\pgfpathmoveto{\pgfpoint{183.982819pt}{117.223633pt}}
\pgflineto{\pgfpoint{177.982819pt}{117.223633pt}}
\pgfusepath{stroke}
\pgfpathmoveto{\pgfpoint{180.982819pt}{114.223633pt}}
\pgflineto{\pgfpoint{180.982819pt}{120.223633pt}}
\pgfusepath{stroke}
\pgfpathmoveto{\pgfpoint{179.547073pt}{115.611145pt}}
\pgflineto{\pgfpoint{173.547073pt}{115.611145pt}}
\pgfusepath{stroke}
\pgfpathmoveto{\pgfpoint{176.547073pt}{112.611145pt}}
\pgflineto{\pgfpoint{176.547073pt}{118.611145pt}}
\pgfusepath{stroke}
\pgfpathmoveto{\pgfpoint{190.868942pt}{106.227837pt}}
\pgflineto{\pgfpoint{184.868942pt}{106.227837pt}}
\pgfusepath{stroke}
\pgfpathmoveto{\pgfpoint{187.868942pt}{103.227837pt}}
\pgflineto{\pgfpoint{187.868942pt}{109.227837pt}}
\pgfusepath{stroke}
\pgfpathmoveto{\pgfpoint{201.759155pt}{109.677612pt}}
\pgflineto{\pgfpoint{195.759155pt}{109.677612pt}}
\pgfusepath{stroke}
\pgfpathmoveto{\pgfpoint{198.759155pt}{106.677612pt}}
\pgflineto{\pgfpoint{198.759155pt}{112.677612pt}}
\pgfusepath{stroke}
\pgfpathmoveto{\pgfpoint{183.758728pt}{108.592598pt}}
\pgflineto{\pgfpoint{177.758728pt}{108.592598pt}}
\pgfusepath{stroke}
\pgfpathmoveto{\pgfpoint{180.758728pt}{105.592598pt}}
\pgflineto{\pgfpoint{180.758728pt}{111.592606pt}}
\pgfusepath{stroke}
\pgfpathmoveto{\pgfpoint{112.574402pt}{75.370621pt}}
\pgflineto{\pgfpoint{106.574402pt}{75.370621pt}}
\pgfusepath{stroke}
\pgfpathmoveto{\pgfpoint{109.574402pt}{72.370621pt}}
\pgflineto{\pgfpoint{109.574402pt}{78.370621pt}}
\pgfusepath{stroke}
\pgfpathmoveto{\pgfpoint{174.193832pt}{101.355026pt}}
\pgflineto{\pgfpoint{168.193832pt}{101.355026pt}}
\pgfusepath{stroke}
\pgfpathmoveto{\pgfpoint{171.193832pt}{98.355026pt}}
\pgflineto{\pgfpoint{171.193832pt}{104.355026pt}}
\pgfusepath{stroke}
\pgfpathmoveto{\pgfpoint{182.152756pt}{112.984955pt}}
\pgflineto{\pgfpoint{176.152756pt}{112.984955pt}}
\pgfusepath{stroke}
\pgfpathmoveto{\pgfpoint{179.152756pt}{109.984955pt}}
\pgflineto{\pgfpoint{179.152756pt}{115.984955pt}}
\pgfusepath{stroke}
\pgfpathmoveto{\pgfpoint{168.108780pt}{110.069618pt}}
\pgflineto{\pgfpoint{162.108780pt}{110.069618pt}}
\pgfusepath{stroke}
\pgfpathmoveto{\pgfpoint{165.108780pt}{107.069618pt}}
\pgflineto{\pgfpoint{165.108780pt}{113.069618pt}}
\pgfusepath{stroke}
\pgfpathmoveto{\pgfpoint{201.415054pt}{114.212738pt}}
\pgflineto{\pgfpoint{195.415054pt}{114.212738pt}}
\pgfusepath{stroke}
\pgfpathmoveto{\pgfpoint{198.415054pt}{111.212738pt}}
\pgflineto{\pgfpoint{198.415054pt}{117.212738pt}}
\pgfusepath{stroke}
\pgfpathmoveto{\pgfpoint{192.300415pt}{121.396606pt}}
\pgflineto{\pgfpoint{186.300415pt}{121.396606pt}}
\pgfusepath{stroke}
\pgfpathmoveto{\pgfpoint{189.300415pt}{118.396606pt}}
\pgflineto{\pgfpoint{189.300415pt}{124.396606pt}}
\pgfusepath{stroke}
\pgfpathmoveto{\pgfpoint{193.598267pt}{124.001534pt}}
\pgflineto{\pgfpoint{187.598267pt}{124.001534pt}}
\pgfusepath{stroke}
\pgfpathmoveto{\pgfpoint{190.598267pt}{121.001534pt}}
\pgflineto{\pgfpoint{190.598267pt}{127.001534pt}}
\pgfusepath{stroke}
\pgfpathmoveto{\pgfpoint{180.524246pt}{109.700615pt}}
\pgflineto{\pgfpoint{174.524246pt}{109.700615pt}}
\pgfusepath{stroke}
\pgfpathmoveto{\pgfpoint{177.524246pt}{106.700615pt}}
\pgflineto{\pgfpoint{177.524246pt}{112.700615pt}}
\pgfusepath{stroke}
\pgfpathmoveto{\pgfpoint{176.954651pt}{103.460312pt}}
\pgflineto{\pgfpoint{170.954651pt}{103.460312pt}}
\pgfusepath{stroke}
\pgfpathmoveto{\pgfpoint{173.954651pt}{100.460312pt}}
\pgflineto{\pgfpoint{173.954651pt}{106.460312pt}}
\pgfusepath{stroke}
\pgfpathmoveto{\pgfpoint{185.662079pt}{110.839188pt}}
\pgflineto{\pgfpoint{179.662079pt}{110.839188pt}}
\pgfusepath{stroke}
\pgfpathmoveto{\pgfpoint{182.662079pt}{107.839188pt}}
\pgflineto{\pgfpoint{182.662079pt}{113.839188pt}}
\pgfusepath{stroke}
\pgfpathmoveto{\pgfpoint{148.840988pt}{87.041306pt}}
\pgflineto{\pgfpoint{142.840988pt}{87.041306pt}}
\pgfusepath{stroke}
\pgfpathmoveto{\pgfpoint{145.840988pt}{84.041306pt}}
\pgflineto{\pgfpoint{145.840988pt}{90.041306pt}}
\pgfusepath{stroke}
\pgfpathmoveto{\pgfpoint{149.264221pt}{74.474411pt}}
\pgflineto{\pgfpoint{143.264221pt}{74.474411pt}}
\pgfusepath{stroke}
\pgfpathmoveto{\pgfpoint{146.264221pt}{71.474411pt}}
\pgflineto{\pgfpoint{146.264221pt}{77.474411pt}}
\pgfusepath{stroke}
\pgfpathmoveto{\pgfpoint{208.496521pt}{104.231644pt}}
\pgflineto{\pgfpoint{202.496521pt}{104.231644pt}}
\pgfusepath{stroke}
\pgfpathmoveto{\pgfpoint{205.496521pt}{101.231644pt}}
\pgflineto{\pgfpoint{205.496521pt}{107.231644pt}}
\pgfusepath{stroke}
\pgfpathmoveto{\pgfpoint{171.510223pt}{118.414749pt}}
\pgflineto{\pgfpoint{165.510223pt}{118.414749pt}}
\pgfusepath{stroke}
\pgfpathmoveto{\pgfpoint{168.510223pt}{115.414749pt}}
\pgflineto{\pgfpoint{168.510223pt}{121.414749pt}}
\pgfusepath{stroke}
\pgfpathmoveto{\pgfpoint{179.501907pt}{118.791878pt}}
\pgflineto{\pgfpoint{173.501907pt}{118.791878pt}}
\pgfusepath{stroke}
\pgfpathmoveto{\pgfpoint{176.501907pt}{115.791878pt}}
\pgflineto{\pgfpoint{176.501907pt}{121.791878pt}}
\pgfusepath{stroke}
\pgfpathmoveto{\pgfpoint{184.015793pt}{119.659134pt}}
\pgflineto{\pgfpoint{178.015793pt}{119.659134pt}}
\pgfusepath{stroke}
\pgfpathmoveto{\pgfpoint{181.015793pt}{116.659134pt}}
\pgflineto{\pgfpoint{181.015793pt}{122.659142pt}}
\pgfusepath{stroke}
\pgfpathmoveto{\pgfpoint{170.695862pt}{117.468170pt}}
\pgflineto{\pgfpoint{164.695862pt}{117.468170pt}}
\pgfusepath{stroke}
\pgfpathmoveto{\pgfpoint{167.695862pt}{114.468163pt}}
\pgflineto{\pgfpoint{167.695862pt}{120.468170pt}}
\pgfusepath{stroke}
\pgfpathmoveto{\pgfpoint{196.609299pt}{118.015953pt}}
\pgflineto{\pgfpoint{190.609299pt}{118.015953pt}}
\pgfusepath{stroke}
\pgfpathmoveto{\pgfpoint{193.609299pt}{115.015953pt}}
\pgflineto{\pgfpoint{193.609299pt}{121.015953pt}}
\pgfusepath{stroke}
\pgfpathmoveto{\pgfpoint{197.507370pt}{115.068123pt}}
\pgflineto{\pgfpoint{191.507370pt}{115.068123pt}}
\pgfusepath{stroke}
\pgfpathmoveto{\pgfpoint{194.507370pt}{112.068123pt}}
\pgflineto{\pgfpoint{194.507370pt}{118.068123pt}}
\pgfusepath{stroke}
\pgfpathmoveto{\pgfpoint{193.406509pt}{114.448425pt}}
\pgflineto{\pgfpoint{187.406509pt}{114.448425pt}}
\pgfusepath{stroke}
\pgfpathmoveto{\pgfpoint{190.406509pt}{111.448425pt}}
\pgflineto{\pgfpoint{190.406509pt}{117.448425pt}}
\pgfusepath{stroke}
\pgfpathmoveto{\pgfpoint{194.718613pt}{114.180740pt}}
\pgflineto{\pgfpoint{188.718613pt}{114.180740pt}}
\pgfusepath{stroke}
\pgfpathmoveto{\pgfpoint{191.718613pt}{111.180733pt}}
\pgflineto{\pgfpoint{191.718613pt}{117.180740pt}}
\pgfusepath{stroke}
\pgfpathmoveto{\pgfpoint{202.550079pt}{120.974258pt}}
\pgflineto{\pgfpoint{196.550079pt}{120.974258pt}}
\pgfusepath{stroke}
\pgfpathmoveto{\pgfpoint{199.550079pt}{117.974258pt}}
\pgflineto{\pgfpoint{199.550079pt}{123.974258pt}}
\pgfusepath{stroke}
\pgfpathmoveto{\pgfpoint{194.683685pt}{124.179276pt}}
\pgflineto{\pgfpoint{188.683685pt}{124.179276pt}}
\pgfusepath{stroke}
\pgfpathmoveto{\pgfpoint{191.683685pt}{121.179276pt}}
\pgflineto{\pgfpoint{191.683685pt}{127.179276pt}}
\pgfusepath{stroke}
\pgfpathmoveto{\pgfpoint{205.879364pt}{112.132126pt}}
\pgflineto{\pgfpoint{199.879364pt}{112.132126pt}}
\pgfusepath{stroke}
\pgfpathmoveto{\pgfpoint{202.879364pt}{109.132126pt}}
\pgflineto{\pgfpoint{202.879364pt}{115.132133pt}}
\pgfusepath{stroke}
\pgfpathmoveto{\pgfpoint{206.979767pt}{90.041245pt}}
\pgflineto{\pgfpoint{200.979767pt}{90.041245pt}}
\pgfusepath{stroke}
\pgfpathmoveto{\pgfpoint{203.979767pt}{87.041245pt}}
\pgflineto{\pgfpoint{203.979767pt}{93.041245pt}}
\pgfusepath{stroke}
\pgfpathmoveto{\pgfpoint{196.589050pt}{123.302505pt}}
\pgflineto{\pgfpoint{190.589050pt}{123.302505pt}}
\pgfusepath{stroke}
\pgfpathmoveto{\pgfpoint{193.589050pt}{120.302505pt}}
\pgflineto{\pgfpoint{193.589050pt}{126.302505pt}}
\pgfusepath{stroke}
\pgfpathmoveto{\pgfpoint{209.073334pt}{97.586403pt}}
\pgflineto{\pgfpoint{203.073318pt}{97.586403pt}}
\pgfusepath{stroke}
\pgfpathmoveto{\pgfpoint{206.073318pt}{94.586403pt}}
\pgflineto{\pgfpoint{206.073318pt}{100.586403pt}}
\pgfusepath{stroke}
\pgfpathmoveto{\pgfpoint{195.115631pt}{118.254936pt}}
\pgflineto{\pgfpoint{189.115631pt}{118.254936pt}}
\pgfusepath{stroke}
\pgfpathmoveto{\pgfpoint{192.115631pt}{115.254936pt}}
\pgflineto{\pgfpoint{192.115631pt}{121.254936pt}}
\pgfusepath{stroke}
\pgfpathmoveto{\pgfpoint{201.029053pt}{98.080841pt}}
\pgflineto{\pgfpoint{195.029053pt}{98.080841pt}}
\pgfusepath{stroke}
\pgfpathmoveto{\pgfpoint{198.029053pt}{95.080841pt}}
\pgflineto{\pgfpoint{198.029053pt}{101.080841pt}}
\pgfusepath{stroke}
\pgfpathmoveto{\pgfpoint{197.477219pt}{106.478531pt}}
\pgflineto{\pgfpoint{191.477219pt}{106.478531pt}}
\pgfusepath{stroke}
\pgfpathmoveto{\pgfpoint{194.477219pt}{103.478531pt}}
\pgflineto{\pgfpoint{194.477219pt}{109.478531pt}}
\pgfusepath{stroke}
\pgfpathmoveto{\pgfpoint{148.010254pt}{79.851120pt}}
\pgflineto{\pgfpoint{142.010254pt}{79.851120pt}}
\pgfusepath{stroke}
\pgfpathmoveto{\pgfpoint{145.010254pt}{76.851120pt}}
\pgflineto{\pgfpoint{145.010254pt}{82.851120pt}}
\pgfusepath{stroke}
\pgfpathmoveto{\pgfpoint{175.602463pt}{102.839371pt}}
\pgflineto{\pgfpoint{169.602463pt}{102.839371pt}}
\pgfusepath{stroke}
\pgfpathmoveto{\pgfpoint{172.602463pt}{99.839371pt}}
\pgflineto{\pgfpoint{172.602463pt}{105.839371pt}}
\pgfusepath{stroke}
\pgfpathmoveto{\pgfpoint{168.293350pt}{102.053452pt}}
\pgflineto{\pgfpoint{162.293350pt}{102.053452pt}}
\pgfusepath{stroke}
\pgfpathmoveto{\pgfpoint{165.293350pt}{99.053444pt}}
\pgflineto{\pgfpoint{165.293350pt}{105.053452pt}}
\pgfusepath{stroke}
\pgfpathmoveto{\pgfpoint{188.499710pt}{114.862000pt}}
\pgflineto{\pgfpoint{182.499710pt}{114.862000pt}}
\pgfusepath{stroke}
\pgfpathmoveto{\pgfpoint{185.499710pt}{111.862000pt}}
\pgflineto{\pgfpoint{185.499710pt}{117.862007pt}}
\pgfusepath{stroke}
\pgfpathmoveto{\pgfpoint{174.017639pt}{119.228752pt}}
\pgflineto{\pgfpoint{168.017639pt}{119.228752pt}}
\pgfusepath{stroke}
\pgfpathmoveto{\pgfpoint{171.017639pt}{116.228752pt}}
\pgflineto{\pgfpoint{171.017639pt}{122.228752pt}}
\pgfusepath{stroke}
\pgfpathmoveto{\pgfpoint{166.606094pt}{107.932907pt}}
\pgflineto{\pgfpoint{160.606094pt}{107.932907pt}}
\pgfusepath{stroke}
\pgfpathmoveto{\pgfpoint{163.606094pt}{104.932907pt}}
\pgflineto{\pgfpoint{163.606094pt}{110.932907pt}}
\pgfusepath{stroke}
\pgfpathmoveto{\pgfpoint{178.223022pt}{102.634354pt}}
\pgflineto{\pgfpoint{172.223022pt}{102.634354pt}}
\pgfusepath{stroke}
\pgfpathmoveto{\pgfpoint{175.223022pt}{99.634346pt}}
\pgflineto{\pgfpoint{175.223022pt}{105.634354pt}}
\pgfusepath{stroke}
\pgfpathmoveto{\pgfpoint{214.268692pt}{114.793839pt}}
\pgflineto{\pgfpoint{208.268692pt}{114.793839pt}}
\pgfusepath{stroke}
\pgfpathmoveto{\pgfpoint{211.268692pt}{111.793839pt}}
\pgflineto{\pgfpoint{211.268692pt}{117.793839pt}}
\pgfusepath{stroke}
\pgfpathmoveto{\pgfpoint{222.330811pt}{111.535378pt}}
\pgflineto{\pgfpoint{216.330811pt}{111.535378pt}}
\pgfusepath{stroke}
\pgfpathmoveto{\pgfpoint{219.330811pt}{108.535378pt}}
\pgflineto{\pgfpoint{219.330811pt}{114.535378pt}}
\pgfusepath{stroke}
\pgfpathmoveto{\pgfpoint{212.852081pt}{120.428650pt}}
\pgflineto{\pgfpoint{206.852081pt}{120.428650pt}}
\pgfusepath{stroke}
\pgfpathmoveto{\pgfpoint{209.852081pt}{117.428650pt}}
\pgflineto{\pgfpoint{209.852081pt}{123.428650pt}}
\pgfusepath{stroke}
\pgfpathmoveto{\pgfpoint{201.270599pt}{119.242432pt}}
\pgflineto{\pgfpoint{195.270584pt}{119.242432pt}}
\pgfusepath{stroke}
\pgfpathmoveto{\pgfpoint{198.270584pt}{116.242432pt}}
\pgflineto{\pgfpoint{198.270584pt}{122.242432pt}}
\pgfusepath{stroke}
\pgfpathmoveto{\pgfpoint{170.743149pt}{91.735825pt}}
\pgflineto{\pgfpoint{164.743149pt}{91.735825pt}}
\pgfusepath{stroke}
\pgfpathmoveto{\pgfpoint{167.743149pt}{88.735825pt}}
\pgflineto{\pgfpoint{167.743149pt}{94.735825pt}}
\pgfusepath{stroke}
\pgfpathmoveto{\pgfpoint{208.540649pt}{102.282532pt}}
\pgflineto{\pgfpoint{202.540634pt}{102.282532pt}}
\pgfusepath{stroke}
\pgfpathmoveto{\pgfpoint{205.540634pt}{99.282532pt}}
\pgflineto{\pgfpoint{205.540634pt}{105.282532pt}}
\pgfusepath{stroke}
\pgfpathmoveto{\pgfpoint{199.218811pt}{68.839760pt}}
\pgflineto{\pgfpoint{193.218811pt}{68.839760pt}}
\pgfusepath{stroke}
\pgfpathmoveto{\pgfpoint{196.218811pt}{65.839760pt}}
\pgflineto{\pgfpoint{196.218811pt}{71.839760pt}}
\pgfusepath{stroke}
\pgfpathmoveto{\pgfpoint{187.196762pt}{116.726044pt}}
\pgflineto{\pgfpoint{181.196762pt}{116.726044pt}}
\pgfusepath{stroke}
\pgfpathmoveto{\pgfpoint{184.196762pt}{113.726044pt}}
\pgflineto{\pgfpoint{184.196762pt}{119.726044pt}}
\pgfusepath{stroke}
\pgfpathmoveto{\pgfpoint{147.139359pt}{71.754990pt}}
\pgflineto{\pgfpoint{141.139359pt}{71.754990pt}}
\pgfusepath{stroke}
\pgfpathmoveto{\pgfpoint{144.139359pt}{68.754990pt}}
\pgflineto{\pgfpoint{144.139359pt}{74.754990pt}}
\pgfusepath{stroke}
\pgfpathmoveto{\pgfpoint{215.177032pt}{136.581879pt}}
\pgflineto{\pgfpoint{209.177032pt}{136.581879pt}}
\pgfusepath{stroke}
\pgfpathmoveto{\pgfpoint{212.177032pt}{133.581879pt}}
\pgflineto{\pgfpoint{212.177032pt}{139.581879pt}}
\pgfusepath{stroke}
\pgfpathmoveto{\pgfpoint{184.478500pt}{114.593369pt}}
\pgflineto{\pgfpoint{178.478500pt}{114.593369pt}}
\pgfusepath{stroke}
\pgfpathmoveto{\pgfpoint{181.478500pt}{111.593369pt}}
\pgflineto{\pgfpoint{181.478500pt}{117.593369pt}}
\pgfusepath{stroke}
\pgfpathmoveto{\pgfpoint{190.108887pt}{122.127441pt}}
\pgflineto{\pgfpoint{184.108887pt}{122.127441pt}}
\pgfusepath{stroke}
\pgfpathmoveto{\pgfpoint{187.108887pt}{119.127441pt}}
\pgflineto{\pgfpoint{187.108887pt}{125.127441pt}}
\pgfusepath{stroke}
\pgfpathmoveto{\pgfpoint{179.439209pt}{116.649216pt}}
\pgflineto{\pgfpoint{173.439209pt}{116.649216pt}}
\pgfusepath{stroke}
\pgfpathmoveto{\pgfpoint{176.439209pt}{113.649216pt}}
\pgflineto{\pgfpoint{176.439209pt}{119.649216pt}}
\pgfusepath{stroke}
\pgfpathmoveto{\pgfpoint{104.465897pt}{107.626564pt}}
\pgflineto{\pgfpoint{98.465889pt}{107.626564pt}}
\pgfusepath{stroke}
\pgfpathmoveto{\pgfpoint{101.465897pt}{104.626564pt}}
\pgflineto{\pgfpoint{101.465897pt}{110.626564pt}}
\pgfusepath{stroke}
\pgfpathmoveto{\pgfpoint{177.118210pt}{119.713600pt}}
\pgflineto{\pgfpoint{171.118210pt}{119.713600pt}}
\pgfusepath{stroke}
\pgfpathmoveto{\pgfpoint{174.118210pt}{116.713600pt}}
\pgflineto{\pgfpoint{174.118210pt}{122.713600pt}}
\pgfusepath{stroke}
\pgfpathmoveto{\pgfpoint{164.368271pt}{113.844521pt}}
\pgflineto{\pgfpoint{158.368271pt}{113.844521pt}}
\pgfusepath{stroke}
\pgfpathmoveto{\pgfpoint{161.368271pt}{110.844521pt}}
\pgflineto{\pgfpoint{161.368271pt}{116.844521pt}}
\pgfusepath{stroke}
\pgfpathmoveto{\pgfpoint{174.302505pt}{114.449341pt}}
\pgflineto{\pgfpoint{168.302505pt}{114.449341pt}}
\pgfusepath{stroke}
\pgfpathmoveto{\pgfpoint{171.302505pt}{111.449341pt}}
\pgflineto{\pgfpoint{171.302505pt}{117.449341pt}}
\pgfusepath{stroke}
\pgfpathmoveto{\pgfpoint{188.850342pt}{120.117348pt}}
\pgflineto{\pgfpoint{182.850342pt}{120.117348pt}}
\pgfusepath{stroke}
\pgfpathmoveto{\pgfpoint{185.850342pt}{117.117348pt}}
\pgflineto{\pgfpoint{185.850342pt}{123.117348pt}}
\pgfusepath{stroke}
\pgfpathmoveto{\pgfpoint{189.115555pt}{120.904907pt}}
\pgflineto{\pgfpoint{183.115555pt}{120.904907pt}}
\pgfusepath{stroke}
\pgfpathmoveto{\pgfpoint{186.115555pt}{117.904907pt}}
\pgflineto{\pgfpoint{186.115555pt}{123.904907pt}}
\pgfusepath{stroke}
\pgfpathmoveto{\pgfpoint{164.011871pt}{105.423592pt}}
\pgflineto{\pgfpoint{158.011871pt}{105.423592pt}}
\pgfusepath{stroke}
\pgfpathmoveto{\pgfpoint{161.011871pt}{102.423584pt}}
\pgflineto{\pgfpoint{161.011871pt}{108.423592pt}}
\pgfusepath{stroke}
\pgfpathmoveto{\pgfpoint{123.379929pt}{92.500244pt}}
\pgflineto{\pgfpoint{117.379929pt}{92.500244pt}}
\pgfusepath{stroke}
\pgfpathmoveto{\pgfpoint{120.379936pt}{89.500244pt}}
\pgflineto{\pgfpoint{120.379936pt}{95.500244pt}}
\pgfusepath{stroke}
\pgfpathmoveto{\pgfpoint{224.852325pt}{104.413704pt}}
\pgflineto{\pgfpoint{218.852325pt}{104.413704pt}}
\pgfusepath{stroke}
\pgfpathmoveto{\pgfpoint{221.852325pt}{101.413704pt}}
\pgflineto{\pgfpoint{221.852325pt}{107.413704pt}}
\pgfusepath{stroke}
\pgfpathmoveto{\pgfpoint{158.911072pt}{77.804001pt}}
\pgflineto{\pgfpoint{152.911072pt}{77.804001pt}}
\pgfusepath{stroke}
\pgfpathmoveto{\pgfpoint{155.911072pt}{74.804001pt}}
\pgflineto{\pgfpoint{155.911072pt}{80.804001pt}}
\pgfusepath{stroke}
\pgfpathmoveto{\pgfpoint{200.542542pt}{112.245041pt}}
\pgflineto{\pgfpoint{194.542542pt}{112.245041pt}}
\pgfusepath{stroke}
\pgfpathmoveto{\pgfpoint{197.542542pt}{109.245041pt}}
\pgflineto{\pgfpoint{197.542542pt}{115.245041pt}}
\pgfusepath{stroke}
\pgfpathmoveto{\pgfpoint{202.097809pt}{102.256493pt}}
\pgflineto{\pgfpoint{196.097809pt}{102.256493pt}}
\pgfusepath{stroke}
\pgfpathmoveto{\pgfpoint{199.097809pt}{99.256493pt}}
\pgflineto{\pgfpoint{199.097809pt}{105.256493pt}}
\pgfusepath{stroke}
\pgfpathmoveto{\pgfpoint{209.810730pt}{94.916466pt}}
\pgflineto{\pgfpoint{203.810730pt}{94.916466pt}}
\pgfusepath{stroke}
\pgfpathmoveto{\pgfpoint{206.810730pt}{91.916466pt}}
\pgflineto{\pgfpoint{206.810730pt}{97.916466pt}}
\pgfusepath{stroke}
\pgfpathmoveto{\pgfpoint{188.358261pt}{110.721474pt}}
\pgflineto{\pgfpoint{182.358261pt}{110.721474pt}}
\pgfusepath{stroke}
\pgfpathmoveto{\pgfpoint{185.358261pt}{107.721474pt}}
\pgflineto{\pgfpoint{185.358261pt}{113.721474pt}}
\pgfusepath{stroke}
\pgfpathmoveto{\pgfpoint{202.170609pt}{103.766815pt}}
\pgflineto{\pgfpoint{196.170609pt}{103.766815pt}}
\pgfusepath{stroke}
\pgfpathmoveto{\pgfpoint{199.170609pt}{100.766815pt}}
\pgflineto{\pgfpoint{199.170609pt}{106.766815pt}}
\pgfusepath{stroke}
\pgfpathmoveto{\pgfpoint{193.927582pt}{120.353355pt}}
\pgflineto{\pgfpoint{187.927582pt}{120.353355pt}}
\pgfusepath{stroke}
\pgfpathmoveto{\pgfpoint{190.927582pt}{117.353355pt}}
\pgflineto{\pgfpoint{190.927582pt}{123.353355pt}}
\pgfusepath{stroke}
\pgfpathmoveto{\pgfpoint{214.340698pt}{107.736900pt}}
\pgflineto{\pgfpoint{208.340698pt}{107.736900pt}}
\pgfusepath{stroke}
\pgfpathmoveto{\pgfpoint{211.340698pt}{104.736893pt}}
\pgflineto{\pgfpoint{211.340698pt}{110.736900pt}}
\pgfusepath{stroke}
\pgfpathmoveto{\pgfpoint{192.854523pt}{116.522400pt}}
\pgflineto{\pgfpoint{186.854523pt}{116.522400pt}}
\pgfusepath{stroke}
\pgfpathmoveto{\pgfpoint{189.854523pt}{113.522400pt}}
\pgflineto{\pgfpoint{189.854523pt}{119.522400pt}}
\pgfusepath{stroke}
\pgfpathmoveto{\pgfpoint{180.888123pt}{119.808556pt}}
\pgflineto{\pgfpoint{174.888123pt}{119.808556pt}}
\pgfusepath{stroke}
\pgfpathmoveto{\pgfpoint{177.888123pt}{116.808556pt}}
\pgflineto{\pgfpoint{177.888123pt}{122.808556pt}}
\pgfusepath{stroke}
\pgfpathmoveto{\pgfpoint{176.316513pt}{120.338478pt}}
\pgflineto{\pgfpoint{170.316513pt}{120.338478pt}}
\pgfusepath{stroke}
\pgfpathmoveto{\pgfpoint{173.316513pt}{117.338478pt}}
\pgflineto{\pgfpoint{173.316513pt}{123.338478pt}}
\pgfusepath{stroke}
\pgfpathmoveto{\pgfpoint{183.214355pt}{122.327942pt}}
\pgflineto{\pgfpoint{177.214355pt}{122.327942pt}}
\pgfusepath{stroke}
\pgfpathmoveto{\pgfpoint{180.214355pt}{119.327942pt}}
\pgflineto{\pgfpoint{180.214355pt}{125.327942pt}}
\pgfusepath{stroke}
\pgfpathmoveto{\pgfpoint{191.410797pt}{122.499756pt}}
\pgflineto{\pgfpoint{185.410797pt}{122.499756pt}}
\pgfusepath{stroke}
\pgfpathmoveto{\pgfpoint{188.410797pt}{119.499756pt}}
\pgflineto{\pgfpoint{188.410797pt}{125.499756pt}}
\pgfusepath{stroke}
\pgfpathmoveto{\pgfpoint{197.348083pt}{116.337822pt}}
\pgflineto{\pgfpoint{191.348083pt}{116.337822pt}}
\pgfusepath{stroke}
\pgfpathmoveto{\pgfpoint{194.348083pt}{113.337822pt}}
\pgflineto{\pgfpoint{194.348083pt}{119.337822pt}}
\pgfusepath{stroke}
\pgfpathmoveto{\pgfpoint{150.088013pt}{108.014786pt}}
\pgflineto{\pgfpoint{144.088013pt}{108.014786pt}}
\pgfusepath{stroke}
\pgfpathmoveto{\pgfpoint{147.088013pt}{105.014786pt}}
\pgflineto{\pgfpoint{147.088013pt}{111.014786pt}}
\pgfusepath{stroke}
\pgfpathmoveto{\pgfpoint{165.265259pt}{92.156952pt}}
\pgflineto{\pgfpoint{159.265259pt}{92.156952pt}}
\pgfusepath{stroke}
\pgfpathmoveto{\pgfpoint{162.265259pt}{89.156952pt}}
\pgflineto{\pgfpoint{162.265259pt}{95.156952pt}}
\pgfusepath{stroke}
\pgfpathmoveto{\pgfpoint{164.344528pt}{89.086136pt}}
\pgflineto{\pgfpoint{158.344528pt}{89.086136pt}}
\pgfusepath{stroke}
\pgfpathmoveto{\pgfpoint{161.344528pt}{86.086136pt}}
\pgflineto{\pgfpoint{161.344528pt}{92.086136pt}}
\pgfusepath{stroke}
\pgfpathmoveto{\pgfpoint{144.495544pt}{104.760262pt}}
\pgflineto{\pgfpoint{138.495544pt}{104.760262pt}}
\pgfusepath{stroke}
\pgfpathmoveto{\pgfpoint{141.495544pt}{101.760262pt}}
\pgflineto{\pgfpoint{141.495544pt}{107.760262pt}}
\pgfusepath{stroke}
\pgfpathmoveto{\pgfpoint{152.836807pt}{126.138474pt}}
\pgflineto{\pgfpoint{146.836807pt}{126.138474pt}}
\pgfusepath{stroke}
\pgfpathmoveto{\pgfpoint{149.836807pt}{123.138474pt}}
\pgflineto{\pgfpoint{149.836807pt}{129.138474pt}}
\pgfusepath{stroke}
\pgfpathmoveto{\pgfpoint{200.675446pt}{115.899910pt}}
\pgflineto{\pgfpoint{194.675446pt}{115.899910pt}}
\pgfusepath{stroke}
\pgfpathmoveto{\pgfpoint{197.675446pt}{112.899910pt}}
\pgflineto{\pgfpoint{197.675446pt}{118.899910pt}}
\pgfusepath{stroke}
\pgfpathmoveto{\pgfpoint{190.927689pt}{96.198410pt}}
\pgflineto{\pgfpoint{184.927689pt}{96.198410pt}}
\pgfusepath{stroke}
\pgfpathmoveto{\pgfpoint{187.927689pt}{93.198410pt}}
\pgflineto{\pgfpoint{187.927689pt}{99.198410pt}}
\pgfusepath{stroke}
\pgfpathmoveto{\pgfpoint{203.814697pt}{103.487259pt}}
\pgflineto{\pgfpoint{197.814697pt}{103.487259pt}}
\pgfusepath{stroke}
\pgfpathmoveto{\pgfpoint{200.814697pt}{100.487259pt}}
\pgflineto{\pgfpoint{200.814697pt}{106.487267pt}}
\pgfusepath{stroke}
\pgfpathmoveto{\pgfpoint{200.864105pt}{116.337234pt}}
\pgflineto{\pgfpoint{194.864105pt}{116.337234pt}}
\pgfusepath{stroke}
\pgfpathmoveto{\pgfpoint{197.864105pt}{113.337234pt}}
\pgflineto{\pgfpoint{197.864105pt}{119.337234pt}}
\pgfusepath{stroke}
\pgfpathmoveto{\pgfpoint{195.630478pt}{117.957169pt}}
\pgflineto{\pgfpoint{189.630478pt}{117.957169pt}}
\pgfusepath{stroke}
\pgfpathmoveto{\pgfpoint{192.630478pt}{114.957169pt}}
\pgflineto{\pgfpoint{192.630478pt}{120.957176pt}}
\pgfusepath{stroke}
\pgfpathmoveto{\pgfpoint{164.992645pt}{119.094711pt}}
\pgflineto{\pgfpoint{158.992645pt}{119.094711pt}}
\pgfusepath{stroke}
\pgfpathmoveto{\pgfpoint{161.992645pt}{116.094711pt}}
\pgflineto{\pgfpoint{161.992645pt}{122.094719pt}}
\pgfusepath{stroke}
\pgfpathmoveto{\pgfpoint{184.135727pt}{114.890152pt}}
\pgflineto{\pgfpoint{178.135727pt}{114.890152pt}}
\pgfusepath{stroke}
\pgfpathmoveto{\pgfpoint{181.135727pt}{111.890152pt}}
\pgflineto{\pgfpoint{181.135727pt}{117.890160pt}}
\pgfusepath{stroke}
\pgfpathmoveto{\pgfpoint{183.212982pt}{107.354736pt}}
\pgflineto{\pgfpoint{177.212982pt}{107.354736pt}}
\pgfusepath{stroke}
\pgfpathmoveto{\pgfpoint{180.212982pt}{104.354736pt}}
\pgflineto{\pgfpoint{180.212982pt}{110.354736pt}}
\pgfusepath{stroke}
\pgfpathmoveto{\pgfpoint{205.889694pt}{104.733597pt}}
\pgflineto{\pgfpoint{199.889694pt}{104.733597pt}}
\pgfusepath{stroke}
\pgfpathmoveto{\pgfpoint{202.889694pt}{101.733597pt}}
\pgflineto{\pgfpoint{202.889694pt}{107.733597pt}}
\pgfusepath{stroke}
\pgfpathmoveto{\pgfpoint{67.810738pt}{29.410484pt}}
\pgflineto{\pgfpoint{61.810730pt}{29.410484pt}}
\pgfusepath{stroke}
\pgfpathmoveto{\pgfpoint{64.810730pt}{26.410492pt}}
\pgflineto{\pgfpoint{64.810730pt}{32.410484pt}}
\pgfusepath{stroke}
\pgfpathmoveto{\pgfpoint{138.959503pt}{82.218102pt}}
\pgflineto{\pgfpoint{132.959503pt}{82.218102pt}}
\pgfusepath{stroke}
\pgfpathmoveto{\pgfpoint{135.959503pt}{79.218102pt}}
\pgflineto{\pgfpoint{135.959503pt}{85.218109pt}}
\pgfusepath{stroke}
\pgfpathmoveto{\pgfpoint{195.715820pt}{105.990662pt}}
\pgflineto{\pgfpoint{189.715820pt}{105.990662pt}}
\pgfusepath{stroke}
\pgfpathmoveto{\pgfpoint{192.715820pt}{102.990662pt}}
\pgflineto{\pgfpoint{192.715820pt}{108.990662pt}}
\pgfusepath{stroke}
\pgfpathmoveto{\pgfpoint{210.786179pt}{108.818954pt}}
\pgflineto{\pgfpoint{204.786179pt}{108.818954pt}}
\pgfusepath{stroke}
\pgfpathmoveto{\pgfpoint{207.786179pt}{105.818947pt}}
\pgflineto{\pgfpoint{207.786179pt}{111.818954pt}}
\pgfusepath{stroke}
\pgfpathmoveto{\pgfpoint{204.065964pt}{84.708786pt}}
\pgflineto{\pgfpoint{198.065964pt}{84.708786pt}}
\pgfusepath{stroke}
\pgfpathmoveto{\pgfpoint{201.065964pt}{81.708786pt}}
\pgflineto{\pgfpoint{201.065964pt}{87.708786pt}}
\pgfusepath{stroke}
\pgfpathmoveto{\pgfpoint{194.247559pt}{84.369720pt}}
\pgflineto{\pgfpoint{188.247559pt}{84.369720pt}}
\pgfusepath{stroke}
\pgfpathmoveto{\pgfpoint{191.247559pt}{81.369720pt}}
\pgflineto{\pgfpoint{191.247559pt}{87.369720pt}}
\pgfusepath{stroke}
\pgfpathmoveto{\pgfpoint{235.681610pt}{74.712463pt}}
\pgflineto{\pgfpoint{229.681595pt}{74.712463pt}}
\pgfusepath{stroke}
\pgfpathmoveto{\pgfpoint{232.681595pt}{71.712463pt}}
\pgflineto{\pgfpoint{232.681595pt}{77.712463pt}}
\pgfusepath{stroke}
\pgfpathmoveto{\pgfpoint{208.190735pt}{79.811440pt}}
\pgflineto{\pgfpoint{202.190735pt}{79.811440pt}}
\pgfusepath{stroke}
\pgfpathmoveto{\pgfpoint{205.190735pt}{76.811440pt}}
\pgflineto{\pgfpoint{205.190735pt}{82.811440pt}}
\pgfusepath{stroke}
\pgfpathmoveto{\pgfpoint{208.186905pt}{79.900948pt}}
\pgflineto{\pgfpoint{202.186890pt}{79.900948pt}}
\pgfusepath{stroke}
\pgfpathmoveto{\pgfpoint{205.186905pt}{76.900948pt}}
\pgflineto{\pgfpoint{205.186905pt}{82.900948pt}}
\pgfusepath{stroke}
\pgfpathmoveto{\pgfpoint{194.421799pt}{101.387360pt}}
\pgflineto{\pgfpoint{188.421799pt}{101.387360pt}}
\pgfusepath{stroke}
\pgfpathmoveto{\pgfpoint{191.421799pt}{98.387360pt}}
\pgflineto{\pgfpoint{191.421799pt}{104.387360pt}}
\pgfusepath{stroke}
\pgfpathmoveto{\pgfpoint{185.854950pt}{114.317383pt}}
\pgflineto{\pgfpoint{179.854950pt}{114.317383pt}}
\pgfusepath{stroke}
\pgfpathmoveto{\pgfpoint{182.854950pt}{111.317375pt}}
\pgflineto{\pgfpoint{182.854950pt}{117.317383pt}}
\pgfusepath{stroke}
\pgfpathmoveto{\pgfpoint{169.837845pt}{103.355423pt}}
\pgflineto{\pgfpoint{163.837845pt}{103.355423pt}}
\pgfusepath{stroke}
\pgfpathmoveto{\pgfpoint{166.837845pt}{100.355423pt}}
\pgflineto{\pgfpoint{166.837845pt}{106.355431pt}}
\pgfusepath{stroke}
\pgfpathmoveto{\pgfpoint{177.191116pt}{103.091187pt}}
\pgflineto{\pgfpoint{171.191116pt}{103.091187pt}}
\pgfusepath{stroke}
\pgfpathmoveto{\pgfpoint{174.191116pt}{100.091187pt}}
\pgflineto{\pgfpoint{174.191116pt}{106.091187pt}}
\pgfusepath{stroke}
\pgfpathmoveto{\pgfpoint{158.701706pt}{103.395920pt}}
\pgflineto{\pgfpoint{152.701706pt}{103.395920pt}}
\pgfusepath{stroke}
\pgfpathmoveto{\pgfpoint{155.701706pt}{100.395920pt}}
\pgflineto{\pgfpoint{155.701706pt}{106.395927pt}}
\pgfusepath{stroke}
\pgfpathmoveto{\pgfpoint{158.003036pt}{119.001534pt}}
\pgflineto{\pgfpoint{152.003036pt}{119.001534pt}}
\pgfusepath{stroke}
\pgfpathmoveto{\pgfpoint{155.003036pt}{116.001534pt}}
\pgflineto{\pgfpoint{155.003036pt}{122.001534pt}}
\pgfusepath{stroke}
\pgfpathmoveto{\pgfpoint{198.435318pt}{121.251968pt}}
\pgflineto{\pgfpoint{192.435318pt}{121.251968pt}}
\pgfusepath{stroke}
\pgfpathmoveto{\pgfpoint{195.435318pt}{118.251968pt}}
\pgflineto{\pgfpoint{195.435318pt}{124.251968pt}}
\pgfusepath{stroke}
\pgfpathmoveto{\pgfpoint{200.801666pt}{114.776733pt}}
\pgflineto{\pgfpoint{194.801666pt}{114.776733pt}}
\pgfusepath{stroke}
\pgfpathmoveto{\pgfpoint{197.801666pt}{111.776733pt}}
\pgflineto{\pgfpoint{197.801666pt}{117.776733pt}}
\pgfusepath{stroke}
\pgfpathmoveto{\pgfpoint{190.116608pt}{115.944687pt}}
\pgflineto{\pgfpoint{184.116608pt}{115.944687pt}}
\pgfusepath{stroke}
\pgfpathmoveto{\pgfpoint{187.116608pt}{112.944687pt}}
\pgflineto{\pgfpoint{187.116608pt}{118.944687pt}}
\pgfusepath{stroke}
\pgfpathmoveto{\pgfpoint{186.450836pt}{117.508553pt}}
\pgflineto{\pgfpoint{180.450836pt}{117.508553pt}}
\pgfusepath{stroke}
\pgfpathmoveto{\pgfpoint{183.450836pt}{114.508553pt}}
\pgflineto{\pgfpoint{183.450836pt}{120.508553pt}}
\pgfusepath{stroke}
\pgfpathmoveto{\pgfpoint{182.945129pt}{94.769684pt}}
\pgflineto{\pgfpoint{176.945129pt}{94.769684pt}}
\pgfusepath{stroke}
\pgfpathmoveto{\pgfpoint{179.945129pt}{91.769684pt}}
\pgflineto{\pgfpoint{179.945129pt}{97.769684pt}}
\pgfusepath{stroke}
\pgfpathmoveto{\pgfpoint{217.117340pt}{103.659065pt}}
\pgflineto{\pgfpoint{211.117340pt}{103.659065pt}}
\pgfusepath{stroke}
\pgfpathmoveto{\pgfpoint{214.117340pt}{100.659065pt}}
\pgflineto{\pgfpoint{214.117340pt}{106.659065pt}}
\pgfusepath{stroke}
\pgfpathmoveto{\pgfpoint{238.411362pt}{84.948723pt}}
\pgflineto{\pgfpoint{232.411346pt}{84.948723pt}}
\pgfusepath{stroke}
\pgfpathmoveto{\pgfpoint{235.411362pt}{81.948715pt}}
\pgflineto{\pgfpoint{235.411362pt}{87.948723pt}}
\pgfusepath{stroke}
\pgfpathmoveto{\pgfpoint{179.200348pt}{89.153976pt}}
\pgflineto{\pgfpoint{173.200348pt}{89.153976pt}}
\pgfusepath{stroke}
\pgfpathmoveto{\pgfpoint{176.200348pt}{86.153976pt}}
\pgflineto{\pgfpoint{176.200348pt}{92.153984pt}}
\pgfusepath{stroke}
\pgfpathmoveto{\pgfpoint{235.074829pt}{94.344696pt}}
\pgflineto{\pgfpoint{229.074829pt}{94.344696pt}}
\pgfusepath{stroke}
\pgfpathmoveto{\pgfpoint{232.074829pt}{91.344696pt}}
\pgflineto{\pgfpoint{232.074829pt}{97.344696pt}}
\pgfusepath{stroke}
\pgfpathmoveto{\pgfpoint{237.887268pt}{67.848312pt}}
\pgflineto{\pgfpoint{231.887253pt}{67.848312pt}}
\pgfusepath{stroke}
\pgfpathmoveto{\pgfpoint{234.887253pt}{64.848312pt}}
\pgflineto{\pgfpoint{234.887253pt}{70.848312pt}}
\pgfusepath{stroke}
\pgfpathmoveto{\pgfpoint{181.916992pt}{116.796455pt}}
\pgflineto{\pgfpoint{175.916992pt}{116.796455pt}}
\pgfusepath{stroke}
\pgfpathmoveto{\pgfpoint{178.916992pt}{113.796455pt}}
\pgflineto{\pgfpoint{178.916992pt}{119.796455pt}}
\pgfusepath{stroke}
\pgfpathmoveto{\pgfpoint{158.279373pt}{108.222610pt}}
\pgflineto{\pgfpoint{152.279373pt}{108.222610pt}}
\pgfusepath{stroke}
\pgfpathmoveto{\pgfpoint{155.279373pt}{105.222610pt}}
\pgflineto{\pgfpoint{155.279373pt}{111.222610pt}}
\pgfusepath{stroke}
\pgfpathmoveto{\pgfpoint{258.962585pt}{110.360901pt}}
\pgflineto{\pgfpoint{252.962585pt}{110.360901pt}}
\pgfusepath{stroke}
\pgfpathmoveto{\pgfpoint{255.962585pt}{107.360901pt}}
\pgflineto{\pgfpoint{255.962585pt}{113.360901pt}}
\pgfusepath{stroke}
\pgfpathmoveto{\pgfpoint{173.935364pt}{104.089211pt}}
\pgflineto{\pgfpoint{167.935364pt}{104.089211pt}}
\pgfusepath{stroke}
\pgfpathmoveto{\pgfpoint{170.935364pt}{101.089203pt}}
\pgflineto{\pgfpoint{170.935364pt}{107.089211pt}}
\pgfusepath{stroke}
\pgfpathmoveto{\pgfpoint{155.325623pt}{92.199280pt}}
\pgflineto{\pgfpoint{149.325623pt}{92.199280pt}}
\pgfusepath{stroke}
\pgfpathmoveto{\pgfpoint{152.325623pt}{89.199280pt}}
\pgflineto{\pgfpoint{152.325623pt}{95.199280pt}}
\pgfusepath{stroke}
\pgfpathmoveto{\pgfpoint{202.332703pt}{106.899551pt}}
\pgflineto{\pgfpoint{196.332703pt}{106.899551pt}}
\pgfusepath{stroke}
\pgfpathmoveto{\pgfpoint{199.332703pt}{103.899551pt}}
\pgflineto{\pgfpoint{199.332703pt}{109.899551pt}}
\pgfusepath{stroke}
\pgfpathmoveto{\pgfpoint{188.291809pt}{106.586739pt}}
\pgflineto{\pgfpoint{182.291809pt}{106.586739pt}}
\pgfusepath{stroke}
\pgfpathmoveto{\pgfpoint{185.291809pt}{103.586739pt}}
\pgflineto{\pgfpoint{185.291809pt}{109.586739pt}}
\pgfusepath{stroke}
\pgfpathmoveto{\pgfpoint{170.988876pt}{117.481316pt}}
\pgflineto{\pgfpoint{164.988876pt}{117.481316pt}}
\pgfusepath{stroke}
\pgfpathmoveto{\pgfpoint{167.988876pt}{114.481316pt}}
\pgflineto{\pgfpoint{167.988876pt}{120.481316pt}}
\pgfusepath{stroke}
\pgfpathmoveto{\pgfpoint{165.308563pt}{111.498383pt}}
\pgflineto{\pgfpoint{159.308563pt}{111.498383pt}}
\pgfusepath{stroke}
\pgfpathmoveto{\pgfpoint{162.308563pt}{108.498383pt}}
\pgflineto{\pgfpoint{162.308563pt}{114.498383pt}}
\pgfusepath{stroke}
\pgfpathmoveto{\pgfpoint{196.641235pt}{106.520065pt}}
\pgflineto{\pgfpoint{190.641235pt}{106.520065pt}}
\pgfusepath{stroke}
\pgfpathmoveto{\pgfpoint{193.641235pt}{103.520065pt}}
\pgflineto{\pgfpoint{193.641235pt}{109.520065pt}}
\pgfusepath{stroke}
\pgfpathmoveto{\pgfpoint{198.887161pt}{112.126419pt}}
\pgflineto{\pgfpoint{192.887161pt}{112.126419pt}}
\pgfusepath{stroke}
\pgfpathmoveto{\pgfpoint{195.887161pt}{109.126419pt}}
\pgflineto{\pgfpoint{195.887161pt}{115.126419pt}}
\pgfusepath{stroke}
\pgfpathmoveto{\pgfpoint{190.139114pt}{118.018402pt}}
\pgflineto{\pgfpoint{184.139114pt}{118.018402pt}}
\pgfusepath{stroke}
\pgfpathmoveto{\pgfpoint{187.139114pt}{115.018394pt}}
\pgflineto{\pgfpoint{187.139114pt}{121.018402pt}}
\pgfusepath{stroke}
\pgfpathmoveto{\pgfpoint{180.520355pt}{106.133606pt}}
\pgflineto{\pgfpoint{174.520355pt}{106.133606pt}}
\pgfusepath{stroke}
\pgfpathmoveto{\pgfpoint{177.520355pt}{103.133606pt}}
\pgflineto{\pgfpoint{177.520355pt}{109.133606pt}}
\pgfusepath{stroke}
\pgfpathmoveto{\pgfpoint{166.703690pt}{93.097610pt}}
\pgflineto{\pgfpoint{160.703690pt}{93.097610pt}}
\pgfusepath{stroke}
\pgfpathmoveto{\pgfpoint{163.703690pt}{90.097610pt}}
\pgflineto{\pgfpoint{163.703690pt}{96.097618pt}}
\pgfusepath{stroke}
\pgfpathmoveto{\pgfpoint{172.526749pt}{104.289711pt}}
\pgflineto{\pgfpoint{166.526749pt}{104.289711pt}}
\pgfusepath{stroke}
\pgfpathmoveto{\pgfpoint{169.526749pt}{101.289703pt}}
\pgflineto{\pgfpoint{169.526749pt}{107.289711pt}}
\pgfusepath{stroke}
\pgfpathmoveto{\pgfpoint{187.477814pt}{117.596985pt}}
\pgflineto{\pgfpoint{181.477814pt}{117.596985pt}}
\pgfusepath{stroke}
\pgfpathmoveto{\pgfpoint{184.477814pt}{114.596985pt}}
\pgflineto{\pgfpoint{184.477814pt}{120.596985pt}}
\pgfusepath{stroke}
\pgfpathmoveto{\pgfpoint{181.620728pt}{109.305641pt}}
\pgflineto{\pgfpoint{175.620728pt}{109.305641pt}}
\pgfusepath{stroke}
\pgfpathmoveto{\pgfpoint{178.620728pt}{106.305641pt}}
\pgflineto{\pgfpoint{178.620728pt}{112.305641pt}}
\pgfusepath{stroke}
\pgfpathmoveto{\pgfpoint{200.743759pt}{116.577286pt}}
\pgflineto{\pgfpoint{194.743759pt}{116.577286pt}}
\pgfusepath{stroke}
\pgfpathmoveto{\pgfpoint{197.743759pt}{113.577286pt}}
\pgflineto{\pgfpoint{197.743759pt}{119.577286pt}}
\pgfusepath{stroke}
\pgfpathmoveto{\pgfpoint{171.920624pt}{108.920357pt}}
\pgflineto{\pgfpoint{165.920624pt}{108.920357pt}}
\pgfusepath{stroke}
\pgfpathmoveto{\pgfpoint{168.920624pt}{105.920357pt}}
\pgflineto{\pgfpoint{168.920624pt}{111.920364pt}}
\pgfusepath{stroke}
\pgfpathmoveto{\pgfpoint{135.909760pt}{73.728447pt}}
\pgflineto{\pgfpoint{129.909760pt}{73.728447pt}}
\pgfusepath{stroke}
\pgfpathmoveto{\pgfpoint{132.909760pt}{70.728447pt}}
\pgflineto{\pgfpoint{132.909760pt}{76.728447pt}}
\pgfusepath{stroke}
\pgfpathmoveto{\pgfpoint{120.742088pt}{87.429214pt}}
\pgflineto{\pgfpoint{114.742088pt}{87.429214pt}}
\pgfusepath{stroke}
\pgfpathmoveto{\pgfpoint{117.742088pt}{84.429214pt}}
\pgflineto{\pgfpoint{117.742088pt}{90.429214pt}}
\pgfusepath{stroke}
\pgfpathmoveto{\pgfpoint{184.925278pt}{115.256699pt}}
\pgflineto{\pgfpoint{178.925278pt}{115.256699pt}}
\pgfusepath{stroke}
\pgfpathmoveto{\pgfpoint{181.925278pt}{112.256699pt}}
\pgflineto{\pgfpoint{181.925278pt}{118.256699pt}}
\pgfusepath{stroke}
\color[rgb]{0.000000,0.500000,0.000000}
\pgfpathmoveto{\pgfpoint{184.936493pt}{137.227509pt}}
\pgflineto{\pgfpoint{178.936493pt}{143.227509pt}}
\pgfusepath{stroke}
\pgfpathmoveto{\pgfpoint{184.936493pt}{143.227509pt}}
\pgflineto{\pgfpoint{178.936493pt}{137.227509pt}}
\pgfusepath{stroke}
\pgfpathmoveto{\pgfpoint{117.511063pt}{203.316452pt}}
\pgflineto{\pgfpoint{111.511063pt}{209.316437pt}}
\pgfusepath{stroke}
\pgfpathmoveto{\pgfpoint{117.511063pt}{209.316437pt}}
\pgflineto{\pgfpoint{111.511063pt}{203.316452pt}}
\pgfusepath{stroke}
\pgfpathmoveto{\pgfpoint{188.206314pt}{125.640877pt}}
\pgflineto{\pgfpoint{182.206314pt}{131.640869pt}}
\pgfusepath{stroke}
\pgfpathmoveto{\pgfpoint{188.206314pt}{131.640869pt}}
\pgflineto{\pgfpoint{182.206314pt}{125.640877pt}}
\pgfusepath{stroke}
\pgfpathmoveto{\pgfpoint{111.670929pt}{135.618835pt}}
\pgflineto{\pgfpoint{105.670929pt}{141.618851pt}}
\pgfusepath{stroke}
\pgfpathmoveto{\pgfpoint{111.670929pt}{141.618851pt}}
\pgflineto{\pgfpoint{105.670929pt}{135.618835pt}}
\pgfusepath{stroke}
\pgfpathmoveto{\pgfpoint{133.251724pt}{139.476685pt}}
\pgflineto{\pgfpoint{127.251724pt}{145.476685pt}}
\pgfusepath{stroke}
\pgfpathmoveto{\pgfpoint{133.251724pt}{145.476685pt}}
\pgflineto{\pgfpoint{127.251724pt}{139.476685pt}}
\pgfusepath{stroke}
\pgfpathmoveto{\pgfpoint{66.771400pt}{212.555969pt}}
\pgflineto{\pgfpoint{60.771400pt}{218.555969pt}}
\pgfusepath{stroke}
\pgfpathmoveto{\pgfpoint{66.771400pt}{218.555969pt}}
\pgflineto{\pgfpoint{60.771400pt}{212.555969pt}}
\pgfusepath{stroke}
\pgfpathmoveto{\pgfpoint{185.577332pt}{146.044998pt}}
\pgflineto{\pgfpoint{179.577332pt}{152.044998pt}}
\pgfusepath{stroke}
\pgfpathmoveto{\pgfpoint{185.577332pt}{152.044998pt}}
\pgflineto{\pgfpoint{179.577332pt}{146.044998pt}}
\pgfusepath{stroke}
\pgfpathmoveto{\pgfpoint{163.424515pt}{129.375381pt}}
\pgflineto{\pgfpoint{157.424515pt}{135.375381pt}}
\pgfusepath{stroke}
\pgfpathmoveto{\pgfpoint{163.424515pt}{135.375381pt}}
\pgflineto{\pgfpoint{157.424515pt}{129.375381pt}}
\pgfusepath{stroke}
\pgfpathmoveto{\pgfpoint{164.779160pt}{174.718796pt}}
\pgflineto{\pgfpoint{158.779160pt}{180.718796pt}}
\pgfusepath{stroke}
\pgfpathmoveto{\pgfpoint{164.779160pt}{180.718796pt}}
\pgflineto{\pgfpoint{158.779160pt}{174.718796pt}}
\pgfusepath{stroke}
\pgfpathmoveto{\pgfpoint{131.375275pt}{150.253448pt}}
\pgflineto{\pgfpoint{125.375275pt}{156.253448pt}}
\pgfusepath{stroke}
\pgfpathmoveto{\pgfpoint{131.375275pt}{156.253448pt}}
\pgflineto{\pgfpoint{125.375275pt}{150.253448pt}}
\pgfusepath{stroke}
\pgfpathmoveto{\pgfpoint{64.678047pt}{165.434067pt}}
\pgflineto{\pgfpoint{58.678047pt}{171.434067pt}}
\pgfusepath{stroke}
\pgfpathmoveto{\pgfpoint{64.678047pt}{171.434067pt}}
\pgflineto{\pgfpoint{58.678047pt}{165.434067pt}}
\pgfusepath{stroke}
\pgfpathmoveto{\pgfpoint{168.128555pt}{128.687103pt}}
\pgflineto{\pgfpoint{162.128555pt}{134.687103pt}}
\pgfusepath{stroke}
\pgfpathmoveto{\pgfpoint{168.128555pt}{134.687103pt}}
\pgflineto{\pgfpoint{162.128555pt}{128.687103pt}}
\pgfusepath{stroke}
\pgfpathmoveto{\pgfpoint{205.642883pt}{147.756851pt}}
\pgflineto{\pgfpoint{199.642883pt}{153.756851pt}}
\pgfusepath{stroke}
\pgfpathmoveto{\pgfpoint{205.642883pt}{153.756851pt}}
\pgflineto{\pgfpoint{199.642883pt}{147.756851pt}}
\pgfusepath{stroke}
\pgfpathmoveto{\pgfpoint{156.901917pt}{135.317474pt}}
\pgflineto{\pgfpoint{150.901917pt}{141.317474pt}}
\pgfusepath{stroke}
\pgfpathmoveto{\pgfpoint{156.901917pt}{141.317474pt}}
\pgflineto{\pgfpoint{150.901917pt}{135.317474pt}}
\pgfusepath{stroke}
\pgfpathmoveto{\pgfpoint{178.512466pt}{125.103767pt}}
\pgflineto{\pgfpoint{172.512466pt}{131.103775pt}}
\pgfusepath{stroke}
\pgfpathmoveto{\pgfpoint{178.512466pt}{131.103775pt}}
\pgflineto{\pgfpoint{172.512466pt}{125.103767pt}}
\pgfusepath{stroke}
\pgfpathmoveto{\pgfpoint{182.324051pt}{109.428757pt}}
\pgflineto{\pgfpoint{176.324051pt}{115.428764pt}}
\pgfusepath{stroke}
\pgfpathmoveto{\pgfpoint{182.324051pt}{115.428764pt}}
\pgflineto{\pgfpoint{176.324051pt}{109.428757pt}}
\pgfusepath{stroke}
\pgfpathmoveto{\pgfpoint{183.976471pt}{163.479218pt}}
\pgflineto{\pgfpoint{177.976471pt}{169.479218pt}}
\pgfusepath{stroke}
\pgfpathmoveto{\pgfpoint{183.976471pt}{169.479218pt}}
\pgflineto{\pgfpoint{177.976471pt}{163.479218pt}}
\pgfusepath{stroke}
\pgfpathmoveto{\pgfpoint{186.273865pt}{123.065292pt}}
\pgflineto{\pgfpoint{180.273865pt}{129.065292pt}}
\pgfusepath{stroke}
\pgfpathmoveto{\pgfpoint{186.273865pt}{129.065292pt}}
\pgflineto{\pgfpoint{180.273865pt}{123.065292pt}}
\pgfusepath{stroke}
\pgfpathmoveto{\pgfpoint{158.661362pt}{148.370621pt}}
\pgflineto{\pgfpoint{152.661362pt}{154.370621pt}}
\pgfusepath{stroke}
\pgfpathmoveto{\pgfpoint{158.661362pt}{154.370621pt}}
\pgflineto{\pgfpoint{152.661362pt}{148.370621pt}}
\pgfusepath{stroke}
\pgfpathmoveto{\pgfpoint{137.910004pt}{126.041534pt}}
\pgflineto{\pgfpoint{131.910004pt}{132.041534pt}}
\pgfusepath{stroke}
\pgfpathmoveto{\pgfpoint{137.910004pt}{132.041534pt}}
\pgflineto{\pgfpoint{131.910004pt}{126.041534pt}}
\pgfusepath{stroke}
\pgfpathmoveto{\pgfpoint{175.946411pt}{127.328346pt}}
\pgflineto{\pgfpoint{169.946411pt}{133.328354pt}}
\pgfusepath{stroke}
\pgfpathmoveto{\pgfpoint{175.946411pt}{133.328354pt}}
\pgflineto{\pgfpoint{169.946411pt}{127.328346pt}}
\pgfusepath{stroke}
\pgfpathmoveto{\pgfpoint{172.102951pt}{131.214188pt}}
\pgflineto{\pgfpoint{166.102951pt}{137.214188pt}}
\pgfusepath{stroke}
\pgfpathmoveto{\pgfpoint{172.102951pt}{137.214188pt}}
\pgflineto{\pgfpoint{166.102951pt}{131.214188pt}}
\pgfusepath{stroke}
\pgfpathmoveto{\pgfpoint{154.296158pt}{144.527710pt}}
\pgflineto{\pgfpoint{148.296158pt}{150.527710pt}}
\pgfusepath{stroke}
\pgfpathmoveto{\pgfpoint{154.296158pt}{150.527710pt}}
\pgflineto{\pgfpoint{148.296158pt}{144.527710pt}}
\pgfusepath{stroke}
\pgfpathmoveto{\pgfpoint{159.834061pt}{132.893829pt}}
\pgflineto{\pgfpoint{153.834061pt}{138.893829pt}}
\pgfusepath{stroke}
\pgfpathmoveto{\pgfpoint{159.834061pt}{138.893829pt}}
\pgflineto{\pgfpoint{153.834061pt}{132.893829pt}}
\pgfusepath{stroke}
\pgfpathmoveto{\pgfpoint{177.900040pt}{134.565033pt}}
\pgflineto{\pgfpoint{171.900040pt}{140.565033pt}}
\pgfusepath{stroke}
\pgfpathmoveto{\pgfpoint{177.900040pt}{140.565033pt}}
\pgflineto{\pgfpoint{171.900040pt}{134.565033pt}}
\pgfusepath{stroke}
\pgfpathmoveto{\pgfpoint{158.810394pt}{146.112701pt}}
\pgflineto{\pgfpoint{152.810394pt}{152.112701pt}}
\pgfusepath{stroke}
\pgfpathmoveto{\pgfpoint{158.810394pt}{152.112701pt}}
\pgflineto{\pgfpoint{152.810394pt}{146.112701pt}}
\pgfusepath{stroke}
\pgfpathmoveto{\pgfpoint{165.751907pt}{148.133469pt}}
\pgflineto{\pgfpoint{159.751907pt}{154.133469pt}}
\pgfusepath{stroke}
\pgfpathmoveto{\pgfpoint{165.751907pt}{154.133469pt}}
\pgflineto{\pgfpoint{159.751907pt}{148.133469pt}}
\pgfusepath{stroke}
\pgfpathmoveto{\pgfpoint{188.722488pt}{128.864990pt}}
\pgflineto{\pgfpoint{182.722488pt}{134.864990pt}}
\pgfusepath{stroke}
\pgfpathmoveto{\pgfpoint{188.722488pt}{134.864990pt}}
\pgflineto{\pgfpoint{182.722488pt}{128.864990pt}}
\pgfusepath{stroke}
\pgfpathmoveto{\pgfpoint{163.783356pt}{136.138657pt}}
\pgflineto{\pgfpoint{157.783356pt}{142.138657pt}}
\pgfusepath{stroke}
\pgfpathmoveto{\pgfpoint{163.783356pt}{142.138657pt}}
\pgflineto{\pgfpoint{157.783356pt}{136.138657pt}}
\pgfusepath{stroke}
\pgfpathmoveto{\pgfpoint{156.204987pt}{139.583450pt}}
\pgflineto{\pgfpoint{150.204987pt}{145.583450pt}}
\pgfusepath{stroke}
\pgfpathmoveto{\pgfpoint{156.204987pt}{145.583450pt}}
\pgflineto{\pgfpoint{150.204987pt}{139.583450pt}}
\pgfusepath{stroke}
\pgfpathmoveto{\pgfpoint{108.090050pt}{167.233994pt}}
\pgflineto{\pgfpoint{102.090042pt}{173.233994pt}}
\pgfusepath{stroke}
\pgfpathmoveto{\pgfpoint{108.090050pt}{173.233994pt}}
\pgflineto{\pgfpoint{102.090042pt}{167.233994pt}}
\pgfusepath{stroke}
\pgfpathmoveto{\pgfpoint{103.367920pt}{194.184723pt}}
\pgflineto{\pgfpoint{97.367920pt}{200.184723pt}}
\pgfusepath{stroke}
\pgfpathmoveto{\pgfpoint{103.367920pt}{200.184723pt}}
\pgflineto{\pgfpoint{97.367920pt}{194.184723pt}}
\pgfusepath{stroke}
\pgfpathmoveto{\pgfpoint{158.083984pt}{140.284790pt}}
\pgflineto{\pgfpoint{152.083984pt}{146.284790pt}}
\pgfusepath{stroke}
\pgfpathmoveto{\pgfpoint{158.083984pt}{146.284790pt}}
\pgflineto{\pgfpoint{152.083984pt}{140.284790pt}}
\pgfusepath{stroke}
\pgfpathmoveto{\pgfpoint{165.972244pt}{160.728363pt}}
\pgflineto{\pgfpoint{159.972244pt}{166.728363pt}}
\pgfusepath{stroke}
\pgfpathmoveto{\pgfpoint{165.972244pt}{166.728363pt}}
\pgflineto{\pgfpoint{159.972244pt}{160.728363pt}}
\pgfusepath{stroke}
\pgfpathmoveto{\pgfpoint{173.461487pt}{128.505020pt}}
\pgflineto{\pgfpoint{167.461487pt}{134.505020pt}}
\pgfusepath{stroke}
\pgfpathmoveto{\pgfpoint{173.461487pt}{134.505020pt}}
\pgflineto{\pgfpoint{167.461487pt}{128.505020pt}}
\pgfusepath{stroke}
\pgfpathmoveto{\pgfpoint{169.788940pt}{130.577332pt}}
\pgflineto{\pgfpoint{163.788940pt}{136.577332pt}}
\pgfusepath{stroke}
\pgfpathmoveto{\pgfpoint{169.788940pt}{136.577332pt}}
\pgflineto{\pgfpoint{163.788940pt}{130.577332pt}}
\pgfusepath{stroke}
\pgfpathmoveto{\pgfpoint{182.993423pt}{124.435440pt}}
\pgflineto{\pgfpoint{176.993423pt}{130.435440pt}}
\pgfusepath{stroke}
\pgfpathmoveto{\pgfpoint{182.993423pt}{130.435440pt}}
\pgflineto{\pgfpoint{176.993423pt}{124.435440pt}}
\pgfusepath{stroke}
\pgfpathmoveto{\pgfpoint{162.180969pt}{139.215729pt}}
\pgflineto{\pgfpoint{156.180969pt}{145.215729pt}}
\pgfusepath{stroke}
\pgfpathmoveto{\pgfpoint{162.180969pt}{145.215729pt}}
\pgflineto{\pgfpoint{156.180969pt}{139.215729pt}}
\pgfusepath{stroke}
\pgfpathmoveto{\pgfpoint{170.219406pt}{131.559265pt}}
\pgflineto{\pgfpoint{164.219406pt}{137.559265pt}}
\pgfusepath{stroke}
\pgfpathmoveto{\pgfpoint{170.219406pt}{137.559265pt}}
\pgflineto{\pgfpoint{164.219406pt}{131.559265pt}}
\pgfusepath{stroke}
\pgfpathmoveto{\pgfpoint{167.697754pt}{131.977951pt}}
\pgflineto{\pgfpoint{161.697754pt}{137.977951pt}}
\pgfusepath{stroke}
\pgfpathmoveto{\pgfpoint{167.697754pt}{137.977951pt}}
\pgflineto{\pgfpoint{161.697754pt}{131.977951pt}}
\pgfusepath{stroke}
\pgfpathmoveto{\pgfpoint{131.257980pt}{152.243408pt}}
\pgflineto{\pgfpoint{125.257980pt}{158.243423pt}}
\pgfusepath{stroke}
\pgfpathmoveto{\pgfpoint{131.257980pt}{158.243423pt}}
\pgflineto{\pgfpoint{125.257980pt}{152.243408pt}}
\pgfusepath{stroke}
\pgfpathmoveto{\pgfpoint{184.625519pt}{119.561653pt}}
\pgflineto{\pgfpoint{178.625519pt}{125.561653pt}}
\pgfusepath{stroke}
\pgfpathmoveto{\pgfpoint{184.625519pt}{125.561653pt}}
\pgflineto{\pgfpoint{178.625519pt}{119.561653pt}}
\pgfusepath{stroke}
\pgfpathmoveto{\pgfpoint{177.656387pt}{108.470901pt}}
\pgflineto{\pgfpoint{171.656387pt}{114.470901pt}}
\pgfusepath{stroke}
\pgfpathmoveto{\pgfpoint{177.656387pt}{114.470901pt}}
\pgflineto{\pgfpoint{171.656387pt}{108.470901pt}}
\pgfusepath{stroke}
\pgfpathmoveto{\pgfpoint{167.816437pt}{119.507477pt}}
\pgflineto{\pgfpoint{161.816437pt}{125.507477pt}}
\pgfusepath{stroke}
\pgfpathmoveto{\pgfpoint{167.816437pt}{125.507477pt}}
\pgflineto{\pgfpoint{161.816437pt}{119.507477pt}}
\pgfusepath{stroke}
\pgfpathmoveto{\pgfpoint{165.924042pt}{123.079765pt}}
\pgflineto{\pgfpoint{159.924042pt}{129.079773pt}}
\pgfusepath{stroke}
\pgfpathmoveto{\pgfpoint{165.924042pt}{129.079773pt}}
\pgflineto{\pgfpoint{159.924042pt}{123.079765pt}}
\pgfusepath{stroke}
\pgfpathmoveto{\pgfpoint{182.292389pt}{126.585098pt}}
\pgflineto{\pgfpoint{176.292389pt}{132.585098pt}}
\pgfusepath{stroke}
\pgfpathmoveto{\pgfpoint{182.292389pt}{132.585098pt}}
\pgflineto{\pgfpoint{176.292389pt}{126.585098pt}}
\pgfusepath{stroke}
\pgfpathmoveto{\pgfpoint{97.353935pt}{165.778976pt}}
\pgflineto{\pgfpoint{91.353928pt}{171.778976pt}}
\pgfusepath{stroke}
\pgfpathmoveto{\pgfpoint{97.353935pt}{171.778976pt}}
\pgflineto{\pgfpoint{91.353928pt}{165.778976pt}}
\pgfusepath{stroke}
\pgfpathmoveto{\pgfpoint{148.885895pt}{141.106918pt}}
\pgflineto{\pgfpoint{142.885895pt}{147.106918pt}}
\pgfusepath{stroke}
\pgfpathmoveto{\pgfpoint{148.885895pt}{147.106918pt}}
\pgflineto{\pgfpoint{142.885895pt}{141.106918pt}}
\pgfusepath{stroke}
\pgfpathmoveto{\pgfpoint{118.961487pt}{144.851547pt}}
\pgflineto{\pgfpoint{112.961487pt}{150.851547pt}}
\pgfusepath{stroke}
\pgfpathmoveto{\pgfpoint{118.961487pt}{150.851547pt}}
\pgflineto{\pgfpoint{112.961487pt}{144.851547pt}}
\pgfusepath{stroke}
\pgfpathmoveto{\pgfpoint{177.716965pt}{129.267120pt}}
\pgflineto{\pgfpoint{171.716965pt}{135.267120pt}}
\pgfusepath{stroke}
\pgfpathmoveto{\pgfpoint{177.716965pt}{135.267120pt}}
\pgflineto{\pgfpoint{171.716965pt}{129.267120pt}}
\pgfusepath{stroke}
\pgfpathmoveto{\pgfpoint{170.739990pt}{134.313614pt}}
\pgflineto{\pgfpoint{164.739990pt}{140.313614pt}}
\pgfusepath{stroke}
\pgfpathmoveto{\pgfpoint{170.739990pt}{140.313614pt}}
\pgflineto{\pgfpoint{164.739990pt}{134.313614pt}}
\pgfusepath{stroke}
\pgfpathmoveto{\pgfpoint{175.963943pt}{158.481049pt}}
\pgflineto{\pgfpoint{169.963943pt}{164.481049pt}}
\pgfusepath{stroke}
\pgfpathmoveto{\pgfpoint{175.963943pt}{164.481049pt}}
\pgflineto{\pgfpoint{169.963943pt}{158.481049pt}}
\pgfusepath{stroke}
\pgfpathmoveto{\pgfpoint{147.413361pt}{133.222565pt}}
\pgflineto{\pgfpoint{141.413361pt}{139.222565pt}}
\pgfusepath{stroke}
\pgfpathmoveto{\pgfpoint{147.413361pt}{139.222565pt}}
\pgflineto{\pgfpoint{141.413361pt}{133.222565pt}}
\pgfusepath{stroke}
\pgfpathmoveto{\pgfpoint{79.925964pt}{149.746185pt}}
\pgflineto{\pgfpoint{73.925964pt}{155.746185pt}}
\pgfusepath{stroke}
\pgfpathmoveto{\pgfpoint{79.925964pt}{155.746185pt}}
\pgflineto{\pgfpoint{73.925964pt}{149.746185pt}}
\pgfusepath{stroke}
\pgfpathmoveto{\pgfpoint{73.745819pt}{168.351334pt}}
\pgflineto{\pgfpoint{67.745819pt}{174.351334pt}}
\pgfusepath{stroke}
\pgfpathmoveto{\pgfpoint{73.745819pt}{174.351334pt}}
\pgflineto{\pgfpoint{67.745819pt}{168.351334pt}}
\pgfusepath{stroke}
\pgfpathmoveto{\pgfpoint{172.578934pt}{127.022873pt}}
\pgflineto{\pgfpoint{166.578934pt}{133.022873pt}}
\pgfusepath{stroke}
\pgfpathmoveto{\pgfpoint{172.578934pt}{133.022873pt}}
\pgflineto{\pgfpoint{166.578934pt}{127.022873pt}}
\pgfusepath{stroke}
\pgfpathmoveto{\pgfpoint{165.846512pt}{137.737061pt}}
\pgflineto{\pgfpoint{159.846512pt}{143.737061pt}}
\pgfusepath{stroke}
\pgfpathmoveto{\pgfpoint{165.846512pt}{143.737061pt}}
\pgflineto{\pgfpoint{159.846512pt}{137.737061pt}}
\pgfusepath{stroke}
\pgfpathmoveto{\pgfpoint{150.903778pt}{157.222382pt}}
\pgflineto{\pgfpoint{144.903778pt}{163.222382pt}}
\pgfusepath{stroke}
\pgfpathmoveto{\pgfpoint{150.903778pt}{163.222382pt}}
\pgflineto{\pgfpoint{144.903778pt}{157.222382pt}}
\pgfusepath{stroke}
\pgfpathmoveto{\pgfpoint{111.097565pt}{150.775970pt}}
\pgflineto{\pgfpoint{105.097565pt}{156.775970pt}}
\pgfusepath{stroke}
\pgfpathmoveto{\pgfpoint{111.097565pt}{156.775970pt}}
\pgflineto{\pgfpoint{105.097565pt}{150.775970pt}}
\pgfusepath{stroke}
\pgfpathmoveto{\pgfpoint{141.830536pt}{141.786835pt}}
\pgflineto{\pgfpoint{135.830536pt}{147.786835pt}}
\pgfusepath{stroke}
\pgfpathmoveto{\pgfpoint{141.830536pt}{147.786835pt}}
\pgflineto{\pgfpoint{135.830536pt}{141.786835pt}}
\pgfusepath{stroke}
\color[rgb]{1.000000,0.000000,0.000000}
\pgfpathmoveto{\pgfpoint{203.971069pt}{127.326637pt}}
\pgflineto{\pgfpoint{204.544022pt}{129.089996pt}}
\pgfusepath{stroke}
\pgfpathmoveto{\pgfpoint{202.471069pt}{126.236824pt}}
\pgflineto{\pgfpoint{203.971069pt}{127.326637pt}}
\pgfusepath{stroke}
\pgfpathmoveto{\pgfpoint{200.616974pt}{126.236824pt}}
\pgflineto{\pgfpoint{202.471069pt}{126.236824pt}}
\pgfusepath{stroke}
\pgfpathmoveto{\pgfpoint{199.116974pt}{127.326637pt}}
\pgflineto{\pgfpoint{200.616974pt}{126.236824pt}}
\pgfusepath{stroke}
\pgfpathmoveto{\pgfpoint{198.544022pt}{129.089996pt}}
\pgflineto{\pgfpoint{199.116974pt}{127.326637pt}}
\pgfusepath{stroke}
\pgfpathmoveto{\pgfpoint{199.116974pt}{130.853348pt}}
\pgflineto{\pgfpoint{198.544022pt}{129.089996pt}}
\pgfusepath{stroke}
\pgfpathmoveto{\pgfpoint{200.616974pt}{131.943161pt}}
\pgflineto{\pgfpoint{199.116974pt}{130.853348pt}}
\pgfusepath{stroke}
\pgfpathmoveto{\pgfpoint{202.471069pt}{131.943161pt}}
\pgflineto{\pgfpoint{200.616974pt}{131.943161pt}}
\pgfusepath{stroke}
\pgfpathmoveto{\pgfpoint{203.971069pt}{130.853348pt}}
\pgflineto{\pgfpoint{202.471069pt}{131.943161pt}}
\pgfusepath{stroke}
\pgfpathmoveto{\pgfpoint{204.544022pt}{129.089996pt}}
\pgflineto{\pgfpoint{203.971069pt}{130.853348pt}}
\pgfusepath{stroke}
\color[rgb]{0.000000,0.000000,0.000000}
\pgfsetdash{{16pt}{0pt}}{0pt}
\pgfpathmoveto{\pgfpoint{288.074158pt}{197.039612pt}}
\pgflineto{\pgfpoint{260.624542pt}{197.039612pt}}
\pgfusepath{stroke}
\pgfpathmoveto{\pgfpoint{288.074158pt}{220.474182pt}}
\pgflineto{\pgfpoint{260.624542pt}{220.474182pt}}
\pgfusepath{stroke}
\pgfpathmoveto{\pgfpoint{260.624542pt}{220.474182pt}}
\pgflineto{\pgfpoint{260.624542pt}{197.039612pt}}
\pgfusepath{stroke}
\pgfpathmoveto{\pgfpoint{288.074158pt}{220.474182pt}}
\pgflineto{\pgfpoint{288.074158pt}{197.039612pt}}
\pgfusepath{stroke}
{
\pgftransformshift{\pgfpoint{275.119781pt}{216.568420pt}}
\pgfnode{rectangle}{west}{\fontsize{10}{0}\selectfont\textcolor[rgb]{0,0,0}{{BUT}}}{}{\pgfusepath{discard}}}
{
\pgftransformshift{\pgfpoint{275.119781pt}{208.756897pt}}
\pgfnode{rectangle}{west}{\fontsize{10}{0}\selectfont\textcolor[rgb]{0,0,0}{{VVJ}}}{}{\pgfusepath{discard}}}
{
\pgftransformshift{\pgfpoint{275.119781pt}{200.945374pt}}
\pgfnode{rectangle}{west}{\fontsize{10}{0}\selectfont\textcolor[rgb]{0,0,0}{{?}}}{}{\pgfusepath{discard}}}
\color[rgb]{0.000000,0.000000,1.000000}
\pgfsetdash{}{0pt}
\pgfpathmoveto{\pgfpoint{270.872192pt}{216.568420pt}}
\pgflineto{\pgfpoint{264.872192pt}{216.568420pt}}
\pgfusepath{stroke}
\pgfpathmoveto{\pgfpoint{267.872192pt}{213.568420pt}}
\pgflineto{\pgfpoint{267.872192pt}{219.568420pt}}
\pgfusepath{stroke}
\color[rgb]{0.000000,0.500000,0.000000}
\pgfpathmoveto{\pgfpoint{270.872192pt}{205.756897pt}}
\pgflineto{\pgfpoint{264.872192pt}{211.756897pt}}
\pgfusepath{stroke}
\pgfpathmoveto{\pgfpoint{270.872192pt}{211.756897pt}}
\pgflineto{\pgfpoint{264.872192pt}{205.756897pt}}
\pgfusepath{stroke}
\color[rgb]{1.000000,0.000000,0.000000}
\pgfpathmoveto{\pgfpoint{270.299255pt}{199.182007pt}}
\pgflineto{\pgfpoint{270.872192pt}{200.945374pt}}
\pgfusepath{stroke}
\pgfpathmoveto{\pgfpoint{268.799255pt}{198.092194pt}}
\pgflineto{\pgfpoint{270.299255pt}{199.182007pt}}
\pgfusepath{stroke}
\pgfpathmoveto{\pgfpoint{266.945160pt}{198.092194pt}}
\pgflineto{\pgfpoint{268.799255pt}{198.092194pt}}
\pgfusepath{stroke}
\pgfpathmoveto{\pgfpoint{265.445129pt}{199.182007pt}}
\pgflineto{\pgfpoint{266.945160pt}{198.092194pt}}
\pgfusepath{stroke}
\pgfpathmoveto{\pgfpoint{264.872192pt}{200.945374pt}}
\pgflineto{\pgfpoint{265.445129pt}{199.182007pt}}
\pgfusepath{stroke}
\pgfpathmoveto{\pgfpoint{265.445129pt}{202.708710pt}}
\pgflineto{\pgfpoint{264.872192pt}{200.945374pt}}
\pgfusepath{stroke}
\pgfpathmoveto{\pgfpoint{266.945160pt}{203.798523pt}}
\pgflineto{\pgfpoint{265.445129pt}{202.708710pt}}
\pgfusepath{stroke}
\pgfpathmoveto{\pgfpoint{268.799255pt}{203.798523pt}}
\pgflineto{\pgfpoint{266.945160pt}{203.798523pt}}
\pgfusepath{stroke}
\pgfpathmoveto{\pgfpoint{270.299255pt}{202.708710pt}}
\pgflineto{\pgfpoint{268.799255pt}{203.798523pt}}
\pgfusepath{stroke}
\pgfpathmoveto{\pgfpoint{270.872192pt}{200.945374pt}}
\pgflineto{\pgfpoint{270.299255pt}{202.708710pt}}
\pgfusepath{stroke}
\end{pgfpicture}

  \only<presentation>{
    \only<2>{\Large \alert{What if it a \textbf{completely different strain}?}}
  }
\end{frame}

\begin{frame}
  \frametitle{Hands on with Python console}
  

\end{frame}

\begin{frame}
  \frametitle{What have we learned}
  \begin{block}{Machine Learning background}
    \begin{itemize}
    \item Many \alert{technical} problem types.
    \item \alert{Scientific} applications.
    \item \alert{Societal} issues.
    \item Machine learning as science.
    \end{itemize}
  \end{block}

  \begin{block}{Methodology}
    \begin{itemize}
    \item \alert{Classification} Problems: \only<3->{$k$-nearest neighbour algorithm.}
    \item Elementary \alert{variable selection}.
    \item \alert{Verifying} your \alert{conclusions}: \only<3->{Validation/Test sets.}
    \item Verifying your \alert{assumptions}: \only<4->{(How?)}
    \end{itemize}
  \end{block}
\end{frame}

\section{Decision problems}

\only<article>{
All machine learning problems are essentially decision problems. This essentially means replacing some human decisions with machine decisions. One of the simplest decision problems is classification, where you want an algorithm to decide the correct class of some data, but even within this simple framework there is a multitude of decisions to be made. The first is how to frame the classification problem the first place. The second is how to collect, process and annotate the data. The third is choosing the type of classification model to use. The fourth is how to use the collected data to find an optimal classifier within the selected type. After all this has been done, there is the problem of classifying new data. In this course, we will take a holistic view of the problem, and consider each problem in turn, starting from the lowest level and working our way up.}

\section{Hierarchies of decision making problems}
\begin{frame}
  \tableofcontents[currentsection]
\end{frame}
\subsection{Simple decision problems}
\begin{frame}
  \frametitle{Preferences}

  \begin{example}
    \begin{block}{Food}
      \begin{itemize}
      \item[A] McDonald's cheeseburger
        \item[B] Surstromming
        \item[C] Oatmeal
        \end{itemize}
      \end{block}
      \begin{block}{Money}
        \begin{itemize}
        \item[A] 10,000,000 SEK
        \item[B] 10,000,000 USD
        \item[C] 10,000,000 BTC
        \end{itemize}
      \end{block}
      \begin{block}{Entertainment}
        \begin{itemize}
        \item[A] Ticket to Liseberg
        \item[B] Ticket to Rebstar
        \item[C] Ticket to Nutcracker
        \end{itemize}
      \end{block}
  \end{example}

  \only<article>{The simplest decision problem involves selecting one item from a set of choices.}  
  \begin{itemize}
  \item Each choice is called a \alert{reward} $r \in \CR$.
  \item There is a \alert{utility function} $U : \CR \to \Reals$, assigning values to reward.
  \item We (weakly) prefer $A$ to $B$ iff $U(A) \geq U(B)$.
  \end{itemize}

  \begin{exercise}
    From your individual preferences, derive a \alert{common utility function} that reflects everybody's preferences in the class.
  \end{exercise}
\end{frame}


\begin{frame}
  \frametitle{Uncertain rewards}
  \only<article>{However, in real life, there are many cases where we can only choose between uncertain outcomes. The simplest example are lottery tickets, where rewards are essentially random. However, in many cases the rewards are not really random, but simply uncertain. In those cases it is useful to represent our uncertainty with probabilities as well, even though there is nothing really random.}
  \begin{example}%ro: rather an exercise?
    You are going to work, and it might rain.
    What do you do?
    \begin{itemize}
    \item $\decision_1$: Take the umbrella.
    \item $\decision_2$: Risk it!
    \item $\outcome_1$: rain
    \item $\outcome_2$: dry
    \end{itemize}
    \begin{table}
      \centering
      \begin{tabular}{c|c|c}
        $\Rew(\outcome,\decision)$ & $\decision_1$ & $\decision_2$ \\ %ro: U has only one argument.
        \hline
        $\outcome_1$ & dry, carrying umbrella & wet\\
        $\outcome_2$ & dry, carrying umbrella & dry\\
        \hline
        \hline
        $U[\Rew(\outcome,\decision)]$ & $\decision_1$ & $\decision_2$ \\
        \hline
        $\outcome_1$ & 0 & -10\\
        $\outcome_2$ & 0 & 1
      \end{tabular}
      \caption{Rewards and utilities.}
      \label{tab:rain-utility-function}
    \end{table}

    \begin{itemize}
    \item<2-> $\max_\decision \min_\outcome U = 0$
    \item<3-> $\min_\outcome \max_\decision  U = 0$
    \end{itemize}
  \end{example}
\end{frame}



\begin{frame}
  \frametitle{Expected utility}
  \[
    \E (U \mid a) = \sum_r U[\Rew(\outcome, \decision)] \Pr(\outcome \mid \decision)
  \]
  \begin{example}%ro: rather an exercise?
    You are going to work, and it might rain. The forecast said that
    the probability of rain $(\outcome_1)$ was $20\%$. What do you do?
    \begin{itemize}
    \item $\decision_1$: Take the umbrella.
    \item $\decision_2$: Risk it!
    \end{itemize}
    \begin{table}
      \centering
      \begin{tabular}{c|c|c}
        $\Rew(\outcome,\decision)$ & $\decision_1$ & $\decision_2$ \\ %ro: U has only one argument.
        \hline
        $\outcome_1$ & dry, carrying umbrella & wet\\
        $\outcome_2$ & dry, carrying umbrella & dry\\
        \hline
        \hline
        $U[\Rew(\outcome,\decision)]$ & $\decision_1$ & $\decision_2$ \\
        \hline
        $\outcome_1$ & 0 & -10\\
        $\outcome_2$ & 0 & 1\\
        \hline
        \hline
        $\E_P(U \mid \decision)$ & 0 &  -1.2 \\ 
      \end{tabular}
      \caption{Rewards, utilities, expected utility for $20\%$ probability of rain.}
      \label{tab:rain-utility-function}
    \end{table}
  \end{example}
\end{frame}

\begin{frame}
  \frametitle{Preferences among random outcomes}
  \begin{example}
    Would you rather \ldots
    \begin{itemize}
    \item[A] Have 100 EUR now?
    \item[B] Flip a coin, and get 200 EUR if it comes heads?
    \end{itemize}    
  \end{example}
  \uncover<2->{
    \begin{block}{The expected utility hypothesis}
      Rational decision makers prefer choice $A$ to $B$ if
      \[
        \E(U | A) \geq \E(U | B),
      \]
      where the expected utility is
      \[
        \E(U | A) = \sum_r U(r) \Pr(r | A).
      \]
    \end{block}
    In the above example, $r \in \{0, 100, 200\}$ and $U(r)$ is
    increasing, and the coin is fair.
  }
  \begin{itemize}
  \item<3-> If $U$ is convex, we prefer B.
  \item<4-> If $U$ is concave, we prefer A.
  \item<5-> If $U$ is linear, we don't care.
  \end{itemize}
\end{frame}




\subsection{Decision rules}
\only<presentation>{
  \begin{frame}
    \tableofcontents[currentsection,currentsubsection]
  \end{frame}
}
\only<article>{We now move from simple decisions to decisions that
  depend on some observation. This is most easily embodied through the
  problem of classification. }
\begin{frame}
  \frametitle{Deciding a class given a model}
  \only<article>{In the simplest classification problem, we observe some features $x_t$ and want to make a guess $\decision_t$ about the true class label $y_t$. Assuming we have some probabilistic model $P(y_t \mid x_t)$, we want to define a decision rule $\pol(\decision_t \mid x_t)$ that is optimal, in the sense that it maximises expected utility for $P$.}
  \begin{itemize}
  \item Features $x_t \in \CX$.
  \item Label $y_t \in \CY$.
  \item Decisions $\decision_t \in \CA$.
  \item Decision rule $\pol(\decision_t \mid x_t)$ assigns probabilities to actions.
  \end{itemize}
  
  \begin{block}{Standard classification problem}
    \only<article>{In the simplest case, the set of decisions we make are the same as the set of classes}
    \[
    \CA = \CY, \qquad
    U(\decision, y) = \ind{\decision = y}
    \]
  \end{block}

  \begin{exercise}
    If we have a model $P(y_t \mid x_t)$, and a suitable $U$, what is the optimal decision to make?
  \end{exercise}
  \only<presentation>{
    \uncover<2->{
      \[
      \decision_t \in \argmax_{\decision \in \Decisions} \sum_y P(y_t = y \mid x_t) \Util(\decision, y)
      \]
    }
    \uncover<3>{
      For standard classification,
      \[
      \decision_t \in \argmax_{\decision \in \Decisions} P(y_t = \decision \mid x_t)
      \]
    }
  }
\end{frame}


\begin{frame}
  \frametitle{Deciding the class given a model family}
  \begin{itemize}
  \item Training data $S = (x_1, y_1, \ldots, x_n, y_n)$
  \item Models $\cset{P_\outcome}{\outcome \in \Outcome}$
  \item Prior $\bel$ on $\Outcome$.
  \end{itemize}
  \[
    \bel(\outcome \mid S)
    = \frac{P_\outcome(y_1, \ldots, y_n \mid x_1, \ldots, x_n) \bel(\outcome)}
    {\sum_{\outcome' \in \Outcome} P_{\outcome'}(y_1, \ldots, y_n \mid x_1, \ldots, x_n) \bel(\outcome')}
  \]
  We can then calculate the posterior marginal marginal label probability
  \[
    P_{\bel \mid S}(y_t \mid x_t) \defn
    P_{\bel}(y_t \mid x_t, S) = 
    \sum_{\outcome \in \Outcome} P_\outcome(y_t \mid x_t) \bel(\omega \mid S).
  \]
  We can then construct the following simple decision rule:
  \[
    \decision_t \in \argmax_{\decision \in \CY}\sum_{\outcome \in \Outcome} P_\outcome(y_t \mid x_t) \bel(\omega \mid S),
  \]
  otherwise known as the \alert{Bayes rule}.
\end{frame}

\section{Bayes decisions}
\begin{frame}
  \frametitle{Bayes decision rules}
  Consider the case where outcomes are independent of decisions:
  \[
    \Util (P, \decision) \defn \sum_{\outcome}  \Util (\outcome, \decision) P(\outcome)
  \]
  This corresponds e.g. to the case where $P(\omega)$ is the belief about an unknown world.
  \begin{definition}[Bayes utility]
    \label{def:bayes-utility}
    The maximising decision for $P$ has an expected utility equal to:
    \begin{equation}
      \BUtil(P) \defn \sup_{\decision \in \Decisions} \Util (P, \decision).
      \label{eq:bayes-utility}
    \end{equation}
  \end{definition}
\end{frame}


\only<article>{
  One of the simplest decision problems is classification. At the simplest level, this is the problem of observing some data point $x_t \in \CX$ and making a decision about what class $\CY$ it belongs to. Typically, a fixed classifier is defined as a decision rule $\pi(a | x)$ making decisions $a \in \CA$, where the decision space includes the class labels, so that if we observe some point $x_t$ and choose $a_t = 1$, we essentially declare that $y_t = 1$.

  Typically, we wish to have a classification policy that minimises classification error.
}
\begin{frame}
  \begin{definition}[Classification error of a fixed decision rule]
    
  \end{definition}
\end{frame}


\bibliographystyle{apalike}
\bibliography{../../bib/mine,../../bib/misc}

\end{document}

%%% Local Variables: 
%%% mode: latex
%%% TeX-master: t
%%% End: 


\only<article>{ Consider a researcher wishing to collect data for a
  statistical analysis. As long as the analysis is eventually
  published,\footnote{If somebody knows that the analysis is being
    conducted, however, they could still learn something private from
    the fact that the analysis has /emph{not} been published.} this
  creates two types of possible privacy violations. The first is
  direct exposure of sensitive data through publication of the
  analysis, if for example the study is about something such as drug
  use. The second is through publication of ``anonymised'' versions of
  the dataset, which create opportunities for linkage attacks.}

\section{Privacy in databases}
\begin{frame}
  \frametitle{Anonymisation}
  \only<article>{If we wish to publish a database, frequently we need to protect identities of people involved. The simplest method for doing that is simply erasing directly identifying information. However, this does not really work most of the time, especially since attackers can have side-information that can reveal the identities of individuals in the original data.}
  
  \begin{example}[Typical relational database in Tinder]
    \begin{table}[H]
      \begin{tabular}{l|l|l|l|l|l|l}
        Birthday & Name & Height  & Weight & Age & Postcode & Profession\\
        \hline
        06/07 & \only<1>{Mike Pence} & 190 & 80 & 60-70 & 1001 & Politician\\
        06/14 & \only<1>{Donald Trump} & 185 & 110 & 70+ & 1001 & Rentier\\
        01/01 & \only<1>{A. B. Student} & 170 & 70 & 40-60 & 6732 & Time Traveller
      \end{tabular} 
    \end{table}
  \end{example}

  \only<2>{The simple act of hiding or using random identifiers is called anonymisation.}
  \only<article>{However this is generally insufficient as other identifying information may be used to re-identify individuals in the data.}
\end{frame}


\begin{frame}
  \frametitle{Record linkage}
  \only<article>{In particular, anonymisation is not enough as record linkage can allow you to still extract information using data from another database through \emph{quasi-identifiers}.}

  \only<1>{
    \def\firstcircle{(0,0) circle (7em)}
    \def\secondcircle{(3,0) circle (7em)}
    
    \begin{figure}[H]
      \centering
      \begin{tikzpicture}

        \draw \firstcircle node[text width=7em] {Ethnicity\newline
          Date\newline Diagnosis \newline Procedure \newline
          Medication \newline Charge }; \draw \secondcircle node [text
        width=2em, align=right] {Name \newline Address \newline
          Registration \newline Party \newline Lastvote};
        \begin{scope}
          \clip \firstcircle; \fill[red] \secondcircle;
        \end{scope}
        \node [text width=4em] (QI) at (1.5, 0) {Postcode \newline
          Birthdate \newline Sex}; \node [text width=4em] (qi-text) at
        (1.5, -3) {Quasi-identifiers}; \path[->]<1-> (qi-text) edge
        [bend left] (QI);
        % Now we want to highlight the intersection of the first and
        % the
        % second circle:


      \end{tikzpicture}
      
      \caption{An example of two datasets, one containing sensitive and the other public information. The two datasets can be linked and individuals identified through the use of quasi-identifiers.}
      \label{fig:quasi-identifiers}
    \end{figure}
  }
  
  \begin{example}[Typical relational database in a tax office]
    \begin{table}[H]
      \begin{tabular}{l|l|l|l|l|l|l}
        ID & Name &  Salary & Deposits & Age & Postcode & Profession\\
        \hline
        1959060783 & Mike Pence & 150,000 & 1e6 & 60 & 1001 & Politician\\
        1946061408 & Donald Trump & 300,000 & -1e9 & 72 & 1001 & Rentier\\
        2100010101 & A. B. Student & 10,000 & 100,000 & 40 & 6732 & Time Traveller
      \end{tabular}
    \end{table}
  \end{example}
  
  \begin{example}[Typical relational database in Tinder]
    \begin{table}[H]
      \begin{tabular}{l|l|l|l|l|l|l}
        Birthday & Name & Height  & Weight & Age & Postcode & Profession\\
        \hline
        06/07 & & 190 & 80 & 60-70 & 1001 & Politician\\
        06/14 &  & 185 & 110 & 70+ & 1001 & Rentier\\
        01/01 &  & 170 & 70 & 40-60 & 6732 & Time Traveller
      \end{tabular}
    \end{table}
  \end{example}
\end{frame}

\section{$k$-anonymity}

\begin{frame}
  \frametitle{$k$-anonymity}
  \begin{figure}[H]
    \centering \subfigure[Samarati]{\includegraphics[width=0.25\textwidth]{../figures/samarati}}
    \subfigure[Sweeney]{\includegraphics[width=0.25\textwidth]{../figures/sweeney}}
  \end{figure}
  \only<article>{The concept of $k$-anonymity was introduced by~\citet{samarati1998protecting} and provides good guarantees when accessing a single database}

  \begin{definition}[$k$-anonymity]
    A database provides $k$-anonymity if for every person in the database is indistinguishable from $k-1$ persons with respect to \emph{quasi-identifiers}.
  \end{definition}
  \alert{It's the analyst's job to define quasi-identifiers}
  
\end{frame}

\begin{frame}
  \only<1>{
    \begin{table}[H]
      \begin{tabular}{l|l|l|l|l|l|l}
        Birthday & Name & Height  & Weight & Age & Postcode & Profession\\
        \hline
        06/07 & Mike Pence & 190 & 80 & 60+ & 1001 & Politician\\
        06/14 & Donald Trump & 185 & 110 & 60+ & 1001 & Rentier\\
        06/12 & John Bolton & 170 & 60 & 60+ & 1243 & Politician\\
        01/01 & A. B. Student & 170 & 70 & 40-60 & 6732 & Time Traveller\\
        05/08 & Li Yang & 175 & 72 & 30-40 & 6910 & Time Traveller
      \end{tabular}
      \caption{1-anonymity.}
    \end{table}

  }
  \only<presentation>{
    \only<2>{
      \begin{tabular}{l|l|l|l|l|l|l}
        Birthday & Name & Height  & Weight & Age & Postcode & Profession\\
        \hline
        06/07 &  & 190 & 80 & 60+ & 1001 & Politician\\
        06/14 &  & 185 & 110 & 60+ & 1001 & Rentier\\
        06/12 &  & 170 & 60 & 60+ & 1243 & Politician\\
        01/01 &  & 170 & 70 & 40-60 & 6732 & Time Traveller\\
        05/08 &  & 175 & 72 & 30-40 & 6910 & Policeman
      \end{tabular}
      1-anonymity
    }

    \only<3>{
      \begin{tabular}{l|l|l|l|l|l|l}
        Birthday & Name & Height  & Weight & Age & Postcode & Profession\\
        \hline
        06/07 &  & 180-190 & 80+ & 60+ & 1* & \\
        06/14 &  & 180-190 & 80+ & 60+ & 1* &\\
        06/12 &  & 170-180 & 60-80 & 60+ & 1* & \\
        01/01 &  & 170-180 & 60-80 & 20-60 & 6* &\\
        05/08 &  & 170-180 & 60-80 & 20-60 & 6* & 
      \end{tabular}
      1-anonymity
    }
  }
  \only<4>{
    \begin{table}[H]
      \begin{tabular}{l|l|l|l|l|l|l}
        Birthday & Name & Height  & Weight & Age & Postcode & Profession\\
        \hline
                 &  & 180-190 & 80+ & 60+ & 1* & \\
                 &  & 180-190 & 80+ & 60+ & 1* &\\
                 &  & 170-180 & 60-80 & 60+ & 1* & \\
                 &  & 170-180 & 60-80 & 20-60 & 6* &\\
                 &  & 170-180 & 60-80 & 20-60 & 6* & 
      \end{tabular}
      \caption{2-anonymity: the database can be partitioned in sets of at least 2 records}
    \end{table}
  }

  \only<article>{However, with enough information, somebody may still be able to infer something about the indivduals}
\end{frame}





\section{Differential privacy}
\only<article>{While $k$-anonymity can protect against specific re-identification attacks when used with care, it says little about what to do when the adversary has a lot of power. For example, if the  adversary knows the data of everybody that has participated in the database,  it is trivial for them to infer what our own data is. Differential privacy offers protection against adversaries with unlimited side-information or computational power. Informally, an algorithmic computation is differentially-private if an adversary cannot distinguish two similar database based on the result of the computation. While the notion of similarity is for the analyst to defined, a common is to say that two databases are similar when they are identical apart from the data of one person.}

\begin{frame}
  \begin{figure}[H]
    \begin{tikzpicture}
      \node[label=left:$x$] at (0,0) (data) {\includegraphics[width=0.2\columnwidth]{../figures/medical}};

      \node[label=$x_1$] at (-2,3)(patient1) {\includegraphics[width=0.1\columnwidth]{../figures/me-recent}};
      \uncover<3->{
        \node[label=$x_2$] at (2,3) (patient2) {\includegraphics[width=0.2\columnwidth]{../figures/judge}};
      }
      \uncover<4->{
        \node[label=$a$] at (4,0)   (statistics) {\includegraphics[width=0.3\columnwidth]{../figures/coronary-disease}};
      }
      \uncover<2->{
        \draw[->] (patient1) -- (data);
      }
      \uncover<3->{
        \draw[->] (patient2) -- (data);
      }
      \uncover<4->{
        \draw[->] (data) -- node[above]{$\pol$} (statistics);
      }
      \uncover<5->{
        \draw[line width=5, red, ->] (statistics) -- (patient2);
      }
    \end{tikzpicture}
    \caption{If two people contribute their data $x = (x_1, x_2)$ to a medical database, and an algorithm $\pol$ computes some public output $a$ from $x$, then it should be hard infer anything about the data from the public output.}
  \end{figure}

\end{frame}

\begin{frame}
  \frametitle{Privacy desiderata}
  \only<article>{
    Consider a scenario where $n$ persons give their data $x_1, \ldots, x_n$ to an analyst. This analyst then performs some calculation $f(x)$ on the data and published the result. The following properties are desirable from a general standpoint.

    \paragraph{Anonymity.} Individual participation in the study remains a secret. From the release of the calculations results, nobody can significantly increase their probability of identifying an individual in the database.

    \paragraph{Secrecy.} The data of individuals is not revealed. The release does not significantly increase the probability of inferring individual's information $x_i$.

    \paragraph{Side-information.} Even if an adversary has arbitrary side-information, he cannot use that to amplify the amount of knowledge he would have obtained from the release.

    \paragraph{Utility.} The released result has, with high probability, only a small error relative to a calculation that does not attempt to safeguard privacy.
  }
  \only<presentation>{
    We wish to calculate something on some private data and publish a \alert{privacy-preserving}, but \alert{useful}, version of the result.
    \begin{itemize}
    \item Anonymity: Individual participation remains hidden.
    \item Secrecy: Individual data $x_i$ is not revealed.
    \item Side-information: Linkage attacks are not possible.
    \item Utility: The calculation remains useful.
    \end{itemize}
  }
\end{frame}

\begin{frame}
  \frametitle{Example: The prevalence of drug use in sport}
  
  \only<article>{
    Let's say you need to perform a statistical analysis of the drug-use habits of athletes. Obviously, even if you promise the athlete not to reveal their information, you still might not convince them. Yet, you'd like them to be truthful. The trick is to allow them to randomly change their answers, so that you can't be \emph{sure} if they take drugs, no matter what they answer.
  }

  \only<presentation>{
    \begin{itemize}
    \item $n$ athletes
    \item Ask whether they have doped in the past year.
    \item Aim: calculate \% of doping.
    \item How can we get truthful / accurate results?
    \end{itemize}
  }
  \only<2>{
    \begin{block}{Algorithm for randomising responses about drug use}
      \begin{enumerate}
      \item Flip a coin.
      \item If it comes heads, respond \texttt{Yes}.
      \item Otherwise, respond truthfully.
      \end{enumerate}
    \end{block}

    If the rate of positive responses is $p$, everybody follows the protocol, and the coin is fair, what is the true rate $q$ of drug use?
  }
  \only<presentation>{
    \uncover<3>{
      \[
      p = 1/2 + q/2 \Rightarrow q = 1/2
      \]
    }
  }
  \only<article>{The problem with this approach, of course, is that we are effectively throwing away half of our data. In particular, if we repeated the experiment with a coin that came heads at a rate $\epsilon$, then our error bounds would scale as $O(1/\sqrt{\epsilon n})$ for $n$ data points.}
\end{frame}

\begin{frame}
  \frametitle{The randomised response mechanism}
  \only<article>{The above idea can be generalised. Consider we have data $x_1, \ldots, x_n$ from $n$ users and we transform it randomly to $y_1, \ldots, y_n$ using the following mapping.}
  \begin{definition}[Randomised response]
    The $i$-th user, whose data is $x_i \in \{0,1\}$ , responds with $a_i \in \{0, 1\}$ with probability
    \[
    \pol(a_i = j \mid x_i = k) = p,  \qquad  \pol(a_i = k \mid x_i = k) = 1 - p,
    \]
    where $j \neq k$.
  \end{definition}

  \uncover<2->{Given the complete data $x$, the mechanism's output is $a = (a_1, \ldots, a_n)$.}
  \uncover<3->{Since the algorithm independently calculates a new value for each data entry, the output is
    \[
    \pol(a \mid x) = \prod_i \pol(a_i \mid x_i)
    \]
  }

  \only<article>{This mechanism satisfies so-called $\epsilon$-differential privacy, which we will define later.}

\end{frame}

\begin{frame}
  \frametitle{The local privacy model}
  \begin{figure}[H]
    \centering
    \begin{tikzpicture}
      \node[RV] at (0,0) (x1) {$x_1$};
      \node[RV] at (0,1) (x2) {$x_2$};
      \node[RV] at (0,2) (xn) {$x_n$};
      \node[select] at (1,-1) (pol) {$\pol$};
      \node[RV] at (2,0) (a1) {$a_1$};
      \node[RV] at (2,1) (a2) {$a_2$};
      \node[RV] at (2,2) (an) {$a_n$};
      \draw[->] (x1) -- (a1);
      \draw[->] (x2) -- (a2);
      \draw[->] (xn) -- (an);
      \draw[->] (pol) -- (a1);
      \draw[->] (pol) -- (a2);
      \draw[->] (pol) -- (an);
    \end{tikzpicture}
    
    \caption{The local privacy model}
    \label{fig:local-privacy}
  \end{figure}
\end{frame}

\begin{frame}
  \frametitle{Differential privacy.}
  \includegraphics[width=0.2\textwidth]{../figures/dwork} \hspace{1em}
  \includegraphics[width=0.2\textwidth]{../figures/mcsherry} \hspace{1em}
  \includegraphics[width=0.2\textwidth]{../figures/nissim} \hspace{1em}
  \includegraphics[width=0.2\textwidth]{../figures/smith}
  \only<article>{Now let us take a look at a way to characterise the  the inherent privacy properties of algorithms. This is called differential privacy, and it can be seen as a bound on the information an adversary with arbitrary power or side-information could extract from a computation.}
  
  \begin{definition}[$\epsilon$-Differential Privacy]
    A stochastic algorithm $\pol : \CX \to \CA$, where $\CX$ is endowed with a neighbourhood relation $N$, is said to be $\epsilon$-differentially private if
    \begin{equation}
      \label{eq:epsilon-dp}
      \left|\ln \frac{\pol(a \mid x)}{\pol(a \mid x')}\right| \leq \epsilon , \qquad \forall x N x'.
    \end{equation}
  \end{definition}
  
  \only<article>{Typically, algorithms are applied to datasets $x = (x_1, \ldots, x_n)$ composed of the data of $n$ individuals. Thus, all privacy guarantees relate to the data contributed by these individuals. 

    In this context, two datasets are usually called neighbouring if $x = (x_1, \ldots, x_{i-1}, x_i, x_{i+1} x_n)$ and 
    $x' = (x_1, \ldots, x_{i-1}, x_{i+1} x_n)$, i.e. if one dataset is missing an element.
    
    A slightly weaker definition of neighbourhood is to say that $x N x'$ if $x' = (x_1, \ldots, x_{i-1}, x'_i, x_{i+1} x_n)$, i.e. if one dataset has an altered element.

  }
\end{frame}

\begin{frame}
  \only<article>{
    \begin{remark}
      Any differentially private algorithm must be stochastic.
    \end{remark}

    To prove that this is necessary, consider the example of counting how many people take drugs in a competition. If the adversary only doesn't know whether you in particular take drugs, but knows whether everybody else takes drugs, it's trivial to discover your own drug habits by looking at the total. This is because in this case, $f(x) = \sum_i x_i$ and the adversary knows $x_i$ for all $i \neq j$. Then, by observing $f(x)$, he can recover $x_j = f(x) - \sum_{i \neq j} x_i$. Consequently, it is not possible to protect against adversaries with arbitrary side information without stochasticity.}
  \begin{remark}
    The randomised response mechanism with $p \leq 1/2$ is $\ln \frac{1 - p}{p}$-DP.
  \end{remark}
  \begin{proof}
    Consider $x = (x_1, \ldots, x_i,  \ldots, x_n)$, $x' = (x_1, \ldots, x'_i,  \ldots, x_n)$. Then
    \begin{align*}
      \pol(a \mid x)
      \uncover<2->{&= \prod_i \pol(a_i \mid x_i)}
      \uncover<3->{\\ &= \pol(a_j \mid x_j') \prod_{i \neq j} \pol(a_i \mid x_i) }
      \uncover<4->{\\ &\leq \frac{p}{1 - p} \pol(a_j \mid x_j) \prod_{i \neq j} \pol(a_i \mid x_i) }
      \uncover<5>{\\ &= \frac{1-p}{p} \pol(a \mid x')}
    \end{align*}
    \only<4>{$\pol(a_j = k\mid x_j = k) = 1 - p$ so the ratio is $\max\{(1-p)/p, p/(1 - p)\} \leq (1 - p)/p$ for $p \leq 1/2$.}
  \end{proof}
\end{frame}

\begin{frame}
  \begin{figure}[H]
    \centering
    \begin{tikzpicture}
      \node[rectangle] at (0,0) (python) {Python program};
      \node[rectangle] at (8,0) (database) {Database System};
      \draw[->, >=latex, blue!20!white, line width=15pt, bend right]   (python) to node[black]{Query $q$} (database) ;
      \draw[->, >=latex, blue!20!white, line width=15pt, bend right]   (database) to node[black]{Private response $a$} (python) ;
    \end{tikzpicture}
    \label{fig:database-access}
    \caption{Private database access model}
  \end{figure}
  \begin{block}{Response policy}
    The  policy defines a distribution over responses $a$ given the data $x$ and the query $q$.
    \[
    \pol(a \mid x, q)
    \]
  \end{block}
\end{frame}

\begin{frame}
  \frametitle{Differentially private queries}
  \begin{block}{The \texttt{DP-SELECT} statement}
    \begin{itemize}
    \item \texttt{DP-SELECT column1, column2 FROM table;}
      \only<article>{This selects only some columns from the table}
    \item \texttt{DP-SELECT * FROM table;}
      \only<article>{This selects all the columns from the table}
    \end{itemize}
  \end{block}

  \begin{block}{Selecting rows}
    \texttt{SELECT * FROM table WHERE column = value;}
  \end{block}

  \begin{exampleblock}{Arithmetic queries}
    \only<article>{Here are some example SQL statements}
    \begin{itemize}
    \item  \texttt{DP-SELECT COUNT(column) FROM table WHERE condition;}
      \only<article>{This allows you to count the number of rows matching \texttt{condition}}
    \item  \texttt{DP-SELECT AVG(column) FROM table WHERE condition;}
      \only<article>{This lets you to count the number of rows matching \texttt{condition}}
    \item  \texttt{DP-SELECT SUM(column) FROM table WHERE condition;}
      \only<article>{This is used to sum up the values in a column.}
    \end{itemize}
  \end{exampleblock}

  \begin{alertblock}{Cumulative privacy loss}
    Depending on the DP scheme, each query answered may leak privacy.
    \only<article>{In particular, if we always respond with an $\epsilon$-DP mechanism, after $T$ queries our privacy guarantee is $T \epsilon$. There exist mechanisms that do not respond to each query independently, which can bound the total privacy loss.}
  \end{alertblock}
\end{frame}

\begin{frame}
  \frametitle{The Laplace mechanism.}
  \only<article>{
    A simple method to obtain a differentially private algorithm from a deterministic function $f : \CX \to \Reals$, is to use additive noise, so that the output of the algorithm is simply 
    \[
    a = f(x) + \omega, \qquad \omega \sim \Laplace.
    \]
    The amount of noise added, together with the smoothness of the function $f$, determine the amount of privacy we have.
  }
  \begin{definition}[The Laplace mechanism]
    For any function $f : \CX \to \Reals$, 
    \begin{equation}
      \label{eq:laplace-mechanism}
      \pol(a \mid x) = \Laplace(f(x), \lambda),
    \end{equation}
    where the Laplace density is defined as
    \[
    p(\omega \mid \mu, \lambda) = \frac{1}{2 \lambda} \exp\left(-\frac{|\omega - \mu|}{\lambda}\right).
    \]
  \end{definition}
  \only<article>{Here, $\Laplace(\mu, \lambda)$ is the density $f(x) = \frac{\lambda}{2} \exp(-\lambda |x - \mu|)$}.
\end{frame}

\begin{frame}
  \begin{example}[Calculating the average salary]
    \begin{itemize}
    \item The $i$-th person receives salary $x_i$
    \item We wish to calculate the average salary in a private manner.
    \end{itemize}
  \end{example}
  \begin{block}{Local privacy model}
    \begin{itemize}
    \item Obtain $y_i = x_i + \omega$, where $\omega \sim \Laplace(\lambda)$.
    \item Return $a = n^{-1} \sum_{i=1}^n y_i$.
    \end{itemize}
  \end{block}
  \begin{block}{Centralised privacy model}
     Return $a = n^{-1} \sum_{i=1}^n x_i + \omega$, where $\omega \sim \Laplace(\lambda')$.
  \end{block}
  
  How should we add noise in order to guarantee privacy?
\end{frame}


\begin{frame}
  \frametitle{DP properties of the Laplace mechanism}
  \begin{definition}[Sensitivity]
    The sensitivity of a function $f$ is
    \[
    \sensitivity{f} \defn \sup_{x N x'} |f(x) - f(x')|
    \]
    \only<article>{
      If we define a metric $d$, so that $d(x, x') = 1$ for $x N x'$, then:
      \[
       |f(x) - f(x')| \leq \sensitivity{f} d(x, x'),
      \]
      i.e. $f$ is $\sensivity{f}$-Lipschitz with respect to $d$.
    }
  \end{definition}
  \begin{theorem}
    The Laplace mechanism on a function $f$ ran with $\Laplace(\lambda)$ is $\sensitivity{f} / \lambda$-DP.
  \end{theorem}
  \begin{proof}
    \begin{align*}
      \frac{\pol(a \mid x)}{\pol(a \mid x')}
      &=
      \frac{e^{|a - f(x')|/\lambda}}{e^{|a - f(x)|/\lambda}}
      \leq
      \frac{e^{|a - f(x)|/\lambda + \sensitivity{f}/\lambda}}{e^{|a - f(x)|/\lambda}}
        = e^{\sensitivity{f} / \lambda}
    \end{align*}
  \end{proof}
\end{frame}

\begin{frame}
  Here let is assume $x_i \in [0, M]$ for all $i$.
  \begin{block}{Laplace in the local privacy model}
    The sensitivity of the individual data is $M$, so to obtain $\epsilon$-DP we need to use $\lambda = M / \epsilon$.

    The variance of $a$ is $M / \epsilon \sqrt{n}$.
  \end{block}
  \begin{block}{Laplace in the centralised privacy model}
    The sensitivity of $f$ is $M / n$, so we need to use $\lambda = M / n\epsilon$.

    The variance of $a$ is $M / \epsilon n$.
  \end{block}
\end{frame}

\begin{frame}
  \frametitle{The Exponential Mechanism.}
  \only<article>{
    Here we assume that we can answer queries $q$, whereby each possible answer $a$ to the query has a different utility to the DM: $\util(q, a, x)$.
    Let $\sensitivity{\util(q)} \defn \sup_{x N x'} |\util(q, a, x) -\util(q, a, x)|$ denote the sensitivity of a query. Then the following mechanism is $\epsilon$-differentially private.
  }
  \begin{definition}[The Exponential mechanism]
    For any utility function $\util : \CQ \times \CA \times \CX \to \Reals$, define the policy
    \begin{equation}
      \label{eq:exponential-mechanism}
      \pol(a \mid x) \defn \frac{e^{\epsilon \util(q, a, x) / \sensitivity{ \util(q)}}}{\sum_{a'} e^{\epsilon \util(q, a', x) / \sensitivity{\util(q)}}}
    \end{equation}
  \end{definition}
  \only<article>{
    Clearly, when $\epsilon \to 0$, this mechanism is uniformly random. When $\epsilon \to \infty$ the action maximising $\util(q,a,x)$ is always chosen.
  }
\end{frame}



%%% Local Variables:
%%% mode: latex
%%% TeX-engine: xetex
%%% TeX-master: "notes"
%%% End:

 %data bases

\chapter{Fairness}
\label{ch:fairness}
\only<article>{
  When machine learning algorithms are applied at scale, it can be difficult to imagine what their effects might be. In this part of the course, we consider notions of fairness as seen through the prism of conditional independence and meritocracy. The first notion requires that we look deeper into directed graphical models.
}
\section{Graphical models}
\only<article>{
  Graphical models are a very useful tool for modelling the relationship between multiple variables. The simplest such models, probabilistic graphical models (otherwise known as Bayesian networks) involve directed acyclic graphs between random variables.}
\begin{frame}
  \frametitle{Graphical models}
  \begin{figure}[H]
    \centering
    \begin{tikzpicture}
      \node[RV] at (2,0) (xi) {$x_3$};
      \node[RV] at (0,0) (xB) {$x_1$};
      \node[RV] at (1,1) (xD) {$x_2$};
      \draw[->] (xB) to (xD);
      \draw[->] (xD) to (xi);
      \draw[->] (xB) to (xi);
    \end{tikzpicture}
    \label{fig:bn}
    \caption{Graphical model for three variables.}
  \end{figure}
  \only<article>{Consider for example the model in Figure~\ref{fig:bn}. It involves three variables, $x_1, x_2, x_3$ and there are three arrows, which show how one variable depends on another. Simply put, if you think of each $x_k$ as a stochastic function, then $x_k$'s value only depends on the values of its parents, i.e. the nodes that are point to it. In this example, $x_1$ does not depend on any other variable, but the value of $x_2$ depends on the value of $x_1$. Such models are useful when we want to describe the joint probability distribution of all the variables in the collection.}
  \begin{block}{Joint probability}
    Let $\bx = (x_1, \ldots, x_n)$. Then $\bx : \Omega \to X$, $X = \prod_i X_i$ and:
    \[
    \Pr(\bx \in A) = P(\cset{\omega \in \Omega}{\bx(\omega) \in A}).
    \]
  \end{block}
  \only<article>{
    When $X_i$ are finite, we can typically write
    \[
    \Pr(\bx = \ba) = P(\cset{\omega \in \Omega}{\bx(\omega) = \ba}),
    \]
    for the probability that $x_i = a_i$ for all $i \in [n]$.
  }
  \begin{block}{Factorisation}
    \only<article>{
      For any subsets $B \subset [n]$ and its complement $C$ so that
      $\bx_B = (x_i)_{i \in B}$,     $\bx_C = (x_i)_{i \notin B}$
    }
    \only<1>{
      \[
      \Pr(\bx) = \Pr(\bx_B \mid \bx_C) \Pr(\bx_C)
      \only<presentation>{,\qquad B, C \subset [n]}
      \]
    }
    \uncover<2->{
      So we can write any joint distribution as
      \[
      \Pr(x_1) \Pr(x_2 \mid x_1) \Pr(x_3 \mid x_1, x_2) \cdots \Pr(x_n \mid x_1, \ldots, x_{n-1}).
      \]
    }
  \end{block}
  \only<article>{Although the above factorisation is always possible to do, sometimes our graphical model has a structure that makes the factors much simpler. In fact, the main reason for introducing graphical models is to represent dependencies between variables. For a given model, we can infer whether some variables are in fact dependent, independent, or conditionally independent.}
\end{frame}
\begin{frame}
  \frametitle{Directed graphical models and conditional independence}
  \begin{figure}[H]
    \centering
    \begin{tikzpicture}
      \node[RV] at (2,0) (xi) {$x_3$};
      \node[RV] at (0,0) (xB) {$x_1$};
      \node[RV] at (1,1) (xD) {$x_2$};
      \draw[->] (xB)--(xD);
      \draw[->] (xD)--(xi);
    \end{tikzpicture}
    \label{fig:bn}
    \caption{Graphical model for the factorisation $\Pr(x_1 \mid x_2) \Pr(x_2 \mid x_3) \Pr(x_3)$.}
  \end{figure}
  \begin{block}{Conditional independence}
    We say $x_i$ is conditionally independent of $\bx_B$ given $\bx_D$ and write $x_i \mid \bx_D \indep \bx_B$ iff
    \[
    \Pr(x_i, \bx_B \mid \bx_D)
    =
    \Pr(x_i \mid \bx_D)
    \Pr(\bx_D \mid \bx_B).
    \]
  \end{block}

  \frametitle{Directed graphical models}
  \only<article>{
    A graphical model is a convenient way to represent conditional independence between variables. There are many variants of graphical models, whose name is context dependent. Other names used in the literature are probabilistic graphical models, Bayesian networks, causal graphs, or decision diagrams. In this set of notes we focus on directed graphical models that depict dependencies between ranom variables.

    \begin{definition}[Directed graphical model] A collection of $n$ random variables $x_i : \Omega \to X_i$, and let $X \defn \prod_i X_i$, with underlying probability measure $P$ on $\Omega$.
      Let $\bx = (x_i)_{i=1}^n$ and for any subset $B \subset[n]$ let
      \begin{align}
        \bx_B &\defn (x_i)_{i \in B}\\
        \bx_{-j} &\defn (x_i)_{i \neq i}
      \end{align}
    \end{definition}
  }
  \only<article>{In a Graphical model, conditional independence is represented through directed edges.}


\end{frame}

\begin{frame}
  \begin{example}[Smoking and lung cancer]
    \begin{figure}[H]
      \centering
      \begin{tikzpicture}
        \node[RV] at (0,0) (x1) {$S$};
        \node[RV] at (1,1) (x2) {$C$};
        \node[RV] at (2,0) (x3) {$A$};
        \draw[->] (x1)--(x2);
        \draw[->] (x3)--(x2);
      \end{tikzpicture}
      \caption{Smoking and lung cancer graphical model, where $S$: Smoking, $C$: cancer, $A$: asbestos exposure.}
    \end{figure}
    \only<article>{
      It has been found by~\citet{lee2001relation} that lung  incidence not only increases with both asbestos exposure and smoking. This is in agreement with the graphical model shown. The study actually found that there is an amplification effect, whereby smoking and asbestos exposure increases cancer risk by 28 times compared to non-smokers. This implies that the risk is not simply additive. The graphical model only tells us that there is a dependency, and does not describe the nature of this dependency precisely.}
  \end{example}
\end{frame}

\begin{frame}
  \begin{example}[Treatment effects]
    \begin{figure}[H]
      \centering
      \begin{tikzpicture}
        \node[RV] at (0,0) (x) {$x$};
        \node[RV] at (1,1) (y) {$y$};
        \node[RV] at (2,0) (a) {$a$};
        \draw[->] (x)--(y);
        \draw[->] (x)--(a);
        \draw[->] (a)--(y);
      \end{tikzpicture}
      \caption{Kidney treatment model, where $x$: severity, $y$: result, $a$: treatment applied}
    \end{figure}
    % \begin{table}[H]
    %   \begin{tabular}{l|r|r}
    %     & Treatment A  & Treatment B\\
    %     \hline
    %     Small stones & 87  & 270\\
    %     Large stones  & 263 &  80
    %   \end{tabular}
    %   \begin{tabular}{l|r|r}
    %     Severity & Treatment A  & Treatment B\\
    %     \hline
    %     Small stones ) & 93\%  & 87\%\\
    %     Large stones  & 73\% &  69\%\\
    %     \hline
    %     Average & 78\% & 83\%
    %   \end{tabular}
    %   X\end{table}
    \only<article>{
      A curious example is that of applying one of two treatments for kidneys. In the data, it is clear that one treatment is best for both large and small stones. However, when the data is aggregated}
  \end{example}
\end{frame}


% \begin{frame}
%   \begin{figure}[H]
%     \centering
%     \subfigure[Conditionally independent case]{
%     \begin{tikzpicture}
%       \node[RV] at (0,0) (z) {$z$};
%       \node[RV] at (1,1) (s) {$s$};
%       \node[RV] at (2,0) (a) {$a$};
%       \draw[->] (z)--(s);
%       \draw[->] (z)--(s);
%       \draw[->] (s)--(a);
%     \end{tikzpicture}
%   }
%     \subfigure[Dependent case]{
%     \begin{tikzpicture}
%       \node[RV] at (0,0) (z) {$z$};
%       \node[RV] at (1,1) (s) {$s$};
%       \node[RV] at (2,0) (a) {$a$};
%       \draw[->] (z)--(s);
%       \draw[->] (z)--(s);
%       \draw[->] (s)--(a);
%       \draw[->] (z)--(a);
%     \end{tikzpicture}
%   }
%     \caption{School admission graphical model, where $z$: gender, $s$: school applied to, $a$: whether you were admitted. }
%   \end{figure}
%     %   \begin{table}[H]
%     %     \begin{tabular}{l|r|r}
              %     %               School & Male  & Female
                                                     %     %             \end{tabular}
                                                     %     %                                                      \end{table}
                                                     %                                                      \begin{example}[School admission]
                                                     %                                                      \end{example}
                                                     %                                                    \end{frame}


\input{graphical-model-exercises.tex}


\begin{frame}
  \begin{alertblock}{Deciding conditional independence}
    There is an algorithm for deciding conditional independence of any two variables in a graphical model.
    \only<article>{However, this is beyond the scope of these notes. Here, we shall just use these models as a way to encode dependencies that we assume exist.}
  \end{alertblock}

\end{frame}

\begin{frame}
  \frametitle{Measuring independence}
  \only<article>{The simplest way to measure independence is by looking at whether or not the distribution of the possibly dependent variable changes when we change the value of the other variables. }

  \begin{theorem}
    If $x_i \mid \bx_D \indep \bx_B$ then
    \[
    \Pr(x_i \mid \bx_B, \bx_D)
    =
    \Pr(x_i \mid \bx_D)
    \]
  \end{theorem}
  \uncover<2->{
    This implies
    \[
    \Pr(x_i \mid \bx_B, \bx_D)
    =
    \Pr(x_i \mid \bx'_B, \bx_D)
    \]
    so we can measure independence by seeing how the distribution of $x_i$ changes when we vary $\bx'_B$, keeping $\bx_D$ fixed.
  }
\end{frame}

%%% Local Variables:
%%% mode: latex
%%% TeX-engine: xetex
%%% TeX-master: "notes"
%%% End:

\chapter{Fairness.}
\label{ch:fairness}
\only<article>{ When machine learning algorithms are applied at scale,
  it can be difficult to imagine what their effects might be. In this
  part of the book, we consider notions of fairness as seen through
  the prism of conditional independence and meritocracy. The first
  notion requires that we look deeper into directed graphical models.
  
  The problem of fairness in machine learning and
  artificial intelligence has only recently been widely
  recognised. When any algorithm is implemented at scale, no matter
  the original objective and whether it is satisfied, it has
  significant societal effects. In particular, even when considering
  the narrow objective of the algorithm, even if it improves it
  overall, it may increase inequality.
  
  In this course we will look at two aspects of fairness. The first
  has to do with disadvantaged populations that form distinct social
  classes due to a shared income stratum, race or gender. The second
  has to do with meritocratic notions of fairness.

  This chapter requires some knowledge of decision theory (See
  Chapter~\ref{ch:decision-problems}) to understand the principles of
  expected utility maximisation and graphical models (See
  Chapter~\ref{ch:graphical-models}) to understand the idea of conditional independence.
}

\section{Introduction.}


\only<article>{ Fairness is a concept that has received much attention
  recently when applied to large-scale algorithmic decision
  making. However, the very concept of fairness is not well-defined
  and encompasses many different ideas. Some of those relate to fair
  treatment of individuals: \emph{Meritocracy} is the idea that people
  should receive rewards according to their merit. \emph{Equal
    treatment} is the related notion that similar people should be
  treated similarly under similar circumstances. Some concepts are
  related more to the treatment of different groups:
  \emph{Proportional representation} is the idea that proportions of
  different groups in society should be reflected in every facet of
  society. Finally \emph{non-discrimination} captures the notion of
  not treating people differently depending on sensitive
  characteristics.  }

\only<presentation>{
  \begin{frame}
    \frametitle{Fairness}
    What is it?
    \begin{itemize}
    \item<2-> \alert{Meritocracy}.
    \item<3-> Proportionality and representation.
    \item<4-> Equal treatment.
    \item<5-> \alert{Non-discrimination}.
    \end{itemize}
  \end{frame}
}

\begin{frame}
  \frametitle{Meritocracy}
  \only<article>{

    Meritocracy embodies the principle that merit should be
    rewarded. A common example are admissions to universities. Some
    type of summary, typically a grade obtained from high school, is
    used to represent the underlying merit of individuals.
  }
  \uncover<2->{
    \begin{example}[College admissions]
      \only<article>{ In this example, we have two students. In terms
        of grades, student $B$ is clearly better. If we can only
        accept one of them, and given no other information, it seems
        like the natural choice is student $B$.  }
      \begin{itemize}
      \item Student $A$ has a grade 4/5 from Gota Highschool.
      \item Student $B$ has a grade 5/5 from Vasa Highschool.
      \end{itemize}
    \end{example}
  }
  \only<article>{Grades, by themselves, are typically insufficient information. It might be that grades from some high-schools are inflated and do not represent the quality of individuals accurately. So, let us suppose we now consider the information.}
  \uncover<3->{
    \begin{example}[Additional information]
      \only<article>{
        In particular, let us suppose that we have statistics on how well students from different high school do, depending on their high school grade.
      }
      \begin{itemize}
      \item 70\% of admitted Gota graduates with 4+ get their degree.
      \item 50\% of admitted Vasa graduates with 5 get their degree.
      \end{itemize}
      \only<article>{
        All other thing being equal, it is now more likely that student $A$ will graduate. So perhaps we should take in $A$ and not $B$.
      }
    \end{example}
  }

  \uncover<4->{We still don't know how a \alert{specific} student will
    do!}

  \only<article>{ I must emphasise that these are only
    statistics, and not necessarily predictive of the students'
    ability.  Ideally, we would like to admit the students that we
    expect to do well, given the information that we have. However,
    this information is typically not enough for us to make reliable
    predictions.  In addition, we might want to also make sure that
    everybody has a chance to obtain a good education. In order to
    achieve this, we might want to promote ethnic or gender equality
    through university admissions.  Unfortunately, there is no ideal
    solution and we must always balance the benefit of individual
    students with that of specific societal groups as well as society
    as a whole.
  }
  

\end{frame}


\only<presentation>{
  \begin{frame}
    \frametitle{Hiring decisions}
    \begin{columns}
      \begin{column}{0.5\textwidth}
        \includegraphics[height=\textheight]{../figures/cmu-headcount}
      \end{column}
      \begin{column}{0.5\textwidth}
        \includegraphics[width=\columnwidth]{../figures/amazon-hiring}
        \\
        \includegraphics[width=\columnwidth]{../figures/recruitement-automation}
      \end{column}
    \end{columns}
  \end{frame}


  \begin{frame}
    \frametitle{Group fairness and proportionality}
    \includegraphics[width=\textwidth]{../figures/genomics-diversity}
    \url{https://qz.com/1367177/}
  \end{frame}

}

\begin{frame}  
  \begin{block}{Solutions}
    \only<article>{These solution methods are not completely exclusive, and can be implemented simultaneously to some extent.}
    \begin{itemize}
    \item<5-> Admit \alert{everybody}? \only<article>{This suggests that everybody is admitted to at least one university, perhaps even their university of choice. However, it requires that there is enough teaching capacity for all students in the first year. Subsequently, we expect the students who were not qualified to drop out. Of course, this is unfair to the qualified students, as it drains resources that could have been used for them.}
    \item<6-> Admit \alert{randomly}? \only<article>{Completely random decisions are not considered fair, because they do not take into account any information. However, randomisation can also be used in conjunction with grades to ensure that everybody has a shot.}
    \item<7-> Use \alert{prediction} of individual academic performance? \only<article>{The more information we have, the better we can predict academic performance. A grade from high school is one indicator, but more data can be used to obtain better predictions. Of course, no prediction is perfect.}
    \item<8-> Should we take into account \alert{group membership} or other population information? \only<article>{For many reasons, students in some groups can perform differently in standardised tests, even though their innate talents may be no different than students not in the group. The classical example of this is high school teachers discouraging girls from mathematics.}
    \end{itemize}
  \end{block}
  
\end{frame}


\only<article>{
  \begin{example}[Hiring decisions.]
    As a further example, consider gender balance in hiring decisions.
    Typically, received applications are screened, so that some
    applicants undergo through an interview process. At the end, some
    of the interviewed applicants will be hired. There are two
    decision points here, with most people being cut off at the first
    point: the screening. To automate this process, Amazon worked on
    resume-screen program.~\footnote{\url{https://www.reuters.com/article/us-amazon-com-jobs-automation-insight-idUSKCN1MK08G}}
    However, this was scrapped after it was discovered that it
    predominantly favoured men. The reason is not entirely clear, but
    it was probably due to the fact that they trained the system on
    their own screening decisions, and given that the tech industry
    predominantly hires men in the first place, women were likely
    rejected in the screening phase. 
  \end{example}
}


b\section{Group fairness.}
\only<article>{ Let us now take a look at concepts of fairness related
  to \emph{group membership}. This includes concepts such as equal
  treatment, equality of opportunity and generally lack of
  discrimination.  The general idea is that we would like for members
  of society to follow trajectories through life that do not strongly
  depend on their membership in sensitive groups. For example, gender
  should not play a role in academic achievement. Ethnic heritage
  should have no influence on annual income. Unfortunately, the
  underlying societal dynamics create situations where group
  membership becomes important.

  In this section, and throughout this chapter, we will imagine that
  an individual is interacting with a system, which makes decisions
  about the individual, such as whether or not to give them a
  loan. These result in a certain outcome , such as the individual
  using the borrowed money to invest in a business and then having a
  particular annual income. Crucially, these outcomes can be
  correlated with group membership, because of societal dynamics. The
  system designer should make sure that not only the system does what
  it intends, such as giving loans to people that are expected to
  repay them, but also that it does not create inequalities between
  different groups. As we will see later, depending on our fairness
  definition, and on the societal dynamics, this is not always easy.
  
}

\begin{frame}
  \frametitle{Bail decisions}

  \only<article>{ For a more detailed
    example, let us consider bail decisions in the US court
    system. When a defendant is charged, the judge has the option to
    either place them in jail pending trial, or set them free, under
    the condition that the defendant pays some amount of 'bail'. The
    amount of bail (if any) is set to deter flight or a relapse.

    This process sometimes includes the use of a software tool called
    \texttt{COMPAS}, which gives risk scores for the possibility of
    flight, recidivism or violent behaviour. These scores are taken
    into account by judges when making decisions.  In some cases, it
    appears as though automating this procedure might lead to better
    outcomes. But is that generally true?
  }


  \only<presentation>{
    \begin{columns}
      \begin{column}{0.5\textwidth}
        \centering
        \begin{tikzpicture}
          \node at (0,0) (judge) {\includegraphics[width=0.3\columnwidth]{../figures/judge}};
          \uncover<2->{
            \node at (-2,-2) (jail) {\includegraphics[width=0.3\columnwidth]{../figures/jail}};
            \draw[->] (judge) -- (jail);
          }
          \uncover<3->{
            \node at (2,-2) (bail) {\includegraphics[width=0.3\columnwidth]{../figures/bail}};
            \draw[->] (judge) -- (bail);
          }

          \uncover<4->{
            \node at (-2,-4) (trial) {\includegraphics[width=0.3\columnwidth]{../figures/trial}};
            \draw[->] (jail) -- (trial);
          }
          \uncover<5->{
            \draw[->] (bail) -- (trial);
          }
          \uncover<6->{
            \node at (2,-4) (arrest) {\includegraphics[width=0.3\columnwidth]{../figures/handcuffs}};
            \draw[->] (bail) -- (arrest);
          }
        \end{tikzpicture}
      \end{column}
      \begin{column}{0.5\textwidth}
        \centering
        \uncover<7->{
          \includegraphics[width=\textwidth]{../figures/judge-fairness}
        }
      \end{column}
    \end{columns}
  }

  \only<article>{In this setting, a defendant $t$ appears before a
    judge with observable features $x_t \in \CX$, and a sensitive group
    variable $z_t \in \CZ$. The judge employs a specific policy $\pol$ to
    select a decision $a_t \in \CA$, with
    \begin{equation}
      \pol(a_t \mid x_t, z_t)
    \end{equation}
    denoting the probability of action $a_t$ given the individual's features, as well as the sensitive variable. We can assume that the policy is fixed ahead of time, and thereafter decisions are made according to the policy.
    After the judge makes their decision, they observe an outcome $y_t$ sampled from some potentially unknown distribution with parameter $\theta$:
    \begin{equation}
      P_\theta(y_t \mid x_t, z_t), \qquad      P_\theta(y_t \mid x_t, z_t, a_t), 
    \end{equation}
    denoting the probability of $y_t$ given the individual's features
    and the action taken. Here, there are two possibilities. Either
    the outcome $y_t$ depends only on the observed features, or the
    action as well. The correct formulation depends on the meaning of
    the action.

    \paragraph{Actions that affect the outcome.}
    Lt us first consider a simple set of actions $\CA = \{0, 1\}$,
    where a defendant is granted $(a_t = 1)$ or denied $(a_t = 0)$
    bail.  If a defendant is not given bail, they must remain in jail,
    and will always be at the trial $(y_t = 0)$. This costs both the
    government and the defendant, so the judge prefers not to deny
    bail too often.  On the other hand, if the defendant is granted
    bail, there is a chance they will re-offend, or will fail to
    attend trial $(y_t = 1)$. Consequently, we can define the
    following utility function\footnote{See
      Chapter~\ref{ch:decision-problems}} for the judge, that roughly
    reflects those preferences:
    \begin{equation}
      U(a, y) = a - y.
    \end{equation}
    While this weighs the judge's two main concerns equally, we can also imagine different formulations: e.g. if the judge thinks it is much more important to keep people out of jail than to prevent re-offences until trial, they might select a function such as : $U(a, y) = 10a - y$.


    \paragraph{Actions that do not affect the outcome.} In this
    scenario, the only way for the actions to not affect the outcome
    is if everybody is released, with the judge's action only being a
    confidential note on the case file. Then their action cannot
    affect the outcome, since it is only revealed to the judge. The
    utility function can be based on the accuracy of predictions, so
    we can simply set it to $U(a,y) = a - y$.
    
    \paragraph{The optimisation problem.} Now, given the parameter $\param$ of the unknown distribution, the judge can find a decision rule $\pol$ maximising utility in expectation
    \begin{equation}
      \label{eq:expected-utility-judge}
      \E_\param^\pol(U) = \sum_{x,z} P_\theta(x, z) \sum_a \pol(a | x, z) \sum_y P_\theta(y | a, x, z) U(a,y).
    \end{equation}
    In practice, of course, $\param$ is not known but we have some training data $D = (x_t, z_t, a_t, y_t)_{t=1}^T$, collected with some historical policy $\pol_0$, and  we have to resort to one of the following solutions. Firstly, estimating some $\hat{\param}$ from the data and replacing that in the expectation. Secondly, calculating a posterior distribution $\beta(\param | D)$ and maximising $\int_\Param \E_\param^\pol(U) \dd \bel(\param | D)$. Thirdly, maximising an unbiased estimate of expected utility.
    \begin{align*}
      \E^\pol_\param(U)
      &=
        \sum_{x,z} P_\theta(x, z) \sum_a \pol(a | x, z) \sum_y P_\theta(y | a, x, z) U(a,y) \\
      &\approx
        \frac{1}{T} \sum_{t} \sum_a \pol(a | x_t, z_t) \sum_y P_\theta(y | a, x_t, z_t) U(a,y)
        \approx
        \frac{1}{T} \sum_{t} \frac{\pol(a_t | x_t, z_t)}{\pol_0(a_t | x_t, z_t)} U(a_t,y_t).
    \end{align*}
    The first approximation simply replaces the feature distribution with its empirical approximation.
    The second approximation is a bit more delicate, and is explained in more detail in Section~\ref{sec:monte-carlo}.
    This is because, in the original data we have only some specific $a_t, y_t$ for every individual $t$, where $a_t$ was selected by some historical policy $\pol_0$. For that reason, we must use importance sampling in order to estimate the utility of any policy $\pol$. Unfortunately, the policy $\pol_0$ is also unknown and must be estimated from the data. So, we are left with deciding between modelling $P_\param(y | a, x, z)$ or $\pol_0(a | x, z)$.

    However, doing so may have unintended effects, as treatment of different groups may be unequal. For that reason, we may have to add explicit fairness constraints to our optimisation problem. 

  }
\end{frame}  

\begin{frame}
  \frametitle{Demographic parity and equality of opportunity.}

  \only<article>{Let us take a look at how many individuals are given
    different scores. Figure~\ref{fig:risk-bias} shows the proportions
    of individuals obtaining different risk scores as a function of
    their ethnic group. While the general population, and Caucasians
    in particular, have a distribution sharply concentrated in low
    scores, Black defendants obtain much higher risk scores: they are
    almost as likely to be ranked a 10 as the are to be scored a 1. }
  \begin{figure}[H]
    \begin{columns}
      \begin{column}{0.5\textwidth}
        \centering
        \includegraphics[width=\columnwidth]{../figures/Scores-by-race}      
      \end{column}
    \end{columns}
    \label{fig:risk-bias}
    \caption{Apparent bias in risk scores towards black versus white defendants.}
  \end{figure}
  \only<article>{ Typically, we would like to demographic parity
    across groups, either with respect to the decisions, or with
    respect to the outcomes.  Decision parity is called \emph{equality
      of opportunity} and satisfies the following condition
    \begin{equation}
      \label{eq:equality-opportunity}
      \Pr_\param^\pol(a_t | z_t) =       \Pr_\param^\pol(a_t ).
    \end{equation}
    In other words, the action (risk score, in this case) is
    independent of the group.  As we see in
    Figure~\ref{fig:risk-bias}, these distributions are quite
    different.  The curve for ``General'' corresponds to
    $\Pr_\param^\pol(a_t )$, while the other two curves correspond to
    $\Pr_\param^\pol(a_t | z_t)$ for $z_t$ being Caucasian and
    African-American respectively.
    
    \emph{Demographic parity} instead relates to outcomes. If this
    condition is satisfied, then the probability distribution of
    outcomes is independent of group membership.
    \begin{equation}
      \label{eq:eqality-outcomes}
      \Pr_\param^\pol(y_t | z_t) =       \Pr_\param^\pol(y_t ).
    \end{equation}

  }
\end{frame}

\begin{frame}
  \frametitle{Calibration.}
  
  \only<article>{On the
    other hand, the scores generated by the software seemed to be very
    predictive on whether or not defendants would re-offend,
    independently of their race. Figure~\ref{fig:imrs} shows that, if
    an individual obtains a high score then they are very likely to
    re-offend, and conversely, they are unlikely to re-offend when
    they have a low score. }
  \begin{figure}[H]
    \centering
    \includegraphics[width=\columnwidth]{../figures/calibration-compas}
    \caption{Recidivism rates by risk score.}
    \label{fig:imrs}
  \end{figure}
  This concept can be quantified in terms of the conditional distribution of outcomes given the score:
  \begin{equation}
    \Pr_\param^\pol(y_t | a_t, z_t) =       \Pr_\param^\pol(y_t | a_t).
    \label{eq:calibration}
  \end{equation}
  This means that $y_t$ is conditionally independent of $z_t$ given
  the score $a_t$. So, in some sense, the score is sufficient for us
  to predict the outcome, and knowing the race does not help us
  predict any better.  
\end{frame}

\begin{frame}
  \frametitle{Balance}
  \only<article>{While the system's predictions seem to be calibrated against the chance of recidivism, this does not mean that race plays no role. Figure~\ref{fig:imrs-risk} breaks down the population in people that re-offended and those that did. For each sub-population, we then plot the proportion of people receiving different scores by race. While people generally have a small probability of obtaining a high risk score, we see that Black defendants obtain much higher scores.}
  
  \begin{figure}[H]
    \centering
    \begin{subfigure}{0.45\textwidth}
      \includegraphics[width=0.95\textwidth]{../figures/balance-non-recidivism-compas}
      \caption{No recidivism}
    \end{subfigure}
    \begin{subfigure}{0.45\textwidth}
      \includegraphics[width=0.95\textwidth]{../figures/balance-recidivism-compas}
      \caption{Recidivism}
    \end{subfigure}
    \caption{Score breakdown based on recidivism rates.}
    \label{fig:imrs-risk}
  \end{figure}
  \only<article>{
    Balance can also be interpreted as a probabilistic condition, of the following form:
    \begin{equation}
      \Pr_\param^\pol(a_t | y_t, z_t) =       \Pr_\param^\pol(a_t | y_t).
      \label{eq:balance}
    \end{equation}
    Here we have that $a_t$ is conditionally independent of $z_t$
    given the outcome $y_t$. So, if we partition the population
    according to their outcome, we should find that given the outcome,
    the distribution of scores is the same no matter what their
    race. However, this is not what the example shows: there is a
    strong dependence on race.  }
  
\end{frame}

\only<presentation>{

  \begin{frame}
    \frametitle{Graphical models and independence}
    \begin{itemize}
    \item Why is it not possible to be fair in all respects?
    \item Different notions of \alert{conditional independence}.
    \item Can only be satisfied rarely simultaneously.
    \end{itemize}
  \end{frame}

}

\only<article>{ How can we explain this discrepancy? We can show that
  in fact, each one of these different measures of bias in our
  decision rules can be seen as a notion of conditional
  independence. Had $z_t$ been an independent variable, i.e. so that
  $P_\param(x,z) = P_\param(x) P_\param(z)$ and
  $P_\param(y | x,a,z) = P_\param(y | x,a)$, then it would have no
  role. Unfortunately, though, it does influence the distributions of
  the other variables.  }


\begin{example}[Classification]
  One setting where the above discussion applies directly are classification problems. There, the outcomes $y$ are simply labels. the decision space $\CA$ is to just predicting a label, and the utility function is the classification accuracy, so that $U(a, y) = \ind{a = y}$. In this setting, $y$ does not directly depend on $a$, since $a$ is just a prediction of the latent label.  More precisely, $y \indep a \mid x$. This allows us to write:
  \[
    \Pr_\param^\pol(y_t \mid a_t, x_t, z_t) =   \Pr_\param^\pol(y_t \mid x_t, z_t).
  \]
\end{example}

\begin{example}[Regression]
  We can also apply these ideas to regression problems. We are again
  asked to simply predict a latent variable $y$. For concreteness,
  assume $y \in \Reals$. Then, the decision space $\CA = \Reals$ as
  well. A standard utility function is the negative squared prediction
  error, so that $U(a, y) = - (a - y)^2$.

  In this setting we can also relax our framework to work with
  expectations instead of probabilities. For example, calibration and balance can be written as the conditions
  \[
    \E_\param^\pol(y_t | a_t, z_t) = \E_\param^\pol(y_t | a_t),
    \qquad
    \E_\param^\pol(a_t | y_t, z_t) = \E_\param^\pol(a_t | y_t),
  \]
  respectively.
\end{example}

\subsection{Measuring group fairness.}

\only<article>{
  The above discussion focused on the setting where we have a precise
  model parameter $\param$ and policy $\pol$. However, we frequently do
  not know at least one of those things. For example, when we only have
  some observational data generated by some historical policy $\pol$,
  need to estimate $\Pr_\theta^\pol$ somehow. Moreover, these conditions
  will not be perfectly satisfied. How can we measure the deviation from
  these conditions?
}


\paragraph{Empirical expected utility.}
\only<article>{
  Let us start with expect utility, and write it in full
  \[
    \E_\param^\pol(U) = \sum_{x,z,y} \Pr_\param^\pol(y,a,x,z) U(a,y),
  \]
  The $y_t, a_t, x_t, z_t$ is already sampled from $\Pr_\param^\pol$, so we can approximate the expected utility with
}
\[
  \hat{E}_n(U) = \sum_{t=1}^n U(a_t,y_t), \qquad a_t, y_t \sim \Pr_\param^\pol
\]


\paragraph{Deviation from balance.}
\only<article>{

  How about measuring whether balance is satisfied?  Unfortunately
  none of these fairness conditions will be satisfied in practice. So
  what are we going to do? A simple idea is to simply look at how far
  away the distribution is from independence. In particular, let us
  look at the difference between those distributions
  \[
    \Pr_\param^\pol(a | y, z), \qquad  \Pr_\param^\pol(a | y).
  \]
  for all values of $y, z$. Let us first look at the total variation distance:
  \[
    \|\Pr_\param^\pol(a | y, z) -   \Pr_\param^\pol(a | y)\|_1
    =
    \sum_a \|\Pr_\param^\pol(a| y, z) -   \Pr_\param^\pol(a | y)\|_1.
  \]
  We can then sum over all possible values of $y, z$ to obtain a
  single measure of how much we deviate from balance. To obtain an
  empirical estimate, we can simply sum over all observations to
  obtain the empirical models:
  \[
    \hat{P}_n(a | y, z) = \sum_t \frac{\ind{a_t = a \wedge y_t = y \wedge z_t = i}}{\ind{y_t = y \wedge z_t = i}}
  \]
  We can then plug those into our original measure:
  \[
    \|\Pr_\param^\pol(a | y, z) -   \Pr_\param^\pol(a | y)\|_1
    \approx
    \sum_a \|\hat{P}_n (a| y, z) -   \hat{P}_n(a| y)\|_1.
  \]

  However, we can
  use something else as well, such as the KL divergence:
  \[
    D(\Pr_\param^\pol(a | y, z)  \|   \Pr_\param^\pol(a | y))
    =
    \sum_a \ln \frac{\Pr_\param^\pol(a | y, z)}{\Pr_\param^\pol(a | y)} \Pr_\param^\pol(a | y, z)
  \]

  
}
\subsection{Trading off  utility and fairness.}
\label{sec:trad-util-fairn}
\only<article>{ We have now selected a notion of utility and of
  fairness. We have also seen how to measure utility and fairness in
  practice. How can we choose a policy $\pol$ that optimises for both? The
  general idea is to formalise the problem in terms of a constrained
  or unconstrained optimisation. }
\begin{frame}
  \frametitle{Formalisation of the problem}

  \only<article>{ Let us use
    $F(\pol, \param)$ for the measure of deviation from fairness and
    $U(\pol, \param)$ for utility.

    We can deal with the problem in two ways. The first is to formulate a combined objective that mixes the utility and fairness objectives. Then we can formalise the problem as an unconstrained optimisation. The second is to optimise one objective and place constrains on the other. For example, we can place a limit on how much unfairness we are willing to tolerate and maximise utility. Or, we can place a limit on how far away we want to be from the solution that optimises for utility only, and optimises for fairness. Which method to use is application-dependent, but also has some computational impact.
  }

  \begin{block}{Unconstrained optimisation.}
    \only<article>{The simnplest idea is to treat the utility and fairness symetrically. Then we can simply optimise for their mixture.}
    Let $\lambda \in [0,1]$:
    \[
      \max_\pol (1 - \lambda) U(\pol, \param) - \lambda F(\pol, \param)
    \]
  \end{block}
  \only<article>{
    In this setting, we have an unconstrained optimisation problem. These are typically easier to solve, and we can use simple gradient methods. In particular, note that the gradient with respect to the policy of the unconstrained objective is
    \[
      \nabla_\pol [(1 - \lambda) U(\pol, \param) - \lambda F(\pol, \param)]
      =
      (1 - \lambda) \nabla_\pol U(\pol, \param) - \lambda \nabla_\pol F(\pol, \param).
    \]
    Consqeuently, we only need to minimise the sum of these gradients.
  }
  
  \begin{block}{Constrained fairness.}
    \only<article>{In this setting, we have an upper bound $\epsilon$ on the amount of fairness deviation we can allow.}
    Let $\epsilon \geq 0$:
    \begin{align*}
      \max_\pol U(\pol, \param)\\
      \textrm{s.t.} F(\pol, \param) \leq \epsilon.
    \end{align*}
  \end{block}

  \begin{block}{Constrained utility.}
    \only<article>{In this setting, we want to limit the amount of sub-optimality $\epsilon$ we have relative to the optimal policy that ignores fairness.}
    Let $\epsilon \geq 0$:
    \begin{align*}
      \in_\pol F(\pol, \param)\\
      \textrm{s.t.} U(\pol, \param) \geq  \max_{\pol'} U(\pol', \param) - \epsilon
    \end{align*}
  \end{block}
  
\end{frame}

\section{Individual fairness}

\only<article>{
  An altogether different type of fairness concept is individual fairness. This relates to the notion of meritocracy. This means that ``better'' people should have ``better'' outcomes. But what do we mean by better? This depends strongly on the context.
}

\begin{frame}\frametitle{Meritocracy.}
  \begin{block}{Inherent worth.}
    \begin{itemize}
    \item $a$: Action taken by the decision maker. \only<article>{For example, $a$ could specify which applicants are admitted to university, with $a_i$ specifying the university to which an applicant is admitted.}
    \item $x_i \in \Reals^d$: Individual features. \only<article>{That's what we observe about the individual.}
    \item $w_i \in \Reals$: Individual worth. \only<article>{This can be taken to mean how well the individual is expected to perform in university, i.e. their expected grade. However, this quantity is not directly observed.}
    \item $u_i(a) \in \Reals$: Individual utility. \only<article>{This is how much the individual gains by the decision. For simplicity, we can imagine that $i$ is only impacted by the decision regarding them, rather than anybody else, so we can just define a functino $u_i(a_i)$ instead.}
    \end{itemize}
    A decision $a$ is fair if, for all $i,j$ it holds that: if $w_i > w_j$ then $u_i(a) > u_j(a)$.
  \end{block}
  \only<article>{The problem in this setting, is that it is not clear what an individual's worth is. It is also not clear what their utility is. In a sense, if $i$ and $j$ are both admitted to the same university, is it worth the same to them? }
\end{frame}

\begin{frame}\frametitle{Smoothness.}
  \begin{block}{Treating similar individuals similarly.}
    \begin{itemize}
    \item $d(x_i,x_j)$: Distance between individuals $i,j$.
    \item $D[\pol(a_i | x_i), \pol(a_j | x_j)]$: Distance between policy decisions.
    \item Utility function $U(\pol)$.
    \end{itemize}
    Our goal is to find a decision rule $\pol$ maximising  $U(\pol)$, 
    so that it is Lipschitz-smooth with respect to $D, d$:
    \[
      D[\pol(a_i | x_i), \pol(a_j | x_j)] \leq d(x_i,x_j)
    \]
  \end{block}
  \only<article>{When $D(P,Q) = \max_{a \in A} |\log [P(a)/Q(a)]|$ then $\pol$ satisfies differential privacy with $d(x,y) = \epsilon \|x - y\|_1$}.
\end{frame}

\begin{frame}\frametitle{Smoothness and parity.}
  To define groups, we select $S, T \subset \CX$.
  \begin{definition}{$\epsilon$-parity}
    \[
      \|\pol(a | x \in S) - \pol(a | x \in T)\|_1 \leq \epsilon.
    \]
  \end{definition}

  \begin{definition}{Bias for some action $a$:}
    \[
      \max_\pol \pol(a | x \in S) - \pol(a | x \in T)
      \qquad
      \textrm{s.t. $\pol$ is Lipschitz}
    \]
  \end{definition}
  \only<article>{The main problem here is that if we are required to be smooth, but one set has most of its members in one subset.}
\end{frame}


\section{Exercises}

\begin{exercise}
  Consider the following problem, where $z$ is Bernoulli(1/2) and $x \Reals^2$ is a normally distributed variable with mean $(0,z)$ and identity covariance matrix. There are two actions and outcomes, with the outcome being $y_t = 1$ with probability $1/(1 + e^{x_a})$.
  [Add more details]
\end{exercise}

%%% Local Variables:
%%% mode: latex
%%% TeX-engine: xetex
%%% TeX-master: "book"
%%% End:

\section{Project: Credit risk for mortgages}

Consider a bank that must design a decision rule for giving loans to individuals. In this particular case, some of each individual's characteristics are partially known to the bank.  We can assume that the insurer has a linear utility for money and wishes to maximise expected utility. Assume that the $t$-th individual is associated with relevant information $x_t$, sensitive information $z_t$ and a potential outcome $y_t$, which is whether or not they will default on their mortgage. For each individual $t$, the decision rule chooses $a \in \CA$ with probability $\pol(a_t = a \mid x_t)$.

As an example, take a look at the historical data in \texttt{data/credit/german.data-mumeric}, described in \texttt{data/credit/german.doc}. Here there are some attributes related to financial situation, as well as some attributes related to personal information such as gender and marital status. 

For this project, you must
\begin{enumerate}
\item Design a policy for giving or denying credit to individuals, given their probability for being credit-worthy. Assuming that if an individual is credit-worthy, you will obtain a return on investement of $r = 0.5\%$ per month. Take into account the length of the loan to calculate the utility. Assume that the loan is either fully repaid at the end of the lending period $n$, or not at all to make things simple. If an individual is not credit-worthy you will lose your investment of $m$ credits, otherwise you will gain $m r^{n}$ . Ignore macroenomic aspects, such as inflation. In this section, simply assume you have a model for predicting creditworthiness as input to your policy.
\item Learn/train a model for calculating the probability of credit-worthiness from the german data. What are the implicit assumptions about the labelling process in the original data?
\item Combine the model with the first policy to obtain a policy for giving credit, given only the information about the individual and previous data seen.
\item How can you ensure that your policy maximises revenue? How can you take into account the uncertainty due to the limited and/or biased data? What if you have to decide for credit for thousands of individuals and your model is wrong? How should you take that type of risk into account? 
\item Does the existence of this database raise any privacy concerns? If the database was secret (and only known by the bank), but the credit decisions were public, how would that affect privacy?
\item Choose one concept of fairness, e.g. balance of decisions with respect to gender. How do you ensure that your policy is fair? How can you measure it?
\end{enumerate}


%%% Local Variables:
%%% mode: latex
%%% TeX-master: notes
%%% End:




\chapter{Simple decision problems}
\label{ch:decision-problems}
\only<article>{This chapter deals with simple decision problems, whereby a decision maker (DM) makes a simple choice among many. In some of this problems the DM has to make a decision after first observing some side-information. Then the DM uses a \emph{decision rule} to assign a probability to each possible decision for each possible side-information. However, designing the decision rule is not trivial, as it relies on previously collected data. A higher-level decision includes choosing the decision rule itself. The problems of classification and regression fall within this framework. While most steps in the process can be automated and formalised, a lot of decisions are actual design choices made by humans. This creates scope for errors and misinterpretation of results.

In this chapter, we shall formalise all these simple decision problems from the point of view of statistical decision theory. The first question is, given a real world application, what type of decision problem does it map to? Then, what kind of machine learning algorithms can we use to solve it? What are the underlying assumptions and how valid are our conclusions? 
}

\section{Hierarchies of decision making problems}
\label{sec:decision-problems}
\only<presentation>{
  \begin{frame}
    \tableofcontents[ 
    currentsection, 
    hideothersubsections, 
    sectionstyle=show/shaded
    ] 
  \end{frame}
}


\only<article>{
  All machine learning problems are essentially decision problems. This essentially means replacing some human decisions with machine decisions. One of the simplest decision problems is classification, where you want an algorithm to decide the correct class of some data, but even within this simple framework there is a multitude of decisions to be made. The first is how to frame the classification problem the first place. The second is how to collect, process and annotate the data. The third is choosing the type of classification model to use. The fourth is how to use the collected data to find an optimal classifier within the selected type. After all this has been done, there is the problem of classifying new data. In this course, we will take a holistic view of the problem, and consider each problem in turn, starting from the lowest level and working our way up.}


\subsection{Simple decision problems}
\begin{frame}
  \frametitle{Preferences}
  \only<article>{The simplest decision problem involves selecting one item from a set of choices, such as in the following examples}  
  \begin{example}
    \begin{block}{Food}
      \begin{itemize}
      \item[A] McDonald's cheeseburger
      \item[B] Surstromming
      \item[C] Oatmeal
      \end{itemize}
    \end{block}
    \begin{block}{Money}
      \begin{itemize}
      \item[A] 10,000,000 SEK
      \item[B] 10,000,000 USD
      \item[C] 10,000,000 BTC
      \end{itemize}
    \end{block}
    \begin{block}{Entertainment}
      \begin{itemize}
      \item[A] Ticket to Liseberg
      \item[B] Ticket to Rebstar
      \item[C] Ticket to Nutcracker
      \end{itemize}
    \end{block}
  \end{example}
\end{frame}

\begin{frame}
  \frametitle{Rewards and utilities}
  \only<article>{In the decision theoretic framework, the things we receive are called rewards, and we assign a utility value to each one of them, showing which one we prefer.}
  \begin{itemize}
  \item Each choice is called a \alert{reward} $r \in \CR$.
  \item There is a \alert{utility function} $U : \CR \to \Reals$, assigning values to reward.
  \item We (weakly) prefer $A$ to $B$ iff $U(A) \geq U(B)$.
  \end{itemize}
  \only<article>{In each case, given $U$ the choice between each reward is trivial. We just select the reward:
    \[
    r^* \in \argmax_r U(r)
    \]
    The main difficult is actually selecting the appropriate utility function. In a behavioural context, we simply assume that humans act with respect to a specific utility function. However, figuring out this function from behavioural data is non trivial. ven when this assumption is correct, individuals do not have a common utility function.
  }
  \begin{exercise}
    From your individual preferences, derive a \alert{common utility function} that reflects everybody's preferences in the class for each of the three examples. Is there a simple algorithm for deciding this? Would you consider the outcome fair?
  \end{exercise}
\end{frame}

\begin{frame}
  \frametitle{Preferences among random outcomes}
  \begin{example}
    Would you rather \ldots
    \begin{itemize}
    \item[A] Have 100 EUR now?
    \item[B] Flip a coin, and get 200 EUR if it comes heads?
    \end{itemize}    
  \end{example}
  \uncover<2->{
    \begin{block}{The expected utility hypothesis}
      Rational decision makers prefer choice $A$ to $B$ if
      \[
      \E(U | A) \geq \E(U | B),
      \]
      where the expected utility is
      \[
      \E(U | A) = \sum_r U(r) \Pr(r | A).
      \]
    \end{block}
    In the above example, $r \in \{0, 100, 200\}$ and $U(r)$ is
    increasing, and the coin is fair.
  }
  \begin{block}{Risk\index{risk} and monetary rewards}
    \only<article>{When $r \in \Reals$, as in the case of monetary rewards, we can use the shape of the utility function determines the amount of risk aversion. In particular:}
    \begin{itemize}
    \item<3-> If $U$ is convex, we are risk-seeking. \only<article>{In the example above, we would prefer B to A, as the expected utility of B would be higher than A, even though they give the same amount of money on average.}
    \item<4-> If $U$ is linear, we are risk neutral. \only<article>{In the example above, we would be indifferent between $A$ and $B$, as the expected amount of money is the same as the amount of money we get.}
    \item<5-> If $U$ is concave, we are risk-averse. \only<article>{Hence, in the example above, we prefer A.}
    \end{itemize}
  \end{block}
  \only<article>{This idea of risk can be used with any other utility function. We can simply replace the original utility function $U$ with a monotonic function $f(U)$ to achieve risk-sensitive behaviour. However, this is not the only risk-sensitive approach possible.}
\end{frame}


\begin{frame}
  \frametitle{Uncertain rewards}
  \only<article>{However, in real life, there are many cases where we can only choose between uncertain outcomes. The simplest example are lottery tickets, where rewards are essentially random. However, in many cases the rewards are not really random, but simply uncertain. In those cases it is useful to represent our uncertainty with probabilities as well, even though there is nothing really random.}
  \begin{itemize}
  \item Decisions $\decision \in \Decision$
  \item Each choice is called a \alert{reward} $r \in \CR$.
  \item There is a \alert{utility function} $U : \CR \to \Reals$, assigning values to reward.
  \item We (weakly) prefer $A$ to $B$ iff $U(A) \geq U(B)$.
  \end{itemize}

  \begin{example}
    \begin{columns}
      \begin{column}{0.5\textwidth}
        You are going to work, and it might rain.  What do you do?
        \begin{itemize}
        \item $\decision_1$: Take the umbrella.
        \item $\decision_2$: Risk it!
        \item $\outcome_1$: rain
        \item $\outcome_2$: dry
        \end{itemize}
      \end{column}
      \begin{column}{0.5\textwidth}
        \begin{table}
          \centering
          \begin{tabular}{c|c|c}
            $\Rew(\outcome,\decision)$ & $\decision_1$ & $\decision_2$ \\ %ro: U has only one argument.
            \hline
            $\outcome_1$ & dry, carrying umbrella & wet\\
            $\outcome_2$ & dry, carrying umbrella & dry\\
            \hline
            \hline
            $U[\Rew(\outcome,\decision)]$ & $\decision_1$ & $\decision_2$ \\
            \hline
            $\outcome_1$ & 0 & -10\\
            $\outcome_2$ & 0 & 1
          \end{tabular}
          \caption{Rewards and utilities.}
          \label{tab:rain-utility-function}
        \end{table}

        \begin{itemize}
        \item<2-> $\max_\decision \min_\outcome U = 0$
        \item<3-> $\min_\outcome \max_\decision U = 0$
        \end{itemize}
      \end{column}

    \end{columns}
  \end{example}
\end{frame}



\begin{frame}
  \frametitle{Expected utility}
  \[
  \E (U \mid a) = \sum_r U[\Rew(\outcome, \decision)] \Pr(\outcome \mid \decision)
  \]
  \begin{example}%ro: rather an exercise?
    You are going to work, and it might rain. The forecast said that
    the probability of rain $(\outcome_1)$ was $20\%$. What do you do?
    \begin{itemize}
    \item $\decision_1$: Take the umbrella.
    \item $\decision_2$: Risk it!
    \end{itemize}
    \begin{table}
      \centering
      \begin{tabular}{c|c|c}
        $\Rew(\outcome,\decision)$ & $\decision_1$ & $\decision_2$ \\ %ro: U has only one argument.
        \hline
        $\outcome_1$ & dry, carrying umbrella & wet\\
        $\outcome_2$ & dry, carrying umbrella & dry\\
        \hline
        \hline
        $U[\Rew(\outcome,\decision)]$ & $\decision_1$ & $\decision_2$ \\
        \hline
        $\outcome_1$ & 0 & -10\\
        $\outcome_2$ & 0 & 1\\
        \hline
        \hline
        $\E(U \mid \decision)$ & 0 &  -1.2 \\ 
      \end{tabular}
      \caption{Rewards, utilities, expected utility for $20\%$ probability of rain.}
      \label{tab:rain-utility-function}
    \end{table}
  \end{example}
\end{frame}





\subsection{Decision rules}

\only<article>{We now move from simple decisions to decisions that
  depend on some observation. We shall start with a simple problem in applied meteorology. Then we will discuss hypothesis testing as a decision making problem. Finally, we will go through an exercise in Bayesian methods for classification.}

\begin{frame}
  \frametitle{Bayes decision rules}
  Consider the case where outcomes are independent of decisions:
  \[
  \util (\bel, \decision) \defn \sum_{\model}  \util (\model, \decision) \bel(\model)
  \]
  This corresponds e.g. to the case where $\bel(\model)$ is the belief about an unknown world.
  \begin{definition}[Bayes utility]
    \label{def:bayes-utility}
    The maximising decision for $\bel$ has an expected utility equal to:
    \begin{equation}
      \BUtil(\bel) \defn \max_{\decision \in \Decision} \util (\bel, \decision).
      \label{eq:bayes-utility}
    \end{equation}
  \end{definition}
\end{frame}




\begin{frame}
  \frametitle{The $n$-meteorologists problem}
  \index{$n$-meteorologists|textbf}
  \only<article>{Of course, we may not always just be interested in classification performance in terms of predicting the most likely class. It strongly depends on the problem we are actually wanting to solve. In  biometric authentication, for example, we want to guard against the unlikely event that an impostor will successfully be authenticated. Even if the decision rule that always says 'OK' has the lowest classification error in practice, the expected cost of impostors means that the optimal decision rule must sometimes say 'Failed' even if this leads to false rejections sometimes.}
  \begin{exercise}
    \only<presentation>{
      \only<1>{
        \begin{itemize}
        \item Meteorological models $\CM = \set{\model_1, \ldots, \model_n}$
        \item Rain predictions at time $t$: $p_{t,\model} \defn  P_{\model}(x_t = \textrm{rain})$.
        \item Prior probability $\bel(\model) = 1/n$ for each model.
        \item Should we take the umbrella?
        \end{itemize}
      }
    }
    \only<article>{Assume you have $n$ meteorologists. At each day $t$, each meteorologist $i$ gives a probability $p_{t,\model_i}\defn P_{\model_i}(x_t = \textrm{rain})$ for rain. Consider the case of there being three meteorologists, and each one making the following prediction for the coming week. Start with a uniform prior $\bel(\model) = 1/3$ for each model.}
    {
      \begin{table}[h]
        \begin{tabular}{c|l|l|l|l|l|l|l}
          &M&T&W&T&F&S&S\\
          \hline
          CNN & 0.5 & 0.6 & 0.7 & 0.9 & 0.5 & 0.3 & 0.1\\
          SMHI & 0.3 & 0.7 & 0.8 & 0.9 & 0.5 & 0.2 & 0.1\\
          YR & 0.6 & 0.9 & 0.8 & 0.5 & 0.4 & 0.1 & 0.1\\
          \hline
          Rain? & Y & Y & Y & N & Y & N & N
        \end{tabular}
        \caption{Predictions by three different entities for the probability of rain on a particular day, along with whether or not it actually rained.}
        \label{tab:meteorologists}
      \end{table}
    }
    \uncover<2->{
      \begin{enumerate}
      \item<2-> What is your belief about the quality of each meteorologist after each day? 
      \item<3-> What is your belief about the probability of rain each day? 
        \[
        P_\bel(x_t = \textrm{rain} \mid x_1, x_2, \ldots x_{t-1})
        =
        \sum_{\model \in \Model} P_\model(x_t = \textrm{rain} \mid x_1, x_2, \ldots x_{t-1})
        \bel(\model \mid x_1, x_2, \ldots x_{t-1}) 
        \]
      \item<4-> Assume you can decide whether or not to go running each
        day. If you go running and it does not rain, your utility is 1. If
        it rains, it's -10. If you don't go running, your utility is
        0. What is the decision maximising utility in expectation (with respect to the posterior) each
        day?
      \end{enumerate}
    }
    \label{ex:meteorologists}
  \end{exercise}
\end{frame}


\only<article>{
  \subsection{Statistical testing\textsuperscript{*}}
\only<article>{A common type of decision problem is a statistical test. This arises when we have a set of possible candidate models $\CM$ and we need to be able to decide which model to select after we see the evidence.
  Many times, there is only one model under consideration, $\model_0$, the so-called \alert{null hypothesis}. Then, our only decision is whether or not to accept or reject this hypothesis.}
\begin{frame}
  \frametitle{Simple hypothesis testing}
  \only<article>{Let us start with the simple case of needing to compare two models.}
  \begin{block}{The simple hypothesis test as a decision problem}
    \begin{itemize}
    \item $\CM = \{\model_0, \model_1\}$
    \item $a_0$: Accept model $\model_0$
    \item $a_1$: Accept model $\model_1$
    \end{itemize}
    \begin{table}[H]
      \begin{tabular}{c|cc}
        $\util$& $\model_0$& $\model_1$\\\hline
        $a_0$ & 1 & 0\\
        $a_1$ & 0 & 1
      \end{tabular}
      \caption{Example utility function for simple hypothesis tests.}
    \end{table}
    \only<article>{There is no reason for us to be restricted to this utility function. As it is diagonal, it effectively treats both types of errors in the same way.}
  \end{block}

  \begin{example}[Continuation of the medium example]
    \begin{itemize}
    \item $\model_1$: that John is a medium.
    \item $\model_0$: that John is not a medium.
    \end{itemize}
    \only<article>{
      Let $x_t$ be $0$ if John makes an incorrect prediction at time $t$ and $x_t = 1$ if he makes a correct prediction. Let us once more assume a Bernoulli model, so that John's claim that he can predict our tosses perfectly means that for a sequence of tosses $\bx = x_1, \ldots, x_n$,
      \[
        P_{\model_1}(\bx) = \begin{cases}
          1, & x_t = 1 \forall t \in [n]\\
          0, & \exists t \in [n] : x_t = 0.
        \end{cases}
      \]
      That is, the probability of perfectly correct predictions is 1, and that of one or more incorrect prediction is 0. For the other model, we can assume that all draws are independently and identically distributed from a fair coin. Consequently, no matter what John's predictions are, we have that:
      \[
        P_{\model_0}(\bx = 1 \ldots 1) = 2^{-n}.
      \]
      So, for the given example, as stated, we have the following facts:
      \begin{itemize}
      \item If John makes one or more mistakes, then $\Pr(\bx \mid \model_1) = 0$ and $\Pr(\bx \mid \model_0) = 2^{-n}$. Thus, we should perhaps say that then John is not a medium
      \item If John makes no mistakes at all, then 
        \begin{align}
          \Pr(\bx = 1, \ldots, 1 \mid \model_1) &= 1,
          &
            \Pr(\bx = 1, \ldots, 1 \mid \model_0) &= 2^{-n}.
        \end{align}
      \end{itemize}
      Now we can calculate the posterior distribution, which is
      \[
        \bel(\model_1 \mid \bx = 1, \ldots, 1) = \frac{1 \times \bel(\model_1)}{1 \times \bel(\model_1) + 2^{-n} (1 - \bel(\model_1))}.
      \]
      Our expected utility for taking action $a_0$ is actually
    }
    \[
      \E_\bel(\util \mid a_0) = 1 \times \bel(\model_0 \mid \bx) + 0 \times \bel(\model_1 \mid \bx), \qquad
      \E_\bel(\util \mid a_1) = 0 \times \bel(\model_0 \mid \bx) + 1 \times \bel(\model_1 \mid \bx)
    \]
  \end{example}
  
\end{frame}


\begin{frame}
  \frametitle{Null hypothesis test}
    \index{Null-Hypothesis test}
  Many times, there is only one model under consideration, $\model_0$, the so-called \alert{null hypothesis}. \only<article>{ This happens when, for example, we have no simple way of defining an appropriate alternative. Consider the example of the medium: How should we expect a medium to predict? Then, our only decision is whether or not to accept or reject this hypothesis.}
  \begin{block}{The null hypothesis test as a decision problem}
    \begin{itemize}
    \item $a_0$: Accept model $\model_0$
    \item $a_1$: Reject model $\model_0$
    \end{itemize}
  \end{block}

  \begin{example}{Construction of the test for the medium}
    \index{policy!for statistical testing}
    \begin{itemize}
    \item<2-> $\model_0$ is simply the $\Bernoulli(1/2)$ model: responses are by chance.
    \item<3-> We need to design a policy $\pol(a \mid \bx)$ that accepts or rejects depending on the data.
    \item<4-> Since there is no alternative model, we can only construct this policy according to its properties when $\model_0$ is true.
    \item<5-> In particular, we can fix a policy that only chooses $a_1$ when $\model_0$ is true a proportion $\delta$ of the time.
    \item<6-> This can be done by construcing a threshold test from the inverse-CDF.
    \end{itemize}
  \end{example}
\end{frame}
\begin{frame}
  \frametitle{Using $p$-values to construct statistical tests}
  \begin{definition}[Null statistical test]
    \only<article>{
      A statistical test $\pol$ is a decision rule for accepting or rejecting a hypothesis on the basis of evidence. A $p$-value test rejects a hypothesis whenever the value of the statistic $f(x)$ is smaller than a threshold.}
    The statistic $f : \CX \to [0,1]$ is  designed to have the property:
    \[
      P_{\model_0}(\cset{x}{f(x) \leq \delta}) = \delta.
    \]
    If our decision rule is:
    \[
      \pol(a \mid x) =
      \begin{cases}
        a_0, & f(x) \geq \delta\\
        a_1, & f(x) < \delta,
      \end{cases}
    \]
    the probability of rejecting the null hypothesis when it is true is exactly $\delta$.
  \end{definition}
  \only<presentation>{The value of the statistic $f(x)$, otherwise known as the \alert{$p$-value}, is uninformative.}
  \only<article>{This is because, by definition, $f(x)$ has a uniform distribution under $\model_0$. Hence the value of $f(x)$ itself is uninformative: high and low values are equally likely. In theory we should simply choose $\delta$ before seeing the data and just accept or reject based on whether $f(x) \leq \delta$. However nobody does that in practice, meaning that $p$-values are used incorrectly. Better not to use them at all, if uncertain about their meaning.}
\end{frame}
\begin{frame}
  \begin{block}{Issues with $p$-values}
    \begin{itemize}
    \item They only measure quality of fit \alert{on the data}.
    \item Not robust to model misspecification. \only<article>{For example, zero-mean testing using the $\chi^2$-test has a normality assumption.}
    \item They ignore effect sizes. \only<article>{For example, a linear analysis may determine that there is a significant deviation from zero-mean, but with only a small effect size of 0.01. Thus, reporting only the $p$-value is misleading}
    \item They do not consider prior information. 
    \item They do not represent the probability of having made an error. \only<article>{In particular, a $p$-value of $\delta$ does not mean that the probability that the null hypothesis is false given the data $x$, is $\delta$, i.e. $\delta \neq \Pr(\neg \model_0 \mid x)$.}
    \item The null-rejection error probability is the same irrespective of the amount of data (by design).
    \end{itemize}
  \end{block}
\end{frame}

\begin{frame}
  \only<article>{Let us consider the example of the medium.}
  \begin{block}{$p$-values for the medium example}
    \begin{itemize}
    \item<2->$\model_0$ is simply the $\Bernoulli(1/2)$ model:
      responses are by chance. 
    \item<3->CDF: $P_{\model_0}(N \leq n \mid K = 100)$ \only<article> {is the probability of at most $N$ successes if we throw the coin 100 times. This is in fact the cumulative probability function of the binomial distribution. Recall that the binomial represents the distribution for the number of successes of independent experiments, each following a Bernoulli distribution.}
    \item<4->ICDF:  the number of successes that will happen with probability at least $\delta$
    \item<5->e.g. we'll get at most 50 successes a proportion $\delta = 1/2$ of the time.
    \item<6>Using the (inverse) CDF we can construct a policy $\pol$ that selects $a_1$ when $\model_0$ is true only a $\delta$ portion of the time, for any choice of $\delta$.
    \end{itemize}
  \end{block}
  \begin{columns}
    \setlength\fheight{0.33\columnwidth}
    \setlength\fwidth{0.33\columnwidth}
    \begin{column}{0.5\textwidth}
      \only<3,4,5,6>{% This file was created by matlab2tikz.
%
%The latest updates can be retrieved from
%  http://www.mathworks.com/matlabcentral/fileexchange/22022-matlab2tikz-matlab2tikz
%where you can also make suggestions and rate matlab2tikz.
%
\begin{tikzpicture}

\begin{axis}[%
width=\fwidth,
height=0.831\fheight,
at={(0\fwidth,0\fheight)},
scale only axis,
xmin=0,
xmax=100,
xlabel={Number of successes},
ymin=0,
ymax=1,
ylabel={Probability of less than N successes},
axis background/.style={fill=white}
]
\addplot [color=blue, forget plot]
  table[row sep=crcr]{%
0	7.8886090522101e-31\\
1	7.96749514273217e-29\\
2	3.98453643227134e-27\\
3	1.31543344806511e-25\\
4	3.2248444478818e-24\\
5	6.26162256269277e-23\\
6	1.0029797609618e-21\\
7	1.36307186640302e-20\\
8	1.60428183412199e-19\\
9	1.66102448972682e-18\\
10	1.53164508771899e-17\\
11	1.27042666774617e-16\\
12	9.5567876801385e-16\\
13	6.56490776101789e-15\\
14	4.14222593604009e-14\\
15	2.41271075196861e-13\\
16	1.30296790932804e-12\\
17	6.54899932503508e-12\\
18	3.07390330752401e-11\\
19	1.35138126102441e-10\\
20	5.57954452862595e-10\\
21	2.1686833167108e-09\\
22	7.9526642368932e-09\\
23	2.75679038792502e-08\\
24	9.05001310651458e-08\\
25	2.81814101710274e-07\\
26	8.33681324725063e-07\\
27	2.34620630632112e-06\\
28	6.28957500833936e-06\\
29	1.60800076478334e-05\\
30	3.92506982279687e-05\\
31	9.15716124411769e-05\\
32	0.000204388583713412\\
33	0.000436859918456193\\
34	0.000894965195743431\\
35	0.00175882086148508\\
36	0.00331856025796311\\
37	0.00601648786268185\\
38	0.0104893678389258\\
39	0.0176001001088524\\
40	0.0284439668204906\\
41	0.044313040057034\\
42	0.0666053096036057\\
43	0.0966739522478211\\
44	0.135626512036918\\
45	0.184100808663348\\
46	0.242059206803646\\
47	0.308649706794632\\
48	0.382176717201337\\
49	0.460205381306407\\
50	0.539794618693593\\
51	0.617823282798662\\
52	0.691350293205368\\
53	0.757940793196354\\
54	0.815899191336652\\
55	0.864373487963082\\
56	0.903326047752179\\
57	0.933394690396394\\
58	0.955686959942966\\
59	0.971556033179509\\
60	0.982399899891148\\
61	0.989510632161074\\
62	0.993983512137318\\
63	0.996681439742037\\
64	0.998241179138515\\
65	0.999105034804257\\
66	0.999563140081544\\
67	0.999795611416287\\
68	0.999908428387559\\
69	0.999960749301772\\
70	0.999983919992352\\
71	0.999993710424992\\
72	0.999997653793694\\
73	0.999999166318675\\
74	0.999999718185898\\
75	0.999999909499869\\
76	0.999999972432096\\
77	0.999999992047336\\
78	0.999999997831317\\
79	0.999999999442046\\
80	0.999999999864862\\
81	0.999999999969261\\
82	0.999999999993451\\
83	0.999999999998697\\
84	0.999999999999759\\
85	0.999999999999959\\
86	0.999999999999993\\
87	0.999999999999999\\
88	1\\
89	1\\
90	1\\
91	1\\
92	1\\
93	1\\
94	1\\
95	1\\
96	1\\
97	1\\
98	1\\
99	1\\
100	1\\
};
\end{axis}
\end{tikzpicture}%}      
    \end{column}
    \begin{column}{0.5\textwidth}
      \only<4,5,6>{% This file was created by matlab2tikz.
%
%The latest updates can be retrieved from
%  http://www.mathworks.com/matlabcentral/fileexchange/22022-matlab2tikz-matlab2tikz
%where you can also make suggestions and rate matlab2tikz.
%
\begin{tikzpicture}

\begin{axis}[%
width=\fwidth,
height=0.831\fheight,
at={(0\fwidth,0\fheight)},
scale only axis,
xmin=0,
xmax=1,
xlabel={Probability of less than N successes},
ymin=0,
ymax=100,
ylabel={Number of successes},
axis background/.style={fill=white}
]
\addplot [color=blue, forget plot]
  table[row sep=crcr]{%
0	0\\
0.01	38\\
0.02	40\\
0.03	41\\
0.04	41\\
0.05	42\\
0.06	42\\
0.07	43\\
0.08	43\\
0.09	43\\
0.1	44\\
0.11	44\\
0.12	44\\
0.13	44\\
0.14	45\\
0.15	45\\
0.16	45\\
0.17	45\\
0.18	45\\
0.19	46\\
0.2	46\\
0.21	46\\
0.22	46\\
0.23	46\\
0.24	46\\
0.25	47\\
0.26	47\\
0.27	47\\
0.28	47\\
0.29	47\\
0.3	47\\
0.31	48\\
0.32	48\\
0.33	48\\
0.34	48\\
0.35	48\\
0.36	48\\
0.37	48\\
0.38	48\\
0.39	49\\
0.4	49\\
0.41	49\\
0.42	49\\
0.43	49\\
0.44	49\\
0.45	49\\
0.46	49\\
0.47	50\\
0.48	50\\
0.49	50\\
0.5	50\\
0.51	50\\
0.52	50\\
0.53	50\\
0.54	51\\
0.55	51\\
0.56	51\\
0.57	51\\
0.58	51\\
0.59	51\\
0.6	51\\
0.61	51\\
0.62	52\\
0.63	52\\
0.64	52\\
0.65	52\\
0.66	52\\
0.67	52\\
0.68	52\\
0.69	52\\
0.7	53\\
0.71	53\\
0.72	53\\
0.73	53\\
0.74	53\\
0.75	53\\
0.76	54\\
0.77	54\\
0.78	54\\
0.79	54\\
0.8	54\\
0.81	54\\
0.82	55\\
0.83	55\\
0.84	55\\
0.85	55\\
0.86	55\\
0.87	56\\
0.88	56\\
0.89	56\\
0.9	56\\
0.91	57\\
0.92	57\\
0.93	57\\
0.94	58\\
0.95	58\\
0.96	59\\
0.97	59\\
0.98	60\\
0.99	62\\
1	86\\
};
\end{axis}
\end{tikzpicture}%}
    \end{column}
  \end{columns}    
\end{frame}



\begin{frame}
  \frametitle{Constructing a Null-Hypothesis test with frequentist properties}
  \index{Null-Hypothesis test!frequentist}
  \begin{block}{The test statistic}
    We want the test to reflect that we don't have a significant number of failures.
    \[
      f(x) = 1 - \textrm{binocdf}(\sum_{t=1}^n x_t, n, 0.5)
    \]
  \end{block}
  \begin{alertblock}{What $f(x)$ is and is not}
    \begin{itemize}
    \item It is a \textbf{statistic} which is $\leq \delta$ a $\delta$ portion of the time when $\model_0$ is true.
    \item It is \textbf{not} the probability of observing $x$ under $\model_0$.
    \item It is \textbf{not} the probability of $\model_0$ given $x$.
    \end{itemize}
  \end{alertblock}
\end{frame}
\begin{frame}
  \begin{exercise}
    \begin{itemize}
    \item<1-> Let us throw a coin 8 times, and try and predict the outcome.
    \item<2-> Select a $p$-value threshold so that $\delta = 0.05$. 
      For 8 throws, this corresponds to \uncover<3->{$ > 6$ successes or $\geq 87.5\%$ success rate}.
    \item<3-> Let's calculate the $p$-value for each one of you
    \item<4-> What is the rejection performance of the test?
    \end{itemize}
    \setlength\fheight{0.25\columnwidth}
    \setlength\fwidth{0.5\columnwidth}
    \only<2,3>{
      \begin{figure}[H]
        \centering
        % This file was created by matlab2tikz.
%
%The latest updates can be retrieved from
%  http://www.mathworks.com/matlabcentral/fileexchange/22022-matlab2tikz-matlab2tikz
%where you can also make suggestions and rate matlab2tikz.
%
\begin{tikzpicture}

\begin{axis}[%
width=0.951\fwidth,
height=\fheight,
at={(0\fwidth,0\fheight)},
scale only axis,
xmin=0,
xmax=1000,
ymin=0,
ymax=0.5,
axis background/.style={fill=white},
title={The rejection threshold as data increases}
]
\addplot [color=blue, forget plot]
  table[row sep=crcr]{%
1	0\\
2	0\\
3	0\\
4	0\\
5	0.2\\
6	0.166666666666667\\
7	0.142857142857143\\
8	0.25\\
9	0.222222222222222\\
10	0.2\\
11	0.272727272727273\\
12	0.25\\
13	0.307692307692308\\
14	0.285714285714286\\
15	0.266666666666667\\
16	0.3125\\
17	0.294117647058824\\
18	0.333333333333333\\
19	0.315789473684211\\
20	0.3\\
21	0.333333333333333\\
22	0.318181818181818\\
23	0.347826086956522\\
24	0.333333333333333\\
25	0.32\\
26	0.346153846153846\\
27	0.333333333333333\\
28	0.357142857142857\\
29	0.344827586206897\\
30	0.366666666666667\\
31	0.354838709677419\\
32	0.34375\\
33	0.363636363636364\\
34	0.352941176470588\\
35	0.371428571428571\\
36	0.361111111111111\\
37	0.378378378378378\\
38	0.368421052631579\\
39	0.358974358974359\\
40	0.375\\
41	0.365853658536585\\
42	0.380952380952381\\
43	0.372093023255814\\
44	0.386363636363636\\
45	0.377777777777778\\
46	0.369565217391304\\
47	0.382978723404255\\
48	0.375\\
49	0.387755102040816\\
50	0.38\\
51	0.392156862745098\\
52	0.384615384615385\\
53	0.39622641509434\\
54	0.388888888888889\\
55	0.381818181818182\\
56	0.392857142857143\\
57	0.385964912280702\\
58	0.396551724137931\\
59	0.389830508474576\\
60	0.4\\
61	0.39344262295082\\
62	0.403225806451613\\
63	0.396825396825397\\
64	0.390625\\
65	0.4\\
66	0.393939393939394\\
67	0.402985074626866\\
68	0.397058823529412\\
69	0.405797101449275\\
70	0.4\\
71	0.408450704225352\\
72	0.402777777777778\\
73	0.397260273972603\\
74	0.405405405405405\\
75	0.4\\
76	0.407894736842105\\
77	0.402597402597403\\
78	0.41025641025641\\
79	0.405063291139241\\
80	0.4125\\
81	0.407407407407407\\
82	0.414634146341463\\
83	0.409638554216867\\
84	0.404761904761905\\
85	0.411764705882353\\
86	0.406976744186047\\
87	0.413793103448276\\
88	0.409090909090909\\
89	0.415730337078652\\
90	0.411111111111111\\
91	0.417582417582418\\
92	0.41304347826087\\
93	0.419354838709677\\
94	0.414893617021277\\
95	0.410526315789474\\
96	0.416666666666667\\
97	0.412371134020619\\
98	0.418367346938776\\
99	0.414141414141414\\
100	0.42\\
101	0.415841584158416\\
102	0.42156862745098\\
103	0.41747572815534\\
104	0.423076923076923\\
105	0.419047619047619\\
106	0.424528301886792\\
107	0.420560747663551\\
108	0.416666666666667\\
109	0.422018348623853\\
110	0.418181818181818\\
111	0.423423423423423\\
112	0.419642857142857\\
113	0.424778761061947\\
114	0.421052631578947\\
115	0.426086956521739\\
116	0.422413793103448\\
117	0.427350427350427\\
118	0.423728813559322\\
119	0.428571428571429\\
120	0.425\\
121	0.421487603305785\\
122	0.426229508196721\\
123	0.422764227642276\\
124	0.42741935483871\\
125	0.424\\
126	0.428571428571429\\
127	0.425196850393701\\
128	0.4296875\\
129	0.426356589147287\\
130	0.430769230769231\\
131	0.427480916030534\\
132	0.431818181818182\\
133	0.428571428571429\\
134	0.425373134328358\\
135	0.42962962962963\\
136	0.426470588235294\\
137	0.430656934306569\\
138	0.427536231884058\\
139	0.431654676258993\\
140	0.428571428571429\\
141	0.432624113475177\\
142	0.429577464788732\\
143	0.433566433566434\\
144	0.430555555555556\\
145	0.43448275862069\\
146	0.431506849315068\\
147	0.435374149659864\\
148	0.432432432432432\\
149	0.429530201342282\\
150	0.433333333333333\\
151	0.43046357615894\\
152	0.434210526315789\\
153	0.431372549019608\\
154	0.435064935064935\\
155	0.432258064516129\\
156	0.435897435897436\\
157	0.43312101910828\\
158	0.436708860759494\\
159	0.433962264150943\\
160	0.4375\\
161	0.434782608695652\\
162	0.438271604938272\\
163	0.43558282208589\\
164	0.432926829268293\\
165	0.436363636363636\\
166	0.433734939759036\\
167	0.437125748502994\\
168	0.43452380952381\\
169	0.437869822485207\\
170	0.435294117647059\\
171	0.43859649122807\\
172	0.436046511627907\\
173	0.439306358381503\\
174	0.436781609195402\\
175	0.44\\
176	0.4375\\
177	0.440677966101695\\
178	0.438202247191011\\
179	0.441340782122905\\
180	0.438888888888889\\
181	0.43646408839779\\
182	0.43956043956044\\
183	0.437158469945355\\
184	0.440217391304348\\
185	0.437837837837838\\
186	0.440860215053763\\
187	0.438502673796791\\
188	0.441489361702128\\
189	0.439153439153439\\
190	0.442105263157895\\
191	0.43979057591623\\
192	0.442708333333333\\
193	0.440414507772021\\
194	0.443298969072165\\
195	0.441025641025641\\
196	0.438775510204082\\
197	0.441624365482233\\
198	0.439393939393939\\
199	0.442211055276382\\
200	0.44\\
201	0.442786069651741\\
202	0.440594059405941\\
203	0.443349753694581\\
204	0.441176470588235\\
205	0.44390243902439\\
206	0.441747572815534\\
207	0.444444444444444\\
208	0.442307692307692\\
209	0.444976076555024\\
210	0.442857142857143\\
211	0.445497630331754\\
212	0.443396226415094\\
213	0.446009389671362\\
214	0.44392523364486\\
215	0.441860465116279\\
216	0.444444444444444\\
217	0.442396313364055\\
218	0.444954128440367\\
219	0.442922374429224\\
220	0.445454545454545\\
221	0.443438914027149\\
222	0.445945945945946\\
223	0.443946188340807\\
224	0.446428571428571\\
225	0.444444444444444\\
226	0.446902654867257\\
227	0.444933920704846\\
228	0.447368421052632\\
229	0.445414847161572\\
230	0.447826086956522\\
231	0.445887445887446\\
232	0.443965517241379\\
233	0.446351931330472\\
234	0.444444444444444\\
235	0.446808510638298\\
236	0.444915254237288\\
237	0.447257383966245\\
238	0.445378151260504\\
239	0.447698744769874\\
240	0.445833333333333\\
241	0.448132780082988\\
242	0.446280991735537\\
243	0.448559670781893\\
244	0.44672131147541\\
245	0.448979591836735\\
246	0.447154471544715\\
247	0.449392712550607\\
248	0.44758064516129\\
249	0.449799196787149\\
250	0.448\\
251	0.446215139442231\\
252	0.448412698412698\\
253	0.446640316205534\\
254	0.448818897637795\\
255	0.447058823529412\\
256	0.44921875\\
257	0.447470817120623\\
258	0.449612403100775\\
259	0.447876447876448\\
260	0.45\\
261	0.448275862068966\\
262	0.450381679389313\\
263	0.448669201520913\\
264	0.450757575757576\\
265	0.449056603773585\\
266	0.451127819548872\\
267	0.449438202247191\\
268	0.451492537313433\\
269	0.449814126394052\\
270	0.448148148148148\\
271	0.450184501845018\\
272	0.448529411764706\\
273	0.450549450549451\\
274	0.448905109489051\\
275	0.450909090909091\\
276	0.449275362318841\\
277	0.451263537906137\\
278	0.449640287769784\\
279	0.451612903225806\\
280	0.45\\
281	0.451957295373665\\
282	0.450354609929078\\
283	0.452296819787986\\
284	0.450704225352113\\
285	0.452631578947368\\
286	0.451048951048951\\
287	0.452961672473868\\
288	0.451388888888889\\
289	0.453287197231834\\
290	0.451724137931034\\
291	0.450171821305842\\
292	0.452054794520548\\
293	0.450511945392491\\
294	0.452380952380952\\
295	0.450847457627119\\
296	0.452702702702703\\
297	0.451178451178451\\
298	0.453020134228188\\
299	0.451505016722408\\
300	0.453333333333333\\
301	0.451827242524917\\
302	0.45364238410596\\
303	0.452145214521452\\
304	0.453947368421053\\
305	0.452459016393443\\
306	0.454248366013072\\
307	0.452768729641694\\
308	0.454545454545455\\
309	0.453074433656958\\
310	0.454838709677419\\
311	0.453376205787781\\
312	0.451923076923077\\
313	0.453674121405751\\
314	0.452229299363057\\
315	0.453968253968254\\
316	0.45253164556962\\
317	0.454258675078864\\
318	0.452830188679245\\
319	0.454545454545455\\
320	0.453125\\
321	0.454828660436137\\
322	0.453416149068323\\
323	0.455108359133127\\
324	0.453703703703704\\
325	0.455384615384615\\
326	0.45398773006135\\
327	0.45565749235474\\
328	0.454268292682927\\
329	0.455927051671733\\
330	0.454545454545455\\
331	0.45619335347432\\
332	0.454819277108434\\
333	0.453453453453453\\
334	0.455089820359281\\
335	0.453731343283582\\
336	0.455357142857143\\
337	0.454005934718101\\
338	0.455621301775148\\
339	0.454277286135693\\
340	0.455882352941176\\
341	0.454545454545455\\
342	0.456140350877193\\
343	0.454810495626822\\
344	0.456395348837209\\
345	0.455072463768116\\
346	0.456647398843931\\
347	0.455331412103746\\
348	0.456896551724138\\
349	0.455587392550143\\
350	0.457142857142857\\
351	0.455840455840456\\
352	0.457386363636364\\
353	0.456090651558074\\
354	0.457627118644068\\
355	0.456338028169014\\
356	0.455056179775281\\
357	0.456582633053221\\
358	0.455307262569832\\
359	0.456824512534819\\
360	0.455555555555556\\
361	0.457063711911357\\
362	0.455801104972376\\
363	0.457300275482094\\
364	0.456043956043956\\
365	0.457534246575342\\
366	0.456284153005464\\
367	0.457765667574932\\
368	0.456521739130435\\
369	0.457994579945799\\
370	0.456756756756757\\
371	0.45822102425876\\
372	0.456989247311828\\
373	0.458445040214477\\
374	0.457219251336898\\
375	0.458666666666667\\
376	0.457446808510638\\
377	0.458885941644562\\
378	0.457671957671958\\
379	0.45646437994723\\
380	0.457894736842105\\
381	0.456692913385827\\
382	0.458115183246073\\
383	0.456919060052219\\
384	0.458333333333333\\
385	0.457142857142857\\
386	0.458549222797927\\
387	0.457364341085271\\
388	0.458762886597938\\
389	0.457583547557841\\
390	0.458974358974359\\
391	0.457800511508951\\
392	0.459183673469388\\
393	0.458015267175573\\
394	0.459390862944162\\
395	0.458227848101266\\
396	0.45959595959596\\
397	0.458438287153652\\
398	0.459798994974874\\
399	0.458646616541353\\
400	0.46\\
401	0.458852867830424\\
402	0.460199004975124\\
403	0.459057071960298\\
404	0.457920792079208\\
405	0.459259259259259\\
406	0.458128078817734\\
407	0.459459459459459\\
408	0.458333333333333\\
409	0.459657701711491\\
410	0.458536585365854\\
411	0.45985401459854\\
412	0.45873786407767\\
413	0.460048426150121\\
414	0.458937198067633\\
415	0.460240963855422\\
416	0.459134615384615\\
417	0.460431654676259\\
418	0.45933014354067\\
419	0.460620525059666\\
420	0.45952380952381\\
421	0.460807600950119\\
422	0.459715639810427\\
423	0.460992907801418\\
424	0.459905660377358\\
425	0.461176470588235\\
426	0.460093896713615\\
427	0.46135831381733\\
428	0.460280373831776\\
429	0.459207459207459\\
430	0.46046511627907\\
431	0.459396751740139\\
432	0.460648148148148\\
433	0.459584295612009\\
434	0.460829493087558\\
435	0.459770114942529\\
436	0.461009174311927\\
437	0.459954233409611\\
438	0.461187214611872\\
439	0.460136674259681\\
440	0.461363636363636\\
441	0.46031746031746\\
442	0.461538461538462\\
443	0.460496613995485\\
444	0.461711711711712\\
445	0.460674157303371\\
446	0.461883408071749\\
447	0.460850111856823\\
448	0.462053571428571\\
449	0.461024498886414\\
450	0.462222222222222\\
451	0.46119733924612\\
452	0.462389380530973\\
453	0.461368653421634\\
454	0.460352422907489\\
455	0.461538461538462\\
456	0.460526315789474\\
457	0.461706783369803\\
458	0.460698689956332\\
459	0.461873638344227\\
460	0.460869565217391\\
461	0.462039045553145\\
462	0.461038961038961\\
463	0.462203023758099\\
464	0.461206896551724\\
465	0.462365591397849\\
466	0.46137339055794\\
467	0.462526766595289\\
468	0.461538461538462\\
469	0.462686567164179\\
470	0.461702127659574\\
471	0.462845010615711\\
472	0.461864406779661\\
473	0.463002114164905\\
474	0.462025316455696\\
475	0.463157894736842\\
476	0.46218487394958\\
477	0.463312368972746\\
478	0.46234309623431\\
479	0.463465553235908\\
480	0.4625\\
481	0.461538461538462\\
482	0.462655601659751\\
483	0.461697722567288\\
484	0.462809917355372\\
485	0.461855670103093\\
486	0.462962962962963\\
487	0.462012320328542\\
488	0.463114754098361\\
489	0.462167689161554\\
490	0.463265306122449\\
491	0.462321792260692\\
492	0.463414634146341\\
493	0.462474645030426\\
494	0.463562753036437\\
495	0.462626262626263\\
496	0.463709677419355\\
497	0.462776659959759\\
498	0.463855421686747\\
499	0.462925851703407\\
500	0.464\\
501	0.463073852295409\\
502	0.464143426294821\\
503	0.463220675944334\\
504	0.464285714285714\\
505	0.463366336633663\\
506	0.464426877470356\\
507	0.463510848126233\\
508	0.46259842519685\\
509	0.463654223968566\\
510	0.462745098039216\\
511	0.463796477495108\\
512	0.462890625\\
513	0.463937621832359\\
514	0.463035019455253\\
515	0.464077669902913\\
516	0.463178294573643\\
517	0.4642166344294\\
518	0.463320463320463\\
519	0.464354527938343\\
520	0.463461538461538\\
521	0.464491362763916\\
522	0.46360153256705\\
523	0.464627151051625\\
524	0.463740458015267\\
525	0.464761904761905\\
526	0.463878326996198\\
527	0.464895635673624\\
528	0.464015151515151\\
529	0.465028355387524\\
530	0.464150943396226\\
531	0.465160075329567\\
532	0.464285714285714\\
533	0.465290806754221\\
534	0.464419475655431\\
535	0.463551401869159\\
536	0.46455223880597\\
537	0.463687150837989\\
538	0.464684014869888\\
539	0.463821892393321\\
540	0.464814814814815\\
541	0.463955637707948\\
542	0.464944649446494\\
543	0.464088397790055\\
544	0.465073529411765\\
545	0.464220183486239\\
546	0.465201465201465\\
547	0.464351005484461\\
548	0.465328467153285\\
549	0.46448087431694\\
550	0.465454545454545\\
551	0.464609800362976\\
552	0.465579710144928\\
553	0.464737793851718\\
554	0.465703971119134\\
555	0.464864864864865\\
556	0.465827338129496\\
557	0.464991023339318\\
558	0.46594982078853\\
559	0.465116279069767\\
560	0.466071428571429\\
561	0.46524064171123\\
562	0.466192170818505\\
563	0.465364120781528\\
564	0.464539007092199\\
565	0.465486725663717\\
566	0.464664310954064\\
567	0.465608465608466\\
568	0.464788732394366\\
569	0.46572934973638\\
570	0.464912280701754\\
571	0.46584938704028\\
572	0.465034965034965\\
573	0.465968586387435\\
574	0.465156794425087\\
575	0.466086956521739\\
576	0.465277777777778\\
577	0.466204506065858\\
578	0.465397923875433\\
579	0.466321243523316\\
580	0.46551724137931\\
581	0.466437177280551\\
582	0.465635738831615\\
583	0.466552315608919\\
584	0.465753424657534\\
585	0.466666666666667\\
586	0.465870307167235\\
587	0.466780238500852\\
588	0.465986394557823\\
589	0.466893039049236\\
590	0.466101694915254\\
591	0.467005076142132\\
592	0.466216216216216\\
593	0.465430016863406\\
594	0.466329966329966\\
595	0.465546218487395\\
596	0.466442953020134\\
597	0.465661641541039\\
598	0.466555183946488\\
599	0.465776293823038\\
600	0.466666666666667\\
601	0.465890183028286\\
602	0.466777408637874\\
603	0.466003316749585\\
604	0.466887417218543\\
605	0.466115702479339\\
606	0.466996699669967\\
607	0.466227347611203\\
608	0.467105263157895\\
609	0.466338259441708\\
610	0.467213114754098\\
611	0.466448445171849\\
612	0.467320261437909\\
613	0.466557911908646\\
614	0.46742671009772\\
615	0.466666666666667\\
616	0.467532467532468\\
617	0.46677471636953\\
618	0.467637540453074\\
619	0.466882067851373\\
620	0.467741935483871\\
621	0.466988727858293\\
622	0.466237942122186\\
623	0.467094703049759\\
624	0.466346153846154\\
625	0.4672\\
626	0.466453674121406\\
627	0.467304625199362\\
628	0.46656050955414\\
629	0.467408585055644\\
630	0.466666666666667\\
631	0.467511885895404\\
632	0.466772151898734\\
633	0.467614533965245\\
634	0.466876971608833\\
635	0.467716535433071\\
636	0.466981132075472\\
637	0.467817896389325\\
638	0.467084639498433\\
639	0.4679186228482\\
640	0.4671875\\
641	0.46801872074883\\
642	0.467289719626168\\
643	0.468118195956454\\
644	0.467391304347826\\
645	0.468217054263566\\
646	0.46749226006192\\
647	0.468315301391036\\
648	0.467592592592593\\
649	0.468412942989214\\
650	0.467692307692308\\
651	0.468509984639017\\
652	0.467791411042945\\
653	0.467075038284839\\
654	0.467889908256881\\
655	0.467175572519084\\
656	0.467987804878049\\
657	0.467275494672755\\
658	0.468085106382979\\
659	0.467374810318665\\
660	0.468181818181818\\
661	0.467473524962179\\
662	0.468277945619335\\
663	0.467571644042232\\
664	0.468373493975904\\
665	0.467669172932331\\
666	0.468468468468468\\
667	0.467766116941529\\
668	0.468562874251497\\
669	0.467862481315396\\
670	0.46865671641791\\
671	0.46795827123696\\
672	0.46875\\
673	0.468053491827637\\
674	0.468842729970326\\
675	0.468148148148148\\
676	0.468934911242604\\
677	0.468242245199409\\
678	0.469026548672566\\
679	0.468335787923417\\
680	0.469117647058824\\
681	0.468428781204112\\
682	0.469208211143695\\
683	0.468521229868228\\
684	0.467836257309941\\
685	0.468613138686131\\
686	0.467930029154519\\
687	0.468704512372635\\
688	0.468023255813953\\
689	0.468795355587808\\
690	0.468115942028985\\
691	0.468885672937771\\
692	0.468208092485549\\
693	0.468975468975469\\
694	0.468299711815562\\
695	0.469064748201439\\
696	0.468390804597701\\
697	0.469153515064562\\
698	0.468481375358166\\
699	0.469241773962804\\
700	0.468571428571429\\
701	0.469329529243937\\
702	0.468660968660969\\
703	0.469416785206259\\
704	0.46875\\
705	0.469503546099291\\
706	0.468838526912181\\
707	0.46958981612447\\
708	0.468926553672316\\
709	0.469675599435825\\
710	0.469014084507042\\
711	0.469760900140647\\
712	0.469101123595506\\
713	0.46984572230014\\
714	0.469187675070028\\
715	0.46993006993007\\
716	0.46927374301676\\
717	0.468619246861925\\
718	0.469359331476323\\
719	0.468706536856745\\
720	0.469444444444444\\
721	0.46879334257975\\
722	0.469529085872576\\
723	0.468879668049793\\
724	0.469613259668508\\
725	0.468965517241379\\
726	0.46969696969697\\
727	0.469050894085282\\
728	0.46978021978022\\
729	0.469135802469136\\
730	0.46986301369863\\
731	0.46922024623803\\
732	0.469945355191257\\
733	0.469304229195089\\
734	0.470027247956403\\
735	0.469387755102041\\
736	0.470108695652174\\
737	0.469470827679783\\
738	0.470189701897019\\
739	0.469553450608931\\
740	0.47027027027027\\
741	0.469635627530364\\
742	0.470350404312668\\
743	0.46971736204576\\
744	0.470430107526882\\
745	0.469798657718121\\
746	0.470509383378016\\
747	0.469879518072289\\
748	0.470588235294118\\
749	0.469959946595461\\
750	0.469333333333333\\
751	0.470039946737683\\
752	0.469414893617021\\
753	0.470119521912351\\
754	0.469496021220159\\
755	0.470198675496689\\
756	0.46957671957672\\
757	0.470277410832233\\
758	0.469656992084433\\
759	0.470355731225296\\
760	0.469736842105263\\
761	0.470433639947438\\
762	0.469816272965879\\
763	0.470511140235911\\
764	0.469895287958115\\
765	0.470588235294118\\
766	0.469973890339426\\
767	0.470664928292047\\
768	0.470052083333333\\
769	0.47074122236671\\
770	0.47012987012987\\
771	0.470817120622568\\
772	0.47020725388601\\
773	0.470892626131953\\
774	0.470284237726098\\
775	0.470967741935484\\
776	0.470360824742268\\
777	0.471042471042471\\
778	0.470437017994859\\
779	0.471116816431322\\
780	0.470512820512821\\
781	0.471190781049936\\
782	0.470588235294118\\
783	0.469987228607918\\
784	0.470663265306122\\
785	0.470063694267516\\
786	0.470737913486005\\
787	0.470139771283354\\
788	0.470812182741117\\
789	0.4702154626109\\
790	0.470886075949367\\
791	0.470290771175727\\
792	0.470959595959596\\
793	0.470365699873897\\
794	0.47103274559194\\
795	0.470440251572327\\
796	0.471105527638191\\
797	0.470514429109159\\
798	0.471177944862155\\
799	0.470588235294118\\
800	0.47125\\
801	0.470661672908864\\
802	0.471321695760599\\
803	0.470734744707347\\
804	0.471393034825871\\
805	0.470807453416149\\
806	0.471464019851117\\
807	0.47087980173482\\
808	0.471534653465347\\
809	0.470951792336218\\
810	0.471604938271605\\
811	0.471023427866831\\
812	0.471674876847291\\
813	0.471094710947109\\
814	0.471744471744472\\
815	0.471165644171779\\
816	0.471813725490196\\
817	0.471236230110159\\
818	0.470660146699266\\
819	0.471306471306471\\
820	0.470731707317073\\
821	0.471376370280146\\
822	0.470802919708029\\
823	0.471445929526124\\
824	0.470873786407767\\
825	0.471515151515151\\
826	0.470944309927361\\
827	0.471584038694075\\
828	0.471014492753623\\
829	0.471652593486128\\
830	0.471084337349398\\
831	0.471720818291215\\
832	0.471153846153846\\
833	0.471788715486194\\
834	0.471223021582734\\
835	0.47185628742515\\
836	0.471291866028708\\
837	0.471923536439665\\
838	0.471360381861575\\
839	0.471990464839094\\
840	0.471428571428571\\
841	0.47205707491082\\
842	0.471496437054632\\
843	0.472123368920522\\
844	0.471563981042654\\
845	0.472189349112426\\
846	0.471631205673759\\
847	0.472255017709563\\
848	0.471698113207547\\
849	0.472320376914017\\
850	0.471764705882353\\
851	0.472385428907168\\
852	0.471830985915493\\
853	0.471277842907386\\
854	0.471896955503513\\
855	0.471345029239766\\
856	0.47196261682243\\
857	0.471411901983664\\
858	0.472027972027972\\
859	0.471478463329453\\
860	0.472093023255814\\
861	0.471544715447154\\
862	0.47215777262181\\
863	0.471610660486674\\
864	0.472222222222222\\
865	0.471676300578035\\
866	0.472286374133949\\
867	0.471741637831603\\
868	0.472350230414747\\
869	0.47180667433832\\
870	0.472413793103448\\
871	0.47187141216992\\
872	0.472477064220183\\
873	0.471935853379152\\
874	0.47254004576659\\
875	0.472\\
876	0.472602739726027\\
877	0.472063854047891\\
878	0.472665148063781\\
879	0.472127417519909\\
880	0.472727272727273\\
881	0.472190692395006\\
882	0.472789115646259\\
883	0.472253680634202\\
884	0.472850678733032\\
885	0.472316384180791\\
886	0.472911963882619\\
887	0.472378804960541\\
888	0.471846846846847\\
889	0.47244094488189\\
890	0.471910112359551\\
891	0.472502805836139\\
892	0.471973094170404\\
893	0.472564389697648\\
894	0.472035794183445\\
895	0.472625698324022\\
896	0.472098214285714\\
897	0.472686733556299\\
898	0.472160356347439\\
899	0.472747497219132\\
900	0.472222222222222\\
901	0.472807991120977\\
902	0.472283813747228\\
903	0.472868217054264\\
904	0.472345132743363\\
905	0.47292817679558\\
906	0.472406181015453\\
907	0.472987872105843\\
908	0.472466960352423\\
909	0.473047304730473\\
910	0.472527472527473\\
911	0.473106476399561\\
912	0.472587719298246\\
913	0.473165388828039\\
914	0.472647702407002\\
915	0.473224043715847\\
916	0.472707423580786\\
917	0.473282442748092\\
918	0.47276688453159\\
919	0.473340587595212\\
920	0.472826086956522\\
921	0.473398479913138\\
922	0.472885032537961\\
923	0.473456121343445\\
924	0.472943722943723\\
925	0.472432432432432\\
926	0.473002159827214\\
927	0.472491909385113\\
928	0.473060344827586\\
929	0.472551130247578\\
930	0.473118279569892\\
931	0.472610096670247\\
932	0.473175965665236\\
933	0.472668810289389\\
934	0.473233404710921\\
935	0.472727272727273\\
936	0.473290598290598\\
937	0.472785485592316\\
938	0.473347547974414\\
939	0.472843450479233\\
940	0.473404255319149\\
941	0.472901168969182\\
942	0.473460721868365\\
943	0.472958642629905\\
944	0.473516949152542\\
945	0.473015873015873\\
946	0.473572938689218\\
947	0.473072861668427\\
948	0.473628691983122\\
949	0.473129610115911\\
950	0.473684210526316\\
951	0.473186119873817\\
952	0.473739495798319\\
953	0.473242392444911\\
954	0.473794549266247\\
955	0.473298429319372\\
956	0.473849372384937\\
957	0.473354231974922\\
958	0.473903966597077\\
959	0.473409801876955\\
960	0.473958333333333\\
961	0.473465140478668\\
962	0.472972972972973\\
963	0.473520249221184\\
964	0.473029045643154\\
965	0.473575129533679\\
966	0.473084886128364\\
967	0.473629782833506\\
968	0.473140495867769\\
969	0.473684210526316\\
970	0.47319587628866\\
971	0.473738414006179\\
972	0.473251028806584\\
973	0.473792394655704\\
974	0.473305954825462\\
975	0.473846153846154\\
976	0.473360655737705\\
977	0.473899692937564\\
978	0.473415132924335\\
979	0.473953013278856\\
980	0.473469387755102\\
981	0.474006116207951\\
982	0.473523421588595\\
983	0.474059003051882\\
984	0.473577235772358\\
985	0.474111675126904\\
986	0.473630831643002\\
987	0.474164133738602\\
988	0.473684210526316\\
989	0.474216380182002\\
990	0.473737373737374\\
991	0.474268415741675\\
992	0.473790322580645\\
993	0.474320241691843\\
994	0.473843058350101\\
995	0.474371859296482\\
996	0.473895582329317\\
997	0.474423269809428\\
998	0.473947895791583\\
999	0.474474474474474\\
1000	0.474\\
};
\end{axis}
\end{tikzpicture}%
        \caption{Here we see how the rejection threshold, in terms of the success rate, changes with the number of throws to achieve an error rate of $\delta = 0.05$.}
      \end{figure}
      \only<article>{As the amount of throws goes to infinity, the threshold converges to $0.5$. This means that a statistically significant difference from the null hypothesis can be obtained, even when the actual model from which the data is drawn is only slightly different from 0.5.}
    }
    \only<4>{
      \begin{figure}[H]
        \centering
        % This file was created by matlab2tikz.
%
%The latest updates can be retrieved from
%  http://www.mathworks.com/matlabcentral/fileexchange/22022-matlab2tikz-matlab2tikz
%where you can also make suggestions and rate matlab2tikz.
%
\begin{tikzpicture}

\begin{axis}[%
width=0.951\fwidth,
height=\fheight,
at={(0\fwidth,0\fheight)},
scale only axis,
xmin=0,
xmax=1000,
ymin=0,
ymax=1,
axis background/.style={fill=white},
title={How often we reject the null hypothesis},
legend style={legend cell align=left, align=left, legend plot pos=left, draw=black}
]
\addplot [color=blue]
  table[row sep=crcr]{%
1	0\\
2	0\\
3	0\\
4	0\\
5	0.042\\
6	0.022\\
7	0.014\\
8	0.042\\
9	0.027\\
10	0.014\\
11	0.04\\
12	0.025\\
13	0.061\\
14	0.036\\
15	0.018\\
16	0.047\\
17	0.026\\
18	0.055\\
19	0.039\\
20	0.026\\
21	0.044\\
22	0.035\\
23	0.05\\
24	0.035\\
25	0.025\\
26	0.045\\
27	0.034\\
28	0.054\\
29	0.038\\
30	0.065\\
31	0.046\\
32	0.028\\
33	0.052\\
34	0.03\\
35	0.05\\
36	0.031\\
37	0.053\\
38	0.036\\
39	0.026\\
40	0.046\\
41	0.031\\
42	0.053\\
43	0.038\\
44	0.055\\
45	0.039\\
46	0.024\\
47	0.036\\
48	0.026\\
49	0.037\\
50	0.027\\
51	0.039\\
52	0.032\\
53	0.047\\
54	0.038\\
55	0.027\\
56	0.041\\
57	0.033\\
58	0.048\\
59	0.033\\
60	0.045\\
61	0.037\\
62	0.06\\
63	0.04\\
64	0.029\\
65	0.045\\
66	0.036\\
67	0.05\\
68	0.041\\
69	0.053\\
70	0.046\\
71	0.058\\
72	0.044\\
73	0.036\\
74	0.045\\
75	0.039\\
76	0.046\\
77	0.04\\
78	0.054\\
79	0.045\\
80	0.054\\
81	0.046\\
82	0.053\\
83	0.043\\
84	0.029\\
85	0.041\\
86	0.03\\
87	0.042\\
88	0.035\\
89	0.046\\
90	0.033\\
91	0.042\\
92	0.035\\
93	0.051\\
94	0.044\\
95	0.04\\
96	0.05\\
97	0.036\\
98	0.047\\
99	0.036\\
100	0.047\\
101	0.032\\
102	0.047\\
103	0.038\\
104	0.049\\
105	0.041\\
106	0.053\\
107	0.043\\
108	0.036\\
109	0.047\\
110	0.042\\
111	0.048\\
112	0.04\\
113	0.045\\
114	0.038\\
115	0.047\\
116	0.043\\
117	0.049\\
118	0.035\\
119	0.048\\
120	0.04\\
121	0.026\\
122	0.037\\
123	0.026\\
124	0.035\\
125	0.027\\
126	0.043\\
127	0.031\\
128	0.038\\
129	0.035\\
130	0.045\\
131	0.039\\
132	0.051\\
133	0.046\\
134	0.038\\
135	0.044\\
136	0.037\\
137	0.044\\
138	0.033\\
139	0.048\\
140	0.038\\
141	0.046\\
142	0.037\\
143	0.047\\
144	0.041\\
145	0.046\\
146	0.045\\
147	0.052\\
148	0.047\\
149	0.041\\
150	0.043\\
151	0.037\\
152	0.05\\
153	0.046\\
154	0.05\\
155	0.045\\
156	0.05\\
157	0.044\\
158	0.05\\
159	0.047\\
160	0.052\\
161	0.045\\
162	0.05\\
163	0.046\\
164	0.037\\
165	0.046\\
166	0.037\\
167	0.05\\
168	0.041\\
169	0.047\\
170	0.041\\
171	0.053\\
172	0.048\\
173	0.051\\
174	0.046\\
175	0.049\\
176	0.046\\
177	0.054\\
178	0.047\\
179	0.053\\
180	0.044\\
181	0.036\\
182	0.043\\
183	0.038\\
184	0.041\\
185	0.04\\
186	0.046\\
187	0.041\\
188	0.048\\
189	0.045\\
190	0.05\\
191	0.044\\
192	0.048\\
193	0.043\\
194	0.052\\
195	0.043\\
196	0.036\\
197	0.045\\
198	0.039\\
199	0.047\\
200	0.041\\
201	0.047\\
202	0.04\\
203	0.048\\
204	0.041\\
205	0.05\\
206	0.042\\
207	0.047\\
208	0.04\\
209	0.046\\
210	0.039\\
211	0.044\\
212	0.039\\
213	0.047\\
214	0.041\\
215	0.034\\
216	0.043\\
217	0.037\\
218	0.044\\
219	0.035\\
220	0.042\\
221	0.04\\
222	0.045\\
223	0.038\\
224	0.049\\
225	0.04\\
226	0.047\\
227	0.041\\
228	0.051\\
229	0.041\\
230	0.052\\
231	0.044\\
232	0.035\\
233	0.04\\
234	0.036\\
235	0.039\\
236	0.034\\
237	0.041\\
238	0.037\\
239	0.041\\
240	0.035\\
241	0.04\\
242	0.034\\
243	0.043\\
244	0.035\\
245	0.04\\
246	0.037\\
247	0.041\\
248	0.035\\
249	0.041\\
250	0.033\\
251	0.033\\
252	0.038\\
253	0.036\\
254	0.039\\
255	0.033\\
256	0.038\\
257	0.036\\
258	0.041\\
259	0.035\\
260	0.041\\
261	0.031\\
262	0.04\\
263	0.033\\
264	0.042\\
265	0.034\\
266	0.044\\
267	0.038\\
268	0.043\\
269	0.033\\
270	0.028\\
271	0.036\\
272	0.028\\
273	0.033\\
274	0.027\\
275	0.033\\
276	0.028\\
277	0.033\\
278	0.028\\
279	0.038\\
280	0.03\\
281	0.039\\
282	0.03\\
283	0.038\\
284	0.031\\
285	0.038\\
286	0.035\\
287	0.037\\
288	0.032\\
289	0.038\\
290	0.031\\
291	0.028\\
292	0.033\\
293	0.032\\
294	0.033\\
295	0.031\\
296	0.035\\
297	0.03\\
298	0.038\\
299	0.036\\
300	0.038\\
301	0.036\\
302	0.039\\
303	0.037\\
304	0.041\\
305	0.04\\
306	0.044\\
307	0.04\\
308	0.044\\
309	0.041\\
310	0.042\\
311	0.038\\
312	0.033\\
313	0.038\\
314	0.033\\
315	0.039\\
316	0.036\\
317	0.04\\
318	0.037\\
319	0.042\\
320	0.037\\
321	0.041\\
322	0.036\\
323	0.041\\
324	0.037\\
325	0.042\\
326	0.036\\
327	0.039\\
328	0.035\\
329	0.04\\
330	0.037\\
331	0.045\\
332	0.037\\
333	0.033\\
334	0.04\\
335	0.036\\
336	0.041\\
337	0.039\\
338	0.044\\
339	0.042\\
340	0.043\\
341	0.039\\
342	0.043\\
343	0.04\\
344	0.043\\
345	0.038\\
346	0.041\\
347	0.039\\
348	0.044\\
349	0.042\\
350	0.047\\
351	0.042\\
352	0.046\\
353	0.042\\
354	0.049\\
355	0.04\\
356	0.035\\
357	0.041\\
358	0.037\\
359	0.039\\
360	0.038\\
361	0.039\\
362	0.035\\
363	0.04\\
364	0.034\\
365	0.039\\
366	0.036\\
367	0.039\\
368	0.035\\
369	0.042\\
370	0.032\\
371	0.04\\
372	0.033\\
373	0.039\\
374	0.036\\
375	0.037\\
376	0.031\\
377	0.035\\
378	0.034\\
379	0.029\\
380	0.032\\
381	0.028\\
382	0.033\\
383	0.028\\
384	0.033\\
385	0.03\\
386	0.035\\
387	0.026\\
388	0.033\\
389	0.028\\
390	0.03\\
391	0.029\\
392	0.032\\
393	0.031\\
394	0.034\\
395	0.033\\
396	0.038\\
397	0.031\\
398	0.033\\
399	0.03\\
400	0.035\\
401	0.028\\
402	0.037\\
403	0.033\\
404	0.031\\
405	0.032\\
406	0.027\\
407	0.032\\
408	0.03\\
409	0.034\\
410	0.03\\
411	0.035\\
412	0.029\\
413	0.035\\
414	0.03\\
415	0.037\\
416	0.035\\
417	0.038\\
418	0.032\\
419	0.038\\
420	0.033\\
421	0.035\\
422	0.034\\
423	0.038\\
424	0.036\\
425	0.038\\
426	0.033\\
427	0.036\\
428	0.033\\
429	0.029\\
430	0.032\\
431	0.032\\
432	0.034\\
433	0.03\\
434	0.032\\
435	0.028\\
436	0.034\\
437	0.031\\
438	0.037\\
439	0.034\\
440	0.039\\
441	0.033\\
442	0.038\\
443	0.032\\
444	0.037\\
445	0.034\\
446	0.04\\
447	0.034\\
448	0.04\\
449	0.037\\
450	0.04\\
451	0.035\\
452	0.038\\
453	0.036\\
454	0.031\\
455	0.034\\
456	0.028\\
457	0.031\\
458	0.031\\
459	0.032\\
460	0.031\\
461	0.034\\
462	0.028\\
463	0.031\\
464	0.029\\
465	0.033\\
466	0.029\\
467	0.037\\
468	0.033\\
469	0.038\\
470	0.036\\
471	0.038\\
472	0.037\\
473	0.038\\
474	0.036\\
475	0.037\\
476	0.037\\
477	0.039\\
478	0.036\\
479	0.038\\
480	0.035\\
481	0.032\\
482	0.039\\
483	0.033\\
484	0.039\\
485	0.036\\
486	0.043\\
487	0.036\\
488	0.041\\
489	0.039\\
490	0.042\\
491	0.038\\
492	0.04\\
493	0.036\\
494	0.04\\
495	0.036\\
496	0.041\\
497	0.038\\
498	0.041\\
499	0.036\\
500	0.043\\
501	0.038\\
502	0.044\\
503	0.04\\
504	0.045\\
505	0.042\\
506	0.044\\
507	0.042\\
508	0.038\\
509	0.043\\
510	0.038\\
511	0.038\\
512	0.035\\
513	0.037\\
514	0.034\\
515	0.038\\
516	0.035\\
517	0.039\\
518	0.037\\
519	0.039\\
520	0.036\\
521	0.041\\
522	0.035\\
523	0.039\\
524	0.036\\
525	0.04\\
526	0.038\\
527	0.04\\
528	0.037\\
529	0.042\\
530	0.038\\
531	0.043\\
532	0.039\\
533	0.042\\
534	0.04\\
535	0.038\\
536	0.04\\
537	0.038\\
538	0.04\\
539	0.037\\
540	0.04\\
541	0.037\\
542	0.039\\
543	0.036\\
544	0.038\\
545	0.038\\
546	0.04\\
547	0.037\\
548	0.041\\
549	0.037\\
550	0.042\\
551	0.038\\
552	0.045\\
553	0.041\\
554	0.045\\
555	0.042\\
556	0.046\\
557	0.041\\
558	0.045\\
559	0.041\\
560	0.047\\
561	0.041\\
562	0.045\\
563	0.04\\
564	0.036\\
565	0.043\\
566	0.04\\
567	0.045\\
568	0.04\\
569	0.042\\
570	0.04\\
571	0.045\\
572	0.039\\
573	0.045\\
574	0.04\\
575	0.045\\
576	0.039\\
577	0.049\\
578	0.043\\
579	0.049\\
580	0.042\\
581	0.05\\
582	0.044\\
583	0.05\\
584	0.046\\
585	0.048\\
586	0.043\\
587	0.047\\
588	0.043\\
589	0.047\\
590	0.04\\
591	0.043\\
592	0.04\\
593	0.036\\
594	0.04\\
595	0.036\\
596	0.04\\
597	0.035\\
598	0.037\\
599	0.033\\
600	0.037\\
601	0.034\\
602	0.039\\
603	0.037\\
604	0.041\\
605	0.039\\
606	0.04\\
607	0.037\\
608	0.042\\
609	0.039\\
610	0.044\\
611	0.038\\
612	0.045\\
613	0.04\\
614	0.046\\
615	0.042\\
616	0.045\\
617	0.043\\
618	0.045\\
619	0.039\\
620	0.042\\
621	0.039\\
622	0.036\\
623	0.039\\
624	0.036\\
625	0.04\\
626	0.036\\
627	0.041\\
628	0.035\\
629	0.041\\
630	0.035\\
631	0.041\\
632	0.038\\
633	0.04\\
634	0.037\\
635	0.04\\
636	0.037\\
637	0.042\\
638	0.037\\
639	0.04\\
640	0.037\\
641	0.041\\
642	0.036\\
643	0.042\\
644	0.038\\
645	0.042\\
646	0.036\\
647	0.041\\
648	0.039\\
649	0.041\\
650	0.038\\
651	0.044\\
652	0.041\\
653	0.038\\
654	0.04\\
655	0.04\\
656	0.041\\
657	0.04\\
658	0.041\\
659	0.038\\
660	0.042\\
661	0.038\\
662	0.039\\
663	0.036\\
664	0.042\\
665	0.039\\
666	0.04\\
667	0.039\\
668	0.042\\
669	0.039\\
670	0.04\\
671	0.04\\
672	0.041\\
673	0.038\\
674	0.04\\
675	0.038\\
676	0.042\\
677	0.041\\
678	0.043\\
679	0.042\\
680	0.043\\
681	0.041\\
682	0.042\\
683	0.04\\
684	0.039\\
685	0.041\\
686	0.039\\
687	0.042\\
688	0.037\\
689	0.041\\
690	0.036\\
691	0.04\\
692	0.036\\
693	0.041\\
694	0.038\\
695	0.042\\
696	0.039\\
697	0.043\\
698	0.038\\
699	0.041\\
700	0.035\\
701	0.039\\
702	0.036\\
703	0.038\\
704	0.034\\
705	0.041\\
706	0.034\\
707	0.04\\
708	0.038\\
709	0.042\\
710	0.037\\
711	0.04\\
712	0.037\\
713	0.043\\
714	0.041\\
715	0.043\\
716	0.042\\
717	0.039\\
718	0.042\\
719	0.039\\
720	0.041\\
721	0.037\\
722	0.041\\
723	0.038\\
724	0.043\\
725	0.041\\
726	0.044\\
727	0.041\\
728	0.045\\
729	0.042\\
730	0.044\\
731	0.043\\
732	0.043\\
733	0.042\\
734	0.044\\
735	0.042\\
736	0.044\\
737	0.042\\
738	0.046\\
739	0.042\\
740	0.044\\
741	0.043\\
742	0.044\\
743	0.042\\
744	0.043\\
745	0.041\\
746	0.043\\
747	0.041\\
748	0.043\\
749	0.041\\
750	0.037\\
751	0.042\\
752	0.038\\
753	0.041\\
754	0.039\\
755	0.04\\
756	0.039\\
757	0.04\\
758	0.038\\
759	0.04\\
760	0.038\\
761	0.04\\
762	0.039\\
763	0.042\\
764	0.038\\
765	0.04\\
766	0.039\\
767	0.041\\
768	0.039\\
769	0.043\\
770	0.041\\
771	0.042\\
772	0.04\\
773	0.045\\
774	0.039\\
775	0.045\\
776	0.041\\
777	0.043\\
778	0.041\\
779	0.043\\
780	0.04\\
781	0.048\\
782	0.043\\
783	0.038\\
784	0.043\\
785	0.041\\
786	0.045\\
787	0.045\\
788	0.048\\
789	0.042\\
790	0.044\\
791	0.041\\
792	0.046\\
793	0.042\\
794	0.047\\
795	0.044\\
796	0.046\\
797	0.044\\
798	0.045\\
799	0.044\\
800	0.045\\
801	0.042\\
802	0.044\\
803	0.042\\
804	0.045\\
805	0.04\\
806	0.047\\
807	0.041\\
808	0.046\\
809	0.043\\
810	0.048\\
811	0.041\\
812	0.046\\
813	0.043\\
814	0.047\\
815	0.043\\
816	0.045\\
817	0.043\\
818	0.038\\
819	0.043\\
820	0.041\\
821	0.043\\
822	0.041\\
823	0.041\\
824	0.038\\
825	0.045\\
826	0.041\\
827	0.042\\
828	0.039\\
829	0.044\\
830	0.042\\
831	0.047\\
832	0.044\\
833	0.049\\
834	0.041\\
835	0.046\\
836	0.042\\
837	0.044\\
838	0.041\\
839	0.044\\
840	0.043\\
841	0.045\\
842	0.04\\
843	0.046\\
844	0.041\\
845	0.043\\
846	0.04\\
847	0.047\\
848	0.045\\
849	0.047\\
850	0.044\\
851	0.047\\
852	0.043\\
853	0.041\\
854	0.042\\
855	0.039\\
856	0.041\\
857	0.04\\
858	0.042\\
859	0.04\\
860	0.042\\
861	0.039\\
862	0.041\\
863	0.038\\
864	0.041\\
865	0.039\\
866	0.04\\
867	0.039\\
868	0.041\\
869	0.04\\
870	0.043\\
871	0.039\\
872	0.042\\
873	0.039\\
874	0.041\\
875	0.039\\
876	0.04\\
877	0.037\\
878	0.043\\
879	0.038\\
880	0.045\\
881	0.039\\
882	0.042\\
883	0.038\\
884	0.042\\
885	0.038\\
886	0.043\\
887	0.037\\
888	0.035\\
889	0.038\\
890	0.033\\
891	0.036\\
892	0.034\\
893	0.037\\
894	0.033\\
895	0.036\\
896	0.034\\
897	0.038\\
898	0.034\\
899	0.04\\
900	0.035\\
901	0.036\\
902	0.034\\
903	0.039\\
904	0.037\\
905	0.044\\
906	0.041\\
907	0.043\\
908	0.04\\
909	0.043\\
910	0.041\\
911	0.043\\
912	0.04\\
913	0.043\\
914	0.04\\
915	0.045\\
916	0.04\\
917	0.041\\
918	0.038\\
919	0.04\\
920	0.036\\
921	0.04\\
922	0.038\\
923	0.04\\
924	0.038\\
925	0.037\\
926	0.038\\
927	0.037\\
928	0.04\\
929	0.038\\
930	0.04\\
931	0.038\\
932	0.042\\
933	0.039\\
934	0.04\\
935	0.037\\
936	0.04\\
937	0.038\\
938	0.04\\
939	0.038\\
940	0.042\\
941	0.038\\
942	0.041\\
943	0.039\\
944	0.041\\
945	0.038\\
946	0.042\\
947	0.039\\
948	0.042\\
949	0.038\\
950	0.043\\
951	0.041\\
952	0.044\\
953	0.038\\
954	0.042\\
955	0.038\\
956	0.042\\
957	0.039\\
958	0.044\\
959	0.038\\
960	0.042\\
961	0.037\\
962	0.035\\
963	0.04\\
964	0.037\\
965	0.044\\
966	0.04\\
967	0.043\\
968	0.041\\
969	0.041\\
970	0.038\\
971	0.041\\
972	0.039\\
973	0.041\\
974	0.04\\
975	0.043\\
976	0.04\\
977	0.044\\
978	0.037\\
979	0.043\\
980	0.039\\
981	0.042\\
982	0.041\\
983	0.043\\
984	0.042\\
985	0.042\\
986	0.041\\
987	0.043\\
988	0.042\\
989	0.046\\
990	0.045\\
991	0.047\\
992	0.044\\
993	0.046\\
994	0.043\\
995	0.045\\
996	0.041\\
997	0.044\\
998	0.043\\
999	0.046\\
1000	0.043\\
};
\addlegendentry{null-distributed}

\addplot [color=black!50!green]
  table[row sep=crcr]{%
1	0\\
2	0\\
3	0\\
4	0\\
5	0.056\\
6	0.028\\
7	0.018\\
8	0.076\\
9	0.048\\
10	0.028\\
11	0.071\\
12	0.049\\
13	0.096\\
14	0.06\\
15	0.046\\
16	0.094\\
17	0.069\\
18	0.114\\
19	0.08\\
20	0.056\\
21	0.099\\
22	0.068\\
23	0.103\\
24	0.084\\
25	0.065\\
26	0.101\\
27	0.085\\
28	0.11\\
29	0.084\\
30	0.138\\
31	0.106\\
32	0.087\\
33	0.134\\
34	0.109\\
35	0.149\\
36	0.116\\
37	0.17\\
38	0.134\\
39	0.104\\
40	0.15\\
41	0.117\\
42	0.166\\
43	0.138\\
44	0.184\\
45	0.152\\
46	0.128\\
47	0.169\\
48	0.143\\
49	0.185\\
50	0.166\\
51	0.192\\
52	0.167\\
53	0.205\\
54	0.167\\
55	0.144\\
56	0.19\\
57	0.157\\
58	0.193\\
59	0.165\\
60	0.203\\
61	0.172\\
62	0.213\\
63	0.176\\
64	0.157\\
65	0.189\\
66	0.163\\
67	0.209\\
68	0.172\\
69	0.213\\
70	0.181\\
71	0.221\\
72	0.193\\
73	0.16\\
74	0.21\\
75	0.173\\
76	0.219\\
77	0.186\\
78	0.229\\
79	0.195\\
80	0.227\\
81	0.201\\
82	0.234\\
83	0.212\\
84	0.176\\
85	0.215\\
86	0.183\\
87	0.222\\
88	0.194\\
89	0.229\\
90	0.208\\
91	0.237\\
92	0.214\\
93	0.247\\
94	0.215\\
95	0.191\\
96	0.22\\
97	0.198\\
98	0.224\\
99	0.198\\
100	0.234\\
101	0.206\\
102	0.244\\
103	0.214\\
104	0.257\\
105	0.225\\
106	0.261\\
107	0.233\\
108	0.201\\
109	0.24\\
110	0.215\\
111	0.245\\
112	0.226\\
113	0.264\\
114	0.232\\
115	0.277\\
116	0.239\\
117	0.29\\
118	0.256\\
119	0.283\\
120	0.261\\
121	0.233\\
122	0.271\\
123	0.245\\
124	0.283\\
125	0.248\\
126	0.281\\
127	0.249\\
128	0.289\\
129	0.258\\
130	0.295\\
131	0.267\\
132	0.317\\
133	0.282\\
134	0.252\\
135	0.29\\
136	0.267\\
137	0.285\\
138	0.267\\
139	0.295\\
140	0.276\\
141	0.31\\
142	0.283\\
143	0.317\\
144	0.286\\
145	0.321\\
146	0.297\\
147	0.33\\
148	0.301\\
149	0.276\\
150	0.306\\
151	0.289\\
152	0.318\\
153	0.291\\
154	0.323\\
155	0.292\\
156	0.322\\
157	0.296\\
158	0.331\\
159	0.309\\
160	0.348\\
161	0.32\\
162	0.355\\
163	0.335\\
164	0.302\\
165	0.344\\
166	0.313\\
167	0.351\\
168	0.318\\
169	0.353\\
170	0.317\\
171	0.353\\
172	0.326\\
173	0.363\\
174	0.335\\
175	0.369\\
176	0.345\\
177	0.373\\
178	0.346\\
179	0.391\\
180	0.363\\
181	0.331\\
182	0.368\\
183	0.332\\
184	0.37\\
185	0.348\\
186	0.378\\
187	0.351\\
188	0.382\\
189	0.357\\
190	0.394\\
191	0.363\\
192	0.397\\
193	0.373\\
194	0.402\\
195	0.378\\
196	0.357\\
197	0.385\\
198	0.363\\
199	0.387\\
200	0.37\\
201	0.403\\
202	0.378\\
203	0.405\\
204	0.375\\
205	0.415\\
206	0.383\\
207	0.423\\
208	0.39\\
209	0.424\\
210	0.401\\
211	0.431\\
212	0.407\\
213	0.436\\
214	0.413\\
215	0.383\\
216	0.407\\
217	0.388\\
218	0.418\\
219	0.393\\
220	0.416\\
221	0.396\\
222	0.426\\
223	0.404\\
224	0.426\\
225	0.407\\
226	0.436\\
227	0.416\\
228	0.443\\
229	0.428\\
230	0.456\\
231	0.43\\
232	0.408\\
233	0.44\\
234	0.412\\
235	0.444\\
236	0.421\\
237	0.448\\
238	0.417\\
239	0.448\\
240	0.428\\
241	0.457\\
242	0.439\\
243	0.464\\
244	0.442\\
245	0.468\\
246	0.45\\
247	0.473\\
248	0.448\\
249	0.474\\
250	0.456\\
251	0.435\\
252	0.462\\
253	0.442\\
254	0.469\\
255	0.449\\
256	0.471\\
257	0.449\\
258	0.475\\
259	0.452\\
260	0.477\\
261	0.457\\
262	0.487\\
263	0.469\\
264	0.49\\
265	0.469\\
266	0.497\\
267	0.48\\
268	0.504\\
269	0.481\\
270	0.454\\
271	0.483\\
272	0.462\\
273	0.491\\
274	0.465\\
275	0.493\\
276	0.474\\
277	0.501\\
278	0.48\\
279	0.512\\
280	0.484\\
281	0.512\\
282	0.489\\
283	0.522\\
284	0.494\\
285	0.516\\
286	0.496\\
287	0.521\\
288	0.504\\
289	0.526\\
290	0.51\\
291	0.492\\
292	0.519\\
293	0.498\\
294	0.525\\
295	0.502\\
296	0.53\\
297	0.512\\
298	0.534\\
299	0.514\\
300	0.547\\
301	0.52\\
302	0.548\\
303	0.526\\
304	0.553\\
305	0.525\\
306	0.555\\
307	0.525\\
308	0.564\\
309	0.54\\
310	0.566\\
311	0.546\\
312	0.518\\
313	0.553\\
314	0.528\\
315	0.559\\
316	0.528\\
317	0.555\\
318	0.528\\
319	0.557\\
320	0.532\\
321	0.554\\
322	0.529\\
323	0.568\\
324	0.543\\
325	0.57\\
326	0.551\\
327	0.578\\
328	0.549\\
329	0.581\\
330	0.558\\
331	0.592\\
332	0.564\\
333	0.544\\
334	0.581\\
335	0.553\\
336	0.579\\
337	0.564\\
338	0.587\\
339	0.564\\
340	0.595\\
341	0.564\\
342	0.593\\
343	0.569\\
344	0.597\\
345	0.578\\
346	0.599\\
347	0.584\\
348	0.615\\
349	0.589\\
350	0.62\\
351	0.599\\
352	0.619\\
353	0.599\\
354	0.63\\
355	0.616\\
356	0.588\\
357	0.61\\
358	0.596\\
359	0.615\\
360	0.601\\
361	0.619\\
362	0.598\\
363	0.622\\
364	0.604\\
365	0.624\\
366	0.615\\
367	0.629\\
368	0.618\\
369	0.636\\
370	0.62\\
371	0.635\\
372	0.623\\
373	0.645\\
374	0.625\\
375	0.648\\
376	0.633\\
377	0.653\\
378	0.635\\
379	0.622\\
380	0.642\\
381	0.619\\
382	0.64\\
383	0.629\\
384	0.644\\
385	0.622\\
386	0.642\\
387	0.628\\
388	0.648\\
389	0.628\\
390	0.661\\
391	0.636\\
392	0.659\\
393	0.644\\
394	0.658\\
395	0.638\\
396	0.659\\
397	0.642\\
398	0.662\\
399	0.644\\
400	0.672\\
401	0.651\\
402	0.669\\
403	0.65\\
404	0.634\\
405	0.655\\
406	0.634\\
407	0.648\\
408	0.634\\
409	0.658\\
410	0.637\\
411	0.651\\
412	0.637\\
413	0.661\\
414	0.642\\
415	0.66\\
416	0.639\\
417	0.668\\
418	0.645\\
419	0.668\\
420	0.648\\
421	0.665\\
422	0.644\\
423	0.671\\
424	0.652\\
425	0.676\\
426	0.661\\
427	0.682\\
428	0.662\\
429	0.637\\
430	0.66\\
431	0.637\\
432	0.661\\
433	0.645\\
434	0.663\\
435	0.644\\
436	0.66\\
437	0.649\\
438	0.669\\
439	0.655\\
440	0.671\\
441	0.656\\
442	0.68\\
443	0.663\\
444	0.682\\
445	0.668\\
446	0.685\\
447	0.669\\
448	0.686\\
449	0.666\\
450	0.686\\
451	0.673\\
452	0.693\\
453	0.679\\
454	0.659\\
455	0.688\\
456	0.669\\
457	0.692\\
458	0.672\\
459	0.687\\
460	0.673\\
461	0.688\\
462	0.671\\
463	0.688\\
464	0.67\\
465	0.69\\
466	0.673\\
467	0.69\\
468	0.674\\
469	0.697\\
470	0.678\\
471	0.696\\
472	0.685\\
473	0.705\\
474	0.69\\
475	0.708\\
476	0.694\\
477	0.709\\
478	0.703\\
479	0.712\\
480	0.702\\
481	0.688\\
482	0.703\\
483	0.693\\
484	0.708\\
485	0.693\\
486	0.716\\
487	0.697\\
488	0.712\\
489	0.697\\
490	0.713\\
491	0.699\\
492	0.713\\
493	0.703\\
494	0.716\\
495	0.705\\
496	0.72\\
497	0.705\\
498	0.722\\
499	0.708\\
500	0.722\\
501	0.708\\
502	0.728\\
503	0.713\\
504	0.731\\
505	0.713\\
506	0.732\\
507	0.718\\
508	0.707\\
509	0.723\\
510	0.708\\
511	0.726\\
512	0.709\\
513	0.725\\
514	0.709\\
515	0.73\\
516	0.718\\
517	0.734\\
518	0.719\\
519	0.737\\
520	0.722\\
521	0.741\\
522	0.723\\
523	0.738\\
524	0.721\\
525	0.741\\
526	0.723\\
527	0.747\\
528	0.729\\
529	0.749\\
530	0.731\\
531	0.75\\
532	0.737\\
533	0.752\\
534	0.737\\
535	0.726\\
536	0.745\\
537	0.727\\
538	0.743\\
539	0.736\\
540	0.748\\
541	0.731\\
542	0.749\\
543	0.74\\
544	0.757\\
545	0.742\\
546	0.759\\
547	0.743\\
548	0.767\\
549	0.752\\
550	0.77\\
551	0.752\\
552	0.762\\
553	0.755\\
554	0.768\\
555	0.754\\
556	0.768\\
557	0.755\\
558	0.769\\
559	0.758\\
560	0.772\\
561	0.764\\
562	0.775\\
563	0.769\\
564	0.76\\
565	0.769\\
566	0.758\\
567	0.765\\
568	0.759\\
569	0.77\\
570	0.758\\
571	0.773\\
572	0.76\\
573	0.77\\
574	0.757\\
575	0.774\\
576	0.755\\
577	0.77\\
578	0.759\\
579	0.776\\
580	0.756\\
581	0.774\\
582	0.758\\
583	0.778\\
584	0.765\\
585	0.777\\
586	0.764\\
587	0.783\\
588	0.764\\
589	0.779\\
590	0.766\\
591	0.783\\
592	0.767\\
593	0.762\\
594	0.771\\
595	0.76\\
596	0.781\\
597	0.765\\
598	0.785\\
599	0.766\\
600	0.784\\
601	0.77\\
602	0.783\\
603	0.77\\
604	0.78\\
605	0.771\\
606	0.783\\
607	0.771\\
608	0.782\\
609	0.774\\
610	0.787\\
611	0.778\\
612	0.792\\
613	0.78\\
614	0.787\\
615	0.78\\
616	0.789\\
617	0.781\\
618	0.789\\
619	0.778\\
620	0.792\\
621	0.784\\
622	0.771\\
623	0.788\\
624	0.778\\
625	0.796\\
626	0.783\\
627	0.797\\
628	0.784\\
629	0.796\\
630	0.784\\
631	0.793\\
632	0.787\\
633	0.797\\
634	0.791\\
635	0.8\\
636	0.791\\
637	0.813\\
638	0.794\\
639	0.807\\
640	0.792\\
641	0.808\\
642	0.794\\
643	0.818\\
644	0.798\\
645	0.817\\
646	0.803\\
647	0.822\\
648	0.803\\
649	0.822\\
650	0.816\\
651	0.827\\
652	0.818\\
653	0.81\\
654	0.819\\
655	0.808\\
656	0.822\\
657	0.811\\
658	0.827\\
659	0.819\\
660	0.831\\
661	0.821\\
662	0.831\\
663	0.822\\
664	0.832\\
665	0.825\\
666	0.835\\
667	0.827\\
668	0.839\\
669	0.827\\
670	0.842\\
671	0.833\\
672	0.839\\
673	0.833\\
674	0.843\\
675	0.835\\
676	0.844\\
677	0.836\\
678	0.845\\
679	0.839\\
680	0.849\\
681	0.838\\
682	0.847\\
683	0.839\\
684	0.834\\
685	0.843\\
686	0.833\\
687	0.843\\
688	0.832\\
689	0.839\\
690	0.829\\
691	0.84\\
692	0.83\\
693	0.838\\
694	0.832\\
695	0.846\\
696	0.833\\
697	0.845\\
698	0.835\\
699	0.844\\
700	0.834\\
701	0.846\\
702	0.839\\
703	0.844\\
704	0.838\\
705	0.849\\
706	0.841\\
707	0.854\\
708	0.843\\
709	0.853\\
710	0.844\\
711	0.856\\
712	0.85\\
713	0.856\\
714	0.852\\
715	0.855\\
716	0.848\\
717	0.838\\
718	0.849\\
719	0.844\\
720	0.854\\
721	0.845\\
722	0.852\\
723	0.845\\
724	0.852\\
725	0.846\\
726	0.856\\
727	0.846\\
728	0.856\\
729	0.848\\
730	0.856\\
731	0.849\\
732	0.858\\
733	0.847\\
734	0.861\\
735	0.855\\
736	0.864\\
737	0.854\\
738	0.86\\
739	0.853\\
740	0.862\\
741	0.855\\
742	0.863\\
743	0.854\\
744	0.859\\
745	0.856\\
746	0.859\\
747	0.854\\
748	0.859\\
749	0.856\\
750	0.853\\
751	0.855\\
752	0.85\\
753	0.859\\
754	0.851\\
755	0.857\\
756	0.854\\
757	0.859\\
758	0.857\\
759	0.862\\
760	0.858\\
761	0.866\\
762	0.859\\
763	0.866\\
764	0.857\\
765	0.866\\
766	0.862\\
767	0.869\\
768	0.865\\
769	0.871\\
770	0.865\\
771	0.876\\
772	0.864\\
773	0.872\\
774	0.863\\
775	0.872\\
776	0.864\\
777	0.877\\
778	0.866\\
779	0.88\\
780	0.876\\
781	0.88\\
782	0.871\\
783	0.864\\
784	0.872\\
785	0.865\\
786	0.873\\
787	0.868\\
788	0.877\\
789	0.867\\
790	0.877\\
791	0.867\\
792	0.876\\
793	0.87\\
794	0.88\\
795	0.873\\
796	0.882\\
797	0.874\\
798	0.883\\
799	0.873\\
800	0.884\\
801	0.877\\
802	0.89\\
803	0.883\\
804	0.892\\
805	0.885\\
806	0.894\\
807	0.886\\
808	0.898\\
809	0.887\\
810	0.892\\
811	0.883\\
812	0.892\\
813	0.884\\
814	0.893\\
815	0.885\\
816	0.891\\
817	0.883\\
818	0.88\\
819	0.889\\
820	0.88\\
821	0.892\\
822	0.882\\
823	0.892\\
824	0.88\\
825	0.891\\
826	0.887\\
827	0.896\\
828	0.889\\
829	0.895\\
830	0.892\\
831	0.899\\
832	0.893\\
833	0.898\\
834	0.892\\
835	0.899\\
836	0.897\\
837	0.905\\
838	0.898\\
839	0.906\\
840	0.9\\
841	0.906\\
842	0.898\\
843	0.906\\
844	0.899\\
845	0.904\\
846	0.9\\
847	0.906\\
848	0.901\\
849	0.904\\
850	0.899\\
851	0.908\\
852	0.901\\
853	0.899\\
854	0.908\\
855	0.902\\
856	0.909\\
857	0.901\\
858	0.91\\
859	0.902\\
860	0.909\\
861	0.9\\
862	0.909\\
863	0.903\\
864	0.911\\
865	0.904\\
866	0.91\\
867	0.906\\
868	0.912\\
869	0.91\\
870	0.915\\
871	0.91\\
872	0.913\\
873	0.905\\
874	0.913\\
875	0.909\\
876	0.918\\
877	0.911\\
878	0.919\\
879	0.911\\
880	0.918\\
881	0.914\\
882	0.92\\
883	0.914\\
884	0.922\\
885	0.918\\
886	0.926\\
887	0.919\\
888	0.91\\
889	0.916\\
890	0.914\\
891	0.92\\
892	0.913\\
893	0.921\\
894	0.915\\
895	0.924\\
896	0.92\\
897	0.924\\
898	0.919\\
899	0.927\\
900	0.922\\
901	0.933\\
902	0.925\\
903	0.931\\
904	0.922\\
905	0.93\\
906	0.922\\
907	0.929\\
908	0.925\\
909	0.933\\
910	0.925\\
911	0.932\\
912	0.921\\
913	0.929\\
914	0.924\\
915	0.929\\
916	0.923\\
917	0.935\\
918	0.926\\
919	0.935\\
920	0.928\\
921	0.931\\
922	0.929\\
923	0.933\\
924	0.928\\
925	0.923\\
926	0.931\\
927	0.922\\
928	0.929\\
929	0.924\\
930	0.932\\
931	0.926\\
932	0.93\\
933	0.926\\
934	0.932\\
935	0.927\\
936	0.932\\
937	0.928\\
938	0.937\\
939	0.928\\
940	0.934\\
941	0.929\\
942	0.936\\
943	0.932\\
944	0.94\\
945	0.933\\
946	0.935\\
947	0.934\\
948	0.937\\
949	0.929\\
950	0.938\\
951	0.932\\
952	0.941\\
953	0.936\\
954	0.943\\
955	0.939\\
956	0.941\\
957	0.937\\
958	0.943\\
959	0.936\\
960	0.942\\
961	0.935\\
962	0.931\\
963	0.938\\
964	0.935\\
965	0.938\\
966	0.934\\
967	0.937\\
968	0.931\\
969	0.935\\
970	0.93\\
971	0.934\\
972	0.932\\
973	0.937\\
974	0.935\\
975	0.938\\
976	0.935\\
977	0.942\\
978	0.938\\
979	0.942\\
980	0.939\\
981	0.942\\
982	0.939\\
983	0.945\\
984	0.942\\
985	0.948\\
986	0.941\\
987	0.947\\
988	0.942\\
989	0.947\\
990	0.941\\
991	0.945\\
992	0.941\\
993	0.945\\
994	0.941\\
995	0.946\\
996	0.942\\
997	0.943\\
998	0.94\\
999	0.945\\
1000	0.941\\
};
\addlegendentry{other distribution}

\end{axis}
\end{tikzpicture}%
        \caption{Here we see the rejection rate of the null hypothesis ($\model_0$) for two cases. Firstly, for the case when $\model_0$ is true. Secondly, when the data is generated from $\Bernoulli(0.55)$.}
      \end{figure}
      \only<article>{As we see, this method keeps its promise: the null is only rejected 0.05 of the time when it's true. We can also examine how often the null is rejected when it is false... but what should we compare against? Here we are generating data from a $\Bernoulli(0.55)$ model, and we can see the rejection of the null increases with the amount of data. This is called the \alert{power} of the test with respect to the $\Bernoulli(0.55)$ distribution. }
    }
  \end{exercise}
\end{frame}

\begin{frame}
  \begin{alertblock}{Statistical power and false discovery.}
    Beyond not rejecting the null when it's true, we also want:
    \begin{itemize}
    \item High power: Rejecting the null when it is false.
    \item Low false discovery rate: Accepting the null when it is true.
    \end{itemize}
  \end{alertblock}
  \begin{block}{Power}
    The power depends on what hypothesis we use as an alternative.
    \only<article>{This implies that we cannot simply consider a plain null hypothesis test, but must formulate a specific alternative hypothesis. }
  \end{block}

  \begin{block}{False discovery rate}
    False discovery depends on how likely it is \alert{a priori} that the null is false.
    \only<article>{This implies that we need to consider a prior probability for the null hypothesis being true.}
  \end{block}

  \only<article>{Both of these problems suggest that a Bayesian approach might be more suitable. Firstly, it allows us to consider an infinite number of possible alternative models as the alternative hypothesis, through Bayesian model averaging. Secondly, it allows us to specify prior probabilities for each alternative. This is especially important when we consider some effects unlikely.}
\end{frame}

\begin{frame}
  \frametitle{The Bayesian Null-Hypothesis test}
  \index{Null-Hypothesis test!Bayesian}
  \begin{example}
    \begin{enumerate}
    \item Set $\util(a_i, \model_j) = \ind{i =
        j}$.
      \only<article>{This choice makes sense if we care equally about
        either type of error.}
    \item Set $\bel(\model_i) = 1/2$. \only<article>{Here we place an
        equal probability in both models.}
    \item $\model_0$: $\Bernoulli(1/2)$. \only<article>{This is the
        same as the null hypothesis test.}
    \item $\model_1$: $\Bernoulli(\theta)$,
      $\theta \sim \Uniform([0,1])$. \only<article>{This is an
        extension of the simple hypothesis test, with an alternative
        hypothesis that says ``the data comes from an arbitrary
        Bernoulli model''.}
    \item Calculate $\bel(\model \mid x)$.
    \item Choose $a_i$, where $i = \argmax_{j} \bel(\model_j \mid x)$.
    \end{enumerate}
    \label{ex:bayesian-compound-hypothesis-test}
  \end{example}
  \begin{block}{Bayesian model averaging for the alternative model $\model_1$}
    \only<article>{In this scenario, $\model_0$ is a simple point model, e.g. corresponding to a $\Bernoulli(1/2)$. However $\model_1$ is a marginal distribution integrated over many models, e.g. a $Beta$ distribution over Bernoulli parameters.}
    \begin{align}
      P_{\model_1}(x) &= \int_\Param B_{\param}(x) \dd \beta(\param) \\
      \bel(\model_0 \mid x) &= \frac{P_{\model_0}(x) \bel(\model_0)}
                              {P_{\model_0}(x) \bel(\model_0) + P_{\model_1}(x) \bel(\model_1)}
    \end{align}
  \end{block}
\end{frame}
\begin{frame}
  \only<1>{
    \begin{figure}[H]
      \centering
      % This file was created by matlab2tikz.
%
%The latest updates can be retrieved from
%  http://www.mathworks.com/matlabcentral/fileexchange/22022-matlab2tikz-matlab2tikz
%where you can also make suggestions and rate matlab2tikz.
%
\begin{tikzpicture}

\begin{axis}[%
width=0.951\fwidth,
height=\fheight,
at={(0\fwidth,0\fheight)},
scale only axis,
xmin=0,
xmax=1000,
ymin=0.2,
ymax=1,
axis background/.style={fill=white},
title={Posterior probability of null hypothesis},
legend style={legend cell align=left, align=left, legend plot pos=left, draw=black}
]
\addplot [color=blue]
  table[row sep=crcr]{%
1	0.5\\
2	0.513428571428579\\
3	0.530400000000006\\
4	0.549220151828846\\
5	0.564505794640886\\
6	0.579761837965468\\
7	0.589670652749934\\
8	0.603198327487571\\
9	0.612710507804044\\
10	0.622481529146985\\
11	0.632329674043767\\
12	0.639125976838228\\
13	0.64509265422774\\
14	0.65393716639051\\
15	0.660686705700999\\
16	0.66433370310305\\
17	0.669903576697104\\
18	0.675067487506611\\
19	0.678154119244985\\
20	0.682495406433819\\
21	0.686197022529554\\
22	0.691301941876299\\
23	0.69640361600917\\
24	0.703148601117709\\
25	0.707820032407103\\
26	0.711695426457818\\
27	0.715359221147491\\
28	0.719042731552696\\
29	0.722529450011717\\
30	0.725285344744691\\
31	0.727471083145525\\
32	0.730179261357046\\
33	0.730405967890427\\
34	0.733713933677416\\
35	0.737454528896692\\
36	0.73977980297638\\
37	0.741904551474257\\
38	0.744520423508124\\
39	0.744986671773239\\
40	0.745363908713381\\
41	0.748465389324303\\
42	0.750231315542815\\
43	0.753435009449074\\
44	0.755556048418276\\
45	0.757109318417645\\
46	0.759351632582811\\
47	0.761754420386211\\
48	0.764260506876631\\
49	0.764994921324801\\
50	0.765376283449688\\
51	0.767723380064695\\
52	0.768936949960375\\
53	0.769457408643923\\
54	0.770685404693851\\
55	0.772351845003292\\
56	0.773369631759839\\
57	0.774347180970153\\
58	0.775948445715669\\
59	0.776818685429158\\
60	0.777439073709813\\
61	0.777335339841137\\
62	0.778001639046212\\
63	0.778982040021835\\
64	0.780534866671558\\
65	0.779803783998585\\
66	0.781386887478536\\
67	0.78278197103704\\
68	0.783380152167219\\
69	0.7840731328909\\
70	0.784017181965077\\
71	0.784445682306571\\
72	0.784800069475553\\
73	0.785686887745829\\
74	0.786650447577705\\
75	0.788018835743464\\
76	0.788105451159269\\
77	0.789854872882559\\
78	0.790482543945808\\
79	0.793688505624105\\
80	0.79554715442027\\
81	0.795574803701331\\
82	0.797051373748046\\
83	0.796660450494348\\
84	0.797988172751084\\
85	0.79905313580717\\
86	0.799601144269644\\
87	0.800200629017261\\
88	0.802474583039486\\
89	0.802406977605762\\
90	0.804087871337542\\
91	0.805653690795988\\
92	0.805437195413505\\
93	0.80685229349786\\
94	0.806200747398066\\
95	0.807864765547405\\
96	0.809223617519479\\
97	0.809477093315768\\
98	0.811077861150512\\
99	0.811470621601731\\
100	0.812606162368112\\
101	0.813043579318637\\
102	0.813805885752489\\
103	0.814053466159382\\
104	0.815376290221077\\
105	0.816424441948036\\
106	0.816645130031278\\
107	0.816896625076648\\
108	0.817785934217611\\
109	0.819083677761451\\
110	0.819107329241821\\
111	0.81968545139683\\
112	0.820658616197729\\
113	0.821387455278069\\
114	0.822379257115891\\
115	0.823361096732542\\
116	0.824259438356359\\
117	0.82454759040492\\
118	0.826258249815837\\
119	0.827140446076984\\
120	0.827399756121947\\
121	0.827990766104929\\
122	0.828640690740409\\
123	0.828930421658244\\
124	0.829897592362919\\
125	0.831045838355361\\
126	0.831573281760198\\
127	0.830931825537672\\
128	0.831209844386124\\
129	0.831585759421777\\
130	0.831312993509998\\
131	0.83154726328006\\
132	0.831598582473836\\
133	0.831323673882424\\
134	0.831096903402298\\
135	0.83167758875837\\
136	0.83181725115168\\
137	0.832485777681264\\
138	0.833308219556405\\
139	0.834061767947827\\
140	0.834635531679534\\
141	0.835036672693498\\
142	0.835083514741903\\
143	0.836081970915459\\
144	0.836244757260713\\
145	0.836528146279731\\
146	0.836284320170081\\
147	0.836725885697727\\
148	0.836893109610911\\
149	0.838086965889744\\
150	0.838508394137181\\
151	0.838975040094661\\
152	0.838659274352808\\
153	0.838814156224337\\
154	0.839044598585951\\
155	0.83925006640243\\
156	0.840026750294944\\
157	0.840347844768476\\
158	0.84042692851321\\
159	0.840448463119289\\
160	0.841150275455018\\
161	0.841121210861781\\
162	0.841870879535841\\
163	0.84178622992499\\
164	0.841791278247246\\
165	0.841875108981318\\
166	0.842614442329456\\
167	0.842007978712858\\
168	0.84290931499336\\
169	0.843155726156913\\
170	0.842819095139268\\
171	0.842798475799399\\
172	0.84357544111844\\
173	0.84477523664545\\
174	0.845130419972993\\
175	0.845377854778561\\
176	0.845447461768069\\
177	0.8450837656676\\
178	0.845950381388135\\
179	0.846104519828811\\
180	0.847011120557019\\
181	0.847590616988483\\
182	0.847614015556945\\
183	0.848423025226468\\
184	0.848954941567566\\
185	0.848763009743778\\
186	0.849441098058396\\
187	0.84935717014401\\
188	0.849708376622654\\
189	0.850040767539006\\
190	0.850535287392158\\
191	0.850781320110361\\
192	0.850988162750226\\
193	0.851192944623552\\
194	0.85109179514124\\
195	0.851477549769328\\
196	0.851758967853195\\
197	0.852303523974975\\
198	0.851646878923636\\
199	0.85163120993264\\
200	0.852197864607165\\
201	0.852198749932239\\
202	0.852575908743632\\
203	0.852421263067557\\
204	0.853353360324458\\
205	0.853739835299951\\
206	0.853986296652095\\
207	0.854043314395568\\
208	0.855018923511512\\
209	0.855077818761253\\
210	0.855332099156297\\
211	0.855576867487622\\
212	0.855640358032123\\
213	0.855409392653604\\
214	0.855922731519628\\
215	0.855644576797771\\
216	0.855952064380303\\
217	0.856603082226283\\
218	0.857449517423577\\
219	0.857870405118593\\
220	0.858188556113201\\
221	0.858073617685676\\
222	0.858067049357412\\
223	0.858096759888457\\
224	0.858595451118873\\
225	0.85905102514473\\
226	0.8593462178363\\
227	0.85946198730958\\
228	0.859969218986037\\
229	0.860822500068576\\
230	0.860808939147778\\
231	0.861093757399939\\
232	0.86114605926986\\
233	0.860794329873739\\
234	0.861518410794064\\
235	0.861959093457971\\
236	0.862012373687792\\
237	0.861916019464465\\
238	0.861785254147375\\
239	0.862199612876703\\
240	0.862781703563683\\
241	0.863651054889699\\
242	0.864030493862297\\
243	0.86469605161212\\
244	0.864753337314602\\
245	0.864686724084392\\
246	0.86491270343143\\
247	0.865046520085952\\
248	0.865245869589081\\
249	0.865197150740793\\
250	0.86500103986121\\
251	0.86539275784131\\
252	0.865020727728133\\
253	0.865733901754295\\
254	0.866689098954172\\
255	0.867684927975163\\
256	0.867755440545608\\
257	0.867937046966088\\
258	0.86846105266603\\
259	0.868721080399383\\
260	0.868646876770353\\
261	0.869743699954748\\
262	0.869598665032295\\
263	0.869797153650345\\
264	0.869865819266176\\
265	0.870566460333781\\
266	0.870380531027183\\
267	0.870664776703421\\
268	0.870484384359129\\
269	0.870732374081177\\
270	0.870882542723152\\
271	0.870747605309657\\
272	0.870622908785213\\
273	0.870797590389936\\
274	0.871569905135685\\
275	0.872294068362683\\
276	0.872241655254043\\
277	0.872378312454221\\
278	0.872421919727539\\
279	0.872711866059361\\
280	0.872883058347227\\
281	0.873199456909826\\
282	0.873745458838617\\
283	0.873564688246341\\
284	0.873691022585835\\
285	0.874156370896733\\
286	0.874599389831082\\
287	0.874520312710424\\
288	0.87439745473255\\
289	0.874792519459275\\
290	0.875269482738637\\
291	0.875303815372943\\
292	0.874579971789381\\
293	0.874291553816913\\
294	0.87444424105727\\
295	0.874422435003978\\
296	0.874089438633354\\
297	0.87475300204896\\
298	0.874964453404325\\
299	0.87477197006468\\
300	0.875340418058969\\
301	0.875617904311528\\
302	0.875494464195803\\
303	0.875162144661003\\
304	0.875314884940464\\
305	0.875506457968751\\
306	0.876145936719543\\
307	0.876548469557845\\
308	0.876448374609129\\
309	0.876797127326819\\
310	0.877080823059106\\
311	0.877421879313702\\
312	0.877750141982542\\
313	0.877933294849713\\
314	0.877615498399591\\
315	0.87798847342277\\
316	0.878572354698997\\
317	0.878369407971961\\
318	0.878826037723782\\
319	0.879101897683792\\
320	0.879730737445801\\
321	0.879672420533784\\
322	0.880180699796083\\
323	0.880526632613907\\
324	0.88037436122469\\
325	0.880218704882616\\
326	0.881039344543085\\
327	0.881040892371298\\
328	0.881365482155018\\
329	0.881612659635375\\
330	0.881960637778646\\
331	0.882296783763886\\
332	0.882298893826701\\
333	0.882069207026174\\
334	0.882420189800679\\
335	0.882501043676728\\
336	0.882133451247899\\
337	0.882357731748278\\
338	0.882642975168715\\
339	0.882785471592621\\
340	0.882897635626444\\
341	0.883272587897542\\
342	0.884062422984232\\
343	0.883777873445729\\
344	0.884092886315521\\
345	0.884192694692361\\
346	0.88376664699701\\
347	0.884046409182139\\
348	0.884018371451119\\
349	0.884456190055955\\
350	0.884560371945742\\
351	0.884755364869449\\
352	0.885185277082958\\
353	0.885597634200631\\
354	0.885470780983892\\
355	0.885756152940541\\
356	0.886058834896269\\
357	0.886512643026292\\
358	0.886679395022926\\
359	0.886472910890242\\
360	0.886835278525778\\
361	0.887197562280516\\
362	0.887380344649262\\
363	0.887763310191279\\
364	0.888302879827573\\
365	0.888634478367864\\
366	0.888957744975769\\
367	0.889533396356058\\
368	0.889534781418345\\
369	0.889769508453081\\
370	0.890253312584929\\
371	0.890153372556858\\
372	0.890354872954586\\
373	0.89038426208031\\
374	0.890528443615022\\
375	0.89062487858964\\
376	0.89081070109217\\
377	0.891014239391963\\
378	0.890958167740274\\
379	0.89076131993891\\
380	0.890624286775858\\
381	0.890661908890408\\
382	0.890892638224341\\
383	0.890800081338427\\
384	0.8909673750002\\
385	0.891503058387049\\
386	0.891699465975083\\
387	0.891474351249085\\
388	0.891420189676029\\
389	0.891230765766478\\
390	0.891237084923438\\
391	0.891636801256959\\
392	0.891580996049875\\
393	0.891477251148262\\
394	0.891313743348652\\
395	0.891249600415384\\
396	0.891454554240307\\
397	0.891554744992202\\
398	0.892069102975351\\
399	0.891999885454809\\
400	0.89206007900508\\
401	0.892329506342697\\
402	0.8923732242469\\
403	0.892896691922665\\
404	0.892863008157568\\
405	0.893582068258621\\
406	0.893779769409085\\
407	0.893789830812123\\
408	0.893678632681131\\
409	0.893832185679075\\
410	0.894099779268383\\
411	0.894707514673874\\
412	0.89505550916294\\
413	0.895443180324936\\
414	0.895407342358707\\
415	0.89558480399194\\
416	0.895496618769739\\
417	0.895903593821514\\
418	0.89620801779016\\
419	0.896123525286222\\
420	0.896218840558798\\
421	0.896688380775534\\
422	0.896624676146965\\
423	0.896643996010239\\
424	0.896542313115698\\
425	0.896281971885293\\
426	0.896534116243933\\
427	0.897002589625791\\
428	0.897172931644389\\
429	0.897059387305101\\
430	0.89694892926059\\
431	0.897140638918202\\
432	0.897280144544519\\
433	0.897377512868601\\
434	0.897440570423744\\
435	0.897648978761512\\
436	0.897678515330727\\
437	0.897764996057184\\
438	0.898037306133291\\
439	0.898225494765168\\
440	0.898478667391519\\
441	0.898456264816533\\
442	0.898613822107084\\
443	0.898925983763314\\
444	0.898996446411169\\
445	0.899103436817331\\
446	0.898940061378386\\
447	0.8991564305494\\
448	0.899281553292353\\
449	0.899439616711737\\
450	0.899768621586721\\
451	0.900027383851568\\
452	0.899941304428069\\
453	0.899822904210224\\
454	0.899304527915027\\
455	0.899501774162357\\
456	0.899437542581292\\
457	0.899116896841936\\
458	0.89877565667168\\
459	0.899147587277386\\
460	0.899737307578243\\
461	0.899932980814427\\
462	0.900671841255209\\
463	0.901032654892979\\
464	0.901053185090958\\
465	0.90095393206205\\
466	0.900725046960664\\
467	0.900840788834344\\
468	0.900975663713415\\
469	0.900452752197697\\
470	0.900496088566516\\
471	0.900705662541975\\
472	0.900657838122116\\
473	0.900691207619047\\
474	0.900557862769693\\
475	0.900663339300925\\
476	0.900459504575711\\
477	0.900670209062966\\
478	0.900813873695384\\
479	0.900870562755269\\
480	0.900998203372415\\
481	0.900866255687562\\
482	0.900946820042208\\
483	0.900978272141797\\
484	0.900900499782871\\
485	0.900669249017841\\
486	0.900357009848936\\
487	0.900053443365023\\
488	0.900082721332046\\
489	0.900051105273325\\
490	0.899792874454013\\
491	0.900109269049113\\
492	0.900410554995859\\
493	0.900266778072236\\
494	0.899815227039719\\
495	0.89980204673438\\
496	0.899736201990323\\
497	0.899496735654011\\
498	0.899859792012427\\
499	0.899938787512588\\
500	0.899542472473045\\
501	0.899531104751936\\
502	0.899477594963291\\
503	0.899420837562494\\
504	0.899077943945062\\
505	0.899063597539254\\
506	0.899041444556542\\
507	0.899065908939775\\
508	0.898869674050493\\
509	0.898932982847113\\
510	0.899215654957109\\
511	0.899571847400476\\
512	0.899683164120401\\
513	0.900250476773202\\
514	0.900384066758889\\
515	0.900399041480491\\
516	0.900403022306908\\
517	0.900897958677079\\
518	0.900897671676719\\
519	0.900654028376666\\
520	0.900720074422425\\
521	0.900235811782109\\
522	0.900487354277605\\
523	0.900676963285339\\
524	0.90054527349459\\
525	0.90082592953985\\
526	0.901037056641812\\
527	0.901281657760357\\
528	0.901049044530983\\
529	0.901185617740657\\
530	0.901279046800334\\
531	0.90109564102748\\
532	0.901349878555149\\
533	0.901593338996943\\
534	0.901766728074675\\
535	0.902158100384138\\
536	0.90172938479497\\
537	0.902038312576206\\
538	0.902442744474983\\
539	0.902495513868625\\
540	0.902385215127777\\
541	0.90262571148417\\
542	0.903284332654153\\
543	0.903258289464009\\
544	0.90337170731181\\
545	0.903615326860239\\
546	0.903482881611624\\
547	0.903833219350268\\
548	0.903876632222242\\
549	0.904139949490778\\
550	0.90403676530153\\
551	0.904372104831306\\
552	0.904525607260554\\
553	0.904500478612329\\
554	0.904290438071598\\
555	0.904501987016298\\
556	0.90480004058273\\
557	0.905124826543655\\
558	0.904937333984618\\
559	0.905108063420727\\
560	0.905205803746772\\
561	0.905233367609886\\
562	0.905440466126476\\
563	0.90544757427411\\
564	0.905229025641736\\
565	0.904890575104236\\
566	0.904641355824293\\
567	0.904733699351956\\
568	0.904831165068893\\
569	0.904739375117331\\
570	0.904615482949781\\
571	0.904491706572072\\
572	0.904668378142016\\
573	0.904850461681524\\
574	0.904918157187051\\
575	0.904780868484108\\
576	0.90464758981877\\
577	0.904580125526714\\
578	0.904709325665446\\
579	0.904971946608489\\
580	0.905137774074132\\
581	0.905027800713973\\
582	0.905372212216707\\
583	0.905583314720376\\
584	0.905779422425697\\
585	0.905901398040807\\
586	0.905657586492967\\
587	0.905983219646736\\
588	0.905959594399551\\
589	0.906208180557379\\
590	0.906494979581495\\
591	0.906282614723431\\
592	0.905835247016342\\
593	0.905626789852199\\
594	0.905825373494532\\
595	0.90594868036409\\
596	0.906102395430337\\
597	0.906374705370362\\
598	0.906416759122066\\
599	0.906658942238733\\
600	0.906663070375934\\
601	0.906876308716253\\
602	0.906862618417479\\
603	0.907118405841068\\
604	0.906873969197931\\
605	0.907144804516937\\
606	0.907254101020179\\
607	0.907543373432151\\
608	0.907790628229123\\
609	0.907963221287878\\
610	0.908072934585263\\
611	0.908335051300562\\
612	0.908313358267817\\
613	0.908492737253873\\
614	0.908556403140025\\
615	0.908576266346228\\
616	0.908392129070155\\
617	0.908519627696974\\
618	0.908526540147038\\
619	0.908464109573529\\
620	0.908730112194305\\
621	0.909007509825898\\
622	0.909186804962908\\
623	0.909217281177271\\
624	0.908952152086493\\
625	0.908888985870366\\
626	0.909043357163516\\
627	0.909006674199393\\
628	0.909276453016599\\
629	0.909301647582943\\
630	0.909575705910598\\
631	0.909412588636414\\
632	0.909774225438554\\
633	0.909733475442626\\
634	0.910091446807914\\
635	0.910039137803902\\
636	0.910038447714767\\
637	0.909603325526185\\
638	0.909996893371881\\
639	0.910030032218183\\
640	0.909956283233837\\
641	0.910313640611984\\
642	0.910435322260196\\
643	0.910267548392231\\
644	0.910427058588987\\
645	0.91041502059795\\
646	0.910679025437712\\
647	0.910804099340493\\
648	0.911120163178348\\
649	0.911137208713604\\
650	0.911256388252492\\
651	0.911321933523723\\
652	0.911240847102822\\
653	0.910912918761571\\
654	0.910916479492718\\
655	0.91098076182029\\
656	0.910996384499599\\
657	0.910996551072843\\
658	0.910952912437121\\
659	0.910976500282831\\
660	0.910885015010786\\
661	0.910656179717752\\
662	0.911000513680335\\
663	0.911483791498768\\
664	0.911512190587999\\
665	0.911523125911814\\
666	0.911425678706113\\
667	0.911513293130596\\
668	0.911270247177834\\
669	0.911223207451165\\
670	0.911218182374091\\
671	0.911524836182184\\
672	0.911661116013581\\
673	0.911952978294118\\
674	0.911919559431107\\
675	0.912071214162512\\
676	0.912119806419482\\
677	0.912197826497718\\
678	0.911838819206902\\
679	0.911676613529057\\
680	0.911548070687162\\
681	0.911780366536458\\
682	0.911959485057924\\
683	0.91211476163063\\
684	0.911883664793621\\
685	0.911888380504357\\
686	0.912128542557225\\
687	0.91200770769011\\
688	0.911884147667853\\
689	0.912307118645362\\
690	0.912490292087559\\
691	0.912850183863347\\
692	0.912714655701326\\
693	0.912526307227088\\
694	0.912450493257876\\
695	0.91273034722557\\
696	0.912896065739724\\
697	0.913040884708395\\
698	0.912973067862385\\
699	0.913055115409388\\
700	0.913153789393103\\
701	0.913488893510176\\
702	0.913566823074341\\
703	0.913789976558888\\
704	0.91376584714568\\
705	0.913966937837984\\
706	0.914185149495448\\
707	0.913953509871363\\
708	0.913901525773659\\
709	0.914154724137982\\
710	0.913973730544639\\
711	0.914033510362375\\
712	0.913993905589593\\
713	0.914065163598812\\
714	0.913975554116992\\
715	0.913880248238623\\
716	0.914029971528616\\
717	0.914167117481087\\
718	0.914282606625561\\
719	0.914253247310693\\
720	0.914324521262557\\
721	0.914224060129336\\
722	0.91397275937312\\
723	0.914208274405052\\
724	0.914472362382426\\
725	0.914320458656168\\
726	0.9144320789142\\
727	0.91446325483576\\
728	0.914578266733565\\
729	0.914570061021404\\
730	0.914460480300966\\
731	0.914616337682431\\
732	0.914793890556421\\
733	0.914812514342507\\
734	0.914824411872085\\
735	0.914849427779976\\
736	0.914710688845785\\
737	0.914953260549388\\
738	0.915206585452327\\
739	0.915320132350385\\
740	0.915637553776793\\
741	0.915751856906566\\
742	0.915779625288203\\
743	0.916001941764533\\
744	0.915772035689486\\
745	0.915786384751517\\
746	0.915839826418639\\
747	0.916057267216476\\
748	0.916182489748863\\
749	0.916140169576031\\
750	0.916221735363692\\
751	0.916248024418264\\
752	0.916390878140088\\
753	0.916314187834269\\
754	0.916176395195527\\
755	0.916425669186365\\
756	0.916458372305891\\
757	0.916556661838468\\
758	0.91669008483251\\
759	0.916473304337913\\
760	0.91660229643994\\
761	0.916714459107972\\
762	0.91687739290483\\
763	0.916958508497886\\
764	0.917106238476631\\
765	0.917102399631135\\
766	0.917314692933706\\
767	0.91744145476723\\
768	0.917411740267331\\
769	0.917426893622457\\
770	0.917328978094296\\
771	0.917585243521865\\
772	0.917746100842904\\
773	0.917707357957101\\
774	0.91769561735125\\
775	0.917592364517094\\
776	0.917593597727676\\
777	0.917631294066587\\
778	0.91768865697017\\
779	0.917765265230184\\
780	0.917638541266642\\
781	0.917522159346581\\
782	0.917522899488594\\
783	0.917413929109497\\
784	0.917347636180439\\
785	0.917745363900988\\
786	0.917821863148938\\
787	0.917991069298415\\
788	0.917622711225858\\
789	0.91752888564053\\
790	0.917618830234156\\
791	0.917806631605699\\
792	0.917830456972353\\
793	0.918065201747729\\
794	0.918092704544362\\
795	0.918234064944799\\
796	0.918027422386904\\
797	0.918102372460589\\
798	0.917951052511886\\
799	0.91788074572757\\
800	0.918110369018461\\
801	0.918233783538549\\
802	0.918159191158439\\
803	0.918110965520066\\
804	0.918022422202006\\
805	0.918203832777293\\
806	0.918278134814061\\
807	0.918275410938789\\
808	0.918149021514307\\
809	0.918027467904833\\
810	0.918092601659021\\
811	0.918381208057087\\
812	0.918736624593667\\
813	0.918950894800309\\
814	0.91902077790755\\
815	0.919065124416002\\
816	0.919130229219796\\
817	0.919225402146274\\
818	0.919649641353637\\
819	0.919692090309233\\
820	0.919752425613076\\
821	0.919764368415795\\
822	0.919959357408949\\
823	0.920230047075945\\
824	0.920319192560071\\
825	0.920502990783955\\
826	0.920455868186982\\
827	0.920597593300504\\
828	0.920546100589681\\
829	0.92056747706318\\
830	0.920530976455304\\
831	0.920759736085753\\
832	0.920719888528511\\
833	0.920782207062358\\
834	0.920903756762417\\
835	0.920761431554239\\
836	0.920728180367128\\
837	0.920937836165203\\
838	0.921001330601021\\
839	0.921142622768558\\
840	0.921149107093537\\
841	0.921081547058581\\
842	0.92119626017738\\
843	0.92142753512188\\
844	0.921500151270171\\
845	0.921672440523428\\
846	0.921617824662702\\
847	0.921720849217872\\
848	0.9216845009842\\
849	0.921901731267334\\
850	0.921880773757243\\
851	0.922030310291404\\
852	0.921993061615431\\
853	0.922157776835074\\
854	0.92238195796749\\
855	0.922406857764889\\
856	0.92243671032187\\
857	0.922441901973966\\
858	0.922347563406307\\
859	0.922431713732541\\
860	0.922585234266588\\
861	0.922578088642252\\
862	0.922598666773738\\
863	0.922464868790336\\
864	0.922421638443489\\
865	0.92234812520357\\
866	0.922441483946334\\
867	0.922333398113747\\
868	0.922454818605543\\
869	0.922176453594615\\
870	0.92243226216589\\
871	0.922594093901003\\
872	0.922669801678818\\
873	0.922655612960138\\
874	0.922607377171426\\
875	0.922831778322946\\
876	0.9228911502537\\
877	0.922972472617576\\
878	0.922834945437328\\
879	0.922984208760326\\
880	0.923137164862873\\
881	0.923421717623861\\
882	0.923513456076899\\
883	0.923691041594221\\
884	0.923772654452536\\
885	0.923853641586689\\
886	0.924029579760946\\
887	0.923898109720967\\
888	0.923908915511675\\
889	0.924007246014408\\
890	0.923955666341957\\
891	0.924091953473324\\
892	0.924050373123056\\
893	0.924153000573776\\
894	0.924254243416118\\
895	0.924394795990062\\
896	0.924373385623111\\
897	0.924421752744621\\
898	0.924278659715869\\
899	0.92435648246473\\
900	0.924382852522703\\
901	0.924281185052627\\
902	0.92432572610157\\
903	0.924445138262389\\
904	0.92444789174475\\
905	0.924140320836018\\
906	0.923867496676182\\
907	0.923885955760326\\
908	0.924042343692656\\
909	0.924049216917399\\
910	0.924353594297257\\
911	0.924477768752772\\
912	0.924722089659157\\
913	0.924653994579662\\
914	0.924854056336714\\
915	0.924980388963447\\
916	0.925165197954399\\
917	0.925119324344982\\
918	0.925239379631118\\
919	0.92511088350398\\
920	0.925201942626288\\
921	0.925057742135349\\
922	0.924939125070449\\
923	0.925094123534973\\
924	0.925321663866485\\
925	0.925396189994514\\
926	0.925259733055765\\
927	0.925476973111465\\
928	0.925328506506901\\
929	0.925407835521549\\
930	0.925464376638705\\
931	0.925262181393908\\
932	0.925093297239007\\
933	0.925134381885798\\
934	0.925116134198375\\
935	0.92527691219268\\
936	0.925295338456293\\
937	0.925289505699119\\
938	0.925320056961521\\
939	0.925383296530814\\
940	0.925361124880659\\
941	0.925401013595556\\
942	0.92557540796456\\
943	0.925705946239852\\
944	0.925606180139248\\
945	0.925663755029423\\
946	0.925594942654501\\
947	0.92573712838758\\
948	0.926009109513846\\
949	0.926147242241459\\
950	0.926056575786199\\
951	0.926345354314932\\
952	0.926279762480086\\
953	0.926250387404413\\
954	0.92629259524298\\
955	0.926435477447881\\
956	0.926437775413544\\
957	0.926258808165419\\
958	0.926128244603786\\
959	0.926083064209885\\
960	0.926384606670326\\
961	0.92644577914241\\
962	0.92633391742912\\
963	0.926324456053356\\
964	0.926681102310738\\
965	0.926553207536277\\
966	0.926565846301665\\
967	0.926708237638184\\
968	0.926870158160594\\
969	0.92677869568474\\
970	0.926891240401072\\
971	0.927012415123364\\
972	0.927107335000297\\
973	0.927030325853832\\
974	0.926866636518792\\
975	0.926961416492809\\
976	0.927042850545933\\
977	0.927130456668847\\
978	0.92725351233201\\
979	0.927155345121732\\
980	0.92704222471787\\
981	0.927185915775622\\
982	0.92725212195743\\
983	0.927357889290717\\
984	0.927373205694981\\
985	0.927372922354718\\
986	0.927270362280897\\
987	0.927090897778489\\
988	0.927216119237622\\
989	0.927329967565924\\
990	0.927522553273289\\
991	0.927477948553595\\
992	0.92770715887184\\
993	0.927835107904276\\
994	0.927912124899563\\
995	0.927891276470247\\
996	0.927689673319107\\
997	0.927718547575706\\
998	0.927865815998528\\
999	0.927775320660132\\
1000	0.927969393482381\\
};
\addlegendentry{null-distributed}

\addplot [color=black!50!green]
  table[row sep=crcr]{%
1	0.5\\
2	0.510514285714293\\
3	0.522666666666672\\
4	0.541145617667356\\
5	0.559594891857982\\
6	0.572722077561215\\
7	0.58267357894979\\
8	0.590888765025288\\
9	0.59950513230508\\
10	0.610398456292962\\
11	0.621986171661307\\
12	0.629607685353166\\
13	0.636650008515584\\
14	0.641805836840973\\
15	0.645673117920615\\
16	0.650190060042996\\
17	0.653318035009855\\
18	0.657104898478128\\
19	0.662415063008198\\
20	0.664416325730633\\
21	0.666497196932597\\
22	0.670366953069615\\
23	0.675329406202602\\
24	0.677535740711214\\
25	0.678609752423742\\
26	0.680068798424392\\
27	0.680363078391969\\
28	0.68360476942038\\
29	0.684439503925459\\
30	0.685151524513601\\
31	0.688578805680968\\
32	0.690203422139353\\
33	0.689430643143064\\
34	0.690565546306849\\
35	0.691631658783598\\
36	0.694997724911021\\
37	0.695266391003939\\
38	0.697159731694896\\
39	0.698803417074886\\
40	0.69870982590195\\
41	0.702340202434425\\
42	0.701772883836458\\
43	0.701941014314554\\
44	0.703360953930788\\
45	0.705494681518851\\
46	0.704779724224034\\
47	0.704324217639773\\
48	0.705849622765393\\
49	0.705925862040322\\
50	0.705943013344713\\
51	0.706458704848519\\
52	0.70744761182046\\
53	0.708252374249268\\
54	0.707711088285194\\
55	0.708061481597829\\
56	0.707890651183836\\
57	0.708037477077985\\
58	0.710111925810219\\
59	0.710108793019845\\
60	0.711070115688758\\
61	0.710728493866042\\
62	0.711955914789266\\
63	0.711638925910158\\
64	0.714083642457653\\
65	0.713870043865417\\
66	0.715173493948941\\
67	0.716518112002338\\
68	0.714751688743372\\
69	0.715719062419111\\
70	0.71668503020515\\
71	0.71645867362212\\
72	0.718641838838105\\
73	0.718651935083157\\
74	0.718486897096008\\
75	0.718245502812726\\
76	0.718535959541549\\
77	0.719216253782296\\
78	0.719635507072108\\
79	0.720090989744273\\
80	0.721116499270887\\
81	0.720543134935382\\
82	0.720023391450815\\
83	0.719695942434876\\
84	0.721716193189492\\
85	0.720740504484837\\
86	0.720046852352644\\
87	0.720780880412062\\
88	0.721701383700102\\
89	0.722201559432899\\
90	0.722093608362844\\
91	0.722633531337497\\
92	0.722905012391265\\
93	0.722669433744779\\
94	0.723294063767271\\
95	0.724512473593576\\
96	0.724551508501092\\
97	0.724185072315263\\
98	0.724803387186071\\
99	0.723381821473247\\
100	0.722949691244829\\
101	0.722678049571914\\
102	0.72309092030487\\
103	0.724461402583635\\
104	0.726218160605663\\
105	0.724961532259852\\
106	0.725053884684947\\
107	0.725384438872887\\
108	0.724569953186669\\
109	0.723392513294873\\
110	0.723735109832107\\
111	0.723800390115964\\
112	0.723080153181814\\
113	0.721243691934387\\
114	0.722144077750336\\
115	0.72031671859672\\
116	0.720651850410994\\
117	0.719859038606614\\
118	0.719565108459603\\
119	0.720605295947238\\
120	0.720077433664943\\
121	0.721238909696906\\
122	0.721246736233956\\
123	0.720595553600838\\
124	0.721235648631571\\
125	0.722029607787604\\
126	0.721850406329367\\
127	0.721898513793173\\
128	0.72140608260354\\
129	0.722547200095916\\
130	0.722490297756938\\
131	0.721099387242113\\
132	0.719110935062541\\
133	0.719146763470114\\
134	0.720190229000194\\
135	0.719138514028948\\
136	0.717913961589254\\
137	0.718624712690314\\
138	0.718785350419777\\
139	0.718937277732206\\
140	0.718026228442769\\
141	0.717005434807179\\
142	0.715501671893482\\
143	0.715246196500503\\
144	0.715250037138092\\
145	0.714440502381109\\
146	0.716217892810672\\
147	0.716633499967755\\
148	0.716160136742425\\
149	0.716707934188694\\
150	0.71807232619364\\
151	0.716836968162971\\
152	0.716441929130552\\
153	0.715359278073588\\
154	0.714737815949687\\
155	0.716243581839717\\
156	0.715087506818712\\
157	0.71517871286811\\
158	0.71546051879918\\
159	0.714175317806894\\
160	0.713972642624575\\
161	0.713261957935537\\
162	0.71348793759172\\
163	0.712298602792273\\
164	0.710392996884199\\
165	0.710924760404204\\
166	0.710249705896436\\
167	0.710279325754645\\
168	0.710349415418325\\
169	0.710099563178778\\
170	0.710512303083625\\
171	0.709812021112128\\
172	0.709366724054538\\
173	0.708189558177089\\
174	0.707456007077034\\
175	0.707557762319277\\
176	0.707276963389352\\
177	0.707790744939115\\
178	0.707313231003666\\
179	0.706680483318063\\
180	0.706213917575917\\
181	0.705174454652533\\
182	0.704199222600018\\
183	0.703864233587471\\
184	0.704424886249499\\
185	0.702979693272111\\
186	0.701977282161342\\
187	0.702142928970858\\
188	0.701854700989938\\
189	0.701118633349511\\
190	0.700446927494293\\
191	0.701181075651167\\
192	0.701292982119426\\
193	0.700748283131185\\
194	0.700386600361307\\
195	0.698431404591549\\
196	0.697356443033295\\
197	0.697296097825173\\
198	0.696233980202986\\
199	0.695765474349198\\
200	0.695253580745811\\
201	0.694369592141614\\
202	0.693943738999574\\
203	0.693801082869511\\
204	0.692699977774703\\
205	0.69220459584979\\
206	0.691479979624019\\
207	0.690933609611064\\
208	0.690584112914056\\
209	0.690489214152406\\
210	0.689415728950321\\
211	0.68913548871224\\
212	0.689417704717792\\
213	0.688513708009134\\
214	0.687450861582652\\
215	0.686537374413577\\
216	0.686363399955601\\
217	0.68560910104164\\
218	0.685361148368896\\
219	0.683319047017025\\
220	0.684483734970759\\
221	0.684070054058869\\
222	0.684045148941501\\
223	0.684077467615433\\
224	0.684105599500453\\
225	0.683279452733106\\
226	0.681885503566444\\
227	0.68145110260529\\
228	0.680459797323743\\
229	0.679817900949797\\
230	0.679063854770419\\
231	0.6793764347354\\
232	0.679288861337636\\
233	0.678708567981214\\
234	0.678320360610228\\
235	0.677544706439222\\
236	0.67618450210556\\
237	0.675683945651494\\
238	0.675656889551855\\
239	0.674760768218215\\
240	0.67452116392506\\
241	0.674531197941914\\
242	0.673802024394031\\
243	0.67373899164192\\
244	0.672876933615936\\
245	0.671602451618875\\
246	0.671263940772159\\
247	0.671242647974137\\
248	0.671558932268417\\
249	0.67008739442289\\
250	0.669651586638553\\
251	0.669755915722691\\
252	0.668839480965471\\
253	0.668137773850322\\
254	0.667645818979603\\
255	0.666896330763076\\
256	0.667221433330066\\
257	0.666315148119856\\
258	0.666159528796222\\
259	0.666375308239687\\
260	0.665944368766223\\
261	0.66548962826092\\
262	0.664219852325653\\
263	0.66427322234401\\
264	0.664267749504353\\
265	0.664258878196785\\
266	0.66487016042541\\
267	0.664373298928617\\
268	0.663743682898097\\
269	0.663182295103834\\
270	0.662000201169966\\
271	0.660915015632444\\
272	0.661272609604721\\
273	0.660231406523838\\
274	0.659664546373504\\
275	0.658290476392594\\
276	0.656812613999134\\
277	0.655604080250705\\
278	0.655068521951374\\
279	0.654891559785567\\
280	0.654988429421982\\
281	0.654080656884876\\
282	0.653982552857018\\
283	0.653257622388014\\
284	0.652472565406695\\
285	0.651701621961392\\
286	0.651405392019922\\
287	0.651148804073021\\
288	0.65120746018597\\
289	0.649873332943559\\
290	0.649946694562137\\
291	0.648758484344932\\
292	0.646049105096354\\
293	0.645437728890186\\
294	0.644372730794407\\
295	0.643029531336189\\
296	0.641928595396001\\
297	0.641630340008984\\
298	0.641665051770476\\
299	0.640165863996393\\
300	0.6387061768156\\
301	0.637913983281591\\
302	0.637405677946001\\
303	0.636977863765608\\
304	0.636415754305926\\
305	0.636460545908689\\
306	0.637430231666155\\
307	0.637220008189472\\
308	0.637559027132294\\
309	0.636294904174803\\
310	0.635600377467651\\
311	0.634470135580898\\
312	0.633846866960613\\
313	0.633295704609844\\
314	0.632527688128799\\
315	0.633110494524294\\
316	0.632903057738448\\
317	0.631411673802177\\
318	0.631092800873438\\
319	0.630135291047584\\
320	0.630373705062284\\
321	0.62998882990786\\
322	0.629719820174733\\
323	0.628786699588807\\
324	0.62757254235702\\
325	0.62597060222765\\
326	0.625327410405337\\
327	0.625005484011395\\
328	0.624585688651391\\
329	0.624733581758839\\
330	0.623559433662303\\
331	0.622942782268312\\
332	0.622050959727119\\
333	0.619481249641956\\
334	0.618833102322393\\
335	0.617882382119545\\
336	0.616504880018768\\
337	0.616094372704923\\
338	0.615161764240032\\
339	0.615227052420352\\
340	0.614349326537698\\
341	0.61390810003056\\
342	0.612170840774932\\
343	0.610945743443225\\
344	0.610143962222988\\
345	0.609577378311477\\
346	0.609410827996386\\
347	0.607854388379047\\
348	0.606892270249969\\
349	0.606604804255389\\
350	0.604993655380554\\
351	0.604786184215351\\
352	0.604081436794936\\
353	0.603845724996401\\
354	0.603239364563344\\
355	0.601800678813737\\
356	0.601175832900544\\
357	0.600507700220921\\
358	0.598351682855141\\
359	0.597140660206602\\
360	0.597290111800076\\
361	0.59572278385368\\
362	0.595315653118971\\
363	0.594461646915735\\
364	0.593153019600346\\
365	0.593024673462185\\
366	0.591850443112806\\
367	0.591978008130103\\
368	0.591297499298947\\
369	0.59051393800533\\
370	0.589897577907114\\
371	0.590119435563238\\
372	0.590144035933245\\
373	0.590101357003906\\
374	0.588948864617286\\
375	0.587200553192091\\
376	0.586497307122197\\
377	0.585895956687226\\
378	0.586075012727239\\
379	0.585248191315292\\
380	0.584395711648455\\
381	0.584777809514614\\
382	0.584559681288103\\
383	0.583959823364454\\
384	0.583336185562777\\
385	0.583089534613798\\
386	0.581788296524094\\
387	0.581088976481569\\
388	0.579148120705602\\
389	0.57821007297987\\
390	0.577057273628256\\
391	0.577205128095889\\
392	0.576564116696051\\
393	0.576112115859398\\
394	0.575368569680887\\
395	0.57472859029842\\
396	0.573937433989197\\
397	0.573830089892829\\
398	0.573625963874266\\
399	0.573323858211733\\
400	0.572646765886904\\
401	0.572447462019285\\
402	0.571266399179072\\
403	0.569970752305938\\
404	0.569765301564411\\
405	0.569072092431748\\
406	0.56852339589379\\
407	0.568407929333688\\
408	0.568254628963006\\
409	0.569134704331899\\
410	0.568695058096752\\
411	0.56800460850286\\
412	0.567030651332449\\
413	0.566498155740019\\
414	0.566277230631447\\
415	0.565455023191929\\
416	0.564144692496094\\
417	0.562424755507977\\
418	0.562312243856203\\
419	0.561816481020166\\
420	0.560852146065228\\
421	0.560397088935363\\
422	0.559244876044417\\
423	0.558007253088233\\
424	0.557450946315178\\
425	0.557556452214683\\
426	0.556846291571797\\
427	0.555691324122932\\
428	0.556254521252165\\
429	0.555432435652591\\
430	0.555237577021745\\
431	0.55433009927663\\
432	0.553916543563876\\
433	0.553769712435256\\
434	0.552215144289903\\
435	0.551037792571226\\
436	0.551195202280572\\
437	0.550785736486395\\
438	0.550708759294252\\
439	0.550820062063882\\
440	0.550100746029878\\
441	0.549037425551981\\
442	0.548617874478887\\
443	0.547387909018327\\
444	0.54659746529501\\
445	0.545174186043958\\
446	0.544164803455782\\
447	0.543695692128479\\
448	0.543112616015964\\
449	0.542269881833607\\
450	0.541927842123624\\
451	0.542325084211125\\
452	0.542933704647354\\
453	0.543202728599384\\
454	0.542076185523096\\
455	0.541007295183524\\
456	0.540496593920314\\
457	0.540343293898122\\
458	0.539340137204821\\
459	0.538445632125961\\
460	0.538099832998005\\
461	0.538128944420748\\
462	0.538291526533876\\
463	0.537445470783578\\
464	0.536241967095221\\
465	0.53510672830379\\
466	0.535427113553064\\
467	0.535755398462287\\
468	0.534756746264752\\
469	0.533558641490654\\
470	0.532562970326447\\
471	0.531766175109934\\
472	0.53180456692834\\
473	0.532003198036553\\
474	0.531242885975616\\
475	0.530651085304155\\
476	0.530409205570226\\
477	0.529437742654112\\
478	0.528612229453156\\
479	0.528530490768902\\
480	0.528624948813344\\
481	0.528298324236399\\
482	0.527893219413827\\
483	0.52794850887301\\
484	0.527585562799832\\
485	0.526915363474775\\
486	0.525296777061426\\
487	0.524764677020031\\
488	0.523770606562537\\
489	0.523116217673768\\
490	0.522914514616385\\
491	0.523045779855523\\
492	0.522850582019876\\
493	0.522616393108588\\
494	0.521470272595632\\
495	0.520847490448232\\
496	0.520494723116771\\
497	0.519777378730315\\
498	0.519543735667661\\
499	0.518455631476965\\
500	0.517185962692949\\
501	0.516074415724999\\
502	0.515329626678292\\
503	0.514491594028586\\
504	0.513215068690994\\
505	0.513461351759997\\
506	0.512242606113201\\
507	0.511189640485226\\
508	0.510485568273653\\
509	0.509857827094422\\
510	0.509540066063683\\
511	0.509992369653302\\
512	0.509505100033804\\
513	0.508965609386239\\
514	0.507478059689093\\
515	0.506939913410676\\
516	0.506969090148597\\
517	0.507225620168297\\
518	0.506895899590835\\
519	0.506065113222272\\
520	0.505770741234002\\
521	0.505707046306872\\
522	0.504031330322722\\
523	0.502656550137698\\
524	0.502682283611166\\
525	0.502632311769369\\
526	0.501819928638644\\
527	0.501521453545029\\
528	0.500486282025588\\
529	0.499951516085756\\
530	0.49973799605921\\
531	0.499353265617225\\
532	0.498309138257533\\
533	0.497182927166106\\
534	0.496888331276065\\
535	0.496675542075242\\
536	0.496828088720194\\
537	0.497193323987224\\
538	0.49662542350522\\
539	0.496250686404914\\
540	0.49680040914691\\
541	0.496507371495791\\
542	0.495316376604778\\
543	0.494797920731353\\
544	0.49471811781705\\
545	0.4943907220985\\
546	0.494018888613829\\
547	0.492997890477294\\
548	0.492454942612128\\
549	0.492015104979679\\
550	0.491593593542845\\
551	0.489855129832751\\
552	0.489230104784187\\
553	0.488299263519444\\
554	0.488672889265283\\
555	0.488462261650747\\
556	0.48734440346482\\
557	0.487053517418883\\
558	0.487645711552708\\
559	0.486267376874547\\
560	0.487163997252617\\
561	0.487698385677001\\
562	0.488256183429079\\
563	0.48809133211675\\
564	0.487009901636104\\
565	0.487345873364525\\
566	0.486496959368346\\
567	0.485924833217009\\
568	0.485308592156352\\
569	0.484586362717079\\
570	0.484790651107225\\
571	0.484032104897175\\
572	0.482642927806261\\
573	0.482238919260278\\
574	0.481448931888471\\
575	0.481038538037613\\
576	0.48060767491291\\
577	0.480807225009008\\
578	0.479525837561707\\
579	0.478981590649925\\
580	0.478379824814488\\
581	0.478886659808396\\
582	0.477838347152058\\
583	0.476961230819757\\
584	0.476579008978624\\
585	0.476010785388189\\
586	0.475783994183422\\
587	0.475106497422202\\
588	0.475120180588442\\
589	0.474711655374201\\
590	0.474946874464519\\
591	0.474617999389078\\
592	0.473943637354886\\
593	0.47331483781197\\
594	0.472406166279325\\
595	0.471391230503018\\
596	0.471246834183975\\
597	0.470924801127823\\
598	0.470067913953464\\
599	0.470022357965791\\
600	0.46975806574867\\
601	0.468794485710778\\
602	0.468551971493837\\
603	0.468224873847785\\
604	0.468148270034832\\
605	0.46798518653641\\
606	0.466929615348119\\
607	0.466573185870224\\
608	0.465693074945958\\
609	0.465108230175227\\
610	0.465452914050893\\
611	0.464988888407938\\
612	0.464306486160899\\
613	0.463553703819235\\
614	0.462930705083572\\
615	0.46240344750156\\
616	0.461328732940194\\
617	0.46037314042297\\
618	0.459642466456449\\
619	0.460123297787704\\
620	0.459406538418755\\
621	0.459329110453682\\
622	0.458720534956245\\
623	0.4576171750391\\
624	0.457440014412514\\
625	0.456565830434871\\
626	0.455817359109747\\
627	0.456010104840714\\
628	0.455967125883424\\
629	0.455228251336689\\
630	0.454893804570083\\
631	0.453882143901199\\
632	0.453778520606243\\
633	0.452927963415567\\
634	0.452778947262671\\
635	0.452177120990449\\
636	0.451775149265932\\
637	0.45075957698729\\
638	0.450284279021952\\
639	0.449324293721662\\
640	0.449295324025034\\
641	0.449124274812609\\
642	0.448328277839775\\
643	0.447679871329474\\
644	0.447478021069558\\
645	0.446485344451551\\
646	0.445646046194763\\
647	0.445465725126074\\
648	0.445025171956497\\
649	0.444249961346042\\
650	0.443660748338156\\
651	0.443355743355857\\
652	0.44337647769049\\
653	0.442567592349175\\
654	0.441806843845348\\
655	0.440824265192868\\
656	0.440360720439897\\
657	0.440100397852734\\
658	0.439206064379219\\
659	0.438503295844975\\
660	0.438240790810128\\
661	0.436917882852506\\
662	0.435865823771034\\
663	0.435960106101569\\
664	0.435971202930209\\
665	0.435402584544942\\
666	0.435165895354778\\
667	0.434217184273851\\
668	0.433346693106113\\
669	0.432809579396859\\
670	0.432620358919423\\
671	0.432355314950698\\
672	0.430298450814658\\
673	0.429413324555752\\
674	0.428097777774573\\
675	0.427347750098248\\
676	0.426682653225972\\
677	0.425938418805178\\
678	0.425277626335032\\
679	0.424798650678823\\
680	0.423791653309611\\
681	0.423286432312496\\
682	0.423034662087971\\
683	0.422040985812738\\
684	0.421629912638394\\
685	0.420369767851481\\
686	0.420005029063376\\
687	0.419815284185757\\
688	0.419272962807169\\
689	0.418457939154217\\
690	0.417062570587913\\
691	0.416399503070311\\
692	0.415848711712664\\
693	0.415944549683046\\
694	0.415224818011643\\
695	0.414956200517064\\
696	0.413498420089645\\
697	0.412579553562925\\
698	0.411987132309575\\
699	0.411168969522089\\
700	0.410906710277019\\
701	0.409979530140803\\
702	0.409598239458698\\
703	0.409263780360083\\
704	0.408766690142252\\
705	0.407974161980961\\
706	0.407253021016548\\
707	0.40749077163536\\
708	0.406999802050621\\
709	0.40667065782571\\
710	0.406763946193584\\
711	0.406429837762052\\
712	0.40623281988578\\
713	0.405920889595877\\
714	0.404935448059772\\
715	0.404422810302055\\
716	0.403860024313126\\
717	0.403340333029341\\
718	0.402880624433629\\
719	0.402071213582559\\
720	0.400315102255432\\
721	0.399044595399196\\
722	0.398134005458433\\
723	0.397796730552002\\
724	0.397515289074871\\
725	0.396748457720368\\
726	0.395677748070956\\
727	0.39508195890598\\
728	0.394609117993113\\
729	0.394779042214391\\
730	0.394459870730598\\
731	0.393688898388476\\
732	0.393397735701959\\
733	0.393237776139923\\
734	0.392640110872308\\
735	0.391776361786642\\
736	0.391104290179279\\
737	0.390468987802017\\
738	0.389890412690422\\
739	0.389844251463275\\
740	0.389280766470053\\
741	0.388689930339907\\
742	0.388177681351634\\
743	0.38833067616629\\
744	0.388024240615089\\
745	0.388672094013535\\
746	0.388352778128109\\
747	0.387535168315177\\
748	0.387302310545076\\
749	0.387066119876547\\
750	0.386702590665576\\
751	0.38600634295688\\
752	0.384940468736964\\
753	0.384413806389379\\
754	0.383481740911877\\
755	0.383246951067619\\
756	0.382722041337911\\
757	0.38194780908419\\
758	0.381393459309293\\
759	0.380710664376536\\
760	0.380197608355719\\
761	0.379996875479654\\
762	0.379694522382251\\
763	0.378822125415798\\
764	0.378458139717958\\
765	0.377776351546156\\
766	0.376941253115225\\
767	0.376671690841982\\
768	0.375974512191297\\
769	0.374931987987131\\
770	0.374931888333806\\
771	0.374404397905654\\
772	0.373908191004882\\
773	0.373044366757291\\
774	0.372765298573119\\
775	0.372514303020745\\
776	0.371687077335873\\
777	0.371406022094682\\
778	0.370884167833477\\
779	0.370903145038996\\
780	0.369929025811134\\
781	0.369348619521733\\
782	0.369230300137401\\
783	0.368385816499228\\
784	0.3685921105099\\
785	0.368334540495986\\
786	0.367381078012009\\
787	0.367017660627899\\
788	0.366401879820759\\
789	0.366270028108934\\
790	0.366091868274072\\
791	0.365903517762906\\
792	0.365570233796516\\
793	0.365105432926367\\
794	0.364456318338127\\
795	0.363988630777857\\
796	0.363686779273285\\
797	0.362885694795722\\
798	0.362852101012416\\
799	0.361805063948178\\
800	0.361691300096033\\
801	0.361113379255059\\
802	0.360807487360862\\
803	0.36047037194372\\
804	0.359865230192296\\
805	0.358644684623535\\
806	0.358445085771309\\
807	0.358777143334439\\
808	0.357845914178048\\
809	0.35763066543016\\
810	0.356922942425303\\
811	0.357090846575079\\
812	0.355967718769729\\
813	0.355656137362414\\
814	0.354939614424499\\
815	0.354669442687295\\
816	0.354729130917405\\
817	0.354997796751576\\
818	0.354303569077964\\
819	0.353366857338691\\
820	0.353114389385617\\
821	0.351995454899104\\
822	0.351508592616941\\
823	0.350914611197314\\
824	0.349456336942169\\
825	0.348693969110859\\
826	0.348171154387896\\
827	0.34780828704258\\
828	0.347912146972194\\
829	0.346889423927241\\
830	0.346113695384537\\
831	0.34558675993174\\
832	0.345359251840612\\
833	0.344652003367701\\
834	0.343587032597556\\
835	0.342970571190873\\
836	0.343157574820819\\
837	0.342230661823855\\
838	0.341177368469579\\
839	0.340730848755688\\
840	0.340268571366501\\
841	0.339268109470384\\
842	0.339125834433771\\
843	0.338696730457318\\
844	0.338978110407666\\
845	0.338243081860698\\
846	0.338265964520209\\
847	0.33756528509878\\
848	0.336902043049855\\
849	0.336964725830541\\
850	0.336457400783865\\
851	0.336155199410705\\
852	0.335700952225856\\
853	0.334804346553823\\
854	0.334468045096268\\
855	0.333535238145324\\
856	0.333078532832485\\
857	0.332789689141891\\
858	0.332528286680524\\
859	0.332070140792985\\
860	0.331849577637534\\
861	0.331398495750095\\
862	0.330953967222407\\
863	0.330433489773207\\
864	0.330211431732241\\
865	0.329287340304079\\
866	0.329069938797071\\
867	0.32780817976907\\
868	0.327940769309398\\
869	0.327169808909185\\
870	0.326521312494551\\
871	0.326307117009969\\
872	0.325879130929877\\
873	0.326430369489172\\
874	0.326186642921637\\
875	0.326073838572025\\
876	0.325467265948978\\
877	0.325468541684039\\
878	0.324930532567762\\
879	0.324313894599842\\
880	0.323545322662676\\
881	0.323190118908083\\
882	0.322725301757643\\
883	0.321977351795743\\
884	0.322372873605321\\
885	0.322079338733016\\
886	0.321280094434301\\
887	0.320695608332143\\
888	0.320480817012021\\
889	0.320305004041177\\
890	0.320220546652249\\
891	0.319711527728584\\
892	0.319730834400789\\
893	0.318938282305962\\
894	0.317998456069918\\
895	0.317192813563359\\
896	0.316341878377312\\
897	0.31543650928331\\
898	0.313881573611628\\
899	0.313280571159335\\
900	0.312747159304018\\
901	0.312529846946358\\
902	0.312239265457311\\
903	0.311485327550023\\
904	0.31142567023495\\
905	0.31093128431706\\
906	0.311201910512526\\
907	0.310944098538645\\
908	0.310464716003666\\
909	0.309588738111601\\
910	0.309283837407499\\
911	0.309709020992295\\
912	0.309463946934477\\
913	0.309697783017095\\
914	0.308694564515444\\
915	0.308542474881\\
916	0.307784099737483\\
917	0.307354233203675\\
918	0.30753234007699\\
919	0.306724221734863\\
920	0.306068981649268\\
921	0.305685946194188\\
922	0.304985577194473\\
923	0.304876198397068\\
924	0.304776353918782\\
925	0.303857024155001\\
926	0.303286864730371\\
927	0.302301208193809\\
928	0.302292279545927\\
929	0.301432345902569\\
930	0.30073604844262\\
931	0.300232877720175\\
932	0.300298345457082\\
933	0.300510257080957\\
934	0.300327027066496\\
935	0.299575888926776\\
936	0.299511844584647\\
937	0.298734664384855\\
938	0.297463913410112\\
939	0.296669758099267\\
940	0.29642436840234\\
941	0.295654145096775\\
942	0.295293383673972\\
943	0.294262065272874\\
944	0.293826310670656\\
945	0.293199022757407\\
946	0.292284740947036\\
947	0.292112336436886\\
948	0.291192930580631\\
949	0.29092485088128\\
950	0.290453386465016\\
951	0.28938811314654\\
952	0.288618522830896\\
953	0.287481650755337\\
954	0.287038835950244\\
955	0.286173556376884\\
956	0.285611979744058\\
957	0.284634029714487\\
958	0.284019584018597\\
959	0.283547689008701\\
960	0.283387366456833\\
961	0.28312481434951\\
962	0.283140768191691\\
963	0.282804257083293\\
964	0.282955705737634\\
965	0.282453215514962\\
966	0.281528517381071\\
967	0.280974755578353\\
968	0.281187328408017\\
969	0.280174005013353\\
970	0.280422050371319\\
971	0.27997345771092\\
972	0.279459095560173\\
973	0.279087713893094\\
974	0.279461445138391\\
975	0.279262842825217\\
976	0.278235220994428\\
977	0.278292835354202\\
978	0.277641509294552\\
979	0.277484536818641\\
980	0.277211812633252\\
981	0.276784640132169\\
982	0.276739382096243\\
983	0.276559683248046\\
984	0.275985910514577\\
985	0.275962460123419\\
986	0.27597614972481\\
987	0.275084543265945\\
988	0.274484875274133\\
989	0.274281257836399\\
990	0.273698874673708\\
991	0.273986278566509\\
992	0.272926498513193\\
993	0.272422492790299\\
994	0.271936550709003\\
995	0.271870100449497\\
996	0.270621483412832\\
997	0.270194243503858\\
998	0.269993149171878\\
999	0.270135682718851\\
1000	0.269247337297298\\
};
\addlegendentry{other-distributed}

\end{axis}
\end{tikzpicture}%
      \caption{Here we see the convergence of the posterior probability.}
    \end{figure}
    \only<article>{As can be seen in the figure above, in both cases, the posterior converges to the correct value, so it can be used to indicate our confidence that the null is true.}
  }
  \only<2>{
    \begin{figure}[H]
      \centering
      % This file was created by matlab2tikz.
%
%The latest updates can be retrieved from
%  http://www.mathworks.com/matlabcentral/fileexchange/22022-matlab2tikz-matlab2tikz
%where you can also make suggestions and rate matlab2tikz.
%
\begin{tikzpicture}

\begin{axis}[%
width=0.951\fwidth,
height=\fheight,
at={(0\fwidth,0\fheight)},
scale only axis,
xmin=0,
xmax=1000,
ymin=0,
ymax=0.7,
axis background/.style={fill=white},
title={Rejection of null hypothesis for Bernoulli(0.5)},
legend style={legend cell align=left, align=left, legend plot pos=left, draw=black}
]
\addplot [color=blue]
  table[row sep=crcr]{%
1	0\\
2	0\\
3	0\\
4	0\\
5	0.042\\
6	0.022\\
7	0.014\\
8	0.042\\
9	0.027\\
10	0.014\\
11	0.04\\
12	0.025\\
13	0.061\\
14	0.036\\
15	0.018\\
16	0.047\\
17	0.026\\
18	0.055\\
19	0.039\\
20	0.026\\
21	0.044\\
22	0.035\\
23	0.05\\
24	0.035\\
25	0.025\\
26	0.045\\
27	0.034\\
28	0.054\\
29	0.038\\
30	0.065\\
31	0.046\\
32	0.028\\
33	0.052\\
34	0.03\\
35	0.05\\
36	0.031\\
37	0.053\\
38	0.036\\
39	0.026\\
40	0.046\\
41	0.031\\
42	0.053\\
43	0.038\\
44	0.055\\
45	0.039\\
46	0.024\\
47	0.036\\
48	0.026\\
49	0.037\\
50	0.027\\
51	0.039\\
52	0.032\\
53	0.047\\
54	0.038\\
55	0.027\\
56	0.041\\
57	0.033\\
58	0.048\\
59	0.033\\
60	0.045\\
61	0.037\\
62	0.06\\
63	0.04\\
64	0.029\\
65	0.045\\
66	0.036\\
67	0.05\\
68	0.041\\
69	0.053\\
70	0.046\\
71	0.058\\
72	0.044\\
73	0.036\\
74	0.045\\
75	0.039\\
76	0.046\\
77	0.04\\
78	0.054\\
79	0.045\\
80	0.054\\
81	0.046\\
82	0.053\\
83	0.043\\
84	0.029\\
85	0.041\\
86	0.03\\
87	0.042\\
88	0.035\\
89	0.046\\
90	0.033\\
91	0.042\\
92	0.035\\
93	0.051\\
94	0.044\\
95	0.04\\
96	0.05\\
97	0.036\\
98	0.047\\
99	0.036\\
100	0.047\\
101	0.032\\
102	0.047\\
103	0.038\\
104	0.049\\
105	0.041\\
106	0.053\\
107	0.043\\
108	0.036\\
109	0.047\\
110	0.042\\
111	0.048\\
112	0.04\\
113	0.045\\
114	0.038\\
115	0.047\\
116	0.043\\
117	0.049\\
118	0.035\\
119	0.048\\
120	0.04\\
121	0.026\\
122	0.037\\
123	0.026\\
124	0.035\\
125	0.027\\
126	0.043\\
127	0.031\\
128	0.038\\
129	0.035\\
130	0.045\\
131	0.039\\
132	0.051\\
133	0.046\\
134	0.038\\
135	0.044\\
136	0.037\\
137	0.044\\
138	0.033\\
139	0.048\\
140	0.038\\
141	0.046\\
142	0.037\\
143	0.047\\
144	0.041\\
145	0.046\\
146	0.045\\
147	0.052\\
148	0.047\\
149	0.041\\
150	0.043\\
151	0.037\\
152	0.05\\
153	0.046\\
154	0.05\\
155	0.045\\
156	0.05\\
157	0.044\\
158	0.05\\
159	0.047\\
160	0.052\\
161	0.045\\
162	0.05\\
163	0.046\\
164	0.037\\
165	0.046\\
166	0.037\\
167	0.05\\
168	0.041\\
169	0.047\\
170	0.041\\
171	0.053\\
172	0.048\\
173	0.051\\
174	0.046\\
175	0.049\\
176	0.046\\
177	0.054\\
178	0.047\\
179	0.053\\
180	0.044\\
181	0.036\\
182	0.043\\
183	0.038\\
184	0.041\\
185	0.04\\
186	0.046\\
187	0.041\\
188	0.048\\
189	0.045\\
190	0.05\\
191	0.044\\
192	0.048\\
193	0.043\\
194	0.052\\
195	0.043\\
196	0.036\\
197	0.045\\
198	0.039\\
199	0.047\\
200	0.041\\
201	0.047\\
202	0.04\\
203	0.048\\
204	0.041\\
205	0.05\\
206	0.042\\
207	0.047\\
208	0.04\\
209	0.046\\
210	0.039\\
211	0.044\\
212	0.039\\
213	0.047\\
214	0.041\\
215	0.034\\
216	0.043\\
217	0.037\\
218	0.044\\
219	0.035\\
220	0.042\\
221	0.04\\
222	0.045\\
223	0.038\\
224	0.049\\
225	0.04\\
226	0.047\\
227	0.041\\
228	0.051\\
229	0.041\\
230	0.052\\
231	0.044\\
232	0.035\\
233	0.04\\
234	0.036\\
235	0.039\\
236	0.034\\
237	0.041\\
238	0.037\\
239	0.041\\
240	0.035\\
241	0.04\\
242	0.034\\
243	0.043\\
244	0.035\\
245	0.04\\
246	0.037\\
247	0.041\\
248	0.035\\
249	0.041\\
250	0.033\\
251	0.033\\
252	0.038\\
253	0.036\\
254	0.039\\
255	0.033\\
256	0.038\\
257	0.036\\
258	0.041\\
259	0.035\\
260	0.041\\
261	0.031\\
262	0.04\\
263	0.033\\
264	0.042\\
265	0.034\\
266	0.044\\
267	0.038\\
268	0.043\\
269	0.033\\
270	0.028\\
271	0.036\\
272	0.028\\
273	0.033\\
274	0.027\\
275	0.033\\
276	0.028\\
277	0.033\\
278	0.028\\
279	0.038\\
280	0.03\\
281	0.039\\
282	0.03\\
283	0.038\\
284	0.031\\
285	0.038\\
286	0.035\\
287	0.037\\
288	0.032\\
289	0.038\\
290	0.031\\
291	0.028\\
292	0.033\\
293	0.032\\
294	0.033\\
295	0.031\\
296	0.035\\
297	0.03\\
298	0.038\\
299	0.036\\
300	0.038\\
301	0.036\\
302	0.039\\
303	0.037\\
304	0.041\\
305	0.04\\
306	0.044\\
307	0.04\\
308	0.044\\
309	0.041\\
310	0.042\\
311	0.038\\
312	0.033\\
313	0.038\\
314	0.033\\
315	0.039\\
316	0.036\\
317	0.04\\
318	0.037\\
319	0.042\\
320	0.037\\
321	0.041\\
322	0.036\\
323	0.041\\
324	0.037\\
325	0.042\\
326	0.036\\
327	0.039\\
328	0.035\\
329	0.04\\
330	0.037\\
331	0.045\\
332	0.037\\
333	0.033\\
334	0.04\\
335	0.036\\
336	0.041\\
337	0.039\\
338	0.044\\
339	0.042\\
340	0.043\\
341	0.039\\
342	0.043\\
343	0.04\\
344	0.043\\
345	0.038\\
346	0.041\\
347	0.039\\
348	0.044\\
349	0.042\\
350	0.047\\
351	0.042\\
352	0.046\\
353	0.042\\
354	0.049\\
355	0.04\\
356	0.035\\
357	0.041\\
358	0.037\\
359	0.039\\
360	0.038\\
361	0.039\\
362	0.035\\
363	0.04\\
364	0.034\\
365	0.039\\
366	0.036\\
367	0.039\\
368	0.035\\
369	0.042\\
370	0.032\\
371	0.04\\
372	0.033\\
373	0.039\\
374	0.036\\
375	0.037\\
376	0.031\\
377	0.035\\
378	0.034\\
379	0.029\\
380	0.032\\
381	0.028\\
382	0.033\\
383	0.028\\
384	0.033\\
385	0.03\\
386	0.035\\
387	0.026\\
388	0.033\\
389	0.028\\
390	0.03\\
391	0.029\\
392	0.032\\
393	0.031\\
394	0.034\\
395	0.033\\
396	0.038\\
397	0.031\\
398	0.033\\
399	0.03\\
400	0.035\\
401	0.028\\
402	0.037\\
403	0.033\\
404	0.031\\
405	0.032\\
406	0.027\\
407	0.032\\
408	0.03\\
409	0.034\\
410	0.03\\
411	0.035\\
412	0.029\\
413	0.035\\
414	0.03\\
415	0.037\\
416	0.035\\
417	0.038\\
418	0.032\\
419	0.038\\
420	0.033\\
421	0.035\\
422	0.034\\
423	0.038\\
424	0.036\\
425	0.038\\
426	0.033\\
427	0.036\\
428	0.033\\
429	0.029\\
430	0.032\\
431	0.032\\
432	0.034\\
433	0.03\\
434	0.032\\
435	0.028\\
436	0.034\\
437	0.031\\
438	0.037\\
439	0.034\\
440	0.039\\
441	0.033\\
442	0.038\\
443	0.032\\
444	0.037\\
445	0.034\\
446	0.04\\
447	0.034\\
448	0.04\\
449	0.037\\
450	0.04\\
451	0.035\\
452	0.038\\
453	0.036\\
454	0.031\\
455	0.034\\
456	0.028\\
457	0.031\\
458	0.031\\
459	0.032\\
460	0.031\\
461	0.034\\
462	0.028\\
463	0.031\\
464	0.029\\
465	0.033\\
466	0.029\\
467	0.037\\
468	0.033\\
469	0.038\\
470	0.036\\
471	0.038\\
472	0.037\\
473	0.038\\
474	0.036\\
475	0.037\\
476	0.037\\
477	0.039\\
478	0.036\\
479	0.038\\
480	0.035\\
481	0.032\\
482	0.039\\
483	0.033\\
484	0.039\\
485	0.036\\
486	0.043\\
487	0.036\\
488	0.041\\
489	0.039\\
490	0.042\\
491	0.038\\
492	0.04\\
493	0.036\\
494	0.04\\
495	0.036\\
496	0.041\\
497	0.038\\
498	0.041\\
499	0.036\\
500	0.043\\
501	0.038\\
502	0.044\\
503	0.04\\
504	0.045\\
505	0.042\\
506	0.044\\
507	0.042\\
508	0.038\\
509	0.043\\
510	0.038\\
511	0.038\\
512	0.035\\
513	0.037\\
514	0.034\\
515	0.038\\
516	0.035\\
517	0.039\\
518	0.037\\
519	0.039\\
520	0.036\\
521	0.041\\
522	0.035\\
523	0.039\\
524	0.036\\
525	0.04\\
526	0.038\\
527	0.04\\
528	0.037\\
529	0.042\\
530	0.038\\
531	0.043\\
532	0.039\\
533	0.042\\
534	0.04\\
535	0.038\\
536	0.04\\
537	0.038\\
538	0.04\\
539	0.037\\
540	0.04\\
541	0.037\\
542	0.039\\
543	0.036\\
544	0.038\\
545	0.038\\
546	0.04\\
547	0.037\\
548	0.041\\
549	0.037\\
550	0.042\\
551	0.038\\
552	0.045\\
553	0.041\\
554	0.045\\
555	0.042\\
556	0.046\\
557	0.041\\
558	0.045\\
559	0.041\\
560	0.047\\
561	0.041\\
562	0.045\\
563	0.04\\
564	0.036\\
565	0.043\\
566	0.04\\
567	0.045\\
568	0.04\\
569	0.042\\
570	0.04\\
571	0.045\\
572	0.039\\
573	0.045\\
574	0.04\\
575	0.045\\
576	0.039\\
577	0.049\\
578	0.043\\
579	0.049\\
580	0.042\\
581	0.05\\
582	0.044\\
583	0.05\\
584	0.046\\
585	0.048\\
586	0.043\\
587	0.047\\
588	0.043\\
589	0.047\\
590	0.04\\
591	0.043\\
592	0.04\\
593	0.036\\
594	0.04\\
595	0.036\\
596	0.04\\
597	0.035\\
598	0.037\\
599	0.033\\
600	0.037\\
601	0.034\\
602	0.039\\
603	0.037\\
604	0.041\\
605	0.039\\
606	0.04\\
607	0.037\\
608	0.042\\
609	0.039\\
610	0.044\\
611	0.038\\
612	0.045\\
613	0.04\\
614	0.046\\
615	0.042\\
616	0.045\\
617	0.043\\
618	0.045\\
619	0.039\\
620	0.042\\
621	0.039\\
622	0.036\\
623	0.039\\
624	0.036\\
625	0.04\\
626	0.036\\
627	0.041\\
628	0.035\\
629	0.041\\
630	0.035\\
631	0.041\\
632	0.038\\
633	0.04\\
634	0.037\\
635	0.04\\
636	0.037\\
637	0.042\\
638	0.037\\
639	0.04\\
640	0.037\\
641	0.041\\
642	0.036\\
643	0.042\\
644	0.038\\
645	0.042\\
646	0.036\\
647	0.041\\
648	0.039\\
649	0.041\\
650	0.038\\
651	0.044\\
652	0.041\\
653	0.038\\
654	0.04\\
655	0.04\\
656	0.041\\
657	0.04\\
658	0.041\\
659	0.038\\
660	0.042\\
661	0.038\\
662	0.039\\
663	0.036\\
664	0.042\\
665	0.039\\
666	0.04\\
667	0.039\\
668	0.042\\
669	0.039\\
670	0.04\\
671	0.04\\
672	0.041\\
673	0.038\\
674	0.04\\
675	0.038\\
676	0.042\\
677	0.041\\
678	0.043\\
679	0.042\\
680	0.043\\
681	0.041\\
682	0.042\\
683	0.04\\
684	0.039\\
685	0.041\\
686	0.039\\
687	0.042\\
688	0.037\\
689	0.041\\
690	0.036\\
691	0.04\\
692	0.036\\
693	0.041\\
694	0.038\\
695	0.042\\
696	0.039\\
697	0.043\\
698	0.038\\
699	0.041\\
700	0.035\\
701	0.039\\
702	0.036\\
703	0.038\\
704	0.034\\
705	0.041\\
706	0.034\\
707	0.04\\
708	0.038\\
709	0.042\\
710	0.037\\
711	0.04\\
712	0.037\\
713	0.043\\
714	0.041\\
715	0.043\\
716	0.042\\
717	0.039\\
718	0.042\\
719	0.039\\
720	0.041\\
721	0.037\\
722	0.041\\
723	0.038\\
724	0.043\\
725	0.041\\
726	0.044\\
727	0.041\\
728	0.045\\
729	0.042\\
730	0.044\\
731	0.043\\
732	0.043\\
733	0.042\\
734	0.044\\
735	0.042\\
736	0.044\\
737	0.042\\
738	0.046\\
739	0.042\\
740	0.044\\
741	0.043\\
742	0.044\\
743	0.042\\
744	0.043\\
745	0.041\\
746	0.043\\
747	0.041\\
748	0.043\\
749	0.041\\
750	0.037\\
751	0.042\\
752	0.038\\
753	0.041\\
754	0.039\\
755	0.04\\
756	0.039\\
757	0.04\\
758	0.038\\
759	0.04\\
760	0.038\\
761	0.04\\
762	0.039\\
763	0.042\\
764	0.038\\
765	0.04\\
766	0.039\\
767	0.041\\
768	0.039\\
769	0.043\\
770	0.041\\
771	0.042\\
772	0.04\\
773	0.045\\
774	0.039\\
775	0.045\\
776	0.041\\
777	0.043\\
778	0.041\\
779	0.043\\
780	0.04\\
781	0.048\\
782	0.043\\
783	0.038\\
784	0.043\\
785	0.041\\
786	0.045\\
787	0.045\\
788	0.048\\
789	0.042\\
790	0.044\\
791	0.041\\
792	0.046\\
793	0.042\\
794	0.047\\
795	0.044\\
796	0.046\\
797	0.044\\
798	0.045\\
799	0.044\\
800	0.045\\
801	0.042\\
802	0.044\\
803	0.042\\
804	0.045\\
805	0.04\\
806	0.047\\
807	0.041\\
808	0.046\\
809	0.043\\
810	0.048\\
811	0.041\\
812	0.046\\
813	0.043\\
814	0.047\\
815	0.043\\
816	0.045\\
817	0.043\\
818	0.038\\
819	0.043\\
820	0.041\\
821	0.043\\
822	0.041\\
823	0.041\\
824	0.038\\
825	0.045\\
826	0.041\\
827	0.042\\
828	0.039\\
829	0.044\\
830	0.042\\
831	0.047\\
832	0.044\\
833	0.049\\
834	0.041\\
835	0.046\\
836	0.042\\
837	0.044\\
838	0.041\\
839	0.044\\
840	0.043\\
841	0.045\\
842	0.04\\
843	0.046\\
844	0.041\\
845	0.043\\
846	0.04\\
847	0.047\\
848	0.045\\
849	0.047\\
850	0.044\\
851	0.047\\
852	0.043\\
853	0.041\\
854	0.042\\
855	0.039\\
856	0.041\\
857	0.04\\
858	0.042\\
859	0.04\\
860	0.042\\
861	0.039\\
862	0.041\\
863	0.038\\
864	0.041\\
865	0.039\\
866	0.04\\
867	0.039\\
868	0.041\\
869	0.04\\
870	0.043\\
871	0.039\\
872	0.042\\
873	0.039\\
874	0.041\\
875	0.039\\
876	0.04\\
877	0.037\\
878	0.043\\
879	0.038\\
880	0.045\\
881	0.039\\
882	0.042\\
883	0.038\\
884	0.042\\
885	0.038\\
886	0.043\\
887	0.037\\
888	0.035\\
889	0.038\\
890	0.033\\
891	0.036\\
892	0.034\\
893	0.037\\
894	0.033\\
895	0.036\\
896	0.034\\
897	0.038\\
898	0.034\\
899	0.04\\
900	0.035\\
901	0.036\\
902	0.034\\
903	0.039\\
904	0.037\\
905	0.044\\
906	0.041\\
907	0.043\\
908	0.04\\
909	0.043\\
910	0.041\\
911	0.043\\
912	0.04\\
913	0.043\\
914	0.04\\
915	0.045\\
916	0.04\\
917	0.041\\
918	0.038\\
919	0.04\\
920	0.036\\
921	0.04\\
922	0.038\\
923	0.04\\
924	0.038\\
925	0.037\\
926	0.038\\
927	0.037\\
928	0.04\\
929	0.038\\
930	0.04\\
931	0.038\\
932	0.042\\
933	0.039\\
934	0.04\\
935	0.037\\
936	0.04\\
937	0.038\\
938	0.04\\
939	0.038\\
940	0.042\\
941	0.038\\
942	0.041\\
943	0.039\\
944	0.041\\
945	0.038\\
946	0.042\\
947	0.039\\
948	0.042\\
949	0.038\\
950	0.043\\
951	0.041\\
952	0.044\\
953	0.038\\
954	0.042\\
955	0.038\\
956	0.042\\
957	0.039\\
958	0.044\\
959	0.038\\
960	0.042\\
961	0.037\\
962	0.035\\
963	0.04\\
964	0.037\\
965	0.044\\
966	0.04\\
967	0.043\\
968	0.041\\
969	0.041\\
970	0.038\\
971	0.041\\
972	0.039\\
973	0.041\\
974	0.04\\
975	0.043\\
976	0.04\\
977	0.044\\
978	0.037\\
979	0.043\\
980	0.039\\
981	0.042\\
982	0.041\\
983	0.043\\
984	0.042\\
985	0.042\\
986	0.041\\
987	0.043\\
988	0.042\\
989	0.046\\
990	0.045\\
991	0.047\\
992	0.044\\
993	0.046\\
994	0.043\\
995	0.045\\
996	0.041\\
997	0.044\\
998	0.043\\
999	0.046\\
1000	0.043\\
};
\addlegendentry{null test}

\addplot [color=black!50!green]
  table[row sep=crcr]{%
1	0\\
2	0.505\\
3	0.261\\
4	0.135\\
5	0.395\\
6	0.23\\
7	0.14\\
8	0.306\\
9	0.201\\
10	0.124\\
11	0.232\\
12	0.156\\
13	0.108\\
14	0.176\\
15	0.116\\
16	0.08\\
17	0.147\\
18	0.102\\
19	0.072\\
20	0.119\\
21	0.088\\
22	0.14\\
23	0.096\\
24	0.065\\
25	0.106\\
26	0.075\\
27	0.123\\
28	0.094\\
29	0.066\\
30	0.105\\
31	0.077\\
32	0.114\\
33	0.088\\
34	0.057\\
35	0.087\\
36	0.06\\
37	0.094\\
38	0.067\\
39	0.047\\
40	0.084\\
41	0.056\\
42	0.091\\
43	0.063\\
44	0.049\\
45	0.069\\
46	0.046\\
47	0.066\\
48	0.05\\
49	0.04\\
50	0.058\\
51	0.045\\
52	0.069\\
53	0.053\\
54	0.074\\
55	0.056\\
56	0.036\\
57	0.058\\
58	0.038\\
59	0.064\\
60	0.046\\
61	0.036\\
62	0.051\\
63	0.039\\
64	0.064\\
65	0.049\\
66	0.069\\
67	0.056\\
68	0.041\\
69	0.06\\
70	0.051\\
71	0.071\\
72	0.055\\
73	0.067\\
74	0.056\\
75	0.041\\
76	0.06\\
77	0.043\\
78	0.062\\
79	0.044\\
80	0.058\\
81	0.048\\
82	0.038\\
83	0.05\\
84	0.042\\
85	0.049\\
86	0.041\\
87	0.053\\
88	0.042\\
89	0.038\\
90	0.048\\
91	0.036\\
92	0.052\\
93	0.041\\
94	0.054\\
95	0.039\\
96	0.054\\
97	0.046\\
98	0.034\\
99	0.045\\
100	0.034\\
101	0.046\\
102	0.037\\
103	0.046\\
104	0.034\\
105	0.028\\
106	0.037\\
107	0.033\\
108	0.044\\
109	0.035\\
110	0.045\\
111	0.039\\
112	0.046\\
113	0.037\\
114	0.028\\
115	0.034\\
116	0.03\\
117	0.038\\
118	0.03\\
119	0.037\\
120	0.028\\
121	0.034\\
122	0.031\\
123	0.029\\
124	0.034\\
125	0.031\\
126	0.033\\
127	0.033\\
128	0.039\\
129	0.035\\
130	0.029\\
131	0.031\\
132	0.029\\
133	0.034\\
134	0.03\\
135	0.033\\
136	0.029\\
137	0.037\\
138	0.032\\
139	0.029\\
140	0.032\\
141	0.028\\
142	0.036\\
143	0.031\\
144	0.037\\
145	0.032\\
146	0.042\\
147	0.032\\
148	0.044\\
149	0.034\\
150	0.027\\
151	0.035\\
152	0.03\\
153	0.037\\
154	0.032\\
155	0.037\\
156	0.032\\
157	0.04\\
158	0.031\\
159	0.026\\
160	0.033\\
161	0.031\\
162	0.036\\
163	0.032\\
164	0.035\\
165	0.033\\
166	0.04\\
167	0.032\\
168	0.026\\
169	0.034\\
170	0.029\\
171	0.034\\
172	0.03\\
173	0.035\\
174	0.028\\
175	0.034\\
176	0.032\\
177	0.038\\
178	0.03\\
179	0.027\\
180	0.034\\
181	0.028\\
182	0.033\\
183	0.03\\
184	0.036\\
185	0.031\\
186	0.037\\
187	0.031\\
188	0.026\\
189	0.033\\
190	0.025\\
191	0.032\\
192	0.028\\
193	0.034\\
194	0.028\\
195	0.035\\
196	0.029\\
197	0.038\\
198	0.035\\
199	0.025\\
200	0.031\\
201	0.028\\
202	0.032\\
203	0.032\\
204	0.034\\
205	0.03\\
206	0.037\\
207	0.034\\
208	0.037\\
209	0.03\\
210	0.028\\
211	0.032\\
212	0.028\\
213	0.035\\
214	0.029\\
215	0.04\\
216	0.031\\
217	0.035\\
218	0.031\\
219	0.034\\
220	0.028\\
221	0.026\\
222	0.027\\
223	0.025\\
224	0.028\\
225	0.028\\
226	0.032\\
227	0.026\\
228	0.028\\
229	0.026\\
230	0.03\\
231	0.026\\
232	0.024\\
233	0.029\\
234	0.024\\
235	0.028\\
236	0.025\\
237	0.03\\
238	0.028\\
239	0.031\\
240	0.027\\
241	0.032\\
242	0.027\\
243	0.026\\
244	0.028\\
245	0.022\\
246	0.024\\
247	0.021\\
248	0.027\\
249	0.023\\
250	0.031\\
251	0.028\\
252	0.032\\
253	0.03\\
254	0.031\\
255	0.026\\
256	0.023\\
257	0.028\\
258	0.024\\
259	0.028\\
260	0.025\\
261	0.029\\
262	0.022\\
263	0.027\\
264	0.024\\
265	0.027\\
266	0.026\\
267	0.021\\
268	0.026\\
269	0.02\\
270	0.024\\
271	0.021\\
272	0.025\\
273	0.024\\
274	0.026\\
275	0.023\\
276	0.027\\
277	0.023\\
278	0.027\\
279	0.024\\
280	0.019\\
281	0.022\\
282	0.022\\
283	0.023\\
284	0.023\\
285	0.025\\
286	0.022\\
287	0.023\\
288	0.021\\
289	0.023\\
290	0.022\\
291	0.02\\
292	0.021\\
293	0.021\\
294	0.021\\
295	0.019\\
296	0.021\\
297	0.019\\
298	0.024\\
299	0.022\\
300	0.024\\
301	0.022\\
302	0.025\\
303	0.022\\
304	0.02\\
305	0.022\\
306	0.02\\
307	0.024\\
308	0.022\\
309	0.026\\
310	0.023\\
311	0.024\\
312	0.023\\
313	0.025\\
314	0.022\\
315	0.024\\
316	0.023\\
317	0.019\\
318	0.021\\
319	0.021\\
320	0.022\\
321	0.019\\
322	0.02\\
323	0.019\\
324	0.019\\
325	0.019\\
326	0.019\\
327	0.019\\
328	0.019\\
329	0.018\\
330	0.018\\
331	0.019\\
332	0.018\\
333	0.021\\
334	0.017\\
335	0.02\\
336	0.017\\
337	0.018\\
338	0.017\\
339	0.02\\
340	0.017\\
341	0.019\\
342	0.016\\
343	0.016\\
344	0.017\\
345	0.015\\
346	0.02\\
347	0.016\\
348	0.019\\
349	0.016\\
350	0.018\\
351	0.017\\
352	0.018\\
353	0.016\\
354	0.017\\
355	0.015\\
356	0.016\\
357	0.014\\
358	0.013\\
359	0.017\\
360	0.016\\
361	0.018\\
362	0.016\\
363	0.017\\
364	0.014\\
365	0.016\\
366	0.015\\
367	0.015\\
368	0.015\\
369	0.015\\
370	0.015\\
371	0.014\\
372	0.015\\
373	0.015\\
374	0.015\\
375	0.015\\
376	0.016\\
377	0.015\\
378	0.017\\
379	0.014\\
380	0.016\\
381	0.014\\
382	0.019\\
383	0.015\\
384	0.018\\
385	0.015\\
386	0.013\\
387	0.016\\
388	0.013\\
389	0.015\\
390	0.013\\
391	0.017\\
392	0.014\\
393	0.019\\
394	0.016\\
395	0.018\\
396	0.016\\
397	0.018\\
398	0.016\\
399	0.015\\
400	0.018\\
401	0.016\\
402	0.018\\
403	0.017\\
404	0.017\\
405	0.015\\
406	0.017\\
407	0.015\\
408	0.015\\
409	0.014\\
410	0.017\\
411	0.011\\
412	0.015\\
413	0.012\\
414	0.009\\
415	0.014\\
416	0.012\\
417	0.014\\
418	0.012\\
419	0.013\\
420	0.01\\
421	0.014\\
422	0.012\\
423	0.015\\
424	0.014\\
425	0.016\\
426	0.014\\
427	0.015\\
428	0.013\\
429	0.012\\
430	0.014\\
431	0.013\\
432	0.014\\
433	0.013\\
434	0.016\\
435	0.015\\
436	0.017\\
437	0.016\\
438	0.017\\
439	0.014\\
440	0.016\\
441	0.014\\
442	0.015\\
443	0.015\\
444	0.014\\
445	0.014\\
446	0.014\\
447	0.014\\
448	0.013\\
449	0.014\\
450	0.013\\
451	0.014\\
452	0.014\\
453	0.016\\
454	0.015\\
455	0.017\\
456	0.016\\
457	0.019\\
458	0.018\\
459	0.017\\
460	0.017\\
461	0.015\\
462	0.016\\
463	0.014\\
464	0.016\\
465	0.014\\
466	0.015\\
467	0.013\\
468	0.016\\
469	0.013\\
470	0.015\\
471	0.011\\
472	0.013\\
473	0.011\\
474	0.011\\
475	0.013\\
476	0.011\\
477	0.014\\
478	0.012\\
479	0.015\\
480	0.015\\
481	0.015\\
482	0.014\\
483	0.015\\
484	0.014\\
485	0.017\\
486	0.014\\
487	0.017\\
488	0.015\\
489	0.017\\
490	0.016\\
491	0.016\\
492	0.017\\
493	0.015\\
494	0.017\\
495	0.017\\
496	0.019\\
497	0.017\\
498	0.019\\
499	0.018\\
500	0.019\\
501	0.018\\
502	0.02\\
503	0.019\\
504	0.023\\
505	0.019\\
506	0.017\\
507	0.02\\
508	0.018\\
509	0.02\\
510	0.017\\
511	0.018\\
512	0.017\\
513	0.018\\
514	0.017\\
515	0.02\\
516	0.017\\
517	0.018\\
518	0.017\\
519	0.017\\
520	0.017\\
521	0.017\\
522	0.017\\
523	0.016\\
524	0.018\\
525	0.017\\
526	0.019\\
527	0.017\\
528	0.019\\
529	0.019\\
530	0.019\\
531	0.019\\
532	0.019\\
533	0.016\\
534	0.017\\
535	0.017\\
536	0.02\\
537	0.016\\
538	0.015\\
539	0.018\\
540	0.018\\
541	0.019\\
542	0.015\\
543	0.018\\
544	0.016\\
545	0.019\\
546	0.014\\
547	0.015\\
548	0.014\\
549	0.016\\
550	0.015\\
551	0.018\\
552	0.017\\
553	0.018\\
554	0.017\\
555	0.016\\
556	0.017\\
557	0.016\\
558	0.017\\
559	0.014\\
560	0.015\\
561	0.013\\
562	0.014\\
563	0.014\\
564	0.015\\
565	0.014\\
566	0.018\\
567	0.015\\
568	0.017\\
569	0.017\\
570	0.017\\
571	0.014\\
572	0.014\\
573	0.015\\
574	0.015\\
575	0.016\\
576	0.013\\
577	0.016\\
578	0.013\\
579	0.013\\
580	0.013\\
581	0.016\\
582	0.014\\
583	0.017\\
584	0.015\\
585	0.017\\
586	0.017\\
587	0.017\\
588	0.016\\
589	0.015\\
590	0.016\\
591	0.015\\
592	0.016\\
593	0.016\\
594	0.018\\
595	0.015\\
596	0.016\\
597	0.015\\
598	0.017\\
599	0.014\\
600	0.016\\
601	0.016\\
602	0.017\\
603	0.017\\
604	0.017\\
605	0.016\\
606	0.013\\
607	0.014\\
608	0.014\\
609	0.014\\
610	0.013\\
611	0.016\\
612	0.014\\
613	0.015\\
614	0.014\\
615	0.016\\
616	0.016\\
617	0.016\\
618	0.014\\
619	0.015\\
620	0.013\\
621	0.015\\
622	0.014\\
623	0.012\\
624	0.016\\
625	0.014\\
626	0.015\\
627	0.014\\
628	0.015\\
629	0.015\\
630	0.016\\
631	0.015\\
632	0.016\\
633	0.015\\
634	0.016\\
635	0.015\\
636	0.016\\
637	0.015\\
638	0.015\\
639	0.015\\
640	0.015\\
641	0.015\\
642	0.014\\
643	0.015\\
644	0.014\\
645	0.015\\
646	0.014\\
647	0.016\\
648	0.014\\
649	0.016\\
650	0.016\\
651	0.016\\
652	0.014\\
653	0.017\\
654	0.015\\
655	0.016\\
656	0.014\\
657	0.014\\
658	0.013\\
659	0.013\\
660	0.015\\
661	0.015\\
662	0.016\\
663	0.014\\
664	0.015\\
665	0.012\\
666	0.017\\
667	0.015\\
668	0.016\\
669	0.016\\
670	0.017\\
671	0.015\\
672	0.018\\
673	0.016\\
674	0.017\\
675	0.016\\
676	0.015\\
677	0.018\\
678	0.015\\
679	0.017\\
680	0.016\\
681	0.016\\
682	0.016\\
683	0.017\\
684	0.016\\
685	0.017\\
686	0.014\\
687	0.016\\
688	0.015\\
689	0.016\\
690	0.015\\
691	0.016\\
692	0.016\\
693	0.017\\
694	0.017\\
695	0.014\\
696	0.017\\
697	0.014\\
698	0.015\\
699	0.013\\
700	0.015\\
701	0.014\\
702	0.014\\
703	0.014\\
704	0.016\\
705	0.015\\
706	0.017\\
707	0.015\\
708	0.016\\
709	0.016\\
710	0.018\\
711	0.016\\
712	0.014\\
713	0.016\\
714	0.014\\
715	0.015\\
716	0.013\\
717	0.014\\
718	0.014\\
719	0.014\\
720	0.014\\
721	0.015\\
722	0.014\\
723	0.015\\
724	0.014\\
725	0.017\\
726	0.016\\
727	0.019\\
728	0.014\\
729	0.018\\
730	0.016\\
731	0.013\\
732	0.014\\
733	0.014\\
734	0.014\\
735	0.014\\
736	0.014\\
737	0.014\\
738	0.015\\
739	0.014\\
740	0.015\\
741	0.014\\
742	0.017\\
743	0.013\\
744	0.015\\
745	0.014\\
746	0.015\\
747	0.013\\
748	0.013\\
749	0.013\\
750	0.012\\
751	0.013\\
752	0.013\\
753	0.014\\
754	0.012\\
755	0.012\\
756	0.011\\
757	0.012\\
758	0.011\\
759	0.014\\
760	0.011\\
761	0.015\\
762	0.01\\
763	0.012\\
764	0.011\\
765	0.013\\
766	0.012\\
767	0.015\\
768	0.011\\
769	0.008\\
770	0.01\\
771	0.01\\
772	0.011\\
773	0.011\\
774	0.011\\
775	0.01\\
776	0.011\\
777	0.01\\
778	0.011\\
779	0.011\\
780	0.013\\
781	0.012\\
782	0.014\\
783	0.012\\
784	0.014\\
785	0.011\\
786	0.014\\
787	0.012\\
788	0.011\\
789	0.013\\
790	0.013\\
791	0.013\\
792	0.013\\
793	0.013\\
794	0.013\\
795	0.014\\
796	0.013\\
797	0.014\\
798	0.012\\
799	0.013\\
800	0.012\\
801	0.013\\
802	0.012\\
803	0.014\\
804	0.012\\
805	0.013\\
806	0.011\\
807	0.013\\
808	0.013\\
809	0.013\\
810	0.014\\
811	0.013\\
812	0.013\\
813	0.012\\
814	0.012\\
815	0.012\\
816	0.013\\
817	0.011\\
818	0.012\\
819	0.011\\
820	0.011\\
821	0.011\\
822	0.012\\
823	0.01\\
824	0.012\\
825	0.011\\
826	0.011\\
827	0.011\\
828	0.011\\
829	0.011\\
830	0.01\\
831	0.01\\
832	0.01\\
833	0.01\\
834	0.009\\
835	0.01\\
836	0.01\\
837	0.01\\
838	0.01\\
839	0.01\\
840	0.009\\
841	0.01\\
842	0.008\\
843	0.01\\
844	0.009\\
845	0.01\\
846	0.009\\
847	0.011\\
848	0.008\\
849	0.007\\
850	0.009\\
851	0.007\\
852	0.008\\
853	0.008\\
854	0.008\\
855	0.008\\
856	0.009\\
857	0.009\\
858	0.009\\
859	0.008\\
860	0.009\\
861	0.007\\
862	0.009\\
863	0.008\\
864	0.01\\
865	0.009\\
866	0.009\\
867	0.009\\
868	0.007\\
869	0.009\\
870	0.007\\
871	0.009\\
872	0.008\\
873	0.008\\
874	0.008\\
875	0.008\\
876	0.007\\
877	0.008\\
878	0.007\\
879	0.009\\
880	0.008\\
881	0.008\\
882	0.008\\
883	0.008\\
884	0.008\\
885	0.009\\
886	0.008\\
887	0.008\\
888	0.008\\
889	0.008\\
890	0.008\\
891	0.008\\
892	0.009\\
893	0.007\\
894	0.008\\
895	0.006\\
896	0.007\\
897	0.006\\
898	0.009\\
899	0.008\\
900	0.008\\
901	0.008\\
902	0.008\\
903	0.008\\
904	0.008\\
905	0.008\\
906	0.009\\
907	0.008\\
908	0.01\\
909	0.01\\
910	0.009\\
911	0.009\\
912	0.008\\
913	0.009\\
914	0.008\\
915	0.009\\
916	0.009\\
917	0.009\\
918	0.009\\
919	0.009\\
920	0.009\\
921	0.009\\
922	0.009\\
923	0.009\\
924	0.009\\
925	0.009\\
926	0.009\\
927	0.009\\
928	0.009\\
929	0.008\\
930	0.009\\
931	0.009\\
932	0.009\\
933	0.008\\
934	0.009\\
935	0.008\\
936	0.01\\
937	0.008\\
938	0.009\\
939	0.008\\
940	0.009\\
941	0.008\\
942	0.008\\
943	0.008\\
944	0.009\\
945	0.008\\
946	0.01\\
947	0.008\\
948	0.009\\
949	0.008\\
950	0.007\\
951	0.007\\
952	0.006\\
953	0.007\\
954	0.006\\
955	0.007\\
956	0.007\\
957	0.009\\
958	0.008\\
959	0.008\\
960	0.007\\
961	0.007\\
962	0.007\\
963	0.007\\
964	0.007\\
965	0.007\\
966	0.007\\
967	0.007\\
968	0.006\\
969	0.009\\
970	0.007\\
971	0.006\\
972	0.006\\
973	0.006\\
974	0.008\\
975	0.007\\
976	0.008\\
977	0.007\\
978	0.007\\
979	0.007\\
980	0.008\\
981	0.008\\
982	0.008\\
983	0.007\\
984	0.009\\
985	0.008\\
986	0.009\\
987	0.008\\
988	0.009\\
989	0.008\\
990	0.008\\
991	0.008\\
992	0.009\\
993	0.006\\
994	0.006\\
995	0.006\\
996	0.006\\
997	0.006\\
998	0.006\\
999	0.006\\
1000	0.006\\
};
\addlegendentry{Bayes test}

\end{axis}
\end{tikzpicture}%
      \caption{Comparison of the rejection probability for the null and the Bayesian test when $\model_0$ is true.}
    \end{figure}
    \only<article>{Now we can use this Bayesian test, with uniform prior, to see how well it performs. While the plain null hypothesis test has a fixed rejection rate of $0.05$, the Bayesian test's rejection rate converges to 0 as we collect more data.}
  }
  \only<3>{
    \begin{figure}[H]
      \centering
      % This file was created by matlab2tikz.
%
%The latest updates can be retrieved from
%  http://www.mathworks.com/matlabcentral/fileexchange/22022-matlab2tikz-matlab2tikz
%where you can also make suggestions and rate matlab2tikz.
%
\begin{tikzpicture}

\begin{axis}[%
width=0.951\fwidth,
height=\fheight,
at={(0\fwidth,0\fheight)},
scale only axis,
xmin=0,
xmax=1000,
ymin=0,
ymax=1,
axis background/.style={fill=white},
title={Rejection of null hypothesis for Bernoulli(0.55)},
legend style={legend cell align=left, align=left, legend plot pos=left, draw=black}
]
\addplot [color=blue]
  table[row sep=crcr]{%
1	1\\
2	0.311\\
3	0.173\\
4	0.102\\
5	0.265\\
6	0.173\\
7	0.107\\
8	0.235\\
9	0.162\\
10	0.102\\
11	0.203\\
12	0.146\\
13	0.245\\
14	0.181\\
15	0.132\\
16	0.218\\
17	0.162\\
18	0.261\\
19	0.194\\
20	0.151\\
21	0.213\\
22	0.172\\
23	0.242\\
24	0.19\\
25	0.146\\
26	0.217\\
27	0.163\\
28	0.232\\
29	0.189\\
30	0.249\\
31	0.2\\
32	0.162\\
33	0.227\\
34	0.189\\
35	0.248\\
36	0.211\\
37	0.259\\
38	0.219\\
39	0.183\\
40	0.234\\
41	0.189\\
42	0.251\\
43	0.22\\
44	0.264\\
45	0.228\\
46	0.186\\
47	0.247\\
48	0.198\\
49	0.247\\
50	0.215\\
51	0.263\\
52	0.225\\
53	0.279\\
54	0.24\\
55	0.205\\
56	0.252\\
57	0.218\\
58	0.262\\
59	0.224\\
60	0.268\\
61	0.234\\
62	0.277\\
63	0.245\\
64	0.216\\
65	0.251\\
66	0.219\\
67	0.27\\
68	0.226\\
69	0.28\\
70	0.243\\
71	0.287\\
72	0.259\\
73	0.222\\
74	0.263\\
75	0.234\\
76	0.277\\
77	0.238\\
78	0.282\\
79	0.256\\
80	0.296\\
81	0.262\\
82	0.305\\
83	0.28\\
84	0.247\\
85	0.286\\
86	0.26\\
87	0.3\\
88	0.271\\
89	0.309\\
90	0.282\\
91	0.316\\
92	0.282\\
93	0.324\\
94	0.291\\
95	0.262\\
96	0.305\\
97	0.271\\
98	0.314\\
99	0.285\\
100	0.323\\
101	0.301\\
102	0.346\\
103	0.312\\
104	0.35\\
105	0.312\\
106	0.359\\
107	0.32\\
108	0.294\\
109	0.332\\
110	0.305\\
111	0.339\\
112	0.31\\
113	0.346\\
114	0.318\\
115	0.364\\
116	0.328\\
117	0.372\\
118	0.34\\
119	0.38\\
120	0.355\\
121	0.318\\
122	0.358\\
123	0.327\\
124	0.366\\
125	0.33\\
126	0.366\\
127	0.338\\
128	0.375\\
129	0.34\\
130	0.382\\
131	0.349\\
132	0.379\\
133	0.352\\
134	0.327\\
135	0.363\\
136	0.333\\
137	0.367\\
138	0.338\\
139	0.367\\
140	0.345\\
141	0.388\\
142	0.363\\
143	0.384\\
144	0.364\\
145	0.387\\
146	0.366\\
147	0.401\\
148	0.38\\
149	0.345\\
150	0.378\\
151	0.356\\
152	0.389\\
153	0.363\\
154	0.393\\
155	0.371\\
156	0.411\\
157	0.379\\
158	0.408\\
159	0.38\\
160	0.414\\
161	0.383\\
162	0.412\\
163	0.389\\
164	0.363\\
165	0.393\\
166	0.362\\
167	0.401\\
168	0.374\\
169	0.409\\
170	0.385\\
171	0.412\\
172	0.39\\
173	0.418\\
174	0.397\\
175	0.424\\
176	0.409\\
177	0.441\\
178	0.417\\
179	0.452\\
180	0.431\\
181	0.399\\
182	0.428\\
183	0.403\\
184	0.428\\
185	0.4\\
186	0.443\\
187	0.411\\
188	0.447\\
189	0.421\\
190	0.452\\
191	0.428\\
192	0.455\\
193	0.433\\
194	0.469\\
195	0.446\\
196	0.418\\
197	0.453\\
198	0.426\\
199	0.46\\
200	0.426\\
201	0.461\\
202	0.434\\
203	0.459\\
204	0.443\\
205	0.475\\
206	0.452\\
207	0.483\\
208	0.459\\
209	0.483\\
210	0.464\\
211	0.491\\
212	0.472\\
213	0.504\\
214	0.478\\
215	0.448\\
216	0.482\\
217	0.46\\
218	0.488\\
219	0.463\\
220	0.486\\
221	0.462\\
222	0.485\\
223	0.464\\
224	0.488\\
225	0.461\\
226	0.49\\
227	0.463\\
228	0.505\\
229	0.482\\
230	0.513\\
231	0.485\\
232	0.457\\
233	0.485\\
234	0.465\\
235	0.494\\
236	0.47\\
237	0.503\\
238	0.478\\
239	0.495\\
240	0.472\\
241	0.497\\
242	0.475\\
243	0.507\\
244	0.491\\
245	0.52\\
246	0.492\\
247	0.524\\
248	0.504\\
249	0.526\\
250	0.502\\
251	0.484\\
252	0.502\\
253	0.485\\
254	0.504\\
255	0.489\\
256	0.505\\
257	0.495\\
258	0.505\\
259	0.497\\
260	0.513\\
261	0.497\\
262	0.518\\
263	0.503\\
264	0.533\\
265	0.51\\
266	0.524\\
267	0.508\\
268	0.536\\
269	0.51\\
270	0.494\\
271	0.518\\
272	0.497\\
273	0.523\\
274	0.503\\
275	0.524\\
276	0.501\\
277	0.522\\
278	0.505\\
279	0.526\\
280	0.512\\
281	0.527\\
282	0.506\\
283	0.539\\
284	0.522\\
285	0.543\\
286	0.533\\
287	0.555\\
288	0.537\\
289	0.562\\
290	0.543\\
291	0.525\\
292	0.551\\
293	0.528\\
294	0.551\\
295	0.532\\
296	0.565\\
297	0.545\\
298	0.572\\
299	0.551\\
300	0.579\\
301	0.558\\
302	0.575\\
303	0.558\\
304	0.578\\
305	0.555\\
306	0.584\\
307	0.572\\
308	0.586\\
309	0.571\\
310	0.592\\
311	0.569\\
312	0.549\\
313	0.577\\
314	0.556\\
315	0.583\\
316	0.56\\
317	0.584\\
318	0.561\\
319	0.589\\
320	0.568\\
321	0.592\\
322	0.57\\
323	0.594\\
324	0.571\\
325	0.6\\
326	0.582\\
327	0.606\\
328	0.586\\
329	0.609\\
330	0.587\\
331	0.607\\
332	0.587\\
333	0.563\\
334	0.591\\
335	0.565\\
336	0.6\\
337	0.576\\
338	0.598\\
339	0.578\\
340	0.597\\
341	0.58\\
342	0.602\\
343	0.579\\
344	0.597\\
345	0.584\\
346	0.606\\
347	0.588\\
348	0.616\\
349	0.598\\
350	0.622\\
351	0.606\\
352	0.624\\
353	0.603\\
354	0.627\\
355	0.611\\
356	0.593\\
357	0.615\\
358	0.597\\
359	0.622\\
360	0.602\\
361	0.625\\
362	0.608\\
363	0.63\\
364	0.617\\
365	0.639\\
366	0.619\\
367	0.643\\
368	0.624\\
369	0.645\\
370	0.626\\
371	0.648\\
372	0.63\\
373	0.654\\
374	0.63\\
375	0.652\\
376	0.638\\
377	0.66\\
378	0.641\\
379	0.628\\
380	0.646\\
381	0.627\\
382	0.65\\
383	0.633\\
384	0.654\\
385	0.636\\
386	0.658\\
387	0.641\\
388	0.656\\
389	0.645\\
390	0.667\\
391	0.656\\
392	0.673\\
393	0.659\\
394	0.678\\
395	0.657\\
396	0.679\\
397	0.66\\
398	0.68\\
399	0.655\\
400	0.676\\
401	0.659\\
402	0.692\\
403	0.666\\
404	0.647\\
405	0.67\\
406	0.647\\
407	0.668\\
408	0.646\\
409	0.665\\
410	0.652\\
411	0.669\\
412	0.653\\
413	0.668\\
414	0.647\\
415	0.669\\
416	0.653\\
417	0.675\\
418	0.661\\
419	0.673\\
420	0.66\\
421	0.674\\
422	0.667\\
423	0.679\\
424	0.665\\
425	0.685\\
426	0.667\\
427	0.684\\
428	0.671\\
429	0.658\\
430	0.677\\
431	0.662\\
432	0.677\\
433	0.67\\
434	0.685\\
435	0.669\\
436	0.688\\
437	0.674\\
438	0.688\\
439	0.673\\
440	0.689\\
441	0.679\\
442	0.691\\
443	0.677\\
444	0.694\\
445	0.679\\
446	0.694\\
447	0.676\\
448	0.697\\
449	0.685\\
450	0.703\\
451	0.69\\
452	0.703\\
453	0.693\\
454	0.681\\
455	0.694\\
456	0.682\\
457	0.703\\
458	0.683\\
459	0.702\\
460	0.689\\
461	0.703\\
462	0.687\\
463	0.704\\
464	0.686\\
465	0.699\\
466	0.687\\
467	0.699\\
468	0.687\\
469	0.705\\
470	0.694\\
471	0.713\\
472	0.699\\
473	0.72\\
474	0.706\\
475	0.722\\
476	0.709\\
477	0.724\\
478	0.707\\
479	0.722\\
480	0.711\\
481	0.703\\
482	0.714\\
483	0.706\\
484	0.72\\
485	0.711\\
486	0.725\\
487	0.709\\
488	0.732\\
489	0.71\\
490	0.728\\
491	0.715\\
492	0.728\\
493	0.715\\
494	0.732\\
495	0.727\\
496	0.743\\
497	0.721\\
498	0.743\\
499	0.726\\
500	0.747\\
501	0.729\\
502	0.747\\
503	0.731\\
504	0.748\\
505	0.738\\
506	0.755\\
507	0.745\\
508	0.729\\
509	0.753\\
510	0.737\\
511	0.751\\
512	0.74\\
513	0.754\\
514	0.742\\
515	0.753\\
516	0.743\\
517	0.755\\
518	0.743\\
519	0.758\\
520	0.751\\
521	0.762\\
522	0.752\\
523	0.761\\
524	0.75\\
525	0.765\\
526	0.756\\
527	0.767\\
528	0.761\\
529	0.773\\
530	0.764\\
531	0.782\\
532	0.767\\
533	0.785\\
534	0.772\\
535	0.758\\
536	0.773\\
537	0.761\\
538	0.776\\
539	0.762\\
540	0.774\\
541	0.764\\
542	0.776\\
543	0.762\\
544	0.777\\
545	0.766\\
546	0.781\\
547	0.77\\
548	0.781\\
549	0.765\\
550	0.789\\
551	0.777\\
552	0.785\\
553	0.772\\
554	0.784\\
555	0.776\\
556	0.789\\
557	0.781\\
558	0.798\\
559	0.785\\
560	0.799\\
561	0.787\\
562	0.801\\
563	0.791\\
564	0.784\\
565	0.801\\
566	0.783\\
567	0.8\\
568	0.787\\
569	0.8\\
570	0.79\\
571	0.799\\
572	0.793\\
573	0.803\\
574	0.794\\
575	0.805\\
576	0.791\\
577	0.801\\
578	0.794\\
579	0.806\\
580	0.794\\
581	0.807\\
582	0.794\\
583	0.807\\
584	0.795\\
585	0.811\\
586	0.802\\
587	0.814\\
588	0.802\\
589	0.819\\
590	0.81\\
591	0.817\\
592	0.807\\
593	0.8\\
594	0.809\\
595	0.799\\
596	0.812\\
597	0.802\\
598	0.809\\
599	0.798\\
600	0.815\\
601	0.806\\
602	0.82\\
603	0.807\\
604	0.822\\
605	0.809\\
606	0.819\\
607	0.814\\
608	0.824\\
609	0.812\\
610	0.825\\
611	0.814\\
612	0.823\\
613	0.807\\
614	0.818\\
615	0.805\\
616	0.817\\
617	0.805\\
618	0.819\\
619	0.81\\
620	0.824\\
621	0.807\\
622	0.795\\
623	0.813\\
624	0.804\\
625	0.82\\
626	0.808\\
627	0.818\\
628	0.808\\
629	0.822\\
630	0.811\\
631	0.822\\
632	0.812\\
633	0.826\\
634	0.819\\
635	0.828\\
636	0.821\\
637	0.831\\
638	0.821\\
639	0.83\\
640	0.822\\
641	0.832\\
642	0.825\\
643	0.834\\
644	0.823\\
645	0.836\\
646	0.828\\
647	0.84\\
648	0.833\\
649	0.841\\
650	0.836\\
651	0.845\\
652	0.839\\
653	0.827\\
654	0.839\\
655	0.83\\
656	0.841\\
657	0.826\\
658	0.837\\
659	0.826\\
660	0.835\\
661	0.824\\
662	0.835\\
663	0.822\\
664	0.832\\
665	0.826\\
666	0.833\\
667	0.826\\
668	0.838\\
669	0.828\\
670	0.838\\
671	0.832\\
672	0.838\\
673	0.832\\
674	0.843\\
675	0.838\\
676	0.847\\
677	0.838\\
678	0.846\\
679	0.84\\
680	0.848\\
681	0.838\\
682	0.853\\
683	0.837\\
684	0.827\\
685	0.838\\
686	0.83\\
687	0.84\\
688	0.832\\
689	0.839\\
690	0.835\\
691	0.843\\
692	0.836\\
693	0.845\\
694	0.841\\
695	0.85\\
696	0.841\\
697	0.852\\
698	0.842\\
699	0.85\\
700	0.841\\
701	0.853\\
702	0.846\\
703	0.855\\
704	0.848\\
705	0.855\\
706	0.849\\
707	0.856\\
708	0.851\\
709	0.858\\
710	0.853\\
711	0.86\\
712	0.853\\
713	0.863\\
714	0.856\\
715	0.864\\
716	0.855\\
717	0.845\\
718	0.856\\
719	0.847\\
720	0.861\\
721	0.851\\
722	0.862\\
723	0.856\\
724	0.864\\
725	0.855\\
726	0.868\\
727	0.857\\
728	0.863\\
729	0.858\\
730	0.867\\
731	0.862\\
732	0.868\\
733	0.861\\
734	0.868\\
735	0.862\\
736	0.871\\
737	0.862\\
738	0.87\\
739	0.867\\
740	0.871\\
741	0.869\\
742	0.875\\
743	0.865\\
744	0.878\\
745	0.875\\
746	0.879\\
747	0.871\\
748	0.877\\
749	0.874\\
750	0.867\\
751	0.874\\
752	0.866\\
753	0.872\\
754	0.866\\
755	0.877\\
756	0.867\\
757	0.873\\
758	0.867\\
759	0.872\\
760	0.865\\
761	0.869\\
762	0.862\\
763	0.872\\
764	0.869\\
765	0.874\\
766	0.867\\
767	0.873\\
768	0.87\\
769	0.876\\
770	0.871\\
771	0.876\\
772	0.869\\
773	0.88\\
774	0.874\\
775	0.882\\
776	0.877\\
777	0.883\\
778	0.875\\
779	0.883\\
780	0.878\\
781	0.882\\
782	0.876\\
783	0.872\\
784	0.877\\
785	0.867\\
786	0.875\\
787	0.869\\
788	0.877\\
789	0.873\\
790	0.879\\
791	0.875\\
792	0.88\\
793	0.875\\
794	0.884\\
795	0.877\\
796	0.886\\
797	0.88\\
798	0.887\\
799	0.884\\
800	0.888\\
801	0.884\\
802	0.886\\
803	0.882\\
804	0.888\\
805	0.884\\
806	0.89\\
807	0.884\\
808	0.892\\
809	0.882\\
810	0.889\\
811	0.882\\
812	0.893\\
813	0.886\\
814	0.894\\
815	0.887\\
816	0.896\\
817	0.89\\
818	0.886\\
819	0.889\\
820	0.884\\
821	0.893\\
822	0.885\\
823	0.893\\
824	0.885\\
825	0.892\\
826	0.881\\
827	0.89\\
828	0.888\\
829	0.896\\
830	0.887\\
831	0.896\\
832	0.889\\
833	0.895\\
834	0.888\\
835	0.896\\
836	0.888\\
837	0.898\\
838	0.894\\
839	0.9\\
840	0.895\\
841	0.906\\
842	0.899\\
843	0.905\\
844	0.898\\
845	0.906\\
846	0.902\\
847	0.905\\
848	0.9\\
849	0.909\\
850	0.901\\
851	0.905\\
852	0.901\\
853	0.896\\
854	0.903\\
855	0.898\\
856	0.902\\
857	0.897\\
858	0.903\\
859	0.899\\
860	0.906\\
861	0.902\\
862	0.905\\
863	0.9\\
864	0.908\\
865	0.903\\
866	0.907\\
867	0.904\\
868	0.911\\
869	0.905\\
870	0.91\\
871	0.901\\
872	0.908\\
873	0.905\\
874	0.911\\
875	0.908\\
876	0.914\\
877	0.907\\
878	0.914\\
879	0.908\\
880	0.914\\
881	0.907\\
882	0.915\\
883	0.91\\
884	0.915\\
885	0.912\\
886	0.914\\
887	0.908\\
888	0.906\\
889	0.909\\
890	0.907\\
891	0.912\\
892	0.906\\
893	0.911\\
894	0.907\\
895	0.909\\
896	0.907\\
897	0.908\\
898	0.908\\
899	0.911\\
900	0.908\\
901	0.91\\
902	0.907\\
903	0.909\\
904	0.907\\
905	0.915\\
906	0.91\\
907	0.915\\
908	0.911\\
909	0.915\\
910	0.912\\
911	0.918\\
912	0.913\\
913	0.915\\
914	0.913\\
915	0.916\\
916	0.912\\
917	0.915\\
918	0.914\\
919	0.916\\
920	0.913\\
921	0.918\\
922	0.913\\
923	0.919\\
924	0.914\\
925	0.909\\
926	0.917\\
927	0.913\\
928	0.919\\
929	0.912\\
930	0.917\\
931	0.913\\
932	0.92\\
933	0.917\\
934	0.919\\
935	0.914\\
936	0.919\\
937	0.917\\
938	0.919\\
939	0.915\\
940	0.922\\
941	0.917\\
942	0.922\\
943	0.919\\
944	0.921\\
945	0.918\\
946	0.919\\
947	0.919\\
948	0.925\\
949	0.922\\
950	0.924\\
951	0.92\\
952	0.924\\
953	0.92\\
954	0.922\\
955	0.92\\
956	0.922\\
957	0.92\\
958	0.923\\
959	0.921\\
960	0.925\\
961	0.92\\
962	0.915\\
963	0.923\\
964	0.917\\
965	0.921\\
966	0.918\\
967	0.924\\
968	0.918\\
969	0.925\\
970	0.921\\
971	0.927\\
972	0.923\\
973	0.927\\
974	0.924\\
975	0.928\\
976	0.923\\
977	0.927\\
978	0.926\\
979	0.933\\
980	0.929\\
981	0.932\\
982	0.929\\
983	0.931\\
984	0.929\\
985	0.931\\
986	0.929\\
987	0.932\\
988	0.929\\
989	0.934\\
990	0.931\\
991	0.937\\
992	0.933\\
993	0.936\\
994	0.935\\
995	0.937\\
996	0.936\\
997	0.94\\
998	0.935\\
999	0.94\\
1000	0.936\\
};
\addlegendentry{null test}

\addplot [color=black!50!green]
  table[row sep=crcr]{%
1	0\\
2	0.506\\
3	0.262\\
4	0.138\\
5	0.375\\
6	0.24\\
7	0.139\\
8	0.326\\
9	0.212\\
10	0.131\\
11	0.264\\
12	0.188\\
13	0.124\\
14	0.225\\
15	0.162\\
16	0.118\\
17	0.197\\
18	0.139\\
19	0.102\\
20	0.172\\
21	0.129\\
22	0.197\\
23	0.138\\
24	0.097\\
25	0.168\\
26	0.119\\
27	0.186\\
28	0.141\\
29	0.111\\
30	0.172\\
31	0.142\\
32	0.185\\
33	0.145\\
34	0.119\\
35	0.173\\
36	0.136\\
37	0.184\\
38	0.151\\
39	0.109\\
40	0.152\\
41	0.125\\
42	0.178\\
43	0.142\\
44	0.11\\
45	0.151\\
46	0.119\\
47	0.168\\
48	0.139\\
49	0.109\\
50	0.138\\
51	0.11\\
52	0.143\\
53	0.122\\
54	0.164\\
55	0.134\\
56	0.108\\
57	0.154\\
58	0.121\\
59	0.16\\
60	0.133\\
61	0.106\\
62	0.143\\
63	0.119\\
64	0.155\\
65	0.126\\
66	0.166\\
67	0.141\\
68	0.116\\
69	0.146\\
70	0.123\\
71	0.15\\
72	0.128\\
73	0.158\\
74	0.134\\
75	0.117\\
76	0.144\\
77	0.132\\
78	0.162\\
79	0.134\\
80	0.166\\
81	0.142\\
82	0.122\\
83	0.152\\
84	0.131\\
85	0.155\\
86	0.136\\
87	0.166\\
88	0.143\\
89	0.126\\
90	0.157\\
91	0.138\\
92	0.163\\
93	0.147\\
94	0.173\\
95	0.155\\
96	0.181\\
97	0.164\\
98	0.142\\
99	0.168\\
100	0.141\\
101	0.166\\
102	0.147\\
103	0.173\\
104	0.153\\
105	0.131\\
106	0.163\\
107	0.138\\
108	0.176\\
109	0.149\\
110	0.181\\
111	0.155\\
112	0.184\\
113	0.16\\
114	0.146\\
115	0.168\\
116	0.156\\
117	0.174\\
118	0.158\\
119	0.176\\
120	0.161\\
121	0.192\\
122	0.17\\
123	0.15\\
124	0.176\\
125	0.157\\
126	0.179\\
127	0.168\\
128	0.191\\
129	0.17\\
130	0.154\\
131	0.17\\
132	0.151\\
133	0.183\\
134	0.166\\
135	0.189\\
136	0.177\\
137	0.203\\
138	0.177\\
139	0.161\\
140	0.193\\
141	0.171\\
142	0.193\\
143	0.173\\
144	0.201\\
145	0.179\\
146	0.218\\
147	0.196\\
148	0.225\\
149	0.204\\
150	0.18\\
151	0.207\\
152	0.193\\
153	0.211\\
154	0.195\\
155	0.224\\
156	0.205\\
157	0.228\\
158	0.217\\
159	0.197\\
160	0.218\\
161	0.206\\
162	0.23\\
163	0.205\\
164	0.231\\
165	0.204\\
166	0.236\\
167	0.208\\
168	0.193\\
169	0.212\\
170	0.201\\
171	0.216\\
172	0.197\\
173	0.22\\
174	0.201\\
175	0.226\\
176	0.207\\
177	0.226\\
178	0.211\\
179	0.198\\
180	0.214\\
181	0.201\\
182	0.224\\
183	0.201\\
184	0.226\\
185	0.206\\
186	0.236\\
187	0.21\\
188	0.196\\
189	0.214\\
190	0.196\\
191	0.224\\
192	0.206\\
193	0.231\\
194	0.216\\
195	0.235\\
196	0.219\\
197	0.236\\
198	0.219\\
199	0.205\\
200	0.226\\
201	0.213\\
202	0.234\\
203	0.212\\
204	0.234\\
205	0.218\\
206	0.239\\
207	0.217\\
208	0.236\\
209	0.22\\
210	0.207\\
211	0.226\\
212	0.216\\
213	0.231\\
214	0.219\\
215	0.233\\
216	0.218\\
217	0.237\\
218	0.222\\
219	0.25\\
220	0.227\\
221	0.206\\
222	0.225\\
223	0.208\\
224	0.237\\
225	0.222\\
226	0.242\\
227	0.228\\
228	0.25\\
229	0.236\\
230	0.26\\
231	0.24\\
232	0.228\\
233	0.251\\
234	0.232\\
235	0.258\\
236	0.241\\
237	0.265\\
238	0.249\\
239	0.273\\
240	0.258\\
241	0.284\\
242	0.268\\
243	0.248\\
244	0.27\\
245	0.256\\
246	0.269\\
247	0.257\\
248	0.274\\
249	0.258\\
250	0.278\\
251	0.262\\
252	0.284\\
253	0.277\\
254	0.285\\
255	0.275\\
256	0.265\\
257	0.275\\
258	0.261\\
259	0.277\\
260	0.264\\
261	0.286\\
262	0.271\\
263	0.283\\
264	0.267\\
265	0.286\\
266	0.266\\
267	0.257\\
268	0.273\\
269	0.26\\
270	0.284\\
271	0.261\\
272	0.284\\
273	0.267\\
274	0.292\\
275	0.271\\
276	0.291\\
277	0.277\\
278	0.307\\
279	0.283\\
280	0.275\\
281	0.288\\
282	0.269\\
283	0.292\\
284	0.275\\
285	0.293\\
286	0.277\\
287	0.299\\
288	0.279\\
289	0.3\\
290	0.292\\
291	0.276\\
292	0.293\\
293	0.282\\
294	0.298\\
295	0.282\\
296	0.301\\
297	0.286\\
298	0.308\\
299	0.291\\
300	0.309\\
301	0.285\\
302	0.316\\
303	0.293\\
304	0.282\\
305	0.305\\
306	0.286\\
307	0.301\\
308	0.284\\
309	0.304\\
310	0.295\\
311	0.312\\
312	0.291\\
313	0.314\\
314	0.295\\
315	0.316\\
316	0.3\\
317	0.285\\
318	0.304\\
319	0.291\\
320	0.314\\
321	0.292\\
322	0.317\\
323	0.302\\
324	0.323\\
325	0.309\\
326	0.326\\
327	0.314\\
328	0.334\\
329	0.321\\
330	0.298\\
331	0.319\\
332	0.304\\
333	0.325\\
334	0.304\\
335	0.326\\
336	0.309\\
337	0.334\\
338	0.31\\
339	0.334\\
340	0.32\\
341	0.344\\
342	0.327\\
343	0.311\\
344	0.33\\
345	0.317\\
346	0.339\\
347	0.322\\
348	0.34\\
349	0.325\\
350	0.341\\
351	0.32\\
352	0.352\\
353	0.331\\
354	0.353\\
355	0.334\\
356	0.353\\
357	0.341\\
358	0.32\\
359	0.338\\
360	0.321\\
361	0.35\\
362	0.33\\
363	0.348\\
364	0.332\\
365	0.348\\
366	0.333\\
367	0.358\\
368	0.345\\
369	0.363\\
370	0.348\\
371	0.327\\
372	0.345\\
373	0.325\\
374	0.347\\
375	0.336\\
376	0.357\\
377	0.337\\
378	0.355\\
379	0.334\\
380	0.366\\
381	0.349\\
382	0.363\\
383	0.353\\
384	0.373\\
385	0.36\\
386	0.343\\
387	0.361\\
388	0.341\\
389	0.356\\
390	0.342\\
391	0.358\\
392	0.346\\
393	0.374\\
394	0.356\\
395	0.373\\
396	0.358\\
397	0.38\\
398	0.36\\
399	0.341\\
400	0.367\\
401	0.352\\
402	0.375\\
403	0.349\\
404	0.37\\
405	0.349\\
406	0.368\\
407	0.357\\
408	0.378\\
409	0.357\\
410	0.382\\
411	0.359\\
412	0.384\\
413	0.366\\
414	0.352\\
415	0.368\\
416	0.354\\
417	0.373\\
418	0.357\\
419	0.372\\
420	0.357\\
421	0.385\\
422	0.368\\
423	0.387\\
424	0.371\\
425	0.391\\
426	0.37\\
427	0.399\\
428	0.383\\
429	0.36\\
430	0.385\\
431	0.371\\
432	0.385\\
433	0.371\\
434	0.385\\
435	0.373\\
436	0.394\\
437	0.377\\
438	0.4\\
439	0.383\\
440	0.399\\
441	0.387\\
442	0.409\\
443	0.394\\
444	0.373\\
445	0.4\\
446	0.381\\
447	0.397\\
448	0.378\\
449	0.402\\
450	0.383\\
451	0.404\\
452	0.387\\
453	0.416\\
454	0.402\\
455	0.426\\
456	0.412\\
457	0.425\\
458	0.409\\
459	0.394\\
460	0.409\\
461	0.39\\
462	0.408\\
463	0.392\\
464	0.413\\
465	0.401\\
466	0.415\\
467	0.405\\
468	0.419\\
469	0.406\\
470	0.423\\
471	0.411\\
472	0.425\\
473	0.409\\
474	0.401\\
475	0.415\\
476	0.402\\
477	0.418\\
478	0.404\\
479	0.415\\
480	0.406\\
481	0.423\\
482	0.411\\
483	0.429\\
484	0.417\\
485	0.432\\
486	0.417\\
487	0.442\\
488	0.423\\
489	0.446\\
490	0.427\\
491	0.419\\
492	0.434\\
493	0.415\\
494	0.438\\
495	0.42\\
496	0.443\\
497	0.426\\
498	0.449\\
499	0.435\\
500	0.455\\
501	0.437\\
502	0.457\\
503	0.436\\
504	0.454\\
505	0.443\\
506	0.425\\
507	0.448\\
508	0.438\\
509	0.451\\
510	0.436\\
511	0.453\\
512	0.437\\
513	0.456\\
514	0.44\\
515	0.457\\
516	0.436\\
517	0.459\\
518	0.44\\
519	0.458\\
520	0.446\\
521	0.432\\
522	0.45\\
523	0.434\\
524	0.447\\
525	0.437\\
526	0.452\\
527	0.436\\
528	0.454\\
529	0.439\\
530	0.457\\
531	0.445\\
532	0.458\\
533	0.441\\
534	0.457\\
535	0.443\\
536	0.461\\
537	0.446\\
538	0.434\\
539	0.45\\
540	0.436\\
541	0.449\\
542	0.435\\
543	0.454\\
544	0.442\\
545	0.458\\
546	0.444\\
547	0.454\\
548	0.443\\
549	0.463\\
550	0.446\\
551	0.462\\
552	0.452\\
553	0.468\\
554	0.455\\
555	0.44\\
556	0.461\\
557	0.447\\
558	0.463\\
559	0.448\\
560	0.466\\
561	0.45\\
562	0.464\\
563	0.453\\
564	0.473\\
565	0.458\\
566	0.477\\
567	0.467\\
568	0.479\\
569	0.465\\
570	0.485\\
571	0.469\\
572	0.455\\
573	0.473\\
574	0.46\\
575	0.475\\
576	0.465\\
577	0.475\\
578	0.467\\
579	0.478\\
580	0.471\\
581	0.488\\
582	0.477\\
583	0.491\\
584	0.481\\
585	0.499\\
586	0.483\\
587	0.5\\
588	0.485\\
589	0.469\\
590	0.484\\
591	0.471\\
592	0.484\\
593	0.477\\
594	0.49\\
595	0.472\\
596	0.49\\
597	0.481\\
598	0.49\\
599	0.48\\
600	0.497\\
601	0.48\\
602	0.499\\
603	0.485\\
604	0.506\\
605	0.494\\
606	0.485\\
607	0.503\\
608	0.491\\
609	0.507\\
610	0.494\\
611	0.509\\
612	0.489\\
613	0.506\\
614	0.491\\
615	0.511\\
616	0.495\\
617	0.512\\
618	0.493\\
619	0.512\\
620	0.496\\
621	0.513\\
622	0.495\\
623	0.481\\
624	0.503\\
625	0.482\\
626	0.506\\
627	0.49\\
628	0.51\\
629	0.492\\
630	0.509\\
631	0.495\\
632	0.508\\
633	0.496\\
634	0.518\\
635	0.502\\
636	0.52\\
637	0.508\\
638	0.525\\
639	0.513\\
640	0.499\\
641	0.515\\
642	0.499\\
643	0.521\\
644	0.502\\
645	0.522\\
646	0.502\\
647	0.521\\
648	0.508\\
649	0.52\\
650	0.504\\
651	0.525\\
652	0.496\\
653	0.523\\
654	0.506\\
655	0.533\\
656	0.513\\
657	0.529\\
658	0.509\\
659	0.493\\
660	0.517\\
661	0.505\\
662	0.524\\
663	0.513\\
664	0.536\\
665	0.52\\
666	0.544\\
667	0.528\\
668	0.546\\
669	0.53\\
670	0.546\\
671	0.534\\
672	0.546\\
673	0.533\\
674	0.549\\
675	0.539\\
676	0.527\\
677	0.546\\
678	0.533\\
679	0.556\\
680	0.539\\
681	0.555\\
682	0.533\\
683	0.555\\
684	0.536\\
685	0.549\\
686	0.535\\
687	0.562\\
688	0.548\\
689	0.57\\
690	0.549\\
691	0.572\\
692	0.552\\
693	0.573\\
694	0.555\\
695	0.541\\
696	0.56\\
697	0.55\\
698	0.566\\
699	0.553\\
700	0.57\\
701	0.555\\
702	0.571\\
703	0.545\\
704	0.569\\
705	0.549\\
706	0.572\\
707	0.563\\
708	0.578\\
709	0.565\\
710	0.582\\
711	0.571\\
712	0.558\\
713	0.573\\
714	0.559\\
715	0.577\\
716	0.558\\
717	0.576\\
718	0.562\\
719	0.58\\
720	0.562\\
721	0.581\\
722	0.57\\
723	0.583\\
724	0.569\\
725	0.583\\
726	0.568\\
727	0.588\\
728	0.578\\
729	0.594\\
730	0.58\\
731	0.562\\
732	0.58\\
733	0.57\\
734	0.584\\
735	0.572\\
736	0.586\\
737	0.572\\
738	0.591\\
739	0.58\\
740	0.595\\
741	0.578\\
742	0.595\\
743	0.589\\
744	0.606\\
745	0.59\\
746	0.606\\
747	0.592\\
748	0.608\\
749	0.597\\
750	0.581\\
751	0.597\\
752	0.581\\
753	0.601\\
754	0.594\\
755	0.608\\
756	0.595\\
757	0.608\\
758	0.596\\
759	0.607\\
760	0.595\\
761	0.608\\
762	0.596\\
763	0.61\\
764	0.596\\
765	0.606\\
766	0.596\\
767	0.611\\
768	0.599\\
769	0.587\\
770	0.605\\
771	0.594\\
772	0.616\\
773	0.599\\
774	0.62\\
775	0.606\\
776	0.618\\
777	0.606\\
778	0.62\\
779	0.611\\
780	0.625\\
781	0.609\\
782	0.627\\
783	0.611\\
784	0.631\\
785	0.613\\
786	0.635\\
787	0.619\\
788	0.609\\
789	0.623\\
790	0.611\\
791	0.625\\
792	0.613\\
793	0.624\\
794	0.611\\
795	0.625\\
796	0.613\\
797	0.628\\
798	0.616\\
799	0.629\\
800	0.612\\
801	0.63\\
802	0.618\\
803	0.632\\
804	0.62\\
805	0.636\\
806	0.622\\
807	0.639\\
808	0.624\\
809	0.613\\
810	0.632\\
811	0.615\\
812	0.629\\
813	0.614\\
814	0.631\\
815	0.618\\
816	0.634\\
817	0.62\\
818	0.633\\
819	0.621\\
820	0.635\\
821	0.622\\
822	0.642\\
823	0.628\\
824	0.647\\
825	0.63\\
826	0.647\\
827	0.636\\
828	0.62\\
829	0.641\\
830	0.632\\
831	0.647\\
832	0.635\\
833	0.655\\
834	0.643\\
835	0.651\\
836	0.642\\
837	0.654\\
838	0.641\\
839	0.654\\
840	0.64\\
841	0.655\\
842	0.644\\
843	0.65\\
844	0.64\\
845	0.649\\
846	0.637\\
847	0.656\\
848	0.643\\
849	0.633\\
850	0.646\\
851	0.64\\
852	0.656\\
853	0.64\\
854	0.655\\
855	0.646\\
856	0.659\\
857	0.648\\
858	0.659\\
859	0.654\\
860	0.665\\
861	0.656\\
862	0.668\\
863	0.66\\
864	0.672\\
865	0.662\\
866	0.673\\
867	0.665\\
868	0.654\\
869	0.669\\
870	0.657\\
871	0.669\\
872	0.66\\
873	0.67\\
874	0.661\\
875	0.671\\
876	0.66\\
877	0.671\\
878	0.663\\
879	0.675\\
880	0.665\\
881	0.681\\
882	0.665\\
883	0.676\\
884	0.671\\
885	0.684\\
886	0.672\\
887	0.685\\
888	0.672\\
889	0.658\\
890	0.667\\
891	0.656\\
892	0.666\\
893	0.658\\
894	0.672\\
895	0.659\\
896	0.67\\
897	0.659\\
898	0.676\\
899	0.663\\
900	0.676\\
901	0.668\\
902	0.679\\
903	0.666\\
904	0.68\\
905	0.67\\
906	0.682\\
907	0.669\\
908	0.685\\
909	0.678\\
910	0.666\\
911	0.679\\
912	0.668\\
913	0.678\\
914	0.671\\
915	0.684\\
916	0.673\\
917	0.679\\
918	0.671\\
919	0.68\\
920	0.673\\
921	0.681\\
922	0.672\\
923	0.686\\
924	0.677\\
925	0.686\\
926	0.683\\
927	0.692\\
928	0.688\\
929	0.678\\
930	0.684\\
931	0.679\\
932	0.686\\
933	0.68\\
934	0.693\\
935	0.678\\
936	0.686\\
937	0.681\\
938	0.695\\
939	0.684\\
940	0.699\\
941	0.689\\
942	0.7\\
943	0.689\\
944	0.699\\
945	0.69\\
946	0.702\\
947	0.693\\
948	0.708\\
949	0.696\\
950	0.689\\
951	0.701\\
952	0.69\\
953	0.704\\
954	0.69\\
955	0.704\\
956	0.697\\
957	0.711\\
958	0.705\\
959	0.714\\
960	0.704\\
961	0.712\\
962	0.698\\
963	0.713\\
964	0.702\\
965	0.714\\
966	0.702\\
967	0.718\\
968	0.699\\
969	0.717\\
970	0.703\\
971	0.69\\
972	0.705\\
973	0.695\\
974	0.71\\
975	0.699\\
976	0.715\\
977	0.702\\
978	0.722\\
979	0.705\\
980	0.726\\
981	0.712\\
982	0.725\\
983	0.713\\
984	0.724\\
985	0.717\\
986	0.73\\
987	0.722\\
988	0.736\\
989	0.719\\
990	0.733\\
991	0.723\\
992	0.735\\
993	0.723\\
994	0.71\\
995	0.724\\
996	0.71\\
997	0.725\\
998	0.713\\
999	0.727\\
1000	0.719\\
};
\addlegendentry{Bayes test}

\end{axis}
\end{tikzpicture}%
      \caption{Comparison of the rejection probability for the null and the Bayesian test when $\model_1$ is true.}
    \end{figure}
    \only<article>{However, both methods are able to reject the null hypothesis more often when it is false, as long as we have more data.}
  }
\end{frame}

\begin{frame}
  \frametitle{Concentration inequalities and confidence intervals}
  \only<article>{
    There are a number of inequalities from probability theory that allow us to construct high-probability confidence intervals. The most well-known of those is Hoeffding's inequality \index{Hoeffding inequality|textbf}, the simplest version of which is the following:
    \begin{lemma}[Hoeffding's inequality]
      Let $x_1, \ldots, x_n$ be a series of random variables, $x_i \in [0, 1]$. Then it holds that, for the empirical mean:
      \[
        \mu_n \defn \frac{1}{n} sum_{t=1}^n x_t
      \]
      \begin{equation}
        \Pr(\mu_n \geq \E \mu_n+ \epsilon) \leq e^{-2n\epsilon^2}.
        \label{eq:hoeffding}
      \end{equation}
    \end{lemma}
    When $x_t$ are i.i.d, $\E \mu_n = \E x_t$. This allows us to construct an interval of size $\epsilon$ around the true mean. This can generalise to a two-sided interval:
    \[
      \Pr(|\mu_n - \E \mu_n| \geq \epsilon) \leq 2e^{-2n\epsilon^2}.
    \]
    We can also rewrite the equation to say that, with probability at least $1 - \delta$
    \[
      |\mu_n - \E \mu_n| \leq \sqrt{\frac{\ln 2/\delta}{2n}}
    \]
  }
\end{frame}
\begin{frame}
  \frametitle{Further reading}
  \begin{block}{Points of significance (Nature Methods)}
    \begin{itemize}
    \item Importance of being uncertain \url{https://www.nature.com/articles/nmeth.2613}
    \item Error bars \url{https://www.nature.com/articles/nmeth.2659}
    \item P values and the search for significance \url{https://www.nature.com/articles/nmeth.4120}
    \item Bayes' theorem \url{https://www.nature.com/articles/nmeth.3335}
    \item Sampling distributions and the bootstrap \url{https://www.nature.com/articles/nmeth.3414}
    \end{itemize}
  \end{block}
\end{frame}



%%% Local Variables:
%%% mode: latex
%%% TeX-master: "notes"
%%% End:
 % Missing
}
\section{Formalising classification problems}
\only<article>{
  One of the simplest decision problems is classification. At the simplest level, this is the problem of observing some data point $x_t \in \CX$ and making a decision about what class $\CY$ it belongs to. Typically, a fixed classifier is defined as a decision rule $\pi(a | x)$ making decisions $a \in \CA$, where the decision space includes the class labels, so that if we observe some point $x_t$ and choose $a_t = 1$, we essentially declare that $y_t = 1$.

  Typically, we wish to have a classification policy that minimises classification error.
}
\begin{frame}
  \frametitle{Deciding a class given a model}
  \only<article>{In the simplest classification problem, we observe some features $x_t$ and want to make a guess $\decision_t$ about the true class label $y_t$. Assuming we have some probabilistic model $P_\model(y_t \mid x_t)$, we want to define a decision rule $\pol(\decision_t \mid x_t)$ that is optimal, in the sense that it maximises expected utility for $P_\model$.}
  \begin{itemize}
  \item Features $x_t \in \CX$.
  \item Label $y_t \in \CY$.
  \item Decisions $\decision_t \in \CA$.
  \item Decision rule $\pol(\decision_t \mid x_t)$ assigns probabilities to actions.
  \end{itemize}
  
  \begin{block}{Standard classification problem}
    \only<article>{In the simplest case, the set of decisions we make are the same as the set of classes}
    \[
    \CA = \CY, \qquad
    U(\decision, y) = \ind{\decision = y}
    \]
  \end{block}

  \begin{exercise}
    If we have a model $P_\model(y_t \mid x_t)$, and a suitable $U$, what is the optimal decision to make?
  \end{exercise}
  \only<presentation>{
    \uncover<2->{
      \[
      \decision_t \in \argmax_{\decision \in \Decision} \sum_y P_\model(y_t = y \mid x_t) \util(\decision, y)
      \]
    }
    \uncover<3>{
      For standard classification,
      \[
      \decision_t \in \argmax_{\decision \in \Decision} P_\model(y_t = \decision \mid x_t)
      \]
    }
  }
\end{frame}


\begin{frame}
  \frametitle{Deciding the class given a model family}
  \begin{itemize}
  \item Training data $\Training = \cset{(x_i, y_i)}{i=1, \ldots, \ndata}$
  \item Models $\cset{P_\model}{\model \in \Model}$.
  \item Prior $\bel$ on $\Model$.
  \end{itemize}
  \only<article>{Similarly to our example with the meteorological stations, we can define a posterior distribution over models.}
  \begin{block}{Posterior over classification models}
    \[
    \bel(\model \mid \Training) = \frac{P_\model(y_1, \ldots, y_\ndata \mid
      x_1, \ldots, x_\ndata) \bel(\model)} {\sum_{\model' \in \Model}
      P_{\model'}(y_1, \ldots, y_\ndata \mid x_1, \ldots, x_\ndata)
      \bel(\model')}
    \]
    \only<article>{
      This posterior form can be seen as weighing each model according to how well they can predict the class labels. It is a correct form as long as, for every pair of models $\model, \model'$ we have that $P_\model(x_1, \ldots, x_\ndata) = P_{\model'}(x_1, \ldots, x_\ndata)$. This assumption can be easily satisfied without specifying a particular model for the $x$.}
    \only<2>{
      If not dealing with time-series data, we assume independence between $x_t$:
      \[
      P_\model(y_1, \ldots, y_\ndata \mid  x_1, \ldots, x_\ndata)
      = \prod_{i=1}^T P_\model(y_i \mid x_i)
      \]
    }
  \end{block}
  \uncover<3->{
    \begin{block}{The \alert{Bayes rule} for maximising $\E_\bel(\util \mid a, x_t, \Training)$}
      The decision rule simply chooses the action:
      \begin{align}
        \decision_t &\in
                      \argmax_{\decision \in \Decision}
                      \sum_{y}  \alert<4>{\sum_{\model \in
                      \Model}  P_\model(y_t = y \mid x_t) \bel(\model \mid
                      \Training)} 
                      \util(\decision, y)
                      \only<5>{
        \\ &=
             \argmax_{\decision \in \Decision}
             \sum_{y} \Pr_{\bel \mid \Training}(y_t \mid x_t) 
             \util(\decision, y)
             }
      \end{align}
    \end{block}
  }
  \uncover<4->{
    We can rewrite this by calculating the posterior marginal marginal label probability
    \[
    \Pr_{\bel \mid \Training}(y_t \mid x_t) \defn
    \Pr_{\bel}(y_t \mid x_t, \Training) = 
    \sum_{\model \in \Model} P_\model(y_t \mid x_t) \bel(\model \mid \Training).
    \]
  }

\end{frame}

\begin{frame}
  \frametitle{Approximating the model}
  \begin{block}{Full Bayesian approach for infinite $\Model$}
    Here $\bel$ can be a probability density function and 
    \[
    \bel(\model \mid \Training)  = P_\model(\Training)  \bel(\model)  / \Pr_\bel(\Training),
    \qquad
    \Pr_\bel(\Training) = \int_{\Model} P_\model(\Training)  \bel(\model)  \dd,
    \]
    can be hard to calculate.
  \end{block}
  \onslide<2->{
    \begin{block}{Maximum a posteriori model}
      \index{MAP inference}
      \index{Maximum a posteriori|see MAP inference}
      We only choose a single model through the following optimisation:
      \[
      \MAP(\bel, \Training) 
      \only<2>{
        = \argmax_{\model \in \Model} P_\model(\Training)  \bel(\model) 
      }
      \only<3>{
        = \argmax_{\model \in \Model}
        \overbrace{\ln P_\model(\Training)}^{\textrm{goodness of fit}}  + \underbrace{\ln \bel(\model)}_{\textrm{regulariser}}.
      }
      \]
      \only<article>{You can think of the goodness of fit as how well the model fits the training data, while the regulariser term simply weighs models according to some criterion. Typically, lower weights are used for more complex models.}
    \end{block}
  }
\end{frame}



\begin{frame}
  \frametitle{Learning outcomes}
  \begin{block}{Understanding}
    \begin{itemize}
    \item Preferences, utilities and the expected utility principle.
    \item Hypothesis testing and classification as decision problems.
    \item How to interpret $p$-values Bayesian tests.
    \item The MAP approximation to full Bayesian inference.
    \end{itemize}
  \end{block}
  
  \begin{block}{Skills}
    \begin{itemize}
    \item Being able to implement an optimal decision rule for a given utility and probability.
    \item Being able to construct a simple null hypothesis test.
    \end{itemize}
  \end{block}

  \begin{block}{Reflection}
    \begin{itemize}
    \item When would expected utility maximisation not be a good idea?
    \item What does a $p$ value represent when you see it in a paper?
    \item Can we prevent high false discovery rates when using $p$ values?
    \item When is the MAP approximation good?
    \end{itemize}
  \end{block}
  
\end{frame}



%%% Local Variables:
%%% mode: latex
%%% TeX-master: "notes"
%%% End:

 % decision hierarchies
\section{Beliefs and probabilities}
\only<presentation>{
  \begin{frame}
    \tableofcontents[ 
    currentsection, 
    hideothersubsections, 
    sectionstyle=show/shaded
    ] 
  \end{frame}
}


\only<article>{Probability can be used to describe purely chance events, as in for example quantum physics. However, it is mostly used to describe uncertain events, such as the outcome of a dice roll or a coin flip, which only appear random. In fact, one can take it even further than that, and use it to model subjective uncertainty about any arbitrary event. Although probabilities are not the only way in which we can quantify uncertainty, it is a simple enough model, and with a rich enough history in mathematics, statistics, computer science and engineering that it is the most useful.}
\begin{frame}
  \frametitle{Uncertainty}
  \begin{itemize}
  \item We cannot perfectly predict the future.
  \item We cannot know for sure what happened in the past.
  \item How can we quantify this uncertainty?
  \item Probabilities!
  \end{itemize}
  \begin{block}{Axioms of probability}
    For any probability measure\footnote{$\Sigma$ is the set of possible events, with $A \in \Sigma$ always $A \subset \Omega$. Technically $\Sigma$ is a $\sigma$-algebra} $P$ on $(\Omega, \Sigma)$,
    \begin{enumerate}
    \item<2-> The probability of the certain event is $P(\Omega) = 1$
    \item<3->The probability of the impossible event is
      $P(\emptyset) = 0$
    \item<4->The probability of any event $A \in \Sigma$ is $0 \leq P(A) \leq 1$.
    \item<5-> If $A, B$ are disjoint, i.e. $A \cap B = \emptyset$, meaning
      that they cannot happen at the same time, then
      \[
      P(A \cup B) = P(A) + P(B)
      \]
    \end{enumerate}
  \end{block}
\end{frame}

\begin{frame}
  \only<article>{ Sometimes we would like to calculate the probability
    of some event $A$ happening given that we know that some other
    event $B$ has happened. For this we need to first define the idea
    of conditional probability.  }
  \begin{definition}[Conditional probability]
    The probability of $A$ happening if we know that $B$ has happened
    is defined to be:
    \[
    P(A \mid B) \defn \frac{P(A \cap B) }{P(B)}.
  \]
\end{definition}
\only<1>{
  Conditional probabilities obey the same rules as probabilities. }
  \only<article>{
    Here, the probability measure of any event $A$ given $B$ is defined to be the probability of the intersection of of the events divided by the second event.
    We can rewrite this definition as follows, by using the definition for $P(B \mid A)$}
  \begin{block}{Bayes's theorem}
    For $P(A_1 \cup A_2)  = 1$, $A_1 \cap A_2 = \emptyset$,
    \[
      P(A_i \mid B)
      \uncover<2->{= \frac{P(B \mid A_i) P(A_i)}{P(B)}}
      \uncover<3->{= \frac{P(B \mid A_i) P(A_i)}{P(B \mid A_1) P(A_1) + P(B \mid A_2) P(A_2)}}
    \]
  \end{block}
  \uncover<4->{
  \begin{example}[probability of rain]
    What is the probability of rain given a forecast $x_1$ or $x_2$?
    \begin{columns}
      \begin{column}{0.33\textwidth}
        \begin{table}[H]
          \centering
          \begin{tabular}{c|c}
            $\outcome_1$: rain & $P(\outcome_1) = 80\%$ \\
            $\outcome_2$: dry & $P(\outcome_2) = 20\%$
          \end{tabular}
          \caption{Prior probability of rain tomorrow}
        \end{table}
      \end{column}
      
      \begin{column}{0.33\textwidth}
        \uncover<5->{
        \begin{table}[H]
          \centering
          \begin{tabular}{c|c}
            $x_1$: rain & $P(x_1 \mid \outcome_1) = 90\%$ \\
            $x_2$: dry & $P(x_2 \mid \outcome_2) = 50\%$
          \end{tabular}
          \caption{Probability the forecast is correct}
        \end{table}
        }
      \end{column}
      \uncover<6->{
      \begin{column}{0.33\textwidth}
        \begin{table}[H]
          \centering
          \begin{tabular}{c}
            $P(\outcome_1 \mid x_1) = 87.8\%$ \\
            $P(\outcome_1 \mid x_2) = 44.4\%$
          \end{tabular}
          \caption{Probability that it will rain given the forecast}
        \end{table}
        }
      \end{column}
    \end{columns}
  \end{example}
  }
\end{frame}


\begin{frame}
  \frametitle{Classification in terms of conditional probabilities}
  \only<presentation>{
    \begin{itemize}
    \item Features $x_t \in \CX$.
    \item Class label $y_t \in \CY$.
    \item Probability model $P_\model(x_t \mid y_t)$.
    \item Prior class probability $P_\model(y_t = c)$.
    \end{itemize}
  }
  \only<article>{
    Conditional probability naturally appears in classification problems. Given a new example vector of data $x_t \in \CX$, we would like to calculate the probability of different classes $c \in \CY$ given the data, $P_\model(y_t = c \mid x_t)$.  
If we somehow obtained the distribution of data $P_\model(x_t \mid y_t)$ for each possible class, as well as the prior class probability $P_\model(y_t = c)$, 
from Bayes's theorem, we see that we can obtain the probability of the class:
  }
  \[
  P_\model(y_t = c \mid x_t) = \frac{P_\model(x_t \mid y_t = c) P_\model(y_t = c)}{\sum_{c' \in \CY} P_\model(x_t \mid y_t = c') P_\model(y_t = c')}
  \]
  \only<article>{
    for any class $c$. This directly gives us a method for classifying new data, as long as we have a way to obtain $P_\model(x_t \mid y_t)$ and $P_\model(y_t)$.
  }
  \only<1>{
        \begin{tikzpicture}
          \node[RV] at (0,0) (x) {$y_t$};
          \node[RV] at (0,2) (y) {$x_t$};
          \node[RV] at (1,1) (m) {$\model$};
          \draw[->] (x) to (y);
          \draw[->] (m) to (x);
          \draw[->] (m) to (y);
        \end{tikzpicture}
      }
  \uncover<2->{
    \begin{example}[Normal distribution]
 \only<2,4>{\includegraphics[width=0.5\textwidth]{../figures/equal-variance}Equal prior and variance}
      \only<3>{\includegraphics[width=0.5\textwidth]{../figures/unequal-variance}Unequal variance}
\only<5>{\includegraphics[width=0.5\textwidth]{../figures/unequal-prior}Unequal prior}
    \end{example}
  }
  \uncover<5>{
    \alert{But how can we get a probability model in the first place?}
  }
\end{frame}


  \begin{frame}
    \frametitle{Subjective probability}
    \only<article>{While probabilities apply to truly random events, they are also useful for representing subjective uncertainty. In this course, we will use a special symbol for subjective probability, $\bel$.}
    \begin{block}{Subjective probability measure $\bel$}
      \begin{itemize}
      \item If we think event $A$ is more likely than $B$, then $\bel(A) > \bel(B)$.
      \item Usual rules of probability apply:
        \begin{enumerate}
        \item $\bel(A) \in [0,1]$.
        \item $\bel(\emptyset) = 0$.
        \item If $A \cap B = \emptyset$, then $\bel(A \cup B) = \bel(A) + \bel(B)$.
        \end{enumerate}
      \end{itemize}
    \end{block}
  \end{frame}


  \begin{frame}
    \frametitle{Bayesian inference illustration}
    \begin{columns}
      \begin{column}{0.7\textwidth}
        \begin{block}{Use a subjective belief $\bel(\model)$ on $\Model$}
          \begin{itemize}
          \item<1-> \alert{Prior} belief $\bel(\model)$ represents our initial uncertainty.
          \item<2-> We \alert{observe history} $h$.
          \item<3->Each possible $\model$ assigns a \alert{probability} $P_\model(h)$ to $h$.
          \item<4-> We can use this to \alert{update} our belief via Bayes' theorem to obtain the \alert{posterior} belief:
            \[
            \bel(\model \mid h) \propto P_\model(h) \bel(\model)
            \tag{conclusion = evidence $\times$ prior}
            \]
          \end{itemize}
        \end{block}
      \end{column}
      \begin{column}{0.3\textwidth}
        \centering
\uncover<1->{\includegraphics[width=0.5\fwidth]{../figures/rl_worlds}
          \\
          prior
        }
        \\
        \uncover<2->{\includegraphics[width=0.5\fwidth]{../figures/rl_observations}
          \\
          evidence
        }
        \\
        \uncover<4->{\includegraphics[width=0.5\fwidth]{../figures/rl_worlds2}
          \\ 
          conclusion
        }
      \end{column}
    \end{columns}
  \end{frame}




  \subsection{Probability and Bayesian inference}
  \only<article>{One of the most important methods in machine learning
    and statistics is that of Bayesian inference.  This is the most
    fundamental method of drawing conclusions from data and explicit
    prior assumptions. In Bayesian inference, prior assumptions are
    represented as a probabilities on a space of hypotheses. Each
    hypothesis is seen as a probabilistic model of all possible data
    that we can see.}

  \only<article>{Frequently, we want to draw conclusions from data. However, the conclusions are never solely inferred from data, but also depend on prior assumptions about reality.}




  \begin{frame}
    \frametitle{Some examples}

    \begin{example}
      John claims to be a medium. He throws a coin $n$ times and predicts its value always correctly. Should we believe that he is a medium?
      \begin{itemize}
      \item $\model_1$: John is a medium.
      \item $\model_0$: John is not a medium.
      \end{itemize}
    \end{example}
    The answer depends on what we \alert{expect} a medium to be able to do, and how likely we thought he'd be a medium in the first place.

    \only<article>{
    \begin{example}
      Traces of DNA are found at a murder scene. We perform a DNA test against a database of $10^4$ citizens registered to be living in the area. We know that the probability of a false positive (that is, the test finding a match by mistake) is $10^{-6}$. If there is a match in the database, does that mean that the citizen was at the scene of the crime?
    \end{example}
    }
  \end{frame}





  \begin{frame}
    \frametitle{Bayesian inference}
    \only<article>{
      Now let us apply this idea to our specific problem. We already have the probability of the observation for each model, but we just need to define a \emph{prior probability} for each model. Since this is usually completely subjective, we give it another symbol.
    }
    \only<article>{
      \begin{block}{Prior probability}
        The prior probability $\bel$ on a set of models $\Model$ specifies our subjective belief $\bel(\model)$ that each model is true.\footnote{More generally $\bel$ is a probability measure.}
      \end{block}
    }
    \only<article>{
      This allows us to calculate the probability of John being a medium, given the data:
      \[
      \bel(\model_1 \mid \bx) = \frac{\Pr(\bx \mid \model_1) \bel(\model_1)}{\Pr_\bel(\bx)},
      \]
      where
      \[
      \Pr_\bel(\bx) \defn \Pr(\bx \mid \model_1) \bel(\model_1) + \Pr(\bx \mid \model_0) \bel(\model_0).
      \]
      The only thing left to specify is $\bel(\model_1)$, the probability that John is a medium before seeing the data. This is our subjective prior belief that mediums exist and that John is one of them.
      More generally, we can think of Bayesian inference as follows: }
    \begin{itemize}
    \item<1-> \only<article>{We start with a set of } mutually exclusive models $\Model = \{\model_1, \ldots, \model_k\}$.
    \item<2->\only<article>{Each model $\model$ is represented by a specific probabilistic model for any possible data $x$, that is}
      \only<presentation>{Probability model for any data $x$:} $P_\model(x) \equiv \Pr(x \mid \model)$.
    \item<3-> For each model, we have a prior probability $\bel(\model)$ that it is correct.
    \item<4-> \only<article>{After observing the data, we can calculate a posterior probability that the model is correct:}
      \only<presentation>{Posterior probability}
      \[
      \bel(\model \mid x) = \frac{\Pr(x \mid \model) \bel(\model)}{\sum_{\model' \in \Model} \Pr(x \mid \model') \bel(\model')}
      = \frac{P_\model(x) \bel(\model)}{\sum_{\model' \in \Model} P_{\model'} (x) \bel(\model')}.
      \]
    \end{itemize}
    \only<5->{
      \begin{block}{Interpretation}
        \begin{itemize}
        \item $\CM$: Set of all possible models that could describe the data.
        \item $P_\model(x)$: Probability of $x$ under model $\model$.
        \item Alternative notation $\Pr(x \mid \model)$: Probability of $x$ given that model $\model$ is correct.
        \item $\bel(\model)$: Our belief, before seeing the data, that $\model$ is correct.
        \item $\bel(\model \mid x)$: Our belief, aftering seeing the data, that $\model$ is correct.
      \end{itemize}
    \end{block}
    \only<article>{It must be emphasized that $P_\model(x) = \Pr(x \mid \model)$ as they are simply two different notations for the same thing. In words the first can be seen as the probability that model $\model$ assigns to data $x$, while the second as the probability of $x$ if $\model$ is the true model.}
  }
    \only<article>{
      Combining the prior belief with evidence is key in this procedure. Our posterior belief can then be used as a new prior belief when we get more evidence.}
  \end{frame}
\begin{frame}
\begin{exercise}[Continued example for medium]
    \only<article>{ Now let us apply this idea to our specific
      problem. We first make an independence assumption. In particular, we can assume that success and failure comes from a Bernoulli distribution with a parameter depending on the model.}
    \begin{align}
    P_{\model} (x) &= \prod_{t=1}^n P_{\model} (x_t).
\tag{independence property}
    \end{align}
    \only<article>{We first need to specify how well a medium could predict. Let's assume that a true medium would be able to predict perfectly, and that a non-medium would only predict randomly. This leads to the following models:}
    \begin{align}
      P_{\model_1}(x_t = 1) &= 1, &P_{\model_1}(x_t = 0) &= 0.
                                                           \tag{true medium model}
                                                           \\
      P_{\model_0}(x_t = 1) &= 1/2, &P_{\model_0}(x_t = 0) &= 1/2.
                                                             \tag{non-medium model}
    \end{align}
    \only<article>{
      The only thing left to specify is $\bel(\model_1)$, the probability
      that John is a medium before seeing the data. This is our
      subjective prior belief that mediums exist and that John is one of
      them.}
    \uncover<3->{
      \begin{align}
        \bel(\model_0) &= 1/2,   &  \bel(\model_1) &= 1/2.
                                                     \tag{prior belief}
      \end{align}
    }
    \only<article>{Combining the prior belief with evidence is key in this
      procedure. Our posterior belief can then be used as a new prior
      belief when we get more evidence.  }
    \uncover<4>{
      \begin{align}
      \bel(\model_1 \mid x) & = \frac{P_{\model_1}(x)
        \bel(\model_1)}{\Pr_\bel(x)} \tag{posterior belief}
        \\
      \Pr_\bel(x) &\defn P_{\model_1}(x) \bel(\model_1) + P_{\model_0}(x) \bel(\model_0).
\tag{marginal distribution}
      \end{align}
    }
    Throw a coin 4 times, and have a classmate make a prediction. What your belief that your classmate is a medium? Is the prior you used reasonable?
  \end{exercise}
  \end{frame}


\begin{frame}
  \frametitle{Sequential update of beliefs}
    \only<article>{Assume you have $n$ meteorologists. At each day $t$, each meteorologist $i$ gives a probability $p_{t,\model_i}\defn P_{\model_i}(x_t = \textrm{rain})$ for rain. Consider the case of there being three meteorologists, and each one making the following prediction for the coming week. Start with a uniform prior $\bel(\model) = 1/3$ for each model.}
    {
      \begin{table}[h]
        \begin{tabular}{c|l|l|l|l|l|l|l}
          &M&T&W&T&F&S&S\\
          \hline
          CNN & 0.5 & 0.6 & 0.7 & 0.9 & 0.5 & 0.3 & 0.1\\
          SMHI & 0.3 & 0.7 & 0.8 & 0.9 & 0.5 & 0.2 & 0.1\\
          YR & 0.6 & 0.9 & 0.8 & 0.5 & 0.4 & 0.1 & 0.1\\
          \hline
          Rain? & Y & Y & Y & N & Y & N & N
        \end{tabular}
        \caption{Predictions by three different entities for the probability of rain on a particular day, along with whether or not it actually rained.}
        \label{tab:meteorologists}
      \end{table}
    }
  \begin{exercise}
    \begin{itemize}
    \item $n$ meteorological stations $\cset{\mdp_i}{i=1, \ldots,n}$
    \item The $i$-th station predicts rain $P_{\mdp_i}(x_t \mid x_1, \ldots, x_{t-1})$.
    \item Let $\bel_t(\mdp)$ be our belief at time $t$.
      Derive the next-step belief
      $\bel_{t+1}(\mdp) \defn  \bel_t(\mdp | y_{t})$ in terms of the current belief $\bel_t$.
    \item Write a python function that computes this posterior
    \end{itemize}
  \end{exercise}
  \uncover<2->{
    \[
      \bel_{t+1}(\mdp)
      \defn
      \bel_t(\mdp | x_{t})
      =
      \frac{P_\mdp(x_t \mid x_1, \ldots, x_{t-1}) \bel_t(\mdp)}
      {\sum_{\mdp'} P_{\mdp'}(x_t \mid x_1, \ldots, x_{t-1}) \bel_t(\mdp')}
    \]
  }
\end{frame}



\begin{frame}[label=beta-example]
  \frametitle{Bayesian inference for Bernoulli distributions}
  \only<1>{
    \begin{block}{Estimating a coin's bias}
      A fair coin comes heads $50\%$ of the time. 
      We want to test an unknown coin, which we think may not be completely fair. 
    \end{block}
  }
  \only<1,2>{
    \begin{figure}[h]
      \centering
      \includegraphics[width=\textwidth]{../figures/beta-prior}
      \caption{Prior belief $\bel$ about the coin bias $\theta$.}
    \end{figure}
  }
  \only<2>{
    For a sequence of throws $x_t \in \{0,1\}$,
    \[
    P_\theta(x) \propto \prod_t \theta^{x_t} (1 - \theta)^{1 - x_t}
    = \theta^{\textrm{\#Heads}} (1 - \theta)^{\textrm{\#Tails}}
    \]
  }
  \only<3>{
    \begin{figure}[h]
      \centering
      \includegraphics[width=\textwidth]{../figures/beta-likelihood}
      \caption{Prior belief $\bel$ about the coin bias $\theta$ and likelihood of $\theta$ for the data.}
    \end{figure}
    Say we throw the coin 100 times and obtain 70 heads. Then we plot the \alert{likelihood} $P_\theta(x)$ of different models.
  }
  \only<4>{
    \begin{figure}[h]
      \centering
      \includegraphics[width=\textwidth]{../figures/beta-posterior}
      \caption{Prior belief $\bel(\theta)$ about the coin bias $\theta$, likelihood of $\theta$ for the data, and posterior belief $\bel(\theta \mid x)$}
    \end{figure}
    From these, we calculate a \alert{posterior} distribution over the correct models. This represents our conclusion given our prior and the data.
  }
  \only<article>{If the prior distribution is described by the so-called Beta density
    \[
    f(\theta \mid \alpha, \beta) \propto \theta^{\alpha -1} (1 - \theta)^{\beta -1}
    \]
    where $\alpha, \beta$ describe the shape of the Beta distribution.
  }
\end{frame}



\begin{frame}
  \frametitle{Learning outcomes}
  \begin{block}{Understanding}
    \begin{itemize}
    \item The axioms of probability, marginals and conditional distributions.
    \item The philosophical underpinnings of Bayesianism.
    \item The simple conjugate model for Bernoulli distributions.
    \end{itemize}
  \end{block}
  
  \begin{block}{Skills}
    \begin{itemize}
    \item Be able to calculate with probabilities using the marginal and conditional definitions and Bayes rule.
    \item Being able to implement a simple Bayesian inference algorithm in Python.
    \end{itemize}
  \end{block}

  \begin{block}{Reflection}
    \begin{itemize}
    \item How useful is the Bayesian representation of uncertainty?
    \item How restrictive is the need to select a prior distribution?
    \item Can you think of another way to explicitly represent uncertainty in a way that can incorporate new evidence?
    \end{itemize}
  \end{block}
  
\end{frame}

  %%% Local Variables:
  %%% mode: latex
  %%% TeX-engine: xetex
  %%% TeX-master: "notes.tex"
  %%% End:
 % Bayesian inference
\section{Classification with stochastic gradient descent}
\only<presentation>{
  \begin{frame}
    \tableofcontents[ 
    currentsection, 
    hideothersubsections, 
    sectionstyle=show/shaded
    ] 
  \end{frame}
}

\begin{frame}
  \frametitle{Linear classifiers.}
  \only<article>{Finding the optimal policy for our belief $\bel$ is not normally very difficult. However, it requires that we maintain the complete distribution $\bel$ and that we also nder some probability distribution $P$. In simple decision problems, e.g. where $a$ is finite, it is possible to do this calculation on-the-fly. However, the policies that we wish to find might be much simpler than the Bayes-optimal policy. For example, we might consider linear classifiers.}

  \begin{definition}{Linear classifier}
    \only<article>{A linear classifier is parametrised by the matrix $\Param = [\param_1 \cdots \param_{|\CY|}]$}
    \[
      \pol_\Param(a \mid x) = e^{\param_a^\top x} / \sum_{a'} e^{\param_{a'}^\top x}
    \]
  \end{definition}
  \only<article>{Even though the classifier has a linear structure, the final non-linearity at the end is there to ensure that it defines a proper probability distribution over decisions.}

  \begin{block}{The $\model$-optimal classifier}
    \only<article>{Since the performance measure is simply an expectation, it is intuitive to directly optimise the decision rule with respect to an approximation of the expectation}
    \[
      \max_\Param f(\pol_\Param, \model)  \approx \max_\Param \sum_{t=1}^T  \pol(a_t = y_t \mid x_t ) P_\model(y_t \mid x_t), \qquad (x_t, y_t) \sim P_\model.
    \]
    \only<article>{In practice, this is the empirical expectation on the training set $\cset{(x_t, y_t)}{t=1, \ldots, T}$. However, when the amount of data is insufficient, this expectation may be far from reality, and so our classification rule might be far from optimal.}
  \end{block}

  \begin{block}{The Bayes-optimal classifier}
    \only<article>{An alternative idea is to use our uncertainty to create a distribution over models, and then use this distribution to obtain a single classifier that does take the uncertainty into account.}
    \[
      \max_\Param f(\pol_\Param, \bel)
      \approx
      \max_\Param N^{-1} \sum_{n=1}^N
      \pol(a_t = y_n \mid x_t = x_n),
      \qquad
      (x_n, y_n) \sim P_{\model_n}, \model_n \sim \bel.
    \]
    \only<article>{In this case, the integrals are replaced by sampling models $\model_n$ from the belief, and then sampling $(x_n, y_n)$ pairs from $P_{\model_n}$.}
  \end{block}

\end{frame}

\begin{frame}
  \frametitle{Stochastic gradient methdos}
  \only<article>{To find the maximum of a differentiable function $g$, we can use gradient descent}
  \begin{block}{Gradient ascent}
    \[
      \param_{i+1} = \param_i + \alpha \nabla_\param g(\param_i).
    \]
  \end{block}

  \only<article>{When $f$ is an expectation, we don't need to calculate the full gradient. In fact, we only need to take one sample from the related distribution.}
  \begin{block}{Stochastic gradient ascent}
    \[
      g(\param) = \int_\Model f(\param, \model) \dd \bel(\model)
    \]
    \[
      \param_{i+1} = \param_i + \alpha \nabla_\param f(\param_i, \model_i), \qquad \model_i \sim \bel.
    \]
  \end{block}
  \only<article>{Stochastic gradient methods are commonly employed in neural networks.} 
  
\end{frame}

%%% Local Variables:
%%% mode: latex
%%% TeX-master: "notes"
%%% End:

 % linear models and stochastic gradient descent
\section{Nearest neighbours}
\begin{frame}
  \frametitle{Discriminating between diseases}
  % Title: glps_renderer figure
% Creator: GL2PS 1.3.8, (C) 1999-2012 C. Geuzaine
% For: Octave
% CreationDate: Fri Jun 16 12:38:10 2017
\begin{pgfpicture}
\pgfsetlinewidth{0.01pt}
\color[rgb]{1.000000,1.000000,1.000000}
\pgfpathmoveto{\pgfpoint{41.600006pt}{205.577454pt}}
\pgflineto{\pgfpoint{289.600037pt}{140.777435pt}}
\pgflineto{\pgfpoint{41.600006pt}{140.777435pt}}
\pgfpathclose
\pgfusepath{fill,stroke}
\pgfpathmoveto{\pgfpoint{41.600006pt}{205.577454pt}}
\pgflineto{\pgfpoint{289.600037pt}{205.577454pt}}
\pgflineto{\pgfpoint{289.600037pt}{140.777435pt}}
\pgfpathclose
\pgfusepath{fill,stroke}
\pgfpathmoveto{\pgfpoint{41.600006pt}{91.199989pt}}
\pgflineto{\pgfpoint{289.600037pt}{26.399979pt}}
\pgflineto{\pgfpoint{41.600006pt}{26.399979pt}}
\pgfpathclose
\pgfusepath{fill,stroke}
\pgfpathmoveto{\pgfpoint{41.600006pt}{91.199989pt}}
\pgflineto{\pgfpoint{289.600037pt}{91.199989pt}}
\pgflineto{\pgfpoint{289.600037pt}{26.399979pt}}
\pgfpathclose
\pgfusepath{fill,stroke}
\color[rgb]{1.000000,0.000000,0.000000}
\pgfpathmoveto{\pgfpoint{287.608032pt}{140.777435pt}}
\pgflineto{\pgfpoint{288.604004pt}{140.777435pt}}
\pgflineto{\pgfpoint{288.604004pt}{142.317886pt}}
\pgfpathclose
\pgfusepath{fill,stroke}
\pgfpathmoveto{\pgfpoint{289.600037pt}{141.234161pt}}
\pgflineto{\pgfpoint{288.604004pt}{142.317886pt}}
\pgflineto{\pgfpoint{288.604004pt}{140.777435pt}}
\pgfpathclose
\pgfusepath{fill,stroke}
\pgfpathmoveto{\pgfpoint{289.600037pt}{140.777435pt}}
\pgflineto{\pgfpoint{289.600037pt}{141.234161pt}}
\pgflineto{\pgfpoint{288.604004pt}{140.777435pt}}
\pgfpathclose
\pgfusepath{fill,stroke}
\pgfpathmoveto{\pgfpoint{283.624115pt}{140.777435pt}}
\pgflineto{\pgfpoint{284.620087pt}{140.777435pt}}
\pgflineto{\pgfpoint{284.620087pt}{140.819778pt}}
\pgfpathclose
\pgfusepath{fill,stroke}
\pgfpathmoveto{\pgfpoint{285.616089pt}{141.090790pt}}
\pgflineto{\pgfpoint{284.620087pt}{140.819778pt}}
\pgflineto{\pgfpoint{284.620087pt}{140.777435pt}}
\pgfpathclose
\pgfusepath{fill,stroke}
\pgfpathmoveto{\pgfpoint{285.616089pt}{140.777435pt}}
\pgflineto{\pgfpoint{285.616089pt}{141.090790pt}}
\pgflineto{\pgfpoint{284.620087pt}{140.777435pt}}
\pgfpathclose
\pgfusepath{fill,stroke}
\pgfpathmoveto{\pgfpoint{286.612061pt}{140.777435pt}}
\pgflineto{\pgfpoint{285.616089pt}{141.090790pt}}
\pgflineto{\pgfpoint{285.616089pt}{140.777435pt}}
\pgfpathclose
\pgfusepath{fill,stroke}
\pgfpathmoveto{\pgfpoint{275.656250pt}{140.777435pt}}
\pgflineto{\pgfpoint{276.652222pt}{140.777435pt}}
\pgflineto{\pgfpoint{276.652222pt}{141.883286pt}}
\pgfpathclose
\pgfusepath{fill,stroke}
\pgfpathmoveto{\pgfpoint{277.648193pt}{145.700470pt}}
\pgflineto{\pgfpoint{276.652222pt}{141.883286pt}}
\pgflineto{\pgfpoint{276.652222pt}{140.777435pt}}
\pgfpathclose
\pgfusepath{fill,stroke}
\pgfpathmoveto{\pgfpoint{277.648193pt}{140.777435pt}}
\pgflineto{\pgfpoint{277.648193pt}{145.700470pt}}
\pgflineto{\pgfpoint{276.652222pt}{140.777435pt}}
\pgfpathclose
\pgfusepath{fill,stroke}
\pgfpathmoveto{\pgfpoint{278.644196pt}{141.350510pt}}
\pgflineto{\pgfpoint{277.648193pt}{145.700470pt}}
\pgflineto{\pgfpoint{277.648193pt}{140.777435pt}}
\pgfpathclose
\pgfusepath{fill,stroke}
\pgfpathmoveto{\pgfpoint{278.644196pt}{140.777435pt}}
\pgflineto{\pgfpoint{278.644196pt}{141.350510pt}}
\pgflineto{\pgfpoint{277.648193pt}{140.777435pt}}
\pgfpathclose
\pgfusepath{fill,stroke}
\pgfpathmoveto{\pgfpoint{279.640167pt}{141.685425pt}}
\pgflineto{\pgfpoint{278.644196pt}{141.350510pt}}
\pgflineto{\pgfpoint{278.644196pt}{140.777435pt}}
\pgfpathclose
\pgfusepath{fill,stroke}
\pgfpathmoveto{\pgfpoint{279.640167pt}{140.777435pt}}
\pgflineto{\pgfpoint{279.640167pt}{141.685425pt}}
\pgflineto{\pgfpoint{278.644196pt}{140.777435pt}}
\pgfpathclose
\pgfusepath{fill,stroke}
\pgfpathmoveto{\pgfpoint{280.636139pt}{140.839111pt}}
\pgflineto{\pgfpoint{279.640167pt}{141.685425pt}}
\pgflineto{\pgfpoint{279.640167pt}{140.777435pt}}
\pgfpathclose
\pgfusepath{fill,stroke}
\pgfpathmoveto{\pgfpoint{280.636139pt}{140.777435pt}}
\pgflineto{\pgfpoint{280.636139pt}{140.839111pt}}
\pgflineto{\pgfpoint{279.640167pt}{140.777435pt}}
\pgfpathclose
\pgfusepath{fill,stroke}
\pgfpathmoveto{\pgfpoint{281.632141pt}{140.783401pt}}
\pgflineto{\pgfpoint{280.636139pt}{140.839111pt}}
\pgflineto{\pgfpoint{280.636139pt}{140.777435pt}}
\pgfpathclose
\pgfusepath{fill,stroke}
\pgfpathmoveto{\pgfpoint{281.632141pt}{140.777435pt}}
\pgflineto{\pgfpoint{281.632141pt}{140.783401pt}}
\pgflineto{\pgfpoint{280.636139pt}{140.777435pt}}
\pgfpathclose
\pgfusepath{fill,stroke}
\pgfpathmoveto{\pgfpoint{282.628113pt}{141.247986pt}}
\pgflineto{\pgfpoint{281.632141pt}{140.783401pt}}
\pgflineto{\pgfpoint{281.632141pt}{140.777435pt}}
\pgfpathclose
\pgfusepath{fill,stroke}
\pgfpathmoveto{\pgfpoint{282.628113pt}{140.777435pt}}
\pgflineto{\pgfpoint{282.628113pt}{141.247986pt}}
\pgflineto{\pgfpoint{281.632141pt}{140.777435pt}}
\pgfpathclose
\pgfusepath{fill,stroke}
\pgfpathmoveto{\pgfpoint{283.624115pt}{140.777435pt}}
\pgflineto{\pgfpoint{282.628113pt}{141.247986pt}}
\pgflineto{\pgfpoint{282.628113pt}{140.777435pt}}
\pgfpathclose
\pgfusepath{fill,stroke}
\pgfpathmoveto{\pgfpoint{262.708435pt}{140.777435pt}}
\pgflineto{\pgfpoint{263.704407pt}{140.777435pt}}
\pgflineto{\pgfpoint{263.704407pt}{149.672318pt}}
\pgfpathclose
\pgfusepath{fill,stroke}
\pgfpathmoveto{\pgfpoint{264.700409pt}{150.470245pt}}
\pgflineto{\pgfpoint{263.704407pt}{149.672318pt}}
\pgflineto{\pgfpoint{263.704407pt}{140.777435pt}}
\pgfpathclose
\pgfusepath{fill,stroke}
\pgfpathmoveto{\pgfpoint{264.700409pt}{140.777435pt}}
\pgflineto{\pgfpoint{264.700409pt}{150.470245pt}}
\pgflineto{\pgfpoint{263.704407pt}{140.777435pt}}
\pgfpathclose
\pgfusepath{fill,stroke}
\pgfpathmoveto{\pgfpoint{265.696411pt}{146.772659pt}}
\pgflineto{\pgfpoint{264.700409pt}{150.470245pt}}
\pgflineto{\pgfpoint{264.700409pt}{140.777435pt}}
\pgfpathclose
\pgfusepath{fill,stroke}
\pgfpathmoveto{\pgfpoint{265.696411pt}{140.777435pt}}
\pgflineto{\pgfpoint{265.696411pt}{146.772659pt}}
\pgflineto{\pgfpoint{264.700409pt}{140.777435pt}}
\pgfpathclose
\pgfusepath{fill,stroke}
\pgfpathmoveto{\pgfpoint{266.692383pt}{141.012543pt}}
\pgflineto{\pgfpoint{265.696411pt}{146.772659pt}}
\pgflineto{\pgfpoint{265.696411pt}{140.777435pt}}
\pgfpathclose
\pgfusepath{fill,stroke}
\pgfpathmoveto{\pgfpoint{266.692383pt}{140.777435pt}}
\pgflineto{\pgfpoint{266.692383pt}{141.012543pt}}
\pgflineto{\pgfpoint{265.696411pt}{140.777435pt}}
\pgfpathclose
\pgfusepath{fill,stroke}
\pgfpathmoveto{\pgfpoint{267.688354pt}{140.947601pt}}
\pgflineto{\pgfpoint{266.692383pt}{141.012543pt}}
\pgflineto{\pgfpoint{266.692383pt}{140.777435pt}}
\pgfpathclose
\pgfusepath{fill,stroke}
\pgfpathmoveto{\pgfpoint{267.688354pt}{140.777435pt}}
\pgflineto{\pgfpoint{267.688354pt}{140.947601pt}}
\pgflineto{\pgfpoint{266.692383pt}{140.777435pt}}
\pgfpathclose
\pgfusepath{fill,stroke}
\pgfpathmoveto{\pgfpoint{268.684326pt}{165.780823pt}}
\pgflineto{\pgfpoint{267.688354pt}{140.947601pt}}
\pgflineto{\pgfpoint{267.688354pt}{140.777435pt}}
\pgfpathclose
\pgfusepath{fill,stroke}
\pgfpathmoveto{\pgfpoint{268.684326pt}{140.777435pt}}
\pgflineto{\pgfpoint{268.684326pt}{165.780823pt}}
\pgflineto{\pgfpoint{267.688354pt}{140.777435pt}}
\pgfpathclose
\pgfusepath{fill,stroke}
\pgfpathmoveto{\pgfpoint{269.680328pt}{142.178589pt}}
\pgflineto{\pgfpoint{268.684326pt}{165.780823pt}}
\pgflineto{\pgfpoint{268.684326pt}{140.777435pt}}
\pgfpathclose
\pgfusepath{fill,stroke}
\pgfpathmoveto{\pgfpoint{269.680328pt}{140.777435pt}}
\pgflineto{\pgfpoint{269.680328pt}{142.178589pt}}
\pgflineto{\pgfpoint{268.684326pt}{140.777435pt}}
\pgfpathclose
\pgfusepath{fill,stroke}
\pgfpathmoveto{\pgfpoint{270.676331pt}{140.852936pt}}
\pgflineto{\pgfpoint{269.680328pt}{142.178589pt}}
\pgflineto{\pgfpoint{269.680328pt}{140.777435pt}}
\pgfpathclose
\pgfusepath{fill,stroke}
\pgfpathmoveto{\pgfpoint{270.676331pt}{140.777435pt}}
\pgflineto{\pgfpoint{270.676331pt}{140.852936pt}}
\pgflineto{\pgfpoint{269.680328pt}{140.777435pt}}
\pgfpathclose
\pgfusepath{fill,stroke}
\pgfpathmoveto{\pgfpoint{271.672302pt}{149.853119pt}}
\pgflineto{\pgfpoint{270.676331pt}{140.852936pt}}
\pgflineto{\pgfpoint{270.676331pt}{140.777435pt}}
\pgfpathclose
\pgfusepath{fill,stroke}
\pgfpathmoveto{\pgfpoint{271.672302pt}{140.777435pt}}
\pgflineto{\pgfpoint{271.672302pt}{149.853119pt}}
\pgflineto{\pgfpoint{270.676331pt}{140.777435pt}}
\pgfpathclose
\pgfusepath{fill,stroke}
\pgfpathmoveto{\pgfpoint{272.668274pt}{167.198486pt}}
\pgflineto{\pgfpoint{271.672302pt}{149.853119pt}}
\pgflineto{\pgfpoint{271.672302pt}{140.777435pt}}
\pgfpathclose
\pgfusepath{fill,stroke}
\pgfpathmoveto{\pgfpoint{272.668274pt}{140.777435pt}}
\pgflineto{\pgfpoint{272.668274pt}{167.198486pt}}
\pgflineto{\pgfpoint{271.672302pt}{140.777435pt}}
\pgfpathclose
\pgfusepath{fill,stroke}
\pgfpathmoveto{\pgfpoint{273.664276pt}{140.777435pt}}
\pgflineto{\pgfpoint{272.668274pt}{167.198486pt}}
\pgflineto{\pgfpoint{272.668274pt}{140.777435pt}}
\pgfpathclose
\pgfusepath{fill,stroke}
\pgfpathmoveto{\pgfpoint{255.736542pt}{140.777435pt}}
\pgflineto{\pgfpoint{256.732544pt}{140.777435pt}}
\pgflineto{\pgfpoint{256.732544pt}{160.055664pt}}
\pgfpathclose
\pgfusepath{fill,stroke}
\pgfpathmoveto{\pgfpoint{257.728516pt}{145.276688pt}}
\pgflineto{\pgfpoint{256.732544pt}{160.055664pt}}
\pgflineto{\pgfpoint{256.732544pt}{140.777435pt}}
\pgfpathclose
\pgfusepath{fill,stroke}
\pgfpathmoveto{\pgfpoint{257.728516pt}{140.777435pt}}
\pgflineto{\pgfpoint{257.728516pt}{145.276688pt}}
\pgflineto{\pgfpoint{256.732544pt}{140.777435pt}}
\pgfpathclose
\pgfusepath{fill,stroke}
\pgfpathmoveto{\pgfpoint{258.724518pt}{145.143265pt}}
\pgflineto{\pgfpoint{257.728516pt}{145.276688pt}}
\pgflineto{\pgfpoint{257.728516pt}{140.777435pt}}
\pgfpathclose
\pgfusepath{fill,stroke}
\pgfpathmoveto{\pgfpoint{258.724518pt}{140.777435pt}}
\pgflineto{\pgfpoint{258.724518pt}{145.143265pt}}
\pgflineto{\pgfpoint{257.728516pt}{140.777435pt}}
\pgfpathclose
\pgfusepath{fill,stroke}
\pgfpathmoveto{\pgfpoint{259.720490pt}{141.107025pt}}
\pgflineto{\pgfpoint{258.724518pt}{145.143265pt}}
\pgflineto{\pgfpoint{258.724518pt}{140.777435pt}}
\pgfpathclose
\pgfusepath{fill,stroke}
\pgfpathmoveto{\pgfpoint{259.720490pt}{140.777435pt}}
\pgflineto{\pgfpoint{259.720490pt}{141.107025pt}}
\pgflineto{\pgfpoint{258.724518pt}{140.777435pt}}
\pgfpathclose
\pgfusepath{fill,stroke}
\pgfpathmoveto{\pgfpoint{260.716492pt}{140.785751pt}}
\pgflineto{\pgfpoint{259.720490pt}{141.107025pt}}
\pgflineto{\pgfpoint{259.720490pt}{140.777435pt}}
\pgfpathclose
\pgfusepath{fill,stroke}
\pgfpathmoveto{\pgfpoint{260.716492pt}{140.777435pt}}
\pgflineto{\pgfpoint{260.716492pt}{140.785751pt}}
\pgflineto{\pgfpoint{259.720490pt}{140.777435pt}}
\pgfpathclose
\pgfusepath{fill,stroke}
\pgfpathmoveto{\pgfpoint{261.712463pt}{140.881470pt}}
\pgflineto{\pgfpoint{260.716492pt}{140.785751pt}}
\pgflineto{\pgfpoint{260.716492pt}{140.777435pt}}
\pgfpathclose
\pgfusepath{fill,stroke}
\pgfpathmoveto{\pgfpoint{261.712463pt}{140.777435pt}}
\pgflineto{\pgfpoint{261.712463pt}{140.881470pt}}
\pgflineto{\pgfpoint{260.716492pt}{140.777435pt}}
\pgfpathclose
\pgfusepath{fill,stroke}
\pgfpathmoveto{\pgfpoint{262.708435pt}{140.777435pt}}
\pgflineto{\pgfpoint{261.712463pt}{140.881470pt}}
\pgflineto{\pgfpoint{261.712463pt}{140.777435pt}}
\pgfpathclose
\pgfusepath{fill,stroke}
\pgfpathmoveto{\pgfpoint{252.748611pt}{140.777435pt}}
\pgflineto{\pgfpoint{253.744598pt}{205.642242pt}}
\pgflineto{\pgfpoint{253.234100pt}{205.642242pt}}
\pgfpathclose
\pgfusepath{fill,stroke}
\pgfpathmoveto{\pgfpoint{252.748611pt}{140.777435pt}}
\pgflineto{\pgfpoint{253.744598pt}{140.777435pt}}
\pgflineto{\pgfpoint{253.744598pt}{205.642242pt}}
\pgfpathclose
\pgfusepath{fill,stroke}
\pgfpathmoveto{\pgfpoint{254.325424pt}{205.642242pt}}
\pgflineto{\pgfpoint{253.744598pt}{205.642242pt}}
\pgflineto{\pgfpoint{253.744598pt}{140.777435pt}}
\pgfpathclose
\pgfusepath{fill,stroke}
\pgfpathmoveto{\pgfpoint{254.740570pt}{140.777435pt}}
\pgflineto{\pgfpoint{254.325424pt}{205.642242pt}}
\pgflineto{\pgfpoint{253.744598pt}{140.777435pt}}
\pgfpathclose
\pgfusepath{fill,stroke}
\pgfpathmoveto{\pgfpoint{254.740570pt}{140.777435pt}}
\pgflineto{\pgfpoint{254.740570pt}{205.642242pt}}
\pgflineto{\pgfpoint{254.325424pt}{205.642242pt}}
\pgfpathclose
\pgfusepath{fill,stroke}
\pgfpathmoveto{\pgfpoint{255.736542pt}{140.777435pt}}
\pgflineto{\pgfpoint{254.740570pt}{205.642242pt}}
\pgflineto{\pgfpoint{254.740570pt}{140.777435pt}}
\pgfpathclose
\pgfusepath{fill,stroke}
\pgfpathmoveto{\pgfpoint{255.736542pt}{140.777435pt}}
\pgflineto{\pgfpoint{255.155716pt}{205.642242pt}}
\pgflineto{\pgfpoint{254.740570pt}{205.642242pt}}
\pgfpathclose
\pgfusepath{fill,stroke}
\pgfpathmoveto{\pgfpoint{244.780731pt}{140.777435pt}}
\pgflineto{\pgfpoint{245.776718pt}{140.777435pt}}
\pgflineto{\pgfpoint{245.776718pt}{140.921082pt}}
\pgfpathclose
\pgfusepath{fill,stroke}
\pgfpathmoveto{\pgfpoint{246.231064pt}{205.642242pt}}
\pgflineto{\pgfpoint{245.776718pt}{140.921082pt}}
\pgflineto{\pgfpoint{245.776718pt}{140.777435pt}}
\pgfpathclose
\pgfusepath{fill,stroke}
\pgfpathmoveto{\pgfpoint{246.231064pt}{205.642242pt}}
\pgflineto{\pgfpoint{246.230515pt}{205.642242pt}}
\pgflineto{\pgfpoint{245.776718pt}{140.921082pt}}
\pgfpathclose
\pgfusepath{fill,stroke}
\pgfpathmoveto{\pgfpoint{246.772705pt}{140.777435pt}}
\pgflineto{\pgfpoint{246.231064pt}{205.642242pt}}
\pgflineto{\pgfpoint{245.776718pt}{140.777435pt}}
\pgfpathclose
\pgfusepath{fill,stroke}
\pgfpathmoveto{\pgfpoint{246.772705pt}{140.777435pt}}
\pgflineto{\pgfpoint{246.772705pt}{205.642242pt}}
\pgflineto{\pgfpoint{246.231064pt}{205.642242pt}}
\pgfpathclose
\pgfusepath{fill,stroke}
\pgfpathmoveto{\pgfpoint{247.768677pt}{144.035980pt}}
\pgflineto{\pgfpoint{246.772705pt}{205.642242pt}}
\pgflineto{\pgfpoint{246.772705pt}{140.777435pt}}
\pgfpathclose
\pgfusepath{fill,stroke}
\pgfpathmoveto{\pgfpoint{247.768677pt}{144.035980pt}}
\pgflineto{\pgfpoint{247.327042pt}{205.642242pt}}
\pgflineto{\pgfpoint{246.772705pt}{205.642242pt}}
\pgfpathclose
\pgfusepath{fill,stroke}
\pgfpathmoveto{\pgfpoint{247.768677pt}{140.777435pt}}
\pgflineto{\pgfpoint{247.768677pt}{144.035980pt}}
\pgflineto{\pgfpoint{246.772705pt}{140.777435pt}}
\pgfpathclose
\pgfusepath{fill,stroke}
\pgfpathmoveto{\pgfpoint{248.764679pt}{140.987183pt}}
\pgflineto{\pgfpoint{247.768677pt}{144.035980pt}}
\pgflineto{\pgfpoint{247.768677pt}{140.777435pt}}
\pgfpathclose
\pgfusepath{fill,stroke}
\pgfpathmoveto{\pgfpoint{248.764679pt}{140.777435pt}}
\pgflineto{\pgfpoint{248.764679pt}{140.987183pt}}
\pgflineto{\pgfpoint{247.768677pt}{140.777435pt}}
\pgfpathclose
\pgfusepath{fill,stroke}
\pgfpathmoveto{\pgfpoint{249.760651pt}{140.861908pt}}
\pgflineto{\pgfpoint{248.764679pt}{140.987183pt}}
\pgflineto{\pgfpoint{248.764679pt}{140.777435pt}}
\pgfpathclose
\pgfusepath{fill,stroke}
\pgfpathmoveto{\pgfpoint{249.760651pt}{140.777435pt}}
\pgflineto{\pgfpoint{249.760651pt}{140.861908pt}}
\pgflineto{\pgfpoint{248.764679pt}{140.777435pt}}
\pgfpathclose
\pgfusepath{fill,stroke}
\pgfpathmoveto{\pgfpoint{250.756638pt}{143.997910pt}}
\pgflineto{\pgfpoint{249.760651pt}{140.861908pt}}
\pgflineto{\pgfpoint{249.760651pt}{140.777435pt}}
\pgfpathclose
\pgfusepath{fill,stroke}
\pgfpathmoveto{\pgfpoint{250.756638pt}{140.777435pt}}
\pgflineto{\pgfpoint{250.756638pt}{143.997910pt}}
\pgflineto{\pgfpoint{249.760651pt}{140.777435pt}}
\pgfpathclose
\pgfusepath{fill,stroke}
\pgfpathmoveto{\pgfpoint{251.752625pt}{140.777435pt}}
\pgflineto{\pgfpoint{250.756638pt}{143.997910pt}}
\pgflineto{\pgfpoint{250.756638pt}{140.777435pt}}
\pgfpathclose
\pgfusepath{fill,stroke}
\pgfpathmoveto{\pgfpoint{241.792786pt}{140.777435pt}}
\pgflineto{\pgfpoint{242.788757pt}{140.777435pt}}
\pgflineto{\pgfpoint{242.788757pt}{141.050537pt}}
\pgfpathclose
\pgfusepath{fill,stroke}
\pgfpathmoveto{\pgfpoint{243.784744pt}{140.826157pt}}
\pgflineto{\pgfpoint{242.788757pt}{141.050537pt}}
\pgflineto{\pgfpoint{242.788757pt}{140.777435pt}}
\pgfpathclose
\pgfusepath{fill,stroke}
\pgfpathmoveto{\pgfpoint{243.784744pt}{140.777435pt}}
\pgflineto{\pgfpoint{243.784744pt}{140.826157pt}}
\pgflineto{\pgfpoint{242.788757pt}{140.777435pt}}
\pgfpathclose
\pgfusepath{fill,stroke}
\pgfpathmoveto{\pgfpoint{244.780731pt}{140.777435pt}}
\pgflineto{\pgfpoint{243.784744pt}{140.826157pt}}
\pgflineto{\pgfpoint{243.784744pt}{140.777435pt}}
\pgfpathclose
\pgfusepath{fill,stroke}
\pgfpathmoveto{\pgfpoint{239.800812pt}{140.777435pt}}
\pgflineto{\pgfpoint{240.796814pt}{140.777435pt}}
\pgflineto{\pgfpoint{240.796814pt}{141.635849pt}}
\pgfpathclose
\pgfusepath{fill,stroke}
\pgfpathmoveto{\pgfpoint{241.792786pt}{140.777435pt}}
\pgflineto{\pgfpoint{240.796814pt}{141.635849pt}}
\pgflineto{\pgfpoint{240.796814pt}{140.777435pt}}
\pgfpathclose
\pgfusepath{fill,stroke}
\pgfpathmoveto{\pgfpoint{232.828934pt}{140.777435pt}}
\pgflineto{\pgfpoint{233.824921pt}{140.777435pt}}
\pgflineto{\pgfpoint{233.824921pt}{143.663483pt}}
\pgfpathclose
\pgfusepath{fill,stroke}
\pgfpathmoveto{\pgfpoint{234.820892pt}{144.424850pt}}
\pgflineto{\pgfpoint{233.824921pt}{143.663483pt}}
\pgflineto{\pgfpoint{233.824921pt}{140.777435pt}}
\pgfpathclose
\pgfusepath{fill,stroke}
\pgfpathmoveto{\pgfpoint{234.820892pt}{140.777435pt}}
\pgflineto{\pgfpoint{234.820892pt}{144.424850pt}}
\pgflineto{\pgfpoint{233.824921pt}{140.777435pt}}
\pgfpathclose
\pgfusepath{fill,stroke}
\pgfpathmoveto{\pgfpoint{235.816864pt}{141.070068pt}}
\pgflineto{\pgfpoint{234.820892pt}{144.424850pt}}
\pgflineto{\pgfpoint{234.820892pt}{140.777435pt}}
\pgfpathclose
\pgfusepath{fill,stroke}
\pgfpathmoveto{\pgfpoint{235.816864pt}{140.777435pt}}
\pgflineto{\pgfpoint{235.816864pt}{141.070068pt}}
\pgflineto{\pgfpoint{234.820892pt}{140.777435pt}}
\pgfpathclose
\pgfusepath{fill,stroke}
\pgfpathmoveto{\pgfpoint{236.812866pt}{141.387177pt}}
\pgflineto{\pgfpoint{235.816864pt}{141.070068pt}}
\pgflineto{\pgfpoint{235.816864pt}{140.777435pt}}
\pgfpathclose
\pgfusepath{fill,stroke}
\pgfpathmoveto{\pgfpoint{236.812866pt}{140.777435pt}}
\pgflineto{\pgfpoint{236.812866pt}{141.387177pt}}
\pgflineto{\pgfpoint{235.816864pt}{140.777435pt}}
\pgfpathclose
\pgfusepath{fill,stroke}
\pgfpathmoveto{\pgfpoint{237.808838pt}{141.605896pt}}
\pgflineto{\pgfpoint{236.812866pt}{141.387177pt}}
\pgflineto{\pgfpoint{236.812866pt}{140.777435pt}}
\pgfpathclose
\pgfusepath{fill,stroke}
\pgfpathmoveto{\pgfpoint{237.808838pt}{140.777435pt}}
\pgflineto{\pgfpoint{237.808838pt}{141.605896pt}}
\pgflineto{\pgfpoint{236.812866pt}{140.777435pt}}
\pgfpathclose
\pgfusepath{fill,stroke}
\pgfpathmoveto{\pgfpoint{238.804825pt}{141.892487pt}}
\pgflineto{\pgfpoint{237.808838pt}{141.605896pt}}
\pgflineto{\pgfpoint{237.808838pt}{140.777435pt}}
\pgfpathclose
\pgfusepath{fill,stroke}
\pgfpathmoveto{\pgfpoint{238.804825pt}{140.777435pt}}
\pgflineto{\pgfpoint{238.804825pt}{141.892487pt}}
\pgflineto{\pgfpoint{237.808838pt}{140.777435pt}}
\pgfpathclose
\pgfusepath{fill,stroke}
\pgfpathmoveto{\pgfpoint{239.800812pt}{140.777435pt}}
\pgflineto{\pgfpoint{238.804825pt}{141.892487pt}}
\pgflineto{\pgfpoint{238.804825pt}{140.777435pt}}
\pgfpathclose
\pgfusepath{fill,stroke}
\pgfpathmoveto{\pgfpoint{226.853027pt}{140.777435pt}}
\pgflineto{\pgfpoint{227.849014pt}{140.777435pt}}
\pgflineto{\pgfpoint{227.849014pt}{141.269379pt}}
\pgfpathclose
\pgfusepath{fill,stroke}
\pgfpathmoveto{\pgfpoint{228.845001pt}{143.487183pt}}
\pgflineto{\pgfpoint{227.849014pt}{141.269379pt}}
\pgflineto{\pgfpoint{227.849014pt}{140.777435pt}}
\pgfpathclose
\pgfusepath{fill,stroke}
\pgfpathmoveto{\pgfpoint{228.845001pt}{140.777435pt}}
\pgflineto{\pgfpoint{228.845001pt}{143.487183pt}}
\pgflineto{\pgfpoint{227.849014pt}{140.777435pt}}
\pgfpathclose
\pgfusepath{fill,stroke}
\pgfpathmoveto{\pgfpoint{229.840973pt}{143.864670pt}}
\pgflineto{\pgfpoint{228.845001pt}{143.487183pt}}
\pgflineto{\pgfpoint{228.845001pt}{140.777435pt}}
\pgfpathclose
\pgfusepath{fill,stroke}
\pgfpathmoveto{\pgfpoint{229.840973pt}{140.777435pt}}
\pgflineto{\pgfpoint{229.840973pt}{143.864670pt}}
\pgflineto{\pgfpoint{228.845001pt}{140.777435pt}}
\pgfpathclose
\pgfusepath{fill,stroke}
\pgfpathmoveto{\pgfpoint{230.836945pt}{141.428329pt}}
\pgflineto{\pgfpoint{229.840973pt}{143.864670pt}}
\pgflineto{\pgfpoint{229.840973pt}{140.777435pt}}
\pgfpathclose
\pgfusepath{fill,stroke}
\pgfpathmoveto{\pgfpoint{230.836945pt}{140.777435pt}}
\pgflineto{\pgfpoint{230.836945pt}{141.428329pt}}
\pgflineto{\pgfpoint{229.840973pt}{140.777435pt}}
\pgfpathclose
\pgfusepath{fill,stroke}
\pgfpathmoveto{\pgfpoint{231.832932pt}{162.511902pt}}
\pgflineto{\pgfpoint{230.836945pt}{141.428329pt}}
\pgflineto{\pgfpoint{230.836945pt}{140.777435pt}}
\pgfpathclose
\pgfusepath{fill,stroke}
\pgfpathmoveto{\pgfpoint{231.832932pt}{140.777435pt}}
\pgflineto{\pgfpoint{231.832932pt}{162.511902pt}}
\pgflineto{\pgfpoint{230.836945pt}{140.777435pt}}
\pgfpathclose
\pgfusepath{fill,stroke}
\pgfpathmoveto{\pgfpoint{232.828934pt}{140.777435pt}}
\pgflineto{\pgfpoint{231.832932pt}{162.511902pt}}
\pgflineto{\pgfpoint{231.832932pt}{140.777435pt}}
\pgfpathclose
\pgfusepath{fill,stroke}
\pgfpathmoveto{\pgfpoint{222.869080pt}{140.777435pt}}
\pgflineto{\pgfpoint{223.865082pt}{140.777435pt}}
\pgflineto{\pgfpoint{223.865082pt}{142.648438pt}}
\pgfpathclose
\pgfusepath{fill,stroke}
\pgfpathmoveto{\pgfpoint{224.861053pt}{140.814316pt}}
\pgflineto{\pgfpoint{223.865082pt}{142.648438pt}}
\pgflineto{\pgfpoint{223.865082pt}{140.777435pt}}
\pgfpathclose
\pgfusepath{fill,stroke}
\pgfpathmoveto{\pgfpoint{224.861053pt}{140.777435pt}}
\pgflineto{\pgfpoint{224.861053pt}{140.814316pt}}
\pgflineto{\pgfpoint{223.865082pt}{140.777435pt}}
\pgfpathclose
\pgfusepath{fill,stroke}
\pgfpathmoveto{\pgfpoint{225.857040pt}{142.893295pt}}
\pgflineto{\pgfpoint{224.861053pt}{140.814316pt}}
\pgflineto{\pgfpoint{224.861053pt}{140.777435pt}}
\pgfpathclose
\pgfusepath{fill,stroke}
\pgfpathmoveto{\pgfpoint{225.857040pt}{140.777435pt}}
\pgflineto{\pgfpoint{225.857040pt}{142.893295pt}}
\pgflineto{\pgfpoint{224.861053pt}{140.777435pt}}
\pgfpathclose
\pgfusepath{fill,stroke}
\pgfpathmoveto{\pgfpoint{226.853027pt}{140.777435pt}}
\pgflineto{\pgfpoint{225.857040pt}{142.893295pt}}
\pgflineto{\pgfpoint{225.857040pt}{140.777435pt}}
\pgfpathclose
\pgfusepath{fill,stroke}
\pgfpathmoveto{\pgfpoint{218.885147pt}{140.777435pt}}
\pgflineto{\pgfpoint{219.881134pt}{140.777435pt}}
\pgflineto{\pgfpoint{219.881134pt}{141.240646pt}}
\pgfpathclose
\pgfusepath{fill,stroke}
\pgfpathmoveto{\pgfpoint{220.877121pt}{140.777435pt}}
\pgflineto{\pgfpoint{219.881134pt}{141.240646pt}}
\pgflineto{\pgfpoint{219.881134pt}{140.777435pt}}
\pgfpathclose
\pgfusepath{fill,stroke}
\pgfpathmoveto{\pgfpoint{216.893188pt}{140.777435pt}}
\pgflineto{\pgfpoint{217.889160pt}{140.777435pt}}
\pgflineto{\pgfpoint{217.889160pt}{144.868698pt}}
\pgfpathclose
\pgfusepath{fill,stroke}
\pgfpathmoveto{\pgfpoint{218.885147pt}{140.777435pt}}
\pgflineto{\pgfpoint{217.889160pt}{144.868698pt}}
\pgflineto{\pgfpoint{217.889160pt}{140.777435pt}}
\pgfpathclose
\pgfusepath{fill,stroke}
\pgfpathmoveto{\pgfpoint{207.929337pt}{140.777435pt}}
\pgflineto{\pgfpoint{208.925323pt}{140.777435pt}}
\pgflineto{\pgfpoint{208.925323pt}{168.977936pt}}
\pgfpathclose
\pgfusepath{fill,stroke}
\pgfpathmoveto{\pgfpoint{209.716339pt}{205.642242pt}}
\pgflineto{\pgfpoint{208.925323pt}{168.977936pt}}
\pgflineto{\pgfpoint{208.925323pt}{140.777435pt}}
\pgfpathclose
\pgfusepath{fill,stroke}
\pgfpathmoveto{\pgfpoint{209.716339pt}{205.642242pt}}
\pgflineto{\pgfpoint{209.608246pt}{205.642242pt}}
\pgflineto{\pgfpoint{208.925323pt}{168.977936pt}}
\pgfpathclose
\pgfusepath{fill,stroke}
\pgfpathmoveto{\pgfpoint{209.921295pt}{140.777435pt}}
\pgflineto{\pgfpoint{209.716339pt}{205.642242pt}}
\pgflineto{\pgfpoint{208.925323pt}{140.777435pt}}
\pgfpathclose
\pgfusepath{fill,stroke}
\pgfpathmoveto{\pgfpoint{209.921295pt}{140.777435pt}}
\pgflineto{\pgfpoint{209.921295pt}{205.642242pt}}
\pgflineto{\pgfpoint{209.716339pt}{205.642242pt}}
\pgfpathclose
\pgfusepath{fill,stroke}
\pgfpathmoveto{\pgfpoint{210.917267pt}{156.156219pt}}
\pgflineto{\pgfpoint{209.921295pt}{205.642242pt}}
\pgflineto{\pgfpoint{209.921295pt}{140.777435pt}}
\pgfpathclose
\pgfusepath{fill,stroke}
\pgfpathmoveto{\pgfpoint{210.917267pt}{156.156219pt}}
\pgflineto{\pgfpoint{210.173798pt}{205.642242pt}}
\pgflineto{\pgfpoint{209.921295pt}{205.642242pt}}
\pgfpathclose
\pgfusepath{fill,stroke}
\pgfpathmoveto{\pgfpoint{210.917267pt}{140.777435pt}}
\pgflineto{\pgfpoint{210.917267pt}{156.156219pt}}
\pgflineto{\pgfpoint{209.921295pt}{140.777435pt}}
\pgfpathclose
\pgfusepath{fill,stroke}
\pgfpathmoveto{\pgfpoint{211.913269pt}{142.768951pt}}
\pgflineto{\pgfpoint{210.917267pt}{156.156219pt}}
\pgflineto{\pgfpoint{210.917267pt}{140.777435pt}}
\pgfpathclose
\pgfusepath{fill,stroke}
\pgfpathmoveto{\pgfpoint{211.913269pt}{140.777435pt}}
\pgflineto{\pgfpoint{211.913269pt}{142.768951pt}}
\pgflineto{\pgfpoint{210.917267pt}{140.777435pt}}
\pgfpathclose
\pgfusepath{fill,stroke}
\pgfpathmoveto{\pgfpoint{212.909241pt}{140.796997pt}}
\pgflineto{\pgfpoint{211.913269pt}{142.768951pt}}
\pgflineto{\pgfpoint{211.913269pt}{140.777435pt}}
\pgfpathclose
\pgfusepath{fill,stroke}
\pgfpathmoveto{\pgfpoint{212.909241pt}{140.777435pt}}
\pgflineto{\pgfpoint{212.909241pt}{140.796997pt}}
\pgflineto{\pgfpoint{211.913269pt}{140.777435pt}}
\pgfpathclose
\pgfusepath{fill,stroke}
\pgfpathmoveto{\pgfpoint{213.905228pt}{147.912155pt}}
\pgflineto{\pgfpoint{212.909241pt}{140.796997pt}}
\pgflineto{\pgfpoint{212.909241pt}{140.777435pt}}
\pgfpathclose
\pgfusepath{fill,stroke}
\pgfpathmoveto{\pgfpoint{213.905228pt}{140.777435pt}}
\pgflineto{\pgfpoint{213.905228pt}{147.912155pt}}
\pgflineto{\pgfpoint{212.909241pt}{140.777435pt}}
\pgfpathclose
\pgfusepath{fill,stroke}
\pgfpathmoveto{\pgfpoint{214.901215pt}{141.340469pt}}
\pgflineto{\pgfpoint{213.905228pt}{147.912155pt}}
\pgflineto{\pgfpoint{213.905228pt}{140.777435pt}}
\pgfpathclose
\pgfusepath{fill,stroke}
\pgfpathmoveto{\pgfpoint{214.901215pt}{140.777435pt}}
\pgflineto{\pgfpoint{214.901215pt}{141.340469pt}}
\pgflineto{\pgfpoint{213.905228pt}{140.777435pt}}
\pgfpathclose
\pgfusepath{fill,stroke}
\pgfpathmoveto{\pgfpoint{215.897217pt}{146.916656pt}}
\pgflineto{\pgfpoint{214.901215pt}{141.340469pt}}
\pgflineto{\pgfpoint{214.901215pt}{140.777435pt}}
\pgfpathclose
\pgfusepath{fill,stroke}
\pgfpathmoveto{\pgfpoint{215.897217pt}{140.777435pt}}
\pgflineto{\pgfpoint{215.897217pt}{146.916656pt}}
\pgflineto{\pgfpoint{214.901215pt}{140.777435pt}}
\pgfpathclose
\pgfusepath{fill,stroke}
\pgfpathmoveto{\pgfpoint{216.893188pt}{140.777435pt}}
\pgflineto{\pgfpoint{215.897217pt}{146.916656pt}}
\pgflineto{\pgfpoint{215.897217pt}{140.777435pt}}
\pgfpathclose
\pgfusepath{fill,stroke}
\pgfpathmoveto{\pgfpoint{204.941376pt}{140.777435pt}}
\pgflineto{\pgfpoint{205.937347pt}{140.777435pt}}
\pgflineto{\pgfpoint{205.937347pt}{144.822937pt}}
\pgfpathclose
\pgfusepath{fill,stroke}
\pgfpathmoveto{\pgfpoint{206.933334pt}{140.777435pt}}
\pgflineto{\pgfpoint{205.937347pt}{144.822937pt}}
\pgflineto{\pgfpoint{205.937347pt}{140.777435pt}}
\pgfpathclose
\pgfusepath{fill,stroke}
\pgfpathmoveto{\pgfpoint{199.961456pt}{140.777435pt}}
\pgflineto{\pgfpoint{200.957443pt}{140.777435pt}}
\pgflineto{\pgfpoint{200.957443pt}{141.105011pt}}
\pgfpathclose
\pgfusepath{fill,stroke}
\pgfpathmoveto{\pgfpoint{201.953430pt}{140.916779pt}}
\pgflineto{\pgfpoint{200.957443pt}{141.105011pt}}
\pgflineto{\pgfpoint{200.957443pt}{140.777435pt}}
\pgfpathclose
\pgfusepath{fill,stroke}
\pgfpathmoveto{\pgfpoint{201.953430pt}{140.777435pt}}
\pgflineto{\pgfpoint{201.953430pt}{140.916779pt}}
\pgflineto{\pgfpoint{200.957443pt}{140.777435pt}}
\pgfpathclose
\pgfusepath{fill,stroke}
\pgfpathmoveto{\pgfpoint{202.949402pt}{140.777435pt}}
\pgflineto{\pgfpoint{201.953430pt}{140.916779pt}}
\pgflineto{\pgfpoint{201.953430pt}{140.777435pt}}
\pgfpathclose
\pgfusepath{fill,stroke}
\pgfpathmoveto{\pgfpoint{188.009659pt}{140.777435pt}}
\pgflineto{\pgfpoint{189.005630pt}{140.777435pt}}
\pgflineto{\pgfpoint{189.005630pt}{140.805771pt}}
\pgfpathclose
\pgfusepath{fill,stroke}
\pgfpathmoveto{\pgfpoint{190.001617pt}{149.564423pt}}
\pgflineto{\pgfpoint{189.005630pt}{140.805771pt}}
\pgflineto{\pgfpoint{189.005630pt}{140.777435pt}}
\pgfpathclose
\pgfusepath{fill,stroke}
\pgfpathmoveto{\pgfpoint{190.001617pt}{140.777435pt}}
\pgflineto{\pgfpoint{190.001617pt}{149.564423pt}}
\pgflineto{\pgfpoint{189.005630pt}{140.777435pt}}
\pgfpathclose
\pgfusepath{fill,stroke}
\pgfpathmoveto{\pgfpoint{190.997604pt}{140.863358pt}}
\pgflineto{\pgfpoint{190.001617pt}{149.564423pt}}
\pgflineto{\pgfpoint{190.001617pt}{140.777435pt}}
\pgfpathclose
\pgfusepath{fill,stroke}
\pgfpathmoveto{\pgfpoint{190.997604pt}{140.777435pt}}
\pgflineto{\pgfpoint{190.997604pt}{140.863358pt}}
\pgflineto{\pgfpoint{190.001617pt}{140.777435pt}}
\pgfpathclose
\pgfusepath{fill,stroke}
\pgfpathmoveto{\pgfpoint{191.993591pt}{142.184387pt}}
\pgflineto{\pgfpoint{190.997604pt}{140.863358pt}}
\pgflineto{\pgfpoint{190.997604pt}{140.777435pt}}
\pgfpathclose
\pgfusepath{fill,stroke}
\pgfpathmoveto{\pgfpoint{191.993591pt}{140.777435pt}}
\pgflineto{\pgfpoint{191.993591pt}{142.184387pt}}
\pgflineto{\pgfpoint{190.997604pt}{140.777435pt}}
\pgfpathclose
\pgfusepath{fill,stroke}
\pgfpathmoveto{\pgfpoint{192.989563pt}{195.151352pt}}
\pgflineto{\pgfpoint{191.993591pt}{142.184387pt}}
\pgflineto{\pgfpoint{191.993591pt}{140.777435pt}}
\pgfpathclose
\pgfusepath{fill,stroke}
\pgfpathmoveto{\pgfpoint{192.989563pt}{140.777435pt}}
\pgflineto{\pgfpoint{192.989563pt}{195.151352pt}}
\pgflineto{\pgfpoint{191.993591pt}{140.777435pt}}
\pgfpathclose
\pgfusepath{fill,stroke}
\pgfpathmoveto{\pgfpoint{193.985565pt}{141.529373pt}}
\pgflineto{\pgfpoint{192.989563pt}{195.151352pt}}
\pgflineto{\pgfpoint{192.989563pt}{140.777435pt}}
\pgfpathclose
\pgfusepath{fill,stroke}
\pgfpathmoveto{\pgfpoint{193.985565pt}{140.777435pt}}
\pgflineto{\pgfpoint{193.985565pt}{141.529373pt}}
\pgflineto{\pgfpoint{192.989563pt}{140.777435pt}}
\pgfpathclose
\pgfusepath{fill,stroke}
\pgfpathmoveto{\pgfpoint{194.981537pt}{140.831909pt}}
\pgflineto{\pgfpoint{193.985565pt}{141.529373pt}}
\pgflineto{\pgfpoint{193.985565pt}{140.777435pt}}
\pgfpathclose
\pgfusepath{fill,stroke}
\pgfpathmoveto{\pgfpoint{194.981537pt}{140.777435pt}}
\pgflineto{\pgfpoint{194.981537pt}{140.831909pt}}
\pgflineto{\pgfpoint{193.985565pt}{140.777435pt}}
\pgfpathclose
\pgfusepath{fill,stroke}
\pgfpathmoveto{\pgfpoint{195.977524pt}{149.854721pt}}
\pgflineto{\pgfpoint{194.981537pt}{140.831909pt}}
\pgflineto{\pgfpoint{194.981537pt}{140.777435pt}}
\pgfpathclose
\pgfusepath{fill,stroke}
\pgfpathmoveto{\pgfpoint{195.977524pt}{140.777435pt}}
\pgflineto{\pgfpoint{195.977524pt}{149.854721pt}}
\pgflineto{\pgfpoint{194.981537pt}{140.777435pt}}
\pgfpathclose
\pgfusepath{fill,stroke}
\pgfpathmoveto{\pgfpoint{196.973511pt}{144.710114pt}}
\pgflineto{\pgfpoint{195.977524pt}{149.854721pt}}
\pgflineto{\pgfpoint{195.977524pt}{140.777435pt}}
\pgfpathclose
\pgfusepath{fill,stroke}
\pgfpathmoveto{\pgfpoint{196.973511pt}{140.777435pt}}
\pgflineto{\pgfpoint{196.973511pt}{144.710114pt}}
\pgflineto{\pgfpoint{195.977524pt}{140.777435pt}}
\pgfpathclose
\pgfusepath{fill,stroke}
\pgfpathmoveto{\pgfpoint{197.969498pt}{144.257416pt}}
\pgflineto{\pgfpoint{196.973511pt}{144.710114pt}}
\pgflineto{\pgfpoint{196.973511pt}{140.777435pt}}
\pgfpathclose
\pgfusepath{fill,stroke}
\pgfpathmoveto{\pgfpoint{197.969498pt}{140.777435pt}}
\pgflineto{\pgfpoint{197.969498pt}{144.257416pt}}
\pgflineto{\pgfpoint{196.973511pt}{140.777435pt}}
\pgfpathclose
\pgfusepath{fill,stroke}
\pgfpathmoveto{\pgfpoint{198.965469pt}{197.820709pt}}
\pgflineto{\pgfpoint{197.969498pt}{144.257416pt}}
\pgflineto{\pgfpoint{197.969498pt}{140.777435pt}}
\pgfpathclose
\pgfusepath{fill,stroke}
\pgfpathmoveto{\pgfpoint{198.965469pt}{140.777435pt}}
\pgflineto{\pgfpoint{198.965469pt}{197.820709pt}}
\pgflineto{\pgfpoint{197.969498pt}{140.777435pt}}
\pgfpathclose
\pgfusepath{fill,stroke}
\pgfpathmoveto{\pgfpoint{199.961456pt}{140.777435pt}}
\pgflineto{\pgfpoint{198.965469pt}{197.820709pt}}
\pgflineto{\pgfpoint{198.965469pt}{140.777435pt}}
\pgfpathclose
\pgfusepath{fill,stroke}
\pgfpathmoveto{\pgfpoint{186.017685pt}{140.777435pt}}
\pgflineto{\pgfpoint{187.013672pt}{140.777435pt}}
\pgflineto{\pgfpoint{187.013672pt}{141.098480pt}}
\pgfpathclose
\pgfusepath{fill,stroke}
\pgfpathmoveto{\pgfpoint{188.009659pt}{140.777435pt}}
\pgflineto{\pgfpoint{187.013672pt}{141.098480pt}}
\pgflineto{\pgfpoint{187.013672pt}{140.777435pt}}
\pgfpathclose
\pgfusepath{fill,stroke}
\pgfpathmoveto{\pgfpoint{182.033752pt}{140.777435pt}}
\pgflineto{\pgfpoint{183.029724pt}{140.777435pt}}
\pgflineto{\pgfpoint{183.029724pt}{142.399628pt}}
\pgfpathclose
\pgfusepath{fill,stroke}
\pgfpathmoveto{\pgfpoint{184.025711pt}{140.777435pt}}
\pgflineto{\pgfpoint{183.029724pt}{142.399628pt}}
\pgflineto{\pgfpoint{183.029724pt}{140.777435pt}}
\pgfpathclose
\pgfusepath{fill,stroke}
\pgfpathmoveto{\pgfpoint{176.057846pt}{140.777435pt}}
\pgflineto{\pgfpoint{177.053818pt}{140.777435pt}}
\pgflineto{\pgfpoint{177.053818pt}{144.404053pt}}
\pgfpathclose
\pgfusepath{fill,stroke}
\pgfpathmoveto{\pgfpoint{178.049805pt}{144.353317pt}}
\pgflineto{\pgfpoint{177.053818pt}{144.404053pt}}
\pgflineto{\pgfpoint{177.053818pt}{140.777435pt}}
\pgfpathclose
\pgfusepath{fill,stroke}
\pgfpathmoveto{\pgfpoint{178.049805pt}{140.777435pt}}
\pgflineto{\pgfpoint{178.049805pt}{144.353317pt}}
\pgflineto{\pgfpoint{177.053818pt}{140.777435pt}}
\pgfpathclose
\pgfusepath{fill,stroke}
\pgfpathmoveto{\pgfpoint{179.045792pt}{140.963043pt}}
\pgflineto{\pgfpoint{178.049805pt}{144.353317pt}}
\pgflineto{\pgfpoint{178.049805pt}{140.777435pt}}
\pgfpathclose
\pgfusepath{fill,stroke}
\pgfpathmoveto{\pgfpoint{179.045792pt}{140.777435pt}}
\pgflineto{\pgfpoint{179.045792pt}{140.963043pt}}
\pgflineto{\pgfpoint{178.049805pt}{140.777435pt}}
\pgfpathclose
\pgfusepath{fill,stroke}
\pgfpathmoveto{\pgfpoint{180.041779pt}{140.930664pt}}
\pgflineto{\pgfpoint{179.045792pt}{140.963043pt}}
\pgflineto{\pgfpoint{179.045792pt}{140.777435pt}}
\pgfpathclose
\pgfusepath{fill,stroke}
\pgfpathmoveto{\pgfpoint{180.041779pt}{140.777435pt}}
\pgflineto{\pgfpoint{180.041779pt}{140.930664pt}}
\pgflineto{\pgfpoint{179.045792pt}{140.777435pt}}
\pgfpathclose
\pgfusepath{fill,stroke}
\pgfpathmoveto{\pgfpoint{181.037766pt}{141.619095pt}}
\pgflineto{\pgfpoint{180.041779pt}{140.930664pt}}
\pgflineto{\pgfpoint{180.041779pt}{140.777435pt}}
\pgfpathclose
\pgfusepath{fill,stroke}
\pgfpathmoveto{\pgfpoint{181.037766pt}{140.777435pt}}
\pgflineto{\pgfpoint{181.037766pt}{141.619095pt}}
\pgflineto{\pgfpoint{180.041779pt}{140.777435pt}}
\pgfpathclose
\pgfusepath{fill,stroke}
\pgfpathmoveto{\pgfpoint{182.033752pt}{140.777435pt}}
\pgflineto{\pgfpoint{181.037766pt}{141.619095pt}}
\pgflineto{\pgfpoint{181.037766pt}{140.777435pt}}
\pgfpathclose
\pgfusepath{fill,stroke}
\pgfpathmoveto{\pgfpoint{169.085953pt}{140.777435pt}}
\pgflineto{\pgfpoint{170.081940pt}{140.777435pt}}
\pgflineto{\pgfpoint{170.081940pt}{146.720169pt}}
\pgfpathclose
\pgfusepath{fill,stroke}
\pgfpathmoveto{\pgfpoint{171.077911pt}{145.686310pt}}
\pgflineto{\pgfpoint{170.081940pt}{146.720169pt}}
\pgflineto{\pgfpoint{170.081940pt}{140.777435pt}}
\pgfpathclose
\pgfusepath{fill,stroke}
\pgfpathmoveto{\pgfpoint{171.077911pt}{140.777435pt}}
\pgflineto{\pgfpoint{171.077911pt}{145.686310pt}}
\pgflineto{\pgfpoint{170.081940pt}{140.777435pt}}
\pgfpathclose
\pgfusepath{fill,stroke}
\pgfpathmoveto{\pgfpoint{172.073914pt}{149.736984pt}}
\pgflineto{\pgfpoint{171.077911pt}{145.686310pt}}
\pgflineto{\pgfpoint{171.077911pt}{140.777435pt}}
\pgfpathclose
\pgfusepath{fill,stroke}
\pgfpathmoveto{\pgfpoint{172.073914pt}{140.777435pt}}
\pgflineto{\pgfpoint{172.073914pt}{149.736984pt}}
\pgflineto{\pgfpoint{171.077911pt}{140.777435pt}}
\pgfpathclose
\pgfusepath{fill,stroke}
\pgfpathmoveto{\pgfpoint{173.069885pt}{148.531342pt}}
\pgflineto{\pgfpoint{172.073914pt}{149.736984pt}}
\pgflineto{\pgfpoint{172.073914pt}{140.777435pt}}
\pgfpathclose
\pgfusepath{fill,stroke}
\pgfpathmoveto{\pgfpoint{173.069885pt}{140.777435pt}}
\pgflineto{\pgfpoint{173.069885pt}{148.531342pt}}
\pgflineto{\pgfpoint{172.073914pt}{140.777435pt}}
\pgfpathclose
\pgfusepath{fill,stroke}
\pgfpathmoveto{\pgfpoint{174.065872pt}{141.515808pt}}
\pgflineto{\pgfpoint{173.069885pt}{148.531342pt}}
\pgflineto{\pgfpoint{173.069885pt}{140.777435pt}}
\pgfpathclose
\pgfusepath{fill,stroke}
\pgfpathmoveto{\pgfpoint{174.065872pt}{140.777435pt}}
\pgflineto{\pgfpoint{174.065872pt}{141.515808pt}}
\pgflineto{\pgfpoint{173.069885pt}{140.777435pt}}
\pgfpathclose
\pgfusepath{fill,stroke}
\pgfpathmoveto{\pgfpoint{175.061859pt}{141.323135pt}}
\pgflineto{\pgfpoint{174.065872pt}{141.515808pt}}
\pgflineto{\pgfpoint{174.065872pt}{140.777435pt}}
\pgfpathclose
\pgfusepath{fill,stroke}
\pgfpathmoveto{\pgfpoint{175.061859pt}{140.777435pt}}
\pgflineto{\pgfpoint{175.061859pt}{141.323135pt}}
\pgflineto{\pgfpoint{174.065872pt}{140.777435pt}}
\pgfpathclose
\pgfusepath{fill,stroke}
\pgfpathmoveto{\pgfpoint{176.057846pt}{140.777435pt}}
\pgflineto{\pgfpoint{175.061859pt}{141.323135pt}}
\pgflineto{\pgfpoint{175.061859pt}{140.777435pt}}
\pgfpathclose
\pgfusepath{fill,stroke}
\pgfpathmoveto{\pgfpoint{166.098007pt}{140.777435pt}}
\pgflineto{\pgfpoint{167.093994pt}{140.777435pt}}
\pgflineto{\pgfpoint{167.093994pt}{141.434799pt}}
\pgfpathclose
\pgfusepath{fill,stroke}
\pgfpathmoveto{\pgfpoint{167.417542pt}{205.642242pt}}
\pgflineto{\pgfpoint{167.093994pt}{141.434799pt}}
\pgflineto{\pgfpoint{167.093994pt}{140.777435pt}}
\pgfpathclose
\pgfusepath{fill,stroke}
\pgfpathmoveto{\pgfpoint{167.417542pt}{205.642242pt}}
\pgflineto{\pgfpoint{167.415314pt}{205.642242pt}}
\pgflineto{\pgfpoint{167.093994pt}{141.434799pt}}
\pgfpathclose
\pgfusepath{fill,stroke}
\pgfpathmoveto{\pgfpoint{168.089966pt}{140.777435pt}}
\pgflineto{\pgfpoint{167.417542pt}{205.642242pt}}
\pgflineto{\pgfpoint{167.093994pt}{140.777435pt}}
\pgfpathclose
\pgfusepath{fill,stroke}
\pgfpathmoveto{\pgfpoint{168.089966pt}{140.777435pt}}
\pgflineto{\pgfpoint{168.089966pt}{205.642242pt}}
\pgflineto{\pgfpoint{167.417542pt}{205.642242pt}}
\pgfpathclose
\pgfusepath{fill,stroke}
\pgfpathmoveto{\pgfpoint{169.085953pt}{140.777435pt}}
\pgflineto{\pgfpoint{168.089966pt}{205.642242pt}}
\pgflineto{\pgfpoint{168.089966pt}{140.777435pt}}
\pgfpathclose
\pgfusepath{fill,stroke}
\pgfpathmoveto{\pgfpoint{169.085953pt}{140.777435pt}}
\pgflineto{\pgfpoint{168.762405pt}{205.642242pt}}
\pgflineto{\pgfpoint{168.089966pt}{205.642242pt}}
\pgfpathclose
\pgfusepath{fill,stroke}
\pgfpathmoveto{\pgfpoint{163.110062pt}{140.777435pt}}
\pgflineto{\pgfpoint{164.106033pt}{140.777435pt}}
\pgflineto{\pgfpoint{164.106033pt}{194.127029pt}}
\pgfpathclose
\pgfusepath{fill,stroke}
\pgfpathmoveto{\pgfpoint{165.102020pt}{140.828079pt}}
\pgflineto{\pgfpoint{164.106033pt}{194.127029pt}}
\pgflineto{\pgfpoint{164.106033pt}{140.777435pt}}
\pgfpathclose
\pgfusepath{fill,stroke}
\pgfpathmoveto{\pgfpoint{165.102020pt}{140.777435pt}}
\pgflineto{\pgfpoint{165.102020pt}{140.828079pt}}
\pgflineto{\pgfpoint{164.106033pt}{140.777435pt}}
\pgfpathclose
\pgfusepath{fill,stroke}
\pgfpathmoveto{\pgfpoint{166.098007pt}{140.777435pt}}
\pgflineto{\pgfpoint{165.102020pt}{140.828079pt}}
\pgflineto{\pgfpoint{165.102020pt}{140.777435pt}}
\pgfpathclose
\pgfusepath{fill,stroke}
\pgfpathmoveto{\pgfpoint{159.126114pt}{140.777435pt}}
\pgflineto{\pgfpoint{160.122101pt}{140.777435pt}}
\pgflineto{\pgfpoint{160.122101pt}{158.193100pt}}
\pgfpathclose
\pgfusepath{fill,stroke}
\pgfpathmoveto{\pgfpoint{161.118088pt}{140.795273pt}}
\pgflineto{\pgfpoint{160.122101pt}{158.193100pt}}
\pgflineto{\pgfpoint{160.122101pt}{140.777435pt}}
\pgfpathclose
\pgfusepath{fill,stroke}
\pgfpathmoveto{\pgfpoint{161.118088pt}{140.777435pt}}
\pgflineto{\pgfpoint{161.118088pt}{140.795273pt}}
\pgflineto{\pgfpoint{160.122101pt}{140.777435pt}}
\pgfpathclose
\pgfusepath{fill,stroke}
\pgfpathmoveto{\pgfpoint{162.114075pt}{141.404877pt}}
\pgflineto{\pgfpoint{161.118088pt}{140.795273pt}}
\pgflineto{\pgfpoint{161.118088pt}{140.777435pt}}
\pgfpathclose
\pgfusepath{fill,stroke}
\pgfpathmoveto{\pgfpoint{162.114075pt}{140.777435pt}}
\pgflineto{\pgfpoint{162.114075pt}{141.404877pt}}
\pgflineto{\pgfpoint{161.118088pt}{140.777435pt}}
\pgfpathclose
\pgfusepath{fill,stroke}
\pgfpathmoveto{\pgfpoint{163.110062pt}{140.777435pt}}
\pgflineto{\pgfpoint{162.114075pt}{141.404877pt}}
\pgflineto{\pgfpoint{162.114075pt}{140.777435pt}}
\pgfpathclose
\pgfusepath{fill,stroke}
\pgfpathmoveto{\pgfpoint{156.138168pt}{140.777435pt}}
\pgflineto{\pgfpoint{157.134155pt}{140.777435pt}}
\pgflineto{\pgfpoint{157.134155pt}{140.926575pt}}
\pgfpathclose
\pgfusepath{fill,stroke}
\pgfpathmoveto{\pgfpoint{158.130127pt}{140.979492pt}}
\pgflineto{\pgfpoint{157.134155pt}{140.926575pt}}
\pgflineto{\pgfpoint{157.134155pt}{140.777435pt}}
\pgfpathclose
\pgfusepath{fill,stroke}
\pgfpathmoveto{\pgfpoint{158.130127pt}{140.777435pt}}
\pgflineto{\pgfpoint{158.130127pt}{140.979492pt}}
\pgflineto{\pgfpoint{157.134155pt}{140.777435pt}}
\pgfpathclose
\pgfusepath{fill,stroke}
\pgfpathmoveto{\pgfpoint{159.126114pt}{140.777435pt}}
\pgflineto{\pgfpoint{158.130127pt}{140.979492pt}}
\pgflineto{\pgfpoint{158.130127pt}{140.777435pt}}
\pgfpathclose
\pgfusepath{fill,stroke}
\pgfpathmoveto{\pgfpoint{154.146194pt}{140.777435pt}}
\pgflineto{\pgfpoint{155.142181pt}{140.777435pt}}
\pgflineto{\pgfpoint{155.142181pt}{195.815277pt}}
\pgfpathclose
\pgfusepath{fill,stroke}
\pgfpathmoveto{\pgfpoint{156.138168pt}{140.777435pt}}
\pgflineto{\pgfpoint{155.142181pt}{195.815277pt}}
\pgflineto{\pgfpoint{155.142181pt}{140.777435pt}}
\pgfpathclose
\pgfusepath{fill,stroke}
\pgfpathmoveto{\pgfpoint{150.162262pt}{140.777435pt}}
\pgflineto{\pgfpoint{151.158249pt}{140.777435pt}}
\pgflineto{\pgfpoint{151.158249pt}{141.561523pt}}
\pgfpathclose
\pgfusepath{fill,stroke}
\pgfpathmoveto{\pgfpoint{152.154221pt}{144.148651pt}}
\pgflineto{\pgfpoint{151.158249pt}{141.561523pt}}
\pgflineto{\pgfpoint{151.158249pt}{140.777435pt}}
\pgfpathclose
\pgfusepath{fill,stroke}
\pgfpathmoveto{\pgfpoint{152.154221pt}{140.777435pt}}
\pgflineto{\pgfpoint{152.154221pt}{144.148651pt}}
\pgflineto{\pgfpoint{151.158249pt}{140.777435pt}}
\pgfpathclose
\pgfusepath{fill,stroke}
\pgfpathmoveto{\pgfpoint{153.150208pt}{143.520859pt}}
\pgflineto{\pgfpoint{152.154221pt}{144.148651pt}}
\pgflineto{\pgfpoint{152.154221pt}{140.777435pt}}
\pgfpathclose
\pgfusepath{fill,stroke}
\pgfpathmoveto{\pgfpoint{153.150208pt}{140.777435pt}}
\pgflineto{\pgfpoint{153.150208pt}{143.520859pt}}
\pgflineto{\pgfpoint{152.154221pt}{140.777435pt}}
\pgfpathclose
\pgfusepath{fill,stroke}
\pgfpathmoveto{\pgfpoint{154.146194pt}{140.777435pt}}
\pgflineto{\pgfpoint{153.150208pt}{143.520859pt}}
\pgflineto{\pgfpoint{153.150208pt}{140.777435pt}}
\pgfpathclose
\pgfusepath{fill,stroke}
\pgfpathmoveto{\pgfpoint{144.186356pt}{140.777435pt}}
\pgflineto{\pgfpoint{145.182343pt}{140.777435pt}}
\pgflineto{\pgfpoint{145.182343pt}{141.507751pt}}
\pgfpathclose
\pgfusepath{fill,stroke}
\pgfpathmoveto{\pgfpoint{146.178314pt}{141.062988pt}}
\pgflineto{\pgfpoint{145.182343pt}{141.507751pt}}
\pgflineto{\pgfpoint{145.182343pt}{140.777435pt}}
\pgfpathclose
\pgfusepath{fill,stroke}
\pgfpathmoveto{\pgfpoint{146.178314pt}{140.777435pt}}
\pgflineto{\pgfpoint{146.178314pt}{141.062988pt}}
\pgflineto{\pgfpoint{145.182343pt}{140.777435pt}}
\pgfpathclose
\pgfusepath{fill,stroke}
\pgfpathmoveto{\pgfpoint{147.174316pt}{143.409790pt}}
\pgflineto{\pgfpoint{146.178314pt}{141.062988pt}}
\pgflineto{\pgfpoint{146.178314pt}{140.777435pt}}
\pgfpathclose
\pgfusepath{fill,stroke}
\pgfpathmoveto{\pgfpoint{147.174316pt}{140.777435pt}}
\pgflineto{\pgfpoint{147.174316pt}{143.409790pt}}
\pgflineto{\pgfpoint{146.178314pt}{140.777435pt}}
\pgfpathclose
\pgfusepath{fill,stroke}
\pgfpathmoveto{\pgfpoint{148.170288pt}{140.777435pt}}
\pgflineto{\pgfpoint{147.174316pt}{143.409790pt}}
\pgflineto{\pgfpoint{147.174316pt}{140.777435pt}}
\pgfpathclose
\pgfusepath{fill,stroke}
\pgfpathmoveto{\pgfpoint{142.194382pt}{140.777435pt}}
\pgflineto{\pgfpoint{143.190369pt}{140.777435pt}}
\pgflineto{\pgfpoint{143.190369pt}{141.972321pt}}
\pgfpathclose
\pgfusepath{fill,stroke}
\pgfpathmoveto{\pgfpoint{144.186356pt}{140.777435pt}}
\pgflineto{\pgfpoint{143.190369pt}{141.972321pt}}
\pgflineto{\pgfpoint{143.190369pt}{140.777435pt}}
\pgfpathclose
\pgfusepath{fill,stroke}
\pgfpathmoveto{\pgfpoint{136.218475pt}{140.777435pt}}
\pgflineto{\pgfpoint{137.214478pt}{140.777435pt}}
\pgflineto{\pgfpoint{137.214478pt}{144.493912pt}}
\pgfpathclose
\pgfusepath{fill,stroke}
\pgfpathmoveto{\pgfpoint{138.210449pt}{140.824677pt}}
\pgflineto{\pgfpoint{137.214478pt}{144.493912pt}}
\pgflineto{\pgfpoint{137.214478pt}{140.777435pt}}
\pgfpathclose
\pgfusepath{fill,stroke}
\pgfpathmoveto{\pgfpoint{138.210449pt}{140.777435pt}}
\pgflineto{\pgfpoint{138.210449pt}{140.824677pt}}
\pgflineto{\pgfpoint{137.214478pt}{140.777435pt}}
\pgfpathclose
\pgfusepath{fill,stroke}
\pgfpathmoveto{\pgfpoint{139.206436pt}{155.555008pt}}
\pgflineto{\pgfpoint{138.210449pt}{140.824677pt}}
\pgflineto{\pgfpoint{138.210449pt}{140.777435pt}}
\pgfpathclose
\pgfusepath{fill,stroke}
\pgfpathmoveto{\pgfpoint{139.206436pt}{140.777435pt}}
\pgflineto{\pgfpoint{139.206436pt}{155.555008pt}}
\pgflineto{\pgfpoint{138.210449pt}{140.777435pt}}
\pgfpathclose
\pgfusepath{fill,stroke}
\pgfpathmoveto{\pgfpoint{140.202423pt}{141.880554pt}}
\pgflineto{\pgfpoint{139.206436pt}{155.555008pt}}
\pgflineto{\pgfpoint{139.206436pt}{140.777435pt}}
\pgfpathclose
\pgfusepath{fill,stroke}
\pgfpathmoveto{\pgfpoint{140.202423pt}{140.777435pt}}
\pgflineto{\pgfpoint{140.202423pt}{141.880554pt}}
\pgflineto{\pgfpoint{139.206436pt}{140.777435pt}}
\pgfpathclose
\pgfusepath{fill,stroke}
\pgfpathmoveto{\pgfpoint{141.198410pt}{141.096649pt}}
\pgflineto{\pgfpoint{140.202423pt}{141.880554pt}}
\pgflineto{\pgfpoint{140.202423pt}{140.777435pt}}
\pgfpathclose
\pgfusepath{fill,stroke}
\pgfpathmoveto{\pgfpoint{141.198410pt}{140.777435pt}}
\pgflineto{\pgfpoint{141.198410pt}{141.096649pt}}
\pgflineto{\pgfpoint{140.202423pt}{140.777435pt}}
\pgfpathclose
\pgfusepath{fill,stroke}
\pgfpathmoveto{\pgfpoint{142.194382pt}{140.777435pt}}
\pgflineto{\pgfpoint{141.198410pt}{141.096649pt}}
\pgflineto{\pgfpoint{141.198410pt}{140.777435pt}}
\pgfpathclose
\pgfusepath{fill,stroke}
\pgfpathmoveto{\pgfpoint{134.226517pt}{140.777435pt}}
\pgflineto{\pgfpoint{135.222504pt}{140.777435pt}}
\pgflineto{\pgfpoint{135.222504pt}{141.478424pt}}
\pgfpathclose
\pgfusepath{fill,stroke}
\pgfpathmoveto{\pgfpoint{136.218475pt}{140.777435pt}}
\pgflineto{\pgfpoint{135.222504pt}{141.478424pt}}
\pgflineto{\pgfpoint{135.222504pt}{140.777435pt}}
\pgfpathclose
\pgfusepath{fill,stroke}
\pgfpathmoveto{\pgfpoint{128.250610pt}{140.777435pt}}
\pgflineto{\pgfpoint{129.246597pt}{140.777435pt}}
\pgflineto{\pgfpoint{129.246597pt}{146.889862pt}}
\pgfpathclose
\pgfusepath{fill,stroke}
\pgfpathmoveto{\pgfpoint{130.242584pt}{141.305695pt}}
\pgflineto{\pgfpoint{129.246597pt}{146.889862pt}}
\pgflineto{\pgfpoint{129.246597pt}{140.777435pt}}
\pgfpathclose
\pgfusepath{fill,stroke}
\pgfpathmoveto{\pgfpoint{130.242584pt}{140.777435pt}}
\pgflineto{\pgfpoint{130.242584pt}{141.305695pt}}
\pgflineto{\pgfpoint{129.246597pt}{140.777435pt}}
\pgfpathclose
\pgfusepath{fill,stroke}
\pgfpathmoveto{\pgfpoint{131.238571pt}{142.924377pt}}
\pgflineto{\pgfpoint{130.242584pt}{141.305695pt}}
\pgflineto{\pgfpoint{130.242584pt}{140.777435pt}}
\pgfpathclose
\pgfusepath{fill,stroke}
\pgfpathmoveto{\pgfpoint{131.238571pt}{140.777435pt}}
\pgflineto{\pgfpoint{131.238571pt}{142.924377pt}}
\pgflineto{\pgfpoint{130.242584pt}{140.777435pt}}
\pgfpathclose
\pgfusepath{fill,stroke}
\pgfpathmoveto{\pgfpoint{132.234558pt}{140.784973pt}}
\pgflineto{\pgfpoint{131.238571pt}{142.924377pt}}
\pgflineto{\pgfpoint{131.238571pt}{140.777435pt}}
\pgfpathclose
\pgfusepath{fill,stroke}
\pgfpathmoveto{\pgfpoint{132.234558pt}{140.777435pt}}
\pgflineto{\pgfpoint{132.234558pt}{140.784973pt}}
\pgflineto{\pgfpoint{131.238571pt}{140.777435pt}}
\pgfpathclose
\pgfusepath{fill,stroke}
\pgfpathmoveto{\pgfpoint{133.230530pt}{140.777435pt}}
\pgflineto{\pgfpoint{132.234558pt}{140.784973pt}}
\pgflineto{\pgfpoint{132.234558pt}{140.777435pt}}
\pgfpathclose
\pgfusepath{fill,stroke}
\pgfpathmoveto{\pgfpoint{125.262665pt}{140.777435pt}}
\pgflineto{\pgfpoint{126.258652pt}{140.777435pt}}
\pgflineto{\pgfpoint{126.258652pt}{141.187393pt}}
\pgfpathclose
\pgfusepath{fill,stroke}
\pgfpathmoveto{\pgfpoint{127.254631pt}{158.624695pt}}
\pgflineto{\pgfpoint{126.258652pt}{141.187393pt}}
\pgflineto{\pgfpoint{126.258652pt}{140.777435pt}}
\pgfpathclose
\pgfusepath{fill,stroke}
\pgfpathmoveto{\pgfpoint{127.254631pt}{140.777435pt}}
\pgflineto{\pgfpoint{127.254631pt}{158.624695pt}}
\pgflineto{\pgfpoint{126.258652pt}{140.777435pt}}
\pgfpathclose
\pgfusepath{fill,stroke}
\pgfpathmoveto{\pgfpoint{128.250610pt}{140.777435pt}}
\pgflineto{\pgfpoint{127.254631pt}{158.624695pt}}
\pgflineto{\pgfpoint{127.254631pt}{140.777435pt}}
\pgfpathclose
\pgfusepath{fill,stroke}
\pgfpathmoveto{\pgfpoint{122.274712pt}{140.777435pt}}
\pgflineto{\pgfpoint{123.270691pt}{140.777435pt}}
\pgflineto{\pgfpoint{123.270691pt}{141.045074pt}}
\pgfpathclose
\pgfusepath{fill,stroke}
\pgfpathmoveto{\pgfpoint{124.266678pt}{140.777435pt}}
\pgflineto{\pgfpoint{123.270691pt}{141.045074pt}}
\pgflineto{\pgfpoint{123.270691pt}{140.777435pt}}
\pgfpathclose
\pgfusepath{fill,stroke}
\pgfpathmoveto{\pgfpoint{120.282745pt}{140.777435pt}}
\pgflineto{\pgfpoint{121.278725pt}{140.777435pt}}
\pgflineto{\pgfpoint{121.278725pt}{141.992584pt}}
\pgfpathclose
\pgfusepath{fill,stroke}
\pgfpathmoveto{\pgfpoint{122.274712pt}{140.777435pt}}
\pgflineto{\pgfpoint{121.278725pt}{141.992584pt}}
\pgflineto{\pgfpoint{121.278725pt}{140.777435pt}}
\pgfpathclose
\pgfusepath{fill,stroke}
\pgfpathmoveto{\pgfpoint{115.302826pt}{140.777435pt}}
\pgflineto{\pgfpoint{116.298813pt}{140.777435pt}}
\pgflineto{\pgfpoint{116.298813pt}{141.607086pt}}
\pgfpathclose
\pgfusepath{fill,stroke}
\pgfpathmoveto{\pgfpoint{117.294792pt}{143.559708pt}}
\pgflineto{\pgfpoint{116.298813pt}{141.607086pt}}
\pgflineto{\pgfpoint{116.298813pt}{140.777435pt}}
\pgfpathclose
\pgfusepath{fill,stroke}
\pgfpathmoveto{\pgfpoint{117.294792pt}{140.777435pt}}
\pgflineto{\pgfpoint{117.294792pt}{143.559708pt}}
\pgflineto{\pgfpoint{116.298813pt}{140.777435pt}}
\pgfpathclose
\pgfusepath{fill,stroke}
\pgfpathmoveto{\pgfpoint{118.290779pt}{140.848618pt}}
\pgflineto{\pgfpoint{117.294792pt}{143.559708pt}}
\pgflineto{\pgfpoint{117.294792pt}{140.777435pt}}
\pgfpathclose
\pgfusepath{fill,stroke}
\pgfpathmoveto{\pgfpoint{118.290779pt}{140.777435pt}}
\pgflineto{\pgfpoint{118.290779pt}{140.848618pt}}
\pgflineto{\pgfpoint{117.294792pt}{140.777435pt}}
\pgfpathclose
\pgfusepath{fill,stroke}
\pgfpathmoveto{\pgfpoint{119.286758pt}{140.997116pt}}
\pgflineto{\pgfpoint{118.290779pt}{140.848618pt}}
\pgflineto{\pgfpoint{118.290779pt}{140.777435pt}}
\pgfpathclose
\pgfusepath{fill,stroke}
\pgfpathmoveto{\pgfpoint{119.286758pt}{140.777435pt}}
\pgflineto{\pgfpoint{119.286758pt}{140.997116pt}}
\pgflineto{\pgfpoint{118.290779pt}{140.777435pt}}
\pgfpathclose
\pgfusepath{fill,stroke}
\pgfpathmoveto{\pgfpoint{120.282745pt}{140.777435pt}}
\pgflineto{\pgfpoint{119.286758pt}{140.997116pt}}
\pgflineto{\pgfpoint{119.286758pt}{140.777435pt}}
\pgfpathclose
\pgfusepath{fill,stroke}
\pgfpathmoveto{\pgfpoint{111.318893pt}{140.777435pt}}
\pgflineto{\pgfpoint{112.314873pt}{140.777435pt}}
\pgflineto{\pgfpoint{112.314873pt}{142.841614pt}}
\pgfpathclose
\pgfusepath{fill,stroke}
\pgfpathmoveto{\pgfpoint{113.310852pt}{141.335541pt}}
\pgflineto{\pgfpoint{112.314873pt}{142.841614pt}}
\pgflineto{\pgfpoint{112.314873pt}{140.777435pt}}
\pgfpathclose
\pgfusepath{fill,stroke}
\pgfpathmoveto{\pgfpoint{113.310852pt}{140.777435pt}}
\pgflineto{\pgfpoint{113.310852pt}{141.335541pt}}
\pgflineto{\pgfpoint{112.314873pt}{140.777435pt}}
\pgfpathclose
\pgfusepath{fill,stroke}
\pgfpathmoveto{\pgfpoint{114.306839pt}{143.259003pt}}
\pgflineto{\pgfpoint{113.310852pt}{141.335541pt}}
\pgflineto{\pgfpoint{113.310852pt}{140.777435pt}}
\pgfpathclose
\pgfusepath{fill,stroke}
\pgfpathmoveto{\pgfpoint{114.306839pt}{140.777435pt}}
\pgflineto{\pgfpoint{114.306839pt}{143.259003pt}}
\pgflineto{\pgfpoint{113.310852pt}{140.777435pt}}
\pgfpathclose
\pgfusepath{fill,stroke}
\pgfpathmoveto{\pgfpoint{115.302826pt}{140.777435pt}}
\pgflineto{\pgfpoint{114.306839pt}{143.259003pt}}
\pgflineto{\pgfpoint{114.306839pt}{140.777435pt}}
\pgfpathclose
\pgfusepath{fill,stroke}
\pgfpathmoveto{\pgfpoint{107.334953pt}{140.777435pt}}
\pgflineto{\pgfpoint{108.330933pt}{140.777435pt}}
\pgflineto{\pgfpoint{108.330933pt}{156.427597pt}}
\pgfpathclose
\pgfusepath{fill,stroke}
\pgfpathmoveto{\pgfpoint{109.326920pt}{140.854156pt}}
\pgflineto{\pgfpoint{108.330933pt}{156.427597pt}}
\pgflineto{\pgfpoint{108.330933pt}{140.777435pt}}
\pgfpathclose
\pgfusepath{fill,stroke}
\pgfpathmoveto{\pgfpoint{109.326920pt}{140.777435pt}}
\pgflineto{\pgfpoint{109.326920pt}{140.854156pt}}
\pgflineto{\pgfpoint{108.330933pt}{140.777435pt}}
\pgfpathclose
\pgfusepath{fill,stroke}
\pgfpathmoveto{\pgfpoint{110.322906pt}{143.840607pt}}
\pgflineto{\pgfpoint{109.326920pt}{140.854156pt}}
\pgflineto{\pgfpoint{109.326920pt}{140.777435pt}}
\pgfpathclose
\pgfusepath{fill,stroke}
\pgfpathmoveto{\pgfpoint{110.322906pt}{140.777435pt}}
\pgflineto{\pgfpoint{110.322906pt}{143.840607pt}}
\pgflineto{\pgfpoint{109.326920pt}{140.777435pt}}
\pgfpathclose
\pgfusepath{fill,stroke}
\pgfpathmoveto{\pgfpoint{111.318893pt}{140.777435pt}}
\pgflineto{\pgfpoint{110.322906pt}{143.840607pt}}
\pgflineto{\pgfpoint{110.322906pt}{140.777435pt}}
\pgfpathclose
\pgfusepath{fill,stroke}
\pgfpathmoveto{\pgfpoint{103.351013pt}{140.777435pt}}
\pgflineto{\pgfpoint{104.347000pt}{140.777435pt}}
\pgflineto{\pgfpoint{104.347000pt}{152.852020pt}}
\pgfpathclose
\pgfusepath{fill,stroke}
\pgfpathmoveto{\pgfpoint{105.342987pt}{140.777435pt}}
\pgflineto{\pgfpoint{104.347000pt}{152.852020pt}}
\pgflineto{\pgfpoint{104.347000pt}{140.777435pt}}
\pgfpathclose
\pgfusepath{fill,stroke}
\pgfpathmoveto{\pgfpoint{94.387161pt}{140.777435pt}}
\pgflineto{\pgfpoint{95.383141pt}{140.777435pt}}
\pgflineto{\pgfpoint{95.383141pt}{141.040619pt}}
\pgfpathclose
\pgfusepath{fill,stroke}
\pgfpathmoveto{\pgfpoint{96.379128pt}{141.052612pt}}
\pgflineto{\pgfpoint{95.383141pt}{141.040619pt}}
\pgflineto{\pgfpoint{95.383141pt}{140.777435pt}}
\pgfpathclose
\pgfusepath{fill,stroke}
\pgfpathmoveto{\pgfpoint{96.379128pt}{140.777435pt}}
\pgflineto{\pgfpoint{96.379128pt}{141.052612pt}}
\pgflineto{\pgfpoint{95.383141pt}{140.777435pt}}
\pgfpathclose
\pgfusepath{fill,stroke}
\pgfpathmoveto{\pgfpoint{97.375107pt}{142.021271pt}}
\pgflineto{\pgfpoint{96.379128pt}{141.052612pt}}
\pgflineto{\pgfpoint{96.379128pt}{140.777435pt}}
\pgfpathclose
\pgfusepath{fill,stroke}
\pgfpathmoveto{\pgfpoint{97.375107pt}{140.777435pt}}
\pgflineto{\pgfpoint{97.375107pt}{142.021271pt}}
\pgflineto{\pgfpoint{96.379128pt}{140.777435pt}}
\pgfpathclose
\pgfusepath{fill,stroke}
\pgfpathmoveto{\pgfpoint{98.371094pt}{187.727631pt}}
\pgflineto{\pgfpoint{97.375107pt}{142.021271pt}}
\pgflineto{\pgfpoint{97.375107pt}{140.777435pt}}
\pgfpathclose
\pgfusepath{fill,stroke}
\pgfpathmoveto{\pgfpoint{98.371094pt}{140.777435pt}}
\pgflineto{\pgfpoint{98.371094pt}{187.727631pt}}
\pgflineto{\pgfpoint{97.375107pt}{140.777435pt}}
\pgfpathclose
\pgfusepath{fill,stroke}
\pgfpathmoveto{\pgfpoint{99.367081pt}{143.174622pt}}
\pgflineto{\pgfpoint{98.371094pt}{187.727631pt}}
\pgflineto{\pgfpoint{98.371094pt}{140.777435pt}}
\pgfpathclose
\pgfusepath{fill,stroke}
\pgfpathmoveto{\pgfpoint{99.367081pt}{140.777435pt}}
\pgflineto{\pgfpoint{99.367081pt}{143.174622pt}}
\pgflineto{\pgfpoint{98.371094pt}{140.777435pt}}
\pgfpathclose
\pgfusepath{fill,stroke}
\pgfpathmoveto{\pgfpoint{100.363068pt}{142.326782pt}}
\pgflineto{\pgfpoint{99.367081pt}{143.174622pt}}
\pgflineto{\pgfpoint{99.367081pt}{140.777435pt}}
\pgfpathclose
\pgfusepath{fill,stroke}
\pgfpathmoveto{\pgfpoint{100.363068pt}{140.777435pt}}
\pgflineto{\pgfpoint{100.363068pt}{142.326782pt}}
\pgflineto{\pgfpoint{99.367081pt}{140.777435pt}}
\pgfpathclose
\pgfusepath{fill,stroke}
\pgfpathmoveto{\pgfpoint{101.359047pt}{141.473938pt}}
\pgflineto{\pgfpoint{100.363068pt}{142.326782pt}}
\pgflineto{\pgfpoint{100.363068pt}{140.777435pt}}
\pgfpathclose
\pgfusepath{fill,stroke}
\pgfpathmoveto{\pgfpoint{101.359047pt}{140.777435pt}}
\pgflineto{\pgfpoint{101.359047pt}{141.473938pt}}
\pgflineto{\pgfpoint{100.363068pt}{140.777435pt}}
\pgfpathclose
\pgfusepath{fill,stroke}
\pgfpathmoveto{\pgfpoint{102.355034pt}{153.173050pt}}
\pgflineto{\pgfpoint{101.359047pt}{141.473938pt}}
\pgflineto{\pgfpoint{101.359047pt}{140.777435pt}}
\pgfpathclose
\pgfusepath{fill,stroke}
\pgfpathmoveto{\pgfpoint{102.355034pt}{140.777435pt}}
\pgflineto{\pgfpoint{102.355034pt}{153.173050pt}}
\pgflineto{\pgfpoint{101.359047pt}{140.777435pt}}
\pgfpathclose
\pgfusepath{fill,stroke}
\pgfpathmoveto{\pgfpoint{103.351013pt}{140.777435pt}}
\pgflineto{\pgfpoint{102.355034pt}{153.173050pt}}
\pgflineto{\pgfpoint{102.355034pt}{140.777435pt}}
\pgfpathclose
\pgfusepath{fill,stroke}
\pgfpathmoveto{\pgfpoint{91.399208pt}{140.777435pt}}
\pgflineto{\pgfpoint{92.395187pt}{140.777435pt}}
\pgflineto{\pgfpoint{92.395187pt}{140.789795pt}}
\pgfpathclose
\pgfusepath{fill,stroke}
\pgfpathmoveto{\pgfpoint{93.391174pt}{141.209564pt}}
\pgflineto{\pgfpoint{92.395187pt}{140.789795pt}}
\pgflineto{\pgfpoint{92.395187pt}{140.777435pt}}
\pgfpathclose
\pgfusepath{fill,stroke}
\pgfpathmoveto{\pgfpoint{93.391174pt}{140.777435pt}}
\pgflineto{\pgfpoint{93.391174pt}{141.209564pt}}
\pgflineto{\pgfpoint{92.395187pt}{140.777435pt}}
\pgfpathclose
\pgfusepath{fill,stroke}
\pgfpathmoveto{\pgfpoint{94.387161pt}{140.777435pt}}
\pgflineto{\pgfpoint{93.391174pt}{141.209564pt}}
\pgflineto{\pgfpoint{93.391174pt}{140.777435pt}}
\pgfpathclose
\pgfusepath{fill,stroke}
\pgfpathmoveto{\pgfpoint{86.419289pt}{140.777435pt}}
\pgflineto{\pgfpoint{87.415276pt}{140.777435pt}}
\pgflineto{\pgfpoint{87.415276pt}{152.008636pt}}
\pgfpathclose
\pgfusepath{fill,stroke}
\pgfpathmoveto{\pgfpoint{88.411255pt}{163.097900pt}}
\pgflineto{\pgfpoint{87.415276pt}{152.008636pt}}
\pgflineto{\pgfpoint{87.415276pt}{140.777435pt}}
\pgfpathclose
\pgfusepath{fill,stroke}
\pgfpathmoveto{\pgfpoint{88.411255pt}{140.777435pt}}
\pgflineto{\pgfpoint{88.411255pt}{163.097900pt}}
\pgflineto{\pgfpoint{87.415276pt}{140.777435pt}}
\pgfpathclose
\pgfusepath{fill,stroke}
\pgfpathmoveto{\pgfpoint{89.407242pt}{147.608887pt}}
\pgflineto{\pgfpoint{88.411255pt}{163.097900pt}}
\pgflineto{\pgfpoint{88.411255pt}{140.777435pt}}
\pgfpathclose
\pgfusepath{fill,stroke}
\pgfpathmoveto{\pgfpoint{89.407242pt}{140.777435pt}}
\pgflineto{\pgfpoint{89.407242pt}{147.608887pt}}
\pgflineto{\pgfpoint{88.411255pt}{140.777435pt}}
\pgfpathclose
\pgfusepath{fill,stroke}
\pgfpathmoveto{\pgfpoint{90.403221pt}{189.355881pt}}
\pgflineto{\pgfpoint{89.407242pt}{147.608887pt}}
\pgflineto{\pgfpoint{89.407242pt}{140.777435pt}}
\pgfpathclose
\pgfusepath{fill,stroke}
\pgfpathmoveto{\pgfpoint{90.403221pt}{140.777435pt}}
\pgflineto{\pgfpoint{90.403221pt}{189.355881pt}}
\pgflineto{\pgfpoint{89.407242pt}{140.777435pt}}
\pgfpathclose
\pgfusepath{fill,stroke}
\pgfpathmoveto{\pgfpoint{91.399208pt}{140.777435pt}}
\pgflineto{\pgfpoint{90.403221pt}{189.355881pt}}
\pgflineto{\pgfpoint{90.403221pt}{140.777435pt}}
\pgfpathclose
\pgfusepath{fill,stroke}
\pgfpathmoveto{\pgfpoint{78.451424pt}{140.777435pt}}
\pgflineto{\pgfpoint{79.447403pt}{140.777435pt}}
\pgflineto{\pgfpoint{79.447403pt}{164.721649pt}}
\pgfpathclose
\pgfusepath{fill,stroke}
\pgfpathmoveto{\pgfpoint{80.443390pt}{141.021835pt}}
\pgflineto{\pgfpoint{79.447403pt}{164.721649pt}}
\pgflineto{\pgfpoint{79.447403pt}{140.777435pt}}
\pgfpathclose
\pgfusepath{fill,stroke}
\pgfpathmoveto{\pgfpoint{80.443390pt}{140.777435pt}}
\pgflineto{\pgfpoint{80.443390pt}{141.021835pt}}
\pgflineto{\pgfpoint{79.447403pt}{140.777435pt}}
\pgfpathclose
\pgfusepath{fill,stroke}
\pgfpathmoveto{\pgfpoint{81.439369pt}{141.797729pt}}
\pgflineto{\pgfpoint{80.443390pt}{141.021835pt}}
\pgflineto{\pgfpoint{80.443390pt}{140.777435pt}}
\pgfpathclose
\pgfusepath{fill,stroke}
\pgfpathmoveto{\pgfpoint{81.439369pt}{140.777435pt}}
\pgflineto{\pgfpoint{81.439369pt}{141.797729pt}}
\pgflineto{\pgfpoint{80.443390pt}{140.777435pt}}
\pgfpathclose
\pgfusepath{fill,stroke}
\pgfpathmoveto{\pgfpoint{82.435356pt}{141.784378pt}}
\pgflineto{\pgfpoint{81.439369pt}{141.797729pt}}
\pgflineto{\pgfpoint{81.439369pt}{140.777435pt}}
\pgfpathclose
\pgfusepath{fill,stroke}
\pgfpathmoveto{\pgfpoint{82.435356pt}{140.777435pt}}
\pgflineto{\pgfpoint{82.435356pt}{141.784378pt}}
\pgflineto{\pgfpoint{81.439369pt}{140.777435pt}}
\pgfpathclose
\pgfusepath{fill,stroke}
\pgfpathmoveto{\pgfpoint{83.431335pt}{147.373245pt}}
\pgflineto{\pgfpoint{82.435356pt}{141.784378pt}}
\pgflineto{\pgfpoint{82.435356pt}{140.777435pt}}
\pgfpathclose
\pgfusepath{fill,stroke}
\pgfpathmoveto{\pgfpoint{83.431335pt}{140.777435pt}}
\pgflineto{\pgfpoint{83.431335pt}{147.373245pt}}
\pgflineto{\pgfpoint{82.435356pt}{140.777435pt}}
\pgfpathclose
\pgfusepath{fill,stroke}
\pgfpathmoveto{\pgfpoint{84.427322pt}{140.853500pt}}
\pgflineto{\pgfpoint{83.431335pt}{147.373245pt}}
\pgflineto{\pgfpoint{83.431335pt}{140.777435pt}}
\pgfpathclose
\pgfusepath{fill,stroke}
\pgfpathmoveto{\pgfpoint{84.427322pt}{140.777435pt}}
\pgflineto{\pgfpoint{84.427322pt}{140.853500pt}}
\pgflineto{\pgfpoint{83.431335pt}{140.777435pt}}
\pgfpathclose
\pgfusepath{fill,stroke}
\pgfpathmoveto{\pgfpoint{85.423309pt}{140.987396pt}}
\pgflineto{\pgfpoint{84.427322pt}{140.853500pt}}
\pgflineto{\pgfpoint{84.427322pt}{140.777435pt}}
\pgfpathclose
\pgfusepath{fill,stroke}
\pgfpathmoveto{\pgfpoint{85.423309pt}{140.777435pt}}
\pgflineto{\pgfpoint{85.423309pt}{140.987396pt}}
\pgflineto{\pgfpoint{84.427322pt}{140.777435pt}}
\pgfpathclose
\pgfusepath{fill,stroke}
\pgfpathmoveto{\pgfpoint{86.419289pt}{140.777435pt}}
\pgflineto{\pgfpoint{85.423309pt}{140.987396pt}}
\pgflineto{\pgfpoint{85.423309pt}{140.777435pt}}
\pgfpathclose
\pgfusepath{fill,stroke}
\pgfpathmoveto{\pgfpoint{73.471497pt}{140.777435pt}}
\pgflineto{\pgfpoint{74.467484pt}{140.777435pt}}
\pgflineto{\pgfpoint{74.467484pt}{141.606003pt}}
\pgfpathclose
\pgfusepath{fill,stroke}
\pgfpathmoveto{\pgfpoint{75.463470pt}{145.536850pt}}
\pgflineto{\pgfpoint{74.467484pt}{141.606003pt}}
\pgflineto{\pgfpoint{74.467484pt}{140.777435pt}}
\pgfpathclose
\pgfusepath{fill,stroke}
\pgfpathmoveto{\pgfpoint{75.463470pt}{140.777435pt}}
\pgflineto{\pgfpoint{75.463470pt}{145.536850pt}}
\pgflineto{\pgfpoint{74.467484pt}{140.777435pt}}
\pgfpathclose
\pgfusepath{fill,stroke}
\pgfpathmoveto{\pgfpoint{76.459442pt}{143.382019pt}}
\pgflineto{\pgfpoint{75.463470pt}{145.536850pt}}
\pgflineto{\pgfpoint{75.463470pt}{140.777435pt}}
\pgfpathclose
\pgfusepath{fill,stroke}
\pgfpathmoveto{\pgfpoint{76.459442pt}{140.777435pt}}
\pgflineto{\pgfpoint{76.459442pt}{143.382019pt}}
\pgflineto{\pgfpoint{75.463470pt}{140.777435pt}}
\pgfpathclose
\pgfusepath{fill,stroke}
\pgfpathmoveto{\pgfpoint{77.455437pt}{142.594910pt}}
\pgflineto{\pgfpoint{76.459442pt}{143.382019pt}}
\pgflineto{\pgfpoint{76.459442pt}{140.777435pt}}
\pgfpathclose
\pgfusepath{fill,stroke}
\pgfpathmoveto{\pgfpoint{77.455437pt}{140.777435pt}}
\pgflineto{\pgfpoint{77.455437pt}{142.594910pt}}
\pgflineto{\pgfpoint{76.459442pt}{140.777435pt}}
\pgfpathclose
\pgfusepath{fill,stroke}
\pgfpathmoveto{\pgfpoint{78.451424pt}{140.777435pt}}
\pgflineto{\pgfpoint{77.455437pt}{142.594910pt}}
\pgflineto{\pgfpoint{77.455437pt}{140.777435pt}}
\pgfpathclose
\pgfusepath{fill,stroke}
\pgfpathmoveto{\pgfpoint{68.491577pt}{140.777435pt}}
\pgflineto{\pgfpoint{69.487564pt}{140.777435pt}}
\pgflineto{\pgfpoint{69.487564pt}{149.515793pt}}
\pgfpathclose
\pgfusepath{fill,stroke}
\pgfpathmoveto{\pgfpoint{70.483551pt}{141.832703pt}}
\pgflineto{\pgfpoint{69.487564pt}{149.515793pt}}
\pgflineto{\pgfpoint{69.487564pt}{140.777435pt}}
\pgfpathclose
\pgfusepath{fill,stroke}
\pgfpathmoveto{\pgfpoint{70.483551pt}{140.777435pt}}
\pgflineto{\pgfpoint{70.483551pt}{141.832703pt}}
\pgflineto{\pgfpoint{69.487564pt}{140.777435pt}}
\pgfpathclose
\pgfusepath{fill,stroke}
\pgfpathmoveto{\pgfpoint{71.479530pt}{190.848999pt}}
\pgflineto{\pgfpoint{70.483551pt}{141.832703pt}}
\pgflineto{\pgfpoint{70.483551pt}{140.777435pt}}
\pgfpathclose
\pgfusepath{fill,stroke}
\pgfpathmoveto{\pgfpoint{71.479530pt}{140.777435pt}}
\pgflineto{\pgfpoint{71.479530pt}{190.848999pt}}
\pgflineto{\pgfpoint{70.483551pt}{140.777435pt}}
\pgfpathclose
\pgfusepath{fill,stroke}
\pgfpathmoveto{\pgfpoint{72.475510pt}{148.664902pt}}
\pgflineto{\pgfpoint{71.479530pt}{190.848999pt}}
\pgflineto{\pgfpoint{71.479530pt}{140.777435pt}}
\pgfpathclose
\pgfusepath{fill,stroke}
\pgfpathmoveto{\pgfpoint{72.475510pt}{140.777435pt}}
\pgflineto{\pgfpoint{72.475510pt}{148.664902pt}}
\pgflineto{\pgfpoint{71.479530pt}{140.777435pt}}
\pgfpathclose
\pgfusepath{fill,stroke}
\pgfpathmoveto{\pgfpoint{73.471497pt}{140.777435pt}}
\pgflineto{\pgfpoint{72.475510pt}{148.664902pt}}
\pgflineto{\pgfpoint{72.475510pt}{140.777435pt}}
\pgfpathclose
\pgfusepath{fill,stroke}
\pgfpathmoveto{\pgfpoint{50.563873pt}{140.777435pt}}
\pgflineto{\pgfpoint{51.559845pt}{140.777435pt}}
\pgflineto{\pgfpoint{51.559845pt}{146.765289pt}}
\pgfpathclose
\pgfusepath{fill,stroke}
\pgfpathmoveto{\pgfpoint{52.555840pt}{142.530624pt}}
\pgflineto{\pgfpoint{51.559845pt}{146.765289pt}}
\pgflineto{\pgfpoint{51.559845pt}{140.777435pt}}
\pgfpathclose
\pgfusepath{fill,stroke}
\pgfpathmoveto{\pgfpoint{52.555840pt}{140.777435pt}}
\pgflineto{\pgfpoint{52.555840pt}{142.530624pt}}
\pgflineto{\pgfpoint{51.559845pt}{140.777435pt}}
\pgfpathclose
\pgfusepath{fill,stroke}
\pgfpathmoveto{\pgfpoint{53.551819pt}{146.137848pt}}
\pgflineto{\pgfpoint{52.555840pt}{142.530624pt}}
\pgflineto{\pgfpoint{52.555840pt}{140.777435pt}}
\pgfpathclose
\pgfusepath{fill,stroke}
\pgfpathmoveto{\pgfpoint{53.551819pt}{140.777435pt}}
\pgflineto{\pgfpoint{53.551819pt}{146.137848pt}}
\pgflineto{\pgfpoint{52.555840pt}{140.777435pt}}
\pgfpathclose
\pgfusepath{fill,stroke}
\pgfpathmoveto{\pgfpoint{54.547806pt}{142.199554pt}}
\pgflineto{\pgfpoint{53.551819pt}{146.137848pt}}
\pgflineto{\pgfpoint{53.551819pt}{140.777435pt}}
\pgfpathclose
\pgfusepath{fill,stroke}
\pgfpathmoveto{\pgfpoint{54.547806pt}{140.777435pt}}
\pgflineto{\pgfpoint{54.547806pt}{142.199554pt}}
\pgflineto{\pgfpoint{53.551819pt}{140.777435pt}}
\pgfpathclose
\pgfusepath{fill,stroke}
\pgfpathmoveto{\pgfpoint{55.543785pt}{196.136078pt}}
\pgflineto{\pgfpoint{54.547806pt}{142.199554pt}}
\pgflineto{\pgfpoint{54.547806pt}{140.777435pt}}
\pgfpathclose
\pgfusepath{fill,stroke}
\pgfpathmoveto{\pgfpoint{55.543785pt}{140.777435pt}}
\pgflineto{\pgfpoint{55.543785pt}{196.136078pt}}
\pgflineto{\pgfpoint{54.547806pt}{140.777435pt}}
\pgfpathclose
\pgfusepath{fill,stroke}
\pgfpathmoveto{\pgfpoint{56.539772pt}{168.281158pt}}
\pgflineto{\pgfpoint{55.543785pt}{196.136078pt}}
\pgflineto{\pgfpoint{55.543785pt}{140.777435pt}}
\pgfpathclose
\pgfusepath{fill,stroke}
\pgfpathmoveto{\pgfpoint{56.539772pt}{140.777435pt}}
\pgflineto{\pgfpoint{56.539772pt}{168.281158pt}}
\pgflineto{\pgfpoint{55.543785pt}{140.777435pt}}
\pgfpathclose
\pgfusepath{fill,stroke}
\pgfpathmoveto{\pgfpoint{57.535751pt}{147.868973pt}}
\pgflineto{\pgfpoint{56.539772pt}{168.281158pt}}
\pgflineto{\pgfpoint{56.539772pt}{140.777435pt}}
\pgfpathclose
\pgfusepath{fill,stroke}
\pgfpathmoveto{\pgfpoint{57.535751pt}{140.777435pt}}
\pgflineto{\pgfpoint{57.535751pt}{147.868973pt}}
\pgflineto{\pgfpoint{56.539772pt}{140.777435pt}}
\pgfpathclose
\pgfusepath{fill,stroke}
\pgfpathmoveto{\pgfpoint{58.531738pt}{141.160339pt}}
\pgflineto{\pgfpoint{57.535751pt}{147.868973pt}}
\pgflineto{\pgfpoint{57.535751pt}{140.777435pt}}
\pgfpathclose
\pgfusepath{fill,stroke}
\pgfpathmoveto{\pgfpoint{58.531738pt}{140.777435pt}}
\pgflineto{\pgfpoint{58.531738pt}{141.160339pt}}
\pgflineto{\pgfpoint{57.535751pt}{140.777435pt}}
\pgfpathclose
\pgfusepath{fill,stroke}
\pgfpathmoveto{\pgfpoint{59.527725pt}{141.894562pt}}
\pgflineto{\pgfpoint{58.531738pt}{141.160339pt}}
\pgflineto{\pgfpoint{58.531738pt}{140.777435pt}}
\pgfpathclose
\pgfusepath{fill,stroke}
\pgfpathmoveto{\pgfpoint{59.527725pt}{140.777435pt}}
\pgflineto{\pgfpoint{59.527725pt}{141.894562pt}}
\pgflineto{\pgfpoint{58.531738pt}{140.777435pt}}
\pgfpathclose
\pgfusepath{fill,stroke}
\pgfpathmoveto{\pgfpoint{60.523712pt}{142.042740pt}}
\pgflineto{\pgfpoint{59.527725pt}{141.894562pt}}
\pgflineto{\pgfpoint{59.527725pt}{140.777435pt}}
\pgfpathclose
\pgfusepath{fill,stroke}
\pgfpathmoveto{\pgfpoint{60.523712pt}{140.777435pt}}
\pgflineto{\pgfpoint{60.523712pt}{142.042740pt}}
\pgflineto{\pgfpoint{59.527725pt}{140.777435pt}}
\pgfpathclose
\pgfusepath{fill,stroke}
\pgfpathmoveto{\pgfpoint{61.519691pt}{141.419281pt}}
\pgflineto{\pgfpoint{60.523712pt}{142.042740pt}}
\pgflineto{\pgfpoint{60.523712pt}{140.777435pt}}
\pgfpathclose
\pgfusepath{fill,stroke}
\pgfpathmoveto{\pgfpoint{61.519691pt}{140.777435pt}}
\pgflineto{\pgfpoint{61.519691pt}{141.419281pt}}
\pgflineto{\pgfpoint{60.523712pt}{140.777435pt}}
\pgfpathclose
\pgfusepath{fill,stroke}
\pgfpathmoveto{\pgfpoint{62.515678pt}{202.358673pt}}
\pgflineto{\pgfpoint{61.519691pt}{141.419281pt}}
\pgflineto{\pgfpoint{61.519691pt}{140.777435pt}}
\pgfpathclose
\pgfusepath{fill,stroke}
\pgfpathmoveto{\pgfpoint{62.515678pt}{140.777435pt}}
\pgflineto{\pgfpoint{62.515678pt}{202.358673pt}}
\pgflineto{\pgfpoint{61.519691pt}{140.777435pt}}
\pgfpathclose
\pgfusepath{fill,stroke}
\pgfpathmoveto{\pgfpoint{63.511658pt}{143.879868pt}}
\pgflineto{\pgfpoint{62.515678pt}{202.358673pt}}
\pgflineto{\pgfpoint{62.515678pt}{140.777435pt}}
\pgfpathclose
\pgfusepath{fill,stroke}
\pgfpathmoveto{\pgfpoint{63.511658pt}{140.777435pt}}
\pgflineto{\pgfpoint{63.511658pt}{143.879868pt}}
\pgflineto{\pgfpoint{62.515678pt}{140.777435pt}}
\pgfpathclose
\pgfusepath{fill,stroke}
\pgfpathmoveto{\pgfpoint{64.507637pt}{144.233887pt}}
\pgflineto{\pgfpoint{63.511658pt}{143.879868pt}}
\pgflineto{\pgfpoint{63.511658pt}{140.777435pt}}
\pgfpathclose
\pgfusepath{fill,stroke}
\pgfpathmoveto{\pgfpoint{64.507637pt}{140.777435pt}}
\pgflineto{\pgfpoint{64.507637pt}{144.233887pt}}
\pgflineto{\pgfpoint{63.511658pt}{140.777435pt}}
\pgfpathclose
\pgfusepath{fill,stroke}
\pgfpathmoveto{\pgfpoint{65.503624pt}{142.070892pt}}
\pgflineto{\pgfpoint{64.507637pt}{144.233887pt}}
\pgflineto{\pgfpoint{64.507637pt}{140.777435pt}}
\pgfpathclose
\pgfusepath{fill,stroke}
\pgfpathmoveto{\pgfpoint{65.503624pt}{140.777435pt}}
\pgflineto{\pgfpoint{65.503624pt}{142.070892pt}}
\pgflineto{\pgfpoint{64.507637pt}{140.777435pt}}
\pgfpathclose
\pgfusepath{fill,stroke}
\pgfpathmoveto{\pgfpoint{66.499619pt}{141.913147pt}}
\pgflineto{\pgfpoint{65.503624pt}{142.070892pt}}
\pgflineto{\pgfpoint{65.503624pt}{140.777435pt}}
\pgfpathclose
\pgfusepath{fill,stroke}
\pgfpathmoveto{\pgfpoint{66.499619pt}{140.777435pt}}
\pgflineto{\pgfpoint{66.499619pt}{141.913147pt}}
\pgflineto{\pgfpoint{65.503624pt}{140.777435pt}}
\pgfpathclose
\pgfusepath{fill,stroke}
\pgfpathmoveto{\pgfpoint{67.495590pt}{140.777435pt}}
\pgflineto{\pgfpoint{66.499619pt}{141.913147pt}}
\pgflineto{\pgfpoint{66.499619pt}{140.777435pt}}
\pgfpathclose
\pgfusepath{fill,stroke}
\pgfpathmoveto{\pgfpoint{46.579933pt}{140.777435pt}}
\pgflineto{\pgfpoint{47.575912pt}{140.777435pt}}
\pgflineto{\pgfpoint{47.575912pt}{141.646210pt}}
\pgfpathclose
\pgfusepath{fill,stroke}
\pgfpathmoveto{\pgfpoint{48.571899pt}{141.530609pt}}
\pgflineto{\pgfpoint{47.575912pt}{141.646210pt}}
\pgflineto{\pgfpoint{47.575912pt}{140.777435pt}}
\pgfpathclose
\pgfusepath{fill,stroke}
\pgfpathmoveto{\pgfpoint{48.571899pt}{140.777435pt}}
\pgflineto{\pgfpoint{48.571899pt}{141.530609pt}}
\pgflineto{\pgfpoint{47.575912pt}{140.777435pt}}
\pgfpathclose
\pgfusepath{fill,stroke}
\pgfpathmoveto{\pgfpoint{49.567879pt}{143.458160pt}}
\pgflineto{\pgfpoint{48.571899pt}{141.530609pt}}
\pgflineto{\pgfpoint{48.571899pt}{140.777435pt}}
\pgfpathclose
\pgfusepath{fill,stroke}
\pgfpathmoveto{\pgfpoint{49.567879pt}{140.777435pt}}
\pgflineto{\pgfpoint{49.567879pt}{143.458160pt}}
\pgflineto{\pgfpoint{48.571899pt}{140.777435pt}}
\pgfpathclose
\pgfusepath{fill,stroke}
\pgfpathmoveto{\pgfpoint{50.563873pt}{140.777435pt}}
\pgflineto{\pgfpoint{49.567879pt}{143.458160pt}}
\pgflineto{\pgfpoint{49.567879pt}{140.777435pt}}
\pgfpathclose
\pgfusepath{fill,stroke}
\pgfpathmoveto{\pgfpoint{44.587967pt}{140.777435pt}}
\pgflineto{\pgfpoint{45.583946pt}{140.777435pt}}
\pgflineto{\pgfpoint{45.583946pt}{148.946091pt}}
\pgfpathclose
\pgfusepath{fill,stroke}
\pgfpathmoveto{\pgfpoint{46.579933pt}{140.777435pt}}
\pgflineto{\pgfpoint{45.583946pt}{148.946091pt}}
\pgflineto{\pgfpoint{45.583946pt}{140.777435pt}}
\pgfpathclose
\pgfusepath{fill,stroke}
\pgfpathmoveto{\pgfpoint{42.595993pt}{140.777435pt}}
\pgflineto{\pgfpoint{43.591980pt}{140.777435pt}}
\pgflineto{\pgfpoint{43.591980pt}{186.324295pt}}
\pgfpathclose
\pgfusepath{fill,stroke}
\pgfpathmoveto{\pgfpoint{44.587967pt}{140.777435pt}}
\pgflineto{\pgfpoint{43.591980pt}{186.324295pt}}
\pgflineto{\pgfpoint{43.591980pt}{140.777435pt}}
\pgfpathclose
\pgfusepath{fill,stroke}
\color[rgb]{0.000000,1.000000,0.000000}
\pgfpathmoveto{\pgfpoint{284.620087pt}{26.399979pt}}
\pgflineto{\pgfpoint{285.616089pt}{26.399979pt}}
\pgflineto{\pgfpoint{285.616089pt}{26.474419pt}}
\pgfpathclose
\pgfusepath{fill,stroke}
\pgfpathmoveto{\pgfpoint{286.612061pt}{26.399979pt}}
\pgflineto{\pgfpoint{285.616089pt}{26.474419pt}}
\pgflineto{\pgfpoint{285.616089pt}{26.399979pt}}
\pgfpathclose
\pgfusepath{fill,stroke}
\pgfpathmoveto{\pgfpoint{281.632141pt}{26.399979pt}}
\pgflineto{\pgfpoint{282.628113pt}{26.399979pt}}
\pgflineto{\pgfpoint{282.628113pt}{29.464996pt}}
\pgfpathclose
\pgfusepath{fill,stroke}
\pgfpathmoveto{\pgfpoint{283.624115pt}{28.196754pt}}
\pgflineto{\pgfpoint{282.628113pt}{29.464996pt}}
\pgflineto{\pgfpoint{282.628113pt}{26.399979pt}}
\pgfpathclose
\pgfusepath{fill,stroke}
\pgfpathmoveto{\pgfpoint{283.624115pt}{26.399979pt}}
\pgflineto{\pgfpoint{283.624115pt}{28.196754pt}}
\pgflineto{\pgfpoint{282.628113pt}{26.399979pt}}
\pgfpathclose
\pgfusepath{fill,stroke}
\pgfpathmoveto{\pgfpoint{284.620087pt}{26.399979pt}}
\pgflineto{\pgfpoint{283.624115pt}{28.196754pt}}
\pgflineto{\pgfpoint{283.624115pt}{26.399979pt}}
\pgfpathclose
\pgfusepath{fill,stroke}
\pgfpathmoveto{\pgfpoint{277.648193pt}{26.399979pt}}
\pgflineto{\pgfpoint{278.644196pt}{26.399979pt}}
\pgflineto{\pgfpoint{278.644196pt}{30.203644pt}}
\pgfpathclose
\pgfusepath{fill,stroke}
\pgfpathmoveto{\pgfpoint{279.640167pt}{35.907738pt}}
\pgflineto{\pgfpoint{278.644196pt}{30.203644pt}}
\pgflineto{\pgfpoint{278.644196pt}{26.399979pt}}
\pgfpathclose
\pgfusepath{fill,stroke}
\pgfpathmoveto{\pgfpoint{279.640167pt}{26.399979pt}}
\pgflineto{\pgfpoint{279.640167pt}{35.907738pt}}
\pgflineto{\pgfpoint{278.644196pt}{26.399979pt}}
\pgfpathclose
\pgfusepath{fill,stroke}
\pgfpathmoveto{\pgfpoint{280.636139pt}{26.399979pt}}
\pgflineto{\pgfpoint{279.640167pt}{35.907738pt}}
\pgflineto{\pgfpoint{279.640167pt}{26.399979pt}}
\pgfpathclose
\pgfusepath{fill,stroke}
\pgfpathmoveto{\pgfpoint{267.688354pt}{26.399979pt}}
\pgflineto{\pgfpoint{268.684326pt}{26.399979pt}}
\pgflineto{\pgfpoint{268.684326pt}{26.647873pt}}
\pgfpathclose
\pgfusepath{fill,stroke}
\pgfpathmoveto{\pgfpoint{269.680328pt}{26.596115pt}}
\pgflineto{\pgfpoint{268.684326pt}{26.647873pt}}
\pgflineto{\pgfpoint{268.684326pt}{26.399979pt}}
\pgfpathclose
\pgfusepath{fill,stroke}
\pgfpathmoveto{\pgfpoint{269.680328pt}{26.399979pt}}
\pgflineto{\pgfpoint{269.680328pt}{26.596115pt}}
\pgflineto{\pgfpoint{268.684326pt}{26.399979pt}}
\pgfpathclose
\pgfusepath{fill,stroke}
\pgfpathmoveto{\pgfpoint{270.676331pt}{27.976303pt}}
\pgflineto{\pgfpoint{269.680328pt}{26.596115pt}}
\pgflineto{\pgfpoint{269.680328pt}{26.399979pt}}
\pgfpathclose
\pgfusepath{fill,stroke}
\pgfpathmoveto{\pgfpoint{270.676331pt}{26.399979pt}}
\pgflineto{\pgfpoint{270.676331pt}{27.976303pt}}
\pgflineto{\pgfpoint{269.680328pt}{26.399979pt}}
\pgfpathclose
\pgfusepath{fill,stroke}
\pgfpathmoveto{\pgfpoint{271.672302pt}{42.660454pt}}
\pgflineto{\pgfpoint{270.676331pt}{27.976303pt}}
\pgflineto{\pgfpoint{270.676331pt}{26.399979pt}}
\pgfpathclose
\pgfusepath{fill,stroke}
\pgfpathmoveto{\pgfpoint{271.672302pt}{26.399979pt}}
\pgflineto{\pgfpoint{271.672302pt}{42.660454pt}}
\pgflineto{\pgfpoint{270.676331pt}{26.399979pt}}
\pgfpathclose
\pgfusepath{fill,stroke}
\pgfpathmoveto{\pgfpoint{272.668274pt}{26.867470pt}}
\pgflineto{\pgfpoint{271.672302pt}{42.660454pt}}
\pgflineto{\pgfpoint{271.672302pt}{26.399979pt}}
\pgfpathclose
\pgfusepath{fill,stroke}
\pgfpathmoveto{\pgfpoint{272.668274pt}{26.399979pt}}
\pgflineto{\pgfpoint{272.668274pt}{26.867470pt}}
\pgflineto{\pgfpoint{271.672302pt}{26.399979pt}}
\pgfpathclose
\pgfusepath{fill,stroke}
\pgfpathmoveto{\pgfpoint{273.664276pt}{26.399979pt}}
\pgflineto{\pgfpoint{272.668274pt}{26.867470pt}}
\pgflineto{\pgfpoint{272.668274pt}{26.399979pt}}
\pgfpathclose
\pgfusepath{fill,stroke}
\pgfpathmoveto{\pgfpoint{265.696411pt}{26.399979pt}}
\pgflineto{\pgfpoint{266.692383pt}{26.399979pt}}
\pgflineto{\pgfpoint{266.692383pt}{26.914734pt}}
\pgfpathclose
\pgfusepath{fill,stroke}
\pgfpathmoveto{\pgfpoint{267.688354pt}{26.399979pt}}
\pgflineto{\pgfpoint{266.692383pt}{26.914734pt}}
\pgflineto{\pgfpoint{266.692383pt}{26.399979pt}}
\pgfpathclose
\pgfusepath{fill,stroke}
\pgfpathmoveto{\pgfpoint{263.704407pt}{26.399979pt}}
\pgflineto{\pgfpoint{264.700409pt}{26.399979pt}}
\pgflineto{\pgfpoint{264.700409pt}{26.932861pt}}
\pgfpathclose
\pgfusepath{fill,stroke}
\pgfpathmoveto{\pgfpoint{265.696411pt}{26.399979pt}}
\pgflineto{\pgfpoint{264.700409pt}{26.932861pt}}
\pgflineto{\pgfpoint{264.700409pt}{26.399979pt}}
\pgfpathclose
\pgfusepath{fill,stroke}
\pgfpathmoveto{\pgfpoint{260.716492pt}{26.399979pt}}
\pgflineto{\pgfpoint{261.712463pt}{26.399979pt}}
\pgflineto{\pgfpoint{261.712463pt}{27.286751pt}}
\pgfpathclose
\pgfusepath{fill,stroke}
\pgfpathmoveto{\pgfpoint{262.708435pt}{26.399979pt}}
\pgflineto{\pgfpoint{261.712463pt}{27.286751pt}}
\pgflineto{\pgfpoint{261.712463pt}{26.399979pt}}
\pgfpathclose
\pgfusepath{fill,stroke}
\pgfpathmoveto{\pgfpoint{257.728516pt}{26.399979pt}}
\pgflineto{\pgfpoint{258.724518pt}{26.399979pt}}
\pgflineto{\pgfpoint{258.724518pt}{26.886185pt}}
\pgfpathclose
\pgfusepath{fill,stroke}
\pgfpathmoveto{\pgfpoint{259.720490pt}{26.399979pt}}
\pgflineto{\pgfpoint{258.724518pt}{26.886185pt}}
\pgflineto{\pgfpoint{258.724518pt}{26.399979pt}}
\pgfpathclose
\pgfusepath{fill,stroke}
\pgfpathmoveto{\pgfpoint{253.744598pt}{26.399979pt}}
\pgflineto{\pgfpoint{254.740570pt}{26.399979pt}}
\pgflineto{\pgfpoint{254.740570pt}{27.508163pt}}
\pgfpathclose
\pgfusepath{fill,stroke}
\pgfpathmoveto{\pgfpoint{255.736542pt}{26.399979pt}}
\pgflineto{\pgfpoint{254.740570pt}{27.508163pt}}
\pgflineto{\pgfpoint{254.740570pt}{26.399979pt}}
\pgfpathclose
\pgfusepath{fill,stroke}
\pgfpathmoveto{\pgfpoint{248.764679pt}{26.399979pt}}
\pgflineto{\pgfpoint{249.760651pt}{26.399979pt}}
\pgflineto{\pgfpoint{249.760651pt}{26.474113pt}}
\pgfpathclose
\pgfusepath{fill,stroke}
\pgfpathmoveto{\pgfpoint{250.756638pt}{26.708969pt}}
\pgflineto{\pgfpoint{249.760651pt}{26.474113pt}}
\pgflineto{\pgfpoint{249.760651pt}{26.399979pt}}
\pgfpathclose
\pgfusepath{fill,stroke}
\pgfpathmoveto{\pgfpoint{250.756638pt}{26.399979pt}}
\pgflineto{\pgfpoint{250.756638pt}{26.708969pt}}
\pgflineto{\pgfpoint{249.760651pt}{26.399979pt}}
\pgfpathclose
\pgfusepath{fill,stroke}
\pgfpathmoveto{\pgfpoint{251.752625pt}{26.399979pt}}
\pgflineto{\pgfpoint{250.756638pt}{26.708969pt}}
\pgflineto{\pgfpoint{250.756638pt}{26.399979pt}}
\pgfpathclose
\pgfusepath{fill,stroke}
\pgfpathmoveto{\pgfpoint{244.780731pt}{26.399979pt}}
\pgflineto{\pgfpoint{245.776718pt}{26.399979pt}}
\pgflineto{\pgfpoint{245.776718pt}{26.477646pt}}
\pgfpathclose
\pgfusepath{fill,stroke}
\pgfpathmoveto{\pgfpoint{246.772705pt}{26.550812pt}}
\pgflineto{\pgfpoint{245.776718pt}{26.477646pt}}
\pgflineto{\pgfpoint{245.776718pt}{26.399979pt}}
\pgfpathclose
\pgfusepath{fill,stroke}
\pgfpathmoveto{\pgfpoint{246.772705pt}{26.399979pt}}
\pgflineto{\pgfpoint{246.772705pt}{26.550812pt}}
\pgflineto{\pgfpoint{245.776718pt}{26.399979pt}}
\pgfpathclose
\pgfusepath{fill,stroke}
\pgfpathmoveto{\pgfpoint{247.768677pt}{41.200562pt}}
\pgflineto{\pgfpoint{246.772705pt}{26.550812pt}}
\pgflineto{\pgfpoint{246.772705pt}{26.399979pt}}
\pgfpathclose
\pgfusepath{fill,stroke}
\pgfpathmoveto{\pgfpoint{247.768677pt}{26.399979pt}}
\pgflineto{\pgfpoint{247.768677pt}{41.200562pt}}
\pgflineto{\pgfpoint{246.772705pt}{26.399979pt}}
\pgfpathclose
\pgfusepath{fill,stroke}
\pgfpathmoveto{\pgfpoint{248.764679pt}{26.399979pt}}
\pgflineto{\pgfpoint{247.768677pt}{41.200562pt}}
\pgflineto{\pgfpoint{247.768677pt}{26.399979pt}}
\pgfpathclose
\pgfusepath{fill,stroke}
\pgfpathmoveto{\pgfpoint{241.792786pt}{26.399979pt}}
\pgflineto{\pgfpoint{242.788757pt}{26.399979pt}}
\pgflineto{\pgfpoint{242.788757pt}{26.819893pt}}
\pgfpathclose
\pgfusepath{fill,stroke}
\pgfpathmoveto{\pgfpoint{243.784744pt}{27.221687pt}}
\pgflineto{\pgfpoint{242.788757pt}{26.819893pt}}
\pgflineto{\pgfpoint{242.788757pt}{26.399979pt}}
\pgfpathclose
\pgfusepath{fill,stroke}
\pgfpathmoveto{\pgfpoint{243.784744pt}{26.399979pt}}
\pgflineto{\pgfpoint{243.784744pt}{27.221687pt}}
\pgflineto{\pgfpoint{242.788757pt}{26.399979pt}}
\pgfpathclose
\pgfusepath{fill,stroke}
\pgfpathmoveto{\pgfpoint{244.780731pt}{26.399979pt}}
\pgflineto{\pgfpoint{243.784744pt}{27.221687pt}}
\pgflineto{\pgfpoint{243.784744pt}{26.399979pt}}
\pgfpathclose
\pgfusepath{fill,stroke}
\pgfpathmoveto{\pgfpoint{235.816864pt}{26.399979pt}}
\pgflineto{\pgfpoint{236.812866pt}{26.399979pt}}
\pgflineto{\pgfpoint{236.812866pt}{35.730797pt}}
\pgfpathclose
\pgfusepath{fill,stroke}
\pgfpathmoveto{\pgfpoint{237.808838pt}{26.516998pt}}
\pgflineto{\pgfpoint{236.812866pt}{35.730797pt}}
\pgflineto{\pgfpoint{236.812866pt}{26.399979pt}}
\pgfpathclose
\pgfusepath{fill,stroke}
\pgfpathmoveto{\pgfpoint{237.808838pt}{26.399979pt}}
\pgflineto{\pgfpoint{237.808838pt}{26.516998pt}}
\pgflineto{\pgfpoint{236.812866pt}{26.399979pt}}
\pgfpathclose
\pgfusepath{fill,stroke}
\pgfpathmoveto{\pgfpoint{238.804825pt}{26.399979pt}}
\pgflineto{\pgfpoint{237.808838pt}{26.516998pt}}
\pgflineto{\pgfpoint{237.808838pt}{26.399979pt}}
\pgfpathclose
\pgfusepath{fill,stroke}
\pgfpathmoveto{\pgfpoint{227.849014pt}{26.399979pt}}
\pgflineto{\pgfpoint{228.845001pt}{26.399979pt}}
\pgflineto{\pgfpoint{228.845001pt}{27.012459pt}}
\pgfpathclose
\pgfusepath{fill,stroke}
\pgfpathmoveto{\pgfpoint{229.840973pt}{26.558662pt}}
\pgflineto{\pgfpoint{228.845001pt}{27.012459pt}}
\pgflineto{\pgfpoint{228.845001pt}{26.399979pt}}
\pgfpathclose
\pgfusepath{fill,stroke}
\pgfpathmoveto{\pgfpoint{229.840973pt}{26.399979pt}}
\pgflineto{\pgfpoint{229.840973pt}{26.558662pt}}
\pgflineto{\pgfpoint{228.845001pt}{26.399979pt}}
\pgfpathclose
\pgfusepath{fill,stroke}
\pgfpathmoveto{\pgfpoint{230.836945pt}{26.682472pt}}
\pgflineto{\pgfpoint{229.840973pt}{26.558662pt}}
\pgflineto{\pgfpoint{229.840973pt}{26.399979pt}}
\pgfpathclose
\pgfusepath{fill,stroke}
\pgfpathmoveto{\pgfpoint{230.836945pt}{26.399979pt}}
\pgflineto{\pgfpoint{230.836945pt}{26.682472pt}}
\pgflineto{\pgfpoint{229.840973pt}{26.399979pt}}
\pgfpathclose
\pgfusepath{fill,stroke}
\pgfpathmoveto{\pgfpoint{231.832932pt}{26.399979pt}}
\pgflineto{\pgfpoint{230.836945pt}{26.682472pt}}
\pgflineto{\pgfpoint{230.836945pt}{26.399979pt}}
\pgfpathclose
\pgfusepath{fill,stroke}
\pgfpathmoveto{\pgfpoint{224.861053pt}{26.399979pt}}
\pgflineto{\pgfpoint{225.857040pt}{26.399979pt}}
\pgflineto{\pgfpoint{225.857040pt}{28.125343pt}}
\pgfpathclose
\pgfusepath{fill,stroke}
\pgfpathmoveto{\pgfpoint{226.853027pt}{26.399979pt}}
\pgflineto{\pgfpoint{225.857040pt}{28.125343pt}}
\pgflineto{\pgfpoint{225.857040pt}{26.399979pt}}
\pgfpathclose
\pgfusepath{fill,stroke}
\pgfpathmoveto{\pgfpoint{216.893188pt}{26.399979pt}}
\pgflineto{\pgfpoint{217.889160pt}{26.399979pt}}
\pgflineto{\pgfpoint{217.889160pt}{66.077660pt}}
\pgfpathclose
\pgfusepath{fill,stroke}
\pgfpathmoveto{\pgfpoint{218.885147pt}{26.399979pt}}
\pgflineto{\pgfpoint{217.889160pt}{66.077660pt}}
\pgflineto{\pgfpoint{217.889160pt}{26.399979pt}}
\pgfpathclose
\pgfusepath{fill,stroke}
\pgfpathmoveto{\pgfpoint{213.905228pt}{26.399979pt}}
\pgflineto{\pgfpoint{214.901215pt}{26.399979pt}}
\pgflineto{\pgfpoint{214.901215pt}{27.407372pt}}
\pgfpathclose
\pgfusepath{fill,stroke}
\pgfpathmoveto{\pgfpoint{215.897217pt}{26.495361pt}}
\pgflineto{\pgfpoint{214.901215pt}{27.407372pt}}
\pgflineto{\pgfpoint{214.901215pt}{26.399979pt}}
\pgfpathclose
\pgfusepath{fill,stroke}
\pgfpathmoveto{\pgfpoint{215.897217pt}{26.399979pt}}
\pgflineto{\pgfpoint{215.897217pt}{26.495361pt}}
\pgflineto{\pgfpoint{214.901215pt}{26.399979pt}}
\pgfpathclose
\pgfusepath{fill,stroke}
\pgfpathmoveto{\pgfpoint{216.893188pt}{26.399979pt}}
\pgflineto{\pgfpoint{215.897217pt}{26.495361pt}}
\pgflineto{\pgfpoint{215.897217pt}{26.399979pt}}
\pgfpathclose
\pgfusepath{fill,stroke}
\pgfpathmoveto{\pgfpoint{209.921295pt}{26.399979pt}}
\pgflineto{\pgfpoint{210.917267pt}{26.399979pt}}
\pgflineto{\pgfpoint{210.917267pt}{56.628197pt}}
\pgfpathclose
\pgfusepath{fill,stroke}
\pgfpathmoveto{\pgfpoint{211.913269pt}{30.417183pt}}
\pgflineto{\pgfpoint{210.917267pt}{56.628197pt}}
\pgflineto{\pgfpoint{210.917267pt}{26.399979pt}}
\pgfpathclose
\pgfusepath{fill,stroke}
\pgfpathmoveto{\pgfpoint{211.913269pt}{26.399979pt}}
\pgflineto{\pgfpoint{211.913269pt}{30.417183pt}}
\pgflineto{\pgfpoint{210.917267pt}{26.399979pt}}
\pgfpathclose
\pgfusepath{fill,stroke}
\pgfpathmoveto{\pgfpoint{212.909241pt}{47.374290pt}}
\pgflineto{\pgfpoint{211.913269pt}{30.417183pt}}
\pgflineto{\pgfpoint{211.913269pt}{26.399979pt}}
\pgfpathclose
\pgfusepath{fill,stroke}
\pgfpathmoveto{\pgfpoint{212.909241pt}{26.399979pt}}
\pgflineto{\pgfpoint{212.909241pt}{47.374290pt}}
\pgflineto{\pgfpoint{211.913269pt}{26.399979pt}}
\pgfpathclose
\pgfusepath{fill,stroke}
\pgfpathmoveto{\pgfpoint{213.905228pt}{26.399979pt}}
\pgflineto{\pgfpoint{212.909241pt}{47.374290pt}}
\pgflineto{\pgfpoint{212.909241pt}{26.399979pt}}
\pgfpathclose
\pgfusepath{fill,stroke}
\pgfpathmoveto{\pgfpoint{207.929337pt}{26.399979pt}}
\pgflineto{\pgfpoint{208.925323pt}{26.399979pt}}
\pgflineto{\pgfpoint{208.925323pt}{29.200577pt}}
\pgfpathclose
\pgfusepath{fill,stroke}
\pgfpathmoveto{\pgfpoint{209.921295pt}{26.399979pt}}
\pgflineto{\pgfpoint{208.925323pt}{29.200577pt}}
\pgflineto{\pgfpoint{208.925323pt}{26.399979pt}}
\pgfpathclose
\pgfusepath{fill,stroke}
\pgfpathmoveto{\pgfpoint{204.941376pt}{26.399979pt}}
\pgflineto{\pgfpoint{205.937347pt}{26.399979pt}}
\pgflineto{\pgfpoint{205.937347pt}{29.413773pt}}
\pgfpathclose
\pgfusepath{fill,stroke}
\pgfpathmoveto{\pgfpoint{206.933334pt}{28.372978pt}}
\pgflineto{\pgfpoint{205.937347pt}{29.413773pt}}
\pgflineto{\pgfpoint{205.937347pt}{26.399979pt}}
\pgfpathclose
\pgfusepath{fill,stroke}
\pgfpathmoveto{\pgfpoint{206.933334pt}{26.399979pt}}
\pgflineto{\pgfpoint{206.933334pt}{28.372978pt}}
\pgflineto{\pgfpoint{205.937347pt}{26.399979pt}}
\pgfpathclose
\pgfusepath{fill,stroke}
\pgfpathmoveto{\pgfpoint{207.929337pt}{26.399979pt}}
\pgflineto{\pgfpoint{206.933334pt}{28.372978pt}}
\pgflineto{\pgfpoint{206.933334pt}{26.399979pt}}
\pgfpathclose
\pgfusepath{fill,stroke}
\pgfpathmoveto{\pgfpoint{202.949402pt}{26.399979pt}}
\pgflineto{\pgfpoint{203.945404pt}{26.399979pt}}
\pgflineto{\pgfpoint{203.945404pt}{26.673592pt}}
\pgfpathclose
\pgfusepath{fill,stroke}
\pgfpathmoveto{\pgfpoint{204.941376pt}{26.399979pt}}
\pgflineto{\pgfpoint{203.945404pt}{26.673592pt}}
\pgflineto{\pgfpoint{203.945404pt}{26.399979pt}}
\pgfpathclose
\pgfusepath{fill,stroke}
\pgfpathmoveto{\pgfpoint{196.973511pt}{26.399979pt}}
\pgflineto{\pgfpoint{197.969498pt}{26.399979pt}}
\pgflineto{\pgfpoint{197.969498pt}{27.127144pt}}
\pgfpathclose
\pgfusepath{fill,stroke}
\pgfpathmoveto{\pgfpoint{198.965469pt}{26.399979pt}}
\pgflineto{\pgfpoint{197.969498pt}{27.127144pt}}
\pgflineto{\pgfpoint{197.969498pt}{26.399979pt}}
\pgfpathclose
\pgfusepath{fill,stroke}
\pgfpathmoveto{\pgfpoint{193.985565pt}{26.399979pt}}
\pgflineto{\pgfpoint{194.981537pt}{26.399979pt}}
\pgflineto{\pgfpoint{194.981537pt}{28.872696pt}}
\pgfpathclose
\pgfusepath{fill,stroke}
\pgfpathmoveto{\pgfpoint{195.977524pt}{27.054726pt}}
\pgflineto{\pgfpoint{194.981537pt}{28.872696pt}}
\pgflineto{\pgfpoint{194.981537pt}{26.399979pt}}
\pgfpathclose
\pgfusepath{fill,stroke}
\pgfpathmoveto{\pgfpoint{195.977524pt}{26.399979pt}}
\pgflineto{\pgfpoint{195.977524pt}{27.054726pt}}
\pgflineto{\pgfpoint{194.981537pt}{26.399979pt}}
\pgfpathclose
\pgfusepath{fill,stroke}
\pgfpathmoveto{\pgfpoint{196.973511pt}{26.399979pt}}
\pgflineto{\pgfpoint{195.977524pt}{27.054726pt}}
\pgflineto{\pgfpoint{195.977524pt}{26.399979pt}}
\pgfpathclose
\pgfusepath{fill,stroke}
\pgfpathmoveto{\pgfpoint{190.997604pt}{26.399979pt}}
\pgflineto{\pgfpoint{191.993591pt}{26.399979pt}}
\pgflineto{\pgfpoint{191.993591pt}{31.034332pt}}
\pgfpathclose
\pgfusepath{fill,stroke}
\pgfpathmoveto{\pgfpoint{192.989563pt}{26.399979pt}}
\pgflineto{\pgfpoint{191.993591pt}{31.034332pt}}
\pgflineto{\pgfpoint{191.993591pt}{26.399979pt}}
\pgfpathclose
\pgfusepath{fill,stroke}
\pgfpathmoveto{\pgfpoint{189.005630pt}{26.399979pt}}
\pgflineto{\pgfpoint{190.001617pt}{26.399979pt}}
\pgflineto{\pgfpoint{190.001617pt}{34.817444pt}}
\pgfpathclose
\pgfusepath{fill,stroke}
\pgfpathmoveto{\pgfpoint{190.997604pt}{26.399979pt}}
\pgflineto{\pgfpoint{190.001617pt}{34.817444pt}}
\pgflineto{\pgfpoint{190.001617pt}{26.399979pt}}
\pgfpathclose
\pgfusepath{fill,stroke}
\pgfpathmoveto{\pgfpoint{182.033752pt}{26.399979pt}}
\pgflineto{\pgfpoint{183.029724pt}{26.399979pt}}
\pgflineto{\pgfpoint{183.029724pt}{26.463455pt}}
\pgfpathclose
\pgfusepath{fill,stroke}
\pgfpathmoveto{\pgfpoint{184.025711pt}{26.399979pt}}
\pgflineto{\pgfpoint{183.029724pt}{26.463455pt}}
\pgflineto{\pgfpoint{183.029724pt}{26.399979pt}}
\pgfpathclose
\pgfusepath{fill,stroke}
\pgfpathmoveto{\pgfpoint{177.053818pt}{26.399979pt}}
\pgflineto{\pgfpoint{178.049805pt}{91.264786pt}}
\pgflineto{\pgfpoint{177.581543pt}{91.264786pt}}
\pgfpathclose
\pgfusepath{fill,stroke}
\pgfpathmoveto{\pgfpoint{177.053818pt}{26.399979pt}}
\pgflineto{\pgfpoint{178.049805pt}{26.399979pt}}
\pgflineto{\pgfpoint{178.049805pt}{91.264786pt}}
\pgfpathclose
\pgfusepath{fill,stroke}
\pgfpathmoveto{\pgfpoint{179.045792pt}{26.504753pt}}
\pgflineto{\pgfpoint{178.049805pt}{91.264786pt}}
\pgflineto{\pgfpoint{178.049805pt}{26.399979pt}}
\pgfpathclose
\pgfusepath{fill,stroke}
\pgfpathmoveto{\pgfpoint{179.045792pt}{26.504753pt}}
\pgflineto{\pgfpoint{178.518478pt}{91.264793pt}}
\pgflineto{\pgfpoint{178.049805pt}{91.264786pt}}
\pgfpathclose
\pgfusepath{fill,stroke}
\pgfpathmoveto{\pgfpoint{179.045792pt}{26.399979pt}}
\pgflineto{\pgfpoint{179.045792pt}{26.504753pt}}
\pgflineto{\pgfpoint{178.049805pt}{26.399979pt}}
\pgfpathclose
\pgfusepath{fill,stroke}
\pgfpathmoveto{\pgfpoint{180.041779pt}{30.030716pt}}
\pgflineto{\pgfpoint{179.045792pt}{26.504753pt}}
\pgflineto{\pgfpoint{179.045792pt}{26.399979pt}}
\pgfpathclose
\pgfusepath{fill,stroke}
\pgfpathmoveto{\pgfpoint{180.041779pt}{26.399979pt}}
\pgflineto{\pgfpoint{180.041779pt}{30.030716pt}}
\pgflineto{\pgfpoint{179.045792pt}{26.399979pt}}
\pgfpathclose
\pgfusepath{fill,stroke}
\pgfpathmoveto{\pgfpoint{181.037766pt}{26.399979pt}}
\pgflineto{\pgfpoint{180.041779pt}{30.030716pt}}
\pgflineto{\pgfpoint{180.041779pt}{26.399979pt}}
\pgfpathclose
\pgfusepath{fill,stroke}
\pgfpathmoveto{\pgfpoint{169.085953pt}{26.399979pt}}
\pgflineto{\pgfpoint{170.081940pt}{26.399979pt}}
\pgflineto{\pgfpoint{170.081940pt}{26.839508pt}}
\pgfpathclose
\pgfusepath{fill,stroke}
\pgfpathmoveto{\pgfpoint{171.077911pt}{26.399979pt}}
\pgflineto{\pgfpoint{170.081940pt}{26.839508pt}}
\pgflineto{\pgfpoint{170.081940pt}{26.399979pt}}
\pgfpathclose
\pgfusepath{fill,stroke}
\pgfpathmoveto{\pgfpoint{166.098007pt}{26.399979pt}}
\pgflineto{\pgfpoint{167.093994pt}{26.399979pt}}
\pgflineto{\pgfpoint{167.093994pt}{30.536880pt}}
\pgfpathclose
\pgfusepath{fill,stroke}
\pgfpathmoveto{\pgfpoint{168.089966pt}{26.447098pt}}
\pgflineto{\pgfpoint{167.093994pt}{30.536880pt}}
\pgflineto{\pgfpoint{167.093994pt}{26.399979pt}}
\pgfpathclose
\pgfusepath{fill,stroke}
\pgfpathmoveto{\pgfpoint{168.089966pt}{26.399979pt}}
\pgflineto{\pgfpoint{168.089966pt}{26.447098pt}}
\pgflineto{\pgfpoint{167.093994pt}{26.399979pt}}
\pgfpathclose
\pgfusepath{fill,stroke}
\pgfpathmoveto{\pgfpoint{169.085953pt}{26.399979pt}}
\pgflineto{\pgfpoint{168.089966pt}{26.447098pt}}
\pgflineto{\pgfpoint{168.089966pt}{26.399979pt}}
\pgfpathclose
\pgfusepath{fill,stroke}
\pgfpathmoveto{\pgfpoint{164.106033pt}{26.399979pt}}
\pgflineto{\pgfpoint{165.102020pt}{26.399979pt}}
\pgflineto{\pgfpoint{165.102020pt}{26.737396pt}}
\pgfpathclose
\pgfusepath{fill,stroke}
\pgfpathmoveto{\pgfpoint{166.098007pt}{26.399979pt}}
\pgflineto{\pgfpoint{165.102020pt}{26.737396pt}}
\pgflineto{\pgfpoint{165.102020pt}{26.399979pt}}
\pgfpathclose
\pgfusepath{fill,stroke}
\pgfpathmoveto{\pgfpoint{160.122101pt}{26.399979pt}}
\pgflineto{\pgfpoint{161.118088pt}{26.399979pt}}
\pgflineto{\pgfpoint{161.118088pt}{26.483330pt}}
\pgfpathclose
\pgfusepath{fill,stroke}
\pgfpathmoveto{\pgfpoint{162.114075pt}{26.604111pt}}
\pgflineto{\pgfpoint{161.118088pt}{26.483330pt}}
\pgflineto{\pgfpoint{161.118088pt}{26.399979pt}}
\pgfpathclose
\pgfusepath{fill,stroke}
\pgfpathmoveto{\pgfpoint{162.114075pt}{26.399979pt}}
\pgflineto{\pgfpoint{162.114075pt}{26.604111pt}}
\pgflineto{\pgfpoint{161.118088pt}{26.399979pt}}
\pgfpathclose
\pgfusepath{fill,stroke}
\pgfpathmoveto{\pgfpoint{163.110062pt}{28.794250pt}}
\pgflineto{\pgfpoint{162.114075pt}{26.604111pt}}
\pgflineto{\pgfpoint{162.114075pt}{26.399979pt}}
\pgfpathclose
\pgfusepath{fill,stroke}
\pgfpathmoveto{\pgfpoint{163.110062pt}{26.399979pt}}
\pgflineto{\pgfpoint{163.110062pt}{28.794250pt}}
\pgflineto{\pgfpoint{162.114075pt}{26.399979pt}}
\pgfpathclose
\pgfusepath{fill,stroke}
\pgfpathmoveto{\pgfpoint{164.106033pt}{26.399979pt}}
\pgflineto{\pgfpoint{163.110062pt}{28.794250pt}}
\pgflineto{\pgfpoint{163.110062pt}{26.399979pt}}
\pgfpathclose
\pgfusepath{fill,stroke}
\pgfpathmoveto{\pgfpoint{156.138168pt}{26.399979pt}}
\pgflineto{\pgfpoint{157.134155pt}{26.399979pt}}
\pgflineto{\pgfpoint{157.134155pt}{26.492256pt}}
\pgfpathclose
\pgfusepath{fill,stroke}
\pgfpathmoveto{\pgfpoint{158.130127pt}{26.480469pt}}
\pgflineto{\pgfpoint{157.134155pt}{26.492256pt}}
\pgflineto{\pgfpoint{157.134155pt}{26.399979pt}}
\pgfpathclose
\pgfusepath{fill,stroke}
\pgfpathmoveto{\pgfpoint{158.130127pt}{26.399979pt}}
\pgflineto{\pgfpoint{158.130127pt}{26.480469pt}}
\pgflineto{\pgfpoint{157.134155pt}{26.399979pt}}
\pgfpathclose
\pgfusepath{fill,stroke}
\pgfpathmoveto{\pgfpoint{159.126114pt}{26.399979pt}}
\pgflineto{\pgfpoint{158.130127pt}{26.480469pt}}
\pgflineto{\pgfpoint{158.130127pt}{26.399979pt}}
\pgfpathclose
\pgfusepath{fill,stroke}
\pgfpathmoveto{\pgfpoint{154.146194pt}{26.399979pt}}
\pgflineto{\pgfpoint{155.142181pt}{26.399979pt}}
\pgflineto{\pgfpoint{155.142181pt}{28.098244pt}}
\pgfpathclose
\pgfusepath{fill,stroke}
\pgfpathmoveto{\pgfpoint{156.138168pt}{26.399979pt}}
\pgflineto{\pgfpoint{155.142181pt}{28.098244pt}}
\pgflineto{\pgfpoint{155.142181pt}{26.399979pt}}
\pgfpathclose
\pgfusepath{fill,stroke}
\pgfpathmoveto{\pgfpoint{150.162262pt}{26.399979pt}}
\pgflineto{\pgfpoint{151.158249pt}{26.399979pt}}
\pgflineto{\pgfpoint{151.158249pt}{26.439781pt}}
\pgfpathclose
\pgfusepath{fill,stroke}
\pgfpathmoveto{\pgfpoint{152.154221pt}{26.910515pt}}
\pgflineto{\pgfpoint{151.158249pt}{26.439781pt}}
\pgflineto{\pgfpoint{151.158249pt}{26.399979pt}}
\pgfpathclose
\pgfusepath{fill,stroke}
\pgfpathmoveto{\pgfpoint{152.154221pt}{26.399979pt}}
\pgflineto{\pgfpoint{152.154221pt}{26.910515pt}}
\pgflineto{\pgfpoint{151.158249pt}{26.399979pt}}
\pgfpathclose
\pgfusepath{fill,stroke}
\pgfpathmoveto{\pgfpoint{153.150208pt}{26.399979pt}}
\pgflineto{\pgfpoint{152.154221pt}{26.910515pt}}
\pgflineto{\pgfpoint{152.154221pt}{26.399979pt}}
\pgfpathclose
\pgfusepath{fill,stroke}
\pgfpathmoveto{\pgfpoint{145.182343pt}{26.399979pt}}
\pgflineto{\pgfpoint{146.178314pt}{26.399979pt}}
\pgflineto{\pgfpoint{146.178314pt}{26.535568pt}}
\pgfpathclose
\pgfusepath{fill,stroke}
\pgfpathmoveto{\pgfpoint{147.174316pt}{26.399979pt}}
\pgflineto{\pgfpoint{146.178314pt}{26.535568pt}}
\pgflineto{\pgfpoint{146.178314pt}{26.399979pt}}
\pgfpathclose
\pgfusepath{fill,stroke}
\pgfpathmoveto{\pgfpoint{142.194382pt}{26.399979pt}}
\pgflineto{\pgfpoint{143.190369pt}{26.399979pt}}
\pgflineto{\pgfpoint{143.190369pt}{31.207901pt}}
\pgfpathclose
\pgfusepath{fill,stroke}
\pgfpathmoveto{\pgfpoint{144.186356pt}{26.399979pt}}
\pgflineto{\pgfpoint{143.190369pt}{31.207901pt}}
\pgflineto{\pgfpoint{143.190369pt}{26.399979pt}}
\pgfpathclose
\pgfusepath{fill,stroke}
\pgfpathmoveto{\pgfpoint{138.210449pt}{26.399979pt}}
\pgflineto{\pgfpoint{139.206436pt}{26.399979pt}}
\pgflineto{\pgfpoint{139.206436pt}{27.485779pt}}
\pgfpathclose
\pgfusepath{fill,stroke}
\pgfpathmoveto{\pgfpoint{140.202423pt}{27.782883pt}}
\pgflineto{\pgfpoint{139.206436pt}{27.485779pt}}
\pgflineto{\pgfpoint{139.206436pt}{26.399979pt}}
\pgfpathclose
\pgfusepath{fill,stroke}
\pgfpathmoveto{\pgfpoint{140.202423pt}{26.399979pt}}
\pgflineto{\pgfpoint{140.202423pt}{27.782883pt}}
\pgflineto{\pgfpoint{139.206436pt}{26.399979pt}}
\pgfpathclose
\pgfusepath{fill,stroke}
\pgfpathmoveto{\pgfpoint{141.198410pt}{26.399979pt}}
\pgflineto{\pgfpoint{140.202423pt}{27.782883pt}}
\pgflineto{\pgfpoint{140.202423pt}{26.399979pt}}
\pgfpathclose
\pgfusepath{fill,stroke}
\pgfpathmoveto{\pgfpoint{125.262665pt}{26.399979pt}}
\pgflineto{\pgfpoint{126.258652pt}{26.399979pt}}
\pgflineto{\pgfpoint{126.258652pt}{28.660576pt}}
\pgfpathclose
\pgfusepath{fill,stroke}
\pgfpathmoveto{\pgfpoint{127.254631pt}{29.838959pt}}
\pgflineto{\pgfpoint{126.258652pt}{28.660576pt}}
\pgflineto{\pgfpoint{126.258652pt}{26.399979pt}}
\pgfpathclose
\pgfusepath{fill,stroke}
\pgfpathmoveto{\pgfpoint{127.254631pt}{26.399979pt}}
\pgflineto{\pgfpoint{127.254631pt}{29.838959pt}}
\pgflineto{\pgfpoint{126.258652pt}{26.399979pt}}
\pgfpathclose
\pgfusepath{fill,stroke}
\pgfpathmoveto{\pgfpoint{128.250610pt}{26.614319pt}}
\pgflineto{\pgfpoint{127.254631pt}{29.838959pt}}
\pgflineto{\pgfpoint{127.254631pt}{26.399979pt}}
\pgfpathclose
\pgfusepath{fill,stroke}
\pgfpathmoveto{\pgfpoint{128.250610pt}{26.399979pt}}
\pgflineto{\pgfpoint{128.250610pt}{26.614319pt}}
\pgflineto{\pgfpoint{127.254631pt}{26.399979pt}}
\pgfpathclose
\pgfusepath{fill,stroke}
\pgfpathmoveto{\pgfpoint{129.246597pt}{26.451225pt}}
\pgflineto{\pgfpoint{128.250610pt}{26.614319pt}}
\pgflineto{\pgfpoint{128.250610pt}{26.399979pt}}
\pgfpathclose
\pgfusepath{fill,stroke}
\pgfpathmoveto{\pgfpoint{129.246597pt}{26.399979pt}}
\pgflineto{\pgfpoint{129.246597pt}{26.451225pt}}
\pgflineto{\pgfpoint{128.250610pt}{26.399979pt}}
\pgfpathclose
\pgfusepath{fill,stroke}
\pgfpathmoveto{\pgfpoint{130.242584pt}{26.430634pt}}
\pgflineto{\pgfpoint{129.246597pt}{26.451225pt}}
\pgflineto{\pgfpoint{129.246597pt}{26.399979pt}}
\pgfpathclose
\pgfusepath{fill,stroke}
\pgfpathmoveto{\pgfpoint{130.242584pt}{26.399979pt}}
\pgflineto{\pgfpoint{130.242584pt}{26.430634pt}}
\pgflineto{\pgfpoint{129.246597pt}{26.399979pt}}
\pgfpathclose
\pgfusepath{fill,stroke}
\pgfpathmoveto{\pgfpoint{131.238571pt}{27.958008pt}}
\pgflineto{\pgfpoint{130.242584pt}{26.430634pt}}
\pgflineto{\pgfpoint{130.242584pt}{26.399979pt}}
\pgfpathclose
\pgfusepath{fill,stroke}
\pgfpathmoveto{\pgfpoint{131.238571pt}{26.399979pt}}
\pgflineto{\pgfpoint{131.238571pt}{27.958008pt}}
\pgflineto{\pgfpoint{130.242584pt}{26.399979pt}}
\pgfpathclose
\pgfusepath{fill,stroke}
\pgfpathmoveto{\pgfpoint{132.234558pt}{27.682930pt}}
\pgflineto{\pgfpoint{131.238571pt}{27.958008pt}}
\pgflineto{\pgfpoint{131.238571pt}{26.399979pt}}
\pgfpathclose
\pgfusepath{fill,stroke}
\pgfpathmoveto{\pgfpoint{132.234558pt}{26.399979pt}}
\pgflineto{\pgfpoint{132.234558pt}{27.682930pt}}
\pgflineto{\pgfpoint{131.238571pt}{26.399979pt}}
\pgfpathclose
\pgfusepath{fill,stroke}
\pgfpathmoveto{\pgfpoint{133.230530pt}{26.399979pt}}
\pgflineto{\pgfpoint{132.234558pt}{27.682930pt}}
\pgflineto{\pgfpoint{132.234558pt}{26.399979pt}}
\pgfpathclose
\pgfusepath{fill,stroke}
\pgfpathmoveto{\pgfpoint{122.274712pt}{26.399979pt}}
\pgflineto{\pgfpoint{123.270691pt}{26.399979pt}}
\pgflineto{\pgfpoint{123.270691pt}{26.975601pt}}
\pgfpathclose
\pgfusepath{fill,stroke}
\pgfpathmoveto{\pgfpoint{124.266678pt}{26.399979pt}}
\pgflineto{\pgfpoint{123.270691pt}{26.975601pt}}
\pgflineto{\pgfpoint{123.270691pt}{26.399979pt}}
\pgfpathclose
\pgfusepath{fill,stroke}
\pgfpathmoveto{\pgfpoint{120.282745pt}{26.399979pt}}
\pgflineto{\pgfpoint{121.278725pt}{26.399979pt}}
\pgflineto{\pgfpoint{121.278725pt}{27.190407pt}}
\pgfpathclose
\pgfusepath{fill,stroke}
\pgfpathmoveto{\pgfpoint{122.274712pt}{26.399979pt}}
\pgflineto{\pgfpoint{121.278725pt}{27.190407pt}}
\pgflineto{\pgfpoint{121.278725pt}{26.399979pt}}
\pgfpathclose
\pgfusepath{fill,stroke}
\pgfpathmoveto{\pgfpoint{115.302826pt}{26.399979pt}}
\pgflineto{\pgfpoint{116.298813pt}{26.399979pt}}
\pgflineto{\pgfpoint{116.298813pt}{29.134193pt}}
\pgfpathclose
\pgfusepath{fill,stroke}
\pgfpathmoveto{\pgfpoint{117.294792pt}{26.815964pt}}
\pgflineto{\pgfpoint{116.298813pt}{29.134193pt}}
\pgflineto{\pgfpoint{116.298813pt}{26.399979pt}}
\pgfpathclose
\pgfusepath{fill,stroke}
\pgfpathmoveto{\pgfpoint{117.294792pt}{26.399979pt}}
\pgflineto{\pgfpoint{117.294792pt}{26.815964pt}}
\pgflineto{\pgfpoint{116.298813pt}{26.399979pt}}
\pgfpathclose
\pgfusepath{fill,stroke}
\pgfpathmoveto{\pgfpoint{117.967636pt}{91.264793pt}}
\pgflineto{\pgfpoint{117.294792pt}{26.815964pt}}
\pgflineto{\pgfpoint{117.294792pt}{26.399979pt}}
\pgfpathclose
\pgfusepath{fill,stroke}
\pgfpathmoveto{\pgfpoint{117.967636pt}{91.264793pt}}
\pgflineto{\pgfpoint{117.966232pt}{91.264786pt}}
\pgflineto{\pgfpoint{117.294792pt}{26.815964pt}}
\pgfpathclose
\pgfusepath{fill,stroke}
\pgfpathmoveto{\pgfpoint{118.290779pt}{26.399979pt}}
\pgflineto{\pgfpoint{117.967636pt}{91.264793pt}}
\pgflineto{\pgfpoint{117.294792pt}{26.399979pt}}
\pgfpathclose
\pgfusepath{fill,stroke}
\pgfpathmoveto{\pgfpoint{118.290779pt}{26.399979pt}}
\pgflineto{\pgfpoint{118.290779pt}{91.264793pt}}
\pgflineto{\pgfpoint{117.967636pt}{91.264793pt}}
\pgfpathclose
\pgfusepath{fill,stroke}
\pgfpathmoveto{\pgfpoint{119.286758pt}{26.399979pt}}
\pgflineto{\pgfpoint{118.290779pt}{91.264793pt}}
\pgflineto{\pgfpoint{118.290779pt}{26.399979pt}}
\pgfpathclose
\pgfusepath{fill,stroke}
\pgfpathmoveto{\pgfpoint{119.286758pt}{26.399979pt}}
\pgflineto{\pgfpoint{118.613922pt}{91.264793pt}}
\pgflineto{\pgfpoint{118.290779pt}{91.264793pt}}
\pgfpathclose
\pgfusepath{fill,stroke}
\pgfpathmoveto{\pgfpoint{111.318893pt}{26.399979pt}}
\pgflineto{\pgfpoint{112.314873pt}{26.399979pt}}
\pgflineto{\pgfpoint{112.314873pt}{37.362213pt}}
\pgfpathclose
\pgfusepath{fill,stroke}
\pgfpathmoveto{\pgfpoint{113.310852pt}{26.517036pt}}
\pgflineto{\pgfpoint{112.314873pt}{37.362213pt}}
\pgflineto{\pgfpoint{112.314873pt}{26.399979pt}}
\pgfpathclose
\pgfusepath{fill,stroke}
\pgfpathmoveto{\pgfpoint{113.310852pt}{26.399979pt}}
\pgflineto{\pgfpoint{113.310852pt}{26.517036pt}}
\pgflineto{\pgfpoint{112.314873pt}{26.399979pt}}
\pgfpathclose
\pgfusepath{fill,stroke}
\pgfpathmoveto{\pgfpoint{114.306839pt}{26.399979pt}}
\pgflineto{\pgfpoint{113.310852pt}{26.517036pt}}
\pgflineto{\pgfpoint{113.310852pt}{26.399979pt}}
\pgfpathclose
\pgfusepath{fill,stroke}
\pgfpathmoveto{\pgfpoint{107.334953pt}{26.399979pt}}
\pgflineto{\pgfpoint{108.330933pt}{26.399979pt}}
\pgflineto{\pgfpoint{108.330933pt}{27.150108pt}}
\pgfpathclose
\pgfusepath{fill,stroke}
\pgfpathmoveto{\pgfpoint{109.326920pt}{28.632256pt}}
\pgflineto{\pgfpoint{108.330933pt}{27.150108pt}}
\pgflineto{\pgfpoint{108.330933pt}{26.399979pt}}
\pgfpathclose
\pgfusepath{fill,stroke}
\pgfpathmoveto{\pgfpoint{109.326920pt}{26.399979pt}}
\pgflineto{\pgfpoint{109.326920pt}{28.632256pt}}
\pgflineto{\pgfpoint{108.330933pt}{26.399979pt}}
\pgfpathclose
\pgfusepath{fill,stroke}
\pgfpathmoveto{\pgfpoint{110.322906pt}{26.399979pt}}
\pgflineto{\pgfpoint{109.326920pt}{28.632256pt}}
\pgflineto{\pgfpoint{109.326920pt}{26.399979pt}}
\pgfpathclose
\pgfusepath{fill,stroke}
\pgfpathmoveto{\pgfpoint{103.351013pt}{26.399979pt}}
\pgflineto{\pgfpoint{104.347000pt}{26.399979pt}}
\pgflineto{\pgfpoint{104.347000pt}{35.689026pt}}
\pgfpathclose
\pgfusepath{fill,stroke}
\pgfpathmoveto{\pgfpoint{105.342987pt}{26.399979pt}}
\pgflineto{\pgfpoint{104.347000pt}{35.689026pt}}
\pgflineto{\pgfpoint{104.347000pt}{26.399979pt}}
\pgfpathclose
\pgfusepath{fill,stroke}
\pgfpathmoveto{\pgfpoint{101.359047pt}{26.399979pt}}
\pgflineto{\pgfpoint{102.355034pt}{26.399979pt}}
\pgflineto{\pgfpoint{102.355034pt}{26.959930pt}}
\pgfpathclose
\pgfusepath{fill,stroke}
\pgfpathmoveto{\pgfpoint{103.351013pt}{26.399979pt}}
\pgflineto{\pgfpoint{102.355034pt}{26.959930pt}}
\pgflineto{\pgfpoint{102.355034pt}{26.399979pt}}
\pgfpathclose
\pgfusepath{fill,stroke}
\pgfpathmoveto{\pgfpoint{97.375107pt}{26.399979pt}}
\pgflineto{\pgfpoint{98.371094pt}{26.399979pt}}
\pgflineto{\pgfpoint{98.371094pt}{47.782463pt}}
\pgfpathclose
\pgfusepath{fill,stroke}
\pgfpathmoveto{\pgfpoint{99.367081pt}{54.046951pt}}
\pgflineto{\pgfpoint{98.371094pt}{47.782463pt}}
\pgflineto{\pgfpoint{98.371094pt}{26.399979pt}}
\pgfpathclose
\pgfusepath{fill,stroke}
\pgfpathmoveto{\pgfpoint{99.367081pt}{26.399979pt}}
\pgflineto{\pgfpoint{99.367081pt}{54.046951pt}}
\pgflineto{\pgfpoint{98.371094pt}{26.399979pt}}
\pgfpathclose
\pgfusepath{fill,stroke}
\pgfpathmoveto{\pgfpoint{100.363068pt}{32.883598pt}}
\pgflineto{\pgfpoint{99.367081pt}{54.046951pt}}
\pgflineto{\pgfpoint{99.367081pt}{26.399979pt}}
\pgfpathclose
\pgfusepath{fill,stroke}
\pgfpathmoveto{\pgfpoint{100.363068pt}{26.399979pt}}
\pgflineto{\pgfpoint{100.363068pt}{32.883598pt}}
\pgflineto{\pgfpoint{99.367081pt}{26.399979pt}}
\pgfpathclose
\pgfusepath{fill,stroke}
\pgfpathmoveto{\pgfpoint{101.359047pt}{26.399979pt}}
\pgflineto{\pgfpoint{100.363068pt}{32.883598pt}}
\pgflineto{\pgfpoint{100.363068pt}{26.399979pt}}
\pgfpathclose
\pgfusepath{fill,stroke}
\pgfpathmoveto{\pgfpoint{91.399208pt}{26.399979pt}}
\pgflineto{\pgfpoint{92.395187pt}{26.399979pt}}
\pgflineto{\pgfpoint{92.395187pt}{26.889816pt}}
\pgfpathclose
\pgfusepath{fill,stroke}
\pgfpathmoveto{\pgfpoint{93.391174pt}{27.006203pt}}
\pgflineto{\pgfpoint{92.395187pt}{26.889816pt}}
\pgflineto{\pgfpoint{92.395187pt}{26.399979pt}}
\pgfpathclose
\pgfusepath{fill,stroke}
\pgfpathmoveto{\pgfpoint{93.391174pt}{26.399979pt}}
\pgflineto{\pgfpoint{93.391174pt}{27.006203pt}}
\pgflineto{\pgfpoint{92.395187pt}{26.399979pt}}
\pgfpathclose
\pgfusepath{fill,stroke}
\pgfpathmoveto{\pgfpoint{94.387161pt}{26.399979pt}}
\pgflineto{\pgfpoint{93.391174pt}{27.006203pt}}
\pgflineto{\pgfpoint{93.391174pt}{26.399979pt}}
\pgfpathclose
\pgfusepath{fill,stroke}
\pgfpathmoveto{\pgfpoint{89.407242pt}{26.399979pt}}
\pgflineto{\pgfpoint{90.403221pt}{26.399979pt}}
\pgflineto{\pgfpoint{90.403221pt}{26.454811pt}}
\pgfpathclose
\pgfusepath{fill,stroke}
\pgfpathmoveto{\pgfpoint{91.399208pt}{26.399979pt}}
\pgflineto{\pgfpoint{90.403221pt}{26.454811pt}}
\pgflineto{\pgfpoint{90.403221pt}{26.399979pt}}
\pgfpathclose
\pgfusepath{fill,stroke}
\pgfpathmoveto{\pgfpoint{87.415276pt}{26.399979pt}}
\pgflineto{\pgfpoint{88.411255pt}{26.399979pt}}
\pgflineto{\pgfpoint{88.411255pt}{31.944191pt}}
\pgfpathclose
\pgfusepath{fill,stroke}
\pgfpathmoveto{\pgfpoint{89.407242pt}{26.399979pt}}
\pgflineto{\pgfpoint{88.411255pt}{31.944191pt}}
\pgflineto{\pgfpoint{88.411255pt}{26.399979pt}}
\pgfpathclose
\pgfusepath{fill,stroke}
\pgfpathmoveto{\pgfpoint{85.423309pt}{26.399979pt}}
\pgflineto{\pgfpoint{86.419289pt}{26.399979pt}}
\pgflineto{\pgfpoint{86.419289pt}{26.871178pt}}
\pgfpathclose
\pgfusepath{fill,stroke}
\pgfpathmoveto{\pgfpoint{87.415276pt}{26.399979pt}}
\pgflineto{\pgfpoint{86.419289pt}{26.871178pt}}
\pgflineto{\pgfpoint{86.419289pt}{26.399979pt}}
\pgfpathclose
\pgfusepath{fill,stroke}
\pgfpathmoveto{\pgfpoint{79.447403pt}{26.399979pt}}
\pgflineto{\pgfpoint{80.443390pt}{26.399979pt}}
\pgflineto{\pgfpoint{80.443390pt}{28.913742pt}}
\pgfpathclose
\pgfusepath{fill,stroke}
\pgfpathmoveto{\pgfpoint{81.439369pt}{28.404305pt}}
\pgflineto{\pgfpoint{80.443390pt}{28.913742pt}}
\pgflineto{\pgfpoint{80.443390pt}{26.399979pt}}
\pgfpathclose
\pgfusepath{fill,stroke}
\pgfpathmoveto{\pgfpoint{81.439369pt}{26.399979pt}}
\pgflineto{\pgfpoint{81.439369pt}{28.404305pt}}
\pgflineto{\pgfpoint{80.443390pt}{26.399979pt}}
\pgfpathclose
\pgfusepath{fill,stroke}
\pgfpathmoveto{\pgfpoint{82.435356pt}{28.310295pt}}
\pgflineto{\pgfpoint{81.439369pt}{28.404305pt}}
\pgflineto{\pgfpoint{81.439369pt}{26.399979pt}}
\pgfpathclose
\pgfusepath{fill,stroke}
\pgfpathmoveto{\pgfpoint{82.435356pt}{26.399979pt}}
\pgflineto{\pgfpoint{82.435356pt}{28.310295pt}}
\pgflineto{\pgfpoint{81.439369pt}{26.399979pt}}
\pgfpathclose
\pgfusepath{fill,stroke}
\pgfpathmoveto{\pgfpoint{83.431335pt}{26.399979pt}}
\pgflineto{\pgfpoint{82.435356pt}{28.310295pt}}
\pgflineto{\pgfpoint{82.435356pt}{26.399979pt}}
\pgfpathclose
\pgfusepath{fill,stroke}
\pgfpathmoveto{\pgfpoint{73.471497pt}{26.399979pt}}
\pgflineto{\pgfpoint{74.467484pt}{26.399979pt}}
\pgflineto{\pgfpoint{74.467484pt}{26.868034pt}}
\pgfpathclose
\pgfusepath{fill,stroke}
\pgfpathmoveto{\pgfpoint{75.463470pt}{26.470932pt}}
\pgflineto{\pgfpoint{74.467484pt}{26.868034pt}}
\pgflineto{\pgfpoint{74.467484pt}{26.399979pt}}
\pgfpathclose
\pgfusepath{fill,stroke}
\pgfpathmoveto{\pgfpoint{75.463470pt}{26.399979pt}}
\pgflineto{\pgfpoint{75.463470pt}{26.470932pt}}
\pgflineto{\pgfpoint{74.467484pt}{26.399979pt}}
\pgfpathclose
\pgfusepath{fill,stroke}
\pgfpathmoveto{\pgfpoint{76.459442pt}{26.399979pt}}
\pgflineto{\pgfpoint{75.463470pt}{26.470932pt}}
\pgflineto{\pgfpoint{75.463470pt}{26.399979pt}}
\pgfpathclose
\pgfusepath{fill,stroke}
\pgfpathmoveto{\pgfpoint{70.483551pt}{26.399979pt}}
\pgflineto{\pgfpoint{71.479530pt}{26.399979pt}}
\pgflineto{\pgfpoint{71.479530pt}{52.145065pt}}
\pgfpathclose
\pgfusepath{fill,stroke}
\pgfpathmoveto{\pgfpoint{72.475510pt}{31.143661pt}}
\pgflineto{\pgfpoint{71.479530pt}{52.145065pt}}
\pgflineto{\pgfpoint{71.479530pt}{26.399979pt}}
\pgfpathclose
\pgfusepath{fill,stroke}
\pgfpathmoveto{\pgfpoint{72.475510pt}{26.399979pt}}
\pgflineto{\pgfpoint{72.475510pt}{31.143661pt}}
\pgflineto{\pgfpoint{71.479530pt}{26.399979pt}}
\pgfpathclose
\pgfusepath{fill,stroke}
\pgfpathmoveto{\pgfpoint{73.471497pt}{26.399979pt}}
\pgflineto{\pgfpoint{72.475510pt}{31.143661pt}}
\pgflineto{\pgfpoint{72.475510pt}{26.399979pt}}
\pgfpathclose
\pgfusepath{fill,stroke}
\pgfpathmoveto{\pgfpoint{67.495590pt}{26.399979pt}}
\pgflineto{\pgfpoint{68.491577pt}{26.399979pt}}
\pgflineto{\pgfpoint{68.491577pt}{26.687355pt}}
\pgfpathclose
\pgfusepath{fill,stroke}
\pgfpathmoveto{\pgfpoint{69.487564pt}{27.948082pt}}
\pgflineto{\pgfpoint{68.491577pt}{26.687355pt}}
\pgflineto{\pgfpoint{68.491577pt}{26.399979pt}}
\pgfpathclose
\pgfusepath{fill,stroke}
\pgfpathmoveto{\pgfpoint{69.487564pt}{26.399979pt}}
\pgflineto{\pgfpoint{69.487564pt}{27.948082pt}}
\pgflineto{\pgfpoint{68.491577pt}{26.399979pt}}
\pgfpathclose
\pgfusepath{fill,stroke}
\pgfpathmoveto{\pgfpoint{70.483551pt}{26.399979pt}}
\pgflineto{\pgfpoint{69.487564pt}{27.948082pt}}
\pgflineto{\pgfpoint{69.487564pt}{26.399979pt}}
\pgfpathclose
\pgfusepath{fill,stroke}
\pgfpathmoveto{\pgfpoint{63.511658pt}{26.399979pt}}
\pgflineto{\pgfpoint{64.507637pt}{26.399979pt}}
\pgflineto{\pgfpoint{64.507637pt}{26.602051pt}}
\pgfpathclose
\pgfusepath{fill,stroke}
\pgfpathmoveto{\pgfpoint{65.503624pt}{27.212898pt}}
\pgflineto{\pgfpoint{64.507637pt}{26.602051pt}}
\pgflineto{\pgfpoint{64.507637pt}{26.399979pt}}
\pgfpathclose
\pgfusepath{fill,stroke}
\pgfpathmoveto{\pgfpoint{65.503624pt}{26.399979pt}}
\pgflineto{\pgfpoint{65.503624pt}{27.212898pt}}
\pgflineto{\pgfpoint{64.507637pt}{26.399979pt}}
\pgfpathclose
\pgfusepath{fill,stroke}
\pgfpathmoveto{\pgfpoint{66.499619pt}{29.321190pt}}
\pgflineto{\pgfpoint{65.503624pt}{27.212898pt}}
\pgflineto{\pgfpoint{65.503624pt}{26.399979pt}}
\pgfpathclose
\pgfusepath{fill,stroke}
\pgfpathmoveto{\pgfpoint{66.499619pt}{26.399979pt}}
\pgflineto{\pgfpoint{66.499619pt}{29.321190pt}}
\pgflineto{\pgfpoint{65.503624pt}{26.399979pt}}
\pgfpathclose
\pgfusepath{fill,stroke}
\pgfpathmoveto{\pgfpoint{67.495590pt}{26.399979pt}}
\pgflineto{\pgfpoint{66.499619pt}{29.321190pt}}
\pgflineto{\pgfpoint{66.499619pt}{26.399979pt}}
\pgfpathclose
\pgfusepath{fill,stroke}
\pgfpathmoveto{\pgfpoint{61.519691pt}{26.399979pt}}
\pgflineto{\pgfpoint{62.515678pt}{26.399979pt}}
\pgflineto{\pgfpoint{62.515678pt}{26.488380pt}}
\pgfpathclose
\pgfusepath{fill,stroke}
\pgfpathmoveto{\pgfpoint{63.511658pt}{26.399979pt}}
\pgflineto{\pgfpoint{62.515678pt}{26.488380pt}}
\pgflineto{\pgfpoint{62.515678pt}{26.399979pt}}
\pgfpathclose
\pgfusepath{fill,stroke}
\pgfpathmoveto{\pgfpoint{57.535751pt}{26.399979pt}}
\pgflineto{\pgfpoint{58.531738pt}{26.399979pt}}
\pgflineto{\pgfpoint{58.531738pt}{27.217796pt}}
\pgfpathclose
\pgfusepath{fill,stroke}
\pgfpathmoveto{\pgfpoint{59.527725pt}{26.723701pt}}
\pgflineto{\pgfpoint{58.531738pt}{27.217796pt}}
\pgflineto{\pgfpoint{58.531738pt}{26.399979pt}}
\pgfpathclose
\pgfusepath{fill,stroke}
\pgfpathmoveto{\pgfpoint{59.527725pt}{26.399979pt}}
\pgflineto{\pgfpoint{59.527725pt}{26.723701pt}}
\pgflineto{\pgfpoint{58.531738pt}{26.399979pt}}
\pgfpathclose
\pgfusepath{fill,stroke}
\pgfpathmoveto{\pgfpoint{60.523712pt}{26.399979pt}}
\pgflineto{\pgfpoint{59.527725pt}{26.723701pt}}
\pgflineto{\pgfpoint{59.527725pt}{26.399979pt}}
\pgfpathclose
\pgfusepath{fill,stroke}
\pgfpathmoveto{\pgfpoint{54.547806pt}{26.399979pt}}
\pgflineto{\pgfpoint{55.543785pt}{26.399979pt}}
\pgflineto{\pgfpoint{55.543785pt}{26.577354pt}}
\pgfpathclose
\pgfusepath{fill,stroke}
\pgfpathmoveto{\pgfpoint{56.539772pt}{65.803238pt}}
\pgflineto{\pgfpoint{55.543785pt}{26.577354pt}}
\pgflineto{\pgfpoint{55.543785pt}{26.399979pt}}
\pgfpathclose
\pgfusepath{fill,stroke}
\pgfpathmoveto{\pgfpoint{56.539772pt}{26.399979pt}}
\pgflineto{\pgfpoint{56.539772pt}{65.803238pt}}
\pgflineto{\pgfpoint{55.543785pt}{26.399979pt}}
\pgfpathclose
\pgfusepath{fill,stroke}
\pgfpathmoveto{\pgfpoint{57.535751pt}{26.399979pt}}
\pgflineto{\pgfpoint{56.539772pt}{65.803238pt}}
\pgflineto{\pgfpoint{56.539772pt}{26.399979pt}}
\pgfpathclose
\pgfusepath{fill,stroke}
\pgfpathmoveto{\pgfpoint{52.555840pt}{26.399979pt}}
\pgflineto{\pgfpoint{53.551819pt}{26.399979pt}}
\pgflineto{\pgfpoint{53.551819pt}{32.692734pt}}
\pgfpathclose
\pgfusepath{fill,stroke}
\pgfpathmoveto{\pgfpoint{54.547806pt}{26.399979pt}}
\pgflineto{\pgfpoint{53.551819pt}{32.692734pt}}
\pgflineto{\pgfpoint{53.551819pt}{26.399979pt}}
\pgfpathclose
\pgfusepath{fill,stroke}
\pgfpathmoveto{\pgfpoint{46.579933pt}{26.399979pt}}
\pgflineto{\pgfpoint{47.575912pt}{26.399979pt}}
\pgflineto{\pgfpoint{47.575912pt}{27.559311pt}}
\pgfpathclose
\pgfusepath{fill,stroke}
\pgfpathmoveto{\pgfpoint{48.571899pt}{26.399979pt}}
\pgflineto{\pgfpoint{47.575912pt}{27.559311pt}}
\pgflineto{\pgfpoint{47.575912pt}{26.399979pt}}
\pgfpathclose
\pgfusepath{fill,stroke}
\pgfpathmoveto{\pgfpoint{42.595993pt}{26.399979pt}}
\pgflineto{\pgfpoint{43.591980pt}{26.399979pt}}
\pgflineto{\pgfpoint{43.591980pt}{33.198105pt}}
\pgfpathclose
\pgfusepath{fill,stroke}
\pgfpathmoveto{\pgfpoint{44.587967pt}{26.399979pt}}
\pgflineto{\pgfpoint{43.591980pt}{33.198105pt}}
\pgflineto{\pgfpoint{43.591980pt}{26.399979pt}}
\pgfpathclose
\pgfusepath{fill,stroke}
\color[rgb]{0.000000,0.000000,0.000000}
\pgfsetlinewidth{0.500000pt}
\pgfsetdash{{16pt}{0pt}}{0pt}
\pgfpathmoveto{\pgfpoint{289.600037pt}{140.777435pt}}
\pgflineto{\pgfpoint{41.600006pt}{140.777435pt}}
\pgfusepath{stroke}
\pgfpathmoveto{\pgfpoint{289.600037pt}{205.577454pt}}
\pgflineto{\pgfpoint{41.600006pt}{205.577454pt}}
\pgfusepath{stroke}
\pgfpathmoveto{\pgfpoint{41.600006pt}{205.577454pt}}
\pgflineto{\pgfpoint{41.600006pt}{140.777435pt}}
\pgfusepath{stroke}
\pgfpathmoveto{\pgfpoint{289.600037pt}{205.577454pt}}
\pgflineto{\pgfpoint{289.600037pt}{140.777435pt}}
\pgfusepath{stroke}
\pgfpathmoveto{\pgfpoint{90.403221pt}{143.249802pt}}
\pgflineto{\pgfpoint{90.403221pt}{140.777435pt}}
\pgfusepath{stroke}
\pgfpathmoveto{\pgfpoint{90.403221pt}{203.105072pt}}
\pgflineto{\pgfpoint{90.403221pt}{205.577454pt}}
\pgfusepath{stroke}
\pgfpathmoveto{\pgfpoint{140.202423pt}{143.249802pt}}
\pgflineto{\pgfpoint{140.202423pt}{140.777435pt}}
\pgfusepath{stroke}
\pgfpathmoveto{\pgfpoint{140.202423pt}{203.105072pt}}
\pgflineto{\pgfpoint{140.202423pt}{205.577454pt}}
\pgfusepath{stroke}
\pgfpathmoveto{\pgfpoint{190.001617pt}{143.249802pt}}
\pgflineto{\pgfpoint{190.001617pt}{140.777435pt}}
\pgfusepath{stroke}
\pgfpathmoveto{\pgfpoint{190.001617pt}{203.105072pt}}
\pgflineto{\pgfpoint{190.001617pt}{205.577454pt}}
\pgfusepath{stroke}
\pgfpathmoveto{\pgfpoint{239.800812pt}{143.249802pt}}
\pgflineto{\pgfpoint{239.800812pt}{140.777435pt}}
\pgfusepath{stroke}
\pgfpathmoveto{\pgfpoint{239.800812pt}{203.105072pt}}
\pgflineto{\pgfpoint{239.800812pt}{205.577454pt}}
\pgfusepath{stroke}
\pgfpathmoveto{\pgfpoint{289.600037pt}{143.249802pt}}
\pgflineto{\pgfpoint{289.600037pt}{140.777435pt}}
\pgfusepath{stroke}
\pgfpathmoveto{\pgfpoint{289.600037pt}{203.105072pt}}
\pgflineto{\pgfpoint{289.600037pt}{205.577454pt}}
\pgfusepath{stroke}
{
\pgftransformshift{\pgfpoint{90.403229pt}{135.792816pt}}
\pgfnode{rectangle}{north}{\fontsize{10}{0}\selectfont\textcolor[rgb]{0,0,0}{{50}}}{}{\pgfusepath{discard}}}
{
\pgftransformshift{\pgfpoint{140.202423pt}{135.792816pt}}
\pgfnode{rectangle}{north}{\fontsize{10}{0}\selectfont\textcolor[rgb]{0,0,0}{{100}}}{}{\pgfusepath{discard}}}
{
\pgftransformshift{\pgfpoint{190.001617pt}{135.792816pt}}
\pgfnode{rectangle}{north}{\fontsize{10}{0}\selectfont\textcolor[rgb]{0,0,0}{{150}}}{}{\pgfusepath{discard}}}
{
\pgftransformshift{\pgfpoint{239.800812pt}{135.792816pt}}
\pgfnode{rectangle}{north}{\fontsize{10}{0}\selectfont\textcolor[rgb]{0,0,0}{{200}}}{}{\pgfusepath{discard}}}
{
\pgftransformshift{\pgfpoint{289.600037pt}{135.792816pt}}
\pgfnode{rectangle}{north}{\fontsize{10}{0}\selectfont\textcolor[rgb]{0,0,0}{{250}}}{}{\pgfusepath{discard}}}
\pgfpathmoveto{\pgfpoint{44.080009pt}{140.777435pt}}
\pgflineto{\pgfpoint{41.600006pt}{140.777435pt}}
\pgfusepath{stroke}
\pgfpathmoveto{\pgfpoint{287.119995pt}{140.777435pt}}
\pgflineto{\pgfpoint{289.600037pt}{140.777435pt}}
\pgfusepath{stroke}
\pgfpathmoveto{\pgfpoint{44.080009pt}{153.737442pt}}
\pgflineto{\pgfpoint{41.600006pt}{153.737442pt}}
\pgfusepath{stroke}
\pgfpathmoveto{\pgfpoint{287.119995pt}{153.737442pt}}
\pgflineto{\pgfpoint{289.600037pt}{153.737442pt}}
\pgfusepath{stroke}
\pgfpathmoveto{\pgfpoint{44.080009pt}{166.697433pt}}
\pgflineto{\pgfpoint{41.600006pt}{166.697433pt}}
\pgfusepath{stroke}
\pgfpathmoveto{\pgfpoint{287.119995pt}{166.697433pt}}
\pgflineto{\pgfpoint{289.600037pt}{166.697433pt}}
\pgfusepath{stroke}
\pgfpathmoveto{\pgfpoint{44.080009pt}{179.657440pt}}
\pgflineto{\pgfpoint{41.600006pt}{179.657440pt}}
\pgfusepath{stroke}
\pgfpathmoveto{\pgfpoint{287.119995pt}{179.657440pt}}
\pgflineto{\pgfpoint{289.600037pt}{179.657440pt}}
\pgfusepath{stroke}
\pgfpathmoveto{\pgfpoint{44.080009pt}{192.617432pt}}
\pgflineto{\pgfpoint{41.600006pt}{192.617432pt}}
\pgfusepath{stroke}
\pgfpathmoveto{\pgfpoint{287.119995pt}{192.617432pt}}
\pgflineto{\pgfpoint{289.600037pt}{192.617432pt}}
\pgfusepath{stroke}
\pgfpathmoveto{\pgfpoint{44.080009pt}{205.577454pt}}
\pgflineto{\pgfpoint{41.600006pt}{205.577454pt}}
\pgfusepath{stroke}
\pgfpathmoveto{\pgfpoint{287.119995pt}{205.577454pt}}
\pgflineto{\pgfpoint{289.600037pt}{205.577454pt}}
\pgfusepath{stroke}
{
\pgftransformshift{\pgfpoint{36.600006pt}{140.777435pt}}
\pgfnode{rectangle}{east}{\fontsize{10}{0}\selectfont\textcolor[rgb]{0,0,0}{{0}}}{}{\pgfusepath{discard}}}
{
\pgftransformshift{\pgfpoint{36.600006pt}{153.737442pt}}
\pgfnode{rectangle}{east}{\fontsize{10}{0}\selectfont\textcolor[rgb]{0,0,0}{{2e+06}}}{}{\pgfusepath{discard}}}
{
\pgftransformshift{\pgfpoint{36.600006pt}{166.697433pt}}
\pgfnode{rectangle}{east}{\fontsize{10}{0}\selectfont\textcolor[rgb]{0,0,0}{{4e+06}}}{}{\pgfusepath{discard}}}
{
\pgftransformshift{\pgfpoint{36.600006pt}{179.657440pt}}
\pgfnode{rectangle}{east}{\fontsize{10}{0}\selectfont\textcolor[rgb]{0,0,0}{{6e+06}}}{}{\pgfusepath{discard}}}
{
\pgftransformshift{\pgfpoint{36.600006pt}{192.617432pt}}
\pgfnode{rectangle}{east}{\fontsize{10}{0}\selectfont\textcolor[rgb]{0,0,0}{{8e+06}}}{}{\pgfusepath{discard}}}
{
\pgftransformshift{\pgfpoint{36.600006pt}{205.577454pt}}
\pgfnode{rectangle}{east}{\fontsize{10}{0}\selectfont\textcolor[rgb]{0,0,0}{{1e+07}}}{}{\pgfusepath{discard}}}
\pgfsetlinewidth{0.000100pt}
\pgfsetdash{}{0pt}
\pgfpathmoveto{\pgfpoint{43.591980pt}{140.777435pt}}
\pgflineto{\pgfpoint{44.587967pt}{140.777435pt}}
\pgfusepath{stroke}
\pgfpathmoveto{\pgfpoint{42.595993pt}{140.777435pt}}
\pgflineto{\pgfpoint{43.591980pt}{140.777435pt}}
\pgfusepath{stroke}
\pgfpathmoveto{\pgfpoint{43.591980pt}{186.324295pt}}
\pgflineto{\pgfpoint{42.595993pt}{140.777435pt}}
\pgfusepath{stroke}
\pgfpathmoveto{\pgfpoint{44.587967pt}{140.777435pt}}
\pgflineto{\pgfpoint{43.591980pt}{186.324295pt}}
\pgfusepath{stroke}
\pgfpathmoveto{\pgfpoint{45.583946pt}{140.777435pt}}
\pgflineto{\pgfpoint{46.579933pt}{140.777435pt}}
\pgfusepath{stroke}
\pgfpathmoveto{\pgfpoint{44.587967pt}{140.777435pt}}
\pgflineto{\pgfpoint{45.583946pt}{140.777435pt}}
\pgfusepath{stroke}
\pgfpathmoveto{\pgfpoint{45.583946pt}{148.946091pt}}
\pgflineto{\pgfpoint{44.587967pt}{140.777435pt}}
\pgfusepath{stroke}
\pgfpathmoveto{\pgfpoint{46.579933pt}{140.777435pt}}
\pgflineto{\pgfpoint{45.583946pt}{148.946091pt}}
\pgfusepath{stroke}
\pgfpathmoveto{\pgfpoint{49.567879pt}{140.777435pt}}
\pgflineto{\pgfpoint{50.563873pt}{140.777435pt}}
\pgfusepath{stroke}
\pgfpathmoveto{\pgfpoint{48.571899pt}{140.777435pt}}
\pgflineto{\pgfpoint{49.567879pt}{140.777435pt}}
\pgfusepath{stroke}
\pgfpathmoveto{\pgfpoint{47.575912pt}{140.777435pt}}
\pgflineto{\pgfpoint{48.571899pt}{140.777435pt}}
\pgfusepath{stroke}
\pgfpathmoveto{\pgfpoint{46.579933pt}{140.777435pt}}
\pgflineto{\pgfpoint{47.575912pt}{140.777435pt}}
\pgfusepath{stroke}
\pgfpathmoveto{\pgfpoint{47.575912pt}{141.646210pt}}
\pgflineto{\pgfpoint{46.579933pt}{140.777435pt}}
\pgfusepath{stroke}
\pgfpathmoveto{\pgfpoint{48.571899pt}{141.530609pt}}
\pgflineto{\pgfpoint{47.575912pt}{141.646210pt}}
\pgfusepath{stroke}
\pgfpathmoveto{\pgfpoint{49.567879pt}{143.458160pt}}
\pgflineto{\pgfpoint{48.571899pt}{141.530609pt}}
\pgfusepath{stroke}
\pgfpathmoveto{\pgfpoint{50.563873pt}{140.777435pt}}
\pgflineto{\pgfpoint{49.567879pt}{143.458160pt}}
\pgfusepath{stroke}
\pgfpathmoveto{\pgfpoint{66.499619pt}{140.777435pt}}
\pgflineto{\pgfpoint{67.495590pt}{140.777435pt}}
\pgfusepath{stroke}
\pgfpathmoveto{\pgfpoint{65.503624pt}{140.777435pt}}
\pgflineto{\pgfpoint{66.499619pt}{140.777435pt}}
\pgfusepath{stroke}
\pgfpathmoveto{\pgfpoint{64.507637pt}{140.777435pt}}
\pgflineto{\pgfpoint{65.503624pt}{140.777435pt}}
\pgfusepath{stroke}
\pgfpathmoveto{\pgfpoint{63.511658pt}{140.777435pt}}
\pgflineto{\pgfpoint{64.507637pt}{140.777435pt}}
\pgfusepath{stroke}
\pgfpathmoveto{\pgfpoint{62.515678pt}{140.777435pt}}
\pgflineto{\pgfpoint{63.511658pt}{140.777435pt}}
\pgfusepath{stroke}
\pgfpathmoveto{\pgfpoint{61.519691pt}{140.777435pt}}
\pgflineto{\pgfpoint{62.515678pt}{140.777435pt}}
\pgfusepath{stroke}
\pgfpathmoveto{\pgfpoint{60.523712pt}{140.777435pt}}
\pgflineto{\pgfpoint{61.519691pt}{140.777435pt}}
\pgfusepath{stroke}
\pgfpathmoveto{\pgfpoint{59.527725pt}{140.777435pt}}
\pgflineto{\pgfpoint{60.523712pt}{140.777435pt}}
\pgfusepath{stroke}
\pgfpathmoveto{\pgfpoint{58.531738pt}{140.777435pt}}
\pgflineto{\pgfpoint{59.527725pt}{140.777435pt}}
\pgfusepath{stroke}
\pgfpathmoveto{\pgfpoint{57.535751pt}{140.777435pt}}
\pgflineto{\pgfpoint{58.531738pt}{140.777435pt}}
\pgfusepath{stroke}
\pgfpathmoveto{\pgfpoint{56.539772pt}{140.777435pt}}
\pgflineto{\pgfpoint{57.535751pt}{140.777435pt}}
\pgfusepath{stroke}
\pgfpathmoveto{\pgfpoint{55.543785pt}{140.777435pt}}
\pgflineto{\pgfpoint{56.539772pt}{140.777435pt}}
\pgfusepath{stroke}
\pgfpathmoveto{\pgfpoint{54.547806pt}{140.777435pt}}
\pgflineto{\pgfpoint{55.543785pt}{140.777435pt}}
\pgfusepath{stroke}
\pgfpathmoveto{\pgfpoint{53.551819pt}{140.777435pt}}
\pgflineto{\pgfpoint{54.547806pt}{140.777435pt}}
\pgfusepath{stroke}
\pgfpathmoveto{\pgfpoint{52.555840pt}{140.777435pt}}
\pgflineto{\pgfpoint{53.551819pt}{140.777435pt}}
\pgfusepath{stroke}
\pgfpathmoveto{\pgfpoint{51.559845pt}{140.777435pt}}
\pgflineto{\pgfpoint{52.555840pt}{140.777435pt}}
\pgfusepath{stroke}
\pgfpathmoveto{\pgfpoint{50.563873pt}{140.777435pt}}
\pgflineto{\pgfpoint{51.559845pt}{140.777435pt}}
\pgfusepath{stroke}
\pgfpathmoveto{\pgfpoint{51.559845pt}{146.765289pt}}
\pgflineto{\pgfpoint{50.563873pt}{140.777435pt}}
\pgfusepath{stroke}
\pgfpathmoveto{\pgfpoint{52.555840pt}{142.530624pt}}
\pgflineto{\pgfpoint{51.559845pt}{146.765289pt}}
\pgfusepath{stroke}
\pgfpathmoveto{\pgfpoint{53.551819pt}{146.137848pt}}
\pgflineto{\pgfpoint{52.555840pt}{142.530624pt}}
\pgfusepath{stroke}
\pgfpathmoveto{\pgfpoint{54.547806pt}{142.199554pt}}
\pgflineto{\pgfpoint{53.551819pt}{146.137848pt}}
\pgfusepath{stroke}
\pgfpathmoveto{\pgfpoint{55.543785pt}{196.136078pt}}
\pgflineto{\pgfpoint{54.547806pt}{142.199554pt}}
\pgfusepath{stroke}
\pgfpathmoveto{\pgfpoint{56.539772pt}{168.281158pt}}
\pgflineto{\pgfpoint{55.543785pt}{196.136078pt}}
\pgfusepath{stroke}
\pgfpathmoveto{\pgfpoint{57.535751pt}{147.868973pt}}
\pgflineto{\pgfpoint{56.539772pt}{168.281158pt}}
\pgfusepath{stroke}
\pgfpathmoveto{\pgfpoint{58.531738pt}{141.160339pt}}
\pgflineto{\pgfpoint{57.535751pt}{147.868973pt}}
\pgfusepath{stroke}
\pgfpathmoveto{\pgfpoint{59.527725pt}{141.894562pt}}
\pgflineto{\pgfpoint{58.531738pt}{141.160339pt}}
\pgfusepath{stroke}
\pgfpathmoveto{\pgfpoint{60.523712pt}{142.042740pt}}
\pgflineto{\pgfpoint{59.527725pt}{141.894562pt}}
\pgfusepath{stroke}
\pgfpathmoveto{\pgfpoint{61.519691pt}{141.419281pt}}
\pgflineto{\pgfpoint{60.523712pt}{142.042740pt}}
\pgfusepath{stroke}
\pgfpathmoveto{\pgfpoint{62.515678pt}{202.358673pt}}
\pgflineto{\pgfpoint{61.519691pt}{141.419281pt}}
\pgfusepath{stroke}
\pgfpathmoveto{\pgfpoint{63.511658pt}{143.879868pt}}
\pgflineto{\pgfpoint{62.515678pt}{202.358673pt}}
\pgfusepath{stroke}
\pgfpathmoveto{\pgfpoint{64.507637pt}{144.233887pt}}
\pgflineto{\pgfpoint{63.511658pt}{143.879868pt}}
\pgfusepath{stroke}
\pgfpathmoveto{\pgfpoint{65.503624pt}{142.070892pt}}
\pgflineto{\pgfpoint{64.507637pt}{144.233887pt}}
\pgfusepath{stroke}
\pgfpathmoveto{\pgfpoint{66.499619pt}{141.913147pt}}
\pgflineto{\pgfpoint{65.503624pt}{142.070892pt}}
\pgfusepath{stroke}
\pgfpathmoveto{\pgfpoint{67.495590pt}{140.777435pt}}
\pgflineto{\pgfpoint{66.499619pt}{141.913147pt}}
\pgfusepath{stroke}
\pgfpathmoveto{\pgfpoint{72.475510pt}{140.777435pt}}
\pgflineto{\pgfpoint{73.471497pt}{140.777435pt}}
\pgfusepath{stroke}
\pgfpathmoveto{\pgfpoint{71.479530pt}{140.777435pt}}
\pgflineto{\pgfpoint{72.475510pt}{140.777435pt}}
\pgfusepath{stroke}
\pgfpathmoveto{\pgfpoint{70.483551pt}{140.777435pt}}
\pgflineto{\pgfpoint{71.479530pt}{140.777435pt}}
\pgfusepath{stroke}
\pgfpathmoveto{\pgfpoint{69.487564pt}{140.777435pt}}
\pgflineto{\pgfpoint{70.483551pt}{140.777435pt}}
\pgfusepath{stroke}
\pgfpathmoveto{\pgfpoint{68.491577pt}{140.777435pt}}
\pgflineto{\pgfpoint{69.487564pt}{140.777435pt}}
\pgfusepath{stroke}
\pgfpathmoveto{\pgfpoint{69.487564pt}{149.515793pt}}
\pgflineto{\pgfpoint{68.491577pt}{140.777435pt}}
\pgfusepath{stroke}
\pgfpathmoveto{\pgfpoint{70.483551pt}{141.832703pt}}
\pgflineto{\pgfpoint{69.487564pt}{149.515793pt}}
\pgfusepath{stroke}
\pgfpathmoveto{\pgfpoint{71.479530pt}{190.848999pt}}
\pgflineto{\pgfpoint{70.483551pt}{141.832703pt}}
\pgfusepath{stroke}
\pgfpathmoveto{\pgfpoint{72.475510pt}{148.664902pt}}
\pgflineto{\pgfpoint{71.479530pt}{190.848999pt}}
\pgfusepath{stroke}
\pgfpathmoveto{\pgfpoint{73.471497pt}{140.777435pt}}
\pgflineto{\pgfpoint{72.475510pt}{148.664902pt}}
\pgfusepath{stroke}
\pgfpathmoveto{\pgfpoint{77.455437pt}{140.777435pt}}
\pgflineto{\pgfpoint{78.451424pt}{140.777435pt}}
\pgfusepath{stroke}
\pgfpathmoveto{\pgfpoint{76.459442pt}{140.777435pt}}
\pgflineto{\pgfpoint{77.455437pt}{140.777435pt}}
\pgfusepath{stroke}
\pgfpathmoveto{\pgfpoint{75.463470pt}{140.777435pt}}
\pgflineto{\pgfpoint{76.459442pt}{140.777435pt}}
\pgfusepath{stroke}
\pgfpathmoveto{\pgfpoint{74.467484pt}{140.777435pt}}
\pgflineto{\pgfpoint{75.463470pt}{140.777435pt}}
\pgfusepath{stroke}
\pgfpathmoveto{\pgfpoint{73.471497pt}{140.777435pt}}
\pgflineto{\pgfpoint{74.467484pt}{140.777435pt}}
\pgfusepath{stroke}
\pgfpathmoveto{\pgfpoint{74.467484pt}{141.606003pt}}
\pgflineto{\pgfpoint{73.471497pt}{140.777435pt}}
\pgfusepath{stroke}
\pgfpathmoveto{\pgfpoint{75.463470pt}{145.536850pt}}
\pgflineto{\pgfpoint{74.467484pt}{141.606003pt}}
\pgfusepath{stroke}
\pgfpathmoveto{\pgfpoint{76.459442pt}{143.382019pt}}
\pgflineto{\pgfpoint{75.463470pt}{145.536850pt}}
\pgfusepath{stroke}
\pgfpathmoveto{\pgfpoint{77.455437pt}{142.594910pt}}
\pgflineto{\pgfpoint{76.459442pt}{143.382019pt}}
\pgfusepath{stroke}
\pgfpathmoveto{\pgfpoint{78.451424pt}{140.777435pt}}
\pgflineto{\pgfpoint{77.455437pt}{142.594910pt}}
\pgfusepath{stroke}
\pgfpathmoveto{\pgfpoint{85.423309pt}{140.777435pt}}
\pgflineto{\pgfpoint{86.419289pt}{140.777435pt}}
\pgfusepath{stroke}
\pgfpathmoveto{\pgfpoint{84.427322pt}{140.777435pt}}
\pgflineto{\pgfpoint{85.423309pt}{140.777435pt}}
\pgfusepath{stroke}
\pgfpathmoveto{\pgfpoint{83.431335pt}{140.777435pt}}
\pgflineto{\pgfpoint{84.427322pt}{140.777435pt}}
\pgfusepath{stroke}
\pgfpathmoveto{\pgfpoint{82.435356pt}{140.777435pt}}
\pgflineto{\pgfpoint{83.431335pt}{140.777435pt}}
\pgfusepath{stroke}
\pgfpathmoveto{\pgfpoint{81.439369pt}{140.777435pt}}
\pgflineto{\pgfpoint{82.435356pt}{140.777435pt}}
\pgfusepath{stroke}
\pgfpathmoveto{\pgfpoint{80.443390pt}{140.777435pt}}
\pgflineto{\pgfpoint{81.439369pt}{140.777435pt}}
\pgfusepath{stroke}
\pgfpathmoveto{\pgfpoint{79.447403pt}{140.777435pt}}
\pgflineto{\pgfpoint{80.443390pt}{140.777435pt}}
\pgfusepath{stroke}
\pgfpathmoveto{\pgfpoint{78.451424pt}{140.777435pt}}
\pgflineto{\pgfpoint{79.447403pt}{140.777435pt}}
\pgfusepath{stroke}
\pgfpathmoveto{\pgfpoint{79.447403pt}{164.721649pt}}
\pgflineto{\pgfpoint{78.451424pt}{140.777435pt}}
\pgfusepath{stroke}
\pgfpathmoveto{\pgfpoint{80.443390pt}{141.021835pt}}
\pgflineto{\pgfpoint{79.447403pt}{164.721649pt}}
\pgfusepath{stroke}
\pgfpathmoveto{\pgfpoint{81.439369pt}{141.797729pt}}
\pgflineto{\pgfpoint{80.443390pt}{141.021835pt}}
\pgfusepath{stroke}
\pgfpathmoveto{\pgfpoint{82.435356pt}{141.784378pt}}
\pgflineto{\pgfpoint{81.439369pt}{141.797729pt}}
\pgfusepath{stroke}
\pgfpathmoveto{\pgfpoint{83.431335pt}{147.373245pt}}
\pgflineto{\pgfpoint{82.435356pt}{141.784378pt}}
\pgfusepath{stroke}
\pgfpathmoveto{\pgfpoint{84.427322pt}{140.853500pt}}
\pgflineto{\pgfpoint{83.431335pt}{147.373245pt}}
\pgfusepath{stroke}
\pgfpathmoveto{\pgfpoint{85.423309pt}{140.987396pt}}
\pgflineto{\pgfpoint{84.427322pt}{140.853500pt}}
\pgfusepath{stroke}
\pgfpathmoveto{\pgfpoint{86.419289pt}{140.777435pt}}
\pgflineto{\pgfpoint{85.423309pt}{140.987396pt}}
\pgfusepath{stroke}
\pgfpathmoveto{\pgfpoint{90.403221pt}{140.777435pt}}
\pgflineto{\pgfpoint{91.399208pt}{140.777435pt}}
\pgfusepath{stroke}
\pgfpathmoveto{\pgfpoint{89.407242pt}{140.777435pt}}
\pgflineto{\pgfpoint{90.403221pt}{140.777435pt}}
\pgfusepath{stroke}
\pgfpathmoveto{\pgfpoint{88.411255pt}{140.777435pt}}
\pgflineto{\pgfpoint{89.407242pt}{140.777435pt}}
\pgfusepath{stroke}
\pgfpathmoveto{\pgfpoint{87.415276pt}{140.777435pt}}
\pgflineto{\pgfpoint{88.411255pt}{140.777435pt}}
\pgfusepath{stroke}
\pgfpathmoveto{\pgfpoint{86.419289pt}{140.777435pt}}
\pgflineto{\pgfpoint{87.415276pt}{140.777435pt}}
\pgfusepath{stroke}
\pgfpathmoveto{\pgfpoint{87.415276pt}{152.008636pt}}
\pgflineto{\pgfpoint{86.419289pt}{140.777435pt}}
\pgfusepath{stroke}
\pgfpathmoveto{\pgfpoint{88.411255pt}{163.097900pt}}
\pgflineto{\pgfpoint{87.415276pt}{152.008636pt}}
\pgfusepath{stroke}
\pgfpathmoveto{\pgfpoint{89.407242pt}{147.608887pt}}
\pgflineto{\pgfpoint{88.411255pt}{163.097900pt}}
\pgfusepath{stroke}
\pgfpathmoveto{\pgfpoint{90.403221pt}{189.355881pt}}
\pgflineto{\pgfpoint{89.407242pt}{147.608887pt}}
\pgfusepath{stroke}
\pgfpathmoveto{\pgfpoint{91.399208pt}{140.777435pt}}
\pgflineto{\pgfpoint{90.403221pt}{189.355881pt}}
\pgfusepath{stroke}
\pgfpathmoveto{\pgfpoint{93.391174pt}{140.777435pt}}
\pgflineto{\pgfpoint{94.387161pt}{140.777435pt}}
\pgfusepath{stroke}
\pgfpathmoveto{\pgfpoint{92.395187pt}{140.777435pt}}
\pgflineto{\pgfpoint{93.391174pt}{140.777435pt}}
\pgfusepath{stroke}
\pgfpathmoveto{\pgfpoint{91.399208pt}{140.777435pt}}
\pgflineto{\pgfpoint{92.395187pt}{140.777435pt}}
\pgfusepath{stroke}
\pgfpathmoveto{\pgfpoint{92.395187pt}{140.789795pt}}
\pgflineto{\pgfpoint{91.399208pt}{140.777435pt}}
\pgfusepath{stroke}
\pgfpathmoveto{\pgfpoint{93.391174pt}{141.209564pt}}
\pgflineto{\pgfpoint{92.395187pt}{140.789795pt}}
\pgfusepath{stroke}
\pgfpathmoveto{\pgfpoint{94.387161pt}{140.777435pt}}
\pgflineto{\pgfpoint{93.391174pt}{141.209564pt}}
\pgfusepath{stroke}
\pgfpathmoveto{\pgfpoint{102.355034pt}{140.777435pt}}
\pgflineto{\pgfpoint{103.351013pt}{140.777435pt}}
\pgfusepath{stroke}
\pgfpathmoveto{\pgfpoint{101.359047pt}{140.777435pt}}
\pgflineto{\pgfpoint{102.355034pt}{140.777435pt}}
\pgfusepath{stroke}
\pgfpathmoveto{\pgfpoint{100.363068pt}{140.777435pt}}
\pgflineto{\pgfpoint{101.359047pt}{140.777435pt}}
\pgfusepath{stroke}
\pgfpathmoveto{\pgfpoint{99.367081pt}{140.777435pt}}
\pgflineto{\pgfpoint{100.363068pt}{140.777435pt}}
\pgfusepath{stroke}
\pgfpathmoveto{\pgfpoint{98.371094pt}{140.777435pt}}
\pgflineto{\pgfpoint{99.367081pt}{140.777435pt}}
\pgfusepath{stroke}
\pgfpathmoveto{\pgfpoint{97.375107pt}{140.777435pt}}
\pgflineto{\pgfpoint{98.371094pt}{140.777435pt}}
\pgfusepath{stroke}
\pgfpathmoveto{\pgfpoint{96.379128pt}{140.777435pt}}
\pgflineto{\pgfpoint{97.375107pt}{140.777435pt}}
\pgfusepath{stroke}
\pgfpathmoveto{\pgfpoint{95.383141pt}{140.777435pt}}
\pgflineto{\pgfpoint{96.379128pt}{140.777435pt}}
\pgfusepath{stroke}
\pgfpathmoveto{\pgfpoint{94.387161pt}{140.777435pt}}
\pgflineto{\pgfpoint{95.383141pt}{140.777435pt}}
\pgfusepath{stroke}
\pgfpathmoveto{\pgfpoint{95.383141pt}{141.040619pt}}
\pgflineto{\pgfpoint{94.387161pt}{140.777435pt}}
\pgfusepath{stroke}
\pgfpathmoveto{\pgfpoint{96.379128pt}{141.052612pt}}
\pgflineto{\pgfpoint{95.383141pt}{141.040619pt}}
\pgfusepath{stroke}
\pgfpathmoveto{\pgfpoint{97.375107pt}{142.021271pt}}
\pgflineto{\pgfpoint{96.379128pt}{141.052612pt}}
\pgfusepath{stroke}
\pgfpathmoveto{\pgfpoint{98.371094pt}{187.727631pt}}
\pgflineto{\pgfpoint{97.375107pt}{142.021271pt}}
\pgfusepath{stroke}
\pgfpathmoveto{\pgfpoint{99.367081pt}{143.174622pt}}
\pgflineto{\pgfpoint{98.371094pt}{187.727631pt}}
\pgfusepath{stroke}
\pgfpathmoveto{\pgfpoint{100.363068pt}{142.326782pt}}
\pgflineto{\pgfpoint{99.367081pt}{143.174622pt}}
\pgfusepath{stroke}
\pgfpathmoveto{\pgfpoint{101.359047pt}{141.473938pt}}
\pgflineto{\pgfpoint{100.363068pt}{142.326782pt}}
\pgfusepath{stroke}
\pgfpathmoveto{\pgfpoint{102.355034pt}{153.173050pt}}
\pgflineto{\pgfpoint{101.359047pt}{141.473938pt}}
\pgfusepath{stroke}
\pgfpathmoveto{\pgfpoint{103.351013pt}{140.777435pt}}
\pgflineto{\pgfpoint{102.355034pt}{153.173050pt}}
\pgfusepath{stroke}
\pgfpathmoveto{\pgfpoint{104.347000pt}{140.777435pt}}
\pgflineto{\pgfpoint{105.342987pt}{140.777435pt}}
\pgfusepath{stroke}
\pgfpathmoveto{\pgfpoint{103.351013pt}{140.777435pt}}
\pgflineto{\pgfpoint{104.347000pt}{140.777435pt}}
\pgfusepath{stroke}
\pgfpathmoveto{\pgfpoint{104.347000pt}{152.852020pt}}
\pgflineto{\pgfpoint{103.351013pt}{140.777435pt}}
\pgfusepath{stroke}
\pgfpathmoveto{\pgfpoint{105.342987pt}{140.777435pt}}
\pgflineto{\pgfpoint{104.347000pt}{152.852020pt}}
\pgfusepath{stroke}
\pgfpathmoveto{\pgfpoint{110.322906pt}{140.777435pt}}
\pgflineto{\pgfpoint{111.318893pt}{140.777435pt}}
\pgfusepath{stroke}
\pgfpathmoveto{\pgfpoint{109.326920pt}{140.777435pt}}
\pgflineto{\pgfpoint{110.322906pt}{140.777435pt}}
\pgfusepath{stroke}
\pgfpathmoveto{\pgfpoint{108.330933pt}{140.777435pt}}
\pgflineto{\pgfpoint{109.326920pt}{140.777435pt}}
\pgfusepath{stroke}
\pgfpathmoveto{\pgfpoint{107.334953pt}{140.777435pt}}
\pgflineto{\pgfpoint{108.330933pt}{140.777435pt}}
\pgfusepath{stroke}
\pgfpathmoveto{\pgfpoint{108.330933pt}{156.427597pt}}
\pgflineto{\pgfpoint{107.334953pt}{140.777435pt}}
\pgfusepath{stroke}
\pgfpathmoveto{\pgfpoint{109.326920pt}{140.854156pt}}
\pgflineto{\pgfpoint{108.330933pt}{156.427597pt}}
\pgfusepath{stroke}
\pgfpathmoveto{\pgfpoint{110.322906pt}{143.840607pt}}
\pgflineto{\pgfpoint{109.326920pt}{140.854156pt}}
\pgfusepath{stroke}
\pgfpathmoveto{\pgfpoint{111.318893pt}{140.777435pt}}
\pgflineto{\pgfpoint{110.322906pt}{143.840607pt}}
\pgfusepath{stroke}
\pgfpathmoveto{\pgfpoint{114.306839pt}{140.777435pt}}
\pgflineto{\pgfpoint{115.302826pt}{140.777435pt}}
\pgfusepath{stroke}
\pgfpathmoveto{\pgfpoint{113.310852pt}{140.777435pt}}
\pgflineto{\pgfpoint{114.306839pt}{140.777435pt}}
\pgfusepath{stroke}
\pgfpathmoveto{\pgfpoint{112.314873pt}{140.777435pt}}
\pgflineto{\pgfpoint{113.310852pt}{140.777435pt}}
\pgfusepath{stroke}
\pgfpathmoveto{\pgfpoint{111.318893pt}{140.777435pt}}
\pgflineto{\pgfpoint{112.314873pt}{140.777435pt}}
\pgfusepath{stroke}
\pgfpathmoveto{\pgfpoint{112.314873pt}{142.841614pt}}
\pgflineto{\pgfpoint{111.318893pt}{140.777435pt}}
\pgfusepath{stroke}
\pgfpathmoveto{\pgfpoint{113.310852pt}{141.335541pt}}
\pgflineto{\pgfpoint{112.314873pt}{142.841614pt}}
\pgfusepath{stroke}
\pgfpathmoveto{\pgfpoint{114.306839pt}{143.259003pt}}
\pgflineto{\pgfpoint{113.310852pt}{141.335541pt}}
\pgfusepath{stroke}
\pgfpathmoveto{\pgfpoint{115.302826pt}{140.777435pt}}
\pgflineto{\pgfpoint{114.306839pt}{143.259003pt}}
\pgfusepath{stroke}
\pgfpathmoveto{\pgfpoint{119.286758pt}{140.777435pt}}
\pgflineto{\pgfpoint{120.282745pt}{140.777435pt}}
\pgfusepath{stroke}
\pgfpathmoveto{\pgfpoint{118.290779pt}{140.777435pt}}
\pgflineto{\pgfpoint{119.286758pt}{140.777435pt}}
\pgfusepath{stroke}
\pgfpathmoveto{\pgfpoint{117.294792pt}{140.777435pt}}
\pgflineto{\pgfpoint{118.290779pt}{140.777435pt}}
\pgfusepath{stroke}
\pgfpathmoveto{\pgfpoint{116.298813pt}{140.777435pt}}
\pgflineto{\pgfpoint{117.294792pt}{140.777435pt}}
\pgfusepath{stroke}
\pgfpathmoveto{\pgfpoint{115.302826pt}{140.777435pt}}
\pgflineto{\pgfpoint{116.298813pt}{140.777435pt}}
\pgfusepath{stroke}
\pgfpathmoveto{\pgfpoint{116.298813pt}{141.607086pt}}
\pgflineto{\pgfpoint{115.302826pt}{140.777435pt}}
\pgfusepath{stroke}
\pgfpathmoveto{\pgfpoint{117.294792pt}{143.559708pt}}
\pgflineto{\pgfpoint{116.298813pt}{141.607086pt}}
\pgfusepath{stroke}
\pgfpathmoveto{\pgfpoint{118.290779pt}{140.848618pt}}
\pgflineto{\pgfpoint{117.294792pt}{143.559708pt}}
\pgfusepath{stroke}
\pgfpathmoveto{\pgfpoint{119.286758pt}{140.997116pt}}
\pgflineto{\pgfpoint{118.290779pt}{140.848618pt}}
\pgfusepath{stroke}
\pgfpathmoveto{\pgfpoint{120.282745pt}{140.777435pt}}
\pgflineto{\pgfpoint{119.286758pt}{140.997116pt}}
\pgfusepath{stroke}
\pgfpathmoveto{\pgfpoint{121.278725pt}{140.777435pt}}
\pgflineto{\pgfpoint{122.274712pt}{140.777435pt}}
\pgfusepath{stroke}
\pgfpathmoveto{\pgfpoint{120.282745pt}{140.777435pt}}
\pgflineto{\pgfpoint{121.278725pt}{140.777435pt}}
\pgfusepath{stroke}
\pgfpathmoveto{\pgfpoint{121.278725pt}{141.992584pt}}
\pgflineto{\pgfpoint{120.282745pt}{140.777435pt}}
\pgfusepath{stroke}
\pgfpathmoveto{\pgfpoint{122.274712pt}{140.777435pt}}
\pgflineto{\pgfpoint{121.278725pt}{141.992584pt}}
\pgfusepath{stroke}
\pgfpathmoveto{\pgfpoint{123.270691pt}{140.777435pt}}
\pgflineto{\pgfpoint{124.266678pt}{140.777435pt}}
\pgfusepath{stroke}
\pgfpathmoveto{\pgfpoint{122.274712pt}{140.777435pt}}
\pgflineto{\pgfpoint{123.270691pt}{140.777435pt}}
\pgfusepath{stroke}
\pgfpathmoveto{\pgfpoint{123.270691pt}{141.045074pt}}
\pgflineto{\pgfpoint{122.274712pt}{140.777435pt}}
\pgfusepath{stroke}
\pgfpathmoveto{\pgfpoint{124.266678pt}{140.777435pt}}
\pgflineto{\pgfpoint{123.270691pt}{141.045074pt}}
\pgfusepath{stroke}
\pgfpathmoveto{\pgfpoint{127.254631pt}{140.777435pt}}
\pgflineto{\pgfpoint{128.250610pt}{140.777435pt}}
\pgfusepath{stroke}
\pgfpathmoveto{\pgfpoint{126.258652pt}{140.777435pt}}
\pgflineto{\pgfpoint{127.254631pt}{140.777435pt}}
\pgfusepath{stroke}
\pgfpathmoveto{\pgfpoint{125.262665pt}{140.777435pt}}
\pgflineto{\pgfpoint{126.258652pt}{140.777435pt}}
\pgfusepath{stroke}
\pgfpathmoveto{\pgfpoint{126.258652pt}{141.187393pt}}
\pgflineto{\pgfpoint{125.262665pt}{140.777435pt}}
\pgfusepath{stroke}
\pgfpathmoveto{\pgfpoint{127.254631pt}{158.624695pt}}
\pgflineto{\pgfpoint{126.258652pt}{141.187393pt}}
\pgfusepath{stroke}
\pgfpathmoveto{\pgfpoint{128.250610pt}{140.777435pt}}
\pgflineto{\pgfpoint{127.254631pt}{158.624695pt}}
\pgfusepath{stroke}
\pgfpathmoveto{\pgfpoint{132.234558pt}{140.777435pt}}
\pgflineto{\pgfpoint{133.230530pt}{140.777435pt}}
\pgfusepath{stroke}
\pgfpathmoveto{\pgfpoint{131.238571pt}{140.777435pt}}
\pgflineto{\pgfpoint{132.234558pt}{140.777435pt}}
\pgfusepath{stroke}
\pgfpathmoveto{\pgfpoint{130.242584pt}{140.777435pt}}
\pgflineto{\pgfpoint{131.238571pt}{140.777435pt}}
\pgfusepath{stroke}
\pgfpathmoveto{\pgfpoint{129.246597pt}{140.777435pt}}
\pgflineto{\pgfpoint{130.242584pt}{140.777435pt}}
\pgfusepath{stroke}
\pgfpathmoveto{\pgfpoint{128.250610pt}{140.777435pt}}
\pgflineto{\pgfpoint{129.246597pt}{140.777435pt}}
\pgfusepath{stroke}
\pgfpathmoveto{\pgfpoint{129.246597pt}{146.889862pt}}
\pgflineto{\pgfpoint{128.250610pt}{140.777435pt}}
\pgfusepath{stroke}
\pgfpathmoveto{\pgfpoint{130.242584pt}{141.305695pt}}
\pgflineto{\pgfpoint{129.246597pt}{146.889862pt}}
\pgfusepath{stroke}
\pgfpathmoveto{\pgfpoint{131.238571pt}{142.924377pt}}
\pgflineto{\pgfpoint{130.242584pt}{141.305695pt}}
\pgfusepath{stroke}
\pgfpathmoveto{\pgfpoint{132.234558pt}{140.784973pt}}
\pgflineto{\pgfpoint{131.238571pt}{142.924377pt}}
\pgfusepath{stroke}
\pgfpathmoveto{\pgfpoint{133.230530pt}{140.777435pt}}
\pgflineto{\pgfpoint{132.234558pt}{140.784973pt}}
\pgfusepath{stroke}
\pgfpathmoveto{\pgfpoint{135.222504pt}{140.777435pt}}
\pgflineto{\pgfpoint{136.218475pt}{140.777435pt}}
\pgfusepath{stroke}
\pgfpathmoveto{\pgfpoint{134.226517pt}{140.777435pt}}
\pgflineto{\pgfpoint{135.222504pt}{140.777435pt}}
\pgfusepath{stroke}
\pgfpathmoveto{\pgfpoint{135.222504pt}{141.478424pt}}
\pgflineto{\pgfpoint{134.226517pt}{140.777435pt}}
\pgfusepath{stroke}
\pgfpathmoveto{\pgfpoint{136.218475pt}{140.777435pt}}
\pgflineto{\pgfpoint{135.222504pt}{141.478424pt}}
\pgfusepath{stroke}
\pgfpathmoveto{\pgfpoint{141.198410pt}{140.777435pt}}
\pgflineto{\pgfpoint{142.194382pt}{140.777435pt}}
\pgfusepath{stroke}
\pgfpathmoveto{\pgfpoint{140.202423pt}{140.777435pt}}
\pgflineto{\pgfpoint{141.198410pt}{140.777435pt}}
\pgfusepath{stroke}
\pgfpathmoveto{\pgfpoint{139.206436pt}{140.777435pt}}
\pgflineto{\pgfpoint{140.202423pt}{140.777435pt}}
\pgfusepath{stroke}
\pgfpathmoveto{\pgfpoint{138.210449pt}{140.777435pt}}
\pgflineto{\pgfpoint{139.206436pt}{140.777435pt}}
\pgfusepath{stroke}
\pgfpathmoveto{\pgfpoint{137.214478pt}{140.777435pt}}
\pgflineto{\pgfpoint{138.210449pt}{140.777435pt}}
\pgfusepath{stroke}
\pgfpathmoveto{\pgfpoint{136.218475pt}{140.777435pt}}
\pgflineto{\pgfpoint{137.214478pt}{140.777435pt}}
\pgfusepath{stroke}
\pgfpathmoveto{\pgfpoint{137.214478pt}{144.493912pt}}
\pgflineto{\pgfpoint{136.218475pt}{140.777435pt}}
\pgfusepath{stroke}
\pgfpathmoveto{\pgfpoint{138.210449pt}{140.824677pt}}
\pgflineto{\pgfpoint{137.214478pt}{144.493912pt}}
\pgfusepath{stroke}
\pgfpathmoveto{\pgfpoint{139.206436pt}{155.555008pt}}
\pgflineto{\pgfpoint{138.210449pt}{140.824677pt}}
\pgfusepath{stroke}
\pgfpathmoveto{\pgfpoint{140.202423pt}{141.880554pt}}
\pgflineto{\pgfpoint{139.206436pt}{155.555008pt}}
\pgfusepath{stroke}
\pgfpathmoveto{\pgfpoint{141.198410pt}{141.096649pt}}
\pgflineto{\pgfpoint{140.202423pt}{141.880554pt}}
\pgfusepath{stroke}
\pgfpathmoveto{\pgfpoint{142.194382pt}{140.777435pt}}
\pgflineto{\pgfpoint{141.198410pt}{141.096649pt}}
\pgfusepath{stroke}
\pgfpathmoveto{\pgfpoint{143.190369pt}{140.777435pt}}
\pgflineto{\pgfpoint{144.186356pt}{140.777435pt}}
\pgfusepath{stroke}
\pgfpathmoveto{\pgfpoint{142.194382pt}{140.777435pt}}
\pgflineto{\pgfpoint{143.190369pt}{140.777435pt}}
\pgfusepath{stroke}
\pgfpathmoveto{\pgfpoint{143.190369pt}{141.972321pt}}
\pgflineto{\pgfpoint{142.194382pt}{140.777435pt}}
\pgfusepath{stroke}
\pgfpathmoveto{\pgfpoint{144.186356pt}{140.777435pt}}
\pgflineto{\pgfpoint{143.190369pt}{141.972321pt}}
\pgfusepath{stroke}
\pgfpathmoveto{\pgfpoint{147.174316pt}{140.777435pt}}
\pgflineto{\pgfpoint{148.170288pt}{140.777435pt}}
\pgfusepath{stroke}
\pgfpathmoveto{\pgfpoint{146.178314pt}{140.777435pt}}
\pgflineto{\pgfpoint{147.174316pt}{140.777435pt}}
\pgfusepath{stroke}
\pgfpathmoveto{\pgfpoint{145.182343pt}{140.777435pt}}
\pgflineto{\pgfpoint{146.178314pt}{140.777435pt}}
\pgfusepath{stroke}
\pgfpathmoveto{\pgfpoint{144.186356pt}{140.777435pt}}
\pgflineto{\pgfpoint{145.182343pt}{140.777435pt}}
\pgfusepath{stroke}
\pgfpathmoveto{\pgfpoint{145.182343pt}{141.507751pt}}
\pgflineto{\pgfpoint{144.186356pt}{140.777435pt}}
\pgfusepath{stroke}
\pgfpathmoveto{\pgfpoint{146.178314pt}{141.062988pt}}
\pgflineto{\pgfpoint{145.182343pt}{141.507751pt}}
\pgfusepath{stroke}
\pgfpathmoveto{\pgfpoint{147.174316pt}{143.409790pt}}
\pgflineto{\pgfpoint{146.178314pt}{141.062988pt}}
\pgfusepath{stroke}
\pgfpathmoveto{\pgfpoint{148.170288pt}{140.777435pt}}
\pgflineto{\pgfpoint{147.174316pt}{143.409790pt}}
\pgfusepath{stroke}
\pgfpathmoveto{\pgfpoint{153.150208pt}{140.777435pt}}
\pgflineto{\pgfpoint{154.146194pt}{140.777435pt}}
\pgfusepath{stroke}
\pgfpathmoveto{\pgfpoint{152.154221pt}{140.777435pt}}
\pgflineto{\pgfpoint{153.150208pt}{140.777435pt}}
\pgfusepath{stroke}
\pgfpathmoveto{\pgfpoint{151.158249pt}{140.777435pt}}
\pgflineto{\pgfpoint{152.154221pt}{140.777435pt}}
\pgfusepath{stroke}
\pgfpathmoveto{\pgfpoint{150.162262pt}{140.777435pt}}
\pgflineto{\pgfpoint{151.158249pt}{140.777435pt}}
\pgfusepath{stroke}
\pgfpathmoveto{\pgfpoint{151.158249pt}{141.561523pt}}
\pgflineto{\pgfpoint{150.162262pt}{140.777435pt}}
\pgfusepath{stroke}
\pgfpathmoveto{\pgfpoint{152.154221pt}{144.148651pt}}
\pgflineto{\pgfpoint{151.158249pt}{141.561523pt}}
\pgfusepath{stroke}
\pgfpathmoveto{\pgfpoint{153.150208pt}{143.520859pt}}
\pgflineto{\pgfpoint{152.154221pt}{144.148651pt}}
\pgfusepath{stroke}
\pgfpathmoveto{\pgfpoint{154.146194pt}{140.777435pt}}
\pgflineto{\pgfpoint{153.150208pt}{143.520859pt}}
\pgfusepath{stroke}
\pgfpathmoveto{\pgfpoint{155.142181pt}{140.777435pt}}
\pgflineto{\pgfpoint{156.138168pt}{140.777435pt}}
\pgfusepath{stroke}
\pgfpathmoveto{\pgfpoint{154.146194pt}{140.777435pt}}
\pgflineto{\pgfpoint{155.142181pt}{140.777435pt}}
\pgfusepath{stroke}
\pgfpathmoveto{\pgfpoint{155.142181pt}{195.815277pt}}
\pgflineto{\pgfpoint{154.146194pt}{140.777435pt}}
\pgfusepath{stroke}
\pgfpathmoveto{\pgfpoint{156.138168pt}{140.777435pt}}
\pgflineto{\pgfpoint{155.142181pt}{195.815277pt}}
\pgfusepath{stroke}
\pgfpathmoveto{\pgfpoint{158.130127pt}{140.777435pt}}
\pgflineto{\pgfpoint{159.126114pt}{140.777435pt}}
\pgfusepath{stroke}
\pgfpathmoveto{\pgfpoint{157.134155pt}{140.777435pt}}
\pgflineto{\pgfpoint{158.130127pt}{140.777435pt}}
\pgfusepath{stroke}
\pgfpathmoveto{\pgfpoint{156.138168pt}{140.777435pt}}
\pgflineto{\pgfpoint{157.134155pt}{140.777435pt}}
\pgfusepath{stroke}
\pgfpathmoveto{\pgfpoint{157.134155pt}{140.926575pt}}
\pgflineto{\pgfpoint{156.138168pt}{140.777435pt}}
\pgfusepath{stroke}
\pgfpathmoveto{\pgfpoint{158.130127pt}{140.979492pt}}
\pgflineto{\pgfpoint{157.134155pt}{140.926575pt}}
\pgfusepath{stroke}
\pgfpathmoveto{\pgfpoint{159.126114pt}{140.777435pt}}
\pgflineto{\pgfpoint{158.130127pt}{140.979492pt}}
\pgfusepath{stroke}
\pgfpathmoveto{\pgfpoint{162.114075pt}{140.777435pt}}
\pgflineto{\pgfpoint{163.110062pt}{140.777435pt}}
\pgfusepath{stroke}
\pgfpathmoveto{\pgfpoint{161.118088pt}{140.777435pt}}
\pgflineto{\pgfpoint{162.114075pt}{140.777435pt}}
\pgfusepath{stroke}
\pgfpathmoveto{\pgfpoint{160.122101pt}{140.777435pt}}
\pgflineto{\pgfpoint{161.118088pt}{140.777435pt}}
\pgfusepath{stroke}
\pgfpathmoveto{\pgfpoint{159.126114pt}{140.777435pt}}
\pgflineto{\pgfpoint{160.122101pt}{140.777435pt}}
\pgfusepath{stroke}
\pgfpathmoveto{\pgfpoint{160.122101pt}{158.193100pt}}
\pgflineto{\pgfpoint{159.126114pt}{140.777435pt}}
\pgfusepath{stroke}
\pgfpathmoveto{\pgfpoint{161.118088pt}{140.795273pt}}
\pgflineto{\pgfpoint{160.122101pt}{158.193100pt}}
\pgfusepath{stroke}
\pgfpathmoveto{\pgfpoint{162.114075pt}{141.404877pt}}
\pgflineto{\pgfpoint{161.118088pt}{140.795273pt}}
\pgfusepath{stroke}
\pgfpathmoveto{\pgfpoint{163.110062pt}{140.777435pt}}
\pgflineto{\pgfpoint{162.114075pt}{141.404877pt}}
\pgfusepath{stroke}
\pgfpathmoveto{\pgfpoint{165.102020pt}{140.777435pt}}
\pgflineto{\pgfpoint{166.098007pt}{140.777435pt}}
\pgfusepath{stroke}
\pgfpathmoveto{\pgfpoint{164.106033pt}{140.777435pt}}
\pgflineto{\pgfpoint{165.102020pt}{140.777435pt}}
\pgfusepath{stroke}
\pgfpathmoveto{\pgfpoint{163.110062pt}{140.777435pt}}
\pgflineto{\pgfpoint{164.106033pt}{140.777435pt}}
\pgfusepath{stroke}
\pgfpathmoveto{\pgfpoint{164.106033pt}{194.127029pt}}
\pgflineto{\pgfpoint{163.110062pt}{140.777435pt}}
\pgfusepath{stroke}
\pgfpathmoveto{\pgfpoint{165.102020pt}{140.828079pt}}
\pgflineto{\pgfpoint{164.106033pt}{194.127029pt}}
\pgfusepath{stroke}
\pgfpathmoveto{\pgfpoint{166.098007pt}{140.777435pt}}
\pgflineto{\pgfpoint{165.102020pt}{140.828079pt}}
\pgfusepath{stroke}
\pgfpathmoveto{\pgfpoint{168.089966pt}{140.777435pt}}
\pgflineto{\pgfpoint{169.085953pt}{140.777435pt}}
\pgfusepath{stroke}
\pgfpathmoveto{\pgfpoint{167.093994pt}{140.777435pt}}
\pgflineto{\pgfpoint{168.089966pt}{140.777435pt}}
\pgfusepath{stroke}
\pgfpathmoveto{\pgfpoint{166.098007pt}{140.777435pt}}
\pgflineto{\pgfpoint{167.093994pt}{140.777435pt}}
\pgfusepath{stroke}
\pgfpathmoveto{\pgfpoint{167.093994pt}{141.434799pt}}
\pgflineto{\pgfpoint{166.098007pt}{140.777435pt}}
\pgfusepath{stroke}
\pgfpathmoveto{\pgfpoint{167.415314pt}{205.642242pt}}
\pgflineto{\pgfpoint{167.093994pt}{141.434799pt}}
\pgfusepath{stroke}
\pgfpathmoveto{\pgfpoint{169.085953pt}{140.777435pt}}
\pgflineto{\pgfpoint{168.762405pt}{205.642242pt}}
\pgfusepath{stroke}
\pgfpathmoveto{\pgfpoint{175.061859pt}{140.777435pt}}
\pgflineto{\pgfpoint{176.057846pt}{140.777435pt}}
\pgfusepath{stroke}
\pgfpathmoveto{\pgfpoint{174.065872pt}{140.777435pt}}
\pgflineto{\pgfpoint{175.061859pt}{140.777435pt}}
\pgfusepath{stroke}
\pgfpathmoveto{\pgfpoint{173.069885pt}{140.777435pt}}
\pgflineto{\pgfpoint{174.065872pt}{140.777435pt}}
\pgfusepath{stroke}
\pgfpathmoveto{\pgfpoint{172.073914pt}{140.777435pt}}
\pgflineto{\pgfpoint{173.069885pt}{140.777435pt}}
\pgfusepath{stroke}
\pgfpathmoveto{\pgfpoint{171.077911pt}{140.777435pt}}
\pgflineto{\pgfpoint{172.073914pt}{140.777435pt}}
\pgfusepath{stroke}
\pgfpathmoveto{\pgfpoint{170.081940pt}{140.777435pt}}
\pgflineto{\pgfpoint{171.077911pt}{140.777435pt}}
\pgfusepath{stroke}
\pgfpathmoveto{\pgfpoint{169.085953pt}{140.777435pt}}
\pgflineto{\pgfpoint{170.081940pt}{140.777435pt}}
\pgfusepath{stroke}
\pgfpathmoveto{\pgfpoint{170.081940pt}{146.720169pt}}
\pgflineto{\pgfpoint{169.085953pt}{140.777435pt}}
\pgfusepath{stroke}
\pgfpathmoveto{\pgfpoint{171.077911pt}{145.686310pt}}
\pgflineto{\pgfpoint{170.081940pt}{146.720169pt}}
\pgfusepath{stroke}
\pgfpathmoveto{\pgfpoint{172.073914pt}{149.736984pt}}
\pgflineto{\pgfpoint{171.077911pt}{145.686310pt}}
\pgfusepath{stroke}
\pgfpathmoveto{\pgfpoint{173.069885pt}{148.531342pt}}
\pgflineto{\pgfpoint{172.073914pt}{149.736984pt}}
\pgfusepath{stroke}
\pgfpathmoveto{\pgfpoint{174.065872pt}{141.515808pt}}
\pgflineto{\pgfpoint{173.069885pt}{148.531342pt}}
\pgfusepath{stroke}
\pgfpathmoveto{\pgfpoint{175.061859pt}{141.323135pt}}
\pgflineto{\pgfpoint{174.065872pt}{141.515808pt}}
\pgfusepath{stroke}
\pgfpathmoveto{\pgfpoint{176.057846pt}{140.777435pt}}
\pgflineto{\pgfpoint{175.061859pt}{141.323135pt}}
\pgfusepath{stroke}
\pgfpathmoveto{\pgfpoint{181.037766pt}{140.777435pt}}
\pgflineto{\pgfpoint{182.033752pt}{140.777435pt}}
\pgfusepath{stroke}
\pgfpathmoveto{\pgfpoint{180.041779pt}{140.777435pt}}
\pgflineto{\pgfpoint{181.037766pt}{140.777435pt}}
\pgfusepath{stroke}
\pgfpathmoveto{\pgfpoint{179.045792pt}{140.777435pt}}
\pgflineto{\pgfpoint{180.041779pt}{140.777435pt}}
\pgfusepath{stroke}
\pgfpathmoveto{\pgfpoint{178.049805pt}{140.777435pt}}
\pgflineto{\pgfpoint{179.045792pt}{140.777435pt}}
\pgfusepath{stroke}
\pgfpathmoveto{\pgfpoint{177.053818pt}{140.777435pt}}
\pgflineto{\pgfpoint{178.049805pt}{140.777435pt}}
\pgfusepath{stroke}
\pgfpathmoveto{\pgfpoint{176.057846pt}{140.777435pt}}
\pgflineto{\pgfpoint{177.053818pt}{140.777435pt}}
\pgfusepath{stroke}
\pgfpathmoveto{\pgfpoint{177.053818pt}{144.404053pt}}
\pgflineto{\pgfpoint{176.057846pt}{140.777435pt}}
\pgfusepath{stroke}
\pgfpathmoveto{\pgfpoint{178.049805pt}{144.353317pt}}
\pgflineto{\pgfpoint{177.053818pt}{144.404053pt}}
\pgfusepath{stroke}
\pgfpathmoveto{\pgfpoint{179.045792pt}{140.963043pt}}
\pgflineto{\pgfpoint{178.049805pt}{144.353317pt}}
\pgfusepath{stroke}
\pgfpathmoveto{\pgfpoint{180.041779pt}{140.930664pt}}
\pgflineto{\pgfpoint{179.045792pt}{140.963043pt}}
\pgfusepath{stroke}
\pgfpathmoveto{\pgfpoint{181.037766pt}{141.619095pt}}
\pgflineto{\pgfpoint{180.041779pt}{140.930664pt}}
\pgfusepath{stroke}
\pgfpathmoveto{\pgfpoint{182.033752pt}{140.777435pt}}
\pgflineto{\pgfpoint{181.037766pt}{141.619095pt}}
\pgfusepath{stroke}
\pgfpathmoveto{\pgfpoint{183.029724pt}{140.777435pt}}
\pgflineto{\pgfpoint{184.025711pt}{140.777435pt}}
\pgfusepath{stroke}
\pgfpathmoveto{\pgfpoint{182.033752pt}{140.777435pt}}
\pgflineto{\pgfpoint{183.029724pt}{140.777435pt}}
\pgfusepath{stroke}
\pgfpathmoveto{\pgfpoint{183.029724pt}{142.399628pt}}
\pgflineto{\pgfpoint{182.033752pt}{140.777435pt}}
\pgfusepath{stroke}
\pgfpathmoveto{\pgfpoint{184.025711pt}{140.777435pt}}
\pgflineto{\pgfpoint{183.029724pt}{142.399628pt}}
\pgfusepath{stroke}
\pgfpathmoveto{\pgfpoint{187.013672pt}{140.777435pt}}
\pgflineto{\pgfpoint{188.009659pt}{140.777435pt}}
\pgfusepath{stroke}
\pgfpathmoveto{\pgfpoint{186.017685pt}{140.777435pt}}
\pgflineto{\pgfpoint{187.013672pt}{140.777435pt}}
\pgfusepath{stroke}
\pgfpathmoveto{\pgfpoint{187.013672pt}{141.098480pt}}
\pgflineto{\pgfpoint{186.017685pt}{140.777435pt}}
\pgfusepath{stroke}
\pgfpathmoveto{\pgfpoint{188.009659pt}{140.777435pt}}
\pgflineto{\pgfpoint{187.013672pt}{141.098480pt}}
\pgfusepath{stroke}
\pgfpathmoveto{\pgfpoint{198.965469pt}{140.777435pt}}
\pgflineto{\pgfpoint{199.961456pt}{140.777435pt}}
\pgfusepath{stroke}
\pgfpathmoveto{\pgfpoint{197.969498pt}{140.777435pt}}
\pgflineto{\pgfpoint{198.965469pt}{140.777435pt}}
\pgfusepath{stroke}
\pgfpathmoveto{\pgfpoint{196.973511pt}{140.777435pt}}
\pgflineto{\pgfpoint{197.969498pt}{140.777435pt}}
\pgfusepath{stroke}
\pgfpathmoveto{\pgfpoint{195.977524pt}{140.777435pt}}
\pgflineto{\pgfpoint{196.973511pt}{140.777435pt}}
\pgfusepath{stroke}
\pgfpathmoveto{\pgfpoint{194.981537pt}{140.777435pt}}
\pgflineto{\pgfpoint{195.977524pt}{140.777435pt}}
\pgfusepath{stroke}
\pgfpathmoveto{\pgfpoint{193.985565pt}{140.777435pt}}
\pgflineto{\pgfpoint{194.981537pt}{140.777435pt}}
\pgfusepath{stroke}
\pgfpathmoveto{\pgfpoint{192.989563pt}{140.777435pt}}
\pgflineto{\pgfpoint{193.985565pt}{140.777435pt}}
\pgfusepath{stroke}
\pgfpathmoveto{\pgfpoint{191.993591pt}{140.777435pt}}
\pgflineto{\pgfpoint{192.989563pt}{140.777435pt}}
\pgfusepath{stroke}
\pgfpathmoveto{\pgfpoint{190.997604pt}{140.777435pt}}
\pgflineto{\pgfpoint{191.993591pt}{140.777435pt}}
\pgfusepath{stroke}
\pgfpathmoveto{\pgfpoint{190.001617pt}{140.777435pt}}
\pgflineto{\pgfpoint{190.997604pt}{140.777435pt}}
\pgfusepath{stroke}
\pgfpathmoveto{\pgfpoint{189.005630pt}{140.777435pt}}
\pgflineto{\pgfpoint{190.001617pt}{140.777435pt}}
\pgfusepath{stroke}
\pgfpathmoveto{\pgfpoint{188.009659pt}{140.777435pt}}
\pgflineto{\pgfpoint{189.005630pt}{140.777435pt}}
\pgfusepath{stroke}
\pgfpathmoveto{\pgfpoint{189.005630pt}{140.805771pt}}
\pgflineto{\pgfpoint{188.009659pt}{140.777435pt}}
\pgfusepath{stroke}
\pgfpathmoveto{\pgfpoint{190.001617pt}{149.564423pt}}
\pgflineto{\pgfpoint{189.005630pt}{140.805771pt}}
\pgfusepath{stroke}
\pgfpathmoveto{\pgfpoint{190.997604pt}{140.863358pt}}
\pgflineto{\pgfpoint{190.001617pt}{149.564423pt}}
\pgfusepath{stroke}
\pgfpathmoveto{\pgfpoint{191.993591pt}{142.184387pt}}
\pgflineto{\pgfpoint{190.997604pt}{140.863358pt}}
\pgfusepath{stroke}
\pgfpathmoveto{\pgfpoint{192.989563pt}{195.151352pt}}
\pgflineto{\pgfpoint{191.993591pt}{142.184387pt}}
\pgfusepath{stroke}
\pgfpathmoveto{\pgfpoint{193.985565pt}{141.529373pt}}
\pgflineto{\pgfpoint{192.989563pt}{195.151352pt}}
\pgfusepath{stroke}
\pgfpathmoveto{\pgfpoint{194.981537pt}{140.831909pt}}
\pgflineto{\pgfpoint{193.985565pt}{141.529373pt}}
\pgfusepath{stroke}
\pgfpathmoveto{\pgfpoint{195.977524pt}{149.854721pt}}
\pgflineto{\pgfpoint{194.981537pt}{140.831909pt}}
\pgfusepath{stroke}
\pgfpathmoveto{\pgfpoint{196.973511pt}{144.710114pt}}
\pgflineto{\pgfpoint{195.977524pt}{149.854721pt}}
\pgfusepath{stroke}
\pgfpathmoveto{\pgfpoint{197.969498pt}{144.257416pt}}
\pgflineto{\pgfpoint{196.973511pt}{144.710114pt}}
\pgfusepath{stroke}
\pgfpathmoveto{\pgfpoint{198.965469pt}{197.820709pt}}
\pgflineto{\pgfpoint{197.969498pt}{144.257416pt}}
\pgfusepath{stroke}
\pgfpathmoveto{\pgfpoint{199.961456pt}{140.777435pt}}
\pgflineto{\pgfpoint{198.965469pt}{197.820709pt}}
\pgfusepath{stroke}
\pgfpathmoveto{\pgfpoint{201.953430pt}{140.777435pt}}
\pgflineto{\pgfpoint{202.949402pt}{140.777435pt}}
\pgfusepath{stroke}
\pgfpathmoveto{\pgfpoint{200.957443pt}{140.777435pt}}
\pgflineto{\pgfpoint{201.953430pt}{140.777435pt}}
\pgfusepath{stroke}
\pgfpathmoveto{\pgfpoint{199.961456pt}{140.777435pt}}
\pgflineto{\pgfpoint{200.957443pt}{140.777435pt}}
\pgfusepath{stroke}
\pgfpathmoveto{\pgfpoint{200.957443pt}{141.105011pt}}
\pgflineto{\pgfpoint{199.961456pt}{140.777435pt}}
\pgfusepath{stroke}
\pgfpathmoveto{\pgfpoint{201.953430pt}{140.916779pt}}
\pgflineto{\pgfpoint{200.957443pt}{141.105011pt}}
\pgfusepath{stroke}
\pgfpathmoveto{\pgfpoint{202.949402pt}{140.777435pt}}
\pgflineto{\pgfpoint{201.953430pt}{140.916779pt}}
\pgfusepath{stroke}
\pgfpathmoveto{\pgfpoint{205.937347pt}{140.777435pt}}
\pgflineto{\pgfpoint{206.933334pt}{140.777435pt}}
\pgfusepath{stroke}
\pgfpathmoveto{\pgfpoint{204.941376pt}{140.777435pt}}
\pgflineto{\pgfpoint{205.937347pt}{140.777435pt}}
\pgfusepath{stroke}
\pgfpathmoveto{\pgfpoint{205.937347pt}{144.822937pt}}
\pgflineto{\pgfpoint{204.941376pt}{140.777435pt}}
\pgfusepath{stroke}
\pgfpathmoveto{\pgfpoint{206.933334pt}{140.777435pt}}
\pgflineto{\pgfpoint{205.937347pt}{144.822937pt}}
\pgfusepath{stroke}
\pgfpathmoveto{\pgfpoint{215.897217pt}{140.777435pt}}
\pgflineto{\pgfpoint{216.893188pt}{140.777435pt}}
\pgfusepath{stroke}
\pgfpathmoveto{\pgfpoint{214.901215pt}{140.777435pt}}
\pgflineto{\pgfpoint{215.897217pt}{140.777435pt}}
\pgfusepath{stroke}
\pgfpathmoveto{\pgfpoint{213.905228pt}{140.777435pt}}
\pgflineto{\pgfpoint{214.901215pt}{140.777435pt}}
\pgfusepath{stroke}
\pgfpathmoveto{\pgfpoint{212.909241pt}{140.777435pt}}
\pgflineto{\pgfpoint{213.905228pt}{140.777435pt}}
\pgfusepath{stroke}
\pgfpathmoveto{\pgfpoint{211.913269pt}{140.777435pt}}
\pgflineto{\pgfpoint{212.909241pt}{140.777435pt}}
\pgfusepath{stroke}
\pgfpathmoveto{\pgfpoint{210.917267pt}{140.777435pt}}
\pgflineto{\pgfpoint{211.913269pt}{140.777435pt}}
\pgfusepath{stroke}
\pgfpathmoveto{\pgfpoint{209.921295pt}{140.777435pt}}
\pgflineto{\pgfpoint{210.917267pt}{140.777435pt}}
\pgfusepath{stroke}
\pgfpathmoveto{\pgfpoint{208.925323pt}{140.777435pt}}
\pgflineto{\pgfpoint{209.921295pt}{140.777435pt}}
\pgfusepath{stroke}
\pgfpathmoveto{\pgfpoint{207.929337pt}{140.777435pt}}
\pgflineto{\pgfpoint{208.925323pt}{140.777435pt}}
\pgfusepath{stroke}
\pgfpathmoveto{\pgfpoint{208.925323pt}{168.977936pt}}
\pgflineto{\pgfpoint{207.929337pt}{140.777435pt}}
\pgfusepath{stroke}
\pgfpathmoveto{\pgfpoint{209.608246pt}{205.642242pt}}
\pgflineto{\pgfpoint{208.925323pt}{168.977936pt}}
\pgfusepath{stroke}
\pgfpathmoveto{\pgfpoint{210.917267pt}{156.156219pt}}
\pgflineto{\pgfpoint{210.173798pt}{205.642242pt}}
\pgfusepath{stroke}
\pgfpathmoveto{\pgfpoint{211.913269pt}{142.768951pt}}
\pgflineto{\pgfpoint{210.917267pt}{156.156219pt}}
\pgfusepath{stroke}
\pgfpathmoveto{\pgfpoint{212.909241pt}{140.796997pt}}
\pgflineto{\pgfpoint{211.913269pt}{142.768951pt}}
\pgfusepath{stroke}
\pgfpathmoveto{\pgfpoint{213.905228pt}{147.912155pt}}
\pgflineto{\pgfpoint{212.909241pt}{140.796997pt}}
\pgfusepath{stroke}
\pgfpathmoveto{\pgfpoint{214.901215pt}{141.340469pt}}
\pgflineto{\pgfpoint{213.905228pt}{147.912155pt}}
\pgfusepath{stroke}
\pgfpathmoveto{\pgfpoint{215.897217pt}{146.916656pt}}
\pgflineto{\pgfpoint{214.901215pt}{141.340469pt}}
\pgfusepath{stroke}
\pgfpathmoveto{\pgfpoint{216.893188pt}{140.777435pt}}
\pgflineto{\pgfpoint{215.897217pt}{146.916656pt}}
\pgfusepath{stroke}
\pgfpathmoveto{\pgfpoint{217.889160pt}{140.777435pt}}
\pgflineto{\pgfpoint{218.885147pt}{140.777435pt}}
\pgfusepath{stroke}
\pgfpathmoveto{\pgfpoint{216.893188pt}{140.777435pt}}
\pgflineto{\pgfpoint{217.889160pt}{140.777435pt}}
\pgfusepath{stroke}
\pgfpathmoveto{\pgfpoint{217.889160pt}{144.868698pt}}
\pgflineto{\pgfpoint{216.893188pt}{140.777435pt}}
\pgfusepath{stroke}
\pgfpathmoveto{\pgfpoint{218.885147pt}{140.777435pt}}
\pgflineto{\pgfpoint{217.889160pt}{144.868698pt}}
\pgfusepath{stroke}
\pgfpathmoveto{\pgfpoint{219.881134pt}{140.777435pt}}
\pgflineto{\pgfpoint{220.877121pt}{140.777435pt}}
\pgfusepath{stroke}
\pgfpathmoveto{\pgfpoint{218.885147pt}{140.777435pt}}
\pgflineto{\pgfpoint{219.881134pt}{140.777435pt}}
\pgfusepath{stroke}
\pgfpathmoveto{\pgfpoint{219.881134pt}{141.240646pt}}
\pgflineto{\pgfpoint{218.885147pt}{140.777435pt}}
\pgfusepath{stroke}
\pgfpathmoveto{\pgfpoint{220.877121pt}{140.777435pt}}
\pgflineto{\pgfpoint{219.881134pt}{141.240646pt}}
\pgfusepath{stroke}
\pgfpathmoveto{\pgfpoint{225.857040pt}{140.777435pt}}
\pgflineto{\pgfpoint{226.853027pt}{140.777435pt}}
\pgfusepath{stroke}
\pgfpathmoveto{\pgfpoint{224.861053pt}{140.777435pt}}
\pgflineto{\pgfpoint{225.857040pt}{140.777435pt}}
\pgfusepath{stroke}
\pgfpathmoveto{\pgfpoint{223.865082pt}{140.777435pt}}
\pgflineto{\pgfpoint{224.861053pt}{140.777435pt}}
\pgfusepath{stroke}
\pgfpathmoveto{\pgfpoint{222.869080pt}{140.777435pt}}
\pgflineto{\pgfpoint{223.865082pt}{140.777435pt}}
\pgfusepath{stroke}
\pgfpathmoveto{\pgfpoint{223.865082pt}{142.648438pt}}
\pgflineto{\pgfpoint{222.869080pt}{140.777435pt}}
\pgfusepath{stroke}
\pgfpathmoveto{\pgfpoint{224.861053pt}{140.814316pt}}
\pgflineto{\pgfpoint{223.865082pt}{142.648438pt}}
\pgfusepath{stroke}
\pgfpathmoveto{\pgfpoint{225.857040pt}{142.893295pt}}
\pgflineto{\pgfpoint{224.861053pt}{140.814316pt}}
\pgfusepath{stroke}
\pgfpathmoveto{\pgfpoint{226.853027pt}{140.777435pt}}
\pgflineto{\pgfpoint{225.857040pt}{142.893295pt}}
\pgfusepath{stroke}
\pgfpathmoveto{\pgfpoint{231.832932pt}{140.777435pt}}
\pgflineto{\pgfpoint{232.828934pt}{140.777435pt}}
\pgfusepath{stroke}
\pgfpathmoveto{\pgfpoint{230.836945pt}{140.777435pt}}
\pgflineto{\pgfpoint{231.832932pt}{140.777435pt}}
\pgfusepath{stroke}
\pgfpathmoveto{\pgfpoint{229.840973pt}{140.777435pt}}
\pgflineto{\pgfpoint{230.836945pt}{140.777435pt}}
\pgfusepath{stroke}
\pgfpathmoveto{\pgfpoint{228.845001pt}{140.777435pt}}
\pgflineto{\pgfpoint{229.840973pt}{140.777435pt}}
\pgfusepath{stroke}
\pgfpathmoveto{\pgfpoint{227.849014pt}{140.777435pt}}
\pgflineto{\pgfpoint{228.845001pt}{140.777435pt}}
\pgfusepath{stroke}
\pgfpathmoveto{\pgfpoint{226.853027pt}{140.777435pt}}
\pgflineto{\pgfpoint{227.849014pt}{140.777435pt}}
\pgfusepath{stroke}
\pgfpathmoveto{\pgfpoint{227.849014pt}{141.269379pt}}
\pgflineto{\pgfpoint{226.853027pt}{140.777435pt}}
\pgfusepath{stroke}
\pgfpathmoveto{\pgfpoint{228.845001pt}{143.487183pt}}
\pgflineto{\pgfpoint{227.849014pt}{141.269379pt}}
\pgfusepath{stroke}
\pgfpathmoveto{\pgfpoint{229.840973pt}{143.864670pt}}
\pgflineto{\pgfpoint{228.845001pt}{143.487183pt}}
\pgfusepath{stroke}
\pgfpathmoveto{\pgfpoint{230.836945pt}{141.428329pt}}
\pgflineto{\pgfpoint{229.840973pt}{143.864670pt}}
\pgfusepath{stroke}
\pgfpathmoveto{\pgfpoint{231.832932pt}{162.511902pt}}
\pgflineto{\pgfpoint{230.836945pt}{141.428329pt}}
\pgfusepath{stroke}
\pgfpathmoveto{\pgfpoint{232.828934pt}{140.777435pt}}
\pgflineto{\pgfpoint{231.832932pt}{162.511902pt}}
\pgfusepath{stroke}
\pgfpathmoveto{\pgfpoint{238.804825pt}{140.777435pt}}
\pgflineto{\pgfpoint{239.800812pt}{140.777435pt}}
\pgfusepath{stroke}
\pgfpathmoveto{\pgfpoint{237.808838pt}{140.777435pt}}
\pgflineto{\pgfpoint{238.804825pt}{140.777435pt}}
\pgfusepath{stroke}
\pgfpathmoveto{\pgfpoint{236.812866pt}{140.777435pt}}
\pgflineto{\pgfpoint{237.808838pt}{140.777435pt}}
\pgfusepath{stroke}
\pgfpathmoveto{\pgfpoint{235.816864pt}{140.777435pt}}
\pgflineto{\pgfpoint{236.812866pt}{140.777435pt}}
\pgfusepath{stroke}
\pgfpathmoveto{\pgfpoint{234.820892pt}{140.777435pt}}
\pgflineto{\pgfpoint{235.816864pt}{140.777435pt}}
\pgfusepath{stroke}
\pgfpathmoveto{\pgfpoint{233.824921pt}{140.777435pt}}
\pgflineto{\pgfpoint{234.820892pt}{140.777435pt}}
\pgfusepath{stroke}
\pgfpathmoveto{\pgfpoint{232.828934pt}{140.777435pt}}
\pgflineto{\pgfpoint{233.824921pt}{140.777435pt}}
\pgfusepath{stroke}
\pgfpathmoveto{\pgfpoint{233.824921pt}{143.663483pt}}
\pgflineto{\pgfpoint{232.828934pt}{140.777435pt}}
\pgfusepath{stroke}
\pgfpathmoveto{\pgfpoint{234.820892pt}{144.424850pt}}
\pgflineto{\pgfpoint{233.824921pt}{143.663483pt}}
\pgfusepath{stroke}
\pgfpathmoveto{\pgfpoint{235.816864pt}{141.070068pt}}
\pgflineto{\pgfpoint{234.820892pt}{144.424850pt}}
\pgfusepath{stroke}
\pgfpathmoveto{\pgfpoint{236.812866pt}{141.387177pt}}
\pgflineto{\pgfpoint{235.816864pt}{141.070068pt}}
\pgfusepath{stroke}
\pgfpathmoveto{\pgfpoint{237.808838pt}{141.605896pt}}
\pgflineto{\pgfpoint{236.812866pt}{141.387177pt}}
\pgfusepath{stroke}
\pgfpathmoveto{\pgfpoint{238.804825pt}{141.892487pt}}
\pgflineto{\pgfpoint{237.808838pt}{141.605896pt}}
\pgfusepath{stroke}
\pgfpathmoveto{\pgfpoint{239.800812pt}{140.777435pt}}
\pgflineto{\pgfpoint{238.804825pt}{141.892487pt}}
\pgfusepath{stroke}
\pgfpathmoveto{\pgfpoint{240.796814pt}{140.777435pt}}
\pgflineto{\pgfpoint{241.792786pt}{140.777435pt}}
\pgfusepath{stroke}
\pgfpathmoveto{\pgfpoint{239.800812pt}{140.777435pt}}
\pgflineto{\pgfpoint{240.796814pt}{140.777435pt}}
\pgfusepath{stroke}
\pgfpathmoveto{\pgfpoint{240.796814pt}{141.635849pt}}
\pgflineto{\pgfpoint{239.800812pt}{140.777435pt}}
\pgfusepath{stroke}
\pgfpathmoveto{\pgfpoint{241.792786pt}{140.777435pt}}
\pgflineto{\pgfpoint{240.796814pt}{141.635849pt}}
\pgfusepath{stroke}
\pgfpathmoveto{\pgfpoint{243.784744pt}{140.777435pt}}
\pgflineto{\pgfpoint{244.780731pt}{140.777435pt}}
\pgfusepath{stroke}
\pgfpathmoveto{\pgfpoint{242.788757pt}{140.777435pt}}
\pgflineto{\pgfpoint{243.784744pt}{140.777435pt}}
\pgfusepath{stroke}
\pgfpathmoveto{\pgfpoint{241.792786pt}{140.777435pt}}
\pgflineto{\pgfpoint{242.788757pt}{140.777435pt}}
\pgfusepath{stroke}
\pgfpathmoveto{\pgfpoint{242.788757pt}{141.050537pt}}
\pgflineto{\pgfpoint{241.792786pt}{140.777435pt}}
\pgfusepath{stroke}
\pgfpathmoveto{\pgfpoint{243.784744pt}{140.826157pt}}
\pgflineto{\pgfpoint{242.788757pt}{141.050537pt}}
\pgfusepath{stroke}
\pgfpathmoveto{\pgfpoint{244.780731pt}{140.777435pt}}
\pgflineto{\pgfpoint{243.784744pt}{140.826157pt}}
\pgfusepath{stroke}
\pgfpathmoveto{\pgfpoint{250.756638pt}{140.777435pt}}
\pgflineto{\pgfpoint{251.752625pt}{140.777435pt}}
\pgfusepath{stroke}
\pgfpathmoveto{\pgfpoint{249.760651pt}{140.777435pt}}
\pgflineto{\pgfpoint{250.756638pt}{140.777435pt}}
\pgfusepath{stroke}
\pgfpathmoveto{\pgfpoint{248.764679pt}{140.777435pt}}
\pgflineto{\pgfpoint{249.760651pt}{140.777435pt}}
\pgfusepath{stroke}
\pgfpathmoveto{\pgfpoint{247.768677pt}{140.777435pt}}
\pgflineto{\pgfpoint{248.764679pt}{140.777435pt}}
\pgfusepath{stroke}
\pgfpathmoveto{\pgfpoint{246.772705pt}{140.777435pt}}
\pgflineto{\pgfpoint{247.768677pt}{140.777435pt}}
\pgfusepath{stroke}
\pgfpathmoveto{\pgfpoint{245.776718pt}{140.777435pt}}
\pgflineto{\pgfpoint{246.772705pt}{140.777435pt}}
\pgfusepath{stroke}
\pgfpathmoveto{\pgfpoint{244.780731pt}{140.777435pt}}
\pgflineto{\pgfpoint{245.776718pt}{140.777435pt}}
\pgfusepath{stroke}
\pgfpathmoveto{\pgfpoint{245.776718pt}{140.921082pt}}
\pgflineto{\pgfpoint{244.780731pt}{140.777435pt}}
\pgfusepath{stroke}
\pgfpathmoveto{\pgfpoint{246.230515pt}{205.642242pt}}
\pgflineto{\pgfpoint{245.776718pt}{140.921082pt}}
\pgfusepath{stroke}
\pgfpathmoveto{\pgfpoint{247.768677pt}{144.035980pt}}
\pgflineto{\pgfpoint{247.327026pt}{205.642242pt}}
\pgfusepath{stroke}
\pgfpathmoveto{\pgfpoint{248.764679pt}{140.987183pt}}
\pgflineto{\pgfpoint{247.768677pt}{144.035980pt}}
\pgfusepath{stroke}
\pgfpathmoveto{\pgfpoint{249.760651pt}{140.861908pt}}
\pgflineto{\pgfpoint{248.764679pt}{140.987183pt}}
\pgfusepath{stroke}
\pgfpathmoveto{\pgfpoint{250.756638pt}{143.997910pt}}
\pgflineto{\pgfpoint{249.760651pt}{140.861908pt}}
\pgfusepath{stroke}
\pgfpathmoveto{\pgfpoint{251.752625pt}{140.777435pt}}
\pgflineto{\pgfpoint{250.756638pt}{143.997910pt}}
\pgfusepath{stroke}
\pgfpathmoveto{\pgfpoint{254.740570pt}{140.777435pt}}
\pgflineto{\pgfpoint{255.736542pt}{140.777435pt}}
\pgfusepath{stroke}
\pgfpathmoveto{\pgfpoint{253.744598pt}{140.777435pt}}
\pgflineto{\pgfpoint{254.740570pt}{140.777435pt}}
\pgfusepath{stroke}
\pgfpathmoveto{\pgfpoint{252.748611pt}{140.777435pt}}
\pgflineto{\pgfpoint{253.744598pt}{140.777435pt}}
\pgfusepath{stroke}
\pgfpathmoveto{\pgfpoint{253.234100pt}{205.642242pt}}
\pgflineto{\pgfpoint{252.748611pt}{140.777435pt}}
\pgfusepath{stroke}
\pgfpathmoveto{\pgfpoint{255.736542pt}{140.777435pt}}
\pgflineto{\pgfpoint{255.155716pt}{205.642242pt}}
\pgfusepath{stroke}
\pgfpathmoveto{\pgfpoint{261.712463pt}{140.777435pt}}
\pgflineto{\pgfpoint{262.708435pt}{140.777435pt}}
\pgfusepath{stroke}
\pgfpathmoveto{\pgfpoint{260.716492pt}{140.777435pt}}
\pgflineto{\pgfpoint{261.712463pt}{140.777435pt}}
\pgfusepath{stroke}
\pgfpathmoveto{\pgfpoint{259.720490pt}{140.777435pt}}
\pgflineto{\pgfpoint{260.716492pt}{140.777435pt}}
\pgfusepath{stroke}
\pgfpathmoveto{\pgfpoint{258.724518pt}{140.777435pt}}
\pgflineto{\pgfpoint{259.720490pt}{140.777435pt}}
\pgfusepath{stroke}
\pgfpathmoveto{\pgfpoint{257.728516pt}{140.777435pt}}
\pgflineto{\pgfpoint{258.724518pt}{140.777435pt}}
\pgfusepath{stroke}
\pgfpathmoveto{\pgfpoint{256.732544pt}{140.777435pt}}
\pgflineto{\pgfpoint{257.728516pt}{140.777435pt}}
\pgfusepath{stroke}
\pgfpathmoveto{\pgfpoint{255.736542pt}{140.777435pt}}
\pgflineto{\pgfpoint{256.732544pt}{140.777435pt}}
\pgfusepath{stroke}
\pgfpathmoveto{\pgfpoint{256.732544pt}{160.055664pt}}
\pgflineto{\pgfpoint{255.736542pt}{140.777435pt}}
\pgfusepath{stroke}
\pgfpathmoveto{\pgfpoint{257.728516pt}{145.276688pt}}
\pgflineto{\pgfpoint{256.732544pt}{160.055664pt}}
\pgfusepath{stroke}
\pgfpathmoveto{\pgfpoint{258.724518pt}{145.143265pt}}
\pgflineto{\pgfpoint{257.728516pt}{145.276688pt}}
\pgfusepath{stroke}
\pgfpathmoveto{\pgfpoint{259.720490pt}{141.107025pt}}
\pgflineto{\pgfpoint{258.724518pt}{145.143265pt}}
\pgfusepath{stroke}
\pgfpathmoveto{\pgfpoint{260.716492pt}{140.785751pt}}
\pgflineto{\pgfpoint{259.720490pt}{141.107025pt}}
\pgfusepath{stroke}
\pgfpathmoveto{\pgfpoint{261.712463pt}{140.881470pt}}
\pgflineto{\pgfpoint{260.716492pt}{140.785751pt}}
\pgfusepath{stroke}
\pgfpathmoveto{\pgfpoint{262.708435pt}{140.777435pt}}
\pgflineto{\pgfpoint{261.712463pt}{140.881470pt}}
\pgfusepath{stroke}
\pgfpathmoveto{\pgfpoint{272.668274pt}{140.777435pt}}
\pgflineto{\pgfpoint{273.664276pt}{140.777435pt}}
\pgfusepath{stroke}
\pgfpathmoveto{\pgfpoint{271.672302pt}{140.777435pt}}
\pgflineto{\pgfpoint{272.668274pt}{140.777435pt}}
\pgfusepath{stroke}
\pgfpathmoveto{\pgfpoint{270.676331pt}{140.777435pt}}
\pgflineto{\pgfpoint{271.672302pt}{140.777435pt}}
\pgfusepath{stroke}
\pgfpathmoveto{\pgfpoint{269.680328pt}{140.777435pt}}
\pgflineto{\pgfpoint{270.676331pt}{140.777435pt}}
\pgfusepath{stroke}
\pgfpathmoveto{\pgfpoint{268.684326pt}{140.777435pt}}
\pgflineto{\pgfpoint{269.680328pt}{140.777435pt}}
\pgfusepath{stroke}
\pgfpathmoveto{\pgfpoint{267.688354pt}{140.777435pt}}
\pgflineto{\pgfpoint{268.684326pt}{140.777435pt}}
\pgfusepath{stroke}
\pgfpathmoveto{\pgfpoint{266.692383pt}{140.777435pt}}
\pgflineto{\pgfpoint{267.688354pt}{140.777435pt}}
\pgfusepath{stroke}
\pgfpathmoveto{\pgfpoint{265.696411pt}{140.777435pt}}
\pgflineto{\pgfpoint{266.692383pt}{140.777435pt}}
\pgfusepath{stroke}
\pgfpathmoveto{\pgfpoint{264.700409pt}{140.777435pt}}
\pgflineto{\pgfpoint{265.696411pt}{140.777435pt}}
\pgfusepath{stroke}
\pgfpathmoveto{\pgfpoint{263.704407pt}{140.777435pt}}
\pgflineto{\pgfpoint{264.700409pt}{140.777435pt}}
\pgfusepath{stroke}
\pgfpathmoveto{\pgfpoint{262.708435pt}{140.777435pt}}
\pgflineto{\pgfpoint{263.704407pt}{140.777435pt}}
\pgfusepath{stroke}
\pgfpathmoveto{\pgfpoint{263.704407pt}{149.672318pt}}
\pgflineto{\pgfpoint{262.708435pt}{140.777435pt}}
\pgfusepath{stroke}
\pgfpathmoveto{\pgfpoint{264.700409pt}{150.470245pt}}
\pgflineto{\pgfpoint{263.704407pt}{149.672318pt}}
\pgfusepath{stroke}
\pgfpathmoveto{\pgfpoint{265.696411pt}{146.772659pt}}
\pgflineto{\pgfpoint{264.700409pt}{150.470245pt}}
\pgfusepath{stroke}
\pgfpathmoveto{\pgfpoint{266.692383pt}{141.012543pt}}
\pgflineto{\pgfpoint{265.696411pt}{146.772659pt}}
\pgfusepath{stroke}
\pgfpathmoveto{\pgfpoint{267.688354pt}{140.947601pt}}
\pgflineto{\pgfpoint{266.692383pt}{141.012543pt}}
\pgfusepath{stroke}
\pgfpathmoveto{\pgfpoint{268.684326pt}{165.780823pt}}
\pgflineto{\pgfpoint{267.688354pt}{140.947601pt}}
\pgfusepath{stroke}
\pgfpathmoveto{\pgfpoint{269.680328pt}{142.178589pt}}
\pgflineto{\pgfpoint{268.684326pt}{165.780823pt}}
\pgfusepath{stroke}
\pgfpathmoveto{\pgfpoint{270.676331pt}{140.852936pt}}
\pgflineto{\pgfpoint{269.680328pt}{142.178589pt}}
\pgfusepath{stroke}
\pgfpathmoveto{\pgfpoint{271.672302pt}{149.853119pt}}
\pgflineto{\pgfpoint{270.676331pt}{140.852936pt}}
\pgfusepath{stroke}
\pgfpathmoveto{\pgfpoint{272.668274pt}{167.198486pt}}
\pgflineto{\pgfpoint{271.672302pt}{149.853119pt}}
\pgfusepath{stroke}
\pgfpathmoveto{\pgfpoint{273.664276pt}{140.777435pt}}
\pgflineto{\pgfpoint{272.668274pt}{167.198486pt}}
\pgfusepath{stroke}
\pgfpathmoveto{\pgfpoint{282.628113pt}{140.777435pt}}
\pgflineto{\pgfpoint{283.624115pt}{140.777435pt}}
\pgfusepath{stroke}
\pgfpathmoveto{\pgfpoint{281.632141pt}{140.777435pt}}
\pgflineto{\pgfpoint{282.628113pt}{140.777435pt}}
\pgfusepath{stroke}
\pgfpathmoveto{\pgfpoint{280.636139pt}{140.777435pt}}
\pgflineto{\pgfpoint{281.632141pt}{140.777435pt}}
\pgfusepath{stroke}
\pgfpathmoveto{\pgfpoint{279.640167pt}{140.777435pt}}
\pgflineto{\pgfpoint{280.636139pt}{140.777435pt}}
\pgfusepath{stroke}
\pgfpathmoveto{\pgfpoint{278.644196pt}{140.777435pt}}
\pgflineto{\pgfpoint{279.640167pt}{140.777435pt}}
\pgfusepath{stroke}
\pgfpathmoveto{\pgfpoint{277.648193pt}{140.777435pt}}
\pgflineto{\pgfpoint{278.644196pt}{140.777435pt}}
\pgfusepath{stroke}
\pgfpathmoveto{\pgfpoint{276.652222pt}{140.777435pt}}
\pgflineto{\pgfpoint{277.648193pt}{140.777435pt}}
\pgfusepath{stroke}
\pgfpathmoveto{\pgfpoint{275.656250pt}{140.777435pt}}
\pgflineto{\pgfpoint{276.652222pt}{140.777435pt}}
\pgfusepath{stroke}
\pgfpathmoveto{\pgfpoint{276.652222pt}{141.883286pt}}
\pgflineto{\pgfpoint{275.656250pt}{140.777435pt}}
\pgfusepath{stroke}
\pgfpathmoveto{\pgfpoint{277.648193pt}{145.700470pt}}
\pgflineto{\pgfpoint{276.652222pt}{141.883286pt}}
\pgfusepath{stroke}
\pgfpathmoveto{\pgfpoint{278.644196pt}{141.350510pt}}
\pgflineto{\pgfpoint{277.648193pt}{145.700470pt}}
\pgfusepath{stroke}
\pgfpathmoveto{\pgfpoint{279.640167pt}{141.685425pt}}
\pgflineto{\pgfpoint{278.644196pt}{141.350510pt}}
\pgfusepath{stroke}
\pgfpathmoveto{\pgfpoint{280.636139pt}{140.839111pt}}
\pgflineto{\pgfpoint{279.640167pt}{141.685425pt}}
\pgfusepath{stroke}
\pgfpathmoveto{\pgfpoint{281.632141pt}{140.783401pt}}
\pgflineto{\pgfpoint{280.636139pt}{140.839111pt}}
\pgfusepath{stroke}
\pgfpathmoveto{\pgfpoint{282.628113pt}{141.247986pt}}
\pgflineto{\pgfpoint{281.632141pt}{140.783401pt}}
\pgfusepath{stroke}
\pgfpathmoveto{\pgfpoint{283.624115pt}{140.777435pt}}
\pgflineto{\pgfpoint{282.628113pt}{141.247986pt}}
\pgfusepath{stroke}
\pgfpathmoveto{\pgfpoint{285.616089pt}{140.777435pt}}
\pgflineto{\pgfpoint{286.612061pt}{140.777435pt}}
\pgfusepath{stroke}
\pgfpathmoveto{\pgfpoint{284.620087pt}{140.777435pt}}
\pgflineto{\pgfpoint{285.616089pt}{140.777435pt}}
\pgfusepath{stroke}
\pgfpathmoveto{\pgfpoint{283.624115pt}{140.777435pt}}
\pgflineto{\pgfpoint{284.620087pt}{140.777435pt}}
\pgfusepath{stroke}
\pgfpathmoveto{\pgfpoint{284.620087pt}{140.819778pt}}
\pgflineto{\pgfpoint{283.624115pt}{140.777435pt}}
\pgfusepath{stroke}
\pgfpathmoveto{\pgfpoint{285.616089pt}{141.090790pt}}
\pgflineto{\pgfpoint{284.620087pt}{140.819778pt}}
\pgfusepath{stroke}
\pgfpathmoveto{\pgfpoint{286.612061pt}{140.777435pt}}
\pgflineto{\pgfpoint{285.616089pt}{141.090790pt}}
\pgfusepath{stroke}
\pgfpathmoveto{\pgfpoint{288.604004pt}{140.777435pt}}
\pgflineto{\pgfpoint{289.600037pt}{140.777435pt}}
\pgfusepath{stroke}
\pgfpathmoveto{\pgfpoint{287.608032pt}{140.777435pt}}
\pgflineto{\pgfpoint{288.604004pt}{140.777435pt}}
\pgfusepath{stroke}
\pgfpathmoveto{\pgfpoint{288.604004pt}{142.317886pt}}
\pgflineto{\pgfpoint{287.608032pt}{140.777435pt}}
\pgfusepath{stroke}
\pgfpathmoveto{\pgfpoint{289.600037pt}{141.234161pt}}
\pgflineto{\pgfpoint{288.604004pt}{142.317886pt}}
\pgfusepath{stroke}
\pgfpathmoveto{\pgfpoint{289.600037pt}{140.777435pt}}
\pgflineto{\pgfpoint{289.600037pt}{141.234161pt}}
\pgfusepath{stroke}
\color[rgb]{0.000000,0.000000,1.000000}
\pgfsetlinewidth{2.000000pt}
\pgfpathmoveto{\pgfpoint{42.595993pt}{140.777435pt}}
\pgflineto{\pgfpoint{41.600006pt}{140.777435pt}}
\pgfusepath{stroke}
\pgfpathmoveto{\pgfpoint{43.591980pt}{142.291229pt}}
\pgflineto{\pgfpoint{42.595993pt}{140.777435pt}}
\pgfusepath{stroke}
\pgfpathmoveto{\pgfpoint{44.587967pt}{140.777435pt}}
\pgflineto{\pgfpoint{43.591980pt}{142.291229pt}}
\pgfusepath{stroke}
\pgfpathmoveto{\pgfpoint{45.583946pt}{140.910431pt}}
\pgflineto{\pgfpoint{44.587967pt}{140.777435pt}}
\pgfusepath{stroke}
\pgfpathmoveto{\pgfpoint{46.579933pt}{140.777435pt}}
\pgflineto{\pgfpoint{45.583946pt}{140.910431pt}}
\pgfusepath{stroke}
\pgfpathmoveto{\pgfpoint{47.575912pt}{140.789795pt}}
\pgflineto{\pgfpoint{46.579933pt}{140.777435pt}}
\pgfusepath{stroke}
\pgfpathmoveto{\pgfpoint{48.571899pt}{140.788818pt}}
\pgflineto{\pgfpoint{47.575912pt}{140.789795pt}}
\pgfusepath{stroke}
\pgfpathmoveto{\pgfpoint{49.567879pt}{140.835007pt}}
\pgflineto{\pgfpoint{48.571899pt}{140.788818pt}}
\pgfusepath{stroke}
\pgfpathmoveto{\pgfpoint{50.563873pt}{140.777435pt}}
\pgflineto{\pgfpoint{49.567879pt}{140.835007pt}}
\pgfusepath{stroke}
\pgfpathmoveto{\pgfpoint{51.559845pt}{140.826508pt}}
\pgflineto{\pgfpoint{50.563873pt}{140.777435pt}}
\pgfusepath{stroke}
\pgfpathmoveto{\pgfpoint{52.555840pt}{140.814529pt}}
\pgflineto{\pgfpoint{51.559845pt}{140.826508pt}}
\pgfusepath{stroke}
\pgfpathmoveto{\pgfpoint{53.551819pt}{141.029465pt}}
\pgflineto{\pgfpoint{52.555840pt}{140.814529pt}}
\pgfusepath{stroke}
\pgfpathmoveto{\pgfpoint{54.547806pt}{140.800766pt}}
\pgflineto{\pgfpoint{53.551819pt}{141.029465pt}}
\pgfusepath{stroke}
\pgfpathmoveto{\pgfpoint{55.543785pt}{141.743744pt}}
\pgflineto{\pgfpoint{54.547806pt}{140.800766pt}}
\pgfusepath{stroke}
\pgfpathmoveto{\pgfpoint{56.539772pt}{143.031479pt}}
\pgflineto{\pgfpoint{55.543785pt}{141.743744pt}}
\pgfusepath{stroke}
\pgfpathmoveto{\pgfpoint{57.535751pt}{140.964340pt}}
\pgflineto{\pgfpoint{56.539772pt}{143.031479pt}}
\pgfusepath{stroke}
\pgfpathmoveto{\pgfpoint{58.531738pt}{140.780151pt}}
\pgflineto{\pgfpoint{57.535751pt}{140.964340pt}}
\pgfusepath{stroke}
\pgfpathmoveto{\pgfpoint{59.527725pt}{140.855103pt}}
\pgflineto{\pgfpoint{58.531738pt}{140.780151pt}}
\pgfusepath{stroke}
\pgfpathmoveto{\pgfpoint{60.523712pt}{140.787338pt}}
\pgflineto{\pgfpoint{59.527725pt}{140.855103pt}}
\pgfusepath{stroke}
\pgfpathmoveto{\pgfpoint{61.519691pt}{140.782700pt}}
\pgflineto{\pgfpoint{60.523712pt}{140.787338pt}}
\pgfusepath{stroke}
\pgfpathmoveto{\pgfpoint{62.515678pt}{148.013794pt}}
\pgflineto{\pgfpoint{61.519691pt}{140.782700pt}}
\pgfusepath{stroke}
\pgfpathmoveto{\pgfpoint{63.511658pt}{140.808426pt}}
\pgflineto{\pgfpoint{62.515678pt}{148.013794pt}}
\pgfusepath{stroke}
\pgfpathmoveto{\pgfpoint{64.507637pt}{140.848022pt}}
\pgflineto{\pgfpoint{63.511658pt}{140.808426pt}}
\pgfusepath{stroke}
\pgfpathmoveto{\pgfpoint{65.503624pt}{140.813904pt}}
\pgflineto{\pgfpoint{64.507637pt}{140.848022pt}}
\pgfusepath{stroke}
\pgfpathmoveto{\pgfpoint{66.499619pt}{140.822861pt}}
\pgflineto{\pgfpoint{65.503624pt}{140.813904pt}}
\pgfusepath{stroke}
\pgfpathmoveto{\pgfpoint{67.495590pt}{140.777435pt}}
\pgflineto{\pgfpoint{66.499619pt}{140.822861pt}}
\pgfusepath{stroke}
\pgfpathmoveto{\pgfpoint{68.491577pt}{140.777435pt}}
\pgflineto{\pgfpoint{67.495590pt}{140.777435pt}}
\pgfusepath{stroke}
\pgfpathmoveto{\pgfpoint{69.487564pt}{140.957489pt}}
\pgflineto{\pgfpoint{68.491577pt}{140.777435pt}}
\pgfusepath{stroke}
\pgfpathmoveto{\pgfpoint{70.483551pt}{140.814529pt}}
\pgflineto{\pgfpoint{69.487564pt}{140.957489pt}}
\pgfusepath{stroke}
\pgfpathmoveto{\pgfpoint{71.479530pt}{141.474945pt}}
\pgflineto{\pgfpoint{70.483551pt}{140.814529pt}}
\pgfusepath{stroke}
\pgfpathmoveto{\pgfpoint{72.475510pt}{140.931519pt}}
\pgflineto{\pgfpoint{71.479530pt}{141.474945pt}}
\pgfusepath{stroke}
\pgfpathmoveto{\pgfpoint{73.471497pt}{140.777435pt}}
\pgflineto{\pgfpoint{72.475510pt}{140.931519pt}}
\pgfusepath{stroke}
\pgfpathmoveto{\pgfpoint{74.467484pt}{140.789795pt}}
\pgflineto{\pgfpoint{73.471497pt}{140.777435pt}}
\pgfusepath{stroke}
\pgfpathmoveto{\pgfpoint{75.463470pt}{140.988373pt}}
\pgflineto{\pgfpoint{74.467484pt}{140.789795pt}}
\pgfusepath{stroke}
\pgfpathmoveto{\pgfpoint{76.459442pt}{140.912750pt}}
\pgflineto{\pgfpoint{75.463470pt}{140.988373pt}}
\pgfusepath{stroke}
\pgfpathmoveto{\pgfpoint{77.455437pt}{140.810455pt}}
\pgflineto{\pgfpoint{76.459442pt}{140.912750pt}}
\pgfusepath{stroke}
\pgfpathmoveto{\pgfpoint{78.451424pt}{140.777435pt}}
\pgflineto{\pgfpoint{77.455437pt}{140.810455pt}}
\pgfusepath{stroke}
\pgfpathmoveto{\pgfpoint{79.447403pt}{142.779175pt}}
\pgflineto{\pgfpoint{78.451424pt}{140.777435pt}}
\pgfusepath{stroke}
\pgfpathmoveto{\pgfpoint{80.443390pt}{140.781250pt}}
\pgflineto{\pgfpoint{79.447403pt}{142.779175pt}}
\pgfusepath{stroke}
\pgfpathmoveto{\pgfpoint{81.439369pt}{140.786606pt}}
\pgflineto{\pgfpoint{80.443390pt}{140.781250pt}}
\pgfusepath{stroke}
\pgfpathmoveto{\pgfpoint{82.435356pt}{140.785400pt}}
\pgflineto{\pgfpoint{81.439369pt}{140.786606pt}}
\pgfusepath{stroke}
\pgfpathmoveto{\pgfpoint{83.431335pt}{141.342422pt}}
\pgflineto{\pgfpoint{82.435356pt}{140.785400pt}}
\pgfusepath{stroke}
\pgfpathmoveto{\pgfpoint{84.427322pt}{140.777969pt}}
\pgflineto{\pgfpoint{83.431335pt}{141.342422pt}}
\pgfusepath{stroke}
\pgfpathmoveto{\pgfpoint{85.423309pt}{140.779861pt}}
\pgflineto{\pgfpoint{84.427322pt}{140.777969pt}}
\pgfusepath{stroke}
\pgfpathmoveto{\pgfpoint{86.419289pt}{140.777435pt}}
\pgflineto{\pgfpoint{85.423309pt}{140.779861pt}}
\pgfusepath{stroke}
\pgfpathmoveto{\pgfpoint{87.415276pt}{141.062363pt}}
\pgflineto{\pgfpoint{86.419289pt}{140.777435pt}}
\pgfusepath{stroke}
\pgfpathmoveto{\pgfpoint{88.411255pt}{141.826538pt}}
\pgflineto{\pgfpoint{87.415276pt}{141.062363pt}}
\pgfusepath{stroke}
\pgfpathmoveto{\pgfpoint{89.407242pt}{141.208725pt}}
\pgflineto{\pgfpoint{88.411255pt}{141.826538pt}}
\pgfusepath{stroke}
\pgfpathmoveto{\pgfpoint{90.403221pt}{141.961151pt}}
\pgflineto{\pgfpoint{89.407242pt}{141.208725pt}}
\pgfusepath{stroke}
\pgfpathmoveto{\pgfpoint{91.399208pt}{140.777435pt}}
\pgflineto{\pgfpoint{90.403221pt}{141.961151pt}}
\pgfusepath{stroke}
\pgfpathmoveto{\pgfpoint{92.395187pt}{140.777527pt}}
\pgflineto{\pgfpoint{91.399208pt}{140.777435pt}}
\pgfusepath{stroke}
\pgfpathmoveto{\pgfpoint{93.391174pt}{140.785782pt}}
\pgflineto{\pgfpoint{92.395187pt}{140.777527pt}}
\pgfusepath{stroke}
\pgfpathmoveto{\pgfpoint{94.387161pt}{140.777435pt}}
\pgflineto{\pgfpoint{93.391174pt}{140.785782pt}}
\pgfusepath{stroke}
\pgfpathmoveto{\pgfpoint{95.383141pt}{140.780762pt}}
\pgflineto{\pgfpoint{94.387161pt}{140.777435pt}}
\pgfusepath{stroke}
\pgfpathmoveto{\pgfpoint{96.379128pt}{140.781006pt}}
\pgflineto{\pgfpoint{95.383141pt}{140.780762pt}}
\pgfusepath{stroke}
\pgfpathmoveto{\pgfpoint{97.375107pt}{140.790512pt}}
\pgflineto{\pgfpoint{96.379128pt}{140.781006pt}}
\pgfusepath{stroke}
\pgfpathmoveto{\pgfpoint{98.371094pt}{141.396454pt}}
\pgflineto{\pgfpoint{97.375107pt}{140.790512pt}}
\pgfusepath{stroke}
\pgfpathmoveto{\pgfpoint{99.367081pt}{140.828232pt}}
\pgflineto{\pgfpoint{98.371094pt}{141.396454pt}}
\pgfusepath{stroke}
\pgfpathmoveto{\pgfpoint{100.363068pt}{140.788437pt}}
\pgflineto{\pgfpoint{99.367081pt}{140.828232pt}}
\pgfusepath{stroke}
\pgfpathmoveto{\pgfpoint{101.359047pt}{140.782562pt}}
\pgflineto{\pgfpoint{100.363068pt}{140.788437pt}}
\pgfusepath{stroke}
\pgfpathmoveto{\pgfpoint{102.355034pt}{140.893219pt}}
\pgflineto{\pgfpoint{101.359047pt}{140.782562pt}}
\pgfusepath{stroke}
\pgfpathmoveto{\pgfpoint{103.351013pt}{140.777435pt}}
\pgflineto{\pgfpoint{102.355034pt}{140.893219pt}}
\pgfusepath{stroke}
\pgfpathmoveto{\pgfpoint{104.347000pt}{141.194229pt}}
\pgflineto{\pgfpoint{103.351013pt}{140.777435pt}}
\pgfusepath{stroke}
\pgfpathmoveto{\pgfpoint{105.342987pt}{140.777435pt}}
\pgflineto{\pgfpoint{104.347000pt}{141.194229pt}}
\pgfusepath{stroke}
\pgfpathmoveto{\pgfpoint{106.338966pt}{140.777435pt}}
\pgflineto{\pgfpoint{105.342987pt}{140.777435pt}}
\pgfusepath{stroke}
\pgfpathmoveto{\pgfpoint{107.334953pt}{140.777435pt}}
\pgflineto{\pgfpoint{106.338966pt}{140.777435pt}}
\pgfusepath{stroke}
\pgfpathmoveto{\pgfpoint{108.330933pt}{141.164261pt}}
\pgflineto{\pgfpoint{107.334953pt}{140.777435pt}}
\pgfusepath{stroke}
\pgfpathmoveto{\pgfpoint{109.326920pt}{140.777985pt}}
\pgflineto{\pgfpoint{108.330933pt}{141.164261pt}}
\pgfusepath{stroke}
\pgfpathmoveto{\pgfpoint{110.322906pt}{140.811081pt}}
\pgflineto{\pgfpoint{109.326920pt}{140.777985pt}}
\pgfusepath{stroke}
\pgfpathmoveto{\pgfpoint{111.318893pt}{140.777435pt}}
\pgflineto{\pgfpoint{110.322906pt}{140.811081pt}}
\pgfusepath{stroke}
\pgfpathmoveto{\pgfpoint{112.314873pt}{140.798782pt}}
\pgflineto{\pgfpoint{111.318893pt}{140.777435pt}}
\pgfusepath{stroke}
\pgfpathmoveto{\pgfpoint{113.310852pt}{140.789963pt}}
\pgflineto{\pgfpoint{112.314873pt}{140.798782pt}}
\pgfusepath{stroke}
\pgfpathmoveto{\pgfpoint{114.306839pt}{140.825577pt}}
\pgflineto{\pgfpoint{113.310852pt}{140.789963pt}}
\pgfusepath{stroke}
\pgfpathmoveto{\pgfpoint{115.302826pt}{140.777435pt}}
\pgflineto{\pgfpoint{114.306839pt}{140.825577pt}}
\pgfusepath{stroke}
\pgfpathmoveto{\pgfpoint{116.298813pt}{140.784241pt}}
\pgflineto{\pgfpoint{115.302826pt}{140.777435pt}}
\pgfusepath{stroke}
\pgfpathmoveto{\pgfpoint{117.294792pt}{140.853699pt}}
\pgflineto{\pgfpoint{116.298813pt}{140.784241pt}}
\pgfusepath{stroke}
\pgfpathmoveto{\pgfpoint{118.290779pt}{140.778458pt}}
\pgflineto{\pgfpoint{117.294792pt}{140.853699pt}}
\pgfusepath{stroke}
\pgfpathmoveto{\pgfpoint{119.286758pt}{140.778992pt}}
\pgflineto{\pgfpoint{118.290779pt}{140.778458pt}}
\pgfusepath{stroke}
\pgfpathmoveto{\pgfpoint{120.282745pt}{140.777435pt}}
\pgflineto{\pgfpoint{119.286758pt}{140.778992pt}}
\pgfusepath{stroke}
\pgfpathmoveto{\pgfpoint{121.278725pt}{140.821732pt}}
\pgflineto{\pgfpoint{120.282745pt}{140.777435pt}}
\pgfusepath{stroke}
\pgfpathmoveto{\pgfpoint{122.274712pt}{140.777435pt}}
\pgflineto{\pgfpoint{121.278725pt}{140.821732pt}}
\pgfusepath{stroke}
\pgfpathmoveto{\pgfpoint{123.270691pt}{140.781097pt}}
\pgflineto{\pgfpoint{122.274712pt}{140.777435pt}}
\pgfusepath{stroke}
\pgfpathmoveto{\pgfpoint{124.266678pt}{140.777435pt}}
\pgflineto{\pgfpoint{123.270691pt}{140.781097pt}}
\pgfusepath{stroke}
\pgfpathmoveto{\pgfpoint{125.262665pt}{140.777435pt}}
\pgflineto{\pgfpoint{124.266678pt}{140.777435pt}}
\pgfusepath{stroke}
\pgfpathmoveto{\pgfpoint{126.258652pt}{140.787552pt}}
\pgflineto{\pgfpoint{125.262665pt}{140.777435pt}}
\pgfusepath{stroke}
\pgfpathmoveto{\pgfpoint{127.254631pt}{142.211929pt}}
\pgflineto{\pgfpoint{126.258652pt}{140.787552pt}}
\pgfusepath{stroke}
\pgfpathmoveto{\pgfpoint{128.250610pt}{140.777435pt}}
\pgflineto{\pgfpoint{127.254631pt}{142.211929pt}}
\pgfusepath{stroke}
\pgfpathmoveto{\pgfpoint{129.246597pt}{140.893631pt}}
\pgflineto{\pgfpoint{128.250610pt}{140.777435pt}}
\pgfusepath{stroke}
\pgfpathmoveto{\pgfpoint{130.242584pt}{140.790909pt}}
\pgflineto{\pgfpoint{129.246597pt}{140.893631pt}}
\pgfusepath{stroke}
\pgfpathmoveto{\pgfpoint{131.238571pt}{140.817093pt}}
\pgflineto{\pgfpoint{130.242584pt}{140.790909pt}}
\pgfusepath{stroke}
\pgfpathmoveto{\pgfpoint{132.234558pt}{140.777496pt}}
\pgflineto{\pgfpoint{131.238571pt}{140.817093pt}}
\pgfusepath{stroke}
\pgfpathmoveto{\pgfpoint{133.230530pt}{140.777435pt}}
\pgflineto{\pgfpoint{132.234558pt}{140.777496pt}}
\pgfusepath{stroke}
\pgfpathmoveto{\pgfpoint{134.226517pt}{140.777435pt}}
\pgflineto{\pgfpoint{133.230530pt}{140.777435pt}}
\pgfusepath{stroke}
\pgfpathmoveto{\pgfpoint{135.222504pt}{140.782410pt}}
\pgflineto{\pgfpoint{134.226517pt}{140.777435pt}}
\pgfusepath{stroke}
\pgfpathmoveto{\pgfpoint{136.218475pt}{140.777435pt}}
\pgflineto{\pgfpoint{135.222504pt}{140.782410pt}}
\pgfusepath{stroke}
\pgfpathmoveto{\pgfpoint{137.214478pt}{140.868332pt}}
\pgflineto{\pgfpoint{136.218475pt}{140.777435pt}}
\pgfusepath{stroke}
\pgfpathmoveto{\pgfpoint{138.210449pt}{140.777908pt}}
\pgflineto{\pgfpoint{137.214478pt}{140.868332pt}}
\pgfusepath{stroke}
\pgfpathmoveto{\pgfpoint{139.206436pt}{141.098373pt}}
\pgflineto{\pgfpoint{138.210449pt}{140.777908pt}}
\pgfusepath{stroke}
\pgfpathmoveto{\pgfpoint{140.202423pt}{140.810928pt}}
\pgflineto{\pgfpoint{139.206436pt}{141.098373pt}}
\pgfusepath{stroke}
\pgfpathmoveto{\pgfpoint{141.198410pt}{140.779694pt}}
\pgflineto{\pgfpoint{140.202423pt}{140.810928pt}}
\pgfusepath{stroke}
\pgfpathmoveto{\pgfpoint{142.194382pt}{140.777435pt}}
\pgflineto{\pgfpoint{141.198410pt}{140.779694pt}}
\pgfusepath{stroke}
\pgfpathmoveto{\pgfpoint{143.190369pt}{140.818817pt}}
\pgflineto{\pgfpoint{142.194382pt}{140.777435pt}}
\pgfusepath{stroke}
\pgfpathmoveto{\pgfpoint{144.186356pt}{140.777435pt}}
\pgflineto{\pgfpoint{143.190369pt}{140.818817pt}}
\pgfusepath{stroke}
\pgfpathmoveto{\pgfpoint{145.182343pt}{140.783417pt}}
\pgflineto{\pgfpoint{144.186356pt}{140.777435pt}}
\pgfusepath{stroke}
\pgfpathmoveto{\pgfpoint{146.178314pt}{140.779510pt}}
\pgflineto{\pgfpoint{145.182343pt}{140.783417pt}}
\pgfusepath{stroke}
\pgfpathmoveto{\pgfpoint{147.174316pt}{140.817200pt}}
\pgflineto{\pgfpoint{146.178314pt}{140.779510pt}}
\pgfusepath{stroke}
\pgfpathmoveto{\pgfpoint{148.170288pt}{140.777435pt}}
\pgflineto{\pgfpoint{147.174316pt}{140.817200pt}}
\pgfusepath{stroke}
\pgfpathmoveto{\pgfpoint{149.166275pt}{140.777435pt}}
\pgflineto{\pgfpoint{148.170288pt}{140.777435pt}}
\pgfusepath{stroke}
\pgfpathmoveto{\pgfpoint{150.162262pt}{140.777435pt}}
\pgflineto{\pgfpoint{149.166275pt}{140.777435pt}}
\pgfusepath{stroke}
\pgfpathmoveto{\pgfpoint{151.158249pt}{140.784195pt}}
\pgflineto{\pgfpoint{150.162262pt}{140.777435pt}}
\pgfusepath{stroke}
\pgfpathmoveto{\pgfpoint{152.154221pt}{140.816559pt}}
\pgflineto{\pgfpoint{151.158249pt}{140.784195pt}}
\pgfusepath{stroke}
\pgfpathmoveto{\pgfpoint{153.150208pt}{140.826141pt}}
\pgflineto{\pgfpoint{152.154221pt}{140.816559pt}}
\pgfusepath{stroke}
\pgfpathmoveto{\pgfpoint{154.146194pt}{140.777435pt}}
\pgflineto{\pgfpoint{153.150208pt}{140.826141pt}}
\pgfusepath{stroke}
\pgfpathmoveto{\pgfpoint{155.142181pt}{145.339951pt}}
\pgflineto{\pgfpoint{154.146194pt}{140.777435pt}}
\pgfusepath{stroke}
\pgfpathmoveto{\pgfpoint{156.138168pt}{140.777435pt}}
\pgflineto{\pgfpoint{155.142181pt}{145.339951pt}}
\pgfusepath{stroke}
\pgfpathmoveto{\pgfpoint{157.134155pt}{140.779984pt}}
\pgflineto{\pgfpoint{156.138168pt}{140.777435pt}}
\pgfusepath{stroke}
\pgfpathmoveto{\pgfpoint{158.130127pt}{140.780472pt}}
\pgflineto{\pgfpoint{157.134155pt}{140.779984pt}}
\pgfusepath{stroke}
\pgfpathmoveto{\pgfpoint{159.126114pt}{140.777435pt}}
\pgflineto{\pgfpoint{158.130127pt}{140.780472pt}}
\pgfusepath{stroke}
\pgfpathmoveto{\pgfpoint{160.122101pt}{140.912704pt}}
\pgflineto{\pgfpoint{159.126114pt}{140.777435pt}}
\pgfusepath{stroke}
\pgfpathmoveto{\pgfpoint{161.118088pt}{140.777557pt}}
\pgflineto{\pgfpoint{160.122101pt}{140.912704pt}}
\pgfusepath{stroke}
\pgfpathmoveto{\pgfpoint{162.114075pt}{140.788925pt}}
\pgflineto{\pgfpoint{161.118088pt}{140.777557pt}}
\pgfusepath{stroke}
\pgfpathmoveto{\pgfpoint{163.110062pt}{140.777435pt}}
\pgflineto{\pgfpoint{162.114075pt}{140.788925pt}}
\pgfusepath{stroke}
\pgfpathmoveto{\pgfpoint{164.106033pt}{141.622849pt}}
\pgflineto{\pgfpoint{163.110062pt}{140.777435pt}}
\pgfusepath{stroke}
\pgfpathmoveto{\pgfpoint{165.102020pt}{140.777802pt}}
\pgflineto{\pgfpoint{164.106033pt}{141.622849pt}}
\pgfusepath{stroke}
\pgfpathmoveto{\pgfpoint{166.098007pt}{140.777435pt}}
\pgflineto{\pgfpoint{165.102020pt}{140.777802pt}}
\pgfusepath{stroke}
\pgfpathmoveto{\pgfpoint{167.093994pt}{140.785172pt}}
\pgflineto{\pgfpoint{166.098007pt}{140.777435pt}}
\pgfusepath{stroke}
\pgfpathmoveto{\pgfpoint{168.089966pt}{165.120728pt}}
\pgflineto{\pgfpoint{167.093994pt}{140.785172pt}}
\pgfusepath{stroke}
\pgfpathmoveto{\pgfpoint{169.085953pt}{140.777435pt}}
\pgflineto{\pgfpoint{168.089966pt}{165.120728pt}}
\pgfusepath{stroke}
\pgfpathmoveto{\pgfpoint{170.081940pt}{141.018890pt}}
\pgflineto{\pgfpoint{169.085953pt}{140.777435pt}}
\pgfusepath{stroke}
\pgfpathmoveto{\pgfpoint{171.077911pt}{140.882141pt}}
\pgflineto{\pgfpoint{170.081940pt}{141.018890pt}}
\pgfusepath{stroke}
\pgfpathmoveto{\pgfpoint{172.073914pt}{141.041473pt}}
\pgflineto{\pgfpoint{171.077911pt}{140.882141pt}}
\pgfusepath{stroke}
\pgfpathmoveto{\pgfpoint{173.069885pt}{140.914948pt}}
\pgflineto{\pgfpoint{172.073914pt}{141.041473pt}}
\pgfusepath{stroke}
\pgfpathmoveto{\pgfpoint{174.065872pt}{140.782669pt}}
\pgflineto{\pgfpoint{173.069885pt}{140.914948pt}}
\pgfusepath{stroke}
\pgfpathmoveto{\pgfpoint{175.061859pt}{140.783112pt}}
\pgflineto{\pgfpoint{174.065872pt}{140.782669pt}}
\pgfusepath{stroke}
\pgfpathmoveto{\pgfpoint{176.057846pt}{140.777435pt}}
\pgflineto{\pgfpoint{175.061859pt}{140.783112pt}}
\pgfusepath{stroke}
\pgfpathmoveto{\pgfpoint{177.053818pt}{140.905045pt}}
\pgflineto{\pgfpoint{176.057846pt}{140.777435pt}}
\pgfusepath{stroke}
\pgfpathmoveto{\pgfpoint{178.049805pt}{140.866837pt}}
\pgflineto{\pgfpoint{177.053818pt}{140.905045pt}}
\pgfusepath{stroke}
\pgfpathmoveto{\pgfpoint{179.045792pt}{140.781509pt}}
\pgflineto{\pgfpoint{178.049805pt}{140.866837pt}}
\pgfusepath{stroke}
\pgfpathmoveto{\pgfpoint{180.041779pt}{140.779846pt}}
\pgflineto{\pgfpoint{179.045792pt}{140.781509pt}}
\pgfusepath{stroke}
\pgfpathmoveto{\pgfpoint{181.037766pt}{140.783401pt}}
\pgflineto{\pgfpoint{180.041779pt}{140.779846pt}}
\pgfusepath{stroke}
\pgfpathmoveto{\pgfpoint{182.033752pt}{140.777435pt}}
\pgflineto{\pgfpoint{181.037766pt}{140.783401pt}}
\pgfusepath{stroke}
\pgfpathmoveto{\pgfpoint{183.029724pt}{140.817596pt}}
\pgflineto{\pgfpoint{182.033752pt}{140.777435pt}}
\pgfusepath{stroke}
\pgfpathmoveto{\pgfpoint{184.025711pt}{140.777435pt}}
\pgflineto{\pgfpoint{183.029724pt}{140.817596pt}}
\pgfusepath{stroke}
\pgfpathmoveto{\pgfpoint{185.021698pt}{140.777435pt}}
\pgflineto{\pgfpoint{184.025711pt}{140.777435pt}}
\pgfusepath{stroke}
\pgfpathmoveto{\pgfpoint{186.017685pt}{140.777435pt}}
\pgflineto{\pgfpoint{185.021698pt}{140.777435pt}}
\pgfusepath{stroke}
\pgfpathmoveto{\pgfpoint{187.013672pt}{140.781067pt}}
\pgflineto{\pgfpoint{186.017685pt}{140.777435pt}}
\pgfusepath{stroke}
\pgfpathmoveto{\pgfpoint{188.009659pt}{140.777435pt}}
\pgflineto{\pgfpoint{187.013672pt}{140.781067pt}}
\pgfusepath{stroke}
\pgfpathmoveto{\pgfpoint{189.005630pt}{140.777634pt}}
\pgflineto{\pgfpoint{188.009659pt}{140.777435pt}}
\pgfusepath{stroke}
\pgfpathmoveto{\pgfpoint{190.001617pt}{141.032303pt}}
\pgflineto{\pgfpoint{189.005630pt}{140.777634pt}}
\pgfusepath{stroke}
\pgfpathmoveto{\pgfpoint{190.997604pt}{140.778046pt}}
\pgflineto{\pgfpoint{190.001617pt}{141.032303pt}}
\pgfusepath{stroke}
\pgfpathmoveto{\pgfpoint{191.993591pt}{140.805786pt}}
\pgflineto{\pgfpoint{190.997604pt}{140.778046pt}}
\pgfusepath{stroke}
\pgfpathmoveto{\pgfpoint{192.989563pt}{141.541718pt}}
\pgflineto{\pgfpoint{191.993591pt}{140.805786pt}}
\pgfusepath{stroke}
\pgfpathmoveto{\pgfpoint{193.985565pt}{140.790375pt}}
\pgflineto{\pgfpoint{192.989563pt}{141.541718pt}}
\pgfusepath{stroke}
\pgfpathmoveto{\pgfpoint{194.981537pt}{140.778259pt}}
\pgflineto{\pgfpoint{193.985565pt}{140.790375pt}}
\pgfusepath{stroke}
\pgfpathmoveto{\pgfpoint{195.977524pt}{141.314301pt}}
\pgflineto{\pgfpoint{194.981537pt}{140.778259pt}}
\pgfusepath{stroke}
\pgfpathmoveto{\pgfpoint{196.973511pt}{140.944916pt}}
\pgflineto{\pgfpoint{195.977524pt}{141.314301pt}}
\pgfusepath{stroke}
\pgfpathmoveto{\pgfpoint{197.969498pt}{140.809601pt}}
\pgflineto{\pgfpoint{196.973511pt}{140.944916pt}}
\pgfusepath{stroke}
\pgfpathmoveto{\pgfpoint{198.965469pt}{149.605469pt}}
\pgflineto{\pgfpoint{197.969498pt}{140.809601pt}}
\pgfusepath{stroke}
\pgfpathmoveto{\pgfpoint{199.961456pt}{140.777435pt}}
\pgflineto{\pgfpoint{198.965469pt}{149.605469pt}}
\pgfusepath{stroke}
\pgfpathmoveto{\pgfpoint{200.957443pt}{140.786819pt}}
\pgflineto{\pgfpoint{199.961456pt}{140.777435pt}}
\pgfusepath{stroke}
\pgfpathmoveto{\pgfpoint{201.953430pt}{140.778427pt}}
\pgflineto{\pgfpoint{200.957443pt}{140.786819pt}}
\pgfusepath{stroke}
\pgfpathmoveto{\pgfpoint{202.949402pt}{140.777435pt}}
\pgflineto{\pgfpoint{201.953430pt}{140.778427pt}}
\pgfusepath{stroke}
\pgfpathmoveto{\pgfpoint{203.945404pt}{140.777435pt}}
\pgflineto{\pgfpoint{202.949402pt}{140.777435pt}}
\pgfusepath{stroke}
\pgfpathmoveto{\pgfpoint{204.941376pt}{140.777435pt}}
\pgflineto{\pgfpoint{203.945404pt}{140.777435pt}}
\pgfusepath{stroke}
\pgfpathmoveto{\pgfpoint{205.937347pt}{140.854156pt}}
\pgflineto{\pgfpoint{204.941376pt}{140.777435pt}}
\pgfusepath{stroke}
\pgfpathmoveto{\pgfpoint{206.933334pt}{140.777435pt}}
\pgflineto{\pgfpoint{205.937347pt}{140.854156pt}}
\pgfusepath{stroke}
\pgfpathmoveto{\pgfpoint{207.929337pt}{140.777435pt}}
\pgflineto{\pgfpoint{206.933334pt}{140.777435pt}}
\pgfusepath{stroke}
\pgfpathmoveto{\pgfpoint{208.925323pt}{141.508636pt}}
\pgflineto{\pgfpoint{207.929337pt}{140.777435pt}}
\pgfusepath{stroke}
\pgfpathmoveto{\pgfpoint{209.921295pt}{143.205780pt}}
\pgflineto{\pgfpoint{208.925323pt}{141.508636pt}}
\pgfusepath{stroke}
\pgfpathmoveto{\pgfpoint{210.917267pt}{141.750488pt}}
\pgflineto{\pgfpoint{209.921295pt}{143.205780pt}}
\pgfusepath{stroke}
\pgfpathmoveto{\pgfpoint{211.913269pt}{140.869766pt}}
\pgflineto{\pgfpoint{210.917267pt}{141.750488pt}}
\pgfusepath{stroke}
\pgfpathmoveto{\pgfpoint{212.909241pt}{140.777573pt}}
\pgflineto{\pgfpoint{211.913269pt}{140.869766pt}}
\pgfusepath{stroke}
\pgfpathmoveto{\pgfpoint{213.905228pt}{140.910294pt}}
\pgflineto{\pgfpoint{212.909241pt}{140.777573pt}}
\pgfusepath{stroke}
\pgfpathmoveto{\pgfpoint{214.901215pt}{140.781433pt}}
\pgflineto{\pgfpoint{213.905228pt}{140.910294pt}}
\pgfusepath{stroke}
\pgfpathmoveto{\pgfpoint{215.897217pt}{141.024170pt}}
\pgflineto{\pgfpoint{214.901215pt}{140.781433pt}}
\pgfusepath{stroke}
\pgfpathmoveto{\pgfpoint{216.893188pt}{140.777435pt}}
\pgflineto{\pgfpoint{215.897217pt}{141.024170pt}}
\pgfusepath{stroke}
\pgfpathmoveto{\pgfpoint{217.889160pt}{140.945007pt}}
\pgflineto{\pgfpoint{216.893188pt}{140.777435pt}}
\pgfusepath{stroke}
\pgfpathmoveto{\pgfpoint{218.885147pt}{140.777435pt}}
\pgflineto{\pgfpoint{217.889160pt}{140.945007pt}}
\pgfusepath{stroke}
\pgfpathmoveto{\pgfpoint{219.881134pt}{140.780746pt}}
\pgflineto{\pgfpoint{218.885147pt}{140.777435pt}}
\pgfusepath{stroke}
\pgfpathmoveto{\pgfpoint{220.877121pt}{140.777435pt}}
\pgflineto{\pgfpoint{219.881134pt}{140.780746pt}}
\pgfusepath{stroke}
\pgfpathmoveto{\pgfpoint{221.873108pt}{140.777435pt}}
\pgflineto{\pgfpoint{220.877121pt}{140.777435pt}}
\pgfusepath{stroke}
\pgfpathmoveto{\pgfpoint{222.869080pt}{140.777435pt}}
\pgflineto{\pgfpoint{221.873108pt}{140.777435pt}}
\pgfusepath{stroke}
\pgfpathmoveto{\pgfpoint{223.865082pt}{140.800522pt}}
\pgflineto{\pgfpoint{222.869080pt}{140.777435pt}}
\pgfusepath{stroke}
\pgfpathmoveto{\pgfpoint{224.861053pt}{140.777695pt}}
\pgflineto{\pgfpoint{223.865082pt}{140.800522pt}}
\pgfusepath{stroke}
\pgfpathmoveto{\pgfpoint{225.857040pt}{140.916656pt}}
\pgflineto{\pgfpoint{224.861053pt}{140.777695pt}}
\pgfusepath{stroke}
\pgfpathmoveto{\pgfpoint{226.853027pt}{140.777435pt}}
\pgflineto{\pgfpoint{225.857040pt}{140.916656pt}}
\pgfusepath{stroke}
\pgfpathmoveto{\pgfpoint{227.849014pt}{140.782928pt}}
\pgflineto{\pgfpoint{226.853027pt}{140.777435pt}}
\pgfusepath{stroke}
\pgfpathmoveto{\pgfpoint{228.845001pt}{140.833771pt}}
\pgflineto{\pgfpoint{227.849014pt}{140.782928pt}}
\pgfusepath{stroke}
\pgfpathmoveto{\pgfpoint{229.840973pt}{140.894608pt}}
\pgflineto{\pgfpoint{228.845001pt}{140.833771pt}}
\pgfusepath{stroke}
\pgfpathmoveto{\pgfpoint{230.836945pt}{140.789825pt}}
\pgflineto{\pgfpoint{229.840973pt}{140.894608pt}}
\pgfusepath{stroke}
\pgfpathmoveto{\pgfpoint{231.832932pt}{142.041382pt}}
\pgflineto{\pgfpoint{230.836945pt}{140.789825pt}}
\pgfusepath{stroke}
\pgfpathmoveto{\pgfpoint{232.828934pt}{140.777435pt}}
\pgflineto{\pgfpoint{231.832932pt}{142.041382pt}}
\pgfusepath{stroke}
\pgfpathmoveto{\pgfpoint{233.824921pt}{140.823318pt}}
\pgflineto{\pgfpoint{232.828934pt}{140.777435pt}}
\pgfusepath{stroke}
\pgfpathmoveto{\pgfpoint{234.820892pt}{140.922363pt}}
\pgflineto{\pgfpoint{233.824921pt}{140.823318pt}}
\pgfusepath{stroke}
\pgfpathmoveto{\pgfpoint{235.816864pt}{140.779510pt}}
\pgflineto{\pgfpoint{234.820892pt}{140.922363pt}}
\pgfusepath{stroke}
\pgfpathmoveto{\pgfpoint{236.812866pt}{140.790619pt}}
\pgflineto{\pgfpoint{235.816864pt}{140.779510pt}}
\pgfusepath{stroke}
\pgfpathmoveto{\pgfpoint{237.808838pt}{140.799713pt}}
\pgflineto{\pgfpoint{236.812866pt}{140.790619pt}}
\pgfusepath{stroke}
\pgfpathmoveto{\pgfpoint{238.804825pt}{140.785339pt}}
\pgflineto{\pgfpoint{237.808838pt}{140.799713pt}}
\pgfusepath{stroke}
\pgfpathmoveto{\pgfpoint{239.800812pt}{140.777435pt}}
\pgflineto{\pgfpoint{238.804825pt}{140.785339pt}}
\pgfusepath{stroke}
\pgfpathmoveto{\pgfpoint{240.796814pt}{140.796509pt}}
\pgflineto{\pgfpoint{239.800812pt}{140.777435pt}}
\pgfusepath{stroke}
\pgfpathmoveto{\pgfpoint{241.792786pt}{140.777435pt}}
\pgflineto{\pgfpoint{240.796814pt}{140.796509pt}}
\pgfusepath{stroke}
\pgfpathmoveto{\pgfpoint{242.788757pt}{140.780701pt}}
\pgflineto{\pgfpoint{241.792786pt}{140.777435pt}}
\pgfusepath{stroke}
\pgfpathmoveto{\pgfpoint{243.784744pt}{140.777786pt}}
\pgflineto{\pgfpoint{242.788757pt}{140.780701pt}}
\pgfusepath{stroke}
\pgfpathmoveto{\pgfpoint{244.780731pt}{140.777435pt}}
\pgflineto{\pgfpoint{243.784744pt}{140.777786pt}}
\pgfusepath{stroke}
\pgfpathmoveto{\pgfpoint{245.776718pt}{140.778488pt}}
\pgflineto{\pgfpoint{244.780731pt}{140.777435pt}}
\pgfusepath{stroke}
\pgfpathmoveto{\pgfpoint{246.772705pt}{143.280365pt}}
\pgflineto{\pgfpoint{245.776718pt}{140.778488pt}}
\pgfusepath{stroke}
\pgfpathmoveto{\pgfpoint{247.768677pt}{140.830551pt}}
\pgflineto{\pgfpoint{246.772705pt}{143.280365pt}}
\pgfusepath{stroke}
\pgfpathmoveto{\pgfpoint{248.764679pt}{140.781235pt}}
\pgflineto{\pgfpoint{247.768677pt}{140.830551pt}}
\pgfusepath{stroke}
\pgfpathmoveto{\pgfpoint{249.760651pt}{140.778198pt}}
\pgflineto{\pgfpoint{248.764679pt}{140.781235pt}}
\pgfusepath{stroke}
\pgfpathmoveto{\pgfpoint{250.756638pt}{140.823486pt}}
\pgflineto{\pgfpoint{249.760651pt}{140.778198pt}}
\pgfusepath{stroke}
\pgfpathmoveto{\pgfpoint{251.752625pt}{140.777435pt}}
\pgflineto{\pgfpoint{250.756638pt}{140.823486pt}}
\pgfusepath{stroke}
\pgfpathmoveto{\pgfpoint{252.748611pt}{140.777435pt}}
\pgflineto{\pgfpoint{251.752625pt}{140.777435pt}}
\pgfusepath{stroke}
\pgfpathmoveto{\pgfpoint{253.744598pt}{142.920593pt}}
\pgflineto{\pgfpoint{252.748611pt}{140.777435pt}}
\pgfusepath{stroke}
\pgfpathmoveto{\pgfpoint{254.740570pt}{142.519669pt}}
\pgflineto{\pgfpoint{253.744598pt}{142.920593pt}}
\pgfusepath{stroke}
\pgfpathmoveto{\pgfpoint{255.736542pt}{140.777435pt}}
\pgflineto{\pgfpoint{254.740570pt}{142.519669pt}}
\pgfusepath{stroke}
\pgfpathmoveto{\pgfpoint{256.732544pt}{143.110718pt}}
\pgflineto{\pgfpoint{255.736542pt}{140.777435pt}}
\pgfusepath{stroke}
\pgfpathmoveto{\pgfpoint{257.728516pt}{140.907501pt}}
\pgflineto{\pgfpoint{256.732544pt}{143.110718pt}}
\pgfusepath{stroke}
\pgfpathmoveto{\pgfpoint{258.724518pt}{140.840500pt}}
\pgflineto{\pgfpoint{257.728516pt}{140.907501pt}}
\pgfusepath{stroke}
\pgfpathmoveto{\pgfpoint{259.720490pt}{140.779770pt}}
\pgflineto{\pgfpoint{258.724518pt}{140.840500pt}}
\pgfusepath{stroke}
\pgfpathmoveto{\pgfpoint{260.716492pt}{140.777496pt}}
\pgflineto{\pgfpoint{259.720490pt}{140.779770pt}}
\pgfusepath{stroke}
\pgfpathmoveto{\pgfpoint{261.712463pt}{140.779022pt}}
\pgflineto{\pgfpoint{260.716492pt}{140.777496pt}}
\pgfusepath{stroke}
\pgfpathmoveto{\pgfpoint{262.708435pt}{140.777435pt}}
\pgflineto{\pgfpoint{261.712463pt}{140.779022pt}}
\pgfusepath{stroke}
\pgfpathmoveto{\pgfpoint{263.704407pt}{140.927124pt}}
\pgflineto{\pgfpoint{262.708435pt}{140.777435pt}}
\pgfusepath{stroke}
\pgfpathmoveto{\pgfpoint{264.700409pt}{141.153885pt}}
\pgflineto{\pgfpoint{263.704407pt}{140.927124pt}}
\pgfusepath{stroke}
\pgfpathmoveto{\pgfpoint{265.696411pt}{140.951752pt}}
\pgflineto{\pgfpoint{264.700409pt}{141.153885pt}}
\pgfusepath{stroke}
\pgfpathmoveto{\pgfpoint{266.692383pt}{140.785080pt}}
\pgflineto{\pgfpoint{265.696411pt}{140.951752pt}}
\pgfusepath{stroke}
\pgfpathmoveto{\pgfpoint{267.688354pt}{140.779419pt}}
\pgflineto{\pgfpoint{266.692383pt}{140.785080pt}}
\pgfusepath{stroke}
\pgfpathmoveto{\pgfpoint{268.684326pt}{140.976303pt}}
\pgflineto{\pgfpoint{267.688354pt}{140.779419pt}}
\pgfusepath{stroke}
\pgfpathmoveto{\pgfpoint{269.680328pt}{140.824692pt}}
\pgflineto{\pgfpoint{268.684326pt}{140.976303pt}}
\pgfusepath{stroke}
\pgfpathmoveto{\pgfpoint{270.676331pt}{140.778473pt}}
\pgflineto{\pgfpoint{269.680328pt}{140.824692pt}}
\pgfusepath{stroke}
\pgfpathmoveto{\pgfpoint{271.672302pt}{141.361237pt}}
\pgflineto{\pgfpoint{270.676331pt}{140.778473pt}}
\pgfusepath{stroke}
\pgfpathmoveto{\pgfpoint{272.668274pt}{141.409683pt}}
\pgflineto{\pgfpoint{271.672302pt}{141.361237pt}}
\pgfusepath{stroke}
\pgfpathmoveto{\pgfpoint{273.664276pt}{140.777435pt}}
\pgflineto{\pgfpoint{272.668274pt}{141.409683pt}}
\pgfusepath{stroke}
\pgfpathmoveto{\pgfpoint{274.660248pt}{140.777435pt}}
\pgflineto{\pgfpoint{273.664276pt}{140.777435pt}}
\pgfusepath{stroke}
\pgfpathmoveto{\pgfpoint{275.656250pt}{140.777435pt}}
\pgflineto{\pgfpoint{274.660248pt}{140.777435pt}}
\pgfusepath{stroke}
\pgfpathmoveto{\pgfpoint{276.652222pt}{140.792587pt}}
\pgflineto{\pgfpoint{275.656250pt}{140.777435pt}}
\pgfusepath{stroke}
\pgfpathmoveto{\pgfpoint{277.648193pt}{140.928909pt}}
\pgflineto{\pgfpoint{276.652222pt}{140.792587pt}}
\pgfusepath{stroke}
\pgfpathmoveto{\pgfpoint{278.644196pt}{140.783813pt}}
\pgflineto{\pgfpoint{277.648193pt}{140.928909pt}}
\pgfusepath{stroke}
\pgfpathmoveto{\pgfpoint{279.640167pt}{140.790070pt}}
\pgflineto{\pgfpoint{278.644196pt}{140.783813pt}}
\pgfusepath{stroke}
\pgfpathmoveto{\pgfpoint{280.636139pt}{140.777878pt}}
\pgflineto{\pgfpoint{279.640167pt}{140.790070pt}}
\pgfusepath{stroke}
\pgfpathmoveto{\pgfpoint{281.632141pt}{140.777481pt}}
\pgflineto{\pgfpoint{280.636139pt}{140.777878pt}}
\pgfusepath{stroke}
\pgfpathmoveto{\pgfpoint{282.628113pt}{140.782150pt}}
\pgflineto{\pgfpoint{281.632141pt}{140.777481pt}}
\pgfusepath{stroke}
\pgfpathmoveto{\pgfpoint{283.624115pt}{140.777435pt}}
\pgflineto{\pgfpoint{282.628113pt}{140.782150pt}}
\pgfusepath{stroke}
\pgfpathmoveto{\pgfpoint{284.620087pt}{140.777740pt}}
\pgflineto{\pgfpoint{283.624115pt}{140.777435pt}}
\pgfusepath{stroke}
\pgfpathmoveto{\pgfpoint{285.616089pt}{140.784164pt}}
\pgflineto{\pgfpoint{284.620087pt}{140.777740pt}}
\pgfusepath{stroke}
\pgfpathmoveto{\pgfpoint{286.612061pt}{140.777435pt}}
\pgflineto{\pgfpoint{285.616089pt}{140.784164pt}}
\pgfusepath{stroke}
\pgfpathmoveto{\pgfpoint{287.608032pt}{140.777435pt}}
\pgflineto{\pgfpoint{286.612061pt}{140.777435pt}}
\pgfusepath{stroke}
\pgfpathmoveto{\pgfpoint{288.604004pt}{140.798889pt}}
\pgflineto{\pgfpoint{287.608032pt}{140.777435pt}}
\pgfusepath{stroke}
\pgfpathmoveto{\pgfpoint{289.600037pt}{140.790405pt}}
\pgflineto{\pgfpoint{288.604004pt}{140.798889pt}}
\pgfusepath{stroke}
\color[rgb]{0.000000,0.000000,0.000000}
\pgfsetlinewidth{0.500000pt}
\pgfsetdash{{16pt}{0pt}}{0pt}
\pgfpathmoveto{\pgfpoint{289.600037pt}{26.399979pt}}
\pgflineto{\pgfpoint{41.600006pt}{26.399979pt}}
\pgfusepath{stroke}
\pgfpathmoveto{\pgfpoint{289.600037pt}{91.199989pt}}
\pgflineto{\pgfpoint{41.600006pt}{91.199989pt}}
\pgfusepath{stroke}
\pgfpathmoveto{\pgfpoint{41.600006pt}{91.199989pt}}
\pgflineto{\pgfpoint{41.600006pt}{26.399979pt}}
\pgfusepath{stroke}
\pgfpathmoveto{\pgfpoint{289.600037pt}{91.199989pt}}
\pgflineto{\pgfpoint{289.600037pt}{26.399979pt}}
\pgfusepath{stroke}
\pgfpathmoveto{\pgfpoint{90.403221pt}{28.872353pt}}
\pgflineto{\pgfpoint{90.403221pt}{26.399979pt}}
\pgfusepath{stroke}
\pgfpathmoveto{\pgfpoint{90.403221pt}{88.727623pt}}
\pgflineto{\pgfpoint{90.403221pt}{91.199989pt}}
\pgfusepath{stroke}
\pgfpathmoveto{\pgfpoint{140.202423pt}{28.872353pt}}
\pgflineto{\pgfpoint{140.202423pt}{26.399979pt}}
\pgfusepath{stroke}
\pgfpathmoveto{\pgfpoint{140.202423pt}{88.727623pt}}
\pgflineto{\pgfpoint{140.202423pt}{91.199989pt}}
\pgfusepath{stroke}
\pgfpathmoveto{\pgfpoint{190.001617pt}{28.872353pt}}
\pgflineto{\pgfpoint{190.001617pt}{26.399979pt}}
\pgfusepath{stroke}
\pgfpathmoveto{\pgfpoint{190.001617pt}{88.727623pt}}
\pgflineto{\pgfpoint{190.001617pt}{91.199989pt}}
\pgfusepath{stroke}
\pgfpathmoveto{\pgfpoint{239.800812pt}{28.872353pt}}
\pgflineto{\pgfpoint{239.800812pt}{26.399979pt}}
\pgfusepath{stroke}
\pgfpathmoveto{\pgfpoint{239.800812pt}{88.727623pt}}
\pgflineto{\pgfpoint{239.800812pt}{91.199989pt}}
\pgfusepath{stroke}
\pgfpathmoveto{\pgfpoint{289.600037pt}{28.872353pt}}
\pgflineto{\pgfpoint{289.600037pt}{26.399979pt}}
\pgfusepath{stroke}
\pgfpathmoveto{\pgfpoint{289.600037pt}{88.727623pt}}
\pgflineto{\pgfpoint{289.600037pt}{91.199989pt}}
\pgfusepath{stroke}
{
\pgftransformshift{\pgfpoint{90.403229pt}{21.415367pt}}
\pgfnode{rectangle}{north}{\fontsize{10}{0}\selectfont\textcolor[rgb]{0,0,0}{{50}}}{}{\pgfusepath{discard}}}
{
\pgftransformshift{\pgfpoint{140.202423pt}{21.415367pt}}
\pgfnode{rectangle}{north}{\fontsize{10}{0}\selectfont\textcolor[rgb]{0,0,0}{{100}}}{}{\pgfusepath{discard}}}
{
\pgftransformshift{\pgfpoint{190.001617pt}{21.415367pt}}
\pgfnode{rectangle}{north}{\fontsize{10}{0}\selectfont\textcolor[rgb]{0,0,0}{{150}}}{}{\pgfusepath{discard}}}
{
\pgftransformshift{\pgfpoint{239.800812pt}{21.415367pt}}
\pgfnode{rectangle}{north}{\fontsize{10}{0}\selectfont\textcolor[rgb]{0,0,0}{{200}}}{}{\pgfusepath{discard}}}
{
\pgftransformshift{\pgfpoint{289.600037pt}{21.415367pt}}
\pgfnode{rectangle}{north}{\fontsize{10}{0}\selectfont\textcolor[rgb]{0,0,0}{{250}}}{}{\pgfusepath{discard}}}
\pgfpathmoveto{\pgfpoint{44.080009pt}{26.399979pt}}
\pgflineto{\pgfpoint{41.600006pt}{26.399979pt}}
\pgfusepath{stroke}
\pgfpathmoveto{\pgfpoint{287.119995pt}{26.399979pt}}
\pgflineto{\pgfpoint{289.600037pt}{26.399979pt}}
\pgfusepath{stroke}
\pgfpathmoveto{\pgfpoint{44.080009pt}{39.359985pt}}
\pgflineto{\pgfpoint{41.600006pt}{39.359985pt}}
\pgfusepath{stroke}
\pgfpathmoveto{\pgfpoint{287.119995pt}{39.359985pt}}
\pgflineto{\pgfpoint{289.600037pt}{39.359985pt}}
\pgfusepath{stroke}
\pgfpathmoveto{\pgfpoint{44.080009pt}{52.319984pt}}
\pgflineto{\pgfpoint{41.600006pt}{52.319984pt}}
\pgfusepath{stroke}
\pgfpathmoveto{\pgfpoint{287.119995pt}{52.319984pt}}
\pgflineto{\pgfpoint{289.600037pt}{52.319984pt}}
\pgfusepath{stroke}
\pgfpathmoveto{\pgfpoint{44.080009pt}{65.279991pt}}
\pgflineto{\pgfpoint{41.600006pt}{65.279991pt}}
\pgfusepath{stroke}
\pgfpathmoveto{\pgfpoint{287.119995pt}{65.279991pt}}
\pgflineto{\pgfpoint{289.600037pt}{65.279991pt}}
\pgfusepath{stroke}
\pgfpathmoveto{\pgfpoint{44.080009pt}{78.239990pt}}
\pgflineto{\pgfpoint{41.600006pt}{78.239990pt}}
\pgfusepath{stroke}
\pgfpathmoveto{\pgfpoint{287.119995pt}{78.239990pt}}
\pgflineto{\pgfpoint{289.600037pt}{78.239990pt}}
\pgfusepath{stroke}
\pgfpathmoveto{\pgfpoint{44.080009pt}{91.199989pt}}
\pgflineto{\pgfpoint{41.600006pt}{91.199989pt}}
\pgfusepath{stroke}
\pgfpathmoveto{\pgfpoint{287.119995pt}{91.199989pt}}
\pgflineto{\pgfpoint{289.600037pt}{91.199989pt}}
\pgfusepath{stroke}
{
\pgftransformshift{\pgfpoint{36.600006pt}{26.399979pt}}
\pgfnode{rectangle}{east}{\fontsize{10}{0}\selectfont\textcolor[rgb]{0,0,0}{{0}}}{}{\pgfusepath{discard}}}
{
\pgftransformshift{\pgfpoint{36.600006pt}{39.359978pt}}
\pgfnode{rectangle}{east}{\fontsize{10}{0}\selectfont\textcolor[rgb]{0,0,0}{{2e+06}}}{}{\pgfusepath{discard}}}
{
\pgftransformshift{\pgfpoint{36.600006pt}{52.319984pt}}
\pgfnode{rectangle}{east}{\fontsize{10}{0}\selectfont\textcolor[rgb]{0,0,0}{{4e+06}}}{}{\pgfusepath{discard}}}
{
\pgftransformshift{\pgfpoint{36.600006pt}{65.279991pt}}
\pgfnode{rectangle}{east}{\fontsize{10}{0}\selectfont\textcolor[rgb]{0,0,0}{{6e+06}}}{}{\pgfusepath{discard}}}
{
\pgftransformshift{\pgfpoint{36.600006pt}{78.239990pt}}
\pgfnode{rectangle}{east}{\fontsize{10}{0}\selectfont\textcolor[rgb]{0,0,0}{{8e+06}}}{}{\pgfusepath{discard}}}
{
\pgftransformshift{\pgfpoint{36.600006pt}{91.199989pt}}
\pgfnode{rectangle}{east}{\fontsize{10}{0}\selectfont\textcolor[rgb]{0,0,0}{{1e+07}}}{}{\pgfusepath{discard}}}
\pgfsetlinewidth{0.000100pt}
\pgfsetdash{}{0pt}
\pgfpathmoveto{\pgfpoint{43.591980pt}{26.399979pt}}
\pgflineto{\pgfpoint{44.587967pt}{26.399979pt}}
\pgfusepath{stroke}
\pgfpathmoveto{\pgfpoint{42.595993pt}{26.399979pt}}
\pgflineto{\pgfpoint{43.591980pt}{26.399979pt}}
\pgfusepath{stroke}
\pgfpathmoveto{\pgfpoint{43.591980pt}{33.198105pt}}
\pgflineto{\pgfpoint{42.595993pt}{26.399979pt}}
\pgfusepath{stroke}
\pgfpathmoveto{\pgfpoint{44.587967pt}{26.399979pt}}
\pgflineto{\pgfpoint{43.591980pt}{33.198105pt}}
\pgfusepath{stroke}
\pgfpathmoveto{\pgfpoint{47.575912pt}{26.399979pt}}
\pgflineto{\pgfpoint{48.571899pt}{26.399979pt}}
\pgfusepath{stroke}
\pgfpathmoveto{\pgfpoint{46.579933pt}{26.399979pt}}
\pgflineto{\pgfpoint{47.575912pt}{26.399979pt}}
\pgfusepath{stroke}
\pgfpathmoveto{\pgfpoint{47.575912pt}{27.559311pt}}
\pgflineto{\pgfpoint{46.579933pt}{26.399979pt}}
\pgfusepath{stroke}
\pgfpathmoveto{\pgfpoint{48.571899pt}{26.399979pt}}
\pgflineto{\pgfpoint{47.575912pt}{27.559311pt}}
\pgfusepath{stroke}
\pgfpathmoveto{\pgfpoint{53.551819pt}{26.399979pt}}
\pgflineto{\pgfpoint{54.547806pt}{26.399979pt}}
\pgfusepath{stroke}
\pgfpathmoveto{\pgfpoint{52.555840pt}{26.399979pt}}
\pgflineto{\pgfpoint{53.551819pt}{26.399979pt}}
\pgfusepath{stroke}
\pgfpathmoveto{\pgfpoint{53.551819pt}{32.692734pt}}
\pgflineto{\pgfpoint{52.555840pt}{26.399979pt}}
\pgfusepath{stroke}
\pgfpathmoveto{\pgfpoint{54.547806pt}{26.399979pt}}
\pgflineto{\pgfpoint{53.551819pt}{32.692734pt}}
\pgfusepath{stroke}
\pgfpathmoveto{\pgfpoint{56.539772pt}{26.399979pt}}
\pgflineto{\pgfpoint{57.535751pt}{26.399979pt}}
\pgfusepath{stroke}
\pgfpathmoveto{\pgfpoint{55.543785pt}{26.399979pt}}
\pgflineto{\pgfpoint{56.539772pt}{26.399979pt}}
\pgfusepath{stroke}
\pgfpathmoveto{\pgfpoint{54.547806pt}{26.399979pt}}
\pgflineto{\pgfpoint{55.543785pt}{26.399979pt}}
\pgfusepath{stroke}
\pgfpathmoveto{\pgfpoint{55.543785pt}{26.577354pt}}
\pgflineto{\pgfpoint{54.547806pt}{26.399979pt}}
\pgfusepath{stroke}
\pgfpathmoveto{\pgfpoint{56.539772pt}{65.803238pt}}
\pgflineto{\pgfpoint{55.543785pt}{26.577354pt}}
\pgfusepath{stroke}
\pgfpathmoveto{\pgfpoint{57.535751pt}{26.399979pt}}
\pgflineto{\pgfpoint{56.539772pt}{65.803238pt}}
\pgfusepath{stroke}
\pgfpathmoveto{\pgfpoint{59.527725pt}{26.399979pt}}
\pgflineto{\pgfpoint{60.523712pt}{26.399979pt}}
\pgfusepath{stroke}
\pgfpathmoveto{\pgfpoint{58.531738pt}{26.399979pt}}
\pgflineto{\pgfpoint{59.527725pt}{26.399979pt}}
\pgfusepath{stroke}
\pgfpathmoveto{\pgfpoint{57.535751pt}{26.399979pt}}
\pgflineto{\pgfpoint{58.531738pt}{26.399979pt}}
\pgfusepath{stroke}
\pgfpathmoveto{\pgfpoint{58.531738pt}{27.217796pt}}
\pgflineto{\pgfpoint{57.535751pt}{26.399979pt}}
\pgfusepath{stroke}
\pgfpathmoveto{\pgfpoint{59.527725pt}{26.723701pt}}
\pgflineto{\pgfpoint{58.531738pt}{27.217796pt}}
\pgfusepath{stroke}
\pgfpathmoveto{\pgfpoint{60.523712pt}{26.399979pt}}
\pgflineto{\pgfpoint{59.527725pt}{26.723701pt}}
\pgfusepath{stroke}
\pgfpathmoveto{\pgfpoint{62.515678pt}{26.399979pt}}
\pgflineto{\pgfpoint{63.511658pt}{26.399979pt}}
\pgfusepath{stroke}
\pgfpathmoveto{\pgfpoint{61.519691pt}{26.399979pt}}
\pgflineto{\pgfpoint{62.515678pt}{26.399979pt}}
\pgfusepath{stroke}
\pgfpathmoveto{\pgfpoint{62.515678pt}{26.488380pt}}
\pgflineto{\pgfpoint{61.519691pt}{26.399979pt}}
\pgfusepath{stroke}
\pgfpathmoveto{\pgfpoint{63.511658pt}{26.399979pt}}
\pgflineto{\pgfpoint{62.515678pt}{26.488380pt}}
\pgfusepath{stroke}
\pgfpathmoveto{\pgfpoint{66.499619pt}{26.399979pt}}
\pgflineto{\pgfpoint{67.495590pt}{26.399979pt}}
\pgfusepath{stroke}
\pgfpathmoveto{\pgfpoint{65.503624pt}{26.399979pt}}
\pgflineto{\pgfpoint{66.499619pt}{26.399979pt}}
\pgfusepath{stroke}
\pgfpathmoveto{\pgfpoint{64.507637pt}{26.399979pt}}
\pgflineto{\pgfpoint{65.503624pt}{26.399979pt}}
\pgfusepath{stroke}
\pgfpathmoveto{\pgfpoint{63.511658pt}{26.399979pt}}
\pgflineto{\pgfpoint{64.507637pt}{26.399979pt}}
\pgfusepath{stroke}
\pgfpathmoveto{\pgfpoint{64.507637pt}{26.602051pt}}
\pgflineto{\pgfpoint{63.511658pt}{26.399979pt}}
\pgfusepath{stroke}
\pgfpathmoveto{\pgfpoint{65.503624pt}{27.212898pt}}
\pgflineto{\pgfpoint{64.507637pt}{26.602051pt}}
\pgfusepath{stroke}
\pgfpathmoveto{\pgfpoint{66.499619pt}{29.321190pt}}
\pgflineto{\pgfpoint{65.503624pt}{27.212898pt}}
\pgfusepath{stroke}
\pgfpathmoveto{\pgfpoint{67.495590pt}{26.399979pt}}
\pgflineto{\pgfpoint{66.499619pt}{29.321190pt}}
\pgfusepath{stroke}
\pgfpathmoveto{\pgfpoint{69.487564pt}{26.399979pt}}
\pgflineto{\pgfpoint{70.483551pt}{26.399979pt}}
\pgfusepath{stroke}
\pgfpathmoveto{\pgfpoint{68.491577pt}{26.399979pt}}
\pgflineto{\pgfpoint{69.487564pt}{26.399979pt}}
\pgfusepath{stroke}
\pgfpathmoveto{\pgfpoint{67.495590pt}{26.399979pt}}
\pgflineto{\pgfpoint{68.491577pt}{26.399979pt}}
\pgfusepath{stroke}
\pgfpathmoveto{\pgfpoint{68.491577pt}{26.687355pt}}
\pgflineto{\pgfpoint{67.495590pt}{26.399979pt}}
\pgfusepath{stroke}
\pgfpathmoveto{\pgfpoint{69.487564pt}{27.948082pt}}
\pgflineto{\pgfpoint{68.491577pt}{26.687355pt}}
\pgfusepath{stroke}
\pgfpathmoveto{\pgfpoint{70.483551pt}{26.399979pt}}
\pgflineto{\pgfpoint{69.487564pt}{27.948082pt}}
\pgfusepath{stroke}
\pgfpathmoveto{\pgfpoint{72.475510pt}{26.399979pt}}
\pgflineto{\pgfpoint{73.471497pt}{26.399979pt}}
\pgfusepath{stroke}
\pgfpathmoveto{\pgfpoint{71.479530pt}{26.399979pt}}
\pgflineto{\pgfpoint{72.475510pt}{26.399979pt}}
\pgfusepath{stroke}
\pgfpathmoveto{\pgfpoint{70.483551pt}{26.399979pt}}
\pgflineto{\pgfpoint{71.479530pt}{26.399979pt}}
\pgfusepath{stroke}
\pgfpathmoveto{\pgfpoint{71.479530pt}{52.145065pt}}
\pgflineto{\pgfpoint{70.483551pt}{26.399979pt}}
\pgfusepath{stroke}
\pgfpathmoveto{\pgfpoint{72.475510pt}{31.143661pt}}
\pgflineto{\pgfpoint{71.479530pt}{52.145065pt}}
\pgfusepath{stroke}
\pgfpathmoveto{\pgfpoint{73.471497pt}{26.399979pt}}
\pgflineto{\pgfpoint{72.475510pt}{31.143661pt}}
\pgfusepath{stroke}
\pgfpathmoveto{\pgfpoint{75.463470pt}{26.399979pt}}
\pgflineto{\pgfpoint{76.459442pt}{26.399979pt}}
\pgfusepath{stroke}
\pgfpathmoveto{\pgfpoint{74.467484pt}{26.399979pt}}
\pgflineto{\pgfpoint{75.463470pt}{26.399979pt}}
\pgfusepath{stroke}
\pgfpathmoveto{\pgfpoint{73.471497pt}{26.399979pt}}
\pgflineto{\pgfpoint{74.467484pt}{26.399979pt}}
\pgfusepath{stroke}
\pgfpathmoveto{\pgfpoint{74.467484pt}{26.868034pt}}
\pgflineto{\pgfpoint{73.471497pt}{26.399979pt}}
\pgfusepath{stroke}
\pgfpathmoveto{\pgfpoint{75.463470pt}{26.470932pt}}
\pgflineto{\pgfpoint{74.467484pt}{26.868034pt}}
\pgfusepath{stroke}
\pgfpathmoveto{\pgfpoint{76.459442pt}{26.399979pt}}
\pgflineto{\pgfpoint{75.463470pt}{26.470932pt}}
\pgfusepath{stroke}
\pgfpathmoveto{\pgfpoint{82.435356pt}{26.399979pt}}
\pgflineto{\pgfpoint{83.431335pt}{26.399979pt}}
\pgfusepath{stroke}
\pgfpathmoveto{\pgfpoint{81.439369pt}{26.399979pt}}
\pgflineto{\pgfpoint{82.435356pt}{26.399979pt}}
\pgfusepath{stroke}
\pgfpathmoveto{\pgfpoint{80.443390pt}{26.399979pt}}
\pgflineto{\pgfpoint{81.439369pt}{26.399979pt}}
\pgfusepath{stroke}
\pgfpathmoveto{\pgfpoint{79.447403pt}{26.399979pt}}
\pgflineto{\pgfpoint{80.443390pt}{26.399979pt}}
\pgfusepath{stroke}
\pgfpathmoveto{\pgfpoint{80.443390pt}{28.913742pt}}
\pgflineto{\pgfpoint{79.447403pt}{26.399979pt}}
\pgfusepath{stroke}
\pgfpathmoveto{\pgfpoint{81.439369pt}{28.404305pt}}
\pgflineto{\pgfpoint{80.443390pt}{28.913742pt}}
\pgfusepath{stroke}
\pgfpathmoveto{\pgfpoint{82.435356pt}{28.310295pt}}
\pgflineto{\pgfpoint{81.439369pt}{28.404305pt}}
\pgfusepath{stroke}
\pgfpathmoveto{\pgfpoint{83.431335pt}{26.399979pt}}
\pgflineto{\pgfpoint{82.435356pt}{28.310295pt}}
\pgfusepath{stroke}
\pgfpathmoveto{\pgfpoint{86.419289pt}{26.399979pt}}
\pgflineto{\pgfpoint{87.415276pt}{26.399979pt}}
\pgfusepath{stroke}
\pgfpathmoveto{\pgfpoint{85.423309pt}{26.399979pt}}
\pgflineto{\pgfpoint{86.419289pt}{26.399979pt}}
\pgfusepath{stroke}
\pgfpathmoveto{\pgfpoint{86.419289pt}{26.871178pt}}
\pgflineto{\pgfpoint{85.423309pt}{26.399979pt}}
\pgfusepath{stroke}
\pgfpathmoveto{\pgfpoint{87.415276pt}{26.399979pt}}
\pgflineto{\pgfpoint{86.419289pt}{26.871178pt}}
\pgfusepath{stroke}
\pgfpathmoveto{\pgfpoint{88.411255pt}{26.399979pt}}
\pgflineto{\pgfpoint{89.407242pt}{26.399979pt}}
\pgfusepath{stroke}
\pgfpathmoveto{\pgfpoint{87.415276pt}{26.399979pt}}
\pgflineto{\pgfpoint{88.411255pt}{26.399979pt}}
\pgfusepath{stroke}
\pgfpathmoveto{\pgfpoint{88.411255pt}{31.944191pt}}
\pgflineto{\pgfpoint{87.415276pt}{26.399979pt}}
\pgfusepath{stroke}
\pgfpathmoveto{\pgfpoint{89.407242pt}{26.399979pt}}
\pgflineto{\pgfpoint{88.411255pt}{31.944191pt}}
\pgfusepath{stroke}
\pgfpathmoveto{\pgfpoint{90.403221pt}{26.399979pt}}
\pgflineto{\pgfpoint{91.399208pt}{26.399979pt}}
\pgfusepath{stroke}
\pgfpathmoveto{\pgfpoint{89.407242pt}{26.399979pt}}
\pgflineto{\pgfpoint{90.403221pt}{26.399979pt}}
\pgfusepath{stroke}
\pgfpathmoveto{\pgfpoint{90.403221pt}{26.454811pt}}
\pgflineto{\pgfpoint{89.407242pt}{26.399979pt}}
\pgfusepath{stroke}
\pgfpathmoveto{\pgfpoint{91.399208pt}{26.399979pt}}
\pgflineto{\pgfpoint{90.403221pt}{26.454811pt}}
\pgfusepath{stroke}
\pgfpathmoveto{\pgfpoint{93.391174pt}{26.399979pt}}
\pgflineto{\pgfpoint{94.387161pt}{26.399979pt}}
\pgfusepath{stroke}
\pgfpathmoveto{\pgfpoint{92.395187pt}{26.399979pt}}
\pgflineto{\pgfpoint{93.391174pt}{26.399979pt}}
\pgfusepath{stroke}
\pgfpathmoveto{\pgfpoint{91.399208pt}{26.399979pt}}
\pgflineto{\pgfpoint{92.395187pt}{26.399979pt}}
\pgfusepath{stroke}
\pgfpathmoveto{\pgfpoint{92.395187pt}{26.889816pt}}
\pgflineto{\pgfpoint{91.399208pt}{26.399979pt}}
\pgfusepath{stroke}
\pgfpathmoveto{\pgfpoint{93.391174pt}{27.006203pt}}
\pgflineto{\pgfpoint{92.395187pt}{26.889816pt}}
\pgfusepath{stroke}
\pgfpathmoveto{\pgfpoint{94.387161pt}{26.399979pt}}
\pgflineto{\pgfpoint{93.391174pt}{27.006203pt}}
\pgfusepath{stroke}
\pgfpathmoveto{\pgfpoint{100.363068pt}{26.399979pt}}
\pgflineto{\pgfpoint{101.359047pt}{26.399979pt}}
\pgfusepath{stroke}
\pgfpathmoveto{\pgfpoint{99.367081pt}{26.399979pt}}
\pgflineto{\pgfpoint{100.363068pt}{26.399979pt}}
\pgfusepath{stroke}
\pgfpathmoveto{\pgfpoint{98.371094pt}{26.399979pt}}
\pgflineto{\pgfpoint{99.367081pt}{26.399979pt}}
\pgfusepath{stroke}
\pgfpathmoveto{\pgfpoint{97.375107pt}{26.399979pt}}
\pgflineto{\pgfpoint{98.371094pt}{26.399979pt}}
\pgfusepath{stroke}
\pgfpathmoveto{\pgfpoint{98.371094pt}{47.782463pt}}
\pgflineto{\pgfpoint{97.375107pt}{26.399979pt}}
\pgfusepath{stroke}
\pgfpathmoveto{\pgfpoint{99.367081pt}{54.046951pt}}
\pgflineto{\pgfpoint{98.371094pt}{47.782463pt}}
\pgfusepath{stroke}
\pgfpathmoveto{\pgfpoint{100.363068pt}{32.883598pt}}
\pgflineto{\pgfpoint{99.367081pt}{54.046951pt}}
\pgfusepath{stroke}
\pgfpathmoveto{\pgfpoint{101.359047pt}{26.399979pt}}
\pgflineto{\pgfpoint{100.363068pt}{32.883598pt}}
\pgfusepath{stroke}
\pgfpathmoveto{\pgfpoint{102.355034pt}{26.399979pt}}
\pgflineto{\pgfpoint{103.351013pt}{26.399979pt}}
\pgfusepath{stroke}
\pgfpathmoveto{\pgfpoint{101.359047pt}{26.399979pt}}
\pgflineto{\pgfpoint{102.355034pt}{26.399979pt}}
\pgfusepath{stroke}
\pgfpathmoveto{\pgfpoint{102.355034pt}{26.959930pt}}
\pgflineto{\pgfpoint{101.359047pt}{26.399979pt}}
\pgfusepath{stroke}
\pgfpathmoveto{\pgfpoint{103.351013pt}{26.399979pt}}
\pgflineto{\pgfpoint{102.355034pt}{26.959930pt}}
\pgfusepath{stroke}
\pgfpathmoveto{\pgfpoint{104.347000pt}{26.399979pt}}
\pgflineto{\pgfpoint{105.342987pt}{26.399979pt}}
\pgfusepath{stroke}
\pgfpathmoveto{\pgfpoint{103.351013pt}{26.399979pt}}
\pgflineto{\pgfpoint{104.347000pt}{26.399979pt}}
\pgfusepath{stroke}
\pgfpathmoveto{\pgfpoint{104.347000pt}{35.689026pt}}
\pgflineto{\pgfpoint{103.351013pt}{26.399979pt}}
\pgfusepath{stroke}
\pgfpathmoveto{\pgfpoint{105.342987pt}{26.399979pt}}
\pgflineto{\pgfpoint{104.347000pt}{35.689026pt}}
\pgfusepath{stroke}
\pgfpathmoveto{\pgfpoint{109.326920pt}{26.399979pt}}
\pgflineto{\pgfpoint{110.322906pt}{26.399979pt}}
\pgfusepath{stroke}
\pgfpathmoveto{\pgfpoint{108.330933pt}{26.399979pt}}
\pgflineto{\pgfpoint{109.326920pt}{26.399979pt}}
\pgfusepath{stroke}
\pgfpathmoveto{\pgfpoint{107.334953pt}{26.399979pt}}
\pgflineto{\pgfpoint{108.330933pt}{26.399979pt}}
\pgfusepath{stroke}
\pgfpathmoveto{\pgfpoint{108.330933pt}{27.150108pt}}
\pgflineto{\pgfpoint{107.334953pt}{26.399979pt}}
\pgfusepath{stroke}
\pgfpathmoveto{\pgfpoint{109.326920pt}{28.632256pt}}
\pgflineto{\pgfpoint{108.330933pt}{27.150108pt}}
\pgfusepath{stroke}
\pgfpathmoveto{\pgfpoint{110.322906pt}{26.399979pt}}
\pgflineto{\pgfpoint{109.326920pt}{28.632256pt}}
\pgfusepath{stroke}
\pgfpathmoveto{\pgfpoint{113.310852pt}{26.399979pt}}
\pgflineto{\pgfpoint{114.306839pt}{26.399979pt}}
\pgfusepath{stroke}
\pgfpathmoveto{\pgfpoint{112.314873pt}{26.399979pt}}
\pgflineto{\pgfpoint{113.310852pt}{26.399979pt}}
\pgfusepath{stroke}
\pgfpathmoveto{\pgfpoint{111.318893pt}{26.399979pt}}
\pgflineto{\pgfpoint{112.314873pt}{26.399979pt}}
\pgfusepath{stroke}
\pgfpathmoveto{\pgfpoint{112.314873pt}{37.362213pt}}
\pgflineto{\pgfpoint{111.318893pt}{26.399979pt}}
\pgfusepath{stroke}
\pgfpathmoveto{\pgfpoint{113.310852pt}{26.517036pt}}
\pgflineto{\pgfpoint{112.314873pt}{37.362213pt}}
\pgfusepath{stroke}
\pgfpathmoveto{\pgfpoint{114.306839pt}{26.399979pt}}
\pgflineto{\pgfpoint{113.310852pt}{26.517036pt}}
\pgfusepath{stroke}
\pgfpathmoveto{\pgfpoint{118.290779pt}{26.399979pt}}
\pgflineto{\pgfpoint{119.286758pt}{26.399979pt}}
\pgfusepath{stroke}
\pgfpathmoveto{\pgfpoint{117.294792pt}{26.399979pt}}
\pgflineto{\pgfpoint{118.290779pt}{26.399979pt}}
\pgfusepath{stroke}
\pgfpathmoveto{\pgfpoint{116.298813pt}{26.399979pt}}
\pgflineto{\pgfpoint{117.294792pt}{26.399979pt}}
\pgfusepath{stroke}
\pgfpathmoveto{\pgfpoint{115.302826pt}{26.399979pt}}
\pgflineto{\pgfpoint{116.298813pt}{26.399979pt}}
\pgfusepath{stroke}
\pgfpathmoveto{\pgfpoint{116.298813pt}{29.134193pt}}
\pgflineto{\pgfpoint{115.302826pt}{26.399979pt}}
\pgfusepath{stroke}
\pgfpathmoveto{\pgfpoint{117.294792pt}{26.815964pt}}
\pgflineto{\pgfpoint{116.298813pt}{29.134193pt}}
\pgfusepath{stroke}
\pgfpathmoveto{\pgfpoint{117.966232pt}{91.264786pt}}
\pgflineto{\pgfpoint{117.294792pt}{26.815964pt}}
\pgfusepath{stroke}
\pgfpathmoveto{\pgfpoint{119.286758pt}{26.399979pt}}
\pgflineto{\pgfpoint{118.613922pt}{91.264793pt}}
\pgfusepath{stroke}
\pgfpathmoveto{\pgfpoint{121.278725pt}{26.399979pt}}
\pgflineto{\pgfpoint{122.274712pt}{26.399979pt}}
\pgfusepath{stroke}
\pgfpathmoveto{\pgfpoint{120.282745pt}{26.399979pt}}
\pgflineto{\pgfpoint{121.278725pt}{26.399979pt}}
\pgfusepath{stroke}
\pgfpathmoveto{\pgfpoint{121.278725pt}{27.190407pt}}
\pgflineto{\pgfpoint{120.282745pt}{26.399979pt}}
\pgfusepath{stroke}
\pgfpathmoveto{\pgfpoint{122.274712pt}{26.399979pt}}
\pgflineto{\pgfpoint{121.278725pt}{27.190407pt}}
\pgfusepath{stroke}
\pgfpathmoveto{\pgfpoint{123.270691pt}{26.399979pt}}
\pgflineto{\pgfpoint{124.266678pt}{26.399979pt}}
\pgfusepath{stroke}
\pgfpathmoveto{\pgfpoint{122.274712pt}{26.399979pt}}
\pgflineto{\pgfpoint{123.270691pt}{26.399979pt}}
\pgfusepath{stroke}
\pgfpathmoveto{\pgfpoint{123.270691pt}{26.975601pt}}
\pgflineto{\pgfpoint{122.274712pt}{26.399979pt}}
\pgfusepath{stroke}
\pgfpathmoveto{\pgfpoint{124.266678pt}{26.399979pt}}
\pgflineto{\pgfpoint{123.270691pt}{26.975601pt}}
\pgfusepath{stroke}
\pgfpathmoveto{\pgfpoint{132.234558pt}{26.399979pt}}
\pgflineto{\pgfpoint{133.230530pt}{26.399979pt}}
\pgfusepath{stroke}
\pgfpathmoveto{\pgfpoint{131.238571pt}{26.399979pt}}
\pgflineto{\pgfpoint{132.234558pt}{26.399979pt}}
\pgfusepath{stroke}
\pgfpathmoveto{\pgfpoint{130.242584pt}{26.399979pt}}
\pgflineto{\pgfpoint{131.238571pt}{26.399979pt}}
\pgfusepath{stroke}
\pgfpathmoveto{\pgfpoint{129.246597pt}{26.399979pt}}
\pgflineto{\pgfpoint{130.242584pt}{26.399979pt}}
\pgfusepath{stroke}
\pgfpathmoveto{\pgfpoint{128.250610pt}{26.399979pt}}
\pgflineto{\pgfpoint{129.246597pt}{26.399979pt}}
\pgfusepath{stroke}
\pgfpathmoveto{\pgfpoint{127.254631pt}{26.399979pt}}
\pgflineto{\pgfpoint{128.250610pt}{26.399979pt}}
\pgfusepath{stroke}
\pgfpathmoveto{\pgfpoint{126.258652pt}{26.399979pt}}
\pgflineto{\pgfpoint{127.254631pt}{26.399979pt}}
\pgfusepath{stroke}
\pgfpathmoveto{\pgfpoint{125.262665pt}{26.399979pt}}
\pgflineto{\pgfpoint{126.258652pt}{26.399979pt}}
\pgfusepath{stroke}
\pgfpathmoveto{\pgfpoint{126.258652pt}{28.660576pt}}
\pgflineto{\pgfpoint{125.262665pt}{26.399979pt}}
\pgfusepath{stroke}
\pgfpathmoveto{\pgfpoint{127.254631pt}{29.838959pt}}
\pgflineto{\pgfpoint{126.258652pt}{28.660576pt}}
\pgfusepath{stroke}
\pgfpathmoveto{\pgfpoint{128.250610pt}{26.614319pt}}
\pgflineto{\pgfpoint{127.254631pt}{29.838959pt}}
\pgfusepath{stroke}
\pgfpathmoveto{\pgfpoint{129.246597pt}{26.451225pt}}
\pgflineto{\pgfpoint{128.250610pt}{26.614319pt}}
\pgfusepath{stroke}
\pgfpathmoveto{\pgfpoint{130.242584pt}{26.430634pt}}
\pgflineto{\pgfpoint{129.246597pt}{26.451225pt}}
\pgfusepath{stroke}
\pgfpathmoveto{\pgfpoint{131.238571pt}{27.958008pt}}
\pgflineto{\pgfpoint{130.242584pt}{26.430634pt}}
\pgfusepath{stroke}
\pgfpathmoveto{\pgfpoint{132.234558pt}{27.682930pt}}
\pgflineto{\pgfpoint{131.238571pt}{27.958008pt}}
\pgfusepath{stroke}
\pgfpathmoveto{\pgfpoint{133.230530pt}{26.399979pt}}
\pgflineto{\pgfpoint{132.234558pt}{27.682930pt}}
\pgfusepath{stroke}
\pgfpathmoveto{\pgfpoint{140.202423pt}{26.399979pt}}
\pgflineto{\pgfpoint{141.198410pt}{26.399979pt}}
\pgfusepath{stroke}
\pgfpathmoveto{\pgfpoint{139.206436pt}{26.399979pt}}
\pgflineto{\pgfpoint{140.202423pt}{26.399979pt}}
\pgfusepath{stroke}
\pgfpathmoveto{\pgfpoint{138.210449pt}{26.399979pt}}
\pgflineto{\pgfpoint{139.206436pt}{26.399979pt}}
\pgfusepath{stroke}
\pgfpathmoveto{\pgfpoint{139.206436pt}{27.485779pt}}
\pgflineto{\pgfpoint{138.210449pt}{26.399979pt}}
\pgfusepath{stroke}
\pgfpathmoveto{\pgfpoint{140.202423pt}{27.782883pt}}
\pgflineto{\pgfpoint{139.206436pt}{27.485779pt}}
\pgfusepath{stroke}
\pgfpathmoveto{\pgfpoint{141.198410pt}{26.399979pt}}
\pgflineto{\pgfpoint{140.202423pt}{27.782883pt}}
\pgfusepath{stroke}
\pgfpathmoveto{\pgfpoint{143.190369pt}{26.399979pt}}
\pgflineto{\pgfpoint{144.186356pt}{26.399979pt}}
\pgfusepath{stroke}
\pgfpathmoveto{\pgfpoint{142.194382pt}{26.399979pt}}
\pgflineto{\pgfpoint{143.190369pt}{26.399979pt}}
\pgfusepath{stroke}
\pgfpathmoveto{\pgfpoint{143.190369pt}{31.207901pt}}
\pgflineto{\pgfpoint{142.194382pt}{26.399979pt}}
\pgfusepath{stroke}
\pgfpathmoveto{\pgfpoint{144.186356pt}{26.399979pt}}
\pgflineto{\pgfpoint{143.190369pt}{31.207901pt}}
\pgfusepath{stroke}
\pgfpathmoveto{\pgfpoint{146.178314pt}{26.399979pt}}
\pgflineto{\pgfpoint{147.174316pt}{26.399979pt}}
\pgfusepath{stroke}
\pgfpathmoveto{\pgfpoint{145.182343pt}{26.399979pt}}
\pgflineto{\pgfpoint{146.178314pt}{26.399979pt}}
\pgfusepath{stroke}
\pgfpathmoveto{\pgfpoint{146.178314pt}{26.535568pt}}
\pgflineto{\pgfpoint{145.182343pt}{26.399979pt}}
\pgfusepath{stroke}
\pgfpathmoveto{\pgfpoint{147.174316pt}{26.399979pt}}
\pgflineto{\pgfpoint{146.178314pt}{26.535568pt}}
\pgfusepath{stroke}
\pgfpathmoveto{\pgfpoint{152.154221pt}{26.399979pt}}
\pgflineto{\pgfpoint{153.150208pt}{26.399979pt}}
\pgfusepath{stroke}
\pgfpathmoveto{\pgfpoint{151.158249pt}{26.399979pt}}
\pgflineto{\pgfpoint{152.154221pt}{26.399979pt}}
\pgfusepath{stroke}
\pgfpathmoveto{\pgfpoint{150.162262pt}{26.399979pt}}
\pgflineto{\pgfpoint{151.158249pt}{26.399979pt}}
\pgfusepath{stroke}
\pgfpathmoveto{\pgfpoint{151.158249pt}{26.439781pt}}
\pgflineto{\pgfpoint{150.162262pt}{26.399979pt}}
\pgfusepath{stroke}
\pgfpathmoveto{\pgfpoint{152.154221pt}{26.910515pt}}
\pgflineto{\pgfpoint{151.158249pt}{26.439781pt}}
\pgfusepath{stroke}
\pgfpathmoveto{\pgfpoint{153.150208pt}{26.399979pt}}
\pgflineto{\pgfpoint{152.154221pt}{26.910515pt}}
\pgfusepath{stroke}
\pgfpathmoveto{\pgfpoint{155.142181pt}{26.399979pt}}
\pgflineto{\pgfpoint{156.138168pt}{26.399979pt}}
\pgfusepath{stroke}
\pgfpathmoveto{\pgfpoint{154.146194pt}{26.399979pt}}
\pgflineto{\pgfpoint{155.142181pt}{26.399979pt}}
\pgfusepath{stroke}
\pgfpathmoveto{\pgfpoint{155.142181pt}{28.098244pt}}
\pgflineto{\pgfpoint{154.146194pt}{26.399979pt}}
\pgfusepath{stroke}
\pgfpathmoveto{\pgfpoint{156.138168pt}{26.399979pt}}
\pgflineto{\pgfpoint{155.142181pt}{28.098244pt}}
\pgfusepath{stroke}
\pgfpathmoveto{\pgfpoint{158.130127pt}{26.399979pt}}
\pgflineto{\pgfpoint{159.126114pt}{26.399979pt}}
\pgfusepath{stroke}
\pgfpathmoveto{\pgfpoint{157.134155pt}{26.399979pt}}
\pgflineto{\pgfpoint{158.130127pt}{26.399979pt}}
\pgfusepath{stroke}
\pgfpathmoveto{\pgfpoint{156.138168pt}{26.399979pt}}
\pgflineto{\pgfpoint{157.134155pt}{26.399979pt}}
\pgfusepath{stroke}
\pgfpathmoveto{\pgfpoint{157.134155pt}{26.492256pt}}
\pgflineto{\pgfpoint{156.138168pt}{26.399979pt}}
\pgfusepath{stroke}
\pgfpathmoveto{\pgfpoint{158.130127pt}{26.480469pt}}
\pgflineto{\pgfpoint{157.134155pt}{26.492256pt}}
\pgfusepath{stroke}
\pgfpathmoveto{\pgfpoint{159.126114pt}{26.399979pt}}
\pgflineto{\pgfpoint{158.130127pt}{26.480469pt}}
\pgfusepath{stroke}
\pgfpathmoveto{\pgfpoint{163.110062pt}{26.399979pt}}
\pgflineto{\pgfpoint{164.106033pt}{26.399979pt}}
\pgfusepath{stroke}
\pgfpathmoveto{\pgfpoint{162.114075pt}{26.399979pt}}
\pgflineto{\pgfpoint{163.110062pt}{26.399979pt}}
\pgfusepath{stroke}
\pgfpathmoveto{\pgfpoint{161.118088pt}{26.399979pt}}
\pgflineto{\pgfpoint{162.114075pt}{26.399979pt}}
\pgfusepath{stroke}
\pgfpathmoveto{\pgfpoint{160.122101pt}{26.399979pt}}
\pgflineto{\pgfpoint{161.118088pt}{26.399979pt}}
\pgfusepath{stroke}
\pgfpathmoveto{\pgfpoint{161.118088pt}{26.483330pt}}
\pgflineto{\pgfpoint{160.122101pt}{26.399979pt}}
\pgfusepath{stroke}
\pgfpathmoveto{\pgfpoint{162.114075pt}{26.604111pt}}
\pgflineto{\pgfpoint{161.118088pt}{26.483330pt}}
\pgfusepath{stroke}
\pgfpathmoveto{\pgfpoint{163.110062pt}{28.794250pt}}
\pgflineto{\pgfpoint{162.114075pt}{26.604111pt}}
\pgfusepath{stroke}
\pgfpathmoveto{\pgfpoint{164.106033pt}{26.399979pt}}
\pgflineto{\pgfpoint{163.110062pt}{28.794250pt}}
\pgfusepath{stroke}
\pgfpathmoveto{\pgfpoint{165.102020pt}{26.399979pt}}
\pgflineto{\pgfpoint{166.098007pt}{26.399979pt}}
\pgfusepath{stroke}
\pgfpathmoveto{\pgfpoint{164.106033pt}{26.399979pt}}
\pgflineto{\pgfpoint{165.102020pt}{26.399979pt}}
\pgfusepath{stroke}
\pgfpathmoveto{\pgfpoint{165.102020pt}{26.737396pt}}
\pgflineto{\pgfpoint{164.106033pt}{26.399979pt}}
\pgfusepath{stroke}
\pgfpathmoveto{\pgfpoint{166.098007pt}{26.399979pt}}
\pgflineto{\pgfpoint{165.102020pt}{26.737396pt}}
\pgfusepath{stroke}
\pgfpathmoveto{\pgfpoint{168.089966pt}{26.399979pt}}
\pgflineto{\pgfpoint{169.085953pt}{26.399979pt}}
\pgfusepath{stroke}
\pgfpathmoveto{\pgfpoint{167.093994pt}{26.399979pt}}
\pgflineto{\pgfpoint{168.089966pt}{26.399979pt}}
\pgfusepath{stroke}
\pgfpathmoveto{\pgfpoint{166.098007pt}{26.399979pt}}
\pgflineto{\pgfpoint{167.093994pt}{26.399979pt}}
\pgfusepath{stroke}
\pgfpathmoveto{\pgfpoint{167.093994pt}{30.536880pt}}
\pgflineto{\pgfpoint{166.098007pt}{26.399979pt}}
\pgfusepath{stroke}
\pgfpathmoveto{\pgfpoint{168.089966pt}{26.447098pt}}
\pgflineto{\pgfpoint{167.093994pt}{30.536880pt}}
\pgfusepath{stroke}
\pgfpathmoveto{\pgfpoint{169.085953pt}{26.399979pt}}
\pgflineto{\pgfpoint{168.089966pt}{26.447098pt}}
\pgfusepath{stroke}
\pgfpathmoveto{\pgfpoint{170.081940pt}{26.399979pt}}
\pgflineto{\pgfpoint{171.077911pt}{26.399979pt}}
\pgfusepath{stroke}
\pgfpathmoveto{\pgfpoint{169.085953pt}{26.399979pt}}
\pgflineto{\pgfpoint{170.081940pt}{26.399979pt}}
\pgfusepath{stroke}
\pgfpathmoveto{\pgfpoint{170.081940pt}{26.839508pt}}
\pgflineto{\pgfpoint{169.085953pt}{26.399979pt}}
\pgfusepath{stroke}
\pgfpathmoveto{\pgfpoint{171.077911pt}{26.399979pt}}
\pgflineto{\pgfpoint{170.081940pt}{26.839508pt}}
\pgfusepath{stroke}
\pgfpathmoveto{\pgfpoint{180.041779pt}{26.399979pt}}
\pgflineto{\pgfpoint{181.037766pt}{26.399979pt}}
\pgfusepath{stroke}
\pgfpathmoveto{\pgfpoint{179.045792pt}{26.399979pt}}
\pgflineto{\pgfpoint{180.041779pt}{26.399979pt}}
\pgfusepath{stroke}
\pgfpathmoveto{\pgfpoint{178.049805pt}{26.399979pt}}
\pgflineto{\pgfpoint{179.045792pt}{26.399979pt}}
\pgfusepath{stroke}
\pgfpathmoveto{\pgfpoint{177.053818pt}{26.399979pt}}
\pgflineto{\pgfpoint{178.049805pt}{26.399979pt}}
\pgfusepath{stroke}
\pgfpathmoveto{\pgfpoint{177.581543pt}{91.264786pt}}
\pgflineto{\pgfpoint{177.053818pt}{26.399979pt}}
\pgfusepath{stroke}
\pgfpathmoveto{\pgfpoint{179.045792pt}{26.504753pt}}
\pgflineto{\pgfpoint{178.518478pt}{91.264793pt}}
\pgfusepath{stroke}
\pgfpathmoveto{\pgfpoint{180.041779pt}{30.030716pt}}
\pgflineto{\pgfpoint{179.045792pt}{26.504753pt}}
\pgfusepath{stroke}
\pgfpathmoveto{\pgfpoint{181.037766pt}{26.399979pt}}
\pgflineto{\pgfpoint{180.041779pt}{30.030716pt}}
\pgfusepath{stroke}
\pgfpathmoveto{\pgfpoint{183.029724pt}{26.399979pt}}
\pgflineto{\pgfpoint{184.025711pt}{26.399979pt}}
\pgfusepath{stroke}
\pgfpathmoveto{\pgfpoint{182.033752pt}{26.399979pt}}
\pgflineto{\pgfpoint{183.029724pt}{26.399979pt}}
\pgfusepath{stroke}
\pgfpathmoveto{\pgfpoint{183.029724pt}{26.463455pt}}
\pgflineto{\pgfpoint{182.033752pt}{26.399979pt}}
\pgfusepath{stroke}
\pgfpathmoveto{\pgfpoint{184.025711pt}{26.399979pt}}
\pgflineto{\pgfpoint{183.029724pt}{26.463455pt}}
\pgfusepath{stroke}
\pgfpathmoveto{\pgfpoint{190.001617pt}{26.399979pt}}
\pgflineto{\pgfpoint{190.997604pt}{26.399979pt}}
\pgfusepath{stroke}
\pgfpathmoveto{\pgfpoint{189.005630pt}{26.399979pt}}
\pgflineto{\pgfpoint{190.001617pt}{26.399979pt}}
\pgfusepath{stroke}
\pgfpathmoveto{\pgfpoint{190.001617pt}{34.817444pt}}
\pgflineto{\pgfpoint{189.005630pt}{26.399979pt}}
\pgfusepath{stroke}
\pgfpathmoveto{\pgfpoint{190.997604pt}{26.399979pt}}
\pgflineto{\pgfpoint{190.001617pt}{34.817444pt}}
\pgfusepath{stroke}
\pgfpathmoveto{\pgfpoint{191.993591pt}{26.399979pt}}
\pgflineto{\pgfpoint{192.989563pt}{26.399979pt}}
\pgfusepath{stroke}
\pgfpathmoveto{\pgfpoint{190.997604pt}{26.399979pt}}
\pgflineto{\pgfpoint{191.993591pt}{26.399979pt}}
\pgfusepath{stroke}
\pgfpathmoveto{\pgfpoint{191.993591pt}{31.034332pt}}
\pgflineto{\pgfpoint{190.997604pt}{26.399979pt}}
\pgfusepath{stroke}
\pgfpathmoveto{\pgfpoint{192.989563pt}{26.399979pt}}
\pgflineto{\pgfpoint{191.993591pt}{31.034332pt}}
\pgfusepath{stroke}
\pgfpathmoveto{\pgfpoint{195.977524pt}{26.399979pt}}
\pgflineto{\pgfpoint{196.973511pt}{26.399979pt}}
\pgfusepath{stroke}
\pgfpathmoveto{\pgfpoint{194.981537pt}{26.399979pt}}
\pgflineto{\pgfpoint{195.977524pt}{26.399979pt}}
\pgfusepath{stroke}
\pgfpathmoveto{\pgfpoint{193.985565pt}{26.399979pt}}
\pgflineto{\pgfpoint{194.981537pt}{26.399979pt}}
\pgfusepath{stroke}
\pgfpathmoveto{\pgfpoint{194.981537pt}{28.872696pt}}
\pgflineto{\pgfpoint{193.985565pt}{26.399979pt}}
\pgfusepath{stroke}
\pgfpathmoveto{\pgfpoint{195.977524pt}{27.054726pt}}
\pgflineto{\pgfpoint{194.981537pt}{28.872696pt}}
\pgfusepath{stroke}
\pgfpathmoveto{\pgfpoint{196.973511pt}{26.399979pt}}
\pgflineto{\pgfpoint{195.977524pt}{27.054726pt}}
\pgfusepath{stroke}
\pgfpathmoveto{\pgfpoint{197.969498pt}{26.399979pt}}
\pgflineto{\pgfpoint{198.965469pt}{26.399979pt}}
\pgfusepath{stroke}
\pgfpathmoveto{\pgfpoint{196.973511pt}{26.399979pt}}
\pgflineto{\pgfpoint{197.969498pt}{26.399979pt}}
\pgfusepath{stroke}
\pgfpathmoveto{\pgfpoint{197.969498pt}{27.127144pt}}
\pgflineto{\pgfpoint{196.973511pt}{26.399979pt}}
\pgfusepath{stroke}
\pgfpathmoveto{\pgfpoint{198.965469pt}{26.399979pt}}
\pgflineto{\pgfpoint{197.969498pt}{27.127144pt}}
\pgfusepath{stroke}
\pgfpathmoveto{\pgfpoint{203.945404pt}{26.399979pt}}
\pgflineto{\pgfpoint{204.941376pt}{26.399979pt}}
\pgfusepath{stroke}
\pgfpathmoveto{\pgfpoint{202.949402pt}{26.399979pt}}
\pgflineto{\pgfpoint{203.945404pt}{26.399979pt}}
\pgfusepath{stroke}
\pgfpathmoveto{\pgfpoint{203.945404pt}{26.673592pt}}
\pgflineto{\pgfpoint{202.949402pt}{26.399979pt}}
\pgfusepath{stroke}
\pgfpathmoveto{\pgfpoint{204.941376pt}{26.399979pt}}
\pgflineto{\pgfpoint{203.945404pt}{26.673592pt}}
\pgfusepath{stroke}
\pgfpathmoveto{\pgfpoint{206.933334pt}{26.399979pt}}
\pgflineto{\pgfpoint{207.929337pt}{26.399979pt}}
\pgfusepath{stroke}
\pgfpathmoveto{\pgfpoint{205.937347pt}{26.399979pt}}
\pgflineto{\pgfpoint{206.933334pt}{26.399979pt}}
\pgfusepath{stroke}
\pgfpathmoveto{\pgfpoint{204.941376pt}{26.399979pt}}
\pgflineto{\pgfpoint{205.937347pt}{26.399979pt}}
\pgfusepath{stroke}
\pgfpathmoveto{\pgfpoint{205.937347pt}{29.413773pt}}
\pgflineto{\pgfpoint{204.941376pt}{26.399979pt}}
\pgfusepath{stroke}
\pgfpathmoveto{\pgfpoint{206.933334pt}{28.372978pt}}
\pgflineto{\pgfpoint{205.937347pt}{29.413773pt}}
\pgfusepath{stroke}
\pgfpathmoveto{\pgfpoint{207.929337pt}{26.399979pt}}
\pgflineto{\pgfpoint{206.933334pt}{28.372978pt}}
\pgfusepath{stroke}
\pgfpathmoveto{\pgfpoint{208.925323pt}{26.399979pt}}
\pgflineto{\pgfpoint{209.921295pt}{26.399979pt}}
\pgfusepath{stroke}
\pgfpathmoveto{\pgfpoint{207.929337pt}{26.399979pt}}
\pgflineto{\pgfpoint{208.925323pt}{26.399979pt}}
\pgfusepath{stroke}
\pgfpathmoveto{\pgfpoint{208.925323pt}{29.200577pt}}
\pgflineto{\pgfpoint{207.929337pt}{26.399979pt}}
\pgfusepath{stroke}
\pgfpathmoveto{\pgfpoint{209.921295pt}{26.399979pt}}
\pgflineto{\pgfpoint{208.925323pt}{29.200577pt}}
\pgfusepath{stroke}
\pgfpathmoveto{\pgfpoint{212.909241pt}{26.399979pt}}
\pgflineto{\pgfpoint{213.905228pt}{26.399979pt}}
\pgfusepath{stroke}
\pgfpathmoveto{\pgfpoint{211.913269pt}{26.399979pt}}
\pgflineto{\pgfpoint{212.909241pt}{26.399979pt}}
\pgfusepath{stroke}
\pgfpathmoveto{\pgfpoint{210.917267pt}{26.399979pt}}
\pgflineto{\pgfpoint{211.913269pt}{26.399979pt}}
\pgfusepath{stroke}
\pgfpathmoveto{\pgfpoint{209.921295pt}{26.399979pt}}
\pgflineto{\pgfpoint{210.917267pt}{26.399979pt}}
\pgfusepath{stroke}
\pgfpathmoveto{\pgfpoint{210.917267pt}{56.628197pt}}
\pgflineto{\pgfpoint{209.921295pt}{26.399979pt}}
\pgfusepath{stroke}
\pgfpathmoveto{\pgfpoint{211.913269pt}{30.417183pt}}
\pgflineto{\pgfpoint{210.917267pt}{56.628197pt}}
\pgfusepath{stroke}
\pgfpathmoveto{\pgfpoint{212.909241pt}{47.374290pt}}
\pgflineto{\pgfpoint{211.913269pt}{30.417183pt}}
\pgfusepath{stroke}
\pgfpathmoveto{\pgfpoint{213.905228pt}{26.399979pt}}
\pgflineto{\pgfpoint{212.909241pt}{47.374290pt}}
\pgfusepath{stroke}
\pgfpathmoveto{\pgfpoint{215.897217pt}{26.399979pt}}
\pgflineto{\pgfpoint{216.893188pt}{26.399979pt}}
\pgfusepath{stroke}
\pgfpathmoveto{\pgfpoint{214.901215pt}{26.399979pt}}
\pgflineto{\pgfpoint{215.897217pt}{26.399979pt}}
\pgfusepath{stroke}
\pgfpathmoveto{\pgfpoint{213.905228pt}{26.399979pt}}
\pgflineto{\pgfpoint{214.901215pt}{26.399979pt}}
\pgfusepath{stroke}
\pgfpathmoveto{\pgfpoint{214.901215pt}{27.407372pt}}
\pgflineto{\pgfpoint{213.905228pt}{26.399979pt}}
\pgfusepath{stroke}
\pgfpathmoveto{\pgfpoint{215.897217pt}{26.495361pt}}
\pgflineto{\pgfpoint{214.901215pt}{27.407372pt}}
\pgfusepath{stroke}
\pgfpathmoveto{\pgfpoint{216.893188pt}{26.399979pt}}
\pgflineto{\pgfpoint{215.897217pt}{26.495361pt}}
\pgfusepath{stroke}
\pgfpathmoveto{\pgfpoint{217.889160pt}{26.399979pt}}
\pgflineto{\pgfpoint{218.885147pt}{26.399979pt}}
\pgfusepath{stroke}
\pgfpathmoveto{\pgfpoint{216.893188pt}{26.399979pt}}
\pgflineto{\pgfpoint{217.889160pt}{26.399979pt}}
\pgfusepath{stroke}
\pgfpathmoveto{\pgfpoint{217.889160pt}{66.077660pt}}
\pgflineto{\pgfpoint{216.893188pt}{26.399979pt}}
\pgfusepath{stroke}
\pgfpathmoveto{\pgfpoint{218.885147pt}{26.399979pt}}
\pgflineto{\pgfpoint{217.889160pt}{66.077660pt}}
\pgfusepath{stroke}
\pgfpathmoveto{\pgfpoint{225.857040pt}{26.399979pt}}
\pgflineto{\pgfpoint{226.853027pt}{26.399979pt}}
\pgfusepath{stroke}
\pgfpathmoveto{\pgfpoint{224.861053pt}{26.399979pt}}
\pgflineto{\pgfpoint{225.857040pt}{26.399979pt}}
\pgfusepath{stroke}
\pgfpathmoveto{\pgfpoint{225.857040pt}{28.125343pt}}
\pgflineto{\pgfpoint{224.861053pt}{26.399979pt}}
\pgfusepath{stroke}
\pgfpathmoveto{\pgfpoint{226.853027pt}{26.399979pt}}
\pgflineto{\pgfpoint{225.857040pt}{28.125343pt}}
\pgfusepath{stroke}
\pgfpathmoveto{\pgfpoint{230.836945pt}{26.399979pt}}
\pgflineto{\pgfpoint{231.832932pt}{26.399979pt}}
\pgfusepath{stroke}
\pgfpathmoveto{\pgfpoint{229.840973pt}{26.399979pt}}
\pgflineto{\pgfpoint{230.836945pt}{26.399979pt}}
\pgfusepath{stroke}
\pgfpathmoveto{\pgfpoint{228.845001pt}{26.399979pt}}
\pgflineto{\pgfpoint{229.840973pt}{26.399979pt}}
\pgfusepath{stroke}
\pgfpathmoveto{\pgfpoint{227.849014pt}{26.399979pt}}
\pgflineto{\pgfpoint{228.845001pt}{26.399979pt}}
\pgfusepath{stroke}
\pgfpathmoveto{\pgfpoint{228.845001pt}{27.012459pt}}
\pgflineto{\pgfpoint{227.849014pt}{26.399979pt}}
\pgfusepath{stroke}
\pgfpathmoveto{\pgfpoint{229.840973pt}{26.558662pt}}
\pgflineto{\pgfpoint{228.845001pt}{27.012459pt}}
\pgfusepath{stroke}
\pgfpathmoveto{\pgfpoint{230.836945pt}{26.682472pt}}
\pgflineto{\pgfpoint{229.840973pt}{26.558662pt}}
\pgfusepath{stroke}
\pgfpathmoveto{\pgfpoint{231.832932pt}{26.399979pt}}
\pgflineto{\pgfpoint{230.836945pt}{26.682472pt}}
\pgfusepath{stroke}
\pgfpathmoveto{\pgfpoint{237.808838pt}{26.399979pt}}
\pgflineto{\pgfpoint{238.804825pt}{26.399979pt}}
\pgfusepath{stroke}
\pgfpathmoveto{\pgfpoint{236.812866pt}{26.399979pt}}
\pgflineto{\pgfpoint{237.808838pt}{26.399979pt}}
\pgfusepath{stroke}
\pgfpathmoveto{\pgfpoint{235.816864pt}{26.399979pt}}
\pgflineto{\pgfpoint{236.812866pt}{26.399979pt}}
\pgfusepath{stroke}
\pgfpathmoveto{\pgfpoint{236.812866pt}{35.730797pt}}
\pgflineto{\pgfpoint{235.816864pt}{26.399979pt}}
\pgfusepath{stroke}
\pgfpathmoveto{\pgfpoint{237.808838pt}{26.516998pt}}
\pgflineto{\pgfpoint{236.812866pt}{35.730797pt}}
\pgfusepath{stroke}
\pgfpathmoveto{\pgfpoint{238.804825pt}{26.399979pt}}
\pgflineto{\pgfpoint{237.808838pt}{26.516998pt}}
\pgfusepath{stroke}
\pgfpathmoveto{\pgfpoint{243.784744pt}{26.399979pt}}
\pgflineto{\pgfpoint{244.780731pt}{26.399979pt}}
\pgfusepath{stroke}
\pgfpathmoveto{\pgfpoint{242.788757pt}{26.399979pt}}
\pgflineto{\pgfpoint{243.784744pt}{26.399979pt}}
\pgfusepath{stroke}
\pgfpathmoveto{\pgfpoint{241.792786pt}{26.399979pt}}
\pgflineto{\pgfpoint{242.788757pt}{26.399979pt}}
\pgfusepath{stroke}
\pgfpathmoveto{\pgfpoint{242.788757pt}{26.819893pt}}
\pgflineto{\pgfpoint{241.792786pt}{26.399979pt}}
\pgfusepath{stroke}
\pgfpathmoveto{\pgfpoint{243.784744pt}{27.221687pt}}
\pgflineto{\pgfpoint{242.788757pt}{26.819893pt}}
\pgfusepath{stroke}
\pgfpathmoveto{\pgfpoint{244.780731pt}{26.399979pt}}
\pgflineto{\pgfpoint{243.784744pt}{27.221687pt}}
\pgfusepath{stroke}
\pgfpathmoveto{\pgfpoint{247.768677pt}{26.399979pt}}
\pgflineto{\pgfpoint{248.764679pt}{26.399979pt}}
\pgfusepath{stroke}
\pgfpathmoveto{\pgfpoint{246.772705pt}{26.399979pt}}
\pgflineto{\pgfpoint{247.768677pt}{26.399979pt}}
\pgfusepath{stroke}
\pgfpathmoveto{\pgfpoint{245.776718pt}{26.399979pt}}
\pgflineto{\pgfpoint{246.772705pt}{26.399979pt}}
\pgfusepath{stroke}
\pgfpathmoveto{\pgfpoint{244.780731pt}{26.399979pt}}
\pgflineto{\pgfpoint{245.776718pt}{26.399979pt}}
\pgfusepath{stroke}
\pgfpathmoveto{\pgfpoint{245.776718pt}{26.477646pt}}
\pgflineto{\pgfpoint{244.780731pt}{26.399979pt}}
\pgfusepath{stroke}
\pgfpathmoveto{\pgfpoint{246.772705pt}{26.550812pt}}
\pgflineto{\pgfpoint{245.776718pt}{26.477646pt}}
\pgfusepath{stroke}
\pgfpathmoveto{\pgfpoint{247.768677pt}{41.200562pt}}
\pgflineto{\pgfpoint{246.772705pt}{26.550812pt}}
\pgfusepath{stroke}
\pgfpathmoveto{\pgfpoint{248.764679pt}{26.399979pt}}
\pgflineto{\pgfpoint{247.768677pt}{41.200562pt}}
\pgfusepath{stroke}
\pgfpathmoveto{\pgfpoint{250.756638pt}{26.399979pt}}
\pgflineto{\pgfpoint{251.752625pt}{26.399979pt}}
\pgfusepath{stroke}
\pgfpathmoveto{\pgfpoint{249.760651pt}{26.399979pt}}
\pgflineto{\pgfpoint{250.756638pt}{26.399979pt}}
\pgfusepath{stroke}
\pgfpathmoveto{\pgfpoint{248.764679pt}{26.399979pt}}
\pgflineto{\pgfpoint{249.760651pt}{26.399979pt}}
\pgfusepath{stroke}
\pgfpathmoveto{\pgfpoint{249.760651pt}{26.474113pt}}
\pgflineto{\pgfpoint{248.764679pt}{26.399979pt}}
\pgfusepath{stroke}
\pgfpathmoveto{\pgfpoint{250.756638pt}{26.708969pt}}
\pgflineto{\pgfpoint{249.760651pt}{26.474113pt}}
\pgfusepath{stroke}
\pgfpathmoveto{\pgfpoint{251.752625pt}{26.399979pt}}
\pgflineto{\pgfpoint{250.756638pt}{26.708969pt}}
\pgfusepath{stroke}
\pgfpathmoveto{\pgfpoint{254.740570pt}{26.399979pt}}
\pgflineto{\pgfpoint{255.736542pt}{26.399979pt}}
\pgfusepath{stroke}
\pgfpathmoveto{\pgfpoint{253.744598pt}{26.399979pt}}
\pgflineto{\pgfpoint{254.740570pt}{26.399979pt}}
\pgfusepath{stroke}
\pgfpathmoveto{\pgfpoint{254.740570pt}{27.508163pt}}
\pgflineto{\pgfpoint{253.744598pt}{26.399979pt}}
\pgfusepath{stroke}
\pgfpathmoveto{\pgfpoint{255.736542pt}{26.399979pt}}
\pgflineto{\pgfpoint{254.740570pt}{27.508163pt}}
\pgfusepath{stroke}
\pgfpathmoveto{\pgfpoint{258.724518pt}{26.399979pt}}
\pgflineto{\pgfpoint{259.720490pt}{26.399979pt}}
\pgfusepath{stroke}
\pgfpathmoveto{\pgfpoint{257.728516pt}{26.399979pt}}
\pgflineto{\pgfpoint{258.724518pt}{26.399979pt}}
\pgfusepath{stroke}
\pgfpathmoveto{\pgfpoint{258.724518pt}{26.886185pt}}
\pgflineto{\pgfpoint{257.728516pt}{26.399979pt}}
\pgfusepath{stroke}
\pgfpathmoveto{\pgfpoint{259.720490pt}{26.399979pt}}
\pgflineto{\pgfpoint{258.724518pt}{26.886185pt}}
\pgfusepath{stroke}
\pgfpathmoveto{\pgfpoint{261.712463pt}{26.399979pt}}
\pgflineto{\pgfpoint{262.708435pt}{26.399979pt}}
\pgfusepath{stroke}
\pgfpathmoveto{\pgfpoint{260.716492pt}{26.399979pt}}
\pgflineto{\pgfpoint{261.712463pt}{26.399979pt}}
\pgfusepath{stroke}
\pgfpathmoveto{\pgfpoint{261.712463pt}{27.286751pt}}
\pgflineto{\pgfpoint{260.716492pt}{26.399979pt}}
\pgfusepath{stroke}
\pgfpathmoveto{\pgfpoint{262.708435pt}{26.399979pt}}
\pgflineto{\pgfpoint{261.712463pt}{27.286751pt}}
\pgfusepath{stroke}
\pgfpathmoveto{\pgfpoint{264.700409pt}{26.399979pt}}
\pgflineto{\pgfpoint{265.696411pt}{26.399979pt}}
\pgfusepath{stroke}
\pgfpathmoveto{\pgfpoint{263.704407pt}{26.399979pt}}
\pgflineto{\pgfpoint{264.700409pt}{26.399979pt}}
\pgfusepath{stroke}
\pgfpathmoveto{\pgfpoint{264.700409pt}{26.932861pt}}
\pgflineto{\pgfpoint{263.704407pt}{26.399979pt}}
\pgfusepath{stroke}
\pgfpathmoveto{\pgfpoint{265.696411pt}{26.399979pt}}
\pgflineto{\pgfpoint{264.700409pt}{26.932861pt}}
\pgfusepath{stroke}
\pgfpathmoveto{\pgfpoint{266.692383pt}{26.399979pt}}
\pgflineto{\pgfpoint{267.688354pt}{26.399979pt}}
\pgfusepath{stroke}
\pgfpathmoveto{\pgfpoint{265.696411pt}{26.399979pt}}
\pgflineto{\pgfpoint{266.692383pt}{26.399979pt}}
\pgfusepath{stroke}
\pgfpathmoveto{\pgfpoint{266.692383pt}{26.914734pt}}
\pgflineto{\pgfpoint{265.696411pt}{26.399979pt}}
\pgfusepath{stroke}
\pgfpathmoveto{\pgfpoint{267.688354pt}{26.399979pt}}
\pgflineto{\pgfpoint{266.692383pt}{26.914734pt}}
\pgfusepath{stroke}
\pgfpathmoveto{\pgfpoint{272.668274pt}{26.399979pt}}
\pgflineto{\pgfpoint{273.664276pt}{26.399979pt}}
\pgfusepath{stroke}
\pgfpathmoveto{\pgfpoint{271.672302pt}{26.399979pt}}
\pgflineto{\pgfpoint{272.668274pt}{26.399979pt}}
\pgfusepath{stroke}
\pgfpathmoveto{\pgfpoint{270.676331pt}{26.399979pt}}
\pgflineto{\pgfpoint{271.672302pt}{26.399979pt}}
\pgfusepath{stroke}
\pgfpathmoveto{\pgfpoint{269.680328pt}{26.399979pt}}
\pgflineto{\pgfpoint{270.676331pt}{26.399979pt}}
\pgfusepath{stroke}
\pgfpathmoveto{\pgfpoint{268.684326pt}{26.399979pt}}
\pgflineto{\pgfpoint{269.680328pt}{26.399979pt}}
\pgfusepath{stroke}
\pgfpathmoveto{\pgfpoint{267.688354pt}{26.399979pt}}
\pgflineto{\pgfpoint{268.684326pt}{26.399979pt}}
\pgfusepath{stroke}
\pgfpathmoveto{\pgfpoint{268.684326pt}{26.647873pt}}
\pgflineto{\pgfpoint{267.688354pt}{26.399979pt}}
\pgfusepath{stroke}
\pgfpathmoveto{\pgfpoint{269.680328pt}{26.596115pt}}
\pgflineto{\pgfpoint{268.684326pt}{26.647873pt}}
\pgfusepath{stroke}
\pgfpathmoveto{\pgfpoint{270.676331pt}{27.976303pt}}
\pgflineto{\pgfpoint{269.680328pt}{26.596115pt}}
\pgfusepath{stroke}
\pgfpathmoveto{\pgfpoint{271.672302pt}{42.660454pt}}
\pgflineto{\pgfpoint{270.676331pt}{27.976303pt}}
\pgfusepath{stroke}
\pgfpathmoveto{\pgfpoint{272.668274pt}{26.867470pt}}
\pgflineto{\pgfpoint{271.672302pt}{42.660454pt}}
\pgfusepath{stroke}
\pgfpathmoveto{\pgfpoint{273.664276pt}{26.399979pt}}
\pgflineto{\pgfpoint{272.668274pt}{26.867470pt}}
\pgfusepath{stroke}
\pgfpathmoveto{\pgfpoint{279.640167pt}{26.399979pt}}
\pgflineto{\pgfpoint{280.636139pt}{26.399979pt}}
\pgfusepath{stroke}
\pgfpathmoveto{\pgfpoint{278.644196pt}{26.399979pt}}
\pgflineto{\pgfpoint{279.640167pt}{26.399979pt}}
\pgfusepath{stroke}
\pgfpathmoveto{\pgfpoint{277.648193pt}{26.399979pt}}
\pgflineto{\pgfpoint{278.644196pt}{26.399979pt}}
\pgfusepath{stroke}
\pgfpathmoveto{\pgfpoint{278.644196pt}{30.203644pt}}
\pgflineto{\pgfpoint{277.648193pt}{26.399979pt}}
\pgfusepath{stroke}
\pgfpathmoveto{\pgfpoint{279.640167pt}{35.907738pt}}
\pgflineto{\pgfpoint{278.644196pt}{30.203644pt}}
\pgfusepath{stroke}
\pgfpathmoveto{\pgfpoint{280.636139pt}{26.399979pt}}
\pgflineto{\pgfpoint{279.640167pt}{35.907738pt}}
\pgfusepath{stroke}
\pgfpathmoveto{\pgfpoint{283.624115pt}{26.399979pt}}
\pgflineto{\pgfpoint{284.620087pt}{26.399979pt}}
\pgfusepath{stroke}
\pgfpathmoveto{\pgfpoint{282.628113pt}{26.399979pt}}
\pgflineto{\pgfpoint{283.624115pt}{26.399979pt}}
\pgfusepath{stroke}
\pgfpathmoveto{\pgfpoint{281.632141pt}{26.399979pt}}
\pgflineto{\pgfpoint{282.628113pt}{26.399979pt}}
\pgfusepath{stroke}
\pgfpathmoveto{\pgfpoint{282.628113pt}{29.464996pt}}
\pgflineto{\pgfpoint{281.632141pt}{26.399979pt}}
\pgfusepath{stroke}
\pgfpathmoveto{\pgfpoint{283.624115pt}{28.196754pt}}
\pgflineto{\pgfpoint{282.628113pt}{29.464996pt}}
\pgfusepath{stroke}
\pgfpathmoveto{\pgfpoint{284.620087pt}{26.399979pt}}
\pgflineto{\pgfpoint{283.624115pt}{28.196754pt}}
\pgfusepath{stroke}
\pgfpathmoveto{\pgfpoint{285.616089pt}{26.399979pt}}
\pgflineto{\pgfpoint{286.612061pt}{26.399979pt}}
\pgfusepath{stroke}
\pgfpathmoveto{\pgfpoint{284.620087pt}{26.399979pt}}
\pgflineto{\pgfpoint{285.616089pt}{26.399979pt}}
\pgfusepath{stroke}
\pgfpathmoveto{\pgfpoint{285.616089pt}{26.474419pt}}
\pgflineto{\pgfpoint{284.620087pt}{26.399979pt}}
\pgfusepath{stroke}
\pgfpathmoveto{\pgfpoint{286.612061pt}{26.399979pt}}
\pgflineto{\pgfpoint{285.616089pt}{26.474419pt}}
\pgfusepath{stroke}
\color[rgb]{0.000000,0.000000,1.000000}
\pgfsetlinewidth{2.000000pt}
\pgfpathmoveto{\pgfpoint{42.595993pt}{26.399979pt}}
\pgflineto{\pgfpoint{41.600006pt}{26.399979pt}}
\pgfusepath{stroke}
\pgfpathmoveto{\pgfpoint{43.591980pt}{26.721756pt}}
\pgflineto{\pgfpoint{42.595993pt}{26.399979pt}}
\pgfusepath{stroke}
\pgfpathmoveto{\pgfpoint{44.587967pt}{26.399979pt}}
\pgflineto{\pgfpoint{43.591980pt}{26.721756pt}}
\pgfusepath{stroke}
\pgfpathmoveto{\pgfpoint{45.583946pt}{26.399979pt}}
\pgflineto{\pgfpoint{44.587967pt}{26.399979pt}}
\pgfusepath{stroke}
\pgfpathmoveto{\pgfpoint{46.579933pt}{26.399979pt}}
\pgflineto{\pgfpoint{45.583946pt}{26.399979pt}}
\pgfusepath{stroke}
\pgfpathmoveto{\pgfpoint{47.575912pt}{26.440331pt}}
\pgflineto{\pgfpoint{46.579933pt}{26.399979pt}}
\pgfusepath{stroke}
\pgfpathmoveto{\pgfpoint{48.571899pt}{26.399979pt}}
\pgflineto{\pgfpoint{47.575912pt}{26.440331pt}}
\pgfusepath{stroke}
\pgfpathmoveto{\pgfpoint{49.567879pt}{26.399979pt}}
\pgflineto{\pgfpoint{48.571899pt}{26.399979pt}}
\pgfusepath{stroke}
\pgfpathmoveto{\pgfpoint{50.563873pt}{26.399979pt}}
\pgflineto{\pgfpoint{49.567879pt}{26.399979pt}}
\pgfusepath{stroke}
\pgfpathmoveto{\pgfpoint{51.559845pt}{26.399979pt}}
\pgflineto{\pgfpoint{50.563873pt}{26.399979pt}}
\pgfusepath{stroke}
\pgfpathmoveto{\pgfpoint{52.555840pt}{26.399979pt}}
\pgflineto{\pgfpoint{51.559845pt}{26.399979pt}}
\pgfusepath{stroke}
\pgfpathmoveto{\pgfpoint{53.551819pt}{27.127853pt}}
\pgflineto{\pgfpoint{52.555840pt}{26.399979pt}}
\pgfusepath{stroke}
\pgfpathmoveto{\pgfpoint{54.547806pt}{26.399979pt}}
\pgflineto{\pgfpoint{53.551819pt}{27.127853pt}}
\pgfusepath{stroke}
\pgfpathmoveto{\pgfpoint{55.543785pt}{26.408897pt}}
\pgflineto{\pgfpoint{54.547806pt}{26.399979pt}}
\pgfusepath{stroke}
\pgfpathmoveto{\pgfpoint{56.539772pt}{28.370888pt}}
\pgflineto{\pgfpoint{55.543785pt}{26.408897pt}}
\pgfusepath{stroke}
\pgfpathmoveto{\pgfpoint{57.535751pt}{26.399979pt}}
\pgflineto{\pgfpoint{56.539772pt}{28.370888pt}}
\pgfusepath{stroke}
\pgfpathmoveto{\pgfpoint{58.531738pt}{26.430054pt}}
\pgflineto{\pgfpoint{57.535751pt}{26.399979pt}}
\pgfusepath{stroke}
\pgfpathmoveto{\pgfpoint{59.527725pt}{26.413170pt}}
\pgflineto{\pgfpoint{58.531738pt}{26.430054pt}}
\pgfusepath{stroke}
\pgfpathmoveto{\pgfpoint{60.523712pt}{26.399979pt}}
\pgflineto{\pgfpoint{59.527725pt}{26.413170pt}}
\pgfusepath{stroke}
\pgfpathmoveto{\pgfpoint{61.519691pt}{26.399979pt}}
\pgflineto{\pgfpoint{60.523712pt}{26.399979pt}}
\pgfusepath{stroke}
\pgfpathmoveto{\pgfpoint{62.515678pt}{26.401459pt}}
\pgflineto{\pgfpoint{61.519691pt}{26.399979pt}}
\pgfusepath{stroke}
\pgfpathmoveto{\pgfpoint{63.511658pt}{26.399979pt}}
\pgflineto{\pgfpoint{62.515678pt}{26.401459pt}}
\pgfusepath{stroke}
\pgfpathmoveto{\pgfpoint{64.507637pt}{26.409256pt}}
\pgflineto{\pgfpoint{63.511658pt}{26.399979pt}}
\pgfusepath{stroke}
\pgfpathmoveto{\pgfpoint{65.503624pt}{26.424644pt}}
\pgflineto{\pgfpoint{64.507637pt}{26.409256pt}}
\pgfusepath{stroke}
\pgfpathmoveto{\pgfpoint{66.499619pt}{26.614471pt}}
\pgflineto{\pgfpoint{65.503624pt}{26.424644pt}}
\pgfusepath{stroke}
\pgfpathmoveto{\pgfpoint{67.495590pt}{26.399979pt}}
\pgflineto{\pgfpoint{66.499619pt}{26.614471pt}}
\pgfusepath{stroke}
\pgfpathmoveto{\pgfpoint{68.491577pt}{26.404778pt}}
\pgflineto{\pgfpoint{67.495590pt}{26.399979pt}}
\pgfusepath{stroke}
\pgfpathmoveto{\pgfpoint{69.487564pt}{26.463905pt}}
\pgflineto{\pgfpoint{68.491577pt}{26.404778pt}}
\pgfusepath{stroke}
\pgfpathmoveto{\pgfpoint{70.483551pt}{26.399979pt}}
\pgflineto{\pgfpoint{69.487564pt}{26.463905pt}}
\pgfusepath{stroke}
\pgfpathmoveto{\pgfpoint{71.479530pt}{29.584541pt}}
\pgflineto{\pgfpoint{70.483551pt}{26.399979pt}}
\pgfusepath{stroke}
\pgfpathmoveto{\pgfpoint{72.475510pt}{26.503563pt}}
\pgflineto{\pgfpoint{71.479530pt}{29.584541pt}}
\pgfusepath{stroke}
\pgfpathmoveto{\pgfpoint{73.471497pt}{26.399979pt}}
\pgflineto{\pgfpoint{72.475510pt}{26.503563pt}}
\pgfusepath{stroke}
\pgfpathmoveto{\pgfpoint{74.467484pt}{26.431870pt}}
\pgflineto{\pgfpoint{73.471497pt}{26.399979pt}}
\pgfusepath{stroke}
\pgfpathmoveto{\pgfpoint{75.463470pt}{26.402107pt}}
\pgflineto{\pgfpoint{74.467484pt}{26.431870pt}}
\pgfusepath{stroke}
\pgfpathmoveto{\pgfpoint{76.459442pt}{26.399979pt}}
\pgflineto{\pgfpoint{75.463470pt}{26.402107pt}}
\pgfusepath{stroke}
\pgfpathmoveto{\pgfpoint{77.455437pt}{26.399979pt}}
\pgflineto{\pgfpoint{76.459442pt}{26.399979pt}}
\pgfusepath{stroke}
\pgfpathmoveto{\pgfpoint{78.451424pt}{26.399979pt}}
\pgflineto{\pgfpoint{77.455437pt}{26.399979pt}}
\pgfusepath{stroke}
\pgfpathmoveto{\pgfpoint{79.447403pt}{26.399979pt}}
\pgflineto{\pgfpoint{78.451424pt}{26.399979pt}}
\pgfusepath{stroke}
\pgfpathmoveto{\pgfpoint{80.443390pt}{26.744049pt}}
\pgflineto{\pgfpoint{79.447403pt}{26.399979pt}}
\pgfusepath{stroke}
\pgfpathmoveto{\pgfpoint{81.439369pt}{26.646423pt}}
\pgflineto{\pgfpoint{80.443390pt}{26.744049pt}}
\pgfusepath{stroke}
\pgfpathmoveto{\pgfpoint{82.435356pt}{26.442352pt}}
\pgflineto{\pgfpoint{81.439369pt}{26.646423pt}}
\pgfusepath{stroke}
\pgfpathmoveto{\pgfpoint{83.431335pt}{26.399979pt}}
\pgflineto{\pgfpoint{82.435356pt}{26.442352pt}}
\pgfusepath{stroke}
\pgfpathmoveto{\pgfpoint{84.427322pt}{26.399979pt}}
\pgflineto{\pgfpoint{83.431335pt}{26.399979pt}}
\pgfusepath{stroke}
\pgfpathmoveto{\pgfpoint{85.423309pt}{26.399979pt}}
\pgflineto{\pgfpoint{84.427322pt}{26.399979pt}}
\pgfusepath{stroke}
\pgfpathmoveto{\pgfpoint{86.419289pt}{26.407837pt}}
\pgflineto{\pgfpoint{85.423309pt}{26.399979pt}}
\pgfusepath{stroke}
\pgfpathmoveto{\pgfpoint{87.415276pt}{26.399979pt}}
\pgflineto{\pgfpoint{86.419289pt}{26.407837pt}}
\pgfusepath{stroke}
\pgfpathmoveto{\pgfpoint{88.411255pt}{26.617722pt}}
\pgflineto{\pgfpoint{87.415276pt}{26.399979pt}}
\pgfusepath{stroke}
\pgfpathmoveto{\pgfpoint{89.407242pt}{26.399979pt}}
\pgflineto{\pgfpoint{88.411255pt}{26.617722pt}}
\pgfusepath{stroke}
\pgfpathmoveto{\pgfpoint{90.403221pt}{26.400894pt}}
\pgflineto{\pgfpoint{89.407242pt}{26.399979pt}}
\pgfusepath{stroke}
\pgfpathmoveto{\pgfpoint{91.399208pt}{26.399979pt}}
\pgflineto{\pgfpoint{90.403221pt}{26.400894pt}}
\pgfusepath{stroke}
\pgfpathmoveto{\pgfpoint{92.395187pt}{26.413033pt}}
\pgflineto{\pgfpoint{91.399208pt}{26.399979pt}}
\pgfusepath{stroke}
\pgfpathmoveto{\pgfpoint{93.391174pt}{26.413887pt}}
\pgflineto{\pgfpoint{92.395187pt}{26.413033pt}}
\pgfusepath{stroke}
\pgfpathmoveto{\pgfpoint{94.387161pt}{26.399979pt}}
\pgflineto{\pgfpoint{93.391174pt}{26.413887pt}}
\pgfusepath{stroke}
\pgfpathmoveto{\pgfpoint{95.383141pt}{26.399979pt}}
\pgflineto{\pgfpoint{94.387161pt}{26.399979pt}}
\pgfusepath{stroke}
\pgfpathmoveto{\pgfpoint{96.379128pt}{26.399979pt}}
\pgflineto{\pgfpoint{95.383141pt}{26.399979pt}}
\pgfusepath{stroke}
\pgfpathmoveto{\pgfpoint{97.375107pt}{26.399979pt}}
\pgflineto{\pgfpoint{96.379128pt}{26.399979pt}}
\pgfusepath{stroke}
\pgfpathmoveto{\pgfpoint{98.371094pt}{27.362320pt}}
\pgflineto{\pgfpoint{97.375107pt}{26.399979pt}}
\pgfusepath{stroke}
\pgfpathmoveto{\pgfpoint{99.367081pt}{30.703705pt}}
\pgflineto{\pgfpoint{98.371094pt}{27.362320pt}}
\pgfusepath{stroke}
\pgfpathmoveto{\pgfpoint{100.363068pt}{26.755974pt}}
\pgflineto{\pgfpoint{99.367081pt}{30.703705pt}}
\pgfusepath{stroke}
\pgfpathmoveto{\pgfpoint{101.359047pt}{26.399979pt}}
\pgflineto{\pgfpoint{100.363068pt}{26.755974pt}}
\pgfusepath{stroke}
\pgfpathmoveto{\pgfpoint{102.355034pt}{26.436134pt}}
\pgflineto{\pgfpoint{101.359047pt}{26.399979pt}}
\pgfusepath{stroke}
\pgfpathmoveto{\pgfpoint{103.351013pt}{26.399979pt}}
\pgflineto{\pgfpoint{102.355034pt}{26.436134pt}}
\pgfusepath{stroke}
\pgfpathmoveto{\pgfpoint{104.347000pt}{27.051666pt}}
\pgflineto{\pgfpoint{103.351013pt}{26.399979pt}}
\pgfusepath{stroke}
\pgfpathmoveto{\pgfpoint{105.342987pt}{26.399979pt}}
\pgflineto{\pgfpoint{104.347000pt}{27.051666pt}}
\pgfusepath{stroke}
\pgfpathmoveto{\pgfpoint{106.338966pt}{26.399979pt}}
\pgflineto{\pgfpoint{105.342987pt}{26.399979pt}}
\pgfusepath{stroke}
\pgfpathmoveto{\pgfpoint{107.334953pt}{26.399979pt}}
\pgflineto{\pgfpoint{106.338966pt}{26.399979pt}}
\pgfusepath{stroke}
\pgfpathmoveto{\pgfpoint{108.330933pt}{26.421722pt}}
\pgflineto{\pgfpoint{107.334953pt}{26.399979pt}}
\pgfusepath{stroke}
\pgfpathmoveto{\pgfpoint{109.326920pt}{26.454926pt}}
\pgflineto{\pgfpoint{108.330933pt}{26.421722pt}}
\pgfusepath{stroke}
\pgfpathmoveto{\pgfpoint{110.322906pt}{26.399979pt}}
\pgflineto{\pgfpoint{109.326920pt}{26.454926pt}}
\pgfusepath{stroke}
\pgfpathmoveto{\pgfpoint{111.318893pt}{26.399979pt}}
\pgflineto{\pgfpoint{110.322906pt}{26.399979pt}}
\pgfusepath{stroke}
\pgfpathmoveto{\pgfpoint{112.314873pt}{27.245140pt}}
\pgflineto{\pgfpoint{111.318893pt}{26.399979pt}}
\pgfusepath{stroke}
\pgfpathmoveto{\pgfpoint{113.310852pt}{26.404144pt}}
\pgflineto{\pgfpoint{112.314873pt}{27.245140pt}}
\pgfusepath{stroke}
\pgfpathmoveto{\pgfpoint{114.306839pt}{26.399979pt}}
\pgflineto{\pgfpoint{113.310852pt}{26.404144pt}}
\pgfusepath{stroke}
\pgfpathmoveto{\pgfpoint{115.302826pt}{26.399979pt}}
\pgflineto{\pgfpoint{114.306839pt}{26.399979pt}}
\pgfusepath{stroke}
\pgfpathmoveto{\pgfpoint{116.298813pt}{26.625999pt}}
\pgflineto{\pgfpoint{115.302826pt}{26.399979pt}}
\pgfusepath{stroke}
\pgfpathmoveto{\pgfpoint{117.294792pt}{26.408501pt}}
\pgflineto{\pgfpoint{116.298813pt}{26.625999pt}}
\pgfusepath{stroke}
\pgfpathmoveto{\pgfpoint{118.290779pt}{41.526093pt}}
\pgflineto{\pgfpoint{117.294792pt}{26.408501pt}}
\pgfusepath{stroke}
\pgfpathmoveto{\pgfpoint{119.286758pt}{26.399979pt}}
\pgflineto{\pgfpoint{118.290779pt}{41.526093pt}}
\pgfusepath{stroke}
\pgfpathmoveto{\pgfpoint{120.282745pt}{26.399979pt}}
\pgflineto{\pgfpoint{119.286758pt}{26.399979pt}}
\pgfusepath{stroke}
\pgfpathmoveto{\pgfpoint{121.278725pt}{26.441711pt}}
\pgflineto{\pgfpoint{120.282745pt}{26.399979pt}}
\pgfusepath{stroke}
\pgfpathmoveto{\pgfpoint{122.274712pt}{26.399979pt}}
\pgflineto{\pgfpoint{121.278725pt}{26.441711pt}}
\pgfusepath{stroke}
\pgfpathmoveto{\pgfpoint{123.270691pt}{26.409576pt}}
\pgflineto{\pgfpoint{122.274712pt}{26.399979pt}}
\pgfusepath{stroke}
\pgfpathmoveto{\pgfpoint{124.266678pt}{26.399979pt}}
\pgflineto{\pgfpoint{123.270691pt}{26.409576pt}}
\pgfusepath{stroke}
\pgfpathmoveto{\pgfpoint{125.262665pt}{26.399979pt}}
\pgflineto{\pgfpoint{124.266678pt}{26.399979pt}}
\pgfusepath{stroke}
\pgfpathmoveto{\pgfpoint{126.258652pt}{26.571793pt}}
\pgflineto{\pgfpoint{125.262665pt}{26.399979pt}}
\pgfusepath{stroke}
\pgfpathmoveto{\pgfpoint{127.254631pt}{26.565819pt}}
\pgflineto{\pgfpoint{126.258652pt}{26.571793pt}}
\pgfusepath{stroke}
\pgfpathmoveto{\pgfpoint{128.250610pt}{26.405869pt}}
\pgflineto{\pgfpoint{127.254631pt}{26.565819pt}}
\pgfusepath{stroke}
\pgfpathmoveto{\pgfpoint{129.246597pt}{26.400833pt}}
\pgflineto{\pgfpoint{128.250610pt}{26.405869pt}}
\pgfusepath{stroke}
\pgfpathmoveto{\pgfpoint{130.242584pt}{26.400490pt}}
\pgflineto{\pgfpoint{129.246597pt}{26.400833pt}}
\pgfusepath{stroke}
\pgfpathmoveto{\pgfpoint{131.238571pt}{26.425949pt}}
\pgflineto{\pgfpoint{130.242584pt}{26.400490pt}}
\pgfusepath{stroke}
\pgfpathmoveto{\pgfpoint{132.234558pt}{26.445862pt}}
\pgflineto{\pgfpoint{131.238571pt}{26.425949pt}}
\pgfusepath{stroke}
\pgfpathmoveto{\pgfpoint{133.230530pt}{26.399979pt}}
\pgflineto{\pgfpoint{132.234558pt}{26.445862pt}}
\pgfusepath{stroke}
\pgfpathmoveto{\pgfpoint{134.226517pt}{26.399979pt}}
\pgflineto{\pgfpoint{133.230530pt}{26.399979pt}}
\pgfusepath{stroke}
\pgfpathmoveto{\pgfpoint{135.222504pt}{26.399979pt}}
\pgflineto{\pgfpoint{134.226517pt}{26.399979pt}}
\pgfusepath{stroke}
\pgfpathmoveto{\pgfpoint{136.218475pt}{26.399979pt}}
\pgflineto{\pgfpoint{135.222504pt}{26.399979pt}}
\pgfusepath{stroke}
\pgfpathmoveto{\pgfpoint{137.214478pt}{26.399979pt}}
\pgflineto{\pgfpoint{136.218475pt}{26.399979pt}}
\pgfusepath{stroke}
\pgfpathmoveto{\pgfpoint{138.210449pt}{26.399979pt}}
\pgflineto{\pgfpoint{137.214478pt}{26.399979pt}}
\pgfusepath{stroke}
\pgfpathmoveto{\pgfpoint{139.206436pt}{26.449463pt}}
\pgflineto{\pgfpoint{138.210449pt}{26.399979pt}}
\pgfusepath{stroke}
\pgfpathmoveto{\pgfpoint{140.202423pt}{26.439690pt}}
\pgflineto{\pgfpoint{139.206436pt}{26.449463pt}}
\pgfusepath{stroke}
\pgfpathmoveto{\pgfpoint{141.198410pt}{26.399979pt}}
\pgflineto{\pgfpoint{140.202423pt}{26.439690pt}}
\pgfusepath{stroke}
\pgfpathmoveto{\pgfpoint{142.194382pt}{26.399979pt}}
\pgflineto{\pgfpoint{141.198410pt}{26.399979pt}}
\pgfusepath{stroke}
\pgfpathmoveto{\pgfpoint{143.190369pt}{26.747536pt}}
\pgflineto{\pgfpoint{142.194382pt}{26.399979pt}}
\pgfusepath{stroke}
\pgfpathmoveto{\pgfpoint{144.186356pt}{26.399979pt}}
\pgflineto{\pgfpoint{143.190369pt}{26.747536pt}}
\pgfusepath{stroke}
\pgfpathmoveto{\pgfpoint{145.182343pt}{26.399979pt}}
\pgflineto{\pgfpoint{144.186356pt}{26.399979pt}}
\pgfusepath{stroke}
\pgfpathmoveto{\pgfpoint{146.178314pt}{26.402245pt}}
\pgflineto{\pgfpoint{145.182343pt}{26.399979pt}}
\pgfusepath{stroke}
\pgfpathmoveto{\pgfpoint{147.174316pt}{26.399979pt}}
\pgflineto{\pgfpoint{146.178314pt}{26.402245pt}}
\pgfusepath{stroke}
\pgfpathmoveto{\pgfpoint{148.170288pt}{26.399979pt}}
\pgflineto{\pgfpoint{147.174316pt}{26.399979pt}}
\pgfusepath{stroke}
\pgfpathmoveto{\pgfpoint{149.166275pt}{26.399979pt}}
\pgflineto{\pgfpoint{148.170288pt}{26.399979pt}}
\pgfusepath{stroke}
\pgfpathmoveto{\pgfpoint{150.162262pt}{26.399979pt}}
\pgflineto{\pgfpoint{149.166275pt}{26.399979pt}}
\pgfusepath{stroke}
\pgfpathmoveto{\pgfpoint{151.158249pt}{26.400650pt}}
\pgflineto{\pgfpoint{150.162262pt}{26.399979pt}}
\pgfusepath{stroke}
\pgfpathmoveto{\pgfpoint{152.154221pt}{26.413498pt}}
\pgflineto{\pgfpoint{151.158249pt}{26.400650pt}}
\pgfusepath{stroke}
\pgfpathmoveto{\pgfpoint{153.150208pt}{26.399979pt}}
\pgflineto{\pgfpoint{152.154221pt}{26.413498pt}}
\pgfusepath{stroke}
\pgfpathmoveto{\pgfpoint{154.146194pt}{26.399979pt}}
\pgflineto{\pgfpoint{153.150208pt}{26.399979pt}}
\pgfusepath{stroke}
\pgfpathmoveto{\pgfpoint{155.142181pt}{26.428284pt}}
\pgflineto{\pgfpoint{154.146194pt}{26.399979pt}}
\pgfusepath{stroke}
\pgfpathmoveto{\pgfpoint{156.138168pt}{26.399979pt}}
\pgflineto{\pgfpoint{155.142181pt}{26.428284pt}}
\pgfusepath{stroke}
\pgfpathmoveto{\pgfpoint{157.134155pt}{26.402534pt}}
\pgflineto{\pgfpoint{156.138168pt}{26.399979pt}}
\pgfusepath{stroke}
\pgfpathmoveto{\pgfpoint{158.130127pt}{26.404060pt}}
\pgflineto{\pgfpoint{157.134155pt}{26.402534pt}}
\pgfusepath{stroke}
\pgfpathmoveto{\pgfpoint{159.126114pt}{26.399979pt}}
\pgflineto{\pgfpoint{158.130127pt}{26.404060pt}}
\pgfusepath{stroke}
\pgfpathmoveto{\pgfpoint{160.122101pt}{26.399979pt}}
\pgflineto{\pgfpoint{159.126114pt}{26.399979pt}}
\pgfusepath{stroke}
\pgfpathmoveto{\pgfpoint{161.118088pt}{26.401367pt}}
\pgflineto{\pgfpoint{160.122101pt}{26.399979pt}}
\pgfusepath{stroke}
\pgfpathmoveto{\pgfpoint{162.114075pt}{26.404663pt}}
\pgflineto{\pgfpoint{161.118088pt}{26.401367pt}}
\pgfusepath{stroke}
\pgfpathmoveto{\pgfpoint{163.110062pt}{26.439888pt}}
\pgflineto{\pgfpoint{162.114075pt}{26.404663pt}}
\pgfusepath{stroke}
\pgfpathmoveto{\pgfpoint{164.106033pt}{26.399979pt}}
\pgflineto{\pgfpoint{163.110062pt}{26.439888pt}}
\pgfusepath{stroke}
\pgfpathmoveto{\pgfpoint{165.102020pt}{26.405602pt}}
\pgflineto{\pgfpoint{164.106033pt}{26.399979pt}}
\pgfusepath{stroke}
\pgfpathmoveto{\pgfpoint{166.098007pt}{26.399979pt}}
\pgflineto{\pgfpoint{165.102020pt}{26.405602pt}}
\pgfusepath{stroke}
\pgfpathmoveto{\pgfpoint{167.093994pt}{26.491440pt}}
\pgflineto{\pgfpoint{166.098007pt}{26.399979pt}}
\pgfusepath{stroke}
\pgfpathmoveto{\pgfpoint{168.089966pt}{26.401459pt}}
\pgflineto{\pgfpoint{167.093994pt}{26.491440pt}}
\pgfusepath{stroke}
\pgfpathmoveto{\pgfpoint{169.085953pt}{26.399979pt}}
\pgflineto{\pgfpoint{168.089966pt}{26.401459pt}}
\pgfusepath{stroke}
\pgfpathmoveto{\pgfpoint{170.081940pt}{26.417023pt}}
\pgflineto{\pgfpoint{169.085953pt}{26.399979pt}}
\pgfusepath{stroke}
\pgfpathmoveto{\pgfpoint{171.077911pt}{26.399979pt}}
\pgflineto{\pgfpoint{170.081940pt}{26.417023pt}}
\pgfusepath{stroke}
\pgfpathmoveto{\pgfpoint{172.073914pt}{26.399979pt}}
\pgflineto{\pgfpoint{171.077911pt}{26.399979pt}}
\pgfusepath{stroke}
\pgfpathmoveto{\pgfpoint{173.069885pt}{26.399979pt}}
\pgflineto{\pgfpoint{172.073914pt}{26.399979pt}}
\pgfusepath{stroke}
\pgfpathmoveto{\pgfpoint{174.065872pt}{26.399979pt}}
\pgflineto{\pgfpoint{173.069885pt}{26.399979pt}}
\pgfusepath{stroke}
\pgfpathmoveto{\pgfpoint{175.061859pt}{26.399979pt}}
\pgflineto{\pgfpoint{174.065872pt}{26.399979pt}}
\pgfusepath{stroke}
\pgfpathmoveto{\pgfpoint{176.057846pt}{26.399979pt}}
\pgflineto{\pgfpoint{175.061859pt}{26.399979pt}}
\pgfusepath{stroke}
\pgfpathmoveto{\pgfpoint{177.053818pt}{26.399979pt}}
\pgflineto{\pgfpoint{176.057846pt}{26.399979pt}}
\pgfusepath{stroke}
\pgfpathmoveto{\pgfpoint{178.049805pt}{38.817703pt}}
\pgflineto{\pgfpoint{177.053818pt}{26.399979pt}}
\pgfusepath{stroke}
\pgfpathmoveto{\pgfpoint{179.045792pt}{26.402115pt}}
\pgflineto{\pgfpoint{178.049805pt}{38.817703pt}}
\pgfusepath{stroke}
\pgfpathmoveto{\pgfpoint{180.041779pt}{26.478966pt}}
\pgflineto{\pgfpoint{179.045792pt}{26.402115pt}}
\pgfusepath{stroke}
\pgfpathmoveto{\pgfpoint{181.037766pt}{26.399979pt}}
\pgflineto{\pgfpoint{180.041779pt}{26.478966pt}}
\pgfusepath{stroke}
\pgfpathmoveto{\pgfpoint{182.033752pt}{26.399979pt}}
\pgflineto{\pgfpoint{181.037766pt}{26.399979pt}}
\pgfusepath{stroke}
\pgfpathmoveto{\pgfpoint{183.029724pt}{26.401039pt}}
\pgflineto{\pgfpoint{182.033752pt}{26.399979pt}}
\pgfusepath{stroke}
\pgfpathmoveto{\pgfpoint{184.025711pt}{26.399979pt}}
\pgflineto{\pgfpoint{183.029724pt}{26.401039pt}}
\pgfusepath{stroke}
\pgfpathmoveto{\pgfpoint{185.021698pt}{26.399979pt}}
\pgflineto{\pgfpoint{184.025711pt}{26.399979pt}}
\pgfusepath{stroke}
\pgfpathmoveto{\pgfpoint{186.017685pt}{26.399979pt}}
\pgflineto{\pgfpoint{185.021698pt}{26.399979pt}}
\pgfusepath{stroke}
\pgfpathmoveto{\pgfpoint{187.013672pt}{26.399979pt}}
\pgflineto{\pgfpoint{186.017685pt}{26.399979pt}}
\pgfusepath{stroke}
\pgfpathmoveto{\pgfpoint{188.009659pt}{26.399979pt}}
\pgflineto{\pgfpoint{187.013672pt}{26.399979pt}}
\pgfusepath{stroke}
\pgfpathmoveto{\pgfpoint{189.005630pt}{26.399979pt}}
\pgflineto{\pgfpoint{188.009659pt}{26.399979pt}}
\pgfusepath{stroke}
\pgfpathmoveto{\pgfpoint{190.001617pt}{27.040497pt}}
\pgflineto{\pgfpoint{189.005630pt}{26.399979pt}}
\pgfusepath{stroke}
\pgfpathmoveto{\pgfpoint{190.997604pt}{26.399979pt}}
\pgflineto{\pgfpoint{190.001617pt}{27.040497pt}}
\pgfusepath{stroke}
\pgfpathmoveto{\pgfpoint{191.993591pt}{26.562309pt}}
\pgflineto{\pgfpoint{190.997604pt}{26.399979pt}}
\pgfusepath{stroke}
\pgfpathmoveto{\pgfpoint{192.989563pt}{26.399979pt}}
\pgflineto{\pgfpoint{191.993591pt}{26.562309pt}}
\pgfusepath{stroke}
\pgfpathmoveto{\pgfpoint{193.985565pt}{26.399979pt}}
\pgflineto{\pgfpoint{192.989563pt}{26.399979pt}}
\pgfusepath{stroke}
\pgfpathmoveto{\pgfpoint{194.981537pt}{26.700401pt}}
\pgflineto{\pgfpoint{193.985565pt}{26.399979pt}}
\pgfusepath{stroke}
\pgfpathmoveto{\pgfpoint{195.977524pt}{26.438347pt}}
\pgflineto{\pgfpoint{194.981537pt}{26.700401pt}}
\pgfusepath{stroke}
\pgfpathmoveto{\pgfpoint{196.973511pt}{26.399979pt}}
\pgflineto{\pgfpoint{195.977524pt}{26.438347pt}}
\pgfusepath{stroke}
\pgfpathmoveto{\pgfpoint{197.969498pt}{26.425056pt}}
\pgflineto{\pgfpoint{196.973511pt}{26.399979pt}}
\pgfusepath{stroke}
\pgfpathmoveto{\pgfpoint{198.965469pt}{26.399979pt}}
\pgflineto{\pgfpoint{197.969498pt}{26.425056pt}}
\pgfusepath{stroke}
\pgfpathmoveto{\pgfpoint{199.961456pt}{26.399979pt}}
\pgflineto{\pgfpoint{198.965469pt}{26.399979pt}}
\pgfusepath{stroke}
\pgfpathmoveto{\pgfpoint{200.957443pt}{26.399979pt}}
\pgflineto{\pgfpoint{199.961456pt}{26.399979pt}}
\pgfusepath{stroke}
\pgfpathmoveto{\pgfpoint{201.953430pt}{26.399979pt}}
\pgflineto{\pgfpoint{200.957443pt}{26.399979pt}}
\pgfusepath{stroke}
\pgfpathmoveto{\pgfpoint{202.949402pt}{26.399979pt}}
\pgflineto{\pgfpoint{201.953430pt}{26.399979pt}}
\pgfusepath{stroke}
\pgfpathmoveto{\pgfpoint{203.945404pt}{26.404549pt}}
\pgflineto{\pgfpoint{202.949402pt}{26.399979pt}}
\pgfusepath{stroke}
\pgfpathmoveto{\pgfpoint{204.941376pt}{26.399979pt}}
\pgflineto{\pgfpoint{203.945404pt}{26.404549pt}}
\pgfusepath{stroke}
\pgfpathmoveto{\pgfpoint{205.937347pt}{26.498993pt}}
\pgflineto{\pgfpoint{204.941376pt}{26.399979pt}}
\pgfusepath{stroke}
\pgfpathmoveto{\pgfpoint{206.933334pt}{26.624886pt}}
\pgflineto{\pgfpoint{205.937347pt}{26.498993pt}}
\pgfusepath{stroke}
\pgfpathmoveto{\pgfpoint{207.929337pt}{26.399979pt}}
\pgflineto{\pgfpoint{206.933334pt}{26.624886pt}}
\pgfusepath{stroke}
\pgfpathmoveto{\pgfpoint{208.925323pt}{26.499222pt}}
\pgflineto{\pgfpoint{207.929337pt}{26.399979pt}}
\pgfusepath{stroke}
\pgfpathmoveto{\pgfpoint{209.921295pt}{26.399979pt}}
\pgflineto{\pgfpoint{208.925323pt}{26.499222pt}}
\pgfusepath{stroke}
\pgfpathmoveto{\pgfpoint{210.917267pt}{28.758682pt}}
\pgflineto{\pgfpoint{209.921295pt}{26.399979pt}}
\pgfusepath{stroke}
\pgfpathmoveto{\pgfpoint{211.913269pt}{26.580238pt}}
\pgflineto{\pgfpoint{210.917267pt}{28.758682pt}}
\pgfusepath{stroke}
\pgfpathmoveto{\pgfpoint{212.909241pt}{27.484299pt}}
\pgflineto{\pgfpoint{211.913269pt}{26.580238pt}}
\pgfusepath{stroke}
\pgfpathmoveto{\pgfpoint{213.905228pt}{26.399979pt}}
\pgflineto{\pgfpoint{212.909241pt}{27.484299pt}}
\pgfusepath{stroke}
\pgfpathmoveto{\pgfpoint{214.901215pt}{26.416771pt}}
\pgflineto{\pgfpoint{213.905228pt}{26.399979pt}}
\pgfusepath{stroke}
\pgfpathmoveto{\pgfpoint{215.897217pt}{26.401573pt}}
\pgflineto{\pgfpoint{214.901215pt}{26.416771pt}}
\pgfusepath{stroke}
\pgfpathmoveto{\pgfpoint{216.893188pt}{26.399979pt}}
\pgflineto{\pgfpoint{215.897217pt}{26.401573pt}}
\pgfusepath{stroke}
\pgfpathmoveto{\pgfpoint{217.889160pt}{30.564590pt}}
\pgflineto{\pgfpoint{216.893188pt}{26.399979pt}}
\pgfusepath{stroke}
\pgfpathmoveto{\pgfpoint{218.885147pt}{26.399979pt}}
\pgflineto{\pgfpoint{217.889160pt}{30.564590pt}}
\pgfusepath{stroke}
\pgfpathmoveto{\pgfpoint{219.881134pt}{26.399979pt}}
\pgflineto{\pgfpoint{218.885147pt}{26.399979pt}}
\pgfusepath{stroke}
\pgfpathmoveto{\pgfpoint{220.877121pt}{26.399979pt}}
\pgflineto{\pgfpoint{219.881134pt}{26.399979pt}}
\pgfusepath{stroke}
\pgfpathmoveto{\pgfpoint{221.873108pt}{26.399979pt}}
\pgflineto{\pgfpoint{220.877121pt}{26.399979pt}}
\pgfusepath{stroke}
\pgfpathmoveto{\pgfpoint{222.869080pt}{26.399979pt}}
\pgflineto{\pgfpoint{221.873108pt}{26.399979pt}}
\pgfusepath{stroke}
\pgfpathmoveto{\pgfpoint{223.865082pt}{26.399979pt}}
\pgflineto{\pgfpoint{222.869080pt}{26.399979pt}}
\pgfusepath{stroke}
\pgfpathmoveto{\pgfpoint{224.861053pt}{26.399979pt}}
\pgflineto{\pgfpoint{223.865082pt}{26.399979pt}}
\pgfusepath{stroke}
\pgfpathmoveto{\pgfpoint{225.857040pt}{26.482925pt}}
\pgflineto{\pgfpoint{224.861053pt}{26.399979pt}}
\pgfusepath{stroke}
\pgfpathmoveto{\pgfpoint{226.853027pt}{26.399979pt}}
\pgflineto{\pgfpoint{225.857040pt}{26.482925pt}}
\pgfusepath{stroke}
\pgfpathmoveto{\pgfpoint{227.849014pt}{26.399979pt}}
\pgflineto{\pgfpoint{226.853027pt}{26.399979pt}}
\pgfusepath{stroke}
\pgfpathmoveto{\pgfpoint{228.845001pt}{26.419968pt}}
\pgflineto{\pgfpoint{227.849014pt}{26.399979pt}}
\pgfusepath{stroke}
\pgfpathmoveto{\pgfpoint{229.840973pt}{26.402626pt}}
\pgflineto{\pgfpoint{228.845001pt}{26.419968pt}}
\pgfusepath{stroke}
\pgfpathmoveto{\pgfpoint{230.836945pt}{26.404686pt}}
\pgflineto{\pgfpoint{229.840973pt}{26.402626pt}}
\pgfusepath{stroke}
\pgfpathmoveto{\pgfpoint{231.832932pt}{26.399979pt}}
\pgflineto{\pgfpoint{230.836945pt}{26.404686pt}}
\pgfusepath{stroke}
\pgfpathmoveto{\pgfpoint{232.828934pt}{26.399979pt}}
\pgflineto{\pgfpoint{231.832932pt}{26.399979pt}}
\pgfusepath{stroke}
\pgfpathmoveto{\pgfpoint{233.824921pt}{26.399979pt}}
\pgflineto{\pgfpoint{232.828934pt}{26.399979pt}}
\pgfusepath{stroke}
\pgfpathmoveto{\pgfpoint{234.820892pt}{26.399979pt}}
\pgflineto{\pgfpoint{233.824921pt}{26.399979pt}}
\pgfusepath{stroke}
\pgfpathmoveto{\pgfpoint{235.816864pt}{26.399979pt}}
\pgflineto{\pgfpoint{234.820892pt}{26.399979pt}}
\pgfusepath{stroke}
\pgfpathmoveto{\pgfpoint{236.812866pt}{27.336357pt}}
\pgflineto{\pgfpoint{235.816864pt}{26.399979pt}}
\pgfusepath{stroke}
\pgfpathmoveto{\pgfpoint{237.808838pt}{26.403709pt}}
\pgflineto{\pgfpoint{236.812866pt}{27.336357pt}}
\pgfusepath{stroke}
\pgfpathmoveto{\pgfpoint{238.804825pt}{26.399979pt}}
\pgflineto{\pgfpoint{237.808838pt}{26.403709pt}}
\pgfusepath{stroke}
\pgfpathmoveto{\pgfpoint{239.800812pt}{26.399979pt}}
\pgflineto{\pgfpoint{238.804825pt}{26.399979pt}}
\pgfusepath{stroke}
\pgfpathmoveto{\pgfpoint{240.796814pt}{26.399979pt}}
\pgflineto{\pgfpoint{239.800812pt}{26.399979pt}}
\pgfusepath{stroke}
\pgfpathmoveto{\pgfpoint{241.792786pt}{26.399979pt}}
\pgflineto{\pgfpoint{240.796814pt}{26.399979pt}}
\pgfusepath{stroke}
\pgfpathmoveto{\pgfpoint{242.788757pt}{26.408813pt}}
\pgflineto{\pgfpoint{241.792786pt}{26.399979pt}}
\pgfusepath{stroke}
\pgfpathmoveto{\pgfpoint{243.784744pt}{26.426201pt}}
\pgflineto{\pgfpoint{242.788757pt}{26.408813pt}}
\pgfusepath{stroke}
\pgfpathmoveto{\pgfpoint{244.780731pt}{26.399979pt}}
\pgflineto{\pgfpoint{243.784744pt}{26.426201pt}}
\pgfusepath{stroke}
\pgfpathmoveto{\pgfpoint{245.776718pt}{26.401276pt}}
\pgflineto{\pgfpoint{244.780731pt}{26.399979pt}}
\pgfusepath{stroke}
\pgfpathmoveto{\pgfpoint{246.772705pt}{26.412910pt}}
\pgflineto{\pgfpoint{245.776718pt}{26.401276pt}}
\pgfusepath{stroke}
\pgfpathmoveto{\pgfpoint{247.768677pt}{28.808250pt}}
\pgflineto{\pgfpoint{246.772705pt}{26.412910pt}}
\pgfusepath{stroke}
\pgfpathmoveto{\pgfpoint{248.764679pt}{26.399979pt}}
\pgflineto{\pgfpoint{247.768677pt}{28.808250pt}}
\pgfusepath{stroke}
\pgfpathmoveto{\pgfpoint{249.760651pt}{26.401222pt}}
\pgflineto{\pgfpoint{248.764679pt}{26.399979pt}}
\pgfusepath{stroke}
\pgfpathmoveto{\pgfpoint{250.756638pt}{26.407082pt}}
\pgflineto{\pgfpoint{249.760651pt}{26.401222pt}}
\pgfusepath{stroke}
\pgfpathmoveto{\pgfpoint{251.752625pt}{26.399979pt}}
\pgflineto{\pgfpoint{250.756638pt}{26.407082pt}}
\pgfusepath{stroke}
\pgfpathmoveto{\pgfpoint{252.748611pt}{26.399979pt}}
\pgflineto{\pgfpoint{251.752625pt}{26.399979pt}}
\pgfusepath{stroke}
\pgfpathmoveto{\pgfpoint{253.744598pt}{26.399979pt}}
\pgflineto{\pgfpoint{252.748611pt}{26.399979pt}}
\pgfusepath{stroke}
\pgfpathmoveto{\pgfpoint{254.740570pt}{26.418449pt}}
\pgflineto{\pgfpoint{253.744598pt}{26.399979pt}}
\pgfusepath{stroke}
\pgfpathmoveto{\pgfpoint{255.736542pt}{26.399979pt}}
\pgflineto{\pgfpoint{254.740570pt}{26.418449pt}}
\pgfusepath{stroke}
\pgfpathmoveto{\pgfpoint{256.732544pt}{26.399979pt}}
\pgflineto{\pgfpoint{255.736542pt}{26.399979pt}}
\pgfusepath{stroke}
\pgfpathmoveto{\pgfpoint{257.728516pt}{26.399979pt}}
\pgflineto{\pgfpoint{256.732544pt}{26.399979pt}}
\pgfusepath{stroke}
\pgfpathmoveto{\pgfpoint{258.724518pt}{26.417419pt}}
\pgflineto{\pgfpoint{257.728516pt}{26.399979pt}}
\pgfusepath{stroke}
\pgfpathmoveto{\pgfpoint{259.720490pt}{26.399979pt}}
\pgflineto{\pgfpoint{258.724518pt}{26.417419pt}}
\pgfusepath{stroke}
\pgfpathmoveto{\pgfpoint{260.716492pt}{26.399979pt}}
\pgflineto{\pgfpoint{259.720490pt}{26.399979pt}}
\pgfusepath{stroke}
\pgfpathmoveto{\pgfpoint{261.712463pt}{26.456230pt}}
\pgflineto{\pgfpoint{260.716492pt}{26.399979pt}}
\pgfusepath{stroke}
\pgfpathmoveto{\pgfpoint{262.708435pt}{26.399979pt}}
\pgflineto{\pgfpoint{261.712463pt}{26.456230pt}}
\pgfusepath{stroke}
\pgfpathmoveto{\pgfpoint{263.704407pt}{26.399979pt}}
\pgflineto{\pgfpoint{262.708435pt}{26.399979pt}}
\pgfusepath{stroke}
\pgfpathmoveto{\pgfpoint{264.700409pt}{26.422783pt}}
\pgflineto{\pgfpoint{263.704407pt}{26.399979pt}}
\pgfusepath{stroke}
\pgfpathmoveto{\pgfpoint{265.696411pt}{26.399979pt}}
\pgflineto{\pgfpoint{264.700409pt}{26.422783pt}}
\pgfusepath{stroke}
\pgfpathmoveto{\pgfpoint{266.692383pt}{26.434059pt}}
\pgflineto{\pgfpoint{265.696411pt}{26.399979pt}}
\pgfusepath{stroke}
\pgfpathmoveto{\pgfpoint{267.688354pt}{26.399979pt}}
\pgflineto{\pgfpoint{266.692383pt}{26.434059pt}}
\pgfusepath{stroke}
\pgfpathmoveto{\pgfpoint{268.684326pt}{26.407372pt}}
\pgflineto{\pgfpoint{267.688354pt}{26.399979pt}}
\pgfusepath{stroke}
\pgfpathmoveto{\pgfpoint{269.680328pt}{26.407188pt}}
\pgflineto{\pgfpoint{268.684326pt}{26.407372pt}}
\pgfusepath{stroke}
\pgfpathmoveto{\pgfpoint{270.676331pt}{26.458199pt}}
\pgflineto{\pgfpoint{269.680328pt}{26.407188pt}}
\pgfusepath{stroke}
\pgfpathmoveto{\pgfpoint{271.672302pt}{28.359955pt}}
\pgflineto{\pgfpoint{270.676331pt}{26.458199pt}}
\pgfusepath{stroke}
\pgfpathmoveto{\pgfpoint{272.668274pt}{26.415123pt}}
\pgflineto{\pgfpoint{271.672302pt}{28.359955pt}}
\pgfusepath{stroke}
\pgfpathmoveto{\pgfpoint{273.664276pt}{26.399979pt}}
\pgflineto{\pgfpoint{272.668274pt}{26.415123pt}}
\pgfusepath{stroke}
\pgfpathmoveto{\pgfpoint{274.660248pt}{26.399979pt}}
\pgflineto{\pgfpoint{273.664276pt}{26.399979pt}}
\pgfusepath{stroke}
\pgfpathmoveto{\pgfpoint{275.656250pt}{26.399979pt}}
\pgflineto{\pgfpoint{274.660248pt}{26.399979pt}}
\pgfusepath{stroke}
\pgfpathmoveto{\pgfpoint{276.652222pt}{26.399979pt}}
\pgflineto{\pgfpoint{275.656250pt}{26.399979pt}}
\pgfusepath{stroke}
\pgfpathmoveto{\pgfpoint{277.648193pt}{26.399979pt}}
\pgflineto{\pgfpoint{276.652222pt}{26.399979pt}}
\pgfusepath{stroke}
\pgfpathmoveto{\pgfpoint{278.644196pt}{26.623550pt}}
\pgflineto{\pgfpoint{277.648193pt}{26.399979pt}}
\pgfusepath{stroke}
\pgfpathmoveto{\pgfpoint{279.640167pt}{27.888870pt}}
\pgflineto{\pgfpoint{278.644196pt}{26.623550pt}}
\pgfusepath{stroke}
\pgfpathmoveto{\pgfpoint{280.636139pt}{26.399979pt}}
\pgflineto{\pgfpoint{279.640167pt}{27.888870pt}}
\pgfusepath{stroke}
\pgfpathmoveto{\pgfpoint{281.632141pt}{26.399979pt}}
\pgflineto{\pgfpoint{280.636139pt}{26.399979pt}}
\pgfusepath{stroke}
\pgfpathmoveto{\pgfpoint{282.628113pt}{26.524590pt}}
\pgflineto{\pgfpoint{281.632141pt}{26.399979pt}}
\pgfusepath{stroke}
\pgfpathmoveto{\pgfpoint{283.624115pt}{26.456955pt}}
\pgflineto{\pgfpoint{282.628113pt}{26.524590pt}}
\pgfusepath{stroke}
\pgfpathmoveto{\pgfpoint{284.620087pt}{26.399979pt}}
\pgflineto{\pgfpoint{283.624115pt}{26.456955pt}}
\pgfusepath{stroke}
\pgfpathmoveto{\pgfpoint{285.616089pt}{26.402275pt}}
\pgflineto{\pgfpoint{284.620087pt}{26.399979pt}}
\pgfusepath{stroke}
\pgfpathmoveto{\pgfpoint{286.612061pt}{26.399979pt}}
\pgflineto{\pgfpoint{285.616089pt}{26.402275pt}}
\pgfusepath{stroke}
\pgfpathmoveto{\pgfpoint{287.608032pt}{26.399979pt}}
\pgflineto{\pgfpoint{286.612061pt}{26.399979pt}}
\pgfusepath{stroke}
\pgfpathmoveto{\pgfpoint{288.604004pt}{26.399979pt}}
\pgflineto{\pgfpoint{287.608032pt}{26.399979pt}}
\pgfusepath{stroke}
\pgfpathmoveto{\pgfpoint{289.600037pt}{26.399979pt}}
\pgflineto{\pgfpoint{288.604004pt}{26.399979pt}}
\pgfusepath{stroke}
{
\pgftransformshift{\pgfpoint{165.600006pt}{215.577454pt}}
\pgfnode{rectangle}{south}{\fontsize{10}{0}\selectfont\textcolor[rgb]{0,0,0}{{Spectral statistics VVX strain}}}{}{\pgfusepath{discard}}}
{
\pgftransformshift{\pgfpoint{165.600006pt}{101.199989pt}}
\pgfnode{rectangle}{south}{\fontsize{10}{0}\selectfont\textcolor[rgb]{0,0,0}{{Spectral statistics for BUT }}}{}{\pgfusepath{discard}}}
\end{pgfpicture}

\end{frame}

\only<article>{
  Let's tackle the problem of discriminating between different
  disease vectors. Ideally, we'd like to have a simple test that
  tells us what ails us. One kind of test is mass spectrometry. This
  graph shows spectrometry results for two types of bacteria. There
  is plenty of variation within each type, both due to measurement
  error and due to changes in the bacterial strains. Here, we plot
  the average and maximum energies measured for about 100 different
  examples from each strain.
}

\begin{frame}
  \frametitle{Nearest neighbour: the hidden secret of machine learning}
  \input{../figures/separation1.tikz}
\end{frame}

\only<article>{ Now, is it possible to identify an unknown strain
  based on this data? Actually, this is possible. Sometimes, very
  simple algorithms work very well. One of the simplest one involves
  just measuring the distance between the decsription of a new unknown
  strain and known ones. In this visualisation, I projected the
  1300-dimensional data into a 2-dimensional space. Here you can
  clearly see that it is possible to separate the two strains. We can
  use the distance to examples VVT and BUT in order to decide the type
  of an unknown strain.  }

\begin{frame}
  \frametitle{Comparing spectral data}
  \only<1>{\input{../figures/difference1.tikz}}
  \only<presentation>{\only<2>{% Title: glps_renderer figure
% Creator: GL2PS 1.3.8, (C) 1999-2012 C. Geuzaine
% For: Octave
% CreationDate: Fri Jun 16 12:38:10 2017
\begin{pgfpicture}
\pgfsetlinewidth{0.01pt}
\color[rgb]{1.000000,1.000000,1.000000}
\pgfpathmoveto{\pgfpoint{41.600006pt}{222.000000pt}}
\pgflineto{\pgfpoint{289.600037pt}{26.399979pt}}
\pgflineto{\pgfpoint{41.600006pt}{26.399979pt}}
\pgfpathclose
\pgfusepath{fill,stroke}
\pgfpathmoveto{\pgfpoint{41.600006pt}{222.000000pt}}
\pgflineto{\pgfpoint{289.600037pt}{222.000000pt}}
\pgflineto{\pgfpoint{289.600037pt}{26.399979pt}}
\pgfpathclose
\pgfusepath{fill,stroke}
\color[rgb]{1.000000,0.000000,0.000000}
\pgfpathmoveto{\pgfpoint{46.560013pt}{171.656006pt}}
\pgflineto{\pgfpoint{46.560013pt}{65.485382pt}}
\pgflineto{\pgfpoint{51.272934pt}{60.450993pt}}
\pgfpathclose
\pgfusepath{fill,stroke}
\pgfpathmoveto{\pgfpoint{51.272934pt}{60.450993pt}}
\pgflineto{\pgfpoint{51.520004pt}{54.621078pt}}
\pgflineto{\pgfpoint{51.520004pt}{60.187065pt}}
\pgfpathclose
\pgfusepath{fill,stroke}
\pgfpathmoveto{\pgfpoint{52.729080pt}{55.915833pt}}
\pgflineto{\pgfpoint{51.520004pt}{60.187065pt}}
\pgflineto{\pgfpoint{51.520004pt}{54.621078pt}}
\pgfpathclose
\pgfusepath{fill,stroke}
\pgfpathmoveto{\pgfpoint{52.729080pt}{55.915833pt}}
\pgflineto{\pgfpoint{56.480011pt}{42.665169pt}}
\pgflineto{\pgfpoint{56.480011pt}{59.932564pt}}
\pgfpathclose
\pgfusepath{fill,stroke}
\pgfpathmoveto{\pgfpoint{61.440010pt}{105.284721pt}}
\pgflineto{\pgfpoint{56.480011pt}{59.932564pt}}
\pgflineto{\pgfpoint{56.480011pt}{42.665169pt}}
\pgfpathclose
\pgfusepath{fill,stroke}
\pgfpathmoveto{\pgfpoint{61.440010pt}{46.590111pt}}
\pgflineto{\pgfpoint{61.440010pt}{105.284721pt}}
\pgflineto{\pgfpoint{56.480011pt}{42.665169pt}}
\pgfpathclose
\pgfusepath{fill,stroke}
\pgfpathmoveto{\pgfpoint{66.400009pt}{163.107224pt}}
\pgflineto{\pgfpoint{61.440010pt}{105.284721pt}}
\pgflineto{\pgfpoint{61.440010pt}{46.590111pt}}
\pgfpathclose
\pgfusepath{fill,stroke}
\pgfpathmoveto{\pgfpoint{66.400009pt}{63.127323pt}}
\pgflineto{\pgfpoint{66.400009pt}{163.107224pt}}
\pgflineto{\pgfpoint{61.440010pt}{46.590111pt}}
\pgfpathclose
\pgfusepath{fill,stroke}
\pgfpathmoveto{\pgfpoint{69.454124pt}{89.661858pt}}
\pgflineto{\pgfpoint{66.400009pt}{163.107224pt}}
\pgflineto{\pgfpoint{66.400009pt}{63.127323pt}}
\pgfpathclose
\pgfusepath{fill,stroke}
\pgfpathmoveto{\pgfpoint{69.454124pt}{89.661858pt}}
\pgflineto{\pgfpoint{71.360008pt}{43.829178pt}}
\pgflineto{\pgfpoint{71.360008pt}{106.220398pt}}
\pgfpathclose
\pgfusepath{fill,stroke}
\pgfpathmoveto{\pgfpoint{73.884773pt}{93.204117pt}}
\pgflineto{\pgfpoint{71.360008pt}{106.220398pt}}
\pgflineto{\pgfpoint{71.360008pt}{43.829178pt}}
\pgfpathclose
\pgfusepath{fill,stroke}
\pgfpathmoveto{\pgfpoint{73.884773pt}{93.204117pt}}
\pgflineto{\pgfpoint{76.320007pt}{80.649338pt}}
\pgflineto{\pgfpoint{76.320007pt}{140.828415pt}}
\pgfpathclose
\pgfusepath{fill,stroke}
\pgfpathmoveto{\pgfpoint{80.181015pt}{65.915634pt}}
\pgflineto{\pgfpoint{76.320007pt}{140.828415pt}}
\pgflineto{\pgfpoint{76.320007pt}{80.649338pt}}
\pgfpathclose
\pgfusepath{fill,stroke}
\pgfpathmoveto{\pgfpoint{80.181015pt}{65.915634pt}}
\pgflineto{\pgfpoint{81.280014pt}{44.592613pt}}
\pgflineto{\pgfpoint{81.280014pt}{61.721870pt}}
\pgfpathclose
\pgfusepath{fill,stroke}
\pgfpathmoveto{\pgfpoint{85.898277pt}{60.412163pt}}
\pgflineto{\pgfpoint{81.280014pt}{61.721870pt}}
\pgflineto{\pgfpoint{81.280014pt}{44.592613pt}}
\pgfpathclose
\pgfusepath{fill,stroke}
\pgfpathmoveto{\pgfpoint{85.898277pt}{60.412163pt}}
\pgflineto{\pgfpoint{86.240013pt}{60.315250pt}}
\pgflineto{\pgfpoint{86.240013pt}{61.582748pt}}
\pgfpathclose
\pgfusepath{fill,stroke}
\pgfpathmoveto{\pgfpoint{91.200012pt}{82.405769pt}}
\pgflineto{\pgfpoint{86.240013pt}{61.582748pt}}
\pgflineto{\pgfpoint{86.240013pt}{60.315250pt}}
\pgfpathclose
\pgfusepath{fill,stroke}
\pgfpathmoveto{\pgfpoint{91.200012pt}{67.963776pt}}
\pgflineto{\pgfpoint{91.200012pt}{82.405769pt}}
\pgflineto{\pgfpoint{86.240013pt}{60.315250pt}}
\pgfpathclose
\pgfusepath{fill,stroke}
\pgfpathmoveto{\pgfpoint{96.160011pt}{121.447739pt}}
\pgflineto{\pgfpoint{91.200012pt}{82.405769pt}}
\pgflineto{\pgfpoint{91.200012pt}{67.963776pt}}
\pgfpathclose
\pgfusepath{fill,stroke}
\pgfpathmoveto{\pgfpoint{96.160011pt}{39.422333pt}}
\pgflineto{\pgfpoint{96.160011pt}{121.447739pt}}
\pgflineto{\pgfpoint{91.200012pt}{67.963776pt}}
\pgfpathclose
\pgfusepath{fill,stroke}
\pgfpathmoveto{\pgfpoint{101.120010pt}{45.170807pt}}
\pgflineto{\pgfpoint{96.160011pt}{121.447739pt}}
\pgflineto{\pgfpoint{96.160011pt}{39.422333pt}}
\pgfpathclose
\pgfusepath{fill,stroke}
\pgfpathmoveto{\pgfpoint{101.120010pt}{36.805199pt}}
\pgflineto{\pgfpoint{101.120010pt}{45.170807pt}}
\pgflineto{\pgfpoint{96.160011pt}{39.422333pt}}
\pgfpathclose
\pgfusepath{fill,stroke}
\pgfpathmoveto{\pgfpoint{106.080017pt}{93.045525pt}}
\pgflineto{\pgfpoint{101.120010pt}{45.170807pt}}
\pgflineto{\pgfpoint{101.120010pt}{36.805199pt}}
\pgfpathclose
\pgfusepath{fill,stroke}
\pgfpathmoveto{\pgfpoint{106.080017pt}{57.417770pt}}
\pgflineto{\pgfpoint{106.080017pt}{93.045525pt}}
\pgflineto{\pgfpoint{101.120010pt}{36.805199pt}}
\pgfpathclose
\pgfusepath{fill,stroke}
\pgfpathmoveto{\pgfpoint{111.040009pt}{89.065750pt}}
\pgflineto{\pgfpoint{106.080017pt}{93.045525pt}}
\pgflineto{\pgfpoint{106.080017pt}{57.417770pt}}
\pgfpathclose
\pgfusepath{fill,stroke}
\pgfpathmoveto{\pgfpoint{111.040009pt}{48.803902pt}}
\pgflineto{\pgfpoint{111.040009pt}{89.065750pt}}
\pgflineto{\pgfpoint{106.080017pt}{57.417770pt}}
\pgfpathclose
\pgfusepath{fill,stroke}
\pgfpathmoveto{\pgfpoint{116.000015pt}{43.683624pt}}
\pgflineto{\pgfpoint{111.040009pt}{89.065750pt}}
\pgflineto{\pgfpoint{111.040009pt}{48.803902pt}}
\pgfpathclose
\pgfusepath{fill,stroke}
\pgfpathmoveto{\pgfpoint{116.000015pt}{40.934975pt}}
\pgflineto{\pgfpoint{116.000015pt}{43.683624pt}}
\pgflineto{\pgfpoint{111.040009pt}{48.803902pt}}
\pgfpathclose
\pgfusepath{fill,stroke}
\pgfpathmoveto{\pgfpoint{120.960007pt}{149.515289pt}}
\pgflineto{\pgfpoint{116.000015pt}{43.683624pt}}
\pgflineto{\pgfpoint{116.000015pt}{40.934975pt}}
\pgfpathclose
\pgfusepath{fill,stroke}
\pgfpathmoveto{\pgfpoint{120.960007pt}{115.830338pt}}
\pgflineto{\pgfpoint{120.960007pt}{149.515289pt}}
\pgflineto{\pgfpoint{116.000015pt}{40.934975pt}}
\pgfpathclose
\pgfusepath{fill,stroke}
\pgfpathmoveto{\pgfpoint{125.920013pt}{72.071335pt}}
\pgflineto{\pgfpoint{120.960007pt}{149.515289pt}}
\pgflineto{\pgfpoint{120.960007pt}{115.830338pt}}
\pgfpathclose
\pgfusepath{fill,stroke}
\pgfpathmoveto{\pgfpoint{125.920013pt}{66.937149pt}}
\pgflineto{\pgfpoint{125.920013pt}{72.071335pt}}
\pgflineto{\pgfpoint{120.960007pt}{115.830338pt}}
\pgfpathclose
\pgfusepath{fill,stroke}
\pgfpathmoveto{\pgfpoint{127.352821pt}{64.771484pt}}
\pgflineto{\pgfpoint{125.920013pt}{72.071335pt}}
\pgflineto{\pgfpoint{125.920013pt}{66.937149pt}}
\pgfpathclose
\pgfusepath{fill,stroke}
\pgfpathmoveto{\pgfpoint{127.352821pt}{64.771484pt}}
\pgflineto{\pgfpoint{130.880005pt}{46.801132pt}}
\pgflineto{\pgfpoint{130.880005pt}{59.440186pt}}
\pgfpathclose
\pgfusepath{fill,stroke}
\pgfpathmoveto{\pgfpoint{132.276794pt}{53.129593pt}}
\pgflineto{\pgfpoint{130.880005pt}{59.440186pt}}
\pgflineto{\pgfpoint{130.880005pt}{46.801132pt}}
\pgfpathclose
\pgfusepath{fill,stroke}
\pgfpathmoveto{\pgfpoint{132.276794pt}{53.129593pt}}
\pgflineto{\pgfpoint{135.840012pt}{37.031158pt}}
\pgflineto{\pgfpoint{135.840012pt}{69.273598pt}}
\pgfpathclose
\pgfusepath{fill,stroke}
\pgfpathmoveto{\pgfpoint{138.230530pt}{66.341614pt}}
\pgflineto{\pgfpoint{135.840012pt}{69.273598pt}}
\pgflineto{\pgfpoint{135.840012pt}{37.031158pt}}
\pgfpathclose
\pgfusepath{fill,stroke}
\pgfpathmoveto{\pgfpoint{138.230530pt}{66.341614pt}}
\pgflineto{\pgfpoint{140.800003pt}{63.190159pt}}
\pgflineto{\pgfpoint{140.800003pt}{97.846222pt}}
\pgfpathclose
\pgfusepath{fill,stroke}
\pgfpathmoveto{\pgfpoint{145.760010pt}{49.291016pt}}
\pgflineto{\pgfpoint{140.800003pt}{97.846222pt}}
\pgflineto{\pgfpoint{140.800003pt}{63.190159pt}}
\pgfpathclose
\pgfusepath{fill,stroke}
\pgfpathmoveto{\pgfpoint{145.760010pt}{45.087975pt}}
\pgflineto{\pgfpoint{145.760010pt}{49.291016pt}}
\pgflineto{\pgfpoint{140.800003pt}{63.190159pt}}
\pgfpathclose
\pgfusepath{fill,stroke}
\pgfpathmoveto{\pgfpoint{148.486938pt}{53.263680pt}}
\pgflineto{\pgfpoint{145.760010pt}{49.291016pt}}
\pgflineto{\pgfpoint{145.760010pt}{45.087975pt}}
\pgfpathclose
\pgfusepath{fill,stroke}
\pgfpathmoveto{\pgfpoint{148.486938pt}{53.263680pt}}
\pgflineto{\pgfpoint{150.720016pt}{56.516876pt}}
\pgflineto{\pgfpoint{150.720016pt}{59.958729pt}}
\pgfpathclose
\pgfusepath{fill,stroke}
\pgfpathmoveto{\pgfpoint{155.680023pt}{145.320908pt}}
\pgflineto{\pgfpoint{150.720016pt}{59.958729pt}}
\pgflineto{\pgfpoint{150.720016pt}{56.516876pt}}
\pgfpathclose
\pgfusepath{fill,stroke}
\pgfpathmoveto{\pgfpoint{155.680023pt}{39.056114pt}}
\pgflineto{\pgfpoint{155.680023pt}{145.320908pt}}
\pgflineto{\pgfpoint{150.720016pt}{56.516876pt}}
\pgfpathclose
\pgfusepath{fill,stroke}
\pgfpathmoveto{\pgfpoint{158.851837pt}{88.183441pt}}
\pgflineto{\pgfpoint{155.680023pt}{145.320908pt}}
\pgflineto{\pgfpoint{155.680023pt}{39.056114pt}}
\pgfpathclose
\pgfusepath{fill,stroke}
\pgfpathmoveto{\pgfpoint{158.851837pt}{88.183441pt}}
\pgflineto{\pgfpoint{160.640015pt}{55.971481pt}}
\pgflineto{\pgfpoint{160.640015pt}{115.879601pt}}
\pgfpathclose
\pgfusepath{fill,stroke}
\pgfpathmoveto{\pgfpoint{165.274719pt}{64.387100pt}}
\pgflineto{\pgfpoint{160.640015pt}{115.879601pt}}
\pgflineto{\pgfpoint{160.640015pt}{55.971481pt}}
\pgfpathclose
\pgfusepath{fill,stroke}
\pgfpathmoveto{\pgfpoint{165.274719pt}{64.387100pt}}
\pgflineto{\pgfpoint{165.600006pt}{60.773132pt}}
\pgflineto{\pgfpoint{165.600006pt}{64.977737pt}}
\pgfpathclose
\pgfusepath{fill,stroke}
\pgfpathmoveto{\pgfpoint{166.011597pt}{63.599285pt}}
\pgflineto{\pgfpoint{165.600006pt}{64.977737pt}}
\pgflineto{\pgfpoint{165.600006pt}{60.773132pt}}
\pgfpathclose
\pgfusepath{fill,stroke}
\pgfpathmoveto{\pgfpoint{166.011597pt}{63.599285pt}}
\pgflineto{\pgfpoint{170.560013pt}{48.365959pt}}
\pgflineto{\pgfpoint{170.560013pt}{94.831261pt}}
\pgfpathclose
\pgfusepath{fill,stroke}
\pgfpathmoveto{\pgfpoint{174.223160pt}{57.045738pt}}
\pgflineto{\pgfpoint{170.560013pt}{94.831261pt}}
\pgflineto{\pgfpoint{170.560013pt}{48.365959pt}}
\pgfpathclose
\pgfusepath{fill,stroke}
\pgfpathmoveto{\pgfpoint{174.223160pt}{57.045738pt}}
\pgflineto{\pgfpoint{175.520004pt}{43.668625pt}}
\pgflineto{\pgfpoint{175.520004pt}{60.118607pt}}
\pgfpathclose
\pgfusepath{fill,stroke}
\pgfpathmoveto{\pgfpoint{180.480011pt}{75.116745pt}}
\pgflineto{\pgfpoint{175.520004pt}{60.118607pt}}
\pgflineto{\pgfpoint{175.520004pt}{43.668625pt}}
\pgfpathclose
\pgfusepath{fill,stroke}
\pgfpathmoveto{\pgfpoint{180.480011pt}{40.850761pt}}
\pgflineto{\pgfpoint{180.480011pt}{75.116745pt}}
\pgflineto{\pgfpoint{175.520004pt}{43.668625pt}}
\pgfpathclose
\pgfusepath{fill,stroke}
\pgfpathmoveto{\pgfpoint{182.813766pt}{65.866241pt}}
\pgflineto{\pgfpoint{180.480011pt}{75.116745pt}}
\pgflineto{\pgfpoint{180.480011pt}{40.850761pt}}
\pgfpathclose
\pgfusepath{fill,stroke}
\pgfpathmoveto{\pgfpoint{182.813766pt}{65.866241pt}}
\pgflineto{\pgfpoint{185.440018pt}{55.456352pt}}
\pgflineto{\pgfpoint{185.440018pt}{94.016953pt}}
\pgfpathclose
\pgfusepath{fill,stroke}
\pgfpathmoveto{\pgfpoint{190.400024pt}{74.316742pt}}
\pgflineto{\pgfpoint{185.440018pt}{94.016953pt}}
\pgflineto{\pgfpoint{185.440018pt}{55.456352pt}}
\pgfpathclose
\pgfusepath{fill,stroke}
\pgfpathmoveto{\pgfpoint{190.400024pt}{40.646240pt}}
\pgflineto{\pgfpoint{190.400024pt}{74.316742pt}}
\pgflineto{\pgfpoint{185.440018pt}{55.456352pt}}
\pgfpathclose
\pgfusepath{fill,stroke}
\pgfpathmoveto{\pgfpoint{194.088348pt}{48.193687pt}}
\pgflineto{\pgfpoint{190.400024pt}{74.316742pt}}
\pgflineto{\pgfpoint{190.400024pt}{40.646240pt}}
\pgfpathclose
\pgfusepath{fill,stroke}
\pgfpathmoveto{\pgfpoint{194.088348pt}{48.193687pt}}
\pgflineto{\pgfpoint{195.360016pt}{39.186867pt}}
\pgflineto{\pgfpoint{195.360016pt}{50.795929pt}}
\pgfpathclose
\pgfusepath{fill,stroke}
\pgfpathmoveto{\pgfpoint{200.320007pt}{162.896408pt}}
\pgflineto{\pgfpoint{195.360016pt}{50.795929pt}}
\pgflineto{\pgfpoint{195.360016pt}{39.186867pt}}
\pgfpathclose
\pgfusepath{fill,stroke}
\pgfpathmoveto{\pgfpoint{200.320007pt}{111.419617pt}}
\pgflineto{\pgfpoint{200.320007pt}{162.896408pt}}
\pgflineto{\pgfpoint{195.360016pt}{39.186867pt}}
\pgfpathclose
\pgfusepath{fill,stroke}
\pgfpathmoveto{\pgfpoint{205.279999pt}{99.158791pt}}
\pgflineto{\pgfpoint{200.320007pt}{162.896408pt}}
\pgflineto{\pgfpoint{200.320007pt}{111.419617pt}}
\pgfpathclose
\pgfusepath{fill,stroke}
\pgfpathmoveto{\pgfpoint{205.279999pt}{77.342728pt}}
\pgflineto{\pgfpoint{205.279999pt}{99.158791pt}}
\pgflineto{\pgfpoint{200.320007pt}{111.419617pt}}
\pgfpathclose
\pgfusepath{fill,stroke}
\pgfpathmoveto{\pgfpoint{210.147217pt}{64.919937pt}}
\pgflineto{\pgfpoint{205.279999pt}{99.158791pt}}
\pgflineto{\pgfpoint{205.279999pt}{77.342728pt}}
\pgfpathclose
\pgfusepath{fill,stroke}
\pgfpathmoveto{\pgfpoint{210.147217pt}{64.919937pt}}
\pgflineto{\pgfpoint{210.240021pt}{64.267227pt}}
\pgflineto{\pgfpoint{210.240021pt}{64.683121pt}}
\pgfpathclose
\pgfusepath{fill,stroke}
\pgfpathmoveto{\pgfpoint{215.200012pt}{68.944382pt}}
\pgflineto{\pgfpoint{210.240021pt}{64.683121pt}}
\pgflineto{\pgfpoint{210.240021pt}{64.267227pt}}
\pgfpathclose
\pgfusepath{fill,stroke}
\pgfpathmoveto{\pgfpoint{215.200012pt}{36.115891pt}}
\pgflineto{\pgfpoint{215.200012pt}{68.944382pt}}
\pgflineto{\pgfpoint{210.240021pt}{64.267227pt}}
\pgfpathclose
\pgfusepath{fill,stroke}
\pgfpathmoveto{\pgfpoint{220.160019pt}{56.577995pt}}
\pgflineto{\pgfpoint{215.200012pt}{68.944382pt}}
\pgflineto{\pgfpoint{215.200012pt}{36.115891pt}}
\pgfpathclose
\pgfusepath{fill,stroke}
\pgfpathmoveto{\pgfpoint{220.160019pt}{48.298035pt}}
\pgflineto{\pgfpoint{220.160019pt}{56.577995pt}}
\pgflineto{\pgfpoint{215.200012pt}{36.115891pt}}
\pgfpathclose
\pgfusepath{fill,stroke}
\pgfpathmoveto{\pgfpoint{225.120026pt}{63.517673pt}}
\pgflineto{\pgfpoint{220.160019pt}{56.577995pt}}
\pgflineto{\pgfpoint{220.160019pt}{48.298035pt}}
\pgfpathclose
\pgfusepath{fill,stroke}
\pgfpathmoveto{\pgfpoint{225.120026pt}{51.340904pt}}
\pgflineto{\pgfpoint{225.120026pt}{63.517673pt}}
\pgflineto{\pgfpoint{220.160019pt}{48.298035pt}}
\pgfpathclose
\pgfusepath{fill,stroke}
\pgfpathmoveto{\pgfpoint{228.633514pt}{65.886925pt}}
\pgflineto{\pgfpoint{225.120026pt}{63.517673pt}}
\pgflineto{\pgfpoint{225.120026pt}{51.340904pt}}
\pgfpathclose
\pgfusepath{fill,stroke}
\pgfpathmoveto{\pgfpoint{228.633514pt}{65.886925pt}}
\pgflineto{\pgfpoint{230.080017pt}{66.862328pt}}
\pgflineto{\pgfpoint{230.080017pt}{71.875427pt}}
\pgfpathclose
\pgfusepath{fill,stroke}
\pgfpathmoveto{\pgfpoint{231.822998pt}{62.078987pt}}
\pgflineto{\pgfpoint{230.080017pt}{71.875427pt}}
\pgflineto{\pgfpoint{230.080017pt}{66.862328pt}}
\pgfpathclose
\pgfusepath{fill,stroke}
\pgfpathmoveto{\pgfpoint{231.822998pt}{62.078987pt}}
\pgflineto{\pgfpoint{235.040024pt}{43.997490pt}}
\pgflineto{\pgfpoint{235.040024pt}{53.250259pt}}
\pgfpathclose
\pgfusepath{fill,stroke}
\pgfpathmoveto{\pgfpoint{240.000000pt}{75.645462pt}}
\pgflineto{\pgfpoint{235.040024pt}{53.250259pt}}
\pgflineto{\pgfpoint{235.040024pt}{43.997490pt}}
\pgfpathclose
\pgfusepath{fill,stroke}
\pgfpathmoveto{\pgfpoint{240.000000pt}{61.037575pt}}
\pgflineto{\pgfpoint{240.000000pt}{75.645462pt}}
\pgflineto{\pgfpoint{235.040024pt}{43.997490pt}}
\pgfpathclose
\pgfusepath{fill,stroke}
\pgfpathmoveto{\pgfpoint{244.506226pt}{42.584846pt}}
\pgflineto{\pgfpoint{240.000000pt}{75.645462pt}}
\pgflineto{\pgfpoint{240.000000pt}{61.037575pt}}
\pgfpathclose
\pgfusepath{fill,stroke}
\pgfpathmoveto{\pgfpoint{244.506226pt}{42.584846pt}}
\pgflineto{\pgfpoint{244.960022pt}{39.255608pt}}
\pgflineto{\pgfpoint{244.960022pt}{40.726631pt}}
\pgfpathclose
\pgfusepath{fill,stroke}
\pgfpathmoveto{\pgfpoint{249.920013pt}{116.421631pt}}
\pgflineto{\pgfpoint{244.960022pt}{40.726631pt}}
\pgflineto{\pgfpoint{244.960022pt}{39.255608pt}}
\pgfpathclose
\pgfusepath{fill,stroke}
\pgfpathmoveto{\pgfpoint{249.920013pt}{106.094025pt}}
\pgflineto{\pgfpoint{249.920013pt}{116.421631pt}}
\pgflineto{\pgfpoint{244.960022pt}{39.255608pt}}
\pgfpathclose
\pgfusepath{fill,stroke}
\pgfpathmoveto{\pgfpoint{254.880020pt}{215.171143pt}}
\pgflineto{\pgfpoint{249.920013pt}{116.421631pt}}
\pgflineto{\pgfpoint{249.920013pt}{106.094025pt}}
\pgfpathclose
\pgfusepath{fill,stroke}
\pgfpathmoveto{\pgfpoint{254.880020pt}{70.739304pt}}
\pgflineto{\pgfpoint{254.880020pt}{215.171143pt}}
\pgflineto{\pgfpoint{249.920013pt}{106.094025pt}}
\pgfpathclose
\pgfusepath{fill,stroke}
\pgfpathmoveto{\pgfpoint{259.840027pt}{141.531891pt}}
\pgflineto{\pgfpoint{254.880020pt}{215.171143pt}}
\pgflineto{\pgfpoint{254.880020pt}{70.739304pt}}
\pgfpathclose
\pgfusepath{fill,stroke}
\pgfpathmoveto{\pgfpoint{259.840027pt}{73.925293pt}}
\pgflineto{\pgfpoint{259.840027pt}{141.531891pt}}
\pgflineto{\pgfpoint{254.880020pt}{70.739304pt}}
\pgfpathclose
\pgfusepath{fill,stroke}
\pgfpathmoveto{\pgfpoint{264.800018pt}{190.998230pt}}
\pgflineto{\pgfpoint{259.840027pt}{141.531891pt}}
\pgflineto{\pgfpoint{259.840027pt}{73.925293pt}}
\pgfpathclose
\pgfusepath{fill,stroke}
\pgfpathmoveto{\pgfpoint{264.800018pt}{34.848969pt}}
\pgflineto{\pgfpoint{264.800018pt}{190.998230pt}}
\pgflineto{\pgfpoint{259.840027pt}{73.925293pt}}
\pgfpathclose
\pgfusepath{fill,stroke}
\pgfpathmoveto{\pgfpoint{269.487213pt}{69.737061pt}}
\pgflineto{\pgfpoint{264.800018pt}{190.998230pt}}
\pgflineto{\pgfpoint{264.800018pt}{34.848969pt}}
\pgfpathclose
\pgfusepath{fill,stroke}
\pgfpathmoveto{\pgfpoint{269.487213pt}{69.737061pt}}
\pgflineto{\pgfpoint{269.760010pt}{62.679287pt}}
\pgflineto{\pgfpoint{269.760010pt}{71.767647pt}}
\pgfpathclose
\pgfusepath{fill,stroke}
\pgfpathmoveto{\pgfpoint{274.720001pt}{76.811203pt}}
\pgflineto{\pgfpoint{269.760010pt}{71.767647pt}}
\pgflineto{\pgfpoint{269.760010pt}{62.679287pt}}
\pgfpathclose
\pgfusepath{fill,stroke}
\pgfpathmoveto{\pgfpoint{274.720001pt}{37.509773pt}}
\pgflineto{\pgfpoint{274.720001pt}{76.811203pt}}
\pgflineto{\pgfpoint{269.760010pt}{62.679287pt}}
\pgfpathclose
\pgfusepath{fill,stroke}
\pgfpathmoveto{\pgfpoint{279.680023pt}{69.222656pt}}
\pgflineto{\pgfpoint{274.720001pt}{76.811203pt}}
\pgflineto{\pgfpoint{274.720001pt}{37.509773pt}}
\pgfpathclose
\pgfusepath{fill,stroke}
\pgfpathmoveto{\pgfpoint{279.680023pt}{54.383209pt}}
\pgflineto{\pgfpoint{279.680023pt}{69.222656pt}}
\pgflineto{\pgfpoint{274.720001pt}{37.509773pt}}
\pgfpathclose
\pgfusepath{fill,stroke}
\pgfpathmoveto{\pgfpoint{280.898163pt}{61.260727pt}}
\pgflineto{\pgfpoint{279.680023pt}{69.222656pt}}
\pgflineto{\pgfpoint{279.680023pt}{54.383209pt}}
\pgfpathclose
\pgfusepath{fill,stroke}
\pgfpathmoveto{\pgfpoint{280.898163pt}{61.260727pt}}
\pgflineto{\pgfpoint{284.640015pt}{36.803474pt}}
\pgflineto{\pgfpoint{284.640015pt}{82.386887pt}}
\pgfpathclose
\pgfusepath{fill,stroke}
\pgfpathmoveto{\pgfpoint{288.703094pt}{64.960159pt}}
\pgflineto{\pgfpoint{284.640015pt}{82.386887pt}}
\pgflineto{\pgfpoint{284.640015pt}{36.803474pt}}
\pgfpathclose
\pgfusepath{fill,stroke}
\pgfpathmoveto{\pgfpoint{289.600037pt}{71.175858pt}}
\pgflineto{\pgfpoint{288.703094pt}{64.960159pt}}
\pgflineto{\pgfpoint{289.600037pt}{61.113129pt}}
\pgfpathclose
\pgfusepath{fill,stroke}
\color[rgb]{0.000000,0.000000,0.000000}
\pgfsetlinewidth{0.500000pt}
\pgfsetdash{{16pt}{0pt}}{0pt}
\pgfpathmoveto{\pgfpoint{289.600037pt}{26.399979pt}}
\pgflineto{\pgfpoint{41.600006pt}{26.399979pt}}
\pgfusepath{stroke}
\pgfpathmoveto{\pgfpoint{289.600037pt}{222.000000pt}}
\pgflineto{\pgfpoint{41.600006pt}{222.000000pt}}
\pgfusepath{stroke}
\pgfpathmoveto{\pgfpoint{41.600006pt}{222.000000pt}}
\pgflineto{\pgfpoint{41.600006pt}{26.399979pt}}
\pgfusepath{stroke}
\pgfpathmoveto{\pgfpoint{289.600037pt}{222.000000pt}}
\pgflineto{\pgfpoint{289.600037pt}{26.399979pt}}
\pgfusepath{stroke}
\pgfpathmoveto{\pgfpoint{41.600006pt}{28.874924pt}}
\pgflineto{\pgfpoint{41.600006pt}{26.399979pt}}
\pgfusepath{stroke}
\pgfpathmoveto{\pgfpoint{41.600006pt}{219.525085pt}}
\pgflineto{\pgfpoint{41.600006pt}{222.000000pt}}
\pgfusepath{stroke}
\pgfpathmoveto{\pgfpoint{91.200012pt}{28.874924pt}}
\pgflineto{\pgfpoint{91.200012pt}{26.399979pt}}
\pgfusepath{stroke}
\pgfpathmoveto{\pgfpoint{91.200012pt}{219.525085pt}}
\pgflineto{\pgfpoint{91.200012pt}{222.000000pt}}
\pgfusepath{stroke}
\pgfpathmoveto{\pgfpoint{140.800003pt}{28.874924pt}}
\pgflineto{\pgfpoint{140.800003pt}{26.399979pt}}
\pgfusepath{stroke}
\pgfpathmoveto{\pgfpoint{140.800003pt}{219.525085pt}}
\pgflineto{\pgfpoint{140.800003pt}{222.000000pt}}
\pgfusepath{stroke}
\pgfpathmoveto{\pgfpoint{190.400024pt}{28.874924pt}}
\pgflineto{\pgfpoint{190.400024pt}{26.399979pt}}
\pgfusepath{stroke}
\pgfpathmoveto{\pgfpoint{190.400024pt}{219.525085pt}}
\pgflineto{\pgfpoint{190.400024pt}{222.000000pt}}
\pgfusepath{stroke}
\pgfpathmoveto{\pgfpoint{240.000000pt}{28.874924pt}}
\pgflineto{\pgfpoint{240.000000pt}{26.399979pt}}
\pgfusepath{stroke}
\pgfpathmoveto{\pgfpoint{240.000000pt}{219.525085pt}}
\pgflineto{\pgfpoint{240.000000pt}{222.000000pt}}
\pgfusepath{stroke}
\pgfpathmoveto{\pgfpoint{289.600037pt}{28.874924pt}}
\pgflineto{\pgfpoint{289.600037pt}{26.399979pt}}
\pgfusepath{stroke}
\pgfpathmoveto{\pgfpoint{289.600037pt}{219.525085pt}}
\pgflineto{\pgfpoint{289.600037pt}{222.000000pt}}
\pgfusepath{stroke}
{
\pgftransformshift{\pgfpoint{41.600006pt}{21.410187pt}}
\pgfnode{rectangle}{north}{\fontsize{10}{0}\selectfont\textcolor[rgb]{0,0,0}{{0}}}{}{\pgfusepath{discard}}}
{
\pgftransformshift{\pgfpoint{91.200012pt}{21.410187pt}}
\pgfnode{rectangle}{north}{\fontsize{10}{0}\selectfont\textcolor[rgb]{0,0,0}{{10}}}{}{\pgfusepath{discard}}}
{
\pgftransformshift{\pgfpoint{140.800003pt}{21.410187pt}}
\pgfnode{rectangle}{north}{\fontsize{10}{0}\selectfont\textcolor[rgb]{0,0,0}{{20}}}{}{\pgfusepath{discard}}}
{
\pgftransformshift{\pgfpoint{190.400009pt}{21.410187pt}}
\pgfnode{rectangle}{north}{\fontsize{10}{0}\selectfont\textcolor[rgb]{0,0,0}{{30}}}{}{\pgfusepath{discard}}}
{
\pgftransformshift{\pgfpoint{240.000000pt}{21.410187pt}}
\pgfnode{rectangle}{north}{\fontsize{10}{0}\selectfont\textcolor[rgb]{0,0,0}{{40}}}{}{\pgfusepath{discard}}}
{
\pgftransformshift{\pgfpoint{289.600037pt}{21.410187pt}}
\pgfnode{rectangle}{north}{\fontsize{10}{0}\selectfont\textcolor[rgb]{0,0,0}{{50}}}{}{\pgfusepath{discard}}}
\pgfpathmoveto{\pgfpoint{44.080009pt}{26.399979pt}}
\pgflineto{\pgfpoint{41.600006pt}{26.399979pt}}
\pgfusepath{stroke}
\pgfpathmoveto{\pgfpoint{287.120026pt}{26.399979pt}}
\pgflineto{\pgfpoint{289.600037pt}{26.399979pt}}
\pgfusepath{stroke}
\pgfpathmoveto{\pgfpoint{44.080009pt}{58.999985pt}}
\pgflineto{\pgfpoint{41.600006pt}{58.999985pt}}
\pgfusepath{stroke}
\pgfpathmoveto{\pgfpoint{287.120026pt}{58.999985pt}}
\pgflineto{\pgfpoint{289.600037pt}{58.999985pt}}
\pgfusepath{stroke}
\pgfpathmoveto{\pgfpoint{44.080009pt}{91.599991pt}}
\pgflineto{\pgfpoint{41.600006pt}{91.599991pt}}
\pgfusepath{stroke}
\pgfpathmoveto{\pgfpoint{287.120026pt}{91.599991pt}}
\pgflineto{\pgfpoint{289.600037pt}{91.599991pt}}
\pgfusepath{stroke}
\pgfpathmoveto{\pgfpoint{44.080009pt}{124.199997pt}}
\pgflineto{\pgfpoint{41.600006pt}{124.199997pt}}
\pgfusepath{stroke}
\pgfpathmoveto{\pgfpoint{287.120026pt}{124.199997pt}}
\pgflineto{\pgfpoint{289.600037pt}{124.199997pt}}
\pgfusepath{stroke}
\pgfpathmoveto{\pgfpoint{44.080009pt}{156.800003pt}}
\pgflineto{\pgfpoint{41.600006pt}{156.800003pt}}
\pgfusepath{stroke}
\pgfpathmoveto{\pgfpoint{287.120026pt}{156.800003pt}}
\pgflineto{\pgfpoint{289.600037pt}{156.800003pt}}
\pgfusepath{stroke}
\pgfpathmoveto{\pgfpoint{44.080009pt}{189.400009pt}}
\pgflineto{\pgfpoint{41.600006pt}{189.400009pt}}
\pgfusepath{stroke}
\pgfpathmoveto{\pgfpoint{287.120026pt}{189.400009pt}}
\pgflineto{\pgfpoint{289.600037pt}{189.400009pt}}
\pgfusepath{stroke}
\pgfpathmoveto{\pgfpoint{44.080009pt}{222.000000pt}}
\pgflineto{\pgfpoint{41.600006pt}{222.000000pt}}
\pgfusepath{stroke}
\pgfpathmoveto{\pgfpoint{287.120026pt}{222.000000pt}}
\pgflineto{\pgfpoint{289.600037pt}{222.000000pt}}
\pgfusepath{stroke}
{
\pgftransformshift{\pgfpoint{36.600006pt}{26.399979pt}}
\pgfnode{rectangle}{east}{\fontsize{10}{0}\selectfont\textcolor[rgb]{0,0,0}{{0}}}{}{\pgfusepath{discard}}}
{
\pgftransformshift{\pgfpoint{36.600006pt}{58.999985pt}}
\pgfnode{rectangle}{east}{\fontsize{10}{0}\selectfont\textcolor[rgb]{0,0,0}{{1}}}{}{\pgfusepath{discard}}}
{
\pgftransformshift{\pgfpoint{36.600006pt}{91.599991pt}}
\pgfnode{rectangle}{east}{\fontsize{10}{0}\selectfont\textcolor[rgb]{0,0,0}{{2}}}{}{\pgfusepath{discard}}}
{
\pgftransformshift{\pgfpoint{36.600006pt}{124.199989pt}}
\pgfnode{rectangle}{east}{\fontsize{10}{0}\selectfont\textcolor[rgb]{0,0,0}{{3}}}{}{\pgfusepath{discard}}}
{
\pgftransformshift{\pgfpoint{36.600006pt}{156.799988pt}}
\pgfnode{rectangle}{east}{\fontsize{10}{0}\selectfont\textcolor[rgb]{0,0,0}{{4}}}{}{\pgfusepath{discard}}}
{
\pgftransformshift{\pgfpoint{36.600006pt}{189.399994pt}}
\pgfnode{rectangle}{east}{\fontsize{10}{0}\selectfont\textcolor[rgb]{0,0,0}{{5}}}{}{\pgfusepath{discard}}}
{
\pgftransformshift{\pgfpoint{36.600006pt}{222.000000pt}}
\pgfnode{rectangle}{east}{\fontsize{10}{0}\selectfont\textcolor[rgb]{0,0,0}{{6}}}{}{\pgfusepath{discard}}}
\color[rgb]{0.000000,0.000000,1.000000}
\pgfsetdash{}{0pt}
\pgfpathmoveto{\pgfpoint{51.520004pt}{60.187065pt}}
\pgflineto{\pgfpoint{46.560013pt}{65.485382pt}}
\pgfusepath{stroke}
\pgfpathmoveto{\pgfpoint{56.480011pt}{42.665169pt}}
\pgflineto{\pgfpoint{51.520004pt}{60.187065pt}}
\pgfusepath{stroke}
\pgfpathmoveto{\pgfpoint{61.440010pt}{46.590111pt}}
\pgflineto{\pgfpoint{56.480011pt}{42.665169pt}}
\pgfusepath{stroke}
\pgfpathmoveto{\pgfpoint{66.400009pt}{63.127323pt}}
\pgflineto{\pgfpoint{61.440010pt}{46.590111pt}}
\pgfusepath{stroke}
\pgfpathmoveto{\pgfpoint{71.360008pt}{106.220398pt}}
\pgflineto{\pgfpoint{66.400009pt}{63.127323pt}}
\pgfusepath{stroke}
\pgfpathmoveto{\pgfpoint{76.320007pt}{80.649338pt}}
\pgflineto{\pgfpoint{71.360008pt}{106.220398pt}}
\pgfusepath{stroke}
\pgfpathmoveto{\pgfpoint{81.280014pt}{61.721870pt}}
\pgflineto{\pgfpoint{76.320007pt}{80.649338pt}}
\pgfusepath{stroke}
\pgfpathmoveto{\pgfpoint{86.240013pt}{60.315250pt}}
\pgflineto{\pgfpoint{81.280014pt}{61.721870pt}}
\pgfusepath{stroke}
\pgfpathmoveto{\pgfpoint{91.200012pt}{67.963776pt}}
\pgflineto{\pgfpoint{86.240013pt}{60.315250pt}}
\pgfusepath{stroke}
\pgfpathmoveto{\pgfpoint{96.160011pt}{39.422333pt}}
\pgflineto{\pgfpoint{91.200012pt}{67.963776pt}}
\pgfusepath{stroke}
\pgfpathmoveto{\pgfpoint{101.120010pt}{36.805199pt}}
\pgflineto{\pgfpoint{96.160011pt}{39.422333pt}}
\pgfusepath{stroke}
\pgfpathmoveto{\pgfpoint{106.080017pt}{57.417770pt}}
\pgflineto{\pgfpoint{101.120010pt}{36.805199pt}}
\pgfusepath{stroke}
\pgfpathmoveto{\pgfpoint{111.040009pt}{48.803902pt}}
\pgflineto{\pgfpoint{106.080017pt}{57.417770pt}}
\pgfusepath{stroke}
\pgfpathmoveto{\pgfpoint{116.000015pt}{40.934975pt}}
\pgflineto{\pgfpoint{111.040009pt}{48.803902pt}}
\pgfusepath{stroke}
\pgfpathmoveto{\pgfpoint{120.960007pt}{115.830338pt}}
\pgflineto{\pgfpoint{116.000015pt}{40.934975pt}}
\pgfusepath{stroke}
\pgfpathmoveto{\pgfpoint{125.920013pt}{66.937149pt}}
\pgflineto{\pgfpoint{120.960007pt}{115.830338pt}}
\pgfusepath{stroke}
\pgfpathmoveto{\pgfpoint{130.880005pt}{59.440186pt}}
\pgflineto{\pgfpoint{125.920013pt}{66.937149pt}}
\pgfusepath{stroke}
\pgfpathmoveto{\pgfpoint{135.840012pt}{37.031158pt}}
\pgflineto{\pgfpoint{130.880005pt}{59.440186pt}}
\pgfusepath{stroke}
\pgfpathmoveto{\pgfpoint{140.800003pt}{97.846222pt}}
\pgflineto{\pgfpoint{135.840012pt}{37.031158pt}}
\pgfusepath{stroke}
\pgfpathmoveto{\pgfpoint{145.760010pt}{49.291016pt}}
\pgflineto{\pgfpoint{140.800003pt}{97.846222pt}}
\pgfusepath{stroke}
\pgfpathmoveto{\pgfpoint{150.720016pt}{56.516876pt}}
\pgflineto{\pgfpoint{145.760010pt}{49.291016pt}}
\pgfusepath{stroke}
\pgfpathmoveto{\pgfpoint{155.680023pt}{39.056114pt}}
\pgflineto{\pgfpoint{150.720016pt}{56.516876pt}}
\pgfusepath{stroke}
\pgfpathmoveto{\pgfpoint{160.640015pt}{115.879601pt}}
\pgflineto{\pgfpoint{155.680023pt}{39.056114pt}}
\pgfusepath{stroke}
\pgfpathmoveto{\pgfpoint{165.600006pt}{60.773132pt}}
\pgflineto{\pgfpoint{160.640015pt}{115.879601pt}}
\pgfusepath{stroke}
\pgfpathmoveto{\pgfpoint{170.560013pt}{94.831261pt}}
\pgflineto{\pgfpoint{165.600006pt}{60.773132pt}}
\pgfusepath{stroke}
\pgfpathmoveto{\pgfpoint{175.520004pt}{43.668625pt}}
\pgflineto{\pgfpoint{170.560013pt}{94.831261pt}}
\pgfusepath{stroke}
\pgfpathmoveto{\pgfpoint{180.480011pt}{40.850761pt}}
\pgflineto{\pgfpoint{175.520004pt}{43.668625pt}}
\pgfusepath{stroke}
\pgfpathmoveto{\pgfpoint{185.440018pt}{94.016953pt}}
\pgflineto{\pgfpoint{180.480011pt}{40.850761pt}}
\pgfusepath{stroke}
\pgfpathmoveto{\pgfpoint{190.400024pt}{74.316742pt}}
\pgflineto{\pgfpoint{185.440018pt}{94.016953pt}}
\pgfusepath{stroke}
\pgfpathmoveto{\pgfpoint{195.360016pt}{39.186867pt}}
\pgflineto{\pgfpoint{190.400024pt}{74.316742pt}}
\pgfusepath{stroke}
\pgfpathmoveto{\pgfpoint{200.320007pt}{111.419617pt}}
\pgflineto{\pgfpoint{195.360016pt}{39.186867pt}}
\pgfusepath{stroke}
\pgfpathmoveto{\pgfpoint{205.279999pt}{77.342728pt}}
\pgflineto{\pgfpoint{200.320007pt}{111.419617pt}}
\pgfusepath{stroke}
\pgfpathmoveto{\pgfpoint{210.240021pt}{64.683121pt}}
\pgflineto{\pgfpoint{205.279999pt}{77.342728pt}}
\pgfusepath{stroke}
\pgfpathmoveto{\pgfpoint{215.200012pt}{68.944382pt}}
\pgflineto{\pgfpoint{210.240021pt}{64.683121pt}}
\pgfusepath{stroke}
\pgfpathmoveto{\pgfpoint{220.160019pt}{56.577995pt}}
\pgflineto{\pgfpoint{215.200012pt}{68.944382pt}}
\pgfusepath{stroke}
\pgfpathmoveto{\pgfpoint{225.120026pt}{63.517673pt}}
\pgflineto{\pgfpoint{220.160019pt}{56.577995pt}}
\pgfusepath{stroke}
\pgfpathmoveto{\pgfpoint{230.080017pt}{66.862328pt}}
\pgflineto{\pgfpoint{225.120026pt}{63.517673pt}}
\pgfusepath{stroke}
\pgfpathmoveto{\pgfpoint{235.040024pt}{53.250259pt}}
\pgflineto{\pgfpoint{230.080017pt}{66.862328pt}}
\pgfusepath{stroke}
\pgfpathmoveto{\pgfpoint{240.000000pt}{75.645462pt}}
\pgflineto{\pgfpoint{235.040024pt}{53.250259pt}}
\pgfusepath{stroke}
\pgfpathmoveto{\pgfpoint{244.960022pt}{39.255608pt}}
\pgflineto{\pgfpoint{240.000000pt}{75.645462pt}}
\pgfusepath{stroke}
\pgfpathmoveto{\pgfpoint{249.920013pt}{106.094025pt}}
\pgflineto{\pgfpoint{244.960022pt}{39.255608pt}}
\pgfusepath{stroke}
\pgfpathmoveto{\pgfpoint{254.880020pt}{70.739304pt}}
\pgflineto{\pgfpoint{249.920013pt}{106.094025pt}}
\pgfusepath{stroke}
\pgfpathmoveto{\pgfpoint{259.840027pt}{73.925293pt}}
\pgflineto{\pgfpoint{254.880020pt}{70.739304pt}}
\pgfusepath{stroke}
\pgfpathmoveto{\pgfpoint{264.800018pt}{34.848969pt}}
\pgflineto{\pgfpoint{259.840027pt}{73.925293pt}}
\pgfusepath{stroke}
\pgfpathmoveto{\pgfpoint{269.760010pt}{71.767647pt}}
\pgflineto{\pgfpoint{264.800018pt}{34.848969pt}}
\pgfusepath{stroke}
\pgfpathmoveto{\pgfpoint{274.720001pt}{76.811203pt}}
\pgflineto{\pgfpoint{269.760010pt}{71.767647pt}}
\pgfusepath{stroke}
\pgfpathmoveto{\pgfpoint{279.680023pt}{69.222656pt}}
\pgflineto{\pgfpoint{274.720001pt}{76.811203pt}}
\pgfusepath{stroke}
\pgfpathmoveto{\pgfpoint{284.640015pt}{36.803474pt}}
\pgflineto{\pgfpoint{279.680023pt}{69.222656pt}}
\pgfusepath{stroke}
\pgfpathmoveto{\pgfpoint{289.600037pt}{71.175858pt}}
\pgflineto{\pgfpoint{284.640015pt}{36.803474pt}}
\pgfusepath{stroke}
\color[rgb]{0.000000,0.500000,0.000000}
\pgfpathmoveto{\pgfpoint{51.520004pt}{54.621078pt}}
\pgflineto{\pgfpoint{46.560013pt}{171.656006pt}}
\pgfusepath{stroke}
\pgfpathmoveto{\pgfpoint{56.480011pt}{59.932564pt}}
\pgflineto{\pgfpoint{51.520004pt}{54.621078pt}}
\pgfusepath{stroke}
\pgfpathmoveto{\pgfpoint{61.440010pt}{105.284721pt}}
\pgflineto{\pgfpoint{56.480011pt}{59.932564pt}}
\pgfusepath{stroke}
\pgfpathmoveto{\pgfpoint{66.400009pt}{163.107224pt}}
\pgflineto{\pgfpoint{61.440010pt}{105.284721pt}}
\pgfusepath{stroke}
\pgfpathmoveto{\pgfpoint{71.360008pt}{43.829178pt}}
\pgflineto{\pgfpoint{66.400009pt}{163.107224pt}}
\pgfusepath{stroke}
\pgfpathmoveto{\pgfpoint{76.320007pt}{140.828415pt}}
\pgflineto{\pgfpoint{71.360008pt}{43.829178pt}}
\pgfusepath{stroke}
\pgfpathmoveto{\pgfpoint{81.280014pt}{44.592613pt}}
\pgflineto{\pgfpoint{76.320007pt}{140.828415pt}}
\pgfusepath{stroke}
\pgfpathmoveto{\pgfpoint{86.240013pt}{61.582748pt}}
\pgflineto{\pgfpoint{81.280014pt}{44.592613pt}}
\pgfusepath{stroke}
\pgfpathmoveto{\pgfpoint{91.200012pt}{82.405769pt}}
\pgflineto{\pgfpoint{86.240013pt}{61.582748pt}}
\pgfusepath{stroke}
\pgfpathmoveto{\pgfpoint{96.160011pt}{121.447739pt}}
\pgflineto{\pgfpoint{91.200012pt}{82.405769pt}}
\pgfusepath{stroke}
\pgfpathmoveto{\pgfpoint{101.120010pt}{45.170807pt}}
\pgflineto{\pgfpoint{96.160011pt}{121.447739pt}}
\pgfusepath{stroke}
\pgfpathmoveto{\pgfpoint{106.080017pt}{93.045525pt}}
\pgflineto{\pgfpoint{101.120010pt}{45.170807pt}}
\pgfusepath{stroke}
\pgfpathmoveto{\pgfpoint{111.040009pt}{89.065750pt}}
\pgflineto{\pgfpoint{106.080017pt}{93.045525pt}}
\pgfusepath{stroke}
\pgfpathmoveto{\pgfpoint{116.000015pt}{43.683624pt}}
\pgflineto{\pgfpoint{111.040009pt}{89.065750pt}}
\pgfusepath{stroke}
\pgfpathmoveto{\pgfpoint{120.960007pt}{149.515289pt}}
\pgflineto{\pgfpoint{116.000015pt}{43.683624pt}}
\pgfusepath{stroke}
\pgfpathmoveto{\pgfpoint{125.920013pt}{72.071335pt}}
\pgflineto{\pgfpoint{120.960007pt}{149.515289pt}}
\pgfusepath{stroke}
\pgfpathmoveto{\pgfpoint{130.880005pt}{46.801132pt}}
\pgflineto{\pgfpoint{125.920013pt}{72.071335pt}}
\pgfusepath{stroke}
\pgfpathmoveto{\pgfpoint{135.840012pt}{69.273598pt}}
\pgflineto{\pgfpoint{130.880005pt}{46.801132pt}}
\pgfusepath{stroke}
\pgfpathmoveto{\pgfpoint{140.800003pt}{63.190159pt}}
\pgflineto{\pgfpoint{135.840012pt}{69.273598pt}}
\pgfusepath{stroke}
\pgfpathmoveto{\pgfpoint{145.760010pt}{45.087975pt}}
\pgflineto{\pgfpoint{140.800003pt}{63.190159pt}}
\pgfusepath{stroke}
\pgfpathmoveto{\pgfpoint{150.720016pt}{59.958729pt}}
\pgflineto{\pgfpoint{145.760010pt}{45.087975pt}}
\pgfusepath{stroke}
\pgfpathmoveto{\pgfpoint{155.680023pt}{145.320908pt}}
\pgflineto{\pgfpoint{150.720016pt}{59.958729pt}}
\pgfusepath{stroke}
\pgfpathmoveto{\pgfpoint{160.640015pt}{55.971481pt}}
\pgflineto{\pgfpoint{155.680023pt}{145.320908pt}}
\pgfusepath{stroke}
\pgfpathmoveto{\pgfpoint{165.600006pt}{64.977737pt}}
\pgflineto{\pgfpoint{160.640015pt}{55.971481pt}}
\pgfusepath{stroke}
\pgfpathmoveto{\pgfpoint{170.560013pt}{48.365959pt}}
\pgflineto{\pgfpoint{165.600006pt}{64.977737pt}}
\pgfusepath{stroke}
\pgfpathmoveto{\pgfpoint{175.520004pt}{60.118607pt}}
\pgflineto{\pgfpoint{170.560013pt}{48.365959pt}}
\pgfusepath{stroke}
\pgfpathmoveto{\pgfpoint{180.480011pt}{75.116745pt}}
\pgflineto{\pgfpoint{175.520004pt}{60.118607pt}}
\pgfusepath{stroke}
\pgfpathmoveto{\pgfpoint{185.440018pt}{55.456352pt}}
\pgflineto{\pgfpoint{180.480011pt}{75.116745pt}}
\pgfusepath{stroke}
\pgfpathmoveto{\pgfpoint{190.400024pt}{40.646240pt}}
\pgflineto{\pgfpoint{185.440018pt}{55.456352pt}}
\pgfusepath{stroke}
\pgfpathmoveto{\pgfpoint{195.360016pt}{50.795929pt}}
\pgflineto{\pgfpoint{190.400024pt}{40.646240pt}}
\pgfusepath{stroke}
\pgfpathmoveto{\pgfpoint{200.320007pt}{162.896408pt}}
\pgflineto{\pgfpoint{195.360016pt}{50.795929pt}}
\pgfusepath{stroke}
\pgfpathmoveto{\pgfpoint{205.279999pt}{99.158791pt}}
\pgflineto{\pgfpoint{200.320007pt}{162.896408pt}}
\pgfusepath{stroke}
\pgfpathmoveto{\pgfpoint{210.240021pt}{64.267227pt}}
\pgflineto{\pgfpoint{205.279999pt}{99.158791pt}}
\pgfusepath{stroke}
\pgfpathmoveto{\pgfpoint{215.200012pt}{36.115891pt}}
\pgflineto{\pgfpoint{210.240021pt}{64.267227pt}}
\pgfusepath{stroke}
\pgfpathmoveto{\pgfpoint{220.160019pt}{48.298035pt}}
\pgflineto{\pgfpoint{215.200012pt}{36.115891pt}}
\pgfusepath{stroke}
\pgfpathmoveto{\pgfpoint{225.120026pt}{51.340904pt}}
\pgflineto{\pgfpoint{220.160019pt}{48.298035pt}}
\pgfusepath{stroke}
\pgfpathmoveto{\pgfpoint{230.080017pt}{71.875427pt}}
\pgflineto{\pgfpoint{225.120026pt}{51.340904pt}}
\pgfusepath{stroke}
\pgfpathmoveto{\pgfpoint{235.040024pt}{43.997490pt}}
\pgflineto{\pgfpoint{230.080017pt}{71.875427pt}}
\pgfusepath{stroke}
\pgfpathmoveto{\pgfpoint{240.000000pt}{61.037575pt}}
\pgflineto{\pgfpoint{235.040024pt}{43.997490pt}}
\pgfusepath{stroke}
\pgfpathmoveto{\pgfpoint{244.960022pt}{40.726631pt}}
\pgflineto{\pgfpoint{240.000000pt}{61.037575pt}}
\pgfusepath{stroke}
\pgfpathmoveto{\pgfpoint{249.920013pt}{116.421631pt}}
\pgflineto{\pgfpoint{244.960022pt}{40.726631pt}}
\pgfusepath{stroke}
\pgfpathmoveto{\pgfpoint{254.880020pt}{215.171143pt}}
\pgflineto{\pgfpoint{249.920013pt}{116.421631pt}}
\pgfusepath{stroke}
\pgfpathmoveto{\pgfpoint{259.840027pt}{141.531891pt}}
\pgflineto{\pgfpoint{254.880020pt}{215.171143pt}}
\pgfusepath{stroke}
\pgfpathmoveto{\pgfpoint{264.800018pt}{190.998230pt}}
\pgflineto{\pgfpoint{259.840027pt}{141.531891pt}}
\pgfusepath{stroke}
\pgfpathmoveto{\pgfpoint{269.760010pt}{62.679287pt}}
\pgflineto{\pgfpoint{264.800018pt}{190.998230pt}}
\pgfusepath{stroke}
\pgfpathmoveto{\pgfpoint{274.720001pt}{37.509773pt}}
\pgflineto{\pgfpoint{269.760010pt}{62.679287pt}}
\pgfusepath{stroke}
\pgfpathmoveto{\pgfpoint{279.680023pt}{54.383209pt}}
\pgflineto{\pgfpoint{274.720001pt}{37.509773pt}}
\pgfusepath{stroke}
\pgfpathmoveto{\pgfpoint{284.640015pt}{82.386887pt}}
\pgflineto{\pgfpoint{279.680023pt}{54.383209pt}}
\pgfusepath{stroke}
\pgfpathmoveto{\pgfpoint{289.600037pt}{61.113129pt}}
\pgflineto{\pgfpoint{284.640015pt}{82.386887pt}}
\pgfusepath{stroke}
\color[rgb]{0.000000,0.000000,0.000000}
\pgfpathmoveto{\pgfpoint{288.703094pt}{64.960159pt}}
\pgflineto{\pgfpoint{289.600037pt}{61.113129pt}}
\pgfusepath{stroke}
\pgfpathmoveto{\pgfpoint{289.600037pt}{71.175858pt}}
\pgflineto{\pgfpoint{288.703094pt}{64.960159pt}}
\pgfusepath{stroke}
\pgfpathmoveto{\pgfpoint{289.600037pt}{61.113129pt}}
\pgflineto{\pgfpoint{289.600037pt}{71.175858pt}}
\pgfusepath{stroke}
\pgfpathmoveto{\pgfpoint{284.640015pt}{36.803474pt}}
\pgflineto{\pgfpoint{288.703094pt}{64.960159pt}}
\pgfusepath{stroke}
\pgfpathmoveto{\pgfpoint{280.898163pt}{61.260727pt}}
\pgflineto{\pgfpoint{284.640015pt}{36.803474pt}}
\pgfusepath{stroke}
\pgfpathmoveto{\pgfpoint{284.640015pt}{82.386887pt}}
\pgflineto{\pgfpoint{280.898163pt}{61.260727pt}}
\pgfusepath{stroke}
\pgfpathmoveto{\pgfpoint{288.703094pt}{64.960159pt}}
\pgflineto{\pgfpoint{284.640015pt}{82.386887pt}}
\pgfusepath{stroke}
\pgfpathmoveto{\pgfpoint{279.680023pt}{54.383209pt}}
\pgflineto{\pgfpoint{280.898163pt}{61.260727pt}}
\pgfusepath{stroke}
\pgfpathmoveto{\pgfpoint{274.720001pt}{37.509773pt}}
\pgflineto{\pgfpoint{279.680023pt}{54.383209pt}}
\pgfusepath{stroke}
\pgfpathmoveto{\pgfpoint{269.760010pt}{62.679287pt}}
\pgflineto{\pgfpoint{274.720001pt}{37.509773pt}}
\pgfusepath{stroke}
\pgfpathmoveto{\pgfpoint{269.487213pt}{69.737061pt}}
\pgflineto{\pgfpoint{269.760010pt}{62.679287pt}}
\pgfusepath{stroke}
\pgfpathmoveto{\pgfpoint{269.760010pt}{71.767647pt}}
\pgflineto{\pgfpoint{269.487213pt}{69.737061pt}}
\pgfusepath{stroke}
\pgfpathmoveto{\pgfpoint{274.720001pt}{76.811203pt}}
\pgflineto{\pgfpoint{269.760010pt}{71.767647pt}}
\pgfusepath{stroke}
\pgfpathmoveto{\pgfpoint{279.680023pt}{69.222656pt}}
\pgflineto{\pgfpoint{274.720001pt}{76.811203pt}}
\pgfusepath{stroke}
\pgfpathmoveto{\pgfpoint{280.898163pt}{61.260727pt}}
\pgflineto{\pgfpoint{279.680023pt}{69.222656pt}}
\pgfusepath{stroke}
\pgfpathmoveto{\pgfpoint{264.800018pt}{34.848969pt}}
\pgflineto{\pgfpoint{269.487213pt}{69.737061pt}}
\pgfusepath{stroke}
\pgfpathmoveto{\pgfpoint{259.840027pt}{73.925293pt}}
\pgflineto{\pgfpoint{264.800018pt}{34.848969pt}}
\pgfusepath{stroke}
\pgfpathmoveto{\pgfpoint{254.880020pt}{70.739304pt}}
\pgflineto{\pgfpoint{259.840027pt}{73.925293pt}}
\pgfusepath{stroke}
\pgfpathmoveto{\pgfpoint{249.920013pt}{106.094025pt}}
\pgflineto{\pgfpoint{254.880020pt}{70.739304pt}}
\pgfusepath{stroke}
\pgfpathmoveto{\pgfpoint{244.960022pt}{39.255608pt}}
\pgflineto{\pgfpoint{249.920013pt}{106.094025pt}}
\pgfusepath{stroke}
\pgfpathmoveto{\pgfpoint{244.506226pt}{42.584846pt}}
\pgflineto{\pgfpoint{244.960022pt}{39.255608pt}}
\pgfusepath{stroke}
\pgfpathmoveto{\pgfpoint{244.960022pt}{40.726631pt}}
\pgflineto{\pgfpoint{244.506226pt}{42.584846pt}}
\pgfusepath{stroke}
\pgfpathmoveto{\pgfpoint{249.920013pt}{116.421631pt}}
\pgflineto{\pgfpoint{244.960022pt}{40.726631pt}}
\pgfusepath{stroke}
\pgfpathmoveto{\pgfpoint{254.880020pt}{215.171143pt}}
\pgflineto{\pgfpoint{249.920013pt}{116.421631pt}}
\pgfusepath{stroke}
\pgfpathmoveto{\pgfpoint{259.840027pt}{141.531891pt}}
\pgflineto{\pgfpoint{254.880020pt}{215.171143pt}}
\pgfusepath{stroke}
\pgfpathmoveto{\pgfpoint{264.800018pt}{190.998230pt}}
\pgflineto{\pgfpoint{259.840027pt}{141.531891pt}}
\pgfusepath{stroke}
\pgfpathmoveto{\pgfpoint{269.487213pt}{69.737061pt}}
\pgflineto{\pgfpoint{264.800018pt}{190.998230pt}}
\pgfusepath{stroke}
\pgfpathmoveto{\pgfpoint{240.000000pt}{61.037575pt}}
\pgflineto{\pgfpoint{244.506226pt}{42.584846pt}}
\pgfusepath{stroke}
\pgfpathmoveto{\pgfpoint{235.040024pt}{43.997490pt}}
\pgflineto{\pgfpoint{240.000000pt}{61.037575pt}}
\pgfusepath{stroke}
\pgfpathmoveto{\pgfpoint{231.822998pt}{62.078987pt}}
\pgflineto{\pgfpoint{235.040024pt}{43.997490pt}}
\pgfusepath{stroke}
\pgfpathmoveto{\pgfpoint{235.040024pt}{53.250259pt}}
\pgflineto{\pgfpoint{231.822998pt}{62.078987pt}}
\pgfusepath{stroke}
\pgfpathmoveto{\pgfpoint{240.000000pt}{75.645462pt}}
\pgflineto{\pgfpoint{235.040024pt}{53.250259pt}}
\pgfusepath{stroke}
\pgfpathmoveto{\pgfpoint{244.506226pt}{42.584846pt}}
\pgflineto{\pgfpoint{240.000000pt}{75.645462pt}}
\pgfusepath{stroke}
\pgfpathmoveto{\pgfpoint{230.080017pt}{66.862328pt}}
\pgflineto{\pgfpoint{231.822998pt}{62.078987pt}}
\pgfusepath{stroke}
\pgfpathmoveto{\pgfpoint{228.633514pt}{65.886925pt}}
\pgflineto{\pgfpoint{230.080017pt}{66.862328pt}}
\pgfusepath{stroke}
\pgfpathmoveto{\pgfpoint{230.080017pt}{71.875427pt}}
\pgflineto{\pgfpoint{228.633514pt}{65.886925pt}}
\pgfusepath{stroke}
\pgfpathmoveto{\pgfpoint{231.822998pt}{62.078987pt}}
\pgflineto{\pgfpoint{230.080017pt}{71.875427pt}}
\pgfusepath{stroke}
\pgfpathmoveto{\pgfpoint{225.120026pt}{51.340904pt}}
\pgflineto{\pgfpoint{228.633514pt}{65.886925pt}}
\pgfusepath{stroke}
\pgfpathmoveto{\pgfpoint{220.160019pt}{48.298035pt}}
\pgflineto{\pgfpoint{225.120026pt}{51.340904pt}}
\pgfusepath{stroke}
\pgfpathmoveto{\pgfpoint{215.200012pt}{36.115891pt}}
\pgflineto{\pgfpoint{220.160019pt}{48.298035pt}}
\pgfusepath{stroke}
\pgfpathmoveto{\pgfpoint{210.240021pt}{64.267227pt}}
\pgflineto{\pgfpoint{215.200012pt}{36.115891pt}}
\pgfusepath{stroke}
\pgfpathmoveto{\pgfpoint{210.147217pt}{64.919937pt}}
\pgflineto{\pgfpoint{210.240021pt}{64.267227pt}}
\pgfusepath{stroke}
\pgfpathmoveto{\pgfpoint{210.240021pt}{64.683121pt}}
\pgflineto{\pgfpoint{210.147217pt}{64.919937pt}}
\pgfusepath{stroke}
\pgfpathmoveto{\pgfpoint{215.200012pt}{68.944382pt}}
\pgflineto{\pgfpoint{210.240021pt}{64.683121pt}}
\pgfusepath{stroke}
\pgfpathmoveto{\pgfpoint{220.160019pt}{56.577995pt}}
\pgflineto{\pgfpoint{215.200012pt}{68.944382pt}}
\pgfusepath{stroke}
\pgfpathmoveto{\pgfpoint{225.120026pt}{63.517673pt}}
\pgflineto{\pgfpoint{220.160019pt}{56.577995pt}}
\pgfusepath{stroke}
\pgfpathmoveto{\pgfpoint{228.633514pt}{65.886925pt}}
\pgflineto{\pgfpoint{225.120026pt}{63.517673pt}}
\pgfusepath{stroke}
\pgfpathmoveto{\pgfpoint{205.279999pt}{77.342728pt}}
\pgflineto{\pgfpoint{210.147217pt}{64.919937pt}}
\pgfusepath{stroke}
\pgfpathmoveto{\pgfpoint{200.320007pt}{111.419617pt}}
\pgflineto{\pgfpoint{205.279999pt}{77.342728pt}}
\pgfusepath{stroke}
\pgfpathmoveto{\pgfpoint{195.360016pt}{39.186867pt}}
\pgflineto{\pgfpoint{200.320007pt}{111.419617pt}}
\pgfusepath{stroke}
\pgfpathmoveto{\pgfpoint{194.088348pt}{48.193687pt}}
\pgflineto{\pgfpoint{195.360016pt}{39.186867pt}}
\pgfusepath{stroke}
\pgfpathmoveto{\pgfpoint{195.360016pt}{50.795929pt}}
\pgflineto{\pgfpoint{194.088348pt}{48.193687pt}}
\pgfusepath{stroke}
\pgfpathmoveto{\pgfpoint{200.320007pt}{162.896408pt}}
\pgflineto{\pgfpoint{195.360016pt}{50.795929pt}}
\pgfusepath{stroke}
\pgfpathmoveto{\pgfpoint{205.279999pt}{99.158791pt}}
\pgflineto{\pgfpoint{200.320007pt}{162.896408pt}}
\pgfusepath{stroke}
\pgfpathmoveto{\pgfpoint{210.147217pt}{64.919937pt}}
\pgflineto{\pgfpoint{205.279999pt}{99.158791pt}}
\pgfusepath{stroke}
\pgfpathmoveto{\pgfpoint{190.400024pt}{40.646240pt}}
\pgflineto{\pgfpoint{194.088348pt}{48.193687pt}}
\pgfusepath{stroke}
\pgfpathmoveto{\pgfpoint{185.440018pt}{55.456352pt}}
\pgflineto{\pgfpoint{190.400024pt}{40.646240pt}}
\pgfusepath{stroke}
\pgfpathmoveto{\pgfpoint{182.813766pt}{65.866241pt}}
\pgflineto{\pgfpoint{185.440018pt}{55.456352pt}}
\pgfusepath{stroke}
\pgfpathmoveto{\pgfpoint{185.440018pt}{94.016953pt}}
\pgflineto{\pgfpoint{182.813766pt}{65.866241pt}}
\pgfusepath{stroke}
\pgfpathmoveto{\pgfpoint{190.400024pt}{74.316742pt}}
\pgflineto{\pgfpoint{185.440018pt}{94.016953pt}}
\pgfusepath{stroke}
\pgfpathmoveto{\pgfpoint{194.088348pt}{48.193687pt}}
\pgflineto{\pgfpoint{190.400024pt}{74.316742pt}}
\pgfusepath{stroke}
\pgfpathmoveto{\pgfpoint{180.480011pt}{40.850761pt}}
\pgflineto{\pgfpoint{182.813766pt}{65.866241pt}}
\pgfusepath{stroke}
\pgfpathmoveto{\pgfpoint{175.520004pt}{43.668625pt}}
\pgflineto{\pgfpoint{180.480011pt}{40.850761pt}}
\pgfusepath{stroke}
\pgfpathmoveto{\pgfpoint{174.223160pt}{57.045738pt}}
\pgflineto{\pgfpoint{175.520004pt}{43.668625pt}}
\pgfusepath{stroke}
\pgfpathmoveto{\pgfpoint{175.520004pt}{60.118607pt}}
\pgflineto{\pgfpoint{174.223160pt}{57.045738pt}}
\pgfusepath{stroke}
\pgfpathmoveto{\pgfpoint{180.480011pt}{75.116745pt}}
\pgflineto{\pgfpoint{175.520004pt}{60.118607pt}}
\pgfusepath{stroke}
\pgfpathmoveto{\pgfpoint{182.813766pt}{65.866241pt}}
\pgflineto{\pgfpoint{180.480011pt}{75.116745pt}}
\pgfusepath{stroke}
\pgfpathmoveto{\pgfpoint{170.560013pt}{48.365959pt}}
\pgflineto{\pgfpoint{174.223160pt}{57.045738pt}}
\pgfusepath{stroke}
\pgfpathmoveto{\pgfpoint{166.011597pt}{63.599285pt}}
\pgflineto{\pgfpoint{170.560013pt}{48.365959pt}}
\pgfusepath{stroke}
\pgfpathmoveto{\pgfpoint{170.560013pt}{94.831261pt}}
\pgflineto{\pgfpoint{166.011597pt}{63.599285pt}}
\pgfusepath{stroke}
\pgfpathmoveto{\pgfpoint{174.223160pt}{57.045738pt}}
\pgflineto{\pgfpoint{170.560013pt}{94.831261pt}}
\pgfusepath{stroke}
\pgfpathmoveto{\pgfpoint{165.600006pt}{60.773132pt}}
\pgflineto{\pgfpoint{166.011597pt}{63.599285pt}}
\pgfusepath{stroke}
\pgfpathmoveto{\pgfpoint{165.274719pt}{64.387100pt}}
\pgflineto{\pgfpoint{165.600006pt}{60.773132pt}}
\pgfusepath{stroke}
\pgfpathmoveto{\pgfpoint{165.600006pt}{64.977737pt}}
\pgflineto{\pgfpoint{165.274719pt}{64.387100pt}}
\pgfusepath{stroke}
\pgfpathmoveto{\pgfpoint{166.011597pt}{63.599285pt}}
\pgflineto{\pgfpoint{165.600006pt}{64.977737pt}}
\pgfusepath{stroke}
\pgfpathmoveto{\pgfpoint{160.640015pt}{55.971481pt}}
\pgflineto{\pgfpoint{165.274719pt}{64.387100pt}}
\pgfusepath{stroke}
\pgfpathmoveto{\pgfpoint{158.851837pt}{88.183441pt}}
\pgflineto{\pgfpoint{160.640015pt}{55.971481pt}}
\pgfusepath{stroke}
\pgfpathmoveto{\pgfpoint{160.640015pt}{115.879601pt}}
\pgflineto{\pgfpoint{158.851837pt}{88.183441pt}}
\pgfusepath{stroke}
\pgfpathmoveto{\pgfpoint{165.274719pt}{64.387100pt}}
\pgflineto{\pgfpoint{160.640015pt}{115.879601pt}}
\pgfusepath{stroke}
\pgfpathmoveto{\pgfpoint{155.680023pt}{39.056114pt}}
\pgflineto{\pgfpoint{158.851837pt}{88.183441pt}}
\pgfusepath{stroke}
\pgfpathmoveto{\pgfpoint{150.720016pt}{56.516876pt}}
\pgflineto{\pgfpoint{155.680023pt}{39.056114pt}}
\pgfusepath{stroke}
\pgfpathmoveto{\pgfpoint{148.486938pt}{53.263680pt}}
\pgflineto{\pgfpoint{150.720016pt}{56.516876pt}}
\pgfusepath{stroke}
\pgfpathmoveto{\pgfpoint{150.720016pt}{59.958729pt}}
\pgflineto{\pgfpoint{148.486938pt}{53.263680pt}}
\pgfusepath{stroke}
\pgfpathmoveto{\pgfpoint{155.680023pt}{145.320908pt}}
\pgflineto{\pgfpoint{150.720016pt}{59.958729pt}}
\pgfusepath{stroke}
\pgfpathmoveto{\pgfpoint{158.851837pt}{88.183441pt}}
\pgflineto{\pgfpoint{155.680023pt}{145.320908pt}}
\pgfusepath{stroke}
\pgfpathmoveto{\pgfpoint{145.760010pt}{45.087975pt}}
\pgflineto{\pgfpoint{148.486938pt}{53.263680pt}}
\pgfusepath{stroke}
\pgfpathmoveto{\pgfpoint{140.800003pt}{63.190159pt}}
\pgflineto{\pgfpoint{145.760010pt}{45.087975pt}}
\pgfusepath{stroke}
\pgfpathmoveto{\pgfpoint{138.230530pt}{66.341614pt}}
\pgflineto{\pgfpoint{140.800003pt}{63.190159pt}}
\pgfusepath{stroke}
\pgfpathmoveto{\pgfpoint{140.800003pt}{97.846222pt}}
\pgflineto{\pgfpoint{138.230530pt}{66.341614pt}}
\pgfusepath{stroke}
\pgfpathmoveto{\pgfpoint{145.760010pt}{49.291016pt}}
\pgflineto{\pgfpoint{140.800003pt}{97.846222pt}}
\pgfusepath{stroke}
\pgfpathmoveto{\pgfpoint{148.486938pt}{53.263680pt}}
\pgflineto{\pgfpoint{145.760010pt}{49.291016pt}}
\pgfusepath{stroke}
\pgfpathmoveto{\pgfpoint{135.840012pt}{37.031158pt}}
\pgflineto{\pgfpoint{138.230530pt}{66.341614pt}}
\pgfusepath{stroke}
\pgfpathmoveto{\pgfpoint{132.276794pt}{53.129593pt}}
\pgflineto{\pgfpoint{135.840012pt}{37.031158pt}}
\pgfusepath{stroke}
\pgfpathmoveto{\pgfpoint{135.840012pt}{69.273598pt}}
\pgflineto{\pgfpoint{132.276794pt}{53.129593pt}}
\pgfusepath{stroke}
\pgfpathmoveto{\pgfpoint{138.230530pt}{66.341614pt}}
\pgflineto{\pgfpoint{135.840012pt}{69.273598pt}}
\pgfusepath{stroke}
\pgfpathmoveto{\pgfpoint{130.880005pt}{46.801132pt}}
\pgflineto{\pgfpoint{132.276794pt}{53.129593pt}}
\pgfusepath{stroke}
\pgfpathmoveto{\pgfpoint{127.352821pt}{64.771484pt}}
\pgflineto{\pgfpoint{130.880005pt}{46.801132pt}}
\pgfusepath{stroke}
\pgfpathmoveto{\pgfpoint{130.880005pt}{59.440186pt}}
\pgflineto{\pgfpoint{127.352821pt}{64.771484pt}}
\pgfusepath{stroke}
\pgfpathmoveto{\pgfpoint{132.276794pt}{53.129593pt}}
\pgflineto{\pgfpoint{130.880005pt}{59.440186pt}}
\pgfusepath{stroke}
\pgfpathmoveto{\pgfpoint{125.920013pt}{66.937149pt}}
\pgflineto{\pgfpoint{127.352821pt}{64.771484pt}}
\pgfusepath{stroke}
\pgfpathmoveto{\pgfpoint{120.960007pt}{115.830338pt}}
\pgflineto{\pgfpoint{125.920013pt}{66.937149pt}}
\pgfusepath{stroke}
\pgfpathmoveto{\pgfpoint{116.000015pt}{40.934975pt}}
\pgflineto{\pgfpoint{120.960007pt}{115.830338pt}}
\pgfusepath{stroke}
\pgfpathmoveto{\pgfpoint{111.040009pt}{48.803902pt}}
\pgflineto{\pgfpoint{116.000015pt}{40.934975pt}}
\pgfusepath{stroke}
\pgfpathmoveto{\pgfpoint{106.080017pt}{57.417770pt}}
\pgflineto{\pgfpoint{111.040009pt}{48.803902pt}}
\pgfusepath{stroke}
\pgfpathmoveto{\pgfpoint{101.120010pt}{36.805199pt}}
\pgflineto{\pgfpoint{106.080017pt}{57.417770pt}}
\pgfusepath{stroke}
\pgfpathmoveto{\pgfpoint{96.160011pt}{39.422333pt}}
\pgflineto{\pgfpoint{101.120010pt}{36.805199pt}}
\pgfusepath{stroke}
\pgfpathmoveto{\pgfpoint{91.200012pt}{67.963776pt}}
\pgflineto{\pgfpoint{96.160011pt}{39.422333pt}}
\pgfusepath{stroke}
\pgfpathmoveto{\pgfpoint{86.240013pt}{60.315250pt}}
\pgflineto{\pgfpoint{91.200012pt}{67.963776pt}}
\pgfusepath{stroke}
\pgfpathmoveto{\pgfpoint{85.898277pt}{60.412163pt}}
\pgflineto{\pgfpoint{86.240013pt}{60.315250pt}}
\pgfusepath{stroke}
\pgfpathmoveto{\pgfpoint{86.240013pt}{61.582748pt}}
\pgflineto{\pgfpoint{85.898277pt}{60.412163pt}}
\pgfusepath{stroke}
\pgfpathmoveto{\pgfpoint{91.200012pt}{82.405769pt}}
\pgflineto{\pgfpoint{86.240013pt}{61.582748pt}}
\pgfusepath{stroke}
\pgfpathmoveto{\pgfpoint{96.160011pt}{121.447739pt}}
\pgflineto{\pgfpoint{91.200012pt}{82.405769pt}}
\pgfusepath{stroke}
\pgfpathmoveto{\pgfpoint{101.120010pt}{45.170807pt}}
\pgflineto{\pgfpoint{96.160011pt}{121.447739pt}}
\pgfusepath{stroke}
\pgfpathmoveto{\pgfpoint{106.080017pt}{93.045525pt}}
\pgflineto{\pgfpoint{101.120010pt}{45.170807pt}}
\pgfusepath{stroke}
\pgfpathmoveto{\pgfpoint{111.040009pt}{89.065750pt}}
\pgflineto{\pgfpoint{106.080017pt}{93.045525pt}}
\pgfusepath{stroke}
\pgfpathmoveto{\pgfpoint{116.000015pt}{43.683624pt}}
\pgflineto{\pgfpoint{111.040009pt}{89.065750pt}}
\pgfusepath{stroke}
\pgfpathmoveto{\pgfpoint{120.960007pt}{149.515289pt}}
\pgflineto{\pgfpoint{116.000015pt}{43.683624pt}}
\pgfusepath{stroke}
\pgfpathmoveto{\pgfpoint{125.920013pt}{72.071335pt}}
\pgflineto{\pgfpoint{120.960007pt}{149.515289pt}}
\pgfusepath{stroke}
\pgfpathmoveto{\pgfpoint{127.352821pt}{64.771484pt}}
\pgflineto{\pgfpoint{125.920013pt}{72.071335pt}}
\pgfusepath{stroke}
\pgfpathmoveto{\pgfpoint{81.280014pt}{44.592613pt}}
\pgflineto{\pgfpoint{85.898277pt}{60.412163pt}}
\pgfusepath{stroke}
\pgfpathmoveto{\pgfpoint{80.181015pt}{65.915634pt}}
\pgflineto{\pgfpoint{81.280014pt}{44.592613pt}}
\pgfusepath{stroke}
\pgfpathmoveto{\pgfpoint{81.280014pt}{61.721870pt}}
\pgflineto{\pgfpoint{80.181015pt}{65.915634pt}}
\pgfusepath{stroke}
\pgfpathmoveto{\pgfpoint{85.898277pt}{60.412163pt}}
\pgflineto{\pgfpoint{81.280014pt}{61.721870pt}}
\pgfusepath{stroke}
\pgfpathmoveto{\pgfpoint{76.320007pt}{80.649338pt}}
\pgflineto{\pgfpoint{80.181015pt}{65.915634pt}}
\pgfusepath{stroke}
\pgfpathmoveto{\pgfpoint{73.884773pt}{93.204117pt}}
\pgflineto{\pgfpoint{76.320007pt}{80.649338pt}}
\pgfusepath{stroke}
\pgfpathmoveto{\pgfpoint{76.320007pt}{140.828415pt}}
\pgflineto{\pgfpoint{73.884773pt}{93.204117pt}}
\pgfusepath{stroke}
\pgfpathmoveto{\pgfpoint{80.181015pt}{65.915634pt}}
\pgflineto{\pgfpoint{76.320007pt}{140.828415pt}}
\pgfusepath{stroke}
\pgfpathmoveto{\pgfpoint{71.360008pt}{43.829178pt}}
\pgflineto{\pgfpoint{73.884773pt}{93.204117pt}}
\pgfusepath{stroke}
\pgfpathmoveto{\pgfpoint{69.454124pt}{89.661858pt}}
\pgflineto{\pgfpoint{71.360008pt}{43.829178pt}}
\pgfusepath{stroke}
\pgfpathmoveto{\pgfpoint{71.360008pt}{106.220398pt}}
\pgflineto{\pgfpoint{69.454124pt}{89.661858pt}}
\pgfusepath{stroke}
\pgfpathmoveto{\pgfpoint{73.884773pt}{93.204117pt}}
\pgflineto{\pgfpoint{71.360008pt}{106.220398pt}}
\pgfusepath{stroke}
\pgfpathmoveto{\pgfpoint{66.400009pt}{63.127323pt}}
\pgflineto{\pgfpoint{69.454124pt}{89.661858pt}}
\pgfusepath{stroke}
\pgfpathmoveto{\pgfpoint{61.440010pt}{46.590111pt}}
\pgflineto{\pgfpoint{66.400009pt}{63.127323pt}}
\pgfusepath{stroke}
\pgfpathmoveto{\pgfpoint{56.480011pt}{42.665169pt}}
\pgflineto{\pgfpoint{61.440010pt}{46.590111pt}}
\pgfusepath{stroke}
\pgfpathmoveto{\pgfpoint{52.729080pt}{55.915833pt}}
\pgflineto{\pgfpoint{56.480011pt}{42.665169pt}}
\pgfusepath{stroke}
\pgfpathmoveto{\pgfpoint{56.480011pt}{59.932564pt}}
\pgflineto{\pgfpoint{52.729080pt}{55.915833pt}}
\pgfusepath{stroke}
\pgfpathmoveto{\pgfpoint{61.440010pt}{105.284721pt}}
\pgflineto{\pgfpoint{56.480011pt}{59.932564pt}}
\pgfusepath{stroke}
\pgfpathmoveto{\pgfpoint{66.400009pt}{163.107224pt}}
\pgflineto{\pgfpoint{61.440010pt}{105.284721pt}}
\pgfusepath{stroke}
\pgfpathmoveto{\pgfpoint{69.454124pt}{89.661858pt}}
\pgflineto{\pgfpoint{66.400009pt}{163.107224pt}}
\pgfusepath{stroke}
\pgfpathmoveto{\pgfpoint{51.520004pt}{54.621078pt}}
\pgflineto{\pgfpoint{52.729080pt}{55.915833pt}}
\pgfusepath{stroke}
\pgfpathmoveto{\pgfpoint{51.272934pt}{60.450993pt}}
\pgflineto{\pgfpoint{51.520004pt}{54.621078pt}}
\pgfusepath{stroke}
\pgfpathmoveto{\pgfpoint{51.520004pt}{60.187065pt}}
\pgflineto{\pgfpoint{51.272934pt}{60.450993pt}}
\pgfusepath{stroke}
\pgfpathmoveto{\pgfpoint{52.729080pt}{55.915833pt}}
\pgflineto{\pgfpoint{51.520004pt}{60.187065pt}}
\pgfusepath{stroke}
\pgfpathmoveto{\pgfpoint{46.560013pt}{65.485382pt}}
\pgflineto{\pgfpoint{51.272934pt}{60.450993pt}}
\pgfusepath{stroke}
\pgfpathmoveto{\pgfpoint{46.560013pt}{171.656006pt}}
\pgflineto{\pgfpoint{46.560013pt}{65.485382pt}}
\pgfusepath{stroke}
\pgfpathmoveto{\pgfpoint{51.272934pt}{60.450993pt}}
\pgflineto{\pgfpoint{46.560013pt}{171.656006pt}}
\pgfusepath{stroke}
\end{pgfpicture}
}}
\end{frame}

\only<article>{
  The choice of distance in this kind of algorithm is important,
  particularly for very high dimensions. For something like a
  spectrogram, one idea is look at the total area of the difference
  between two spectral lines. 
}

\begin{frame}
  \frametitle{The nearest neighbour algorithm}
  \only<article>{The nearest neighbour algorithm for classification (Alg.~\ref{alg:kNN-classify}) does not include any complicated learning. Given a training dataset $D$, it returns a classification decision for any new point $x$ by simply comparing it to its closest $k$ neighbours in the dataset. It then estimates the probability $p_y$ of each class $y$ by calculating the average number of times the neighbours take the class $y$.
  }
  \begin{algorithm}[H]
    \begin{algorithmic}[1]
      \State \textbf{Input} Data $D = \{(x_1, y_1), \ldots, (x_T, y_T)\}$, $k \geq 1$,  $d : \CX \times \CX \to \Reals_+$, new point $x \in \CX$
      \State $D = \texttt{Sort}(D, d)$ \% \textsf{ Sort $D$ so that $d(x, x_i) \leq d(x, x_{i+1})$}.
      \State $p_y = \sum_{i=1}^k \ind{y_i = y} / k$ for $y \in \CY$.
      \State \textbf{Return} $\argmax_y p_y$
    \end{algorithmic}
    \caption{$k$-NN Classify}
    \label{alg:kNN-classify}
  \end{algorithm}
  \begin{alertblock}{Algorithm parameters}
    \only<article>{In order to use the algorithm, we must specify some parameters, namely.}
    \begin{itemize}
    \item Neighbourhood $k \geq 1$. \only<article>{The number of neighbours to consider.}
    \item Distance $d : \CX \times \CX \to \Reals_+$. \only<article>{The function we use to determine what is a neighbour.}
    \end{itemize}
  \end{alertblock}
  \only<presentation>{
    What does the algorithm output when $k = T$?
  }
\end{frame}

\begin{frame}
  \frametitle{Nearest neighbour: What type is the new bacterium?}
  % Title: glps_renderer figure
% Creator: GL2PS 1.3.8, (C) 1999-2012 C. Geuzaine
% For: Octave
% CreationDate: Fri Jun 16 12:49:21 2017
\begin{pgfpicture}
\pgfsetlinewidth{0.01pt}
\color[rgb]{1.000000,1.000000,1.000000}
\pgfpathmoveto{\pgfpoint{45.000008pt}{222.000000pt}}
\pgflineto{\pgfpoint{289.600037pt}{26.399979pt}}
\pgflineto{\pgfpoint{45.000008pt}{26.399979pt}}
\pgfpathclose
\pgfusepath{fill,stroke}
\pgfpathmoveto{\pgfpoint{45.000008pt}{222.000000pt}}
\pgflineto{\pgfpoint{289.600037pt}{222.000000pt}}
\pgflineto{\pgfpoint{289.600037pt}{26.399979pt}}
\pgfpathclose
\pgfusepath{fill,stroke}
\pgfpathmoveto{\pgfpoint{260.624542pt}{220.474182pt}}
\pgflineto{\pgfpoint{288.074158pt}{197.039612pt}}
\pgflineto{\pgfpoint{260.624542pt}{197.039612pt}}
\pgfpathclose
\pgfusepath{fill,stroke}
\pgfpathmoveto{\pgfpoint{260.624542pt}{220.474182pt}}
\pgflineto{\pgfpoint{288.074158pt}{220.474182pt}}
\pgflineto{\pgfpoint{288.074158pt}{197.039612pt}}
\pgfpathclose
\pgfusepath{fill,stroke}
\color[rgb]{0.000000,0.000000,0.000000}
\pgfsetlinewidth{0.500000pt}
\pgfsetdash{{16pt}{0pt}}{0pt}
\pgfpathmoveto{\pgfpoint{289.600037pt}{26.399979pt}}
\pgflineto{\pgfpoint{45.000008pt}{26.399979pt}}
\pgfusepath{stroke}
\pgfpathmoveto{\pgfpoint{289.600037pt}{222.000000pt}}
\pgflineto{\pgfpoint{45.000008pt}{222.000000pt}}
\pgfusepath{stroke}
\pgfpathmoveto{\pgfpoint{45.000008pt}{222.000000pt}}
\pgflineto{\pgfpoint{45.000008pt}{26.399979pt}}
\pgfusepath{stroke}
\pgfpathmoveto{\pgfpoint{289.600037pt}{222.000000pt}}
\pgflineto{\pgfpoint{289.600037pt}{26.399979pt}}
\pgfusepath{stroke}
\pgfpathmoveto{\pgfpoint{45.000008pt}{28.840996pt}}
\pgflineto{\pgfpoint{45.000008pt}{26.399979pt}}
\pgfusepath{stroke}
\pgfpathmoveto{\pgfpoint{45.000008pt}{219.558990pt}}
\pgflineto{\pgfpoint{45.000008pt}{222.000000pt}}
\pgfusepath{stroke}
\pgfpathmoveto{\pgfpoint{93.920013pt}{28.840996pt}}
\pgflineto{\pgfpoint{93.920013pt}{26.399979pt}}
\pgfusepath{stroke}
\pgfpathmoveto{\pgfpoint{93.920013pt}{219.558990pt}}
\pgflineto{\pgfpoint{93.920013pt}{222.000000pt}}
\pgfusepath{stroke}
\pgfpathmoveto{\pgfpoint{142.840012pt}{28.840996pt}}
\pgflineto{\pgfpoint{142.840012pt}{26.399979pt}}
\pgfusepath{stroke}
\pgfpathmoveto{\pgfpoint{142.840012pt}{219.558990pt}}
\pgflineto{\pgfpoint{142.840012pt}{222.000000pt}}
\pgfusepath{stroke}
\pgfpathmoveto{\pgfpoint{191.760010pt}{28.840996pt}}
\pgflineto{\pgfpoint{191.760010pt}{26.399979pt}}
\pgfusepath{stroke}
\pgfpathmoveto{\pgfpoint{191.760010pt}{219.558990pt}}
\pgflineto{\pgfpoint{191.760010pt}{222.000000pt}}
\pgfusepath{stroke}
\pgfpathmoveto{\pgfpoint{240.680023pt}{28.840996pt}}
\pgflineto{\pgfpoint{240.680023pt}{26.399979pt}}
\pgfusepath{stroke}
\pgfpathmoveto{\pgfpoint{240.680023pt}{219.558990pt}}
\pgflineto{\pgfpoint{240.680023pt}{222.000000pt}}
\pgfusepath{stroke}
\pgfpathmoveto{\pgfpoint{289.600037pt}{28.840996pt}}
\pgflineto{\pgfpoint{289.600037pt}{26.399979pt}}
\pgfusepath{stroke}
\pgfpathmoveto{\pgfpoint{289.600037pt}{219.558990pt}}
\pgflineto{\pgfpoint{289.600037pt}{222.000000pt}}
\pgfusepath{stroke}
{
\pgftransformshift{\pgfpoint{45.000015pt}{21.410187pt}}
\pgfnode{rectangle}{north}{\fontsize{10}{0}\selectfont\textcolor[rgb]{0,0,0}{{-1.5e+08}}}{}{\pgfusepath{discard}}}
{
\pgftransformshift{\pgfpoint{93.920013pt}{21.410187pt}}
\pgfnode{rectangle}{north}{\fontsize{10}{0}\selectfont\textcolor[rgb]{0,0,0}{{-1e+08}}}{}{\pgfusepath{discard}}}
{
\pgftransformshift{\pgfpoint{142.840012pt}{21.410187pt}}
\pgfnode{rectangle}{north}{\fontsize{10}{0}\selectfont\textcolor[rgb]{0,0,0}{{-5e+07}}}{}{\pgfusepath{discard}}}
{
\pgftransformshift{\pgfpoint{191.760010pt}{21.410187pt}}
\pgfnode{rectangle}{north}{\fontsize{10}{0}\selectfont\textcolor[rgb]{0,0,0}{{0}}}{}{\pgfusepath{discard}}}
{
\pgftransformshift{\pgfpoint{240.680008pt}{21.410187pt}}
\pgfnode{rectangle}{north}{\fontsize{10}{0}\selectfont\textcolor[rgb]{0,0,0}{{5e+07}}}{}{\pgfusepath{discard}}}
{
\pgftransformshift{\pgfpoint{289.600006pt}{21.410187pt}}
\pgfnode{rectangle}{north}{\fontsize{10}{0}\selectfont\textcolor[rgb]{0,0,0}{{1e+08}}}{}{\pgfusepath{discard}}}
\pgfpathmoveto{\pgfpoint{47.442024pt}{26.399979pt}}
\pgflineto{\pgfpoint{45.000008pt}{26.399979pt}}
\pgfusepath{stroke}
\pgfpathmoveto{\pgfpoint{287.158020pt}{26.399979pt}}
\pgflineto{\pgfpoint{289.600037pt}{26.399979pt}}
\pgfusepath{stroke}
\pgfpathmoveto{\pgfpoint{47.442024pt}{75.299988pt}}
\pgflineto{\pgfpoint{45.000008pt}{75.299988pt}}
\pgfusepath{stroke}
\pgfpathmoveto{\pgfpoint{287.158020pt}{75.299988pt}}
\pgflineto{\pgfpoint{289.600037pt}{75.299988pt}}
\pgfusepath{stroke}
\pgfpathmoveto{\pgfpoint{47.442024pt}{124.199989pt}}
\pgflineto{\pgfpoint{45.000008pt}{124.199989pt}}
\pgfusepath{stroke}
\pgfpathmoveto{\pgfpoint{287.158020pt}{124.199989pt}}
\pgflineto{\pgfpoint{289.600037pt}{124.199989pt}}
\pgfusepath{stroke}
\pgfpathmoveto{\pgfpoint{47.442024pt}{173.099991pt}}
\pgflineto{\pgfpoint{45.000008pt}{173.099991pt}}
\pgfusepath{stroke}
\pgfpathmoveto{\pgfpoint{287.158020pt}{173.099991pt}}
\pgflineto{\pgfpoint{289.600037pt}{173.099991pt}}
\pgfusepath{stroke}
\pgfpathmoveto{\pgfpoint{47.442024pt}{222.000000pt}}
\pgflineto{\pgfpoint{45.000008pt}{222.000000pt}}
\pgfusepath{stroke}
\pgfpathmoveto{\pgfpoint{287.158020pt}{222.000000pt}}
\pgflineto{\pgfpoint{289.600037pt}{222.000000pt}}
\pgfusepath{stroke}
{
\pgftransformshift{\pgfpoint{40.008171pt}{26.399979pt}}
\pgfnode{rectangle}{east}{\fontsize{10}{0}\selectfont\textcolor[rgb]{0,0,0}{{-2e+08}}}{}{\pgfusepath{discard}}}
{
\pgftransformshift{\pgfpoint{40.008171pt}{75.299988pt}}
\pgfnode{rectangle}{east}{\fontsize{10}{0}\selectfont\textcolor[rgb]{0,0,0}{{-1e+08}}}{}{\pgfusepath{discard}}}
{
\pgftransformshift{\pgfpoint{40.008171pt}{124.199989pt}}
\pgfnode{rectangle}{east}{\fontsize{10}{0}\selectfont\textcolor[rgb]{0,0,0}{{0}}}{}{\pgfusepath{discard}}}
{
\pgftransformshift{\pgfpoint{40.008171pt}{173.099991pt}}
\pgfnode{rectangle}{east}{\fontsize{10}{0}\selectfont\textcolor[rgb]{0,0,0}{{1e+08}}}{}{\pgfusepath{discard}}}
{
\pgftransformshift{\pgfpoint{40.008171pt}{222.000000pt}}
\pgfnode{rectangle}{east}{\fontsize{10}{0}\selectfont\textcolor[rgb]{0,0,0}{{2e+08}}}{}{\pgfusepath{discard}}}
\color[rgb]{0.000000,0.000000,1.000000}
\pgfsetdash{}{0pt}
\pgfpathmoveto{\pgfpoint{196.935547pt}{119.434402pt}}
\pgflineto{\pgfpoint{190.935547pt}{119.434402pt}}
\pgfusepath{stroke}
\pgfpathmoveto{\pgfpoint{193.935547pt}{116.434402pt}}
\pgflineto{\pgfpoint{193.935547pt}{122.434402pt}}
\pgfusepath{stroke}
\pgfpathmoveto{\pgfpoint{199.886261pt}{119.242699pt}}
\pgflineto{\pgfpoint{193.886261pt}{119.242699pt}}
\pgfusepath{stroke}
\pgfpathmoveto{\pgfpoint{196.886261pt}{116.242699pt}}
\pgflineto{\pgfpoint{196.886261pt}{122.242699pt}}
\pgfusepath{stroke}
\pgfpathmoveto{\pgfpoint{183.982819pt}{117.223633pt}}
\pgflineto{\pgfpoint{177.982819pt}{117.223633pt}}
\pgfusepath{stroke}
\pgfpathmoveto{\pgfpoint{180.982819pt}{114.223633pt}}
\pgflineto{\pgfpoint{180.982819pt}{120.223633pt}}
\pgfusepath{stroke}
\pgfpathmoveto{\pgfpoint{179.547073pt}{115.611145pt}}
\pgflineto{\pgfpoint{173.547073pt}{115.611145pt}}
\pgfusepath{stroke}
\pgfpathmoveto{\pgfpoint{176.547073pt}{112.611145pt}}
\pgflineto{\pgfpoint{176.547073pt}{118.611145pt}}
\pgfusepath{stroke}
\pgfpathmoveto{\pgfpoint{190.868942pt}{106.227837pt}}
\pgflineto{\pgfpoint{184.868942pt}{106.227837pt}}
\pgfusepath{stroke}
\pgfpathmoveto{\pgfpoint{187.868942pt}{103.227837pt}}
\pgflineto{\pgfpoint{187.868942pt}{109.227837pt}}
\pgfusepath{stroke}
\pgfpathmoveto{\pgfpoint{201.759155pt}{109.677612pt}}
\pgflineto{\pgfpoint{195.759155pt}{109.677612pt}}
\pgfusepath{stroke}
\pgfpathmoveto{\pgfpoint{198.759155pt}{106.677612pt}}
\pgflineto{\pgfpoint{198.759155pt}{112.677612pt}}
\pgfusepath{stroke}
\pgfpathmoveto{\pgfpoint{183.758728pt}{108.592598pt}}
\pgflineto{\pgfpoint{177.758728pt}{108.592598pt}}
\pgfusepath{stroke}
\pgfpathmoveto{\pgfpoint{180.758728pt}{105.592598pt}}
\pgflineto{\pgfpoint{180.758728pt}{111.592606pt}}
\pgfusepath{stroke}
\pgfpathmoveto{\pgfpoint{112.574402pt}{75.370621pt}}
\pgflineto{\pgfpoint{106.574402pt}{75.370621pt}}
\pgfusepath{stroke}
\pgfpathmoveto{\pgfpoint{109.574402pt}{72.370621pt}}
\pgflineto{\pgfpoint{109.574402pt}{78.370621pt}}
\pgfusepath{stroke}
\pgfpathmoveto{\pgfpoint{174.193832pt}{101.355026pt}}
\pgflineto{\pgfpoint{168.193832pt}{101.355026pt}}
\pgfusepath{stroke}
\pgfpathmoveto{\pgfpoint{171.193832pt}{98.355026pt}}
\pgflineto{\pgfpoint{171.193832pt}{104.355026pt}}
\pgfusepath{stroke}
\pgfpathmoveto{\pgfpoint{182.152756pt}{112.984955pt}}
\pgflineto{\pgfpoint{176.152756pt}{112.984955pt}}
\pgfusepath{stroke}
\pgfpathmoveto{\pgfpoint{179.152756pt}{109.984955pt}}
\pgflineto{\pgfpoint{179.152756pt}{115.984955pt}}
\pgfusepath{stroke}
\pgfpathmoveto{\pgfpoint{168.108780pt}{110.069618pt}}
\pgflineto{\pgfpoint{162.108780pt}{110.069618pt}}
\pgfusepath{stroke}
\pgfpathmoveto{\pgfpoint{165.108780pt}{107.069618pt}}
\pgflineto{\pgfpoint{165.108780pt}{113.069618pt}}
\pgfusepath{stroke}
\pgfpathmoveto{\pgfpoint{201.415054pt}{114.212738pt}}
\pgflineto{\pgfpoint{195.415054pt}{114.212738pt}}
\pgfusepath{stroke}
\pgfpathmoveto{\pgfpoint{198.415054pt}{111.212738pt}}
\pgflineto{\pgfpoint{198.415054pt}{117.212738pt}}
\pgfusepath{stroke}
\pgfpathmoveto{\pgfpoint{192.300415pt}{121.396606pt}}
\pgflineto{\pgfpoint{186.300415pt}{121.396606pt}}
\pgfusepath{stroke}
\pgfpathmoveto{\pgfpoint{189.300415pt}{118.396606pt}}
\pgflineto{\pgfpoint{189.300415pt}{124.396606pt}}
\pgfusepath{stroke}
\pgfpathmoveto{\pgfpoint{193.598267pt}{124.001534pt}}
\pgflineto{\pgfpoint{187.598267pt}{124.001534pt}}
\pgfusepath{stroke}
\pgfpathmoveto{\pgfpoint{190.598267pt}{121.001534pt}}
\pgflineto{\pgfpoint{190.598267pt}{127.001534pt}}
\pgfusepath{stroke}
\pgfpathmoveto{\pgfpoint{180.524246pt}{109.700615pt}}
\pgflineto{\pgfpoint{174.524246pt}{109.700615pt}}
\pgfusepath{stroke}
\pgfpathmoveto{\pgfpoint{177.524246pt}{106.700615pt}}
\pgflineto{\pgfpoint{177.524246pt}{112.700615pt}}
\pgfusepath{stroke}
\pgfpathmoveto{\pgfpoint{176.954651pt}{103.460312pt}}
\pgflineto{\pgfpoint{170.954651pt}{103.460312pt}}
\pgfusepath{stroke}
\pgfpathmoveto{\pgfpoint{173.954651pt}{100.460312pt}}
\pgflineto{\pgfpoint{173.954651pt}{106.460312pt}}
\pgfusepath{stroke}
\pgfpathmoveto{\pgfpoint{185.662079pt}{110.839188pt}}
\pgflineto{\pgfpoint{179.662079pt}{110.839188pt}}
\pgfusepath{stroke}
\pgfpathmoveto{\pgfpoint{182.662079pt}{107.839188pt}}
\pgflineto{\pgfpoint{182.662079pt}{113.839188pt}}
\pgfusepath{stroke}
\pgfpathmoveto{\pgfpoint{148.840988pt}{87.041306pt}}
\pgflineto{\pgfpoint{142.840988pt}{87.041306pt}}
\pgfusepath{stroke}
\pgfpathmoveto{\pgfpoint{145.840988pt}{84.041306pt}}
\pgflineto{\pgfpoint{145.840988pt}{90.041306pt}}
\pgfusepath{stroke}
\pgfpathmoveto{\pgfpoint{149.264221pt}{74.474411pt}}
\pgflineto{\pgfpoint{143.264221pt}{74.474411pt}}
\pgfusepath{stroke}
\pgfpathmoveto{\pgfpoint{146.264221pt}{71.474411pt}}
\pgflineto{\pgfpoint{146.264221pt}{77.474411pt}}
\pgfusepath{stroke}
\pgfpathmoveto{\pgfpoint{208.496521pt}{104.231644pt}}
\pgflineto{\pgfpoint{202.496521pt}{104.231644pt}}
\pgfusepath{stroke}
\pgfpathmoveto{\pgfpoint{205.496521pt}{101.231644pt}}
\pgflineto{\pgfpoint{205.496521pt}{107.231644pt}}
\pgfusepath{stroke}
\pgfpathmoveto{\pgfpoint{171.510223pt}{118.414749pt}}
\pgflineto{\pgfpoint{165.510223pt}{118.414749pt}}
\pgfusepath{stroke}
\pgfpathmoveto{\pgfpoint{168.510223pt}{115.414749pt}}
\pgflineto{\pgfpoint{168.510223pt}{121.414749pt}}
\pgfusepath{stroke}
\pgfpathmoveto{\pgfpoint{179.501907pt}{118.791878pt}}
\pgflineto{\pgfpoint{173.501907pt}{118.791878pt}}
\pgfusepath{stroke}
\pgfpathmoveto{\pgfpoint{176.501907pt}{115.791878pt}}
\pgflineto{\pgfpoint{176.501907pt}{121.791878pt}}
\pgfusepath{stroke}
\pgfpathmoveto{\pgfpoint{184.015793pt}{119.659134pt}}
\pgflineto{\pgfpoint{178.015793pt}{119.659134pt}}
\pgfusepath{stroke}
\pgfpathmoveto{\pgfpoint{181.015793pt}{116.659134pt}}
\pgflineto{\pgfpoint{181.015793pt}{122.659142pt}}
\pgfusepath{stroke}
\pgfpathmoveto{\pgfpoint{170.695862pt}{117.468170pt}}
\pgflineto{\pgfpoint{164.695862pt}{117.468170pt}}
\pgfusepath{stroke}
\pgfpathmoveto{\pgfpoint{167.695862pt}{114.468163pt}}
\pgflineto{\pgfpoint{167.695862pt}{120.468170pt}}
\pgfusepath{stroke}
\pgfpathmoveto{\pgfpoint{196.609299pt}{118.015953pt}}
\pgflineto{\pgfpoint{190.609299pt}{118.015953pt}}
\pgfusepath{stroke}
\pgfpathmoveto{\pgfpoint{193.609299pt}{115.015953pt}}
\pgflineto{\pgfpoint{193.609299pt}{121.015953pt}}
\pgfusepath{stroke}
\pgfpathmoveto{\pgfpoint{197.507370pt}{115.068123pt}}
\pgflineto{\pgfpoint{191.507370pt}{115.068123pt}}
\pgfusepath{stroke}
\pgfpathmoveto{\pgfpoint{194.507370pt}{112.068123pt}}
\pgflineto{\pgfpoint{194.507370pt}{118.068123pt}}
\pgfusepath{stroke}
\pgfpathmoveto{\pgfpoint{193.406509pt}{114.448425pt}}
\pgflineto{\pgfpoint{187.406509pt}{114.448425pt}}
\pgfusepath{stroke}
\pgfpathmoveto{\pgfpoint{190.406509pt}{111.448425pt}}
\pgflineto{\pgfpoint{190.406509pt}{117.448425pt}}
\pgfusepath{stroke}
\pgfpathmoveto{\pgfpoint{194.718613pt}{114.180740pt}}
\pgflineto{\pgfpoint{188.718613pt}{114.180740pt}}
\pgfusepath{stroke}
\pgfpathmoveto{\pgfpoint{191.718613pt}{111.180733pt}}
\pgflineto{\pgfpoint{191.718613pt}{117.180740pt}}
\pgfusepath{stroke}
\pgfpathmoveto{\pgfpoint{202.550079pt}{120.974258pt}}
\pgflineto{\pgfpoint{196.550079pt}{120.974258pt}}
\pgfusepath{stroke}
\pgfpathmoveto{\pgfpoint{199.550079pt}{117.974258pt}}
\pgflineto{\pgfpoint{199.550079pt}{123.974258pt}}
\pgfusepath{stroke}
\pgfpathmoveto{\pgfpoint{194.683685pt}{124.179276pt}}
\pgflineto{\pgfpoint{188.683685pt}{124.179276pt}}
\pgfusepath{stroke}
\pgfpathmoveto{\pgfpoint{191.683685pt}{121.179276pt}}
\pgflineto{\pgfpoint{191.683685pt}{127.179276pt}}
\pgfusepath{stroke}
\pgfpathmoveto{\pgfpoint{205.879364pt}{112.132126pt}}
\pgflineto{\pgfpoint{199.879364pt}{112.132126pt}}
\pgfusepath{stroke}
\pgfpathmoveto{\pgfpoint{202.879364pt}{109.132126pt}}
\pgflineto{\pgfpoint{202.879364pt}{115.132133pt}}
\pgfusepath{stroke}
\pgfpathmoveto{\pgfpoint{206.979767pt}{90.041245pt}}
\pgflineto{\pgfpoint{200.979767pt}{90.041245pt}}
\pgfusepath{stroke}
\pgfpathmoveto{\pgfpoint{203.979767pt}{87.041245pt}}
\pgflineto{\pgfpoint{203.979767pt}{93.041245pt}}
\pgfusepath{stroke}
\pgfpathmoveto{\pgfpoint{196.589050pt}{123.302505pt}}
\pgflineto{\pgfpoint{190.589050pt}{123.302505pt}}
\pgfusepath{stroke}
\pgfpathmoveto{\pgfpoint{193.589050pt}{120.302505pt}}
\pgflineto{\pgfpoint{193.589050pt}{126.302505pt}}
\pgfusepath{stroke}
\pgfpathmoveto{\pgfpoint{209.073334pt}{97.586403pt}}
\pgflineto{\pgfpoint{203.073318pt}{97.586403pt}}
\pgfusepath{stroke}
\pgfpathmoveto{\pgfpoint{206.073318pt}{94.586403pt}}
\pgflineto{\pgfpoint{206.073318pt}{100.586403pt}}
\pgfusepath{stroke}
\pgfpathmoveto{\pgfpoint{195.115631pt}{118.254936pt}}
\pgflineto{\pgfpoint{189.115631pt}{118.254936pt}}
\pgfusepath{stroke}
\pgfpathmoveto{\pgfpoint{192.115631pt}{115.254936pt}}
\pgflineto{\pgfpoint{192.115631pt}{121.254936pt}}
\pgfusepath{stroke}
\pgfpathmoveto{\pgfpoint{201.029053pt}{98.080841pt}}
\pgflineto{\pgfpoint{195.029053pt}{98.080841pt}}
\pgfusepath{stroke}
\pgfpathmoveto{\pgfpoint{198.029053pt}{95.080841pt}}
\pgflineto{\pgfpoint{198.029053pt}{101.080841pt}}
\pgfusepath{stroke}
\pgfpathmoveto{\pgfpoint{197.477219pt}{106.478531pt}}
\pgflineto{\pgfpoint{191.477219pt}{106.478531pt}}
\pgfusepath{stroke}
\pgfpathmoveto{\pgfpoint{194.477219pt}{103.478531pt}}
\pgflineto{\pgfpoint{194.477219pt}{109.478531pt}}
\pgfusepath{stroke}
\pgfpathmoveto{\pgfpoint{148.010254pt}{79.851120pt}}
\pgflineto{\pgfpoint{142.010254pt}{79.851120pt}}
\pgfusepath{stroke}
\pgfpathmoveto{\pgfpoint{145.010254pt}{76.851120pt}}
\pgflineto{\pgfpoint{145.010254pt}{82.851120pt}}
\pgfusepath{stroke}
\pgfpathmoveto{\pgfpoint{175.602463pt}{102.839371pt}}
\pgflineto{\pgfpoint{169.602463pt}{102.839371pt}}
\pgfusepath{stroke}
\pgfpathmoveto{\pgfpoint{172.602463pt}{99.839371pt}}
\pgflineto{\pgfpoint{172.602463pt}{105.839371pt}}
\pgfusepath{stroke}
\pgfpathmoveto{\pgfpoint{168.293350pt}{102.053452pt}}
\pgflineto{\pgfpoint{162.293350pt}{102.053452pt}}
\pgfusepath{stroke}
\pgfpathmoveto{\pgfpoint{165.293350pt}{99.053444pt}}
\pgflineto{\pgfpoint{165.293350pt}{105.053452pt}}
\pgfusepath{stroke}
\pgfpathmoveto{\pgfpoint{188.499710pt}{114.862000pt}}
\pgflineto{\pgfpoint{182.499710pt}{114.862000pt}}
\pgfusepath{stroke}
\pgfpathmoveto{\pgfpoint{185.499710pt}{111.862000pt}}
\pgflineto{\pgfpoint{185.499710pt}{117.862007pt}}
\pgfusepath{stroke}
\pgfpathmoveto{\pgfpoint{174.017639pt}{119.228752pt}}
\pgflineto{\pgfpoint{168.017639pt}{119.228752pt}}
\pgfusepath{stroke}
\pgfpathmoveto{\pgfpoint{171.017639pt}{116.228752pt}}
\pgflineto{\pgfpoint{171.017639pt}{122.228752pt}}
\pgfusepath{stroke}
\pgfpathmoveto{\pgfpoint{166.606094pt}{107.932907pt}}
\pgflineto{\pgfpoint{160.606094pt}{107.932907pt}}
\pgfusepath{stroke}
\pgfpathmoveto{\pgfpoint{163.606094pt}{104.932907pt}}
\pgflineto{\pgfpoint{163.606094pt}{110.932907pt}}
\pgfusepath{stroke}
\pgfpathmoveto{\pgfpoint{178.223022pt}{102.634354pt}}
\pgflineto{\pgfpoint{172.223022pt}{102.634354pt}}
\pgfusepath{stroke}
\pgfpathmoveto{\pgfpoint{175.223022pt}{99.634346pt}}
\pgflineto{\pgfpoint{175.223022pt}{105.634354pt}}
\pgfusepath{stroke}
\pgfpathmoveto{\pgfpoint{214.268692pt}{114.793839pt}}
\pgflineto{\pgfpoint{208.268692pt}{114.793839pt}}
\pgfusepath{stroke}
\pgfpathmoveto{\pgfpoint{211.268692pt}{111.793839pt}}
\pgflineto{\pgfpoint{211.268692pt}{117.793839pt}}
\pgfusepath{stroke}
\pgfpathmoveto{\pgfpoint{222.330811pt}{111.535378pt}}
\pgflineto{\pgfpoint{216.330811pt}{111.535378pt}}
\pgfusepath{stroke}
\pgfpathmoveto{\pgfpoint{219.330811pt}{108.535378pt}}
\pgflineto{\pgfpoint{219.330811pt}{114.535378pt}}
\pgfusepath{stroke}
\pgfpathmoveto{\pgfpoint{212.852081pt}{120.428650pt}}
\pgflineto{\pgfpoint{206.852081pt}{120.428650pt}}
\pgfusepath{stroke}
\pgfpathmoveto{\pgfpoint{209.852081pt}{117.428650pt}}
\pgflineto{\pgfpoint{209.852081pt}{123.428650pt}}
\pgfusepath{stroke}
\pgfpathmoveto{\pgfpoint{201.270599pt}{119.242432pt}}
\pgflineto{\pgfpoint{195.270584pt}{119.242432pt}}
\pgfusepath{stroke}
\pgfpathmoveto{\pgfpoint{198.270584pt}{116.242432pt}}
\pgflineto{\pgfpoint{198.270584pt}{122.242432pt}}
\pgfusepath{stroke}
\pgfpathmoveto{\pgfpoint{170.743149pt}{91.735825pt}}
\pgflineto{\pgfpoint{164.743149pt}{91.735825pt}}
\pgfusepath{stroke}
\pgfpathmoveto{\pgfpoint{167.743149pt}{88.735825pt}}
\pgflineto{\pgfpoint{167.743149pt}{94.735825pt}}
\pgfusepath{stroke}
\pgfpathmoveto{\pgfpoint{208.540649pt}{102.282532pt}}
\pgflineto{\pgfpoint{202.540634pt}{102.282532pt}}
\pgfusepath{stroke}
\pgfpathmoveto{\pgfpoint{205.540634pt}{99.282532pt}}
\pgflineto{\pgfpoint{205.540634pt}{105.282532pt}}
\pgfusepath{stroke}
\pgfpathmoveto{\pgfpoint{199.218811pt}{68.839760pt}}
\pgflineto{\pgfpoint{193.218811pt}{68.839760pt}}
\pgfusepath{stroke}
\pgfpathmoveto{\pgfpoint{196.218811pt}{65.839760pt}}
\pgflineto{\pgfpoint{196.218811pt}{71.839760pt}}
\pgfusepath{stroke}
\pgfpathmoveto{\pgfpoint{187.196762pt}{116.726044pt}}
\pgflineto{\pgfpoint{181.196762pt}{116.726044pt}}
\pgfusepath{stroke}
\pgfpathmoveto{\pgfpoint{184.196762pt}{113.726044pt}}
\pgflineto{\pgfpoint{184.196762pt}{119.726044pt}}
\pgfusepath{stroke}
\pgfpathmoveto{\pgfpoint{147.139359pt}{71.754990pt}}
\pgflineto{\pgfpoint{141.139359pt}{71.754990pt}}
\pgfusepath{stroke}
\pgfpathmoveto{\pgfpoint{144.139359pt}{68.754990pt}}
\pgflineto{\pgfpoint{144.139359pt}{74.754990pt}}
\pgfusepath{stroke}
\pgfpathmoveto{\pgfpoint{215.177032pt}{136.581879pt}}
\pgflineto{\pgfpoint{209.177032pt}{136.581879pt}}
\pgfusepath{stroke}
\pgfpathmoveto{\pgfpoint{212.177032pt}{133.581879pt}}
\pgflineto{\pgfpoint{212.177032pt}{139.581879pt}}
\pgfusepath{stroke}
\pgfpathmoveto{\pgfpoint{184.478500pt}{114.593369pt}}
\pgflineto{\pgfpoint{178.478500pt}{114.593369pt}}
\pgfusepath{stroke}
\pgfpathmoveto{\pgfpoint{181.478500pt}{111.593369pt}}
\pgflineto{\pgfpoint{181.478500pt}{117.593369pt}}
\pgfusepath{stroke}
\pgfpathmoveto{\pgfpoint{190.108887pt}{122.127441pt}}
\pgflineto{\pgfpoint{184.108887pt}{122.127441pt}}
\pgfusepath{stroke}
\pgfpathmoveto{\pgfpoint{187.108887pt}{119.127441pt}}
\pgflineto{\pgfpoint{187.108887pt}{125.127441pt}}
\pgfusepath{stroke}
\pgfpathmoveto{\pgfpoint{179.439209pt}{116.649216pt}}
\pgflineto{\pgfpoint{173.439209pt}{116.649216pt}}
\pgfusepath{stroke}
\pgfpathmoveto{\pgfpoint{176.439209pt}{113.649216pt}}
\pgflineto{\pgfpoint{176.439209pt}{119.649216pt}}
\pgfusepath{stroke}
\pgfpathmoveto{\pgfpoint{104.465897pt}{107.626564pt}}
\pgflineto{\pgfpoint{98.465889pt}{107.626564pt}}
\pgfusepath{stroke}
\pgfpathmoveto{\pgfpoint{101.465897pt}{104.626564pt}}
\pgflineto{\pgfpoint{101.465897pt}{110.626564pt}}
\pgfusepath{stroke}
\pgfpathmoveto{\pgfpoint{177.118210pt}{119.713600pt}}
\pgflineto{\pgfpoint{171.118210pt}{119.713600pt}}
\pgfusepath{stroke}
\pgfpathmoveto{\pgfpoint{174.118210pt}{116.713600pt}}
\pgflineto{\pgfpoint{174.118210pt}{122.713600pt}}
\pgfusepath{stroke}
\pgfpathmoveto{\pgfpoint{164.368271pt}{113.844521pt}}
\pgflineto{\pgfpoint{158.368271pt}{113.844521pt}}
\pgfusepath{stroke}
\pgfpathmoveto{\pgfpoint{161.368271pt}{110.844521pt}}
\pgflineto{\pgfpoint{161.368271pt}{116.844521pt}}
\pgfusepath{stroke}
\pgfpathmoveto{\pgfpoint{174.302505pt}{114.449341pt}}
\pgflineto{\pgfpoint{168.302505pt}{114.449341pt}}
\pgfusepath{stroke}
\pgfpathmoveto{\pgfpoint{171.302505pt}{111.449341pt}}
\pgflineto{\pgfpoint{171.302505pt}{117.449341pt}}
\pgfusepath{stroke}
\pgfpathmoveto{\pgfpoint{188.850342pt}{120.117348pt}}
\pgflineto{\pgfpoint{182.850342pt}{120.117348pt}}
\pgfusepath{stroke}
\pgfpathmoveto{\pgfpoint{185.850342pt}{117.117348pt}}
\pgflineto{\pgfpoint{185.850342pt}{123.117348pt}}
\pgfusepath{stroke}
\pgfpathmoveto{\pgfpoint{189.115555pt}{120.904907pt}}
\pgflineto{\pgfpoint{183.115555pt}{120.904907pt}}
\pgfusepath{stroke}
\pgfpathmoveto{\pgfpoint{186.115555pt}{117.904907pt}}
\pgflineto{\pgfpoint{186.115555pt}{123.904907pt}}
\pgfusepath{stroke}
\pgfpathmoveto{\pgfpoint{164.011871pt}{105.423592pt}}
\pgflineto{\pgfpoint{158.011871pt}{105.423592pt}}
\pgfusepath{stroke}
\pgfpathmoveto{\pgfpoint{161.011871pt}{102.423584pt}}
\pgflineto{\pgfpoint{161.011871pt}{108.423592pt}}
\pgfusepath{stroke}
\pgfpathmoveto{\pgfpoint{123.379929pt}{92.500244pt}}
\pgflineto{\pgfpoint{117.379929pt}{92.500244pt}}
\pgfusepath{stroke}
\pgfpathmoveto{\pgfpoint{120.379936pt}{89.500244pt}}
\pgflineto{\pgfpoint{120.379936pt}{95.500244pt}}
\pgfusepath{stroke}
\pgfpathmoveto{\pgfpoint{224.852325pt}{104.413704pt}}
\pgflineto{\pgfpoint{218.852325pt}{104.413704pt}}
\pgfusepath{stroke}
\pgfpathmoveto{\pgfpoint{221.852325pt}{101.413704pt}}
\pgflineto{\pgfpoint{221.852325pt}{107.413704pt}}
\pgfusepath{stroke}
\pgfpathmoveto{\pgfpoint{158.911072pt}{77.804001pt}}
\pgflineto{\pgfpoint{152.911072pt}{77.804001pt}}
\pgfusepath{stroke}
\pgfpathmoveto{\pgfpoint{155.911072pt}{74.804001pt}}
\pgflineto{\pgfpoint{155.911072pt}{80.804001pt}}
\pgfusepath{stroke}
\pgfpathmoveto{\pgfpoint{200.542542pt}{112.245041pt}}
\pgflineto{\pgfpoint{194.542542pt}{112.245041pt}}
\pgfusepath{stroke}
\pgfpathmoveto{\pgfpoint{197.542542pt}{109.245041pt}}
\pgflineto{\pgfpoint{197.542542pt}{115.245041pt}}
\pgfusepath{stroke}
\pgfpathmoveto{\pgfpoint{202.097809pt}{102.256493pt}}
\pgflineto{\pgfpoint{196.097809pt}{102.256493pt}}
\pgfusepath{stroke}
\pgfpathmoveto{\pgfpoint{199.097809pt}{99.256493pt}}
\pgflineto{\pgfpoint{199.097809pt}{105.256493pt}}
\pgfusepath{stroke}
\pgfpathmoveto{\pgfpoint{209.810730pt}{94.916466pt}}
\pgflineto{\pgfpoint{203.810730pt}{94.916466pt}}
\pgfusepath{stroke}
\pgfpathmoveto{\pgfpoint{206.810730pt}{91.916466pt}}
\pgflineto{\pgfpoint{206.810730pt}{97.916466pt}}
\pgfusepath{stroke}
\pgfpathmoveto{\pgfpoint{188.358261pt}{110.721474pt}}
\pgflineto{\pgfpoint{182.358261pt}{110.721474pt}}
\pgfusepath{stroke}
\pgfpathmoveto{\pgfpoint{185.358261pt}{107.721474pt}}
\pgflineto{\pgfpoint{185.358261pt}{113.721474pt}}
\pgfusepath{stroke}
\pgfpathmoveto{\pgfpoint{202.170609pt}{103.766815pt}}
\pgflineto{\pgfpoint{196.170609pt}{103.766815pt}}
\pgfusepath{stroke}
\pgfpathmoveto{\pgfpoint{199.170609pt}{100.766815pt}}
\pgflineto{\pgfpoint{199.170609pt}{106.766815pt}}
\pgfusepath{stroke}
\pgfpathmoveto{\pgfpoint{193.927582pt}{120.353355pt}}
\pgflineto{\pgfpoint{187.927582pt}{120.353355pt}}
\pgfusepath{stroke}
\pgfpathmoveto{\pgfpoint{190.927582pt}{117.353355pt}}
\pgflineto{\pgfpoint{190.927582pt}{123.353355pt}}
\pgfusepath{stroke}
\pgfpathmoveto{\pgfpoint{214.340698pt}{107.736900pt}}
\pgflineto{\pgfpoint{208.340698pt}{107.736900pt}}
\pgfusepath{stroke}
\pgfpathmoveto{\pgfpoint{211.340698pt}{104.736893pt}}
\pgflineto{\pgfpoint{211.340698pt}{110.736900pt}}
\pgfusepath{stroke}
\pgfpathmoveto{\pgfpoint{192.854523pt}{116.522400pt}}
\pgflineto{\pgfpoint{186.854523pt}{116.522400pt}}
\pgfusepath{stroke}
\pgfpathmoveto{\pgfpoint{189.854523pt}{113.522400pt}}
\pgflineto{\pgfpoint{189.854523pt}{119.522400pt}}
\pgfusepath{stroke}
\pgfpathmoveto{\pgfpoint{180.888123pt}{119.808556pt}}
\pgflineto{\pgfpoint{174.888123pt}{119.808556pt}}
\pgfusepath{stroke}
\pgfpathmoveto{\pgfpoint{177.888123pt}{116.808556pt}}
\pgflineto{\pgfpoint{177.888123pt}{122.808556pt}}
\pgfusepath{stroke}
\pgfpathmoveto{\pgfpoint{176.316513pt}{120.338478pt}}
\pgflineto{\pgfpoint{170.316513pt}{120.338478pt}}
\pgfusepath{stroke}
\pgfpathmoveto{\pgfpoint{173.316513pt}{117.338478pt}}
\pgflineto{\pgfpoint{173.316513pt}{123.338478pt}}
\pgfusepath{stroke}
\pgfpathmoveto{\pgfpoint{183.214355pt}{122.327942pt}}
\pgflineto{\pgfpoint{177.214355pt}{122.327942pt}}
\pgfusepath{stroke}
\pgfpathmoveto{\pgfpoint{180.214355pt}{119.327942pt}}
\pgflineto{\pgfpoint{180.214355pt}{125.327942pt}}
\pgfusepath{stroke}
\pgfpathmoveto{\pgfpoint{191.410797pt}{122.499756pt}}
\pgflineto{\pgfpoint{185.410797pt}{122.499756pt}}
\pgfusepath{stroke}
\pgfpathmoveto{\pgfpoint{188.410797pt}{119.499756pt}}
\pgflineto{\pgfpoint{188.410797pt}{125.499756pt}}
\pgfusepath{stroke}
\pgfpathmoveto{\pgfpoint{197.348083pt}{116.337822pt}}
\pgflineto{\pgfpoint{191.348083pt}{116.337822pt}}
\pgfusepath{stroke}
\pgfpathmoveto{\pgfpoint{194.348083pt}{113.337822pt}}
\pgflineto{\pgfpoint{194.348083pt}{119.337822pt}}
\pgfusepath{stroke}
\pgfpathmoveto{\pgfpoint{150.088013pt}{108.014786pt}}
\pgflineto{\pgfpoint{144.088013pt}{108.014786pt}}
\pgfusepath{stroke}
\pgfpathmoveto{\pgfpoint{147.088013pt}{105.014786pt}}
\pgflineto{\pgfpoint{147.088013pt}{111.014786pt}}
\pgfusepath{stroke}
\pgfpathmoveto{\pgfpoint{165.265259pt}{92.156952pt}}
\pgflineto{\pgfpoint{159.265259pt}{92.156952pt}}
\pgfusepath{stroke}
\pgfpathmoveto{\pgfpoint{162.265259pt}{89.156952pt}}
\pgflineto{\pgfpoint{162.265259pt}{95.156952pt}}
\pgfusepath{stroke}
\pgfpathmoveto{\pgfpoint{164.344528pt}{89.086136pt}}
\pgflineto{\pgfpoint{158.344528pt}{89.086136pt}}
\pgfusepath{stroke}
\pgfpathmoveto{\pgfpoint{161.344528pt}{86.086136pt}}
\pgflineto{\pgfpoint{161.344528pt}{92.086136pt}}
\pgfusepath{stroke}
\pgfpathmoveto{\pgfpoint{144.495544pt}{104.760262pt}}
\pgflineto{\pgfpoint{138.495544pt}{104.760262pt}}
\pgfusepath{stroke}
\pgfpathmoveto{\pgfpoint{141.495544pt}{101.760262pt}}
\pgflineto{\pgfpoint{141.495544pt}{107.760262pt}}
\pgfusepath{stroke}
\pgfpathmoveto{\pgfpoint{152.836807pt}{126.138474pt}}
\pgflineto{\pgfpoint{146.836807pt}{126.138474pt}}
\pgfusepath{stroke}
\pgfpathmoveto{\pgfpoint{149.836807pt}{123.138474pt}}
\pgflineto{\pgfpoint{149.836807pt}{129.138474pt}}
\pgfusepath{stroke}
\pgfpathmoveto{\pgfpoint{200.675446pt}{115.899910pt}}
\pgflineto{\pgfpoint{194.675446pt}{115.899910pt}}
\pgfusepath{stroke}
\pgfpathmoveto{\pgfpoint{197.675446pt}{112.899910pt}}
\pgflineto{\pgfpoint{197.675446pt}{118.899910pt}}
\pgfusepath{stroke}
\pgfpathmoveto{\pgfpoint{190.927689pt}{96.198410pt}}
\pgflineto{\pgfpoint{184.927689pt}{96.198410pt}}
\pgfusepath{stroke}
\pgfpathmoveto{\pgfpoint{187.927689pt}{93.198410pt}}
\pgflineto{\pgfpoint{187.927689pt}{99.198410pt}}
\pgfusepath{stroke}
\pgfpathmoveto{\pgfpoint{203.814697pt}{103.487259pt}}
\pgflineto{\pgfpoint{197.814697pt}{103.487259pt}}
\pgfusepath{stroke}
\pgfpathmoveto{\pgfpoint{200.814697pt}{100.487259pt}}
\pgflineto{\pgfpoint{200.814697pt}{106.487267pt}}
\pgfusepath{stroke}
\pgfpathmoveto{\pgfpoint{200.864105pt}{116.337234pt}}
\pgflineto{\pgfpoint{194.864105pt}{116.337234pt}}
\pgfusepath{stroke}
\pgfpathmoveto{\pgfpoint{197.864105pt}{113.337234pt}}
\pgflineto{\pgfpoint{197.864105pt}{119.337234pt}}
\pgfusepath{stroke}
\pgfpathmoveto{\pgfpoint{195.630478pt}{117.957169pt}}
\pgflineto{\pgfpoint{189.630478pt}{117.957169pt}}
\pgfusepath{stroke}
\pgfpathmoveto{\pgfpoint{192.630478pt}{114.957169pt}}
\pgflineto{\pgfpoint{192.630478pt}{120.957176pt}}
\pgfusepath{stroke}
\pgfpathmoveto{\pgfpoint{164.992645pt}{119.094711pt}}
\pgflineto{\pgfpoint{158.992645pt}{119.094711pt}}
\pgfusepath{stroke}
\pgfpathmoveto{\pgfpoint{161.992645pt}{116.094711pt}}
\pgflineto{\pgfpoint{161.992645pt}{122.094719pt}}
\pgfusepath{stroke}
\pgfpathmoveto{\pgfpoint{184.135727pt}{114.890152pt}}
\pgflineto{\pgfpoint{178.135727pt}{114.890152pt}}
\pgfusepath{stroke}
\pgfpathmoveto{\pgfpoint{181.135727pt}{111.890152pt}}
\pgflineto{\pgfpoint{181.135727pt}{117.890160pt}}
\pgfusepath{stroke}
\pgfpathmoveto{\pgfpoint{183.212982pt}{107.354736pt}}
\pgflineto{\pgfpoint{177.212982pt}{107.354736pt}}
\pgfusepath{stroke}
\pgfpathmoveto{\pgfpoint{180.212982pt}{104.354736pt}}
\pgflineto{\pgfpoint{180.212982pt}{110.354736pt}}
\pgfusepath{stroke}
\pgfpathmoveto{\pgfpoint{205.889694pt}{104.733597pt}}
\pgflineto{\pgfpoint{199.889694pt}{104.733597pt}}
\pgfusepath{stroke}
\pgfpathmoveto{\pgfpoint{202.889694pt}{101.733597pt}}
\pgflineto{\pgfpoint{202.889694pt}{107.733597pt}}
\pgfusepath{stroke}
\pgfpathmoveto{\pgfpoint{67.810738pt}{29.410484pt}}
\pgflineto{\pgfpoint{61.810730pt}{29.410484pt}}
\pgfusepath{stroke}
\pgfpathmoveto{\pgfpoint{64.810730pt}{26.410492pt}}
\pgflineto{\pgfpoint{64.810730pt}{32.410484pt}}
\pgfusepath{stroke}
\pgfpathmoveto{\pgfpoint{138.959503pt}{82.218102pt}}
\pgflineto{\pgfpoint{132.959503pt}{82.218102pt}}
\pgfusepath{stroke}
\pgfpathmoveto{\pgfpoint{135.959503pt}{79.218102pt}}
\pgflineto{\pgfpoint{135.959503pt}{85.218109pt}}
\pgfusepath{stroke}
\pgfpathmoveto{\pgfpoint{195.715820pt}{105.990662pt}}
\pgflineto{\pgfpoint{189.715820pt}{105.990662pt}}
\pgfusepath{stroke}
\pgfpathmoveto{\pgfpoint{192.715820pt}{102.990662pt}}
\pgflineto{\pgfpoint{192.715820pt}{108.990662pt}}
\pgfusepath{stroke}
\pgfpathmoveto{\pgfpoint{210.786179pt}{108.818954pt}}
\pgflineto{\pgfpoint{204.786179pt}{108.818954pt}}
\pgfusepath{stroke}
\pgfpathmoveto{\pgfpoint{207.786179pt}{105.818947pt}}
\pgflineto{\pgfpoint{207.786179pt}{111.818954pt}}
\pgfusepath{stroke}
\pgfpathmoveto{\pgfpoint{204.065964pt}{84.708786pt}}
\pgflineto{\pgfpoint{198.065964pt}{84.708786pt}}
\pgfusepath{stroke}
\pgfpathmoveto{\pgfpoint{201.065964pt}{81.708786pt}}
\pgflineto{\pgfpoint{201.065964pt}{87.708786pt}}
\pgfusepath{stroke}
\pgfpathmoveto{\pgfpoint{194.247559pt}{84.369720pt}}
\pgflineto{\pgfpoint{188.247559pt}{84.369720pt}}
\pgfusepath{stroke}
\pgfpathmoveto{\pgfpoint{191.247559pt}{81.369720pt}}
\pgflineto{\pgfpoint{191.247559pt}{87.369720pt}}
\pgfusepath{stroke}
\pgfpathmoveto{\pgfpoint{235.681610pt}{74.712463pt}}
\pgflineto{\pgfpoint{229.681595pt}{74.712463pt}}
\pgfusepath{stroke}
\pgfpathmoveto{\pgfpoint{232.681595pt}{71.712463pt}}
\pgflineto{\pgfpoint{232.681595pt}{77.712463pt}}
\pgfusepath{stroke}
\pgfpathmoveto{\pgfpoint{208.190735pt}{79.811440pt}}
\pgflineto{\pgfpoint{202.190735pt}{79.811440pt}}
\pgfusepath{stroke}
\pgfpathmoveto{\pgfpoint{205.190735pt}{76.811440pt}}
\pgflineto{\pgfpoint{205.190735pt}{82.811440pt}}
\pgfusepath{stroke}
\pgfpathmoveto{\pgfpoint{208.186905pt}{79.900948pt}}
\pgflineto{\pgfpoint{202.186890pt}{79.900948pt}}
\pgfusepath{stroke}
\pgfpathmoveto{\pgfpoint{205.186905pt}{76.900948pt}}
\pgflineto{\pgfpoint{205.186905pt}{82.900948pt}}
\pgfusepath{stroke}
\pgfpathmoveto{\pgfpoint{194.421799pt}{101.387360pt}}
\pgflineto{\pgfpoint{188.421799pt}{101.387360pt}}
\pgfusepath{stroke}
\pgfpathmoveto{\pgfpoint{191.421799pt}{98.387360pt}}
\pgflineto{\pgfpoint{191.421799pt}{104.387360pt}}
\pgfusepath{stroke}
\pgfpathmoveto{\pgfpoint{185.854950pt}{114.317383pt}}
\pgflineto{\pgfpoint{179.854950pt}{114.317383pt}}
\pgfusepath{stroke}
\pgfpathmoveto{\pgfpoint{182.854950pt}{111.317375pt}}
\pgflineto{\pgfpoint{182.854950pt}{117.317383pt}}
\pgfusepath{stroke}
\pgfpathmoveto{\pgfpoint{169.837845pt}{103.355423pt}}
\pgflineto{\pgfpoint{163.837845pt}{103.355423pt}}
\pgfusepath{stroke}
\pgfpathmoveto{\pgfpoint{166.837845pt}{100.355423pt}}
\pgflineto{\pgfpoint{166.837845pt}{106.355431pt}}
\pgfusepath{stroke}
\pgfpathmoveto{\pgfpoint{177.191116pt}{103.091187pt}}
\pgflineto{\pgfpoint{171.191116pt}{103.091187pt}}
\pgfusepath{stroke}
\pgfpathmoveto{\pgfpoint{174.191116pt}{100.091187pt}}
\pgflineto{\pgfpoint{174.191116pt}{106.091187pt}}
\pgfusepath{stroke}
\pgfpathmoveto{\pgfpoint{158.701706pt}{103.395920pt}}
\pgflineto{\pgfpoint{152.701706pt}{103.395920pt}}
\pgfusepath{stroke}
\pgfpathmoveto{\pgfpoint{155.701706pt}{100.395920pt}}
\pgflineto{\pgfpoint{155.701706pt}{106.395927pt}}
\pgfusepath{stroke}
\pgfpathmoveto{\pgfpoint{158.003036pt}{119.001534pt}}
\pgflineto{\pgfpoint{152.003036pt}{119.001534pt}}
\pgfusepath{stroke}
\pgfpathmoveto{\pgfpoint{155.003036pt}{116.001534pt}}
\pgflineto{\pgfpoint{155.003036pt}{122.001534pt}}
\pgfusepath{stroke}
\pgfpathmoveto{\pgfpoint{198.435318pt}{121.251968pt}}
\pgflineto{\pgfpoint{192.435318pt}{121.251968pt}}
\pgfusepath{stroke}
\pgfpathmoveto{\pgfpoint{195.435318pt}{118.251968pt}}
\pgflineto{\pgfpoint{195.435318pt}{124.251968pt}}
\pgfusepath{stroke}
\pgfpathmoveto{\pgfpoint{200.801666pt}{114.776733pt}}
\pgflineto{\pgfpoint{194.801666pt}{114.776733pt}}
\pgfusepath{stroke}
\pgfpathmoveto{\pgfpoint{197.801666pt}{111.776733pt}}
\pgflineto{\pgfpoint{197.801666pt}{117.776733pt}}
\pgfusepath{stroke}
\pgfpathmoveto{\pgfpoint{190.116608pt}{115.944687pt}}
\pgflineto{\pgfpoint{184.116608pt}{115.944687pt}}
\pgfusepath{stroke}
\pgfpathmoveto{\pgfpoint{187.116608pt}{112.944687pt}}
\pgflineto{\pgfpoint{187.116608pt}{118.944687pt}}
\pgfusepath{stroke}
\pgfpathmoveto{\pgfpoint{186.450836pt}{117.508553pt}}
\pgflineto{\pgfpoint{180.450836pt}{117.508553pt}}
\pgfusepath{stroke}
\pgfpathmoveto{\pgfpoint{183.450836pt}{114.508553pt}}
\pgflineto{\pgfpoint{183.450836pt}{120.508553pt}}
\pgfusepath{stroke}
\pgfpathmoveto{\pgfpoint{182.945129pt}{94.769684pt}}
\pgflineto{\pgfpoint{176.945129pt}{94.769684pt}}
\pgfusepath{stroke}
\pgfpathmoveto{\pgfpoint{179.945129pt}{91.769684pt}}
\pgflineto{\pgfpoint{179.945129pt}{97.769684pt}}
\pgfusepath{stroke}
\pgfpathmoveto{\pgfpoint{217.117340pt}{103.659065pt}}
\pgflineto{\pgfpoint{211.117340pt}{103.659065pt}}
\pgfusepath{stroke}
\pgfpathmoveto{\pgfpoint{214.117340pt}{100.659065pt}}
\pgflineto{\pgfpoint{214.117340pt}{106.659065pt}}
\pgfusepath{stroke}
\pgfpathmoveto{\pgfpoint{238.411362pt}{84.948723pt}}
\pgflineto{\pgfpoint{232.411346pt}{84.948723pt}}
\pgfusepath{stroke}
\pgfpathmoveto{\pgfpoint{235.411362pt}{81.948715pt}}
\pgflineto{\pgfpoint{235.411362pt}{87.948723pt}}
\pgfusepath{stroke}
\pgfpathmoveto{\pgfpoint{179.200348pt}{89.153976pt}}
\pgflineto{\pgfpoint{173.200348pt}{89.153976pt}}
\pgfusepath{stroke}
\pgfpathmoveto{\pgfpoint{176.200348pt}{86.153976pt}}
\pgflineto{\pgfpoint{176.200348pt}{92.153984pt}}
\pgfusepath{stroke}
\pgfpathmoveto{\pgfpoint{235.074829pt}{94.344696pt}}
\pgflineto{\pgfpoint{229.074829pt}{94.344696pt}}
\pgfusepath{stroke}
\pgfpathmoveto{\pgfpoint{232.074829pt}{91.344696pt}}
\pgflineto{\pgfpoint{232.074829pt}{97.344696pt}}
\pgfusepath{stroke}
\pgfpathmoveto{\pgfpoint{237.887268pt}{67.848312pt}}
\pgflineto{\pgfpoint{231.887253pt}{67.848312pt}}
\pgfusepath{stroke}
\pgfpathmoveto{\pgfpoint{234.887253pt}{64.848312pt}}
\pgflineto{\pgfpoint{234.887253pt}{70.848312pt}}
\pgfusepath{stroke}
\pgfpathmoveto{\pgfpoint{181.916992pt}{116.796455pt}}
\pgflineto{\pgfpoint{175.916992pt}{116.796455pt}}
\pgfusepath{stroke}
\pgfpathmoveto{\pgfpoint{178.916992pt}{113.796455pt}}
\pgflineto{\pgfpoint{178.916992pt}{119.796455pt}}
\pgfusepath{stroke}
\pgfpathmoveto{\pgfpoint{158.279373pt}{108.222610pt}}
\pgflineto{\pgfpoint{152.279373pt}{108.222610pt}}
\pgfusepath{stroke}
\pgfpathmoveto{\pgfpoint{155.279373pt}{105.222610pt}}
\pgflineto{\pgfpoint{155.279373pt}{111.222610pt}}
\pgfusepath{stroke}
\pgfpathmoveto{\pgfpoint{258.962585pt}{110.360901pt}}
\pgflineto{\pgfpoint{252.962585pt}{110.360901pt}}
\pgfusepath{stroke}
\pgfpathmoveto{\pgfpoint{255.962585pt}{107.360901pt}}
\pgflineto{\pgfpoint{255.962585pt}{113.360901pt}}
\pgfusepath{stroke}
\pgfpathmoveto{\pgfpoint{173.935364pt}{104.089211pt}}
\pgflineto{\pgfpoint{167.935364pt}{104.089211pt}}
\pgfusepath{stroke}
\pgfpathmoveto{\pgfpoint{170.935364pt}{101.089203pt}}
\pgflineto{\pgfpoint{170.935364pt}{107.089211pt}}
\pgfusepath{stroke}
\pgfpathmoveto{\pgfpoint{155.325623pt}{92.199280pt}}
\pgflineto{\pgfpoint{149.325623pt}{92.199280pt}}
\pgfusepath{stroke}
\pgfpathmoveto{\pgfpoint{152.325623pt}{89.199280pt}}
\pgflineto{\pgfpoint{152.325623pt}{95.199280pt}}
\pgfusepath{stroke}
\pgfpathmoveto{\pgfpoint{202.332703pt}{106.899551pt}}
\pgflineto{\pgfpoint{196.332703pt}{106.899551pt}}
\pgfusepath{stroke}
\pgfpathmoveto{\pgfpoint{199.332703pt}{103.899551pt}}
\pgflineto{\pgfpoint{199.332703pt}{109.899551pt}}
\pgfusepath{stroke}
\pgfpathmoveto{\pgfpoint{188.291809pt}{106.586739pt}}
\pgflineto{\pgfpoint{182.291809pt}{106.586739pt}}
\pgfusepath{stroke}
\pgfpathmoveto{\pgfpoint{185.291809pt}{103.586739pt}}
\pgflineto{\pgfpoint{185.291809pt}{109.586739pt}}
\pgfusepath{stroke}
\pgfpathmoveto{\pgfpoint{170.988876pt}{117.481316pt}}
\pgflineto{\pgfpoint{164.988876pt}{117.481316pt}}
\pgfusepath{stroke}
\pgfpathmoveto{\pgfpoint{167.988876pt}{114.481316pt}}
\pgflineto{\pgfpoint{167.988876pt}{120.481316pt}}
\pgfusepath{stroke}
\pgfpathmoveto{\pgfpoint{165.308563pt}{111.498383pt}}
\pgflineto{\pgfpoint{159.308563pt}{111.498383pt}}
\pgfusepath{stroke}
\pgfpathmoveto{\pgfpoint{162.308563pt}{108.498383pt}}
\pgflineto{\pgfpoint{162.308563pt}{114.498383pt}}
\pgfusepath{stroke}
\pgfpathmoveto{\pgfpoint{196.641235pt}{106.520065pt}}
\pgflineto{\pgfpoint{190.641235pt}{106.520065pt}}
\pgfusepath{stroke}
\pgfpathmoveto{\pgfpoint{193.641235pt}{103.520065pt}}
\pgflineto{\pgfpoint{193.641235pt}{109.520065pt}}
\pgfusepath{stroke}
\pgfpathmoveto{\pgfpoint{198.887161pt}{112.126419pt}}
\pgflineto{\pgfpoint{192.887161pt}{112.126419pt}}
\pgfusepath{stroke}
\pgfpathmoveto{\pgfpoint{195.887161pt}{109.126419pt}}
\pgflineto{\pgfpoint{195.887161pt}{115.126419pt}}
\pgfusepath{stroke}
\pgfpathmoveto{\pgfpoint{190.139114pt}{118.018402pt}}
\pgflineto{\pgfpoint{184.139114pt}{118.018402pt}}
\pgfusepath{stroke}
\pgfpathmoveto{\pgfpoint{187.139114pt}{115.018394pt}}
\pgflineto{\pgfpoint{187.139114pt}{121.018402pt}}
\pgfusepath{stroke}
\pgfpathmoveto{\pgfpoint{180.520355pt}{106.133606pt}}
\pgflineto{\pgfpoint{174.520355pt}{106.133606pt}}
\pgfusepath{stroke}
\pgfpathmoveto{\pgfpoint{177.520355pt}{103.133606pt}}
\pgflineto{\pgfpoint{177.520355pt}{109.133606pt}}
\pgfusepath{stroke}
\pgfpathmoveto{\pgfpoint{166.703690pt}{93.097610pt}}
\pgflineto{\pgfpoint{160.703690pt}{93.097610pt}}
\pgfusepath{stroke}
\pgfpathmoveto{\pgfpoint{163.703690pt}{90.097610pt}}
\pgflineto{\pgfpoint{163.703690pt}{96.097618pt}}
\pgfusepath{stroke}
\pgfpathmoveto{\pgfpoint{172.526749pt}{104.289711pt}}
\pgflineto{\pgfpoint{166.526749pt}{104.289711pt}}
\pgfusepath{stroke}
\pgfpathmoveto{\pgfpoint{169.526749pt}{101.289703pt}}
\pgflineto{\pgfpoint{169.526749pt}{107.289711pt}}
\pgfusepath{stroke}
\pgfpathmoveto{\pgfpoint{187.477814pt}{117.596985pt}}
\pgflineto{\pgfpoint{181.477814pt}{117.596985pt}}
\pgfusepath{stroke}
\pgfpathmoveto{\pgfpoint{184.477814pt}{114.596985pt}}
\pgflineto{\pgfpoint{184.477814pt}{120.596985pt}}
\pgfusepath{stroke}
\pgfpathmoveto{\pgfpoint{181.620728pt}{109.305641pt}}
\pgflineto{\pgfpoint{175.620728pt}{109.305641pt}}
\pgfusepath{stroke}
\pgfpathmoveto{\pgfpoint{178.620728pt}{106.305641pt}}
\pgflineto{\pgfpoint{178.620728pt}{112.305641pt}}
\pgfusepath{stroke}
\pgfpathmoveto{\pgfpoint{200.743759pt}{116.577286pt}}
\pgflineto{\pgfpoint{194.743759pt}{116.577286pt}}
\pgfusepath{stroke}
\pgfpathmoveto{\pgfpoint{197.743759pt}{113.577286pt}}
\pgflineto{\pgfpoint{197.743759pt}{119.577286pt}}
\pgfusepath{stroke}
\pgfpathmoveto{\pgfpoint{171.920624pt}{108.920357pt}}
\pgflineto{\pgfpoint{165.920624pt}{108.920357pt}}
\pgfusepath{stroke}
\pgfpathmoveto{\pgfpoint{168.920624pt}{105.920357pt}}
\pgflineto{\pgfpoint{168.920624pt}{111.920364pt}}
\pgfusepath{stroke}
\pgfpathmoveto{\pgfpoint{135.909760pt}{73.728447pt}}
\pgflineto{\pgfpoint{129.909760pt}{73.728447pt}}
\pgfusepath{stroke}
\pgfpathmoveto{\pgfpoint{132.909760pt}{70.728447pt}}
\pgflineto{\pgfpoint{132.909760pt}{76.728447pt}}
\pgfusepath{stroke}
\pgfpathmoveto{\pgfpoint{120.742088pt}{87.429214pt}}
\pgflineto{\pgfpoint{114.742088pt}{87.429214pt}}
\pgfusepath{stroke}
\pgfpathmoveto{\pgfpoint{117.742088pt}{84.429214pt}}
\pgflineto{\pgfpoint{117.742088pt}{90.429214pt}}
\pgfusepath{stroke}
\pgfpathmoveto{\pgfpoint{184.925278pt}{115.256699pt}}
\pgflineto{\pgfpoint{178.925278pt}{115.256699pt}}
\pgfusepath{stroke}
\pgfpathmoveto{\pgfpoint{181.925278pt}{112.256699pt}}
\pgflineto{\pgfpoint{181.925278pt}{118.256699pt}}
\pgfusepath{stroke}
\color[rgb]{0.000000,0.500000,0.000000}
\pgfpathmoveto{\pgfpoint{184.936493pt}{137.227509pt}}
\pgflineto{\pgfpoint{178.936493pt}{143.227509pt}}
\pgfusepath{stroke}
\pgfpathmoveto{\pgfpoint{184.936493pt}{143.227509pt}}
\pgflineto{\pgfpoint{178.936493pt}{137.227509pt}}
\pgfusepath{stroke}
\pgfpathmoveto{\pgfpoint{117.511063pt}{203.316452pt}}
\pgflineto{\pgfpoint{111.511063pt}{209.316437pt}}
\pgfusepath{stroke}
\pgfpathmoveto{\pgfpoint{117.511063pt}{209.316437pt}}
\pgflineto{\pgfpoint{111.511063pt}{203.316452pt}}
\pgfusepath{stroke}
\pgfpathmoveto{\pgfpoint{188.206314pt}{125.640877pt}}
\pgflineto{\pgfpoint{182.206314pt}{131.640869pt}}
\pgfusepath{stroke}
\pgfpathmoveto{\pgfpoint{188.206314pt}{131.640869pt}}
\pgflineto{\pgfpoint{182.206314pt}{125.640877pt}}
\pgfusepath{stroke}
\pgfpathmoveto{\pgfpoint{111.670929pt}{135.618835pt}}
\pgflineto{\pgfpoint{105.670929pt}{141.618851pt}}
\pgfusepath{stroke}
\pgfpathmoveto{\pgfpoint{111.670929pt}{141.618851pt}}
\pgflineto{\pgfpoint{105.670929pt}{135.618835pt}}
\pgfusepath{stroke}
\pgfpathmoveto{\pgfpoint{133.251724pt}{139.476685pt}}
\pgflineto{\pgfpoint{127.251724pt}{145.476685pt}}
\pgfusepath{stroke}
\pgfpathmoveto{\pgfpoint{133.251724pt}{145.476685pt}}
\pgflineto{\pgfpoint{127.251724pt}{139.476685pt}}
\pgfusepath{stroke}
\pgfpathmoveto{\pgfpoint{66.771400pt}{212.555969pt}}
\pgflineto{\pgfpoint{60.771400pt}{218.555969pt}}
\pgfusepath{stroke}
\pgfpathmoveto{\pgfpoint{66.771400pt}{218.555969pt}}
\pgflineto{\pgfpoint{60.771400pt}{212.555969pt}}
\pgfusepath{stroke}
\pgfpathmoveto{\pgfpoint{185.577332pt}{146.044998pt}}
\pgflineto{\pgfpoint{179.577332pt}{152.044998pt}}
\pgfusepath{stroke}
\pgfpathmoveto{\pgfpoint{185.577332pt}{152.044998pt}}
\pgflineto{\pgfpoint{179.577332pt}{146.044998pt}}
\pgfusepath{stroke}
\pgfpathmoveto{\pgfpoint{163.424515pt}{129.375381pt}}
\pgflineto{\pgfpoint{157.424515pt}{135.375381pt}}
\pgfusepath{stroke}
\pgfpathmoveto{\pgfpoint{163.424515pt}{135.375381pt}}
\pgflineto{\pgfpoint{157.424515pt}{129.375381pt}}
\pgfusepath{stroke}
\pgfpathmoveto{\pgfpoint{164.779160pt}{174.718796pt}}
\pgflineto{\pgfpoint{158.779160pt}{180.718796pt}}
\pgfusepath{stroke}
\pgfpathmoveto{\pgfpoint{164.779160pt}{180.718796pt}}
\pgflineto{\pgfpoint{158.779160pt}{174.718796pt}}
\pgfusepath{stroke}
\pgfpathmoveto{\pgfpoint{131.375275pt}{150.253448pt}}
\pgflineto{\pgfpoint{125.375275pt}{156.253448pt}}
\pgfusepath{stroke}
\pgfpathmoveto{\pgfpoint{131.375275pt}{156.253448pt}}
\pgflineto{\pgfpoint{125.375275pt}{150.253448pt}}
\pgfusepath{stroke}
\pgfpathmoveto{\pgfpoint{64.678047pt}{165.434067pt}}
\pgflineto{\pgfpoint{58.678047pt}{171.434067pt}}
\pgfusepath{stroke}
\pgfpathmoveto{\pgfpoint{64.678047pt}{171.434067pt}}
\pgflineto{\pgfpoint{58.678047pt}{165.434067pt}}
\pgfusepath{stroke}
\pgfpathmoveto{\pgfpoint{168.128555pt}{128.687103pt}}
\pgflineto{\pgfpoint{162.128555pt}{134.687103pt}}
\pgfusepath{stroke}
\pgfpathmoveto{\pgfpoint{168.128555pt}{134.687103pt}}
\pgflineto{\pgfpoint{162.128555pt}{128.687103pt}}
\pgfusepath{stroke}
\pgfpathmoveto{\pgfpoint{205.642883pt}{147.756851pt}}
\pgflineto{\pgfpoint{199.642883pt}{153.756851pt}}
\pgfusepath{stroke}
\pgfpathmoveto{\pgfpoint{205.642883pt}{153.756851pt}}
\pgflineto{\pgfpoint{199.642883pt}{147.756851pt}}
\pgfusepath{stroke}
\pgfpathmoveto{\pgfpoint{156.901917pt}{135.317474pt}}
\pgflineto{\pgfpoint{150.901917pt}{141.317474pt}}
\pgfusepath{stroke}
\pgfpathmoveto{\pgfpoint{156.901917pt}{141.317474pt}}
\pgflineto{\pgfpoint{150.901917pt}{135.317474pt}}
\pgfusepath{stroke}
\pgfpathmoveto{\pgfpoint{178.512466pt}{125.103767pt}}
\pgflineto{\pgfpoint{172.512466pt}{131.103775pt}}
\pgfusepath{stroke}
\pgfpathmoveto{\pgfpoint{178.512466pt}{131.103775pt}}
\pgflineto{\pgfpoint{172.512466pt}{125.103767pt}}
\pgfusepath{stroke}
\pgfpathmoveto{\pgfpoint{182.324051pt}{109.428757pt}}
\pgflineto{\pgfpoint{176.324051pt}{115.428764pt}}
\pgfusepath{stroke}
\pgfpathmoveto{\pgfpoint{182.324051pt}{115.428764pt}}
\pgflineto{\pgfpoint{176.324051pt}{109.428757pt}}
\pgfusepath{stroke}
\pgfpathmoveto{\pgfpoint{183.976471pt}{163.479218pt}}
\pgflineto{\pgfpoint{177.976471pt}{169.479218pt}}
\pgfusepath{stroke}
\pgfpathmoveto{\pgfpoint{183.976471pt}{169.479218pt}}
\pgflineto{\pgfpoint{177.976471pt}{163.479218pt}}
\pgfusepath{stroke}
\pgfpathmoveto{\pgfpoint{186.273865pt}{123.065292pt}}
\pgflineto{\pgfpoint{180.273865pt}{129.065292pt}}
\pgfusepath{stroke}
\pgfpathmoveto{\pgfpoint{186.273865pt}{129.065292pt}}
\pgflineto{\pgfpoint{180.273865pt}{123.065292pt}}
\pgfusepath{stroke}
\pgfpathmoveto{\pgfpoint{158.661362pt}{148.370621pt}}
\pgflineto{\pgfpoint{152.661362pt}{154.370621pt}}
\pgfusepath{stroke}
\pgfpathmoveto{\pgfpoint{158.661362pt}{154.370621pt}}
\pgflineto{\pgfpoint{152.661362pt}{148.370621pt}}
\pgfusepath{stroke}
\pgfpathmoveto{\pgfpoint{137.910004pt}{126.041534pt}}
\pgflineto{\pgfpoint{131.910004pt}{132.041534pt}}
\pgfusepath{stroke}
\pgfpathmoveto{\pgfpoint{137.910004pt}{132.041534pt}}
\pgflineto{\pgfpoint{131.910004pt}{126.041534pt}}
\pgfusepath{stroke}
\pgfpathmoveto{\pgfpoint{175.946411pt}{127.328346pt}}
\pgflineto{\pgfpoint{169.946411pt}{133.328354pt}}
\pgfusepath{stroke}
\pgfpathmoveto{\pgfpoint{175.946411pt}{133.328354pt}}
\pgflineto{\pgfpoint{169.946411pt}{127.328346pt}}
\pgfusepath{stroke}
\pgfpathmoveto{\pgfpoint{172.102951pt}{131.214188pt}}
\pgflineto{\pgfpoint{166.102951pt}{137.214188pt}}
\pgfusepath{stroke}
\pgfpathmoveto{\pgfpoint{172.102951pt}{137.214188pt}}
\pgflineto{\pgfpoint{166.102951pt}{131.214188pt}}
\pgfusepath{stroke}
\pgfpathmoveto{\pgfpoint{154.296158pt}{144.527710pt}}
\pgflineto{\pgfpoint{148.296158pt}{150.527710pt}}
\pgfusepath{stroke}
\pgfpathmoveto{\pgfpoint{154.296158pt}{150.527710pt}}
\pgflineto{\pgfpoint{148.296158pt}{144.527710pt}}
\pgfusepath{stroke}
\pgfpathmoveto{\pgfpoint{159.834061pt}{132.893829pt}}
\pgflineto{\pgfpoint{153.834061pt}{138.893829pt}}
\pgfusepath{stroke}
\pgfpathmoveto{\pgfpoint{159.834061pt}{138.893829pt}}
\pgflineto{\pgfpoint{153.834061pt}{132.893829pt}}
\pgfusepath{stroke}
\pgfpathmoveto{\pgfpoint{177.900040pt}{134.565033pt}}
\pgflineto{\pgfpoint{171.900040pt}{140.565033pt}}
\pgfusepath{stroke}
\pgfpathmoveto{\pgfpoint{177.900040pt}{140.565033pt}}
\pgflineto{\pgfpoint{171.900040pt}{134.565033pt}}
\pgfusepath{stroke}
\pgfpathmoveto{\pgfpoint{158.810394pt}{146.112701pt}}
\pgflineto{\pgfpoint{152.810394pt}{152.112701pt}}
\pgfusepath{stroke}
\pgfpathmoveto{\pgfpoint{158.810394pt}{152.112701pt}}
\pgflineto{\pgfpoint{152.810394pt}{146.112701pt}}
\pgfusepath{stroke}
\pgfpathmoveto{\pgfpoint{165.751907pt}{148.133469pt}}
\pgflineto{\pgfpoint{159.751907pt}{154.133469pt}}
\pgfusepath{stroke}
\pgfpathmoveto{\pgfpoint{165.751907pt}{154.133469pt}}
\pgflineto{\pgfpoint{159.751907pt}{148.133469pt}}
\pgfusepath{stroke}
\pgfpathmoveto{\pgfpoint{188.722488pt}{128.864990pt}}
\pgflineto{\pgfpoint{182.722488pt}{134.864990pt}}
\pgfusepath{stroke}
\pgfpathmoveto{\pgfpoint{188.722488pt}{134.864990pt}}
\pgflineto{\pgfpoint{182.722488pt}{128.864990pt}}
\pgfusepath{stroke}
\pgfpathmoveto{\pgfpoint{163.783356pt}{136.138657pt}}
\pgflineto{\pgfpoint{157.783356pt}{142.138657pt}}
\pgfusepath{stroke}
\pgfpathmoveto{\pgfpoint{163.783356pt}{142.138657pt}}
\pgflineto{\pgfpoint{157.783356pt}{136.138657pt}}
\pgfusepath{stroke}
\pgfpathmoveto{\pgfpoint{156.204987pt}{139.583450pt}}
\pgflineto{\pgfpoint{150.204987pt}{145.583450pt}}
\pgfusepath{stroke}
\pgfpathmoveto{\pgfpoint{156.204987pt}{145.583450pt}}
\pgflineto{\pgfpoint{150.204987pt}{139.583450pt}}
\pgfusepath{stroke}
\pgfpathmoveto{\pgfpoint{108.090050pt}{167.233994pt}}
\pgflineto{\pgfpoint{102.090042pt}{173.233994pt}}
\pgfusepath{stroke}
\pgfpathmoveto{\pgfpoint{108.090050pt}{173.233994pt}}
\pgflineto{\pgfpoint{102.090042pt}{167.233994pt}}
\pgfusepath{stroke}
\pgfpathmoveto{\pgfpoint{103.367920pt}{194.184723pt}}
\pgflineto{\pgfpoint{97.367920pt}{200.184723pt}}
\pgfusepath{stroke}
\pgfpathmoveto{\pgfpoint{103.367920pt}{200.184723pt}}
\pgflineto{\pgfpoint{97.367920pt}{194.184723pt}}
\pgfusepath{stroke}
\pgfpathmoveto{\pgfpoint{158.083984pt}{140.284790pt}}
\pgflineto{\pgfpoint{152.083984pt}{146.284790pt}}
\pgfusepath{stroke}
\pgfpathmoveto{\pgfpoint{158.083984pt}{146.284790pt}}
\pgflineto{\pgfpoint{152.083984pt}{140.284790pt}}
\pgfusepath{stroke}
\pgfpathmoveto{\pgfpoint{165.972244pt}{160.728363pt}}
\pgflineto{\pgfpoint{159.972244pt}{166.728363pt}}
\pgfusepath{stroke}
\pgfpathmoveto{\pgfpoint{165.972244pt}{166.728363pt}}
\pgflineto{\pgfpoint{159.972244pt}{160.728363pt}}
\pgfusepath{stroke}
\pgfpathmoveto{\pgfpoint{173.461487pt}{128.505020pt}}
\pgflineto{\pgfpoint{167.461487pt}{134.505020pt}}
\pgfusepath{stroke}
\pgfpathmoveto{\pgfpoint{173.461487pt}{134.505020pt}}
\pgflineto{\pgfpoint{167.461487pt}{128.505020pt}}
\pgfusepath{stroke}
\pgfpathmoveto{\pgfpoint{169.788940pt}{130.577332pt}}
\pgflineto{\pgfpoint{163.788940pt}{136.577332pt}}
\pgfusepath{stroke}
\pgfpathmoveto{\pgfpoint{169.788940pt}{136.577332pt}}
\pgflineto{\pgfpoint{163.788940pt}{130.577332pt}}
\pgfusepath{stroke}
\pgfpathmoveto{\pgfpoint{182.993423pt}{124.435440pt}}
\pgflineto{\pgfpoint{176.993423pt}{130.435440pt}}
\pgfusepath{stroke}
\pgfpathmoveto{\pgfpoint{182.993423pt}{130.435440pt}}
\pgflineto{\pgfpoint{176.993423pt}{124.435440pt}}
\pgfusepath{stroke}
\pgfpathmoveto{\pgfpoint{162.180969pt}{139.215729pt}}
\pgflineto{\pgfpoint{156.180969pt}{145.215729pt}}
\pgfusepath{stroke}
\pgfpathmoveto{\pgfpoint{162.180969pt}{145.215729pt}}
\pgflineto{\pgfpoint{156.180969pt}{139.215729pt}}
\pgfusepath{stroke}
\pgfpathmoveto{\pgfpoint{170.219406pt}{131.559265pt}}
\pgflineto{\pgfpoint{164.219406pt}{137.559265pt}}
\pgfusepath{stroke}
\pgfpathmoveto{\pgfpoint{170.219406pt}{137.559265pt}}
\pgflineto{\pgfpoint{164.219406pt}{131.559265pt}}
\pgfusepath{stroke}
\pgfpathmoveto{\pgfpoint{167.697754pt}{131.977951pt}}
\pgflineto{\pgfpoint{161.697754pt}{137.977951pt}}
\pgfusepath{stroke}
\pgfpathmoveto{\pgfpoint{167.697754pt}{137.977951pt}}
\pgflineto{\pgfpoint{161.697754pt}{131.977951pt}}
\pgfusepath{stroke}
\pgfpathmoveto{\pgfpoint{131.257980pt}{152.243408pt}}
\pgflineto{\pgfpoint{125.257980pt}{158.243423pt}}
\pgfusepath{stroke}
\pgfpathmoveto{\pgfpoint{131.257980pt}{158.243423pt}}
\pgflineto{\pgfpoint{125.257980pt}{152.243408pt}}
\pgfusepath{stroke}
\pgfpathmoveto{\pgfpoint{184.625519pt}{119.561653pt}}
\pgflineto{\pgfpoint{178.625519pt}{125.561653pt}}
\pgfusepath{stroke}
\pgfpathmoveto{\pgfpoint{184.625519pt}{125.561653pt}}
\pgflineto{\pgfpoint{178.625519pt}{119.561653pt}}
\pgfusepath{stroke}
\pgfpathmoveto{\pgfpoint{177.656387pt}{108.470901pt}}
\pgflineto{\pgfpoint{171.656387pt}{114.470901pt}}
\pgfusepath{stroke}
\pgfpathmoveto{\pgfpoint{177.656387pt}{114.470901pt}}
\pgflineto{\pgfpoint{171.656387pt}{108.470901pt}}
\pgfusepath{stroke}
\pgfpathmoveto{\pgfpoint{167.816437pt}{119.507477pt}}
\pgflineto{\pgfpoint{161.816437pt}{125.507477pt}}
\pgfusepath{stroke}
\pgfpathmoveto{\pgfpoint{167.816437pt}{125.507477pt}}
\pgflineto{\pgfpoint{161.816437pt}{119.507477pt}}
\pgfusepath{stroke}
\pgfpathmoveto{\pgfpoint{165.924042pt}{123.079765pt}}
\pgflineto{\pgfpoint{159.924042pt}{129.079773pt}}
\pgfusepath{stroke}
\pgfpathmoveto{\pgfpoint{165.924042pt}{129.079773pt}}
\pgflineto{\pgfpoint{159.924042pt}{123.079765pt}}
\pgfusepath{stroke}
\pgfpathmoveto{\pgfpoint{182.292389pt}{126.585098pt}}
\pgflineto{\pgfpoint{176.292389pt}{132.585098pt}}
\pgfusepath{stroke}
\pgfpathmoveto{\pgfpoint{182.292389pt}{132.585098pt}}
\pgflineto{\pgfpoint{176.292389pt}{126.585098pt}}
\pgfusepath{stroke}
\pgfpathmoveto{\pgfpoint{97.353935pt}{165.778976pt}}
\pgflineto{\pgfpoint{91.353928pt}{171.778976pt}}
\pgfusepath{stroke}
\pgfpathmoveto{\pgfpoint{97.353935pt}{171.778976pt}}
\pgflineto{\pgfpoint{91.353928pt}{165.778976pt}}
\pgfusepath{stroke}
\pgfpathmoveto{\pgfpoint{148.885895pt}{141.106918pt}}
\pgflineto{\pgfpoint{142.885895pt}{147.106918pt}}
\pgfusepath{stroke}
\pgfpathmoveto{\pgfpoint{148.885895pt}{147.106918pt}}
\pgflineto{\pgfpoint{142.885895pt}{141.106918pt}}
\pgfusepath{stroke}
\pgfpathmoveto{\pgfpoint{118.961487pt}{144.851547pt}}
\pgflineto{\pgfpoint{112.961487pt}{150.851547pt}}
\pgfusepath{stroke}
\pgfpathmoveto{\pgfpoint{118.961487pt}{150.851547pt}}
\pgflineto{\pgfpoint{112.961487pt}{144.851547pt}}
\pgfusepath{stroke}
\pgfpathmoveto{\pgfpoint{177.716965pt}{129.267120pt}}
\pgflineto{\pgfpoint{171.716965pt}{135.267120pt}}
\pgfusepath{stroke}
\pgfpathmoveto{\pgfpoint{177.716965pt}{135.267120pt}}
\pgflineto{\pgfpoint{171.716965pt}{129.267120pt}}
\pgfusepath{stroke}
\pgfpathmoveto{\pgfpoint{170.739990pt}{134.313614pt}}
\pgflineto{\pgfpoint{164.739990pt}{140.313614pt}}
\pgfusepath{stroke}
\pgfpathmoveto{\pgfpoint{170.739990pt}{140.313614pt}}
\pgflineto{\pgfpoint{164.739990pt}{134.313614pt}}
\pgfusepath{stroke}
\pgfpathmoveto{\pgfpoint{175.963943pt}{158.481049pt}}
\pgflineto{\pgfpoint{169.963943pt}{164.481049pt}}
\pgfusepath{stroke}
\pgfpathmoveto{\pgfpoint{175.963943pt}{164.481049pt}}
\pgflineto{\pgfpoint{169.963943pt}{158.481049pt}}
\pgfusepath{stroke}
\pgfpathmoveto{\pgfpoint{147.413361pt}{133.222565pt}}
\pgflineto{\pgfpoint{141.413361pt}{139.222565pt}}
\pgfusepath{stroke}
\pgfpathmoveto{\pgfpoint{147.413361pt}{139.222565pt}}
\pgflineto{\pgfpoint{141.413361pt}{133.222565pt}}
\pgfusepath{stroke}
\pgfpathmoveto{\pgfpoint{79.925964pt}{149.746185pt}}
\pgflineto{\pgfpoint{73.925964pt}{155.746185pt}}
\pgfusepath{stroke}
\pgfpathmoveto{\pgfpoint{79.925964pt}{155.746185pt}}
\pgflineto{\pgfpoint{73.925964pt}{149.746185pt}}
\pgfusepath{stroke}
\pgfpathmoveto{\pgfpoint{73.745819pt}{168.351334pt}}
\pgflineto{\pgfpoint{67.745819pt}{174.351334pt}}
\pgfusepath{stroke}
\pgfpathmoveto{\pgfpoint{73.745819pt}{174.351334pt}}
\pgflineto{\pgfpoint{67.745819pt}{168.351334pt}}
\pgfusepath{stroke}
\pgfpathmoveto{\pgfpoint{172.578934pt}{127.022873pt}}
\pgflineto{\pgfpoint{166.578934pt}{133.022873pt}}
\pgfusepath{stroke}
\pgfpathmoveto{\pgfpoint{172.578934pt}{133.022873pt}}
\pgflineto{\pgfpoint{166.578934pt}{127.022873pt}}
\pgfusepath{stroke}
\pgfpathmoveto{\pgfpoint{165.846512pt}{137.737061pt}}
\pgflineto{\pgfpoint{159.846512pt}{143.737061pt}}
\pgfusepath{stroke}
\pgfpathmoveto{\pgfpoint{165.846512pt}{143.737061pt}}
\pgflineto{\pgfpoint{159.846512pt}{137.737061pt}}
\pgfusepath{stroke}
\pgfpathmoveto{\pgfpoint{150.903778pt}{157.222382pt}}
\pgflineto{\pgfpoint{144.903778pt}{163.222382pt}}
\pgfusepath{stroke}
\pgfpathmoveto{\pgfpoint{150.903778pt}{163.222382pt}}
\pgflineto{\pgfpoint{144.903778pt}{157.222382pt}}
\pgfusepath{stroke}
\pgfpathmoveto{\pgfpoint{111.097565pt}{150.775970pt}}
\pgflineto{\pgfpoint{105.097565pt}{156.775970pt}}
\pgfusepath{stroke}
\pgfpathmoveto{\pgfpoint{111.097565pt}{156.775970pt}}
\pgflineto{\pgfpoint{105.097565pt}{150.775970pt}}
\pgfusepath{stroke}
\pgfpathmoveto{\pgfpoint{141.830536pt}{141.786835pt}}
\pgflineto{\pgfpoint{135.830536pt}{147.786835pt}}
\pgfusepath{stroke}
\pgfpathmoveto{\pgfpoint{141.830536pt}{147.786835pt}}
\pgflineto{\pgfpoint{135.830536pt}{141.786835pt}}
\pgfusepath{stroke}
\color[rgb]{1.000000,0.000000,0.000000}
\pgfpathmoveto{\pgfpoint{203.971069pt}{127.326637pt}}
\pgflineto{\pgfpoint{204.544022pt}{129.089996pt}}
\pgfusepath{stroke}
\pgfpathmoveto{\pgfpoint{202.471069pt}{126.236824pt}}
\pgflineto{\pgfpoint{203.971069pt}{127.326637pt}}
\pgfusepath{stroke}
\pgfpathmoveto{\pgfpoint{200.616974pt}{126.236824pt}}
\pgflineto{\pgfpoint{202.471069pt}{126.236824pt}}
\pgfusepath{stroke}
\pgfpathmoveto{\pgfpoint{199.116974pt}{127.326637pt}}
\pgflineto{\pgfpoint{200.616974pt}{126.236824pt}}
\pgfusepath{stroke}
\pgfpathmoveto{\pgfpoint{198.544022pt}{129.089996pt}}
\pgflineto{\pgfpoint{199.116974pt}{127.326637pt}}
\pgfusepath{stroke}
\pgfpathmoveto{\pgfpoint{199.116974pt}{130.853348pt}}
\pgflineto{\pgfpoint{198.544022pt}{129.089996pt}}
\pgfusepath{stroke}
\pgfpathmoveto{\pgfpoint{200.616974pt}{131.943161pt}}
\pgflineto{\pgfpoint{199.116974pt}{130.853348pt}}
\pgfusepath{stroke}
\pgfpathmoveto{\pgfpoint{202.471069pt}{131.943161pt}}
\pgflineto{\pgfpoint{200.616974pt}{131.943161pt}}
\pgfusepath{stroke}
\pgfpathmoveto{\pgfpoint{203.971069pt}{130.853348pt}}
\pgflineto{\pgfpoint{202.471069pt}{131.943161pt}}
\pgfusepath{stroke}
\pgfpathmoveto{\pgfpoint{204.544022pt}{129.089996pt}}
\pgflineto{\pgfpoint{203.971069pt}{130.853348pt}}
\pgfusepath{stroke}
\color[rgb]{0.000000,0.000000,0.000000}
\pgfsetdash{{16pt}{0pt}}{0pt}
\pgfpathmoveto{\pgfpoint{288.074158pt}{197.039612pt}}
\pgflineto{\pgfpoint{260.624542pt}{197.039612pt}}
\pgfusepath{stroke}
\pgfpathmoveto{\pgfpoint{288.074158pt}{220.474182pt}}
\pgflineto{\pgfpoint{260.624542pt}{220.474182pt}}
\pgfusepath{stroke}
\pgfpathmoveto{\pgfpoint{260.624542pt}{220.474182pt}}
\pgflineto{\pgfpoint{260.624542pt}{197.039612pt}}
\pgfusepath{stroke}
\pgfpathmoveto{\pgfpoint{288.074158pt}{220.474182pt}}
\pgflineto{\pgfpoint{288.074158pt}{197.039612pt}}
\pgfusepath{stroke}
{
\pgftransformshift{\pgfpoint{275.119781pt}{216.568420pt}}
\pgfnode{rectangle}{west}{\fontsize{10}{0}\selectfont\textcolor[rgb]{0,0,0}{{BUT}}}{}{\pgfusepath{discard}}}
{
\pgftransformshift{\pgfpoint{275.119781pt}{208.756897pt}}
\pgfnode{rectangle}{west}{\fontsize{10}{0}\selectfont\textcolor[rgb]{0,0,0}{{VVJ}}}{}{\pgfusepath{discard}}}
{
\pgftransformshift{\pgfpoint{275.119781pt}{200.945374pt}}
\pgfnode{rectangle}{west}{\fontsize{10}{0}\selectfont\textcolor[rgb]{0,0,0}{{?}}}{}{\pgfusepath{discard}}}
\color[rgb]{0.000000,0.000000,1.000000}
\pgfsetdash{}{0pt}
\pgfpathmoveto{\pgfpoint{270.872192pt}{216.568420pt}}
\pgflineto{\pgfpoint{264.872192pt}{216.568420pt}}
\pgfusepath{stroke}
\pgfpathmoveto{\pgfpoint{267.872192pt}{213.568420pt}}
\pgflineto{\pgfpoint{267.872192pt}{219.568420pt}}
\pgfusepath{stroke}
\color[rgb]{0.000000,0.500000,0.000000}
\pgfpathmoveto{\pgfpoint{270.872192pt}{205.756897pt}}
\pgflineto{\pgfpoint{264.872192pt}{211.756897pt}}
\pgfusepath{stroke}
\pgfpathmoveto{\pgfpoint{270.872192pt}{211.756897pt}}
\pgflineto{\pgfpoint{264.872192pt}{205.756897pt}}
\pgfusepath{stroke}
\color[rgb]{1.000000,0.000000,0.000000}
\pgfpathmoveto{\pgfpoint{270.299255pt}{199.182007pt}}
\pgflineto{\pgfpoint{270.872192pt}{200.945374pt}}
\pgfusepath{stroke}
\pgfpathmoveto{\pgfpoint{268.799255pt}{198.092194pt}}
\pgflineto{\pgfpoint{270.299255pt}{199.182007pt}}
\pgfusepath{stroke}
\pgfpathmoveto{\pgfpoint{266.945160pt}{198.092194pt}}
\pgflineto{\pgfpoint{268.799255pt}{198.092194pt}}
\pgfusepath{stroke}
\pgfpathmoveto{\pgfpoint{265.445129pt}{199.182007pt}}
\pgflineto{\pgfpoint{266.945160pt}{198.092194pt}}
\pgfusepath{stroke}
\pgfpathmoveto{\pgfpoint{264.872192pt}{200.945374pt}}
\pgflineto{\pgfpoint{265.445129pt}{199.182007pt}}
\pgfusepath{stroke}
\pgfpathmoveto{\pgfpoint{265.445129pt}{202.708710pt}}
\pgflineto{\pgfpoint{264.872192pt}{200.945374pt}}
\pgfusepath{stroke}
\pgfpathmoveto{\pgfpoint{266.945160pt}{203.798523pt}}
\pgflineto{\pgfpoint{265.445129pt}{202.708710pt}}
\pgfusepath{stroke}
\pgfpathmoveto{\pgfpoint{268.799255pt}{203.798523pt}}
\pgflineto{\pgfpoint{266.945160pt}{203.798523pt}}
\pgfusepath{stroke}
\pgfpathmoveto{\pgfpoint{270.299255pt}{202.708710pt}}
\pgflineto{\pgfpoint{268.799255pt}{203.798523pt}}
\pgfusepath{stroke}
\pgfpathmoveto{\pgfpoint{270.872192pt}{200.945374pt}}
\pgflineto{\pgfpoint{270.299255pt}{202.708710pt}}
\pgfusepath{stroke}
\end{pgfpicture}

  \only<presentation>{
    \only<2>{\Large \alert{What if it a \textbf{completely different strain}?}}
  }
  \only<article>{Given that the $+$ points represent the BUT type, and the $\times$ points the VVJ type, what type of bacterium could the circle point be?}
\end{frame}

\only<presentation>{
  \begin{frame}
    \frametitle{Hands on with Python console}
    \hyperlink{../src/decision-problems/knn-classify.py}{\beamerbutton{KNN example}}
    %% Use knn-classify to simply demonstrate a kNN classifier.
  \end{frame}
}

\begin{frame}
  \frametitle{Discussion: Shortcomings of $k$-nearest neighbour}
  \begin{itemize}
  \item Choice of $k$
  \item Choice of metric.
  \item Representation of uncertainty.
  \item Scaling with large amounts of data.
  \end{itemize}
  \only<article>{For that reason, it is best to think of $k$-NN as a \alert{model} for predicting the class of a new example from a finite set of existing classes. The model itself might be incorrect, but this should nevertheless be OK for our purposes. In particular, we might later use the model in order to derive classification rules.}

\end{frame}


%%% Local Variables:
%%% mode: latex
%%% TeX-master: "notes.tex"
%%% End:
  % knn, reproducability and bootstrapping
\section{Naive Bayes classifiers}
\only<presentation>{
  \begin{frame}
    \tableofcontents[ 
    currentsection, 
    hideothersubsections, 
    sectionstyle=show/shaded
    ] 
  \end{frame}
}

\begin{frame}
  \only<article>{ One special case of this idea is in classification,
    when each hypothesis corresponds to a specific class. Then, given
    a new example vector of data $\bx$, we would like to calculate the
    probability of different classes $C$ given the data,
    $\Pr(C \mid \bx)$. So here, the class is the hypothesis.

    From Bayes's theorem, we see that we can write this as }
  \[
    \Pr(C \mid \bx) = \frac{\Pr(\bx \mid C) \Pr(C)}{\sum_{i} \Pr(\bx
      \mid C_i) \Pr(C_i)}
  \]
  \only<article>{ for any class $C$. This directly gives us a method
    for classifying new data, as long as we have a way to obtain
    $\Pr(\bx \mid C)$ and $\Pr(C)$.  }

  But should we use for the probability model $\Pr$?

  \subsubsection{Naive Bayes classifier}

  \begin{block}{Calculating the prior probability of classes}
    A simple method is to simply count the number of times each class
    appears in the training data $\Training = ((x_t,
    y_t))_{t=1}^T$. Then we can set
    \[
      \Pr(C) = 1/T \sum_{t=1}^T \ind{y_t = C}
    \]
  \end{block}

  \only<article>{ The Naive Bayes classifier uses the following model
    for observations, where observations are independent of each other
    given the class. Thus, for example the result of three different
    tests for lung cancer (stethoscope, radiography and biopsy) only
    depend on whether you have cancer, and not on each other.  }
  \begin{block}{Probability model for observations}
    \[
      \Pr(\bx \mid C) = \Pr(x(1), \ldots, x(n) \mid C) = \prod_{k=1}^n
      \Pr(x(k) \mid C).
    \]
  \end{block}

\end{frame}

\begin{frame}
  \only<article>{There are two different types of models we can have,
    one of which is mostly useful for continuous attributes and the
    other for discrete attributes.  In the first, we just need to
    count the number of times each feature takes different values in
    different classes.  }
  \begin{block}{Discrete attribute model.}
    \only<article>{Here we simply count the average number of times
      that the attribute $k$ had the value $i$ when the label was
      $C$. This is in fact analogous to the conditional probability
      definition.}
    \[
      \Pr(x(k) = i \mid C) = \frac{\sum_{t=1}^T \ind{x_t(k) = i \wedge
          y_t = C}} {\sum_{t=1}^T \ind{y_t = C}} = \frac{N_k(i,
        C)}{N(C)},
    \]
    where $N_k(i, C)$ is the numb l l .er of examples in class $C$
    whose $k$-th attribute has the value $i$, and $N(C)$ is the number
    of examples in class $C$.
  \end{block}

  \only<article>{ Sometimes we need to be able to deal with cases
    where there are no examples at all of one class. In that case,
    that class would have probability zero. To get around this
    problem, we add ``fake observations'' to our data. This is called
    \emph{Laplace smoothing}.
    \begin{remark} In Laplace smoothing with constant $\lambda$, our
      probability model is
      \[
        \Pr(x(k) = i \mid C) = \frac{\sum_{t=1}^T \ind{x_t(k) = i \wedge
            y_t = C} + \lambda} {\sum_{t=1}^T \ind{y_t = C} + n_k
          \lambda} = \frac{N_k(i, C) + \lambda}{N(C) + n_k \lambda}.
      \]
      where $n_k$ is the number of values that the $k$-th attribute
      can take. This is necessary, because we want
      $\sum_{i=1}^{n_k} \Pr(x(k) = i \mid C) = 1$. (You can check that
      this is indeed the case as a simple exercise).
    \end{remark}

    \begin{remark}
      In fact, the Laplace smoothing model corresponds to a so-called
      Dirichelt prior over polynomial parameters with a marginal
      probability of observation equal to the Laplace smoothing. We
      shall see more of this in the Bayesian inference section.
    \end{remark}
  }
\end{frame}


\begin{frame}
  \begin{block}{Continuous attribute model.}
    \only<article>{Here we can use a Gaussian model for each
      continuous dimension.}
    \[
      \Pr(x(k) = v \mid C) = \frac{1}{\sigma \sqrt{2 \pi}} e^{\frac{(v -
          \mu)^2}{\sigma^2}},
    \]
    where $\mu$ and $\sigma$ are the mean and variance of the
    Gaussian, typically calculated from the training data as:
    \begin{align*}
      \mu &=   \frac{\sum_{t=1}^T x_t(k) \ind{y_t = C}}
            {\sum_{t=1}^T \ind{y_t = C}},
    \end{align*}
    i.e. $\mu$ is the mean of the $k$-th attribute when the label is
    $C$ and
    \begin{align*}
      \sigma &=   \frac{\sum_{t=1}^T [x_t(k) - \mu]^2 \ind{y_t = C}}
               {\sum_{t=1}^T \ind{y_t = C}},
    \end{align*}
    i.e. $\sigma$ is the variance of the $k$-th attribute when the
    label is $C$.  Sometimes we can just fix $\sigma$ to a constant
    value, i.e. $\sigma = 1$.
  \end{block}
\end{frame}

\begin{frame}
  \begin{alertblock}{Estimates versus true probabilities}
    Remember that the probabilities we get from this calculation are
    only \alert{estimates}. We do not really know the probabilities of
    each observation given the classes: we are only estimating them
    from the data. It is also possible that our assumption about the
    independence of features is completely wrong.
  \end{alertblock}
\end{frame}
 % naive Bayes classifiers

\chapter{Reproducibility}
\label{ch:reproducibility}
\section{Reproducibility}
\only<presentation>{
  \begin{frame}
    \tableofcontents[ 
    currentsection, 
    hideothersubsections, 
    sectionstyle=show/shaded
    ] 
  \end{frame}
}
\only<article>{
  One of the main problems in science is reproducibility: when we are trying to draw conclusions from one specific data set, it is easy to make a mistake. For that reason, the scientific process requires us to use our conclusions to make testable predictions, and then test those predictions with new experiments.}


\begin{frame}
  \frametitle{Reproducibility}
  \only<2>{\includegraphics[width=\textwidth]{../figures/2016-election}}
  \only<article>{A simple example is the 2016 election. While we can make models}
  
\end{frame}
\only<article>{The same thing can be done in when dealing purely with data, by making sure we use some of the data as input to the algorithm, and other data to measure the quality of the algorithm itself. In the following, we assume we have some algorithm $\alg : \Datasets \to \CY$, where $\Datasets$ is the universe of possible input data and $\CY$ the possible outputs, e.g. all possible classification rules. We also assume the existence of some quality measure $U$.
}



\begin{frame}
  \begin{figure}[H]
    \begin{center}
      \begin{tikzpicture}[line width=2pt]
        \node<1->[select,label=above:Data Collection] at (0,2) (experiment) {$\chi$};
        \node<3->[select,label=below:{Algorithm, hyperparameters}] at (0,0) (alg) {$\alg$};
        \node<2->[RV,label=above:Training] at (4,2) (training) {$\Training$};
        \node<5->[RV,label=above:Holdout] at (8,2) (holdout) {$\Holdout$};
        \draw<2->[blue,->] (experiment) -- (training);
        \draw<6->[blue,->] (experiment) to [bend left=45] (holdout);
        \node<4->[RV,label=below:Classifier] at (4,0) (pol) {$\pol$};
        \draw<4->[red,->,dashed] (experiment) -- (pol);
        \draw<4->[red,->] (alg) -- (pol);
        \draw<4->[red,->] (training) -- (pol);
        \node<7->[utility,label=below:Measurement] at (8,0) (util) {$\util$};
        \draw<7->[red,->] (pol) -- (util);
        \draw<7->[red,->] (holdout) -- (util);
      \end{tikzpicture}
    \end{center}
    \caption{The decision process in classification.}
  \end{figure}
  \only<article>{One can think of the decision process in classification as follows. First, we decide to collect some data according to some experimental protocol $\chi$. We also decide to use some algorithm (with associated hyperparameters) $\alg$ together with data $\Training$ we will obtain from our data collection in order to obtain a classification policy $\pol$. Typically, we need to measure the quality of a policy according to how well it classifies on unknown data. This is because our policy has been generated using $\Training$, and so any measurement of its accuracy is going to be biased.}
  \uncover<5->{
    \begin{block}{Classification accuracy}
      \[
      \E_\chi[\util(\pol)] = \sum_{x,y} \underset{\textrm{Data probability}}{\underbrace{\Pr_\chi(x, y)}} \overset{\textrm{Decision probability}}{\overbrace{\pol(a = y \mid x)}}
      \]
      \only<article>{The classification accuracy of policy $\poL$ under $\chi$ is the expected number of times the policy decides $\pol$ chooses the correct class.}
    \end{block}
  }
\end{frame}

\subsection{The human as an algorithm}
\begin{frame}
  \frametitle{The human as an algorithm.}
  \only<article>{The same way with which an algorithm creates a model from some prior assumptions and data, so can a human select an algorithm and associated hyperparamters by executing an algorithm herself. This involves trying different algorithms and hyperparametrs on the same training data $\Training$ and then measuring their performance in the holdout set $\Holdout$.}
  \centering
  \begin{tikzpicture}[line width=2pt]
    \node[select,label=above:Data Collection] at (0,2) (experiment) {$\chi$};
    \node[RV,label=above:Training] at (4,2) (training) {$\Training$};
    \node[RV,label=above:Holdout] at (8,2) (holdout) {$\Holdout$};
    \draw[blue,->] (experiment) -- (training);
    \draw[blue,->] (experiment) to [bend left=45] (holdout);
    \node<2->[select,label=below:{Algorithm, hyperparameters}] at (0,0) (alg) {$\alg_1$};     
    \node<3->[RV,label=below:Classifier] at (4,0) (pol) {$\pol_1$};
    \draw<3->[red,->] (alg) -- (pol);
    \draw<3->[red,->] (training) -- (pol);
    \node<4->[utility,label=below:Measurement] at (8,0) (util) {$\util_1$};
    \draw<4->[red,->] (pol) -- (util);
    \draw<4->[red,->] (holdout) -- (util);
    \node<5->[select,label=below:{Algorithm, hyperparameters}] at (0,-2) (alg2) {$\alg_2$};
    \node<6->[RV,label=below:Classifier] at (4,-2) (pol2) {$\pol_2$};
    \node<7->[utility,label=below:Measurement] at (8,-2) (util2) {$\util_2$};
    \draw<6->[red,->] (alg2) -- (pol2);
    \draw<6->[red,->] (training) to [bend left] (pol2);
    \draw<7->[red,->] (pol2) -- (util2);
    \draw<7->[red,->] (holdout) to [bend right] (util2);
  \end{tikzpicture}

\end{frame}

\begin{frame}
  \frametitle{Holdout sets}
  \only<article>{To summarise, holdout sets are used in order to be able to evaluate the performance of specific algorithms, or hyparameter selection.}
  \begin{itemize}
  \item Original data $\Data$, e.g. $\Data = (x_1, \ldots, x_T)$.
  \item Training data $\Training \subset \Data$, e.g. $\Training = x_1, \ldots, x_n$, $n < T$.
  \item Holdout data $\Holdout = D \setminus \Training$, used to measure the quality of the result.
  \item Get algorithm output $\pol = \alg(\Training)$.
  \item Calculate quality of output $U(\pol, \Holdout)$
  \end{itemize}
  \only<article>{
    As typically algorithms are maximising the quality metric on the training data, 
    \[
    \alg(\Training) = \argmax_y U(y, \Training)
    \]
    we typically obtain a biased estimate, which depends both on the algorithm itself and the training data. For \KNN{} in particular, when we measure accuracy on the training data, we can nearly always obtain near-perfect accuracy.\footnote{But not always perfect. Can you explain why?}

    However, using the holdout set in order to select the best $\alg$ means that we must measure performance on another sample.
  }
\end{frame}


\subsection{Algorithmic sensitivity}
\only<article>{The algorithm's output does have a dependence on its input, obviously. So, how sensitive is the algorithm to the input?}
\begin{frame}
  \frametitle{Independent data sets}
  \only<article>{One simple idea is to just collect independent datasets and see how the output of the algorithm changes when the data changes. However, this is quite expensive, as it not might be easy to collect data in the first place.}
  \centering
  \begin{tikzpicture}[line width=2pt]
    \node[select,label=above:Experiment] at (0,0) (experiment) {$\chi$};
    \node[select,label=below:{Algorithm}] at (8,0) (alg) {$\alg$};
    \node[RV,label=below:1st sample] at (4,0) (sample1) {$D_1$};
    \node[RV,label=below:1st Result] at (6,0) (pol1) {$\pol_1$};
    \node<2>[RV,label=above:2nd Sample] at (4,2) (sample2) {$D_2$};
    \node<2>[RV,label=above:2nd Result] at (6,2) (pol2) {$\pol_2$};
        \draw[blue,->] (experiment) -- (sample1);
    \draw[red,->] (alg) -- (pol1);
    \draw[red,->] (sample1) -- (pol1);
    \draw<2>[blue,->] (experiment) -- (sample2);
    \draw<2>[red,->] (alg) -- (pol2);
    \draw<2>[red,->] (sample2) -- (pol2);
  \end{tikzpicture}
\end{frame}
\begin{frame}
  \frametitle{Bootstrap samples}
  \only<article>{A more efficient idea is to only collect one dataset, but then use it to generate more datasets. The simplest way to do that is by sampling with replacement from the original dataset, new datasets of the same size as the original. Then the original dataset is sufficiently large, this is approximately the same as sampling independent datasets.}
  \centering
  \begin{tikzpicture}[line width=2pt]
    \node[select,label=above:Experiment] at (0,0) (experiment) {$\chi$};
    \node[RV,label=below:training] at (2,0) (training) {$\Training$};
    \draw[blue,->] (experiment) -- (training);
    \node[select,label=below:{Algorithm}] at (8,0) (alg) {$\alg$};
    \node[RV,label=below:1st sample] at (4,0) (sample1) {$D_1$};
    \node[RV,label=below:1st Result] at (6,0) (pol1) {$\pol_1$};
    \node[RV,label=above:2nd Sample] at (4,2) (sample2) {$D_2$};
    \node[RV,label=above:2nd Result] at (6,2) (pol2) {$\pol_2$};
    \draw[red,->] (alg) -- (pol1);
    \draw[red,->] (sample1) -- (pol1);
    \draw[red,->] (training) -- (sample1);
    \draw[red,->] (training) -- (sample2);
    \draw[red,->] (alg) -- (pol2);
    \draw[red,->] (sample2) -- (pol2);
  \end{tikzpicture}
  \only<article>{As usual, we can evaluate our algorithm on an independent holdout set.}
\end{frame}



\begin{frame}
  \frametitle{Bootstrapping}
  Bootstrapping is a general technique that can be used to.
  \begin{itemize}
  \item Estimate the sensitivity of $\alg$ to the data $x$
  \item Obtain a distribution of estimates $y$ from $\alg$ and the data $x$.
  \end{itemize}
  \begin{block}{Bootstrapping}
    \begin{itemize}
    \item \textbf{Input} Training data $\Training$, number of samples $k$.
    \item \textbf{For} $i = 1, \ldots, k$
    \item \quad $\Data^{(i)} = \textrm{Bootstrap}(\Training)$
    \item \quad $y_i = \alg(\Data^{(i)})$
    \item \textbf{return} $\{y_1, \ldots, y_i\}$.
    \end{itemize}
    where  $\textrm{Bootstrap}(\Training)$ samples with replacement $|\Training|$ points.
  \end{block}

\end{frame}
\only<article>{
  In general, it is worthwhile to have some indication of how certain we should be about our prediction. Bayesian inference offers a principled way to do this.
}




\subsection{Bayesian credible intervals}


\subsection{Model mismatch}

\subsection{Boot-strapping}

\subsection{Cross-validation}

\subsection{Independent replication}



%%% Local Variables:
%%% mode: latex
%%% TeX-master: "notes"
%%% End:





\chapter{Causality}
\label{ch:causality}
\section{Introduction}
\only<presentation>{
  \begin{frame}
    \tableofcontents[ 
    currentsection, 
    hideothersubsections, 
    sectionstyle=show/shaded
    ] 
  \end{frame}
}

\begin{frame}
  \frametitle{Headaches and aspirins}
  \only<article>{Causal questions do not just deal with statistical relationships. The meaning of these questions is slightly different depending on whether we are talking about the population at large, or a specific individual. For populations, the main question is whether or not our actions have a causal effect. In observational data, we also need to consider the \emph{direction of causation}.}
  \begin{example}[Population effects]
    \begin{figure}[H]
      \centering
      \begin{subfigure}{\fwidth}
        \centering
        \begin{tikzpicture}[scale=0.5]
          \begin{axis}[xmin=0,xmax=5, xlabel={Dose}, ylabel={Response} ]
            % use TeX as calculator:
            \addplot[color=blue] {1/(1 + e^(1-x))};
            \addplot[color=red] {1/(1 + e^(3-x))};
            \legend{Cured, Side-effects}
          \end{axis}
        \end{tikzpicture}
        \caption{Dose-response curve.}
        \label{fig:dose-response}
      \end{subfigure}
      \begin{subfigure}{\fwidth}
        \centering
        \begin{tikzpicture}[scale=0.5]
          \begin{axis}[xmin=-2,xmax=2, xlabel={Sensitivity}, ylabel={Response}, samples=50 ]
            % use TeX as calculator:
            \addplot[color=blue] {e^(-x^2))};
            \addplot[color=red] {0.5*e^(-4*x^2)};
            \legend{High dose, Low dose}
          \end{axis}
        \end{tikzpicture}
        \caption{Response distribution}
        \label{fig:dose-response}
      \end{subfigure}
      \caption{Investigation the response of the population to various doses of the drug. }
    \end{figure}
    \only<article>{
      We can ask ourselves two different questions about the effect of population effect aspirin on headaches.
    }
    \begin{itemize}
    \item Is aspirin an effective cure for headaches?
    \item Does having a headache lead to aspirin-taking?
    \end{itemize}
  \end{example}
\end{frame}
\begin{frame}
  \only<article>{For individuals, the first question is,  what is the possible effect of our actions? This is called the \emph{effect of causes}. The second question is, what was the reason for something happening? That is called the \emph{cause of effects?} }
  \begin{example}[Individual effects]
    \begin{figure}[H]
      \centering
      \begin{subfigure}{\fwidth}
        \includegraphics[width=\fwidth]{../figures/aspirin}
      \end{subfigure}
      \begin{subfigure}{\fwidth}
        \includegraphics[width=\fwidth]{../figures/450px-Migraine.jpg}
      \end{subfigure}
    \end{figure}
    \only<article>{
      We can ask ourselves two different questions about the individual effect of aspirin on headaches.
    }
    \begin{itemize}
    \item Effects of \alert{Causes}: Will \alert{my} headache pass \alert{if I take} an aspirin?
    \item \alert{Causes} of Effects: Would \alert{my} headache have passed if I had \alert{not taken} an aspirin?
    \end{itemize}
  \end{example}
  \only<article>{In order to be able to meaningfully talk about effects and causes we must also introduce decisions. Formally, there is nothing different in the decisions in this section and those introduced in Section~\ref{sec:decision-problems}. However, in this case we will try and use decisions to model outside interventions in a ``natural'' system, whereby a \emph{null} decision means that we do not intervene.}
\end{frame}

\begin{frame}
  \frametitle{Overview}
  \begin{block}{Inferring causal models}
    We can distinguish different \alert{models} from observational or experimental data.
  \end{block}

  \begin{block}{Inferring individual effects}
    The effect of possible intervention on an individual is not generally determinable. We usually require strong assumptions.
  \end{block}
  
  \begin{block}{Decision-theoretic view}
    There are many competing approaches to causality. We will remain within the decision-theoretic framework, which allows us to crisply define both our knowledge and assumptions.
  \end{block}
\end{frame}

\begin{frame}
  \frametitle{What causes what?}
  \begin{example}
    \begin{figure}[H]
      \centering
      \begin{subfigure}{\fwidth}
        \centering
        \begin{tikzpicture}
          \node[RV, hidden] at (0,1) (p) {$\param$};
          \node[RV] at (-1,0) (x1) {$a_t$};
          \node[RV] at (1,0) (x2) {$x_t$};
          \draw[->] (p) -- (x1);
          \draw[->] (p) -- (x2);
          \draw[->] (x1) -- (x2);
        \end{tikzpicture}
        \caption{Independence of $a_t$.}
      \end{subfigure}
      \begin{subfigure}{\fwidth}
        \centering
        \begin{tikzpicture}
          \node[RV, hidden] at (0,1) (p) {$\param$};
          \node[RV] at (-1,0) (x1) {$a_{t}$};
          \node[RV] at (1,0) (x2) {$x_{t}$};
          \draw[->] (p) -- (x1);
          \draw[->] (p) -- (x2);
          \draw[->] (x2) -- (x1);
        \end{tikzpicture}
        \caption{Independence of $x_t$.}
      \end{subfigure}
    \end{figure}
    Suppose we have data $x_{t}, a_{t}$ where
    \begin{itemize}
    \item $x_{t}$: lung cancer
    \item $a_{t}$: smoking
    \end{itemize}
    Does smoking cause lung cancer or does lung cancer make people smoke? Can we compare the two models above to determine it?
  \end{example}
  \only<article>{The answer is no. Let us consider two different parametrisations of the distribution. One in which $a_t$ generates $x_t$, and the converse, for any given parameter value $\param$, as given below:}
  \uncover<2>{
    \[
    P_\param(D) =
    \prod_t P_\param(x_t, a_t)
    = 
    \prod_t P_{\param'}(x_t \mid a_t) P_{\param'}(a_t)
    = 
    \prod_t P_{\param''}(a_t \mid x_t) P_{\param''}(x_t).
    \]
  }
  \only<article>{In particular, for any parametrisation $\param$ of the joint distribution, there exists a $\param'$ and $\param''$ giving the same probabilities for all $x_t, a_t$. For the example above, we can look at Bernoulli distributions for both variables so that $P_\param(x_t = x, a_t = a) = \param_{x,a}$. Then
    \begin{align*}
      \param'_a &= \sum_x \param_{x,a},
      &
        \param'_{x|a} &= \param_{x,a} / \param'_a
      \\
      \param''_x &= \sum_a \param_{x,a},
      &
        \param'_{a|x} &= \param_{x,a} / \param''_x.
    \end{align*}
    This means we can define prior distributions $\bel, \bel', \bel''$ in these three parameter spaces that give exactly the same results, e.g. by modelling each parameter as an independent Beta distribution. So, clearly simply looking at a simple graphical model does not let us answer this question.}
\end{frame}

\subsection{Decision diagrams}
\only<article>{
  However, graphical models \alert{can} be extended  to model causal relations. In particular, we can use \emph{decision diagrams}\footnote{Otherwise called influence diagrams}, which include not only random variables, but also \emph{decision} variables, denoted with squares, as well as utility variables, denoted via diamonds. In the following examples, we assume there are some underlying distributions specified by parameters $\param$, which we include in the diagrams for clarity. Even though it may seem intuitively sensible to suppose it, the arrow directions in the diagrams \emph{do not} indicate direct causes. The only important thing for determining whether some variable influences another is whether or not there is independence between the corresponding decision and random variables.}
\begin{frame}
  \begin{figure}[H]
    \centering
    \begin{tikzpicture}
      \node[RV, hidden] at (-1,1) (p) {$\param$};
      \node[RV] at (0,0) (x) {$x_t$};
      \node[RV] at (1,1) (y) {$y_t$};
      \only<1,2>{
        \node[select] at (2,0) (a) {$a_t$};
      }
      \only<3->{
        \node[RV] at (2,0) (a) {$a_t$};
      }
      \draw[->] (x)--(y);
      \draw[->] (x)--(a);
      \draw[->] (a)--(y);
      \draw[->] (p) to (x);
      \draw[->] (p)--(y);
      \onslide<3->{
        \node[select] at (4,0) (pol) {$\pol$};
        \draw[->] (pol)--(a);
      }
      \onslide<2->{
        \node[utility] at (3,1) (u) {$\util$};
        \draw[->] (a)--(u);
        \draw[->] (y)--(u);
      }
    \end{tikzpicture}
    \caption{A typical decision diagram where $x_t$: individual information, $y_t$: individual result, $a_t$: action, $\pol$: policy}
    \label{fig:decision-diagram}
    \index{policy!intervention}
  \end{figure}
  \only<3>{
    \begin{example}[Taking an aspirin]
      \only<article>{The diagram in Figure~\ref{fig:decision-diagram} does not completely specify the decision problem. For aspirin taking, we can define the following variables:}
      \begin{itemize}
      \item Individual $t$
      \item Individual information $x_t$
      \item $a_t  = 1$ if $t$ takes an aspirin, and $0$ otherwise.
      \item $y_t = 1$ if the headache is cured in 30 minutes, $0$ otherwise.
      \item $\pol$: intervention policy.\index{policy!intervention}
      \end{itemize}
    \end{example}
  }
  \only<4>{
    \begin{example}[A recommendation system]
      \only<article>{Consider the example of a recommendation system, where we have data of the form $(x_t, a_t, y_t)$. The performance of the recommendation system depends not only on the parameter $\param$, but also on the chosen policy $\pol$. }
      \begin{itemize}
      \item $x_t$: User information (random variable)
      \item $a_t$: System action (random variable)
      \item $y_t$: Click (random varaible)
      \item $\pol$: recommendation policy (decision variable).\index{policy!recommendation}
      \end{itemize}
    \end{example}
  }
  \only<article>{In both cases, there are some questions we can ask using the underlying model. The dependency structure is not enough to know \emph{a priori} whether we can obtain meaningful answers. This depends on the specific assumptions we make about the model.}
\end{frame}

\begin{frame}
  \frametitle{Conditional distributions and decision variables.}
  \only<article>{We begin with a parenthesis on conditional distributions. We normally define the conditional distribution of $A$ given $B$ under a probability measure $P$ as:}
  \[
  P(A \mid B) \defn \frac{P(A \cap B)}{P(B)}.
  \]
  \only<article>{However, decision variables are outside the scope of this probability measure, and yet we need to define conditional distributions using them. }
  \begin{block}{The conditional distribution of decisions}
    \only<article>{If $\pol \in \Pol$ is a decision variable, we represent the conditional distribution of any random variable $a$ given $\pol$ simply as a collection of probability measures $\cset{\pol(a)}{\pol \in \Pol}$, one for each possible value $\pol$. The following notations will be equivalent:}
    \[
    \pol(a) \equiv \Pr^\pol(a) \equiv \Pr(a \mid \pol).
    \]
    \only<article>{The reader should note that the standard definition of a conditional distribution also $P(A \mid B)$ creates a collection of distributions on $A$, with elements $P_B(A)$. However, it also specifies a rule for doing so from the complete distribution $P$. 

      If the random variables $a$ also depends on some probability law $P_\param$, then it will be convenient to use the notation
    }
    \[
    \Pr_\param^\pol(a) \equiv \Pr(a \mid \param, \pol).
    \]
  \end{block}
\end{frame}
\subsection{Common structural assumptions}
\only<article>{
  In order to be clear about what constitutes an observation by the experimenter and what is a decision, we must clearly separate random variables from decision variables. The individual actions may be random variables, but they will depend on decisions taken. As we will see later, this is useful for modelling interventions.}

\begin{frame}
  \frametitle{Basic causal structures}
  \only<article>{Directed graphical models are not sufficient to determine causality by themselves, as they only determine correlations between random variables. If we have decision variables, however, we can always determine whether or not our decisions influence outcomes.}
  \begin{block}{Non-cause}
    \begin{figure}[H]
      \centering
      \begin{tikzpicture}
        \node[select] at (0,0) (p) {$\pol$};
        \node[RV] at (1,0) (a) {$a_t$};
        \node[RV] at (2,0) (y) {$y_t$};
        \draw[->] (p) to (a);
        \draw[->] (y) to (a);
        \uncover<2>{
          \node[RV, hidden] at (1,1) (param) {$\param$};
          \draw[->] (param) to (y);
        }
      \end{tikzpicture}
      \caption{$\pol$ does not cause $y$}
      \label{fig:non-cause}
    \end{figure}
    \only<article>{In the diagram above, we see that $y_t \indep \pol$.}
  \end{block}
  \only<article>{
    \begin{example}
      Consider the model
      \begin{align*}
        y_t &\sim \Normal(0,1)\\
        a_t \mid y_t, \pol &\sim \Normal(y_t + \pol, 1)
      \end{align*}
      \begin{figure}[H]
        \centering
        \includegraphics[width=0.5\textwidth]{../src/causality/non-cause}
        \caption{$\Pr^\pol(y_t)$ for $\pol \in \{-1, 1\}$ when $\pol$ is not a cause for $y_t$}
        \label{fig:non-cause-dist}
      \end{figure}
      In this example, we see tht $y_t$ is independent of the policy
      $\pol$.  However, $y_t$ is not independent of the action taken, as the action depends on $y_t$ directly. The correlation between $y, a$ is shown in Figure~\ref{fig:a-y-correlation:non-cause}.
    \end{example}
  }
  
  \begin{block}{No confounding}
    \only<article>{Confounding is a term that indicates the existence of latent variables that create dependencies between $y_t, \pol, a_t$. We are sure that there is no confounding whenever $y_t \indep \pol \mid a_t$, as captured by the diagram in Figure~\ref{fig:no-confounding}. In this case $\pol$ is a direct cause for $y_t$ through $a_t$.}
    \begin{figure}[H]
      \centering
      \begin{tikzpicture}
        \node[select] at (0,0) (p) {$\pol$};
        \node[RV] at (1,0) (a) {$a_t$};
        \node[RV] at (2,0) (y) {$y_t$};
        \draw[->] (p) to (a);
        \draw[->] (a) to (y);
        \uncover<2>{
          \node[RV, hidden] at (1,1) (param) {$\param$};
          \draw[->] (param) to (y);
        }
      \end{tikzpicture}
      \caption{No confounding: $\pol$ causes $y_t$}
      \label{fig:no-confounding}
    \end{figure}
  \end{block}

  \only<article>{
    \begin{example}
      Consider the model
      \begin{align*}
        a_t &\sim \Normal(\pol,1)\\
        y_t \mid a_t, \pol &\sim \Normal(a_t, 1)
      \end{align*}
      \begin{figure}[H]
        \centering
        \includegraphics[width=0.5\textwidth]{../src/causality/direct-cause}
        \caption{$\Pr^\pol(y_t)$ for $\pol \in \{-1, 1\}$ when $\pol$ is a direct cause for $y_t$}
        \label{fig:non-cause-dist}
      \end{figure}
      \only<article>{We can see how the distribution of $y_t$ changes when $\pol$ changes in Figure~\ref{fig:non-cause-dist}. In this case there is also a correlation between $a_t, y_t$ as seen in Figure~\ref{fig:a-y-correlation}.}
    \end{example}
  }

  \only<article>{
    \begin{figure}[H]
      \centering
      \begin{subfigure}{0.45\textwidth}
        \includegraphics[width=0.9\textwidth]{../src/causality/a-y-non-cause}
        \caption{Non-cause}
        \label{fig:a-y-correlation:non-cause}
      \end{subfigure}
      \hspace{1em}
      \begin{subfigure}{0.45\textwidth}
        \includegraphics[width=0.9\textwidth]{../src/causality/a-y-direct-cause}
        \caption{Cause}
        \label{fig:a-y-correlation:cause}
      \end{subfigure}
      \caption{Correlation between $a_t$ and $y_t$}
      \label{fig:a-y-correlation}
    \end{figure}
  }
\end{frame}
\begin{frame}
  \frametitle{Covariates}
  \begin{block}{Sufficient covariate}
    \only<article>{Sometimes the variable of interest is not conditionally independent of the treatment, unless there exists a \emph{sufficient covariate} $x_t$ such that
      $y_t \indep \pol \mid a_t, x_t$. If $x_t$ is not observed, then it is sometimes called a confounder.}
    \begin{figure}[H]
      \centering
      \begin{tikzpicture}
        \node[select] at (0,0) (p) {$\pol$};
        \node[RV] at (1,0) (a) {$a_t$};
        \node[RV] at (2,0) (y) {$y_t$};
        \node[RV] at (2,1) (x) {$x_t$};
        \draw[->] (p) to (a);
        \draw[->] (a) to (y);
        \draw[->] (x) to (a);
        \draw[->] (x) to (y);
        \uncover<2>{
          \node[RV, hidden] at (3,1) (param) {$\param$};
          \draw[->] (param) to (x);
          \draw[->] (param) to (y);
        }
      \end{tikzpicture}
      \caption{Sufficient covariate $x_t$}
      \label{fig:sufficient-covariate}
    \end{figure}
  \end{block}

  \only<article>{
    \begin{example}
      Consider the model
      \begin{align*}
        x_t &\sim \Normal(0, 1)\\
        a_t &\sim \Normal(x_t + \pol, 1)\\
        y_t \mid a_t, \pol &\sim \Normal(x_t + a_t, 1),
      \end{align*}
      \only<article>{Here $x_t$ influences the outcome $y_t$, but also directly influences $a_t$ through the policy $\pol$. As we can see in Figure~\ref{fig:non-cause-dist}, the policy then has an influence on $y_t$}
      \begin{figure}[H]
        \centering
        \includegraphics[width=0.5\textwidth]{../src/causality/sufficient}
        \caption{$\Pr^\pol(y_t)$ for $\pol \in \{-1, 1\}$ when $\pol$ is a direct cause for $y_t$}
        \label{fig:sufficient-covariate-dist}
      \end{figure}
    \end{example}
  }


  \begin{block}{Instrumental variables and confounders}
    \only<article>{If the sufficient covariate $x_t$ is not observed, we may still have another variable available, $z_t$, on the basis of which we make our decisions. This is called an \emph{instrumental variable.} \index{instrument@see{variable!instrumental}}\index{variable!instrumental}. In this case $z_t$ still depends on $x_t$ and the effect of the treatment depends on $x_t$ directly. As $x_t$ is a latent covariate, it can be called a \emph{confounder.} \index{confounder}}
    \begin{figure}[H]
      \centering
      \begin{tikzpicture}
        \node[select] at (0,0) (p) {$\pol$};
        \node[RV] at (1,0) (a) {$a_t$};
        \node[RV] at (2,0) (y) {$y_t$};
        \node[RV, hidden] at (2,1) (x) {$x_t$};
        \node[RV] at (1,1) (z) {$z_t$};
        \draw[->] (p) to (a);
        \draw[->] (a) to (y);
        \draw[->] (x) to (y);
        \draw[->] (x) to (z);
        \draw[->] (z) to (a);
        \uncover<2>{
          \node[RV, hidden] at (3,1) (param) {$\param$};
          \draw[->] (param) to (x);
          \draw[->] (param) to (y);
          \draw[bend right=45,->] (param) to (z);
        }
      \end{tikzpicture}
      \caption{Instrumental variable $z_t$}
      \label{fig:instrumental-variable}
    \end{figure}
  \end{block}

  \only<article>{
    \begin{example}
      Consider the model
      \begin{align*}
        x_t &\sim \Normal(0, 1)\\
        z_t &\sim \Normal(x_t, 1)\\
        a_t &\sim \Normal(z_t + \pol, 1)\\
        y_t \mid a_t, \pol &\sim \Normal(x_t + a_t, 1)
      \end{align*}
      \only<article>{In this scenario, $x_t$ directly influences the outcome $y_t$, but is not observed.\footnote{Hence, it can be called a confounder.}}
      \begin{figure}[H]
        \centering
        \includegraphics[width=0.5\textwidth]{../src/causality/instrumental}
        \caption{$\Pr^\pol(y_t)$ for $\pol \in \{-1, 1\}$ when $\pol$ is a direct cause for $y_t$}
        \label{fig:instrumental-dist}
      \end{figure}
    \end{example}
  }


\end{frame}


\section{Interventions}
\only<article>{Interventions are of primary interest when we have a set of observational data, collected under a \emph{null} or \emph{default} policy $\pol_0$.  We then wish to intervene with some policy $\pol$ in order to maximise our utility function, or to simply try and estimate the exact relationships between variables.}
\begin{frame}
  \frametitle{Modelling interventions}
  \begin{itemize}
  \item Observational data $D$. \only<article>{This represents data we have collected from some previous regime. In order to be able to model the problem precisely, we must posit the existence of some default policy $\pol_0$ under which the data was collected.}
  \item Policy space $\Pol$. \only<article>{This must include $\pol_0$, as well as any other policies that the decision maker may be able to choose in the future.} 
  \end{itemize}
  \begin{block}{Default policy}
    \index{policy!default}
    The space of policies $\Pol$ includes a \alert{default policy} $\pol_0$, under which the data was collected. \only<article>{The policy $\pol_0$ might already be known, if for example the data was collected with a specific algorithm. However, frequently $\pol_0$ is not given, and must also be inferred from the data. In that case, it can be seen as an additional parameter, complementary to $\param$.}
  \end{block}
  \begin{block}{Intervention policies}
    \index{policy!intervention}
    Except $\pol_0$, policies $\pol \in \Pol$ represent different interventions specifying a distribution $\pol(a_t \mid x_t)$.
    \begin{itemize}
    \item Direct interventions. \only<article>{The simplest scenario is when we are able to choose a $\pol$ for which we know $\pol(a_t \mid x_t)$. This counts as a direct intervention, as we can specify any distribution of actions allowed in $\Pol$. If $\Pol$ includes all conditional distributions, we can select an arbitrary action for every individual. This assumption is plausible for algorithmic decision making such as recommendation systems, but implausible when the actions are taken by another agent, such as a human.}
    \item Indirect interventions and non-compliance. \only<article>{In this scenario we assume that, while we are free to choose policies from $\Pol$, we do not know what distribution $\pol(a_t \mid x_t)$ each policy specifies. In algorithmic decision making, this occurs whenever $\pol$ represents hyperparameters and algorithms for which we have insufficient information. Then policies must be evaluated through some type of black-box (e.g. A/B) testing. When the actions are taken by (human) agents, the policy implies making a recommendation, which may not be followed by the agent. If we denote the recommendation by $v_t$, then we can write $\pol(a_t \mid x_t) = \sum_{z_t} \pol(a_t \mid z_t, x_t) \pol(z_t \mid x_t)$. In this scenario we can freely specify $\pol(z_t \mid x_t)$, but $\pol(a_t \mid z_t, x_t)$ must be estimated. For that reason, it is usually simpler to simply marginalise out $a_t$. But perhaps the simplest approach is to consider non-compliance as a confounder $x_t$, and $z_t$ as an instrumental variable. }
    \end{itemize}
  \end{block}
\end{frame}
\begin{frame}
  \begin{example}[Weight loss]
    \only<article>{Consider weight loss. We can collect observational data from a population of overweight adults over a year. We can imagine that $x$ represents the weight and vital statistics of an individual and $y$ their change in weight after a year. We may also observe their individual actions $a$, such as whether or not they are following a particular diet or exercise regime. Under the default policy $\pol_0$, their actions are determined only the individuals. Consider an alternative policy $\pol$, which prescribes diet and exercise regimes. Due to non-compliance, actual actions taken by individuals may differ from prescribed actions. In addition, actions might not be observed.}
    \only<presentation>{
      \only<1>{
        \begin{figure}[H]
          \centering
                \begin{tikzpicture}
        \node[RV, hidden] at (0,1) (p) {$\param$};
        \node[RV] at (0,0) (x) {$x_t$};
        \node[RV] at (1,1) (y) {$y_t$};
        \node[RV] at (2,0) (a) {$a_t$};
        \draw[->] (p)--(x);
        \draw[->] (p)--(y);
        \draw[->] (x)--(y);
        \draw[->] (x)--(a);
        \draw[->] (a)--(y);
        \node[select] at (4,0) (p) {$\pol$};
        \draw[->] (p)--(a);
        \node[utility] at (3,1) (u) {$\util$};
        \draw[->] (a)--(u);
        \draw[->] (y)--(u);
      \end{tikzpicture}

%%% Local Variables:
%%% mode: latex
%%% TeX-master: "notes"
%%% End:

        \end{figure}
      }
    }
    \only<2>{
      \begin{figure}[H]
        \centering
        \begin{tikzpicture}
          \node[RV, hidden] at (0,1) (p) {$\param$};
          \node[RV] at (0,0) (z) {$z_t$};
          \node[RV, hidden] at (1,2) (x) {$x_t$};
          \node[RV] at (1,1) (y) {$y_t$};
          \node[RV] at (2,0) (a) {$a_t$};
          \draw[->] (p)--(z);
          \draw[->] (p)--(x);
          \draw[->] (p)--(y);
          \draw[->] (x)--(y);
          \draw[->] (z)--(y);
          \draw[->] (z)--(a);
          \draw[->] (a)--(y);
          \node[select] at (4,0) (p) {$\pol$};
          \draw[->] (p)--(a);
          \node[utility] at (3,1) (u) {$\util$};
          \draw[->] (a)--(u);
          \draw[->] (y)--(u);
        \end{tikzpicture}
        \caption{Model of non-compliance as a confounder.}
        \label{fig:non-compliance}
      \end{figure}
    }
  \end{example}
\end{frame}  


\section{Policy evaluation and optimisation}
\index{policy!evaluation}
\begin{frame}
  \frametitle{The value of an observed policy}
  \only<presentation>{
    \begin{figure}[H]
      \centering
            \begin{tikzpicture}
        \node[RV, hidden] at (0,1) (p) {$\param$};
        \node[RV] at (0,0) (x) {$x_t$};
        \node[RV] at (1,1) (y) {$y_t$};
        \node[RV] at (2,0) (a) {$a_t$};
        \draw[->] (p)--(x);
        \draw[->] (p)--(y);
        \draw[->] (x)--(y);
        \draw[->] (x)--(a);
        \draw[->] (a)--(y);
        \node[select] at (4,0) (p) {$\pol$};
        \draw[->] (p)--(a);
        \node[utility] at (3,1) (u) {$\util$};
        \draw[->] (a)--(u);
        \draw[->] (y)--(u);
      \end{tikzpicture}

%%% Local Variables:
%%% mode: latex
%%% TeX-master: "notes"
%%% End:

      \caption{Basic decision diagram}
    \end{figure}
  }
  \only<article>{
    In this section, we will focus on the general model of Figure~\ref{fig:decision-diagram}.
    If we have data $D = \cset{(x_t, a_t, y_t)}{t \in [T]}$ generated from some policy $\pol_0$, we can always infer the average quality of each action $a$ under that policy.}
  \only<2>{
    \begin{align}
      \label{eq:observed-expected-utility}
      \hat{\E}_D(U \mid a) 
      &\defn
        \frac{1}{|\cset{t}{a_t = a}|}
        \sum_{t: a_t = a}
        U(a_t, y_t)\\
      &\approx
        \E^{\pol_0}_\param (U \mid a)
      & (a_t, y_t) & \sim \Pr_\param^{\pol_0}.
    \end{align}
  }
  \only<article>{
    Can we calculate the value of another policy? As we have seen from Simpson's paradox\index{Simpson's paradox}, it is folly to simply select
  }
  \[
  \hat{a}^*_D \in \argmax_a \hat{\E}_D(U \mid a),
  \]
  \only<article>{
    as the action also depends on the observations $x$ through the policy.
    To clarify this, let us look again at the model shown in Figure~\ref{fig:decision-diagram}.
  }
\end{frame}
\begin{frame}

  \begin{align*}
    x_t \mid \param &\sim P_\param(x)\\
    y_t \mid \param, x_t, a_t &\sim P_\param(y \mid x_t, a_t)\\
    a_t \mid x_t, \pol &\sim \pol(a \mid x_t).
  \end{align*}
  \only<article>{
    Assume that $x \in \CX$, a continuous space, but $y \in \CY$ is discrete. In this scenario, then the value of an action under a policy $\pol$ is nonsensical, as it does not really depend on the policy itself:
    \begin{align*}
      \E^\pol_\param(\util \mid a)
      &=
        \int_\CX \dd P_\param(x)
        \sum_{y \in \CY} P_\param(y \mid x, a) \util(a, y).
    \end{align*}
    We see that there is a clear dependence on the distribution of $x$, and there is no dependence on the policy any more. In fact, equation above only tells us the expected utility we'd get if we always chose the same action $a$. But what is the optimal policy? First, we have to define the value of a policy.}
  \begin{block}{The value of a policy}
    \begin{align*}
      \E^\pol_\param(\util)
      &=
        \int_\CX \dd P_\param(x)
        \sum_{a \in \CA}  \pol(a \mid x)  \sum_{y \in \CY} P_\param(y \mid x, a) \util(a, y)
    \end{align*}
  \end{block}
  The optimal policy under a known parameter $\param$ is given simply by
  \begin{align*}
    \max_{\pol \in \Pol} \E^\pol_\param(\util),
  \end{align*}
  where $\Pol$ is the set of allowed policies. 
\end{frame}

\begin{frame}
  \frametitle{Monte-Carlo estimation}
  \only<article>{The simplest method to estimate the value of an alternative policy is to use Monte-Carlo estimation and importance sampling. However, this estimate can have a very high variance if the alternative policy is very different from the original policy.}
  \begin{block}{Importance sampling\footnote{Also known as Propensity Scoring}}
    We can obtain an unbiased estimate of the utility in a model-free manner through importance sampling:
    \only<presentation>{
      \begin{align*}
        \E^\pol_\param(\util)
        &=
          \int_\CX \dd P_\param(x)
          \sum_a
          \E_\param(\util \mid a, x)
          \pol(a \mid x)\\
        &\approx
          \frac{1}{T}
          \sum_{t=1}^T
          \util_t 
          \frac{\pol(a_t \mid x_t)}{\pol_0(a_t \mid x_t)}.
      \end{align*}
    }

    \only<article>{
      \begin{align*}
        \E^\pol_\param(\util)
        &=
          \int_\CX \dd P_\param(x)
          \sum_a
          \E_\param(\util \mid a, x)
          \pol(a \mid x)\\
        &\approx
          \frac{1}{T}
          \sum_a \sum_t
          \E_\param(\util \mid a, x_t)
          \pol(a \mid x_t), & x_t & \sim P_\param(x)\\
        &=
          \frac{1}{T}
          \sum_t \sum_a
          \E_\param(\util \mid a, x_t)
          \frac{\pol(a \mid x_t)}{\pol_0(a \mid x_t)} \pol_0(a \mid x_t)\\
        &\approx
          \frac{1}{T}
          \sum_{t=1}^T
          \util_t 
          \frac{\pol(a_t \mid x_t)}{\pol_0(a_t \mid x_t)},
        & a_t \mid x_t \sim & \pol_0, \quad U_t \mid x_t, a_t \sim P_\param(U_t \mid x_t, a_t)
      \end{align*}
    }
  \end{block}
\end{frame}

\begin{frame}
  \frametitle{Bayesian estimation}
  \only<article>{Unforunately this method has high variance.}
  If we $\pol_0$ is given, we can calculate the utility of any policy to whatever degree of accuracy we wish. \only<article>{We begin with a prior $\bel$ on $\Param$ and obtain the following, assuming the policy $\pol_0$ is stationary.}
  \begin{align*}
    \bel(\param \mid D, \pol_0) &\propto \prod_t \Pr^{\pol_0}_\param(x_t, y_t, a_t)\\
    \E_\bel^\pol(\util \mid D) &= \int_\Param \E_\param^\pol(\util) \dd \bel(\param \mid D)\\
                                &= \int_\Param 
                                  \int_\CX \dd P_\param(x)
                                  \sum_{t=1}^T
                                  \sum_a
                                  \E_\param(\util \mid a, x)
                                  \pol(a \mid x)
                                  \dd \bel(\param \mid D).
  \end{align*}
\end{frame}


\begin{frame}
  \frametitle{Causal inference and policy optimisation}
  \index{policy!optimisation}
  \only<article>{Causal inference requires building a complete model for the effect of both the model parameter $\param$, representing nature, and the policy $\pol$, representing the decision maker. This means that we have to be explicit about the dependencies of random variables on the model and the policy.}
  \only<1>{
    \begin{example}
      \begin{figure}[H]
        \centering
        \begin{tikzpicture}
          \node[RV, hidden] at (0,0) (p) {$\param$};
          \node[RV] at (1,0) (y) {$y_t$};
          \node[RV] at (2,0) (a) {$a_t$};
          \node[select] at (3,0) (pol) {$\pol$};
          \draw[->] (pol)--(a);
          \draw[->] (p)--(y);
          \draw[->] (a)--(y);
          \node[utility] at (2,1) (u) {$\util$};
          \draw[->] (a)--(u);
          \draw[->] (y)--(u);
        \end{tikzpicture}
        \caption{Simple decision problem.}
      \end{figure}
      Let $a_t, y_t \in \{0,1\}$, $\param \in [0,1]^2$ and
      \[
      y_t \mid a_t = a \sim \Bernoulli(\param_a)
      \]
      Then, by estimating $\param$, we can predict the effect of any action.
      \only<article>{
        How can we estimate $\param$ from historical data? We simply have to select the right parameter value.
        Simply put, each choice of $a$ corresponds to one part of the parameter vector. This means that the maximum likelihood estimate 
        \[
        \hat{\param}_a \defn \frac{1}{|\cset{t}{a_t = a}|} \sum_{\cset{t}{a_t = a}} y_t
        \]
        is valid. We can also consider a product-Beta prior $\BetaDist(\alpha^0_a, \beta^0_b)$ for each one of the Bernoulli parameters, so that the posterior after $t$ observations is
        \[
        \alpha^t_a = \alpha^0_a + \sum_{\cset{t}{a_t = a}} y_t, \qquad
        \beta^t_a = \beta^0_a + \sum_{\cset{t}{a_t = a}} (1 - y_t).
        \]
        How can we optimise the policy? 
        Let us parametrise our policy with $\pol(a_t = 1) = w$.

        For a fixed $\param$, the value of the policy is
        \[
        \E^\pol_\param \util = 
        \param_1 w + \param_0 (1 - w)
        \]
        The gradient with respect to w is 
        \[
        \nabla \E^\pol_\param \util = 
        \param_1 - \param_0, 
        \]
        so we can use the update
        \[
        w^{(n+1)} = w^{(n)} + \delta^{(n)} \param_1 - \param_0.
        \]
        \alert{However}, $w \in [0,1]$, which means our optimisation must be constrained. Then we obtain that
        $w = 1$ if $\param_1 > \param_0$ and $0$ otherwise.

        When $\param$ is not known, we can use stochastic gradient descent.
        \index{gradient ascent!stochastic}
        \[
        \nabla \E^\pol_\bel \util = 
        \int_\Param [\nabla \E^\pol_\param \util] \dd \bel(\param)
        \]
        to obtain:
        \[
        w^{(n+1)} = w^{(n)} + \delta^{(n)} \param^{(n)}_1 - \param^{(n)}_0.
        \]
        where $\param^{(n)} \sim \bel$.
      }

    \end{example}
  }

  \only<2>{
    \begin{example}
      \begin{figure}[H]
        \centering
        \begin{tikzpicture}
          \node[RV, hidden] at (0,0) (p) {$\param$};
          \node[RV] at (1,0) (y) {$y_t$};
          \node[RV] at (2,0) (a) {$a_t$};
          \node[select] at (3,0) (pol) {$\pol$};
          \draw[->] (pol)--(a);
          \draw[->] (p)--(y);
          \draw[->] (a)--(y);
          \node[utility] at (2,1) (u) {$\util$};
          \draw[->] (a)--(u);
          \draw[->] (y)--(u);
          \node[RV] at (1,-1) (x) {$x_t$};
          \draw[->] (x)--(a);
          \draw[->] (x)--(y);
          \draw[->] (p)--(x);
        \end{tikzpicture}
        \caption{Decision problem with covariates.}
      \end{figure}
      Let $a_t, x_t = \{0,1\}$, $y_t \in \Reals$, $\param \in \Reals^4$ and
      \[
      y_t \mid a_t = a, x_t = x \sim \Bernoulli(\param_{a,x})
      \]
      Then, by estimating $\param$, we can predict the effect of any action.
    \end{example}
  }


\end{frame}
\section{Individual effects and counterfactuals}

\only<article>{
  Counterfactual analysis is mainly about questions relative to individuals, and specifically about what the effects of alternative actions would have been in specific instances in the past. We will assume a decision-theoretic viewpoint throughout, in order to be as clear as possible and avoiding imprecise language.
}
\subsection{Disturbances and structural equation models}

\begin{frame}
  \only<article>{A structural equation model describes the random variables as deterministic functions of the decisions variables and the random exogenous disturbances. This allows us to separate the unobserved randomness from the known functional relationship between the other variables. Structurally, the model is essentially a variant of decision diagrams, as shown in Figure~\ref{fig:disturbance-model}.}
  \begin{figure}[H]
    \centering
    \begin{tikzpicture}
      \node[RV, hidden] at (-1,1) (p) {$\param$};
      \node[RV] at (0,0) (x) {$x_t$};
      \node[RV] at (1,1) (y) {$y_t$};
      \node[RV] at (2,0) (a) {$a_t$};
      \draw[->] (x)--(y);
      \draw[->] (x)--(a);
      \draw[->] (a)--(y);
      \draw[->] (p) to (x);
      \draw[->] (p)--(y);
      \node[select] at (4,0) (pol) {$\pol$};
      \draw[->] (pol)--(a);
      \node[utility] at (3,1) (u) {$\util$};
      \draw[->] (a)--(u);
      \draw[->] (y)--(u);
      \node[RV, hidden, above of=y]  (oy) {$\omega_{t,y}$};
      \node[RV, hidden, below of=x]  (ox) {$\omega_{t,x}$};
      \node[RV, hidden, below of=a]  (oa) {$\omega_{t,a}$};
      \draw[->] (ox) -- (x);
      \draw[->] (oa) -- (a);
      \draw[->] (oy) -- (y);
    \end{tikzpicture}
    \caption{Decision diagram with exogenous disturbances $\omega$.}
    \label{fig:disturbance-model}
  \end{figure}
  \only<article>{We still need to specify particular functional relationships between the variables. Generally speaking, a random variable taking values in $\CX$, is simply a function $\Omega \times \Param \to \CX$. For example, in Figure~\ref{fig:disturbance-model} $y_t = f_y(\omega, \theta)$. Taking into account the dependencies, this can be rewritten in terms of a function of the other random variables, and the local disturbance: $y_t = f_{y|a,x}(a,x, \omega_{t,y}, \theta)$. The choice of the function, together with the distribution of the parameter $\param$ and the disturbances $\omega$, fully determines our model.}
  \begin{example}[Structural equation model  for Figure~\ref{fig:disturbance-model}]
    \only<presentation>{\vspace{-1em}}
    \only<article>{
      In structural equation models, the only random variables are the exogenous disturbances. In a fully Bayesian framework, $\param$ is also a latent random variable. The remaining variables are  deterministic functions. 
    }
    \begin{align*}
      \theta &\sim \Normal(\vectorsym{0}_4, \eye_4),\\
      x_t &= \theta_0 \omega_{t,x},
          & \omega_{t,x} &\sim \Bernoulli(0.5)\\
      y_t &= \theta_1  + \theta_2 x_t + \theta_3 a_t + \omega_{t,y},
          &\omega_{t,y} &\sim \Normal(0,1)\\
      a_t &= \pol(x_t) + \omega_{t,a} \mod |\CA| 
          &\omega_{t,a} &\sim 0.1 \Singular(0) + 0.9 \Uniform(\CA),
    \end{align*}
  \end{example}
  \only<article>{Structural equation models are particularly interesting in applications such as economics, where there are postulated relations between various economic quantities. If relationships between variables satisfy nice properties, then we can perform counterfactual inferences of the type : ``What if I had \emph{not} taken an aspirin?'' In the example above, if we can infer the noise variables $\omega$, we can change the value of some choice variables, i.e. $a_t$ and see the effect on other variables like $y_t$ directly.}
\end{frame}

\begin{frame}
  \frametitle{Treatment-unit additivity}
  \begin{figure}[H]
    \centering
    \begin{tikzpicture}
      \node[RV, hidden] at (-1,1) (p) {$\param$};
      \node[RV] at (1,1) (y) {$y_t$};
      \node[RV] at (2,0) (a) {$a_t$};
      \draw[->] (a)--(y);
      \draw[->] (p)--(y);
      \node[select] at (4,0) (pol) {$\pol$};
      \draw[->] (pol)--(a);
      \node[utility] at (3,1) (u) {$\util$};
      \draw[->] (a)--(u);
      \draw[->] (y)--(u);
      \node[RV, hidden, above of=y]  (oy) {$\omega_{t,y}$};
      \node[RV, hidden, below of=a]  (oa) {$\omega_{t,a}$};
      \draw[->] (oa) -- (a);
      \draw[->] (oy) -- (y);
    \end{tikzpicture}
    \caption{Decision diagram for treatment-unit additivity}
    \label{fig:tua}
  \end{figure}
  \begin{assumption}[TUA]
    For any given treatment $a \in \CA$, the response variable satisfies
    \[
    y_t = g(a_t) + \omega_{t,y}
    \]
  \end{assumption}
  \only<article>{
    This implies that
    \begin{align*}
      \E[y_t \mid a_t = a] = g(a_t) + \E(\omega_{t,y})
    \end{align*}
  }
\end{frame}


\subsection{Example: Learning instrumental variables}
\only<article>{This example is adapted from~\citet{Hartford:CP-DIV}, who use a deep learning to infer causal effects in the presence of instrumental variables. They break down the problem in two prediction tasks: the first is treatment prediction, and the second conditional treatment distribution. Unfortunately they do not use a decision-theoretic framework and so the difference between actual prices and policy changes is unclear.}
\begin{frame}
  \begin{example}[Pricing model]
    \only<article>{In the following pricing model, we wish to understand how sales are affected by different pricing policies. In this example, there is a variable $z_t$ which is an instrument, as it varies for reasons that are independent of demand and only affects sales through ticket prices.}
    \begin{figure}[H]
      \centering
      \begin{tikzpicture}
        \node[RV] at (0,0) (x) {$x_t$};
        \node[RV] at (1,0) (y) {$y_t$};
        \node[RV] at (0,1) (z) {$z_t$};
        \node[RV] at (1,1) (p) {$a_t$};
        \node[select] at (2,1) (pol) {$\pol$};
        \node[RV,hidden] at (2,0) (o) {$\omega_t$};
        \draw[->] (x) to (y);
        \draw[->] (x) to (p);
        \draw[->] (z) to (p);
        \draw[->] (p) to (y);
        \draw[->] (o) to (p);
        \draw[->] (o) to (y);
        \draw[->] (pol) to (p);
      \end{tikzpicture}
      \caption{Graph of structural equation model for airport pricing policy $\pol$: $a_t$ is the actual price, $z_t$ are fuel costs, $x_t$ is the customer type, $y_t$ is the amount of sales, $\omega_t$ is whether there is a conference. The dependency on $\param$ is omitted for clarity.}
    \end{figure}
  \end{example}
  \only<article>{There are a number of assumptions we can make on the instrument $z_t$.}
  \begin{assumption}[Relevance]
    $a_t$ depends on $z_t$.
  \end{assumption}
  \only<article>{In our example, it also depends on $x_t$.}
  \begin{assumption}[Exclusion]
    $z_t \indep y_t \mid x_t, a_t, \omega_t$.
  \end{assumption}
  \only<article>{In other words, the outcome does not depend on the instrument directly. This was also satisfied in our first example of an instrumental variable.}
  \begin{assumption}[Unconfounded instrument]
    $z_t \indep \omega_t \mid x_t$.
  \end{assumption}
\end{frame}

\begin{frame}
  \frametitle{Prediction tasks}
  \only<article>{We can use the following structural equation model}
  \begin{equation}
    \label{eq:price}
    y_t = g_\param(a_t, x_t) + \omega_t, \qquad \E_\param \omega_t = 0, \qquad \forall \param \in \Param
  \end{equation}
  \only<article>{There are two slightly different prediction tasks we can think of in this model.}
  \begin{block}{Standard prediction}
    \only<article>{In standard prediction tasks, we just want to estimate the distribution of sales given the characteristics and price. Since the actions are correlated with the outcome through the confounder, this estimate is biased.}
    \[
    \Pr_\param^\pol(y_t \mid x_t, a_t), \qquad  \E^\pol_\param(y_t \mid x_t, a_t) = g_\param(x_t, a_t) + \E_\param^\pol(\omega_t \mid x_t, a_t).
    \]
  \end{block}

  \begin{block}{Counterfactual prediction}
    \[
    \E^\pol_\param(y_t \mid x_t, z_t) = 
    \int_\CA \underbrace{[g(a_t \mid x_t, z_t) + \E_\param(\omega \mid x_t)]}_{h(a_t, x_t)} \dd \pol(a_t \mid x_t)
    \]
  \end{block}

  
\end{frame}





\subsection{Discussion}
\begin{frame}
  \begin{block}{Further reading}
    \begin{itemize}
    \item Pearl, \emph{Causality}.
    \item \citet{dawid2012decision}
    \end{itemize}
  \end{block}

  \subsection{Exercises}
  In the following exercises, we are taking actions $a_t$ and obtaining outcomes $y_t$. Our utility function is simply $U = y_t$.
  \only<article>{
    \begin{exercise}
      Let us have some data generated from a null treatment policy $\pol_0$ of the form $(a_t, y_t)$. There is a simple model that explains the data of the form
      \[
      y_t \mid a_t = a, \param \sim \Normal(a + \param, 1),
      \]
      where the actions are distribution according to $\pol(a_t)$. 
      \begin{itemize}
      \item Assume that $\pol_0 \in [0,1]$ is given and it is $a_t \mid \pol = \pol_0 \sim \Bernoulli(\pol_0)$. First, estimate $\param$. Then, calculate the distribution of $y_t \mid \pol_0, \param$ for any other policy and plot the resulting mean and variance as $\pol$ changes. You can do this first in a maximum-likelihood manner. Advanced: estimate the posterior distribution of $\param$ for a normal prior on $\param$.
      \item Now assume that $\pol_0$ is not given. This means that you must also estimate $\pol_0$ itself before estimating the effect of any other policy $\pol$ on the data.
      \item In this exercise, can you learn about other actions when you are not taking them? Why?
      \end{itemize}
    \end{exercise}

    \begin{exercise}
      Let us have some data generated from a null treatment policy $\pol_0$ of the form $(a_t, y_t)$. Let us now consider a slightly model where $\param \in \Reals^2$.
      \[
      y_t \mid a_t = a, \param \sim \Normal(\param_a, 1),
      \]
      where the actions are distribution according to $\pol(a_t)$. 
      \begin{itemize}
      \item Assume that $\pol_0 \in [0,1]$ is given and it is $a_t \mid \pol = \pol_0 \sim \Bernoulli(\pol_0)$. First, estimate $\param$. Then, calculate the distribution of $y_t \mid \pol_0, \param$ for any other policy and plot the resulting mean and variance as $\pol$ changes. You can do this first in a maximum-likelihood manner. Advanced: estimate the posterior distribution of $\param$ for a normal prior on $\param$.
      \item Now assume that $\pol_0$ is not given. This means that you must also estimate $\pol_0$ itself before estimating the effect of any other policy $\pol$ on the data.
      \item In this exercise, can you learn about other actions when you are not taking them? Why?
      \end{itemize}
    \end{exercise}

    \begin{exercise}
      Given your estimates, find the optimal policy for each one of those cases. Measure the quality of this policy on
      \begin{itemize}
      \item The actual data you have already (e.g. using importance sampling)
      \item On new simulations (using the testing framework).
      \end{itemize}
      \emph{Advanced: The optimal policy when $\param$ is known is to always take the same action. Does that still hold when $\param$ is not known and you are estimating it all the time from new data?}
    \end{exercise}


    \begin{exercise}[Advanced]
      Let us have some data generated from a null treatment policy $\pol_0$ of the form $(x_t, a_t, y_t)$, with $a_t, x_t \in \{0, 1\}$.
      \[
      y_t \mid a_t = a, x_t = x, \param \sim \Normal(\param_{a,x}, 1),
      \]
      where the actions are distribution according to $\pol_0(a_t \mid x_t)$. 
      \begin{itemize}
      \item Assume that $\pol_0$ is given and it is $a_t \mid x_t = x, \pol = \pol_0 \sim \Bernoulli(w_{x})$. First, estimate $\param$. Repeat your analysis.
      \item Now assume that $\pol_0$ is not given.  Again, repeat your analysis.
      \item Is there now globally better action $a_t$? Should it depend on $x_t$, like in the observed policy? Can you estimate the optimal policy?
      \end{itemize}
    \end{exercise}
  }
\end{frame}
%%% Local Variables:
%%% mode: latex
%%% TeX-engine: xetex
%%% TeX-master: "notes"
%%% End:


\chapter{Experiment design}
\label{ch:bandit}

\section{Introduction}
\label{sec:mdp-introduction}
This unit describes the very general formalism of Markov decision
processes (MDPs) for formalising problems in sequential decision
making.  Thus a \emindex{Markov decision process} can be used to model
stochastic path problems, stopping problems, reinforcement learning
problems, experiment design problems, and control problems.

We being by taking a look at the problem of \emindex{experimental
  design}. One instance of this problem occurs when considering how to
best allocate treatments with unknown efficacy to patients in an
adaptive manner, so that the best treatment is found, or so as to
maximise the number of patients that are treated successfully. The
problem, originally considered
by~\cite{Chernoff:SequentialDesignExperiments,chernoff1966smc},
informally can be stated as follows.

We have a number of treatments of unknown efficacy, i.e. some of them
work better than the others. We observe patients one at a time. When a
new patient arrives, we must choose which treatment to
administer. Afterwards, we observe whether the patient improves or
not. Given that the treatment effects are initially unknown, how can
we maximise the number of cured patients? Alternatively, how can we
discover the best treatment? The two different problems are formalised
below.

\begin{example}\indexmargin{Adaptive treatment allocation}
  Consider $k$ treatments to be administered to $T$ volunteers.  To each
  volunteer only a single treatment can be assigned.  At the $t$-th trial, we treat one volunteer with some treatment $a_t \in \{1, \ldots, k\}$. We then obtain  obtain a reward $r_t = 1$ if the patient is treated and $0$ otherwise.  We wish to choose actions maximising the utility  $U = \sum_t r_t$. This would correspond to maximising the number of patients that get treated over time.
\end{example}

\begin{example}\indexmargin{Adaptive hypothesis testing}
  An alternative goal would be to do a \emph{clinical trail}\index{clinical trial}, in order to find the best possible treatment. For simplicity, consider the problem of trying to find out whether a particular treatment is better or not than a placebo.  We are given a hypothesis set $\Omega$, with each $\omega \in \Omega$ corresponding to different models for the effect of the treatment and the placebo. Since we don't know what is the right model, we place a prior $\bel_0$ on $\Omega$. We can perform $T$ experiments, after which we must make decide whether or not the treatment is significantly better than the placebo. To model this, we define a decision set $\CD = \{d_0, d_1\}$ and a utility function $U : \CD \times \Omega \to \Reals$, which models the effect of each decision $d$ given different versions of reality $\omega$. One hypothesis $\omega \in \Omega$ is true. To distinguish them, we can choose
  from a set of $k$ possible experiments to be performed over $T$
  trials.  At the $t$-th trial, we choose experiment $a_t \in \{1,
  \ldots, k\}$ and observe outcome $x_t \in \CX$, with $x_t \sim
  P_\omega$ drawn from the true hypothesis. Our posterior is
  \[
  \bel_t(\omega) \defn
  \bel_0(\omega \mid a_1, \ldots, a_t, x_1, \ldots, x_t).
  \]
  The reward is $r_t = 0$ for $t < T$ and
  \[
  r_T = \max_{d \in D}\E_{\bel_T}(U \mid d).
  \]
  Our utility in this can again be expressed as a sum over individual rewards,  $U = \sum_{t=1}^T r_t$.
\end{example}
Both formalizations correspond to so-called {\em bandit problems} which we take a closer look at in the following section.

\section{Bandit problems}
\label{sec:exp-design-bandit}
\index{bandit problems}

The simplest bandit problem is the stochastic $n$-armed bandit.\index{bandit problems!stochastic} We are faced with $n$ different one-armed bandit machines, such as those found in casinos. In this problem, at time $t$, you have to choose one \emph{action} (i.e. a machine) $a_t \in \CA = \set{1, \ldots, n}$. In this setting, each time $t$ you play a machine, you receive a reward $r_t$, with fixed expected value $\omega_i = \E (r_t \mid a_t = i)$.
Unfortunately, you do not know $\omega_i$, and consequently the best arm is also unknown. How do you then choose arms so as to maximise the total expected reward? 
\begin{definition}[The stochastic $n$-armed bandit problem.]
  This is the problem of selecting a sequence of actions $a_t \in \CA$, with $\CA = \set{1, \ldots, n}$, so as to maximise expected utility, where the utility is 
  \[
  U = \sum_{t=0}^{T - 1} \disc^t r_t,
  \]
  where $T \in (0, \infty]$ is the horizon and $\gamma \in (0,1]$
  is a \emindex{discount factor}. The reward $r_t$ is stochastic,
  and only depends on the current action, with expectation $\E(r_t
  \mid a_t = i) = \omega_i$.
\end{definition}
In order to select the actions, we must specify some \emindex{policy} or decision rule. This can only depend on the sequence of previously taken actions and observed rewards. Usually, the policy $\pol :  \CA^* \times \Reals^* \to \CA$ is a deterministic mapping from the space of all sequences of actions and rewarsd to actions. That is, for every observation and action history $a_1, r_1, \ldots, a_{t-1}, r_{t-1}$ it suggests a single action $a_t$. However, it could also be a stochastic policy, that specifies a mapping to action distributions. We use the following notation for stochastic history-dependent bandit policies,
\begin{equation}
  \label{eq:history-dependent-bandit}
  \pol(a_t \mid a^{t-1}, r^{t-1})
\end{equation}
to mean the probability of actions $a_t$ given the history until time $t$.

How can we solve bandit problems? One idea is to apply the Bayesian
decision-theoretic framework we have developed earlier to maximise
utility in expectation.  More specifically, given the horizon $T
\in (0, \infty]$ and the discount factor $\disc \in (0,1]$, we
define our utility from time $t$ to be:
\begin{equation}
  \label{eq:reward-utility}
  U_t = \sum_{k=1}^{T-t} \gamma^k r_{t+k}.
\end{equation}
To apply the decision theoretic framework, we need to define a suitable family of probability measures $\family$, indexed by parameter $\omega \in \Omega$ describing the reward distribution of each bandit, together with a prior distribution $\bel$ on $\Omega$. Since $\omega$ is unknown, we cannot maximise the expected utility with respect to it. However, we can always maximise expected utility with respect to our belief $\bel$. That is, we replace the ill-defined problem of maximising utility in an unknown model with that of maximising expected utility given a distribution over possible models. The problem can be written in a simple form:
\begin{equation}
  \label{eq:bel-reward-utility}
  \max_\pol \E_\bel^\pol U_t = 
  \max_\pol \int_\Omega \E_\omega^\pol U_t \dd \bel{\omega}.
\end{equation}
The difficulty lies not in formalising the problem, but in the fact that the set of learning policies is quite large, rendering the optimisation infeasible.
The following figure summarises the statement of the bandit problem in the Bayesian setting.
\begin{block}{Decision-theoretic statement of the bandit problem}
  \begin{itemize}
  \item Let $\CA$ be the set of arms.
  \item Define a family of distributions $\family = \cset{P_{\omega, i}}{\omega \in \Omega, i \in \CA}$ on $\Reals$.
  \item Assume the i.i.d model $r_t \mid \omega, a_t = i \sim P_{\omega, i}$.
  \item Define prior $\bel$ on $\Omega$.
  \item Select a policy $\pol : \CA^* \times \Reals^* \to \CA$ maximising
    \[
    \E^\pol_\bel U = \E^\pol_\bel \sum_{t=0}^{T - 1} \disc^t r_{t}
    \]
  \end{itemize}
\end{block}
There are two main difficulties with this approach. The first is specifying the family and the prior distribution: this is effectively part of the problem formulation and can severely influence the solution. The second is calculating the policy that maximises expected utility given a prior and family. The first problem can be resolved by either specifying a subjective prior distribution, or by selecting a prior distribution that has good worst-case guarantees. The second problem is hard to solve, because in general, such policies are history dependent and the set of all possible histories is exponential in the horizon $T$.

\subsection{An example: Bernoulli bandits}
\label{sec:bernoulli-bandit-example}
As a simple illustration, consider the case when the reward for choosing one of the $n$ actions is either $0$ or $1$, with some fixed, yet unknown probability depending on the chosen action. This can be modelled in the standard Bayesian framework using the Beta-Bernoulli conjugate prior. More specifically, we can formalise the problem as follows.

Consider $n$ Bernoulli distributions with
unknown parameters $\omega_i$ ($i = 1, \ldots, n$) such that 
\begin{align}
  r_t \mid a_t = i &\sim
  \Bernoulli(\omega_i),
  &
  \E(r_t  \mid a_t = i) &= \omega_i.
\end{align}
Each Bernoulli distribution thus corresponds to the distribution of
rewards obtained from each bandit that we can play.  In order to
apply the statistical decision theoretic framework, we have to
quantify our uncertainty about the parameters $\omega$ in terms of a
probability distribution.

We model our belief for each bandit's
parameter $\omega_i$ as a Beta distribution $\Beta(\alpha_i,
\beta_i)$, with density $f(\omega \mid \alpha_i, \beta_i)$ so that
\[
\bel(\omega_1, \ldots, \omega_n)
=
\prod_{i=1}^n f(\omega_i \mid \alpha_i, \beta_i).
\]
Recall that the posterior of a Beta prior is also a Beta. Let
\[
N_{t,i} \defn \sum_{k=1}^t \ind{a_k = i}
\]
be the number of times we played arm $i$ and
\[
\hat{r}_{t,i} \defn \frac{1}{N_{t,i}} \sum_{k=1}^t r_t \ind{a_k = i}
\]
be the
\alert{empirical reward} of arm $i$ at time $t$. We
can let this equal $0$ when $N_{t,i} = 0$.
Then, the posterior distribution for the parameter of arm $i$ is
\[
\bel_t = \Beta(\alpha_i + N_{t,i} \hat{r}_{t,i}~,~ \beta_i + N_{t,i} (1 - \hat{r}_{t,i})).
\]
Since $r_t \in \{0,1\}$ the possible states of our belief given some
prior are $\Naturals^{2n}$.

In order for us to be able to evaluate a policy, we need to be able to
predict the expected utility we obtain. This only depends on our
current belief, and the state of our belief corresponds to the state
of the bandit problem.\indexmargin{belief state} This means that
everything we know about the problem at time $t$ can be summarised by
$\bel_t$. For Bernoulli bandits, sufficient statistic for our belief
is the number of times we played each bandit and the total reward from
each bandit.  Thus, our state at time $t$ is entirely described by our
priors $\alpha, \beta$ (the initial state) and the vectors
\begin{align}
  N_t = (N_{t,1}, \ldots, N_{t,i})\\
  \hat{r}_t = (\hat{r}_{t,1}, \ldots, \hat{r}_{t,i}).
\end{align}
At any time $t$, we can calculate the probability of observing
$r_t = 1$ or $r_t = 0$ if we pull arm $i$ as:
\[
\bel_t(r_t = 1 \mid a_t = i) = \frac{\alpha_i + N_{t,i} \hat{r}_{t,i}}{\alpha_i + \beta_i + N_{t,i}}
\]
So, not only we can predict the immediate reward based on our current
belief, but we can also predict all next possible beliefs: the next
state is well-defined and depends only on the current state.  As we
shall see later, this type of decision problem is more generally called a Markov
decision process (Definition~\ref{def:MDP}). For now, we shall more generally (and precisely) define the bandit process itself.

\subsection{Decision-theoretic bandit process}
\label{sec:decision-theoretic-bandits}

The basic bandit process can be seen in Figure~\ref{fig:basic-bandit-process}. We can now define the general decision-theoretic bandit process, not restricted to independent Bernoulli bandits.
\begin{definition}
  Let $\CA$ be a set of actions, not necessarily finite. Let $\Omega$ be a set of possible parameter values, indexing a family of probability measures $\family = \cset{P_{\omega, a}}{\omega \in \Omega, a \in \CA}$. There is some $\omega \in \Omega$ such that, whenever we take action $a_t = a$, we observe reward $r_t \in \CR \subset \Reals$ with probability measure:
  \begin{equation}
    \label{eq:bandit-reward-probability}
    P_{\omega,a}(R) \defn \Pr_\omega(r_{t} \in R \mid a_t = a),
    \qquad R \subseteq \Reals.
  \end{equation}
  Let $\bel_1$ be a prior distribution on $\Omega$ and let the posterior distributions be defined as:
  \begin{equation}
    \label{eq:bandit-posteriors}
    \bel_{t+1}(B) \propto \int_B P_{\omega, a_t} (r_t) \dd \bel_t(\omega).
  \end{equation}
  The next belief is random, since it depends on the random quantity $r_t$. In fact, the probability of the next reward lying in $R$ if $a_t = a$ is given by the following marginal distribution:
  \begin{equation}
    \label{eq:dt-bandit-reward-probability}
    P_{\bel_t, a} (R) \defn \int_\Omega P_{\omega,a}(R) \dd{\bel_t}(\omega).
  \end{equation}
  \begin{figure}[ht]
  \begin{center}
    \begin{tikzpicture}
      \node[RV] at (2,-1.5) (xn1) {$\bel_{t+1}^0$}; 
      \node[RV] at (2,-0.5) (xn2) {$\bel_{t+1}^1$};
      \node[RV] at (2,0.5) (xn3) {$\bel_{t+1}^2$}; 
      \node[RV] at (2,1.5) (xn4) {$\bel_{t+1}^3$};
      \node[select] at (0,-1) (an1) {$a^1_t$};
      \node[select] at (0,1) (an2) {$a^2_t$};
      \node[RV] at (-2,0) (xp) {$\bel_n$};
      \draw[->] (xp) -- (an1);
      \draw[->] (xp) -- (an2);
      \draw[->] (an1) -- (xn1) node[near start, below] {$r=0$};
      \draw[->] (an1) -- (xn2) node[near start, above] {$r=1$}; 
      \draw[->] (an2) -- (xn3) node[near start, below] {$r=0$}; 
      \draw[->] (an2) -- (xn4) node[near start, above] {$r=1$}; 
    \end{tikzpicture}
  \end{center}
  \caption{A partial view of the multi-stage process. Here, the probability that we obtain $r=1$ if we take action $a_t = i$ is simply $P_{\bel_t,i}(\{1\})$.}
  \label{fig:multi-stage-bandit}
\end{figure}  

  Finally, as $\bel_{t+1}$ deterministically depends on $\bel_t, a_t, r_t$, the probability of obtaining a particular next belief is the same as the probability of obtaining the corresponding rewards leading to the next belief. In more detail, we can write:
  \begin{equation}
    \label{eq:dt-bandit-belief-probability}
    \Pr(\bel_{t+1} = \bel \mid \bel_t, a_t)
    =
    \int_\CR \ind{\bel_{t}(\cdot \mid a_t, r_t = r) = \bel} \dd{P_{\bel_t, a}}(r). 
  \end{equation}
\end{definition}
In practice, although multiple reward sequences may lead to the same beliefs, we frequently ignore that possibility for simplicity. Then the process becomes a tree. A solution to the problem of what action to select is given by a backwards induction algorithm similar to that given in Section~\ref{sec:backwards-induction}.
\begin{equation}
  U^*(\bel_t) = \max_{a_t} \E(r_t \mid \bel_t, a_t) + \sum_{\bel_{t+1}} \Pr(\bel_{t+1} \mid \bel_t, a_t) U^*(\bel_{t+1}).\label{eq:backwards-induction-bandits}
\end{equation}
The above equation is the \emindex{backwards induction} algorithm for bandits.  If you look at this structure, you can see that  next belief only depends on the current belief, action and reward, i.e. it satisfies the Markov property, as seen in Figure~\ref{fig:multi-stage-bandit}. Consequently, a decision-theoretic bandit process can be modelled more generally as a \index{Markov decision process}Markov decision process, explained in the following section. It turns out that backwards induction, as well as other efficient algorithms, can provide optimal solutions for Markov decision processes.
\begin{figure}[htb]
  \centering
  \subfigure[The basic process]{
    \begin{tikzpicture}
      \node[select] at (0,1) (at) {$a_t$};
      \node[RV,hidden] at (0,-2) (omega) {$\omega$};
      \node[utility] at (1,-1) (rt) {$r_{t}$};
      \draw[->] (at) -- (rt);
      \draw[->] (omega) -- (rt);
      \node[select] at (0,1) (at2) {$a_{t+1}$};
      \node[utility] at (1,-1) (rt2) {$r_{t+1}$};
      \draw[->] (at2) -- (rt2);
      \draw[->] (omega) -- (rt2);
    \end{tikzpicture}
    \label{fig:basic-bandit-process}
  }
  \subfigure[The full process]{
    \begin{tikzpicture}
      \node[RV,hidden] at (0,-2) (omega) {$\omega$};
      \node[RV] at (0,0) (bt) {$\bel_t$};
      \node[select] at (0,1) (at) {$a_t$};
      \node[utility] at (1,-1) (rt) {$r_{t}$};
      \draw[->] (omega) -- (rt);
      \draw[->] (at) -- (rt);
      \node[RV] at (2,0) (bt2) {$\bel_{t+1}$};
      \draw[->] (at) -- (bt2);
      \draw[->] (bt) -- (bt2);
      \draw[->] (rt) -- (bt2);
      \node[select] at (2,1) (at2) {$a_{t+1}$};
      \node[utility] at (3,-1) (rt2) {$r_{t+1}$};
      \draw[->] (omega) -- (rt2);
      \draw[->] (at2) -- (rt2);
    \end{tikzpicture}
    \label{fig:dt-bandit-full}
  }
  \subfigure[The lifted process]{
    \begin{tikzpicture}
      \node[RV] at (0,0) (bt) {$\bel_t$};
      \node[select] at (0,1) (at) {$a_t$};
      \node[utility] at (1,-1) (rt) {$r_{t}$};
      \draw[->] (bt) -- (rt);
      \draw[->] (at) -- (rt);
      \node[RV] at (2,0) (bt2) {$\bel_{t+1}$};
      \draw[->] (at) -- (bt2);
      \draw[->] (bt) -- (bt2);
      \draw[->] (rt) -- (bt2);
      \node[select] at (2,1) (at2) {$a_{t+1}$};
      \node[utility] at (3,-1) (rt2) {$r_{t+1}$};
      \draw[->] (bt2) -- (rt2);
      \draw[->] (at2) -- (rt2);
    \end{tikzpicture}
    \label{fig:dt-bandit-lifted}
  }
  \caption{Three views of the bandit process.
    Figure~\ref{fig:basic-bandit-process} shows the basic bandit
    process, from the view of an external observer. The decision maker
    selects $a_t$, while the parameter $\omega$ of the process is
    hidden. It then obtains reward $r_t$. The process repeats for $t =
    1, \ldots, T$.  The decision-theoretic bandit process is shown in
    Figures~\ref{fig:dt-bandit-full} and
    \ref{fig:dt-bandit-lifted}. While $\omega$ is not known, at each
    time step $t$ we maintain a belief $\bel_t$ on $\Omega$. The
    reward distribution is then defined through our belief. In
    Figure~\ref{fig:dt-bandit-full}, we can see that complete process,
    where the dependency on $\omega$ is clear. In
    Figure~\ref{fig:dt-bandit-lifted}, we marginalise out $\omega$ and
    obtain a model where the transitions only depend on the current
    belief and action.}
  \label{fig:bandit-process}
\end{figure}

In reality, the reward depends only on the action and the unknown $\omega$, as can be seen in Figure~\ref{fig:dt-bandit-full}. This is the point of view of an external observer. However, from the point of view of the decision maker, the distribution of $\omega$ only depends on his current belief. Consequently, the distribution of rewards also only depends on the current belief, as we can marginalise over $\omega$. This gives rise to the decision-theoretic bandit process shown in Figure~\ref{fig:dt-bandit-lifted}.
In the following section, we shall consider Markov decision processes more generally.

\section{Markov decision processes and reinforcement learning}
\label{sec:MDP}
Bandit problems are one of the simplest instances of reinforcement learning problems. Informally, speaking, these are problems of learning how to act in an unknown environment, only through interaction with the environment and limited reinforcement signals.
The learning agent interacts with the environment through actions and observations, and simultaneously obtains rewards. For example, we can consider a rat running through a maze designed by an experimenter, which from time to time finds a piece of cheese, the reward. 
The goal of the agent is usually to maximise some measure of the total reward. In summary, we can state the problem as follows.
\begin{block}{The reinforcement learning problem.}
  The reinforcement learning problem is the problem of \alert{learning} how to act in an \alert{unknown} environment, only by \textcolor{blue}{interaction} and \textcolor{blue}{reinforcement}.
\end{block}
Generally, we assume that the environment $\mdp$ that we are acting in
has an underlying state $s_t \in \CS$, which changes with in discrete
time steps $t$. At each step, the agent obtains an observation $x_t \in
\CX$ and chooses actions $a_t \in \CA$. We usually assume that the
environment is such that its next state $s_{t+1}$ only depends on its
current state $s_t$ and the last action taken by the agent, $a_t$. In
addition, the agent observes a reward signal $r_t$, and its goal is to
maximise the total reward during its lifetime.

Doing so when the environment $\mdp$ is unknown, is hard even in
seemingly simple settings, like $n$-armed bandits, where the
underlying state never changes. In many real-world applications, the
problem is even harder, as the state is not directly
observed. Instead, we may simply have some measurements $x_t$, which
give only partial information about the true underlying state $s_t$.

Reinforcement learning problems typically fall into one of the
following three groups: (1) Markov decision processes (MDPs), where
the state $s_t$ is observed directly, i.e. $x_t = s_t$; (2) Partially
observable MDPs (POMDPs), where the state is hidden, i.e. $x_t$ is
only probabilistically dependent on the state; and (3) stochastic
Markov games, where the next state also depends on the move of other
agents. While all of these problem \emph{descriptions} are different,
in the Bayesian setting, they all can be reformulated as MDPs, by
constructing an appropriate belief state, similarly to how we did it
for the decision theoretic formulation of the bandit problem.

In this chapter, we shall restrict our attention to Markov decision processes. Hence, we shall not discuss the existence of other agents, or the case where we cannot observe the state directly. 
\begin{definition}[Markov Decision Process]
  A \index{Markov decision process|textbf}Markov decision process $\mdp$ is a tuple $\mdp = \tuple{\CS, \CA, \SP, \SR}$, where $\CS$ is the \emph{state space} and $\CA$ is the \emph{action space}. The \emindex{transition distribution} being $\SP = \cset{P(\cdot \mid s,a)}{s \in \CS, a \in \CA}$ is a collection of probability measures on $\CS$, indexed in $\CS \times \CA$ and the \emindex{reward distribution}  $\Rews = \cset{\Rew(\cdot \mid s,a)}{s \in \CS, a \in \CA}$ is a collection of probability measures on $\Reals$, such that:  
  \begin{align}
    P(S \mid s, a) &= \Pr_\mdp(s_{t+1} \in S \mid s_t =s, a_t = a)
    \\
    \Rew(R \mid s, a) &= \Pr_\mdp(r_{t} \in R \mid s_t =s, a_t = a).
  \end{align}
  \label{def:MDP}
\end{definition}
For simplicity, we shall also use
\begin{equation}
  \label{eq:expected-reward}
  r_\mdp(s,a) = \E_\mdp(r_{t+1} \mid s_t = s, a_t = a),
\end{equation}
for the expected reward.

Of course, the transition and reward distributions are different
for different environments $\mdp$. For that reason, we shall
usually subscript the relevant probabilities and expectations with
$\mdp$, unless the MDP is clear from the context.


\begin{figure}[ht]
      \begin{center}
        \begin{tikzpicture}
          \node[RV] at (0,3) (mu) {$\mdp$};
          \node[select] at (1,0) (a1) {$a_t$};
          \node[RV] (s1) [above of=a1] {$s_t$};
          \node[RV] (s2) [right of=s1] {$s_{t+1}$};
          \node[utility] (r2) [above of=s2] {$r_{t}$};
          \draw [->] (s1) -- (s2);
          \draw [->] (a1) -- (s2);
          \draw [->] (a1) -- (r2);
          \draw [->] (s1) -- (r2); 
          \draw [->, bend right=45] (mu) -- (s2);
          \draw [->, bend right=45] (mu) -- (r2);
        \end{tikzpicture}
      \end{center}
  \begin{block}{Markov property of the reward and state distribution}
    \begin{align}
      \Pr_\mdp(s_{t+1} \in S \mid s_1, a_1, \ldots, s_t, a_t) = \Pr_\mdp(s_{t+1} \in S \mid s_t, a_t)  \only<presentation>{\tag{Transition distribution}}
      \\
      \Pr_\mdp(r_{t} \in R \mid s_1, a_1, \ldots, s_t, a_t) = \Pr_\mdp(r_{t} \in R \mid s_t, a_t) \only<presentation>{\tag{Reward distribution}}
    \end{align}
    \only<article>{where $S \subset \CS$ and $R \subset \CR$ are reward and state subsets respectively.}  
  \end{block}
  \caption{The structure of a Markov decision process.}
  \label{fig:MDP}
\end{figure}


\paragraph{Dependencies of rewards.}
Sometimes it is more convenient to have rewards that depend on the next state as well, i.e.
\begin{equation}
  \label{eq:next-state-dependent-rewards}
  r_\mdp(s,a,s') = \E_\mdp(r_{t+1} \mid s_t = s, a_t = a, s_{t+1} = s'),
\end{equation}
though this is complicates the notation considerably since now the reward is obtained on the next time step. However, we can always replace this with the expected reward for a given state-action pair:
\begin{align}
  \label{eq:expected-reward-state-action}
  r_\mdp(s, a)
  &= \E_\mdp(r_{t+1} \mid s_t = s, a_t = s)
  = \sum_{s' \in \CS} P_\mdp(s' \mid s, a) r_\mdp(s, a, s')
\end{align}
In fact, it is notationally more convenient to have rewards that only depend on the current state:
\begin{equation}
  \label{eq:state-dependent-rewards}
  r_\mdp(s) = \E_\mdp(r_{t} \mid s_t = s).
\end{equation}
For simplicity, we shall mainly consider the latter case. 

\paragraph{The agent.}
The environment does not exist in isolation. The actions are taken by an agent, who is interested in obtaining high rewards. Instead of defining an algorithm for choosing actions directly, we define an algorithm for computing policies, which define distributions on actions.
\begin{block}{The agent's policy $\pol$}
  \index{policy}
  \begin{align}
    \Pr^\pol (a_t \mid s_t, \ldots, s_1, a_{t-1}, \ldots, a_1)\tag{history-dependent policy}
    \index{policy!history-dependent}
    \\
    \Pr^\pol (a_t \mid s_t) \tag{Markov policy}
    \index{policy!Markov}
  \end{align}
\end{block}

In some sense, the agent is defined by its \alert{policy} $\pol$,
which is a conditional distribution on actions given the history.
The \emindex{policy} $\pol$ is otherwise known as a {\em decision
  function}. In general, the policy can be history-dependent. In
certain cases, however, there are optimal policies that are
Markov. This is for example the case with additive utility
functions.  In paticular, the utility function maps from the
sequence of all possible rewards to a real number $U : \CR^* \to
\Reals$, given below:
\begin{definition}[Utility]
  Given a horizon $T$ and a discount factor $\gamma\in (0,1]$, the utility function $U : \CR^* \to \Reals$ is defined as
  \begin{equation}
    \label{eq:tutility-vector}
    U(r_0, r_1, \ldots, r_T) = \sum_{k=0}^T \disc^k r_k.
  \end{equation}
  It is convenient to give a special name to the utility starting from time $t$, i.e. the sum of rewards from that time on:
  \index{utility}
  \begin{equation}
    U_t \defn  \sum_{k=0}^{T-t} \only<2>{\disc^k} r_{t+k}.
  \end{equation}
\end{definition}
At any time $t$, the agent wants to to find a policy $\pol$ \alert{maximising}
the \alert{expected total future reward}
\begin{equation}
  \E_{\mdp}^{\pol} U_t = \E_{\mdp}^{\pol}
  \sum_{k=0}^{T-t} \disc^k r_{t+k}. 
  \tag{expected utility}
\end{equation}
This is so far identical to the expected utility framework we had seen so far, with the only difference that now the reward space is a sequence of numerical rewards and that we are acting within a dynamical system with state space $\CS$. In fact, it is a good idea to think about the \emph{value} of different states of the system under certain policies, in the same way that one things about how good different positions are in chess.


\subsection{Value functions}
\label{sec:value-functions}
\only<article>{
  A value function represents the expected utility of a given state, or state-and-action pair for a specific policy. They are really useful as shorthand notation and as the basis of algorithm development. The most basic of those is the state value function.}
\begin{frame}
  \begin{block}{State value function}
    \index{value function!state}
    \begin{equation}
      V_{\mdp, t}^{\pol}(s) \defn \E^{\pol}_{\mdp} (U_t \mid s_t = s) 
      \label{eq:state-value-function}
    \end{equation}
  \end{block}
  \only<article>{The state value function for a particular policy $\pol$ can be interpreted as how much utility you should expect if you follow the policy starting from state $s$ at time $t$, for the particular MDP $\mu$.}
  \begin{block}{State-action value function}
    \index{value function!state-action}
    \begin{equation}
      Q_{\mdp, t}^{\pol}(s,a) 
      \defn
      \E^{\pol}_{\mdp} (U_t \mid s_t = s, a_t = a) 
      \label{eq:q-value-function}
    \end{equation}
  \end{block}
  \only<article>{The state-action value function for a particular policy $\pol$ can be interpreted as how much utility you should expect if you play action $a$, at state $s$ at time $t$, and then follow the policy $\pol$, for the particular MDP $\mu$.}

  \only<article>{It is also useful to define the optimal policy and optimal value functions for a given MDP. In the following, a star indicates optimal quantities.
  }
  The \emph{optimal policy $\pol^*$}\index{policy!optimal}
  \begin{equation}
    \pol^*(\mdp) : V^{\pol^*(\mdp)}_{t,\mdp}(s) \geq V^{\pol}_{t,\mdp}(s) 
    \quad \forall \pol, t, s
    \label{eq:optimal-policy}
  \end{equation}    
  dominates all other policies
  $\pol$ everywhere in $\CS$.

  The \alert{optimal value function $V^*$} \index{value function!optimal}
  \begin{equation} 
    V^*_{t,\mdp}(s) \defn V^{\pol^*(\mdp)}_{t,\mdp}(s),
    \quad
    Q^*_{t,\mdp}(s) \defn Q^{\pol^*(\mdp)}_{t,\mdp}(s,a).
    \label{eq:optimal-value}
  \end{equation}
  is the value function of the
  optimal policy $\pol^*$.
\end{frame}

\begin{frame}
  \frametitle{Finding the optimal policy when $\mdp$ is known}
  \only<article>{When the MDP $\mdp$ is known, the expected utility of any policy can be calculated. Therefore, one could find the optimal policy by brute force, i.e. by calculating the utility of every possible policy. This might be as reasonable strategy if the number of policies is small. However, there are many better appr. First, there are iterative/offline methods where an optimal policy is found for all states of the MDP. These either try to estimate the optimal value function directly, or try to iteratively improve a policy until it is optimal. The second type of methods tries to find an optimal policy online. That is, the optimal actions are estimated only for states which can be visited in the future starting from the current state. However, the same main ideas are used in all of these algorithms.}
  \only<presentation>{
    \begin{columns}
      \begin{column}{.49\textwidth}
        \begin{center}
          \begin{tikzpicture}
            \node[place] at (0,0) (w1) {$s_t$};
            \node[place] at (3,2.25) (w2a1) {$s_{t+1}^1$};
            \node[place] at (3,.75) (w2a2) {$s_{t+1}^2$};
            \node[place] at (3,-.75) (w2b1) {$s_{t+1}^3$};     
            \node[place] at (3,-2.25) (w2b2) {$s_{t+1}^4$};     
            \draw[->] (w1) to node [auto] {$a_t^1, r_{t+1}^0$} (w2a1);              
            \draw[->] (w1) to node [auto] {$a_t^1, r_{t+1}^1$} (w2a2);              
            \draw[->] (w1) to node [auto] {$a_t^2, r_{t+1}^0$} (w2b1);
            \draw[->] (w1) to node [auto] {$a_t^2, r_{t+1}^1$} (w2b2);
          \end{tikzpicture}
        \end{center}
      \end{column}
      \begin{column}{.49\textwidth}

        \begin{block}{Iterative/offline methods} %ro: Is this also part of the figure? I don't have this in my pdf.
          \begin{itemize}
          \item Estimate the optimal \alert{value function} $V^*$ (i.e. with backwards induction on all $\CS$).
          \item Iteratively \alert{improve} $\pol$ (i.e. with policy iteration) to obtain $\pol^*$.
          \end{itemize}
        \end{block}
        \begin{block}{Online methods}
          \begin{itemize}
          \item Forward \alert{search} followed by backwards induction (on subset of $\CS)$.
          \end{itemize}
        \end{block}
      \end{column}
    \end{columns}
  }
\end{frame}


\section{Finite horizon, undiscounted problems}
\label{sec:finite-horiz-undisc}
\only<article>{
  The conceptually simplest type of problems are finite horizon problems where $T < \infty$ and $\gamma = 1$.    The first thing we shall try to do is to evaluate a given policy for a given MDP. There are a number of algorithms that can achieve this.
}
\subsection{Policy evaluation}
\index{policy evaluation}
\label{sec:policy-evaluation}
\begin{frame}\only<presentation>{\frametitle{Policy evaluation}}
  Here we are interested in the problem of determining the value function of a policy $\pol$ (for $\disc = 1, T < \infty$). All the algorithms we shall consider can be recovered from the following recursion. Noting that $U_{t+1} = \sum_{k=1}^{T-t} r_{t+k}$ we have:
  \begin{align}
    V_{\mdp,t}^\pol(s)
    \only<1->{
      &\defn \E^\pol_\mdp (U_t \mid s_t = s)
      \label{eq:direct-policy-evaluation-1}
      \\
    }
    \only<2->{
      &=\sum_{k=0}^{T-t} \E^\pol_\mdp( r_{t+k} \mid s_t = s)
      \label{eq:direct-policy-evaluation-2}
      \\
    }
    \only<3->{
      &= \E^\pol_\mdp (r_t \mid s_t = s) + \E^\pol_\mdp (U_{t+1} \mid s_t = s)
      \\
    }
    \only<4->{
      &=
      \E_\mdp^\pol(r_{t} \mid s_t = s) + \sum_{i \in \CS} V_{\mdp,t+1}^\pol(i) \Pr_{\mdp}^{\pol}(s_{t+1} = i | s_t = s).
    }
  \end{align}
  Note that the last term can be calculated easily through marginalisation.
  \[\Pr_{\mdp}^{\pol}(s_{t+1} = i | s_t = s)
  =
  \sum_{a \in \CA} \Pr_{\mdp}(s_{t+1} \eq i | s_t \eq s, a_t \eq a) \Pr^\pol(a_t \eq a | s_t \eq s).
  \]
  This derivation directly gives a number of \alert{policy evaluation algorithms}.
\end{frame}


\begin{frame}
  \paragraph{Direct policy evaluation}
  \only<article>{
    Direct policy evaluation is based on \eqref{eq:direct-policy-evaluation-2}, which can be implemented by Algorithm~\ref{alg:direct-policy-evaluation}. One needs to \emph{marginalise out} all possible state sequences to obtain the expected reward given the state at time $t+k$ giving the following:
    \[
    \E^\pol_\mdp( r_{t+k} \mid s_t = s) = \sum_{\mathclap{s_{t+1}, \ldots, s_{t+k} \in \CS^{k}}} \E^\pol_\mdp( r_{t+k} \mid s_{t+k}) \Pr^\pol_\mdp(s_{t+1}, \ldots, s_{t+1} \mid s_t).
    \]
    By using the Markov property, we calculate the probability of reaching any state from any other state at different times, and then add up the expected reward we would get in that state under our policy. Then $\hat{V_t}(s) = V_{\mdp, t}^\pol(s)$ by definition.

    Unfortunately it is not a very good idea to use direct policy evaluation. The most efficient implementation involves calculating $P(s_t \mid s_0)$ recursively for every state. This would result in a total of $|\CS|^3 T$ operations. Monte-Carlo evaluations should be considerably cheaper, especially when the transition structure is sparse.
  }
  \begin{algorithm}[H]
    \begin{algorithmic}[1]
      \FOR{$s \in \CS$}
      \FOR{$t = 0, \ldots, T$}
      \STATE
      \[
      \hat{V_t}(s) = \sum_{k=t}^T \sum_{j \in \CS} \Pr_{\mdp}^{\pol}(s_k = j \mid s_t = s) \E^\pol_\mdp(r_k \mid s_k = j).
      % ro: Something's seems to be wrong with the indexes here.
      % cd: really? seems OK to me.
      \]
      \ENDFOR
      \ENDFOR
    \end{algorithmic}
    \caption{Direct policy evaluation}
    \label{alg:direct-policy-evaluation}
  \end{algorithm}
\end{frame}

\subsection{Monte-Carlo policy evaluation}
\index{policy evaluation!Monte Carlo}
\label{sec:MC-PE}
\begin{frame}
  \only<article>{ Another conceptually simple algorithm is Monte-Carlo
    policy evaluation shown as Algorithm~\ref{alg:monte-carlo-policy-evaluation}. 
    The idea is that instead of summing over all
    possible states to be visited, we just draw states from the Markov
    chain defined jointly by the policy and the \index{Markov decision process}Markov decision process. Unlike direct policy evaluation the algorithm needs a parameter $K$, the number of trajectories to generate. Nevertheless, this is a very useful method, employed within a number of more complex algorithms. 
    % It is also applicable to any kind of decision process.  %ro: outcommented, since not sure what this refers to
  }
  \begin{algorithm}[H]
    \begin{algorithmic}
      \FOR{$s \in \CS$}
      \FOR{$k = 0, \ldots, K$}
      \STATE Choose initial state $s_1$.
      \FOR{$t = 1, \ldots, T$}
      \STATE $a_t \sim \pol(a_t \mid s_t)$ \hfill \textrm{// Take action}
      \STATE Observe reward $r_t$ and next state $s_{t+1}$.
      \STATE Set $r_{t,k} = r_t$.
      \ENDFOR
      \STATE Save total reward:
      \[
      \hat{V}_k(s) = \sum_{t=1}^T r_{t,k}.
      \]
      \ENDFOR
      \STATE Calculate estimate:
      \[
      \hat{V}(s) = \frac{1}{K} \sum_{k=1}^K \hat{V}_k(s).
      \]
      \ENDFOR
    \end{algorithmic}
    \caption{Monte-Carlo policy evaluation}
    \label{alg:monte-carlo-policy-evaluation}
  \end{algorithm}
  \begin{remark}
    The estimate $\hat{V}$ of the Monte Carlo evaluation algorithm satisfies
    \[
    \|V - \hat{V}\|_\infty \leq \sqrt{\frac{\ln(2|\CS|/\delta)}{2K}}
    \qquad
    \textrm{with probability $1 - \delta$}
    \]
  \end{remark}
  \only<article>{
    \begin{proof}
      From Hoeffding's \index{Hoeffding inequality} inequality \eqref{eq:hoeffding} we have for any state $s$ that
      \[
      \Pr\left(
        |\hat{V}(s) - V(s)| \geq \sqrt{\frac{\ln(2|\CS|/\delta)}{2K}}
      \right)
      \leq \delta/|\CS|.
      \]
      Consequently, using a union bound of the form $P(A_1 \cup A_2 \cup \ldots \cup A_n) \leq \sum_i P(A_i)$ gives the required result.
    \end{proof}
  }
\end{frame}
\only<article>{
  The main advantage of Monte-Carlo policy evaluation is that it can be used in very general settings. It can be used not only in Markovian environments such as MDPs, but also in partially observable and multi-agent settings.
}

  \subsection{Backwards induction policy evaluation}
\begin{frame}

  \only<article>{
    Finally, the backwards induction algorithm shown as Algorithm \ref{alg:bipe} is similar to the backwards induction algorithm we saw for sequential sampling and bandit problems.  However, here we are only evaluating a policy rather than finding the optimal one. This algorithm is slightly less generally applicable than the Monte-Carlo method because it makes Markovian assumptions. The Monte-Carlo algorithm, can be used for environments that with a non-Markovian variable $s_t$.
  }


  
  \index{policy evaluation!backwards induction}
  \begin{algorithm}[H]
    \begin{algorithmic}
      \STATE For each state $s \in S$, for $t = 1, \ldots, T - 1$:
      \begin{equation}
        \hat{V}_{t}(s) = r^\pol_\mdp(s) + \sum_{j \in S} \Pr_{\mdp}^{\pol}(s_{t+1} = j \mid s_t = s) \hat{V}_{t+1}(j),
        \label{eq:bi-pe-recursion}
      \end{equation}
      with $\hat{V}_T(s) = r^\pol_\mdp(s)$.
    \end{algorithmic}
    \caption{Backwards induction policy evaluation}
    \label{alg:bipe}
  \end{algorithm}

  \begin{theorem}
    The backwards induction algorithm gives estimates $\hat{V}_t(s)$ satisfying 
    \begin{equation}
      \hat{V}_t(s) = V^\pol_{\mdp,t}(s)
      \label{eq:bi-pe-property}
    \end{equation}
  \end{theorem}
  \only<article>{
    \begin{proof}
      For $t = T-1$, %ro: corrected this from T to T-1 
      the result is obvious. We can prove the remainder by induction. Let \eqref{eq:bi-pe-property} hold for all $t \geq n + 1$.
      Now we prove that it holds for $n$. Note that from the recursion \eqref{eq:bi-pe-recursion} we have:
      \begin{align*}
        \hat{V}_t(s)
        &=
        r_\mdp(s) + \sum_{j \in S} \Pr_{\mdp, \pol}(s_{t+1} = j \mid s_t = s) \hat{V}_{t+1}(j)
        \\
        &=
        r(s) + \sum_{j \in S} \Pr_{\mdp, \pol}(s_{t+1} = j \mid s_t = s) V^\pol_{\mdp,t+1}(j)
        \\
        &=
        r(s) + \E_{\mdp,\pol}(U_{t+1} \mid s_t = s)
        \\
        &=
        \E_{\mdp,\pol}(U_{t} \mid s_t = s) = V^\pol_{\mdp, t}(s),
      \end{align*}
      where the second equality is by the induction hypothesis, the third and fourth equalities are by the definition of the utility, and the last by definition of $V^\pol_{\mdp, t}$.
    \end{proof}
  }
\end{frame}

\subsection{Backwards induction policy optimisation}
\index{policy optimisation!backwards induction}
\label{sec:finite-horiz-backw}
\only<article>{
  Backwards induction as given in \cref{alg:finite-BI} is the first non-naive algorithm for finding an optimal policy for the sequential problems with $T$ stages. It is basically identical to the backwards induction algorithm we saw in Chapter~\ref{cha:sequential-sampling}, which was for the very simple sequential sampling problem, as well as the backwards induction algorithm for the decision-theoretic bandit problem.
}

\begin{frame}
  \begin{algorithm}[H]
    \begin{algorithmic}
      \STATE Input $\mdp$, set $\CS_T$ of states reachable within $T$ steps.
      \STATE Initialise $V_T(s):=\max_a r(s,a)$, for all $s \in \CS_T$. %ro: I guess this is how you want to initialize this?
      \FOR{$n=T-1, T-2, \ldots, 1$}
      \FOR{$s \in \CS_n$}
      \STATE $\pol_n(s) = \argmax_a r(s, a) + \sum_{s' \in \CS_{n+1}} P_\mdp (s' \mid s, a)  V_{n+1} (s')$ %ro: deleted a star here?
      \STATE $V_n(s) = r(s,a) + \sum_{s' \in \CS_{n+1}} P_\mdp(s' \mid s,\pol_n(s)) V_{n+1} (s')$
      \ENDFOR
      \ENDFOR
      \STATE Return $\pol = (\pol_n)_{n=1}^T$.
    \end{algorithmic}
    \caption{Finite-horizon backwards induction}
    \label{alg:finite-BI}
  \end{algorithm}

  \only<article>{
  }
  \only<presentation>{
    \begin{block}{Notes}
      \begin{itemize}
        \only<1>{\item $\Pr_{\mdp}^{\pol}(s'|s) = \sum_a P(s'|s,a) \Pr_\pol(a|s)$.}
      \item Finite horizon problems only, or approximations (e.g. lookahead in game trees).
      \item For stochastic problems , we marginalize over states.
      \item As we know the optimal choice at the last step, we can find the optimal policy!
      \item Can be used with estimates of the value function.
      \end{itemize}
    \end{block}
  }
\end{frame}

\begin{frame}
  \begin{theorem}
    For $T$-horizon problems, backwards induction is optimal, i.e.
    \begin{equation}
      V_n(s) = V^*_{\mdp,n}(s)\label{eq:finite-horizon-induction-hypothesis}
    \end{equation}
  \end{theorem}
  \begin{proof}
    Note that the proof below also holds for $r(s,a) = r(s)$.
    First we show that $V_t \geq V^*_t$. %ro: Isn't this all we need to show? %cd : now, we also need to show equality!
    For $n=T$ we evidently have $V_T(s) = \max_a r(s, a) = V^*_{\mdp, T}(s)$.
    Now assume that for $n \geq t + 1$, \eqref{eq:finite-horizon-induction-hypothesis} holds.
    Then it also holds for $n = t$, since for any policy $\pi'$
    \begin{align*}
      V_t(s)
      \only<4>{
        &= \max_a \set{r(s, a) + \sum_{j \in \CS} P_\mdp (j \mid s,a) V_{t+1}(j)}
        \\
      }
      \only<5>{
        &\geq \max_a \set{r(s, a) + \sum_{j \in \CS} P_\mdp (j \mid s,a) V^*_{\mdp,t+1}(j)}
        && \textrm{(by induction assumption)}
        \\
      }
      \only<6>{
        &\geq \max_a \set{r(s, a) + \sum_{j \in \CS} P_\mdp (j \mid s,a) V^{\pol'}_{\mdp,t+1}(j)}
        \\
      }
      \only<7>{
        &\geq V_t^{\pol'}(s).
      }
    \end{align*}
    This holds for any policy $\pol'$, including $\pol'=\pol$, the policy returned by backwards induction. Then: %ro: I'm not sure why this argument is needed. % cd: this gives us the lower bound.
    \[
    V_{\mdp,t}^*(s) \geq V_{\mdp,t}^{\pol}(s) = V_t(s) \geq V_{\mdp,t}^*(s).
    \]
  \end{proof}
  \only<article>{
    \begin{remark}
      A similar theorem can be proven for arbitrary $\CS$. This requires using $\sup$ instead of $\max$ and proving the existence of a $\pol'$ that is arbitrary-close in value to $V^*$. For details, see~\citep{Puterman:MDP:1994}.
    \end{remark}
  }
\end{frame}



\section{Infinite-horizon}
\label{sec:infinite-horizon}
\newcommand {\msqr} {\vrule height0.33cm width0.44cm}
\newcommand {\bsqr} {\vrule height0.55cm width0.66cm}
\only<article>{When problems have no fixed horizon, they usually can be modelled as infinite horizon problems, sometimes with help of a \emph{terminating state}, whose visit terminates the problem, or discounted rewards, which indicate that we care less about rewards further in the future. 
  When reward discounting is exponential, these problems can be seen as undiscounted problems with random and geometrically distributed horizon. 
  For problems with no discounting and no termination states there are some complications in the definition of optimal policy. However, we defer discussion of such problems to Chapter~\ref{cha:distr-free-reinf}.} %ro: In my (rather outdated) pdf this reference does not exist.


\subsection{Examples}
\label{sec:IH-examples}

We begin with some examples, which will help elucidate the concept of terminating states and infinite horizon. The first is shortest path problems, where the aim is to find the shortest path to a particular goal. Although the process terminates when the goal is reached, not all policies may be able to reach the goal, and so the process may never terminate.

\subsubsection{Shortest-path problems}

We shall consider two types of shortest path problems, deterministic and stochastic. Although conceptually very different, both problems have essentially the same complexity.

Consider an agent moving in a maze, aiming to get to some terminating goal state $X$. That is, when reaching this state, the agent cannot move anymore, and receives a reward of $0$. In general, the agent can move deterministically in the four cardinal directions, and receives a negative reward at each time step. Consequently, the optimal policy is to move to $X$ as quickly as possible.

  \begin{minipage}{.37\textwidth}
    \begin{tabular}{*{8}{|@{}p{0.23cm}}|}
      \hline
      ~14  & ~13 & ~12 &~11  &~10  & ~9  & ~8  & ~7  \\\hline
      ~15  &\msqr& ~13 &\msqr&\msqr&\msqr&\msqr& ~6  \\\hline
      ~16  & ~15 & ~14 &\msqr& ~4  & ~3  & ~4  & ~5  \\\hline
      17  &\msqr&\msqr&\msqr&\msqr& ~2  &\msqr&\msqr\\\hline
      ~18  & ~19 & ~20 &\msqr& ~2  & ~1  & ~2  &\msqr\\\hline
      ~19  &\msqr& ~21 &\msqr& ~1  & ~0  & ~1  &\msqr\\\hline
      ~20  &\msqr& ~22 &\msqr&\msqr&\msqr&\msqr&\msqr\\\hline
      ~21  &\msqr& ~23 &~24  & ~25 &~26  &~27  & ~28 \\\hline
    \end{tabular}
  \end{minipage}
  \hspace{0.1cm}
  \begin{minipage}{.57\textwidth}
    \begin{block}{Properties}
      \begin{itemize}
        \itemsep 0pt
      \item $\disc = 1$, $T \to \infty$.
      \item $r_t=-1$ unless $s_t=X$, in which case $r_t=0$.
      \item $\Pr_\mdp(s_{t+1}=X | s_{t}=X) = 1$.
      \item $\CA=\{ \textrm{North}, \textrm{South}, \textrm{East}, \textrm{West}\}$
      \item Transitions are deterministic and walls block.
      \end{itemize}
    \end{block}
  \end{minipage}
Solving the shortest path problem can be done simply by looking at the distance of any point to $X$. Then the reward obtained by the optimal policy starting from any point, is simply the negative distance. The optimal policy simply moves to the state with the smallest distance to $X$.


  \paragraph{Stochastic shortest path problem with a pit}
  Now assume the shortest path problem with stochastic dynamics. That
  is, at each time-step there is a small probability $\omega$ that
  move to a random direction.  In addition, there is a pit $O$, that
  is a terminating state with a reward of $-100$.

  \begin{minipage}{.37\textwidth}
    \begin{tabular}{*{8}{|@{}p{0.23cm}}|}
      \hline
      &     &     &     &     &     &     &     \\\hline
      &\msqr&\msqr&\msqr&\msqr&\msqr&\msqr&     \\\hline
      &\msqr&     &     &     &     &     &     \\\hline
      &\msqr&     &\msqr&\msqr&     &\msqr&\msqr\\\hline
      &     &     &\msqr&     &     &     &\msqr\\\hline
      &\msqr&  ~O &\msqr&     & ~X   &     &\msqr\\\hline
      &\msqr&     &\msqr&\msqr&\msqr&\msqr&\msqr\\\hline
      &\msqr&     &     &     &     &     &     \\\hline
    \end{tabular}
  \end{minipage}
  \hspace{0.1cm}
  \begin{minipage}{.57\textwidth}
    \begin{block}{Properties}
      \begin{itemize}
      \item $\disc=1$, $T \to \infty$.
      \item $r_t=-1$, but $r_t=0$ at X and $-100$ at O and episode ends.
      \item $\Pr_\mdp(s_{t+1}=X | s_{t}=X) = 1$.
      \item $\CA=\{ \textrm{North}, \textrm{South}, \textrm{East}, \textrm{West}\}$
      \item Moves to a random direction with probability $\omega$.  Walls block.
      \end{itemize}
    \end{block}
  \end{minipage}
  % For what value of $\omega$ is it better to take the dangerous
  % shortcut?  (However, if we want to take into account risk explicitly we must
  % modify the agent's utility function.)

  \begin{figure}[H]
    \centering
    \subfigure[$\omega = 0.1$]{\includegraphics[width=0.45\textwidth]{pit_random_0_1}}
    \subfigure[$\omega = 0.5$]{\includegraphics[width=0.45\textwidth]{pit_random_0_5}}
    \subfigure[value]{\includegraphics[width=0.45\textwidth]{color-axis}}
    \caption{Pit maze solutions for two values of $\omega$.}
    \label{fig:pit-solution}
  \end{figure}
Randomness changes the solution significantly in this environment. When $\omega$ is relatively small, it is worthwhile (in expectation) for the agent to pass past the pit, even though there is a risk of falling in and getting a reward of $-100$. In the example given, even starting from the third row, the agent prefers taking the short-cut. For high enough $\omega$, the optimal policy avoids approaching the pit. Still, the agent prefers jumping in the pit, than being trapped at the bottom of the maze forever.


\subsubsection{Continuing problems}
Finally, many problems have no natural terminating state, but are continuing \emph{ad infinitum}. Frequently, we model those problems using a utility that discounts future rewards exponentially. This way, we can guarantee that the utility is bounded. In addition, exponential discounting also has some economical sense. This is partially because of the effects of inflation, and partially because money now may be more useful than money in the future. Both these effects diminish the value of money over time.  As an example, consider the following inventory management problem. 

\begin{example}[Inventory management]
  There are $K$ storage locations, and each location $i$ can store
  $n_i$ items.  At each time-step there is a probability $\phi_i$ that
  a client tries to buy an item from location $i$, where $\sum_i
  \phi_i \leq 1$.  If there is an item available, when this happens,
  you gain reward $1$.  There are two types of actions, one for
  ordering a certain number $u$ units of stock, paying $c(u)$.
  Further one may move $u$ units of stock from one location $i$ to
  another location $j$, paying $\psi_{ij}(u)$.
\end{example}

An easy special case is when $K=1$, and we assume that deliveries
happen once every $m$ timesteps, and each time-step a client arrives
with probability $\phi$.  Then the state set $\CS=\{0, 1, \ldots, n
\}$ corresponds to the number of items we have, the action set
$\CA=\{0, 1, \ldots, n\}$ to the number of items we may order.  The
transition probabilities are given by $P(s'|s,a) =
\binom{m}{d}\phi^d(1-\phi)^{m-d}$, where $d=s+a-s'$, for $s+a \leq n$.
  % ro: What's the solution?
  % cd: I guess I should try and give the solution to this problem for these special cases.


\newcommand{\Node}[3]{%
  \pgfnodecircle{#1}[stroke]{#2}{0.3cm}%
  \pgfputat{\pgfrelative{#2}{\pgfxy(0,-.075)}}{\pgfbox[center,base]{#3}}}

\newcommand{\SNode}[3]{%
  \pgfnodebox{#1}[stroke]{#2}{0.3cm}%
  \pgfputat{\pgfrelative{#2}{\pgfxy(0,-.075)}}{\pgfbox[center,base]{#3}}}

\newcommand{\BNode}[3]{%
  \pgfnodecircle{#1}[stroke]{#2}{0.4cm}%
  \pgfputat{\pgfrelative{#2}{\pgfxy(0,-.075)}}{\pgfbox[center,base]{#3}}}

\newcommand{\Claim}[2]{%
  \pgfputat{\pgfrelative{\pgfxy(0.4,-0.075)}{\pgfnodecenter{#1}}}%
  {\pgfbox[left,base]{#2}}}

\newcommand{\LClaim}[2]{%
  \pgfputat{\pgfrelative{\pgfxy(-0.4,-0.075)}{\pgfnodecenter{#1}}}%
  {\pgfbox[right,base]{#2}}}

\newcommand{\Bush}[3]{%
  \pgfnodecircle{#1}[virtual]{\pgfrelative{\pgfnodecenter{#2}}{#3}}{1pt}%
  \pgfnodeconnline{#2}{#1}}


\subsection{Markov chain theory for discounted problems}
\label{sec:markov-chain-theory}
  Here we consider MDPs with infinite horizon and discounted rewards. We shall consider undiscounted rewards only in Chapter~\ref{cha:distr-free-reinf}.
  Our utility in this case is the discounted total reward:
  \[
  U_t = \lim_{T\to \infty} \sum_{k=t}^T \disc^k r_k, \qquad \disc \in (0,1)
  \]
For simplicity, in the following we assume that rewards only depend on the current state instead of both state and action. It can easily be verified that results still hold in the latter case. More importantly, we also assume that the state and action spaces $\CS, \CA$ are finite, and that the transition kernel of the MDP is time-invariant. This allows us to use the following simplified vector notation:
\begin{itemize}
\item $\val^\pol = \left(\E^\pol(U_t \mid s_t = s)\right)_{s \in \CS}$ is a vector in $\Reals^{|\CS|}$ representing the value of policy $\pol$.
\item Sometimes we will use $p(j|s,a)$ as a shorthand for $\Pr_{\mdp}(s_{t+1} = j \mid s_t = s, a_t = a)$.
\item $\trans_{\mdp,\pol}$ is a transition matrix in $\Reals^{|\CS|\times|\CS|}$ for policy $\pol$, such that
  \[
  \trans_{\mdp,\pol}(i,j) = \sum_a p(j \mid i, a) \Pr^\pol(a \mid i).
  \]
\item $\rew$ is a reward vector in $\Reals^{|\CS|}$.
\item The space of value functions $\Vals$ is a Banach space \only<article>{(i.e., a complete, normed vector space)} equipped with the norm
  \[
  \|\val\| = \sup\cset{|\val(s)|}{s \in \CS}
  \]
  
\end{itemize}
  \only<article>{For infinite-horizon discounted MDPs, stationary policies are sufficient. This can be proven by induction, using arguments similar to other proofs given here. For a detailed set of proofs, see~\cite{Puterman:MDP:1994}.}
  \begin{definition}
    A policy $\pol$ is stationary if $\pol(a_t \mid s_t) = \pol(a_n \mid s_n)$ for all $n, t$. %ro: can't understand notation nor definition
    \label{def:stationary-policy}
  \end{definition}
  \only<article>{We now present a set of important results that link Markov decision processes to linear algebra.}
  \begin{remark}
    We can use the Markov chain kernel $\trans$ to write the expected reward vector as
    \begin{align}
      \val^\pol
      &=
      \sum_{t=0}^\infty \disc^{t} \trans_{\mdp,\pol}^{t} \rew
      \label{eq:kernel-value}
    \end{align}
    \label{rem:kernel-value}
  \end{remark}
  \only<article>{
    \begin{proof}
      \begin{align*}
        V^\pol(s)
        &= \E \left(\sum_{t=0}^\infty \disc^t r_t ~\middle|~ s_0 = s\right)
        \\
        &= \sum_{t=0}^\infty \disc^t \E(r_t | s_0 = s)
        \\
        &= \sum_{t=0}^\infty \disc^t \sum_{i \in \CS} \Pr(s_t = i \mid s_0 = s) \E(r_t \mid s_t = i).
      \end{align*}
      Since for any distribution vector $\vectorsym{p}$ over $\CS$, we have $\E_\vectorsym{p} r_t = \transpose{\vectorsym{p}} \rew$, the result follows.%ro: Actually, you need some knowledge about the power of the transition matrix, right?
    \end{proof}
  }

  It is possible to show that the expected discounted total reward of a policy is equal to the expected undiscounted total reward with a geometrically distributed horizon (see exercise~\ref{exercise:geometric-discounting}. As a corollary, it follows a Markov decision process with discounting is equivalent with one where there is no discounting, but a stopping probability $(1 - \disc)$ at every step.

The value of a particular policy can be expressed as a linear equation. This is an important result, as it has led to a number of successful algorithms that employ linear theory.
  \begin{theorem}
    For any stationary policy $\pol$, $\val^\pol$ is the unique solution of
    \begin{equation}
      \label{eq:bellman-equation}
      \val = \rew + \disc \trans_{\mdp,\pol} \val. \only<presentation>{\quad \leftarrow \textrm{fixed point}}
    \end{equation}
    In addition, the solution is:
    \begin{equation}
      \label{eq:bellman-solution}
      \val^\pol = (\ident - \disc \trans_{\mdp,\pol})^{-1} \rew,
    \end{equation}
    \label{the:inverse-value}
    where $\ident$ is the identity matrix.
  \end{theorem}

    To prove this we will need the following important theorem. 
    \begin{theorem}\label{thm:wert}
      For any bounded linear transformation $\matrixsym{A} : S \to S$ on a normed
      linear space $S$ (i.e., there is $c < \infty$ s.t. $\|\matrixsym{A}x\|:=\sup_i \sum_j a_{i,j} \leq c\|x\|$ for all $x \in S$ with \emindex{spectral radius}
      $\sigma(\matrixsym{A}) \defn \lim_{n\to \infty} \|\matrixsym{A}^n\|^{1/n} < 1$), $\matrixsym{A}^{-1}$ exists and is given by
      \begin{equation}
        \matrixsym{A}^{-1} = \lim_{T \to \infty} \sum_{n=0}^T (\ident - \matrixsym{A})^n.
      \end{equation}
    \end{theorem}

  \begin{proof}[Proof of Theorem~\ref{the:inverse-value}]
    First note that by manipulating the infinite sum in Remark~\ref{rem:kernel-value}, one obtains $\rew = (\ident - \disc \trans_{\mdp,\pol}) \val^\pol$.
    Since $\|\disc \trans_{\mdp,\pol}\| < 1 \cdot \|\trans_{\mdp_\pol}\| = 1$, the inverse 
    \[
    (\ident - \disc \trans_{\mdp,\pol})^{-1} = \lim_{n \to \infty} \sum_{t=0}^n (\disc \trans_{\mdp,\pol})^t
    \]
    exists by Theorem \ref{thm:wert}.
    It follows that
    \[
    \val = (\ident - \disc \trans_{\mdp,\pol})^{-1} \rew
    = \sum_{t=0}^\infty \disc^t \trans_{\mdp,\pol}^t \rew = \val^\pol,
    \]
    where the last step is by Remark~\ref{rem:kernel-value} again.%ro: do we need to argue for uniqueness of the solution?
  \end{proof}
It is important to note that the matrix $\matrixsym{X} = (\ident - \disc \trans_{\mdp,\pol})^{-1}$ can be seen as the expected number of discounted cumulative visits to each state $s$, starting from state $s'$ and following policy $\pol$. More specifically, the entries of the matrix are:
\begin{equation}
x(s,s') = \E^\pol_\mdp
\left\{
  \sum_{t=0}^\infty \gamma^t \Pr_\mdp^\pol(s_t = s' \mid s_t = s)
\right\}.
\label{eq:cumulative-visits}
\end{equation}
This interpretation is quite useful, as many algorithms rely on an estimation of $\matrixsym{X}$ for approximating value functions.

\subsection{Optimality equations}
\label{sec:optimality-equations}
\only<article>{Let us now look at the backwards induction algorithms in terms of operators. We introduce the operator of a policy, which is the one-step backwards induction operation for a fixed policy, and the Bellman operator, which is the equivalent operator for the optimal policy. If a value function is optimal, then it satisfies the Bellman optimality equation.}
\begin{frame}
  \begin{definition}[Policy and Bellman operator]
    \only<article>{The linear operator of a policy $\pol$ is:}
    \begin{equation}
      \blm_\pol \val \defn\rew + \disc \trans_\pol \val %ro: Here, the mu is dropped. Maybe we should drop it already earlier. % cd: hm, ok, maybe that makes sense
      \label{eq:policy-operator}
    \end{equation}
    Sby contract    \only<article>{The (non-linear) Bellman operator in the space of value functions $\Vals$ is defined as:}
    \begin{equation}
      \blm \val \defn \sup_\pol \set{\rew + \disc \trans_\pol \val}, \qquad \val \in \Vals
      \label{eq:bellman-operator}
    \end{equation}    
    \label{def:bellman-operator}
  \end{definition}
  \only<article>{
    We now show that the Bellman operator satisfies the following monotonicity properties with respect to an arbitrary value vector $\val$.
  }
  \begin{theorem}
    Let $\val^* \defn \sup_\pol \val^\pol$. Then for any bounded $\rew$, it holds that for $\val \in \Vals$:
    \begin{itemize}
    \item[(1)] If $\val \geq \blm \val$, then $\val \geq \val^*$.
    \item[(2)] If $\val \leq \blm \val$, then $\val \leq \val^*$.
    \item[(3)] If $\val = \blm \val$, then $\val$ is unique and $\val = \sup_\pol \val^\pol$.
      Therefore, $\val = \blm \val$ is called the Bellman optimality equation.
    \end{itemize}
  \end{theorem}
  \only<article>{
    \begin{proof}
      We first prove (1). A simple proof by induction over $n$ %ro: Is there an easier way to see this?
      shows that for any $\pol$
      \begin{align*}
        \val
        &\geq
        \rew + \disc \trans_\pol \val
        \geq
        \sum_{k=0}^{n-1}\disc^k\trans_\pol^k\rew
        + \disc^n \trans_\pol^n \val.
      \end{align*}
      Since $\val^\pol = \sum_{t=0}^\infty \disc^t \trans^t_\pol \rew$ it follows that
      \[
      \val - \val^\pol 
      \geq
      \disc^n \trans_\pol^n \val
      -\sum_{k=n}^{\infty}\disc^k\trans_\pol^k\rew.
      \]
      The first-term on the right-hand side can be bounded by arbitrary $\epsilon/2$ for large enough $n$. Also note that %ro: I'm confused by the second part of the proof. Shouldn't we have lower bounds for the terms here to get what we want?
      \[
      \sum_{k=n}^{\infty}\disc^k\trans_\pol^k\rew \geq -\frac{\disc^n \vectorsym{e}}{1 - \disc},
      \]
      with $\vectorsym{e}$ being a unit vector, so this can be bounded by $\epsilon/2$ as well. So for any $\pol, \epsilon > 0$:
      \[
      \val \geq \val^\pol - \epsilon,
      \]
      so
      \[
      \val \geq \sup_\pol \val^\pol.
      \]
      An equivalent argument shows that
      \[
      \val \leq \val^\pol + \epsilon,
      \]
      proving (2).
      Putting together (1) and (2) gives (3).
    \end{proof}
  }
\end{frame}


\begin{frame}
  \only<article>{
    We eventually want show that repeated application of the Bellman operator converges to the optimal value.
    As a preparation, we need the following theorem. 
  }
  \begin{theorem}[Banach Fixed-Point theorem]
    Suppose $\CS$ is a Banach space (i.e. a complete normed linear space) and $T : \CS \to \CS$ %ro: Using \CS here is quite unfortunate, as this usually denotes the state space, which is a bit confusing at the beginning of the next proof.
    is a contraction mapping (i.e. $\exists \disc \in [0,1)$ s.t. $\|Tu-Tv\| \leq \disc \|u-v\|$ for all $u,v \in \CS$). Then
    \begin{itemize}
    \item there is a unique $u^* \in U$ s.t. $Tu^* = u^*$, and
    \item for any $u^0 \in \CS$ the sequence $\{u^n\}$:
      \[
      u^{n+1}  = Tu^{n} = T^{n+1} u^0
      \]
      converges to $u^*$.
    \end{itemize}
    \label{the:fixed-point}
  \end{theorem}
  \begin{proof}
    For any $m \geq 1$
    \begin{align*}
      \|u^{n+m} - u^{n}\|
      &\leq
      \only<1->{
        \sum_{k=0}^{m-1} \|u^{n+k+1} - u^{n + k}\|
        =
        \sum_{k=0}^{m-1} \|T^{n+k}u^{1} - T^{n+k}u^{0}\|
      }
      \only<2->{
        \\
        &\leq
        \sum_{k=0}^{m-1} \disc^{n+k} \|u^{1} - u^{0}\|
        = 
        \frac{\disc^n(1 - \disc^m)}{1 - \disc}\|u^1 - u^0\|.  %ro: A few words why this is sufficient would be in order.
      }
    \end{align*}
  \end{proof}
\end{frame}

\begin{frame}
  \begin{theorem}
    For $\disc \in [0,1)$ the Bellman operator $\blm$ is a contraction mapping in $\Vals$.
    \label{the:bellman-contraction}
  \end{theorem}
  \begin{proof}
    Let $\val, \val' \in \Vals$. Consider $s \in \CS$ such that $\blm \val(s) \geq \blm \val'(s)$, and let
    \[
    a^*_s \in \argmax_{a \in \CA} \set{r(s) + \sum_{j \in \CS} \disc p_\mdp(j\mid s, a) \val(j)}.
    \]
    Using the fact that $a^*_s$ is optimal for $\val$, but not necessarily for $\val'$, we have:
    \begin{align*}
      0
      &\leq
      \blm \val(s) - \blm \val'(s)
      \leq  
      \only<article>{
        \sum_{j \in S} \disc p(j \mid s, a^*_s) \val(j)
        -
        \sum_{j \in S} \disc p(j \mid s, a^*_s) \val'(j)
        \\
        &=
        \disc \sum_{j \in S}  p(j \mid s, a^*_s) [\val(j) - \val'(j)]
        \\
        &\leq 
        \disc \sum_{j \in S}  p(j \mid s, a^*_s) \|\val - \val'\|
        =
      }
      \disc \|\val - \val'\|.
    \end{align*}
    Repeating the argument for $s$ such that $\blm \val(s) \leq \blm \val'(s)$, we obtain
    \[
    |\blm \val (s) - \blm \val'(s)| \leq \disc\|\val - \val'\|.
    \]
    Taking the supremum over all possible $s$, the required result follows.
  \end{proof}
  It is easy to show the same result for the $\blm_\pol$ operator, as a corollary to this theorem.
\end{frame}

\begin{frame}
  \begin{theorem}
    For discrete $\CS$, bounded $\rew$, and $\disc \in [0,1)$
    \begin{enumerate}[(i)]
    \item there is a unique $\val^* \in \Vals$ such that $\blm \val^* = \val^*$ and such that $\val^* = V^*_\mdp$,
    \item for any stationary policy $\pol$, there is a unique $\val \in \Vals$ such that $\blm_\pol \val = \val$ and $\val = V^\pol_\mdp$.
    \end{enumerate}
    \label{the:bellman-convergence}
  \end{theorem}
  \only<article>{
    \begin{proof}
      As the Bellman operator $\blm$ is a contraction by Theorem~\ref{the:bellman-contraction}, application of the fixed-point Theorem \ref{the:fixed-point}
      shows that there is a unique $\val^* \in \Vals$ such that $\blm \val^* = \val^*$. This is also the optimal value function due to Theorem~\ref{the:bellman-contraction}.%ro: The latter reference seems to be wrong, but I'm not sure what's the correct reference.
      The second part of the theorem follows from the first part when considering only a single policy $\pi$ (which then is optimal).
      % Use part 1 with $\Pols = \{\pol\}$.  %ro: \Pol hasn't been used in this section so far.
    \end{proof}
  }
\end{frame}

\subsection{MDP Algorithms}
\label{sec:mdp-algorithms}
\only<article>{Let us now look at three basic algorithms for solving a known Markov decision process. The first, \textit{value iteration}\index{value iteration}, is a simple extension of the backwards induction algorithm to the infinite horizon case. %This is possible to do, since the value converges and consequently we do not need to store all value vectors. %ro: outcommented this: also for backwards induction you need not store all the values, so didn't understand this
}
\subsubsection{Value iteration}
In this version of the algorithm, we assume that rewards are dependent only on the state. An algorithm for the case where reward only depends on the state can be obtained by replacing $r(s,a)$ with $r(s)$.
\label{sec:value-iteration}
\begin{frame}
  \begin{algorithm}[H]
    \begin{algorithmic}
      \STATE Input $\mdp$, $\CS$.
      \STATE Initialise $\val_0 \in \Vals$. %ro: What is \Vals here?
      \FOR{$n=1, 2, \ldots$}
      \FOR{$s \in \CS_n$}
      \STATE $\pol_n(s) = \argmax_{a \in \CA} \set{r(s, a) + \disc \sum_{s' \in \CS} P_\mdp(s' \mid s, a) \val_{n-1} (s')}$
      \STATE $\val_n(s) = r(s, \pol_n(s)) + \disc \sum_{s' \in \CS} P_\mdp(s' \mid s, \pol_n(s)) \val_{n-1} (s')$
      \ENDFOR
      \STATE \textbf{break} if \texttt{termination-condition} is met %ro: what is this condition?
      \ENDFOR
      \STATE Return $\pol_n, V_n$.
    \end{algorithmic}
    \caption{Value iteration}
  \end{algorithm}
\end{frame}

  The value iteration algortihm is a direct extension of the backwards induction algorithm for an infinite horizon. However, since we know that stationary policies are optimal, we do not need to maintain the values and actions for all time steps. At each step, we can merely keep the previous value $\val_{n-1}$. However, since there is an infinite number of steps, we need to know whether the algorithm converges to the optimal value, and what is the error we make at a particular iteration.
  \begin{theorem}
    The value iteration algorithm satisfies
    \begin{itemize}
    \item $\lim_{n \to \infty} \|\val_n - \val^*\| = 0$. 
    \item For each $\epsilon>0$ there exists $N_\epsilon <\infty$ such that for all $n\geq N_\epsilon$
      \begin{equation}
        \|\val_{n+1} - \val_n\| \leq \epsilon(1 - \disc)/2\disc.
        \label{eq:value-iteration-stopping}
      \end{equation}
    \item For $n\geq N_\epsilon$ the policy $\pol_\epsilon$ that takes action
      \[
      \argmax_a r(s,a) + \disc \sum_j p(j|s,a)\val_n(s')
      \]
      is
      $\epsilon$-optimal, i.e. $V^{\pol_\epsilon}_\mdp(s) \geq V^*_\mdp(s) - \epsilon$ for all states $s$.
    \item $\|\val_{n+1} - \val^*\| < \epsilon/2$ for $n \geq N_\epsilon$.
    \end{itemize}
  \end{theorem}
  \begin{proof}
    The first two statements follow from the fixed-point Theorem \ref{the:fixed-point}.
    Now note that
    \[
    \|V^{\pol_\epsilon}_\mdp - \val^*\|
    \leq
    \|V^{\pol_\epsilon}_\mdp - \val_n\|
    +
    \|\val_n - \val^*\|
    \]
    We can bound these two terms easily:  %ro: Some explanations would be helpful. Actually, I only understand the first equality. % cd: I had typos!
    \only<article>{
      \begin{align*}
        \norm{V^{\pol_\epsilon}_\mdp - \val_{n+1}}
        &= 
        \norm{\blm_{\pol_\epsilon} V^{\pol_\epsilon}_\mdp - \val_{n+1}}
        \tag{by definition of $\blm_{\pol_\epsilon}$}
        \\
        &\leq 
        \norm{\blm_{\pol_\epsilon} V^{\pol_\epsilon}_\mdp - \blm \val_{n+1}}
        +
        \norm{\blm \val_{n+1} - \val_{n+1}}
        \tag{triangle}
        \\
        &=
        \norm{\blm_{\pol_\epsilon} V^{\pol_\epsilon}_\mdp - \blm_{\pol_\epsilon} \val_{n+1}}
        +
        \norm{\blm \val_{n+1} - \blm \val_{n}}
        \tag{by definition}
        \\
        &\leq
        \disc \norm{V^{\pol_\epsilon}_\mdp - \val_{n+1}}
        +
        \disc \norm{\val_{n+1} -\val_{n}}.
        \tag{by contraction}
      \end{align*}
      An analogous argument gives the same bound for the second term  $\|\val_n - \val^*\|$. Then, rearranging we obtain
    }
    \[
    \norm{V^{\pol_\epsilon} - \val_{n+1}} 
    \leq
    \frac{\disc}{1 - \disc}\|\val_{n+1} - \val_{n}\|,
    \qquad
    \|\val_{n+1} - \val^*\|
    \leq
    \frac{\disc}{1 - \disc}\|\val_{n+1} - \val_{n}\|,
    \]
    and the third and fourth statements follow from the second statement.
  \end{proof}

  The \emindex{termination condition} of value iteration has been left unspecified. However, the theorem above shows that if we terminate when \eqref{eq:value-iteration-stopping} is true, then our error will be bounded by $\epsilon$. However, better termination conditions can be obtained.

Now let us prove how fast value iteration converges.
  \begin{theorem}[Value iteration monotonicity]
    Let $\Vals$ be the set of value vectors with Bellman operator $\blm$. Then:
    \begin{enumerate}
    \item Let $\val, \val' \in \Vals$ with $\val' \geq \val$. Then $\blm
      \val' \geq \blm \val$.
    \item Let $\val_{n+1} = \blm \val_n$. If there is an $N$ s.t.\ $\blm
      \val_N \leq \val_N$, then $\blm \val_{N+k} \leq \val_{N+k}$ for
      all $k \geq 0$ and similarly for $\geq$.
    \end{enumerate}
    \label{the:value-iteration-monotonicity}
  \end{theorem}
  \begin{proof}
    Let $\pol \in \argmax_\pol \rew + \disc \trans_{\mdp,\pol} \val$. Then
    \[
    \blm \val
    = \rew + \disc \trans_{\mdp,\pol} \val
    \leq \rew + \disc \trans_{\mdp,\pol} \val'
    \leq \max_{\pol'}\rew + \disc \trans_{\mdp,\pol'} \val',
    \]
    where the first inequality is due to the fact that $\trans \val \geq \trans \val'$ for any $\trans$.
    For the second part, 
    \[
    \blm \val_{N+k} = \val_{N+k+1} = \blm^k \blm \val_N \leq \blm^k \val_N = \val_{N+k}.
    \]
    since $\blm \val_N \leq \val_N$ by assumption and consequently $\blm^k \blm \val_N \leq \blm^k \val_N$ by part one of the theorem.
  \end{proof}

  Thus, value iteration converges monotonically to $V^*_{\mdp}$ if the
  initial value $\val_0 \leq \val'$ for all $\val'$.  If $r \geq 0$,
  it is sufficient to set $\val_0 = \mathbf{0}$. Then $\val_n$ is always
  a lower bound on the optimal value function.
\begin{theorem}
  Value iteration converges with error in $O(\disc^n)$
  More specifically, for $r \in [0,1]$ and $\val_0 = \mathbf{0}$, 
  \begin{align*}
    \|\val_n - V_\mdp^*\| &\leq \frac{\disc^n}{1 - \disc},
    &
    \|V^{\pol_n}_\mdp - V_\mdp^*\| &\leq \frac{2\disc^n}{1 - \disc}.
  \end{align*}
\end{theorem}
\begin{proof}
  The first part follows from the contraction property (Theorem~\ref{the:bellman-contraction}):
  \begin{equation}
    \|\val_{n+1} - \val^*\|
    =
    \|\blm \val_{n} - \blm \val^*\|
    \leq 
    \disc \|\val_{n} - \val^*\|.  %ro: This proves the convergence rate, what about the inequalities? 
  \end{equation}
  Now divide by $\disc^n$ to obtain the final result.
  
\end{proof}

Although value iteration converges exponentially fast, the convergence
is dominated by the discount factor $\disc$. When $\disc$ is very
close to one, convergence can be extremely slow.  In fact,
\citet{tseng1990solving} showed that the number of iterations are on
the order of $1 / (1 - \disc)$, for bounded accuracy of the input
data. The overall complexity is
$\tilde{O}(|\CS|^2 |\CA| L (1 - \disc)^{-1}$, omitting logarithmic
factors, where $L$ is the total number of bits used to represent the
input.\footnote{Thus the result is \emph{weakly} polynomial complexity, due to the dependence on the input size description.}

\subsubsection{Policy iteration}
\label{sec:policy-iteration}
\index{policy iteration}
Unlike value iteration, \textit{policy iteration} attempts to iteratively improve a given policy, rather than a value function. At each iteration, it calculates the value of the current policy and then  calculates the policy that is greedy with respect to this value function. For finite MDPs, the policy evaluation step can be performed with either linear algebra or backwards induction, while the policy improvement step is trivial. The algorithm described below can be extended to the case when the reward also depends on the action, by replacing $\rew$ with the policy-dependent reward vector $\rew_\pol$. 

\begin{algorithm}[H]
  \begin{algorithmic}
    \STATE Input $\mdp$, $\CS$.
    \STATE Initialise $\val_0$.
    \FOR{$n=1, 2, \ldots$}
    \STATE $\pol_{n+1} = \argmax_\pol \set{\rew + \disc \trans_\pol \val_n}$ \qquad \texttt{ // policy improvement}  
    \STATE $\val_{n+1} = V^{\pol_{n+1}}_\mdp$ \qquad \texttt{ // policy evaluation}  
    \STATE \textbf{break} if $\pol_{n+1} = \pol_n$.
    \ENDFOR
    \STATE Return $\pol_n, \val_n$.
  \end{algorithmic}
  \caption{Policy iteration}
  \label{alg:policy-iteration}
\end{algorithm}


The following theorem describes an important property of policy iteration, namely that the policies generated are monotonically improving.

\begin{theorem}
  Let $\val_n, \val_{n+1}$ be the value vectors generated by policy iteration.
  Then $\val_n \leq \val_{n+1}$.
\end{theorem}
\begin{proof}
  From the policy improvement step
  \[
  \rew + \disc \trans_{\pol_{n+1}} \val_n
  \geq
  \rew + \disc \trans_{\pol_{n}} \val_n,
  = \val_n
  \]
  where the equality is due to the policy evaluation step for $\pol_{n}$. Rearranging, we get $ \rew \geq (\ident - \disc \trans_{\pol_{n+1}}) \val_n$
  and hence
  \begin{align*}
    (\ident - \disc \trans_{\pol_{n+1}})^{-1} \rew &\geq \val_n,
  \end{align*}
  noting that the inverse is positive. 
  Since the left side equals $\val_{n+1}$ by the policy evaluation step for $\pol_{n+1}$, the theorem follows.
\end{proof}
We can use the fact that the policies are monotonically improving to show that policy iteration will terminate after a finite number of steps.
\begin{corollary}
  If $\CS, \CA$ are finite, then policy iteration terminates after a finite number of iterations and returns an optimal policy.
\end{corollary}

\begin{proof}
  There is only a finite number of policies, and since policies in policy iteration are monotonically improving, the algorithm must stop after finitely many iterations.
  Finally, the last iteration satisfies
  \begin{equation}
    \val_n = \max_\pol \rew + \disc \trans_\pol \val_n.
  \end{equation}
  Thus $\val_n$ solves the optimality equation.
\end{proof}

However, it is easy to see that the number of policies is $|\CA|^{|\CS|}$, thus the above corollary only guarantees exponential-time convergence in the number of states. However, it is also known that the complexity of policy iteration is strongly polynomial~\cite{ye2011simplex}, for any fixed $\disc$, with the number of iterations required being $\frac{|\CS|^2 (|\CA| - 1)}{1 - \disc} \cdot \ln \left( \frac{|\CS|^2}{1 - \disc} \right)$.

Policy iteration seems to have very different behaviour from value iteration. In fact, one can obtain families of algorithms that lie at the extreme ends of the spectrum between policy iteration and value iteration. The first member of this family is modified policy iteration, and the second member is temporal difference policy iteration.


\subsubsection{Modified policy iteration}
\index{policy iteration!modified}
The astute reader will have noticed that it may be
not necessary to fully evaluate the improved
policy. In fact, we can take advantage of that to speed up policy iteration. Thus, a simple variant of policy iteration involves doing only a $k$-step update for the policy evaluation step. For $k=1$, the algorithm becomes identical to value iteration, while for $k \to \infty$ the algorithm is equivalent to policy iteration, as $\val_n = V^{\pol_n}$.
\begin{algorithm}[H]
  \begin{algorithmic}
    \STATE Input $\mdp$, $\CS$.
    \STATE Initialise $\val_0$.
    \FOR{$n=1, 2, \ldots$}
    \STATE $\pol_{n} = \argmax_\pol \rew + \disc \trans_\pol \val_{n-1}$ \qquad \texttt{ // policy improvement}  
    \STATE $\val_{n} = \blm_{\pol_{n}}^k \val_{n-1}$ \qquad \texttt{// partial policy evaluation}
    \STATE \textbf{break} if $\pol_{n} = \pol_{n+1}$.
    \ENDFOR
    \STATE Return $\pol_n, \val_n$.
  \end{algorithmic}
  \caption{Modified policy iteration}
\end{algorithm}
Modified policy iteration can perform much better than either pure value iteration or pure policy iteration.
\subsubsection{A geometric view}
\index{temporal differences}
It is perhaps interesting to see the problem from a geometric perspective. This also gives rise to the so-called ``temporal-difference'' set of algorithms.
First, we define the difference operator, which is the difference between a value function vector $\val$ and its transformation via the Bellman operator.
\begin{definition}
  The \emindex{difference operator} is defined as 
  \begin{equation}
    \label{eq:td-operator}
    \pim \val \defn \max_\pol \set{\rew + (\disc \trans_\pol - \ident) \val} = \blm \val - \val.
  \end{equation}
  \label{def:td-operator}
\end{definition}
Essentially, it is the change in the value function vector when we apply the Bellman operator. 
Thus the Bellman optimality equation can be rewritten as
\begin{equation}
  \label{eq:td-equation}
  \pim \val = \textbf{0}.
\end{equation}
Now let us define the set of greedy policies with respect to a value vector  $\val \in \Vals$ to be:
\[
\Pols_\val \defn \argmax_{\pol \in \Pols} \set{\rew + (\disc \trans_\pol - \ident)\val}.
\]
We can now show the following inequality between the two different value function vectors.
\begin{theorem}
  For any $\val, \val' \in \Vals$ and $\pol \in \Pols_\val$
  \begin{equation}
    \label{eq:td-support}
    \pim \val' \geq \pim \val + (\disc \trans_\pol - \ident)(\val' - \val).
  \end{equation}
\end{theorem}
\begin{proof}
  By definition, $\pim \val' \geq \rew + (\disc \trans_\pol - \ident)\val'$,
  while $\pim \val = \rew + (\disc \trans_\pol - \ident)\val$. Subtracting the latter from the former gives the result.
\end{proof}
Equation \eqref{eq:td-support} is similar to the convexity of the Bayes-optimal utility \eqref{eq:convex-bayes-util}. 
Geometrically, we can see from a look at Figure~\ref{fig:difference-operator}, that applying the Bellman operator on value function always improves it, yet may have a negative effect on the other value function. If the number of policies is finite, then the figure is also a good illustration of the policy iteration algorithm, where each value function improvement results in a new point on the horizontal axis, and the choice of the best improvement (highest line) for that point. In fact, we can write the policy iteration algorithm in terms of the difference operator.

\begin{figure}[ht]
  \centering
  \if 0
  \begin{tikzpicture}[
      pnt/.style={
        circle,
        fill=black,
        thick,
        inner sep=2pt,
        minimum size=0.1cm
      }
    ] 
    \node at (0,0) (start) {};
    \node at (2,0) (vs) [pnt,label=below:$\val^*$] {};
    \node at (5,0) (v1) [pnt,label=below:$\val$] {};
    \node at (8,0) (v2) [pnt,label=below:$\val'$] {};
    \node at (10,0) (end)  {};
    \draw (start) -- (end);
    \node at (8,1) (Bs2) {};
    \node at (8,2) (B2) {};
    \node at (2,-1) (B2s) {};
    \draw (Bs2) -- (vs);
    \draw (B2) -- (B2s);
  \end{tikzpicture}
  \fi
  \input{figures/difference-operator.tikz}
  \caption{The difference operator. The graph shows the effect of the operator for the optimal value function $\val^*$, and two arbitrary value functions, $\val_1, \val_2$. Each line is the improvement effected by the greedy policy $\pol^*, \pol_1, \pol_2$ with respect to each value function $\val^*, \val_1, \val_2$. }
  \label{fig:difference-operator}
\end{figure}
\begin{theorem}
  Let $\{\val_n\}$ be the sequence of value vectors obtained from policy iteration. Then for any $\pol \in \Pols_{\val_n}$,
  \begin{equation}
    \val_{n+1} = \val_n - (\disc \trans_{\pol} - \ident)^{-1} \pim \val_n.
  \end{equation}
\end{theorem}
\begin{proof}
  By definition, we have for  $\pol \in \Pols_{\val_n}$
  \begin{align*}
    \val_{n+1} 
    &=
    (\ident - \disc \trans_\pol)^{-1} \rew - \val_n + \val_n
    \\
    &=
    (\ident - \disc \trans_\pol)^{-1} [
    \rew - (\ident - \disc \trans_\pol)\val_n] + \val_n.
  \end{align*}
  Since $\rew - (\ident - \disc \trans_\pol)\val_n = \pim \val_n$ the claim follows.
\end{proof}


\subsubsection{Temporal-Difference Policy Iteration}
\index{policy iteration!temporal-difference} \index{temporal
  difference} 

 In \emph{temporal-difference policy iteration},
similarly to the modified policy iteration algorithm, we replace the
next-step value with an approximation $\val_n$ of the $n$-th policy's
value. Informally, this approximation is chosen so as to reduce the
discrepancy of our value function over time.

At the $n$-th iteration of the algorithm, we use a policy improvement step to obtain the next policy $\pol_{n+1}$ given our current approximation $\val_n$:
\begin{equation}
  \blm_{\pol_{n+1}} \val_n = \blm \val_n.
\end{equation}
To update the value from $\val_n$ to $\val_{n+1}$ we rely on the \emindex{temporal difference error}, defined as:
\begin{equation}
  d_n(i,j) = [\rew(i) + \disc \val_n(j)] - \val_n(i).
\end{equation}
This can be seen as the difference in the estimate
when we move from state $i$ to state $j$. In fact, it is easy to see that, if our value function estimate satisfies $\val = V^{\pol_n}$, then the expected error should be zero, as:
\[
\sum_{j \in \CS} d_n(i,j) p(j \mid i, \pol_n (i)) = 
\sum_{j \in \CS} [\rew(i) + \disc \val_n(j)] p(j \mid i, \pol_n(i))
-\val_{n}(i).
\]
Note the similarity to the difference operator in modified policy iteration.  The idea of the temporal-difference policy iteration is to use adjust the current value $\val_n$, using the temporal differences mixed over an infinite number of steps:
\begin{align}
  \vectorsym{\tau}_n(i) &= \sum_{t=0}^\infty \E_{\pol_n} \left[(\disc \lambda)^t d_n(s_t,s_{t+1}) \mid s_0 = i\right],\\
  \val_{n+1} &= \val_n + \vectorsym{\tau}_n.
\end{align}
Here the $\lambda$ parameter is a simple way to mix together the different temporal difference errors. If $\lambda \to 1$, our error will be dominated by the terms far in the future, while if $\lambda \to 0$, our error $\vectorsym{\tau}_n$, will be dominated by the short-term discrepancies in our value function. In the end, we shall adjust our value function in the direction of this error.

Putting all of those steps together, we obtain the following algorithm:
\label{sec:temp-diff-policy}
\begin{frame}
  \begin{algorithm}[H]
    \begin{algorithmic}
      \STATE Input $\mdp$, $\CS, \lambda$.
      \STATE Initialise $\val_0$.
      \FOR{$n=0, 1, 2, \ldots$}
      \STATE $\pol_{n+1} = \argmax_\pol \rew + \disc \trans_\pol \val_{n}$ \qquad \texttt{ // policy improvement}
      \STATE $\val_{n+1} = \val_{n} + \vectorsym{\tau}_{n} \qquad \texttt{// temporal difference update}$.
      \STATE \textbf{break} if $\pol_{n+1} = \pol_{n}$.
      \ENDFOR
      \STATE Return $\pol_n, \val_n$.
    \end{algorithmic}
    \caption{Temporal-Difference Policy Iteration}
  \end{algorithm}


  In fact, $\val_{n+1}$ is the unique fixed point of the following equation:
  \begin{equation}
    \tdm_n \val \defn (1 - \lambda) \blm_{\pol_{n+1}} \val_n + \lambda \blm_{\pol_{n+1}} \val.
  \end{equation}
  That is, if we repeatedly apply the above operator to some vector $\val$, then at some point we shall obtain a fixed point $\val^* = \tdm_n \val^*$. It is interesting to see what happens at the two extreme choices of $\lambda$ in this case. 
  For $\lambda = 1$, this becomes identical to standard policy iteration, as the fixed point satisfies $\val^* = \blm_{\pol_{n+1}} \val^*$, so then $\val^*$ must be the value of policy $\pol_{n+1}$. For $\lambda = 0$, one obtains standard value iteration, as the fixed point is reached under one step and is simply $\val^* =  \blm_{\pol_{n+1}} \val_n$, i.e. the approximate value of the one-step greedy policy.
  In other words, the new value vector is moved only  partially towards the direction of the Bellman update, depending on how we choose $\lambda$. 

\end{frame}


\subsubsection{Linear programming}\index{linear programming}
\label{sec:linear-programming}
Perhaps surprisingly, we can also solve Markov decision processes through linear programming. The main idea is to reformulate the maximisation problem as a linear optimisation problem with linear constraints. 
The first step in our procedure is to recall that there is an easy way to determine whether a particular $\val$ is an upper bound on the optimal value function $\val^*$, since if
\[
\val \geq \blm \val
\]
then $\val \geq \val^*$. In order to transform this into a linear program, we must first define a scalar function to minimise. We can do this by selecting some arbitrary distribution on the states $\vectorsym{y} \in \Simplex^{|\CS|}$. %ro: not sure the \Simplex notation has been introduced before.
Then we can write the following linear program.
\begin{block}{Primal linear program}
  \[
  \min_\val \transpose{\vectorsym{y}} \val,
  \]
  such that
  \[
  \val(s) - \disc \transpose{\vectorsym{p}_{s,a}} \val \geq r(s,a), %ro: has the \vectorsym{p}_{s,a} notation be introduced before?
  \qquad \forall a \in \CA, s \in \CS,
  \]
\end{block}
where we use $\vectorsym{p}_{s,a}$ to denote the vector of next state probabilities $p(j \mid s, a)$.

Note that the inequality condition is equivalent to $\val \geq \blm \val$.
Consequently, the problem is to find the smallest $\val$ that satisfies this inequality. When $\CA, \CS$ are finite, it is easy to see that this will be the optimal value function and the Bellman equation is satisfied.

It also pays to look at the dual linear program, which is in terms of a maximisation. This time, instead of finding the minimal upper bound on the value function, we find the maximal cumulative discounted state-action visits $x(s,a)$ that are consistent with the transition kernel of the process. 

\begin{block}{Dual linear program}
  \[
  \max_x \sum_{s \in \CS} \sum_{a \in \CA} r(s,a) x(s,a)
  \]
  such that $x \in \Reals_+^{|\CS \times \CA|}$ and
  \[
  \sum_{a \in \CA} x(j,a) - \sum_{s \in \CS} \sum_{a \in \CA} \disc
  p(j \mid s,a) x(s,a) = y(j) \qquad \forall j \in \CS.
  \]
  with $\vectorsym{y} \in \Simplex^{|\CS|}$.
\end{block}

In this case, $x$ can be interpreted as the discounted sum of state-action visits, as proved by the following theorem.
\begin{theorem}
  For any policy $\pol$,
  \[
  x_\pol(s,a) = \E_{\pol,\mdp} \left\{ \sum \disc^n \ind{s_t = s, a_t = a \mid s_0 \sim y}\right\}  %ro: what is the policy here?
  \]
  is a feasible solution to the dual problem.
  On the other hand, if $x$ is a feasible solution to the dual problem then $\sum_a x(s,a) > 0$. Finally, if we define the strategy
  \[
  \pi(a \mid s) = \frac{x(s,a)}{\sum_{a' \in \CA} x(s,a')}  %ro: Is this the policy you need for the first claim?
  \]
  then $x_\pol = x$ is a feasible solution.
\end{theorem}
The equality condition ensures that $x$ is consistent with the transition kernel of the Markov decision process. Consequently, the program can be seen as search among all possible cumulative state-action distributions to find the one giving the highest total reward.



\section{Summary}
Markov decision processes can represent shortest path problems,
stopping problems, experiment design problems,
multi-armed bandit problems and reinforcement learning problems.

Bandit problems are the simplest type of Markov decision process, since they have a fixed, never-changing state. However, to solve them, one can  construct a Markov decision processes in belief space, within a Bayesian framework. It is then possible to apply backwards induction to find the optimal policy.

Backwards induction is applicable more generally to arbitrary Markov decision processes. For the case of infinite-horizon problems, it is referred to as value iteration, as it converges to a fixed point.
It is tractable when either the state space $\CS$ or the horizon
$T$ are small (finite).

When the horizon is infinite, policy iteration can also be used to find optimal policies. It is different from value iteration in that at every step, it fully evaluates a policy before the improvement step, while value iteration only performs a partial evaluation. In fact, at the $n$-th iteration, value iteration has calculated the value of an $n$-step policy. 

We can arbitrarily mix between the two extremes of policy iteration and value iteration in two ways. Firstly, we can perform a $k$-step partial evaluation. When $k=1$, we obtain value iteration, and when $k \to \infty$, we obtain policy iteration. The generalised algorithm is called modified policy iteration. Secondly, we can perform adjust our value function by using a temporal difference error of values in future time steps. Again, we can mix liberally between policy iteration and value iteration by focusing on errors far in the future (policy iteration) or on short-term errors (value iteration).

Finally, it is possible to solve MDPs through linear programming. This is done by reformulating the problem as a linear optimisation with constraints. In the primal formulation, we attempt to find a minimal upper bound on the optimal value function. In the dual formulation, our goal is to find a distribution on state-action visitations that maximises expected utility and is consistent with the MDP model.

\section{Further reading}

\only<article>{See the last chapter of~\citep{Degroot:OptimalStatisticalDecisions} for further information on the MDP formulation of bandit problems in the decision theoretic setting. This was explored in more detail in Duff's PhD thesis~\citep{duff2002olc}. 
  When the number of (information) states in the bandit problem is finite, \citet{Gittins:1989} has proven that it is possible to formulate simple index policies. However, this is not generally applicable. Easily computable, near-optimal heuristic strategies for bandit problems will be given in Chapter~\ref{cha:distr-free-reinf}. The decision-theoretic solution to the unknown MDP problem will be given in Chapter~\ref{cha:bayes-reinf-learn}.

  Further theoretical background on  Markov decision processes, including many of the theorems in Section~\ref{sec:infinite-horizon}, can be found in~\citep{Puterman:MDP:1994}. Chapter 2 of~\cite{BertsekasTsitsiklis:NDP} gives a quick overview of MDP theory from the operator perspective. The introductory reinforcement learning book of~\citet{Sutton+Barto:1998} also explains the basic \index{Markov decision process}Markov decision process framework.}







\chapter{Recommendation systems}
\label{ch:recommendation}
\only<article>{Structured learning problems involve multiple latent variables with a complex structure. These range from clustering and spech recognition to DNA and biological and social network analysis. Since structured problems include relationships between many variables, they can be analysed using graphical models.}
\section{Recommendation systems}
\only<presentation>{
  \begin{frame}
    \tableofcontents[ 
    currentsection, 
    hideothersubsections, 
    sectionstyle=show/shaded
    ] 
  \end{frame}
}

\begin{frame}
  \begin{figure}[H]
    \centering
    \includegraphics[width=0.4\textwidth]{../figures/recommendation}
    \only<article>{\caption{The recommendation problem}}
    \label{fig:recommendation}
  \end{figure}
  \only<article>{In many machine learning applications, we are dealing with the problem of proposing one or more alternatives to a human. The human can accept zero or more of these choices. As an example, when using an internet search engine, we typically see two things: (a) A list of webpages matching our search terms (b) A smaller list of advertisements that might be relevant to our search. At a high level, }
  \begin{block}{The recommendation problem}
    At time $t$
    \begin{enumerate}
    \item A customer $\bx_t$ appears. \only<article>{For the internet search problem, $\bx_t$ would at least involve the search term used.}
    \item We present a choice $a_t$. \only<article>{For the matching website, the choice is ranked list of websites. For the advertisements, however, it is typical }
    \item The customer chooses $y_t$. \only<article>{This might include selecting one or more of items suggested in $a_t$. The choice of the customer may not be directly visible.}
    \item We obtain a reward $r_t = \rho(a_t, y_t) \in \Reals$. \only<article>{Typically this is a payment either from the customer or an advertiser.}
    \end{enumerate}
  \end{block}
\end{frame}

\begin{frame}
  \begin{alertblock}{The two problems in recommendation systems}
    \begin{itemize}
    \item The modelling (or prediction) problem. \only<article>{Given the data, how to e.g. predict what movies a user likes and dislikes.}
    \item The recommendation problem. \only<article>{What movie(s) to actually recommend to a user.}
    \end{itemize}
    \only<article>{Although closely linked, those two problems have different evaluation metrics. The recommendation problem is harder, especially because the data do not tell us what our recommendation have been in the past, but only what users watched.}
  \end{alertblock}
\end{frame}


\begin{frame}
  \frametitle{How to predict user preferences?}
  \only<article>{In order for us to be able to provide good recommendations, we need to be able to predict user preferences. In the case of movies, preferences of a person for a movie can be expressed in terms of the hypothetical rating that a user would give to a movie. As Fig.~\ref{fig:user-ratings} shows, we frequently do have data about individual user ratings for movies, and we would like to somehow use those.}
  \begin{figure}[H]
    \centering \includegraphics[width=0.9\textwidth]{../figures/recommendationExample-1}
    \caption{User ratings}
    \label{fig:user-ratings}
  \end{figure}

\end{frame}

\begin{frame}
  \begin{figure}[H]
    \centering
    \includegraphics[width=0.9\textwidth]{../figures/netflix}
    \caption{What to recommend?}
    \label{fig:netflix}
  \end{figure}
  \only<article>{
    \begin{example}
      In the case of Netflix and related services, we would like to suggest movies to users which they are more likely to watch, as shown in Figure~\ref{fig:netflix}. However, how can we tell which movies those can be? It is probably not useful to just recommend them to rewatch a previously watched movie. We need to somehow take into account information across our user database: if somebody watched mostly the same films as you, then maybe you'd be interested in watching those movies she has that you haven't seen.

      In the Netflix catalogue, in particular, users also post reviews of the movies they have watched, as shown in Figure~\ref{fig:user-ratings}. This allows us to be able to guess the ratings of users from previous user's ratings.
    \end{example}
  }
\end{frame}



\begin{frame}
  \frametitle{Predictions based on similarity}
  \begin{block}{Content-based filtering.}
    \begin{itemize}
    \item Users typically like similar items. \only<article>{For example, a horror movie fan typically rates horror movies highly.}
    \item That means we can one user's ratings and \alert{item information} to predict their ratings for other items. \only<article>{In this scenario, we do not need to take into account the ratings of other people.}
    \end{itemize}
  \end{block}

  \begin{block}{Collaborative filtering}
    \begin{itemize}
    \item \alert{Similar users have similar tastes}. \only<article>{For example, consider two users $t, u$ who have each watched a set of movies $\CM_t$ and $\CM_u$ respectively, and $\CM_{t,u} = \CM_t \cap \CM_u$ is the set of common movies. If their ratings are the same for those movies, i.e. $x_{t.m} = x_{u,m} \forall m \in \CM_{t,u}$, then it's a good guess that they might have the same ratings for movies they have not both watched. }
    \item That means we can use similar user's \alert{ratings} to predict the ratings for other users. 
      \only<article>{The advantage is that ratings are readily available. The disadvantage is that new users have too few data to be matched to other users.}
    \end{itemize}
  \end{block}
\end{frame}


\begin{frame}
  \frametitle{$k$-NN for similarity}
  \begin{exercise}
    \begin{itemize}
    \item Define a distance $d : \CX^M \times \CX^M \to \Reals_+$ between user ratings.
    \item Apply a $k$-NN-like algorithm to prediction of user ratings from the dataset.
    \end{itemize}
  \end{exercise}
\end{frame}
\begin{frame}
  \begin{block}{Similarity between users}
    \only<article>{Let us define a similarity $w_{ij} \geq 0$ between two users so that}
    \[
    \sum_{j \neq i} w_{i,j} = 1,
    \qquad
    w^m_{i,j} \defn w_{i,j} \ind{x_{j,m}} / \sum_k w_{i,k} \ind{x_{k,m}}.
    \]
    \only<article>{There $w^m_{i,j}$ only considers those users who have rated movie $m$, so that $\sum_{j \neq i : x_{j,m} > 0} x^m_{i,j} = 1$. The overall similarity $w_i,j$ itself does not have to sum up to one, but it does need to be non-negative.}
  \end{block}

  \begin{example}[$k$-nearest neighbours]
    $w_{i,j} = 1/k$ for the $k$ nearest neighbours with respect to $d$.
  \end{example}


  \begin{example}[Weighted distance]
    \[
    w_{i,j} = \frac{\exp[-d(i,j)]}{\sum_{k \neq i} \exp[-d(i,j)]}
    \]
  \end{example}

  \begin{block}{Inferred ratings}
    \only<article>{Then we can define the inferred ratings to be the weighted average rating.}
    \[
    \hat{x}_{u.m} = \sum_{j \neq u} w^m_{u,j} x_{u,m}.
    \]
    \only<article>{But if the value $x_{u,m}$ is missing, we need to only}
  \end{block}
  
\end{frame}

\begin{frame}
  \begin{block}{A naive distance metric}
    \only<article>{A simple idea is to just look at the difference between the raw ratings in terms of the L1 norm:}
    \[
    d(i,j) \defn \|\bx_i - \bx_j\|_1.
    \]
    \only<article>{However this has the problem that this makes users who have watched different amounts of movies look very different.}
  \end{block}
  \begin{block}{Ignoring movies which are not shared.}
    \only<article>{
      We perhaps would only like to look at similarity for users with respect to which movies they have rated. This would lead to a distance such as}
    \[
    d(i,j) \defn \sum_{m} \ind{x_{i,m} \wedge x_{j,m}} |x_{i,m} - x_{j,m}|
    \]
  \end{block}
  \begin{block}{Using side-information}
    \only<article>{In some cases, we have additional information about the products or users. For example, users might be connected in a social network. Then we can tag users as similar if they are strongly connected, even if one of them has few or no movie ratings.}
    \only<presentation>{Social network data}
  \end{block}
  \begin{block}{Inferring a latent representation}
    \only<article>{Rather than going through specific algorithms for calculating similarities, we can think about learning a latent representation from the data. For example we could infer a network of users, and so a distance, from the individual ratings:}
    \[
    d(i,j) \defn f(\bx_i, \bx_j, \theta)
    \]
    \only<article>{More generally, we can try and infer some other latent representation.}
  \end{block}
\end{frame}

\subsection{Least squares representation}
\begin{frame}
  \frametitle{Latent representation}
  \begin{block}{The predictive model}
    \begin{itemize}
    \item $x_{um}$ rating of user $u$ for movie $m$.
    \item $r_{um} = \ind{x_{um} > 0}$ indicates which movies are rated.
    \item $\bz_{m} \in \Reals^n$: an $n$-dimensional representation of a movie.
    \item $\bc_{u} \in \Reals^n$: an $n$-dimensional representation of a user.
    \end{itemize}

    Given $\MC, \MZ$, our predicted movie rating can be written as
    \[
    \hat{x}_{u,m} \defn \bc_u^\top \bz_m, \qquad \hat{\MX} \defn \MC^\top \MZ.
    \]
    \only<article>{On the left side we have individual ratings and on the right side, in matrix form.}
  \end{block}
  \only<article>{We can now try and see how well this prediction fits the data for a movie. One idea is to simply use $|\hat{x}_{m,u} - x_{m,u}|$ to show the difference between the prediction and the actual rating. 
    Now can now define the prediction \alert{error} for a given representation $\MC, \MZ$ as}
  \[
  f(\MC, \MZ) = \|(\MR \circ \hat{\MX} - \MR \circ \MX)^\top (\MR \circ \hat{\MX} - \MR \circ\MX)\|_1
  \]

  \only<article>{Here multiplying with the $\MR$ matrix ensures that we ignore pairs with no ratings.
    \begin{block}{Alternating Least Squares~\cite{takacs2012alternating}}
      \begin{align}
        \label{eq:als}
        \MC_{t+1} &= \argmin_{\MC} f(\MC, \MZ_t) + \lambda g(\MZ, \MZ_t)\\
        \MC_{t} &= \argmin_{\MZ} f(\MC_t, \MZ) +  + \lambda g(\MZ_t, \MZ)
      \end{align}
    \end{block}
  }
\end{frame}


\subsection{Preferences as a latent variable}
\begin{frame}
  \frametitle{A simple preference model}
  \only<article>{As a simple model, we can assume that each person belongs to a \alert{type}. Every type has the same preferences over films. In the simplest possible model, a user of type $c_i$ that has watched a movie $m$ will rate the film deterministically $x_{c,m}$. More generally, we can assume the following model.}
  \begin{figure}[H]
    \centering
    \begin{tikzpicture}
      \node[RV] at (2,0) (data) {$\bx_t$};
      \node[RV,hidden] at (0,0) (cluster) {$c_{t}$};
      \draw[->] (cluster)--(data);
      \uncover<2>{
        \node[RV,hidden] at (2,1) (param) {$\param$};
        \draw[->] (param)--(cluster);
        \draw[->] (param)--(data);
      }
    \end{tikzpicture}
    \caption{Basic preference model}
    \label{fig:basic-preference-model}
  \end{figure}
  \begin{example}[Discrete preference model]
    \begin{itemize}
    \item User type $c \in \CC$. \only<article>{For simplicity, we can think of there being a finite number of types $\CC = \{1, \ldots, n\}$.}
    \item User ratings  $\bx$ with $x_{m} \in \CX = \{0, 1\}$ rating for movie $m$.
    \item Preference distribution 
      \[
      P_\param(\bx | \bc) = \prod_{m=1}^M \param_{m,c}^{x_m} (1 - \param_{m,c})^{(1 - x_m)}.
      \]
    \item $P_\param(c) = \param_c$, $\sum_c \param_c = 1$.
    \end{itemize}
  \end{example}
\end{frame}

\begin{frame}
  \frametitle{A more complex preference model}
  \only<article>{As a simple model, we can assume that each person belongs to a \alert{type}. Every type has the same preferences over films. In the simplest possible model, a user of type $c_i$ that has watched a movie $m$ will rate the film deterministically $x_{c,m}$. More generally, we can assume the following model.}
  \begin{figure}[H]
    \centering
    \begin{tikzpicture}
      \node[RV] at (2,0) (data) {$x$};
      \node[RV,hidden] at (0,0) (cluster) {$c$};
      \node[RV,hidden] at (4,0) (movie) {$z$};
      \draw[->] (cluster)--(data);
      \draw[->] (movie)--(data);
      \node[RV,hidden] at (2,1) (param) {$\param$};
      \draw[->] (param)--(cluster);
      \draw[->] (param)--(data);
      \draw[->] (param)--(movie);
    \end{tikzpicture}
    \caption{Preference model}
    \label{fig:preference-model}
  \end{figure}
  \begin{block}{Preference model}
    \begin{itemize}
    \item User type $\bc \in \CC$. \only<article>{For simplicity, we can think of there being a finite number of types $\CC = \{1, \ldots, n\}$.}
    \item Movie type $\bz \in \CZ$.
    \item Preference distribution 
      \[
      P_\param(x | \bc, \bz) = \Normal(\bc^\top \bz, \sigma_\theta)
      \]
    \item Feature prior
      \[
      P_\param(\bc) = \Normal(0, \lambda_\theta)
      \]
    \end{itemize}
  \end{block}

  
\end{frame}


\subsection{The recommendation problem}

\begin{frame}
  \frametitle{What to recommend}
  \begin{figure}[H]
    \centering
    \begin{tikzpicture}
      \node[RV] at (2,0) (data) {$\bx$};
      \node[RV,hidden] at (2,-1) (movie1) {$x_1$};
      \node[RV,hidden] at (2,-2) (movie2) {$x_2$};
      \node[RV,hidden] at (0,0) (cluster) {$c$};
      \node[RV,hidden] at (4,0) (movie) {$z$};
      \draw[->] (cluster)--(data);
      \draw[->] (movie)--(data);
      \draw[->] (cluster)--(data);
      \draw[->] (movie)--(movie1);
      \draw[->] (movie)--(movie2);
      \draw[->] (cluster)--(movie1);
      \draw[->] (cluster)--(movie2);
    \end{tikzpicture}
    \caption{Preference model}
  \end{figure}
  \begin{block}{The recommendation problem for a given $\param$}
    \uncover<2->{
      \begin{align}
        \max_\pol \E^\pol_\param(U \mid \bx)
        &=
          \max_a \sum_{c, z} U(a, y) \alert{\Pr(y \mid a, c, z)} P_\param(c, z \mid \bx)\\
        &=
          \max_a \sum_{c, z} U(a, y) \sum_{x_a} \alert{\Pr(y \mid a, x_a)} P_\param(x_a \mid c, z) P_\param(c, z \mid \bx)
      \end{align}
    }


  \end{block}
\end{frame}


\begin{frame}
  \frametitle{Two ways to model the effect of actions}
  \begin{figure}[H]
    \centering
    \begin{tikzpicture}
      \node[RV] at (2,0) (data) {$\bx$};
      \node[RV,hidden] at (0,0) (cluster) {$c$};
      \node[RV,hidden] at (4,0) (movie) {$z$};
      \draw[->] (cluster)--(data);
      \draw[->] (movie)--(data);
      \draw[->] (cluster)--(data);
      \node[select] at (-2,0) (action) {$a$};
      \node[RV] at (0,-2) (outcome) {$y$};
      \draw[->] (action) -- (outcome);
      \only<1>{
        \node[RV,hidden] at (2,-1) (movie1) {$x_1$};
        \node[RV,hidden] at (2,-2) (movie2) {$x_2$};
        \draw[->] (movie)--(movie1);
        \draw[->] (movie)--(movie2);
        \draw[->] (cluster)--(movie1);
        \draw[->] (cluster)--(movie2);
        \draw[->] (movie1)--(outcome);
        \draw[->] (movie2)--(outcome);
      }
      \only<2>{
        \draw[->] (cluster)--(outcome);
        \draw[->] (movie)--(outcome);
      }
    \end{tikzpicture}
    \caption{Preference model}
  \end{figure}
  \only<presentation>{
    \begin{align}
      \E_\param (U \mid a, \bx) =
      \only<2>{
      \sum_{c, z} U(a, y) \alert{\Pr(y \mid a, c, z)} P_\param(c, z \mid \bx)
      }
      \only<1>{
      \sum_{c, z} U(a, y) \sum_{x_a} \alert{\Pr(y \mid a, x_a)} P_\param(x_a \mid c, z) P_\param(c, z \mid \bx)
      }
    \end{align}
  }
  \only<article>{
    What is the right model for the effects of our actions? In the most general case, the model could depend on the complete set of latent ratings of all the movies. However, it is hard to interpret this, as the user is also probably not aware of what these ratings are themselves. So it seems simpler and more appropriate to predict the outcome based on our action and the latent representation, especially since we will be marginalising over the individual ratings anyway.}
\end{frame}




  %%% Local Variables:
  %%% mode: latex
  %%% TeX-engine: xetex
  %%% TeX-master: "notes.tex"
  %%% End:

%\section{More fun with latent variable models}
\only<presentation>{
  \begin{frame}
    \tableofcontents[ 
    currentsection, 
    hideothersubsections, 
    sectionstyle=show/shaded
    ] 
  \end{frame}
}

\only<article>{Clustering is the problem of automatically seggregating data of different types into clusters. When the goal is \alert{anomaly detection}, then there are typically two clusters. When the goal is \alert{compression} or \alert{auto-encoding} then there are typically as many clusters as needed for sufficienlty good accuracy.}


\begin{frame}
  \frametitle{Clusters as latent variables}
  \begin{figure}[H]
    \centering
    \begin{tikzpicture}
      \node[RV] at (2,0) (data) {$x_t$};
      \node[RV,hidden] at (0,0) (cluster) {$c_{t}$};
      \draw[->] (cluster)--(data);
      \uncover<2>{
        \node[RV,hidden] at (0,1) (param) {$\param$};
        \draw[->] (param)--(cluster);
        \draw[->] (param)--(data);
      }
    \end{tikzpicture}
    \label{fig:bn-cluster}
    \caption{Graphical model for independent data from a cluster distribution.}
  \end{figure}
  \begin{block}{The clustering distribution}
    \only<presentation>{
      \begin{itemize}
      \item Cluster $c_t$
      \item Observation $x_t$
      \item Parameter $\param$.
      \end{itemize}
    }
    \only<article>{The learning problem is to estimate the parameter $\param$ describing the distribution of observations $x_t$ and clusters $c_t$.}
    \[
    x_t \mid c_t = c, \param \sim P_\param(x | c), \qquad c_t \mid \param \sim P_\param(c), \qquad \param \sim \bel(\param)
    \]
    \only<article>{
      Given a parameter $\param$, the clustering problem is to estimate the probability of each cluster for each new observation.
    }
    \[
    P_\param(c_t \mid x_t) = 
    \uncover<2>{\frac{P_\param(x_t \mid c_t) P_\param(c_t)}{\sum_{c'}P_\param(x_t \mid c_t = c') P_\param(c_t = c')}}
    \]
  \end{block}
\end{frame}

\begin{frame}
  \frametitle{Bayesian formulation of the clustering problem}
  \begin{itemize}
  \item Prior $\bel$ on parameter space $\Param$.
  \item Data $x^T = x_1, \ldots, x_T$. Cluster assignments $c^T$ unknown.
  \item Posterior $\bel(\cdot \mid x^T)$.
  \end{itemize}
  \begin{block}{Posterior distribution}
    \only<article>{The data we obtain do not include the cluster assignments, but we can still formulate the posterior distribution of parameters given the data.}
    \begin{align}
      \bel(\param \mid x^T)
      &=
      \frac{P_\param(x^T) \bel(\param)}{\int_\Param P_{\param'}(x^T) \dd \bel(\param')}, &
      P_\param(x^T) &=  \sum_{c^T \in \CC^T} \overbrace{P_\param(x^T \mid c^T)}^{\mathclap{\textrm{Cluster Density}}}  \underbrace{P_\param(c^T)}_{\mathclap{\textrm{Cluster prior}}}
    \end{align}
    \only<article>{We simply need to expand the data-dependent term to include all possible cluster assignments. This is of course not trivial, since the number of assignments is exponential in $T$. However, algorithms such as Markov Chain Monte Carlo can be used instead.}
  \end{block}
  \uncover<2>{
  \begin{block}{Marginal posterior prediction}
    \[
    P_\bel(c_t \mid x_t, x^T) = 
    \int_\Param P_\param(c_t \mid x_t) \dd \bel(\param \mid x^T)
    \]
  \end{block}
}
\end{frame}

\begin{frame}
  \begin{example}[Preference clustering]
    \only<article>{The learning problem is to estimate the parameter $\param$ describing the distribution of observations $x_t$ and clusters $c_t$. In this example, we can assume}
    \[
      \CC = \{1, \ldots, C\}, \qquad x_{t.m} \in \{0,1\}.
    \]
    \only<article>{This means that all movies are either watched or not, and we'd simply want to predict which movie somebody is likely to watch. This allows us to use the following simple priors, splitting the parameters in two parts} $\param  = (\param_1, \param_2)$.
    \begin{block}{Model family}
    \begin{align}
      P_{\param_1}(c_t = c) &=  \param_{1,c},
      & c_t \sim \Multinomial(\param_1)
      \\
      P_{\param_2}(x_{t,m} = 1 \mid c_t = c)           &=\param_{2, m, c}
      & x_{t,m} \mid c_t = c \sim \Bernoulli(\param_{2,m,c})                                           
    \end{align}
  \end{block}
    \only<article>{Since everything is discrete, it makes sense that we can use a Multinomial model for the cluster distribution and a Bernoulli model for whether or not a movie was watched. Now we only need to specify a useful prior for each one of those. The standard priors to use, are a Beta prior for the Bernoulli and the Dirichlet for the Multinomial, as they are conjuate.}
    \begin{block}{Prior}
    \begin{align}
      \param_1 &\sim \Dirichlet(\vg), & \param_2 &\sim \BetaDist(\alpha, \beta)
    \end{align}
    \only<article>{Typically $\vg = (1/2, \ldots, 1/2)$ and $\alpha = \beta = 1/2$ to allow for the possibility of nearly deterministic behaviour.}
  \end{block}
  See \url{src/pymc/beta_bernoulli_clustering.py}
  \end{example}
\end{frame}

\begin{frame}
  \frametitle{Supervised learning}
  \begin{figure}[H]
    \centering
    \begin{tikzpicture}
      \node[RV] at (0,0) (x1) {$x_1$};
      \node[RV] at (0,2) (x2) {$x_T$};
      \node[RV] at (0,4) (x3) {$x_t$};
      \node[RV] at (2,0) (y1) {$y_1$};
      \node[RV] at (2,2) (y2) {$y_T$};
      \node[RV, hidden] at (2,4) (y3) {$y_t$};
      \draw[->] (x1)--(y1);
      \draw[->] (x2)--(y2);
      \draw[->] (x3)--(y3);
    \end{tikzpicture}
    \label{fig:supervised learning}
    \caption{Graphical model for a classical supervised learning problem.}
  \end{figure}

\end{frame}


\begin{frame}
  \frametitle{Semi-supervised learning}
  \begin{figure}[H]
    \centering
    \begin{tikzpicture}
      \node[RV] at (0,0) (x1) {$x_1$};
      \node[RV] at (0,1) (xs) {$x_i$};
      \node[RV] at (0,2) (x2) {$x_T$};
      \node[RV] at (0,4) (x3) {$x_t$};
      \node[RV] at (2,0) (y1) {$y_1$};
      \node[RV,hidden] at (2,1) (ys) {$y_i$};
      \node[RV] at (2,2) (y2) {$y_T$};
      \node[RV, hidden] at (2,4) (y3) {$y_t$};
      \draw[->] (x1)--(y1);
      \draw[->] (xs)--(ys);
      \draw[->] (x2)--(y2);
      \draw[->] (x3)--(y3);
    \end{tikzpicture}
    \label{fig:semi-supervised learning}
    \caption{Graphical model for a classical semi-supervised learning problem.}
  \end{figure}

\end{frame}

\begin{frame}
  \begin{figure}[H]
    \centering
    \begin{tikzpicture}
      \node[RV] at (0,0) (x1) {$x_1$};
      \node[RV] at (0,1) (xs) {$x_i$};
      \node[RV] at (0,2) (x2) {$x_T$};
      \node[RV] at (0,4) (x3) {$x_t$};
      \node[RV] at (2,0) (y1) {$y_1$};
      \node[RV,hidden] at (2,1) (ys) {$y_i$};
      \node[RV] at (2,2) (y2) {$y_T$};
      \node[RV, hidden] at (2,4) (y3) {$y_t$};
      \draw[<-] (x1)--(y1);
      \draw[<-] (xs)--(ys);
      \draw[<-] (x2)--(y2);
      \draw[<-] (x3)--(y3);
    \end{tikzpicture}
    \label{fig:semi-supervised learning}
    \caption{Generative version of the semi-supervised model}
  \end{figure}

\end{frame}

\begin{frame}
  \begin{figure}[H]
    \centering
    \begin{tikzpicture}
      \node[RV] at (0,0) (x1) {$x_1$};
      \node[RV] at (0,2) (x2) {$x_T$};
      \node[RV] at (0,4) (x3) {$x_t$};
      \node[RV, hidden] at (2,0) (y1) {$y_1$};
      \node[RV, hidden] at (2,2) (y2) {$y_T$};
      \node[RV, hidden] at (2,4) (y3) {$y_t$};
      \draw[<-] (x1)--(y1);
      \draw[<-] (x2)--(y2);
      \draw[<-] (x3)--(y3);
    \end{tikzpicture}
    \label{fig:unsupervised learning}
    \caption{Basic unsupervised learning model}
  \end{figure}
  \begin{block}{Applications}
    \begin{itemize}
    \item Clustering
    \item Compression
    \end{itemize}
  \end{block}

\end{frame}






%%% Local Variables:
%%% mode: latex
%%% TeX-engine: xetex
%%% TeX-master: "notes.tex"
%%% End:

%\section{Social networks}

\begin{frame}
  \frametitle{Network model}
  \begin{figure}[H]
    \centering
    \begin{tikzpicture}
      \node[RV,hidden] at (0,0) (x1) {$x_1$};
      \node[RV,hidden] at (2,0) (x2) {$x_2$};
      \node[RV,hidden] at (0,2) (x3) {$x_3$};
      \node[RV,hidden] at (2,2) (x4) {$x_4$};
      \node[RV] at (1,1) (y1) {$y_1$};
      \node[RV] at (3,1) (y2) {$y_2$};
      \node[RV] at (1,3) (y3) {$y_3$};
      \node[RV] at (3,3) (y4) {$y_4$};
    \end{tikzpicture}
    \label{fig:network-model}
    \caption{Graphical model for independent data from a cluster distribution.}
  \end{figure}
\end{frame}

%%% Local Variables:
%%% mode: latex
%%% TeX-master: "notes.tex"
%%% End:

%\section{Sequential structures}
\only<article>{The simplest type of structure in data is sequences. Examples include speech, text and DNA sequences, as well as data acquired in any sequential decision making problem such as recommendation systems or robotics. Sequential data is always thought to arise from some Markovian processes, defined below.}

\only<presentation>{
  \begin{frame}
    \centering
    \includegraphics[height=\textheight]{../figures/Snake-And-Ladder}
  \end{frame}
}
\begin{frame}
\frametitle{Markov process}
  \begin{figure}[H]
    \centering
    \begin{tikzpicture}
      \node[RV] at (-2,0) (xt1) {$x_{t-1}$};
      \node[RV] at (0,0) (xt2) {$x_t$};
      \node[RV] at (2,0) (xt3) {$x_{t+1}$};
      \draw[->] (xt1)--(xt2);
      \draw[->] (xt2)--(xt3);
    \end{tikzpicture}
    \label{fig:markov-chain}
    \caption{Graphical model for a Markov process.}
  \end{figure}
  \begin{definition}[Markov process]
    A Markov process is a sequence of variables $x_t : \Omega \to \CX$ such that $x_{t+1} \mid x_t \indep x_{t-k} \forall k \leq 1$.
  \end{definition}
  \begin{block}{Application}
    \begin{itemize}
    \item Sequence compression (especially with variable order Models).
    \item Web-search (Page-Rank)
    \item Hidden Markov Models.
    \item MCMC.
    \end{itemize}
  \end{block}

\end{frame}



\begin{frame}
\frametitle{Hidden Markov model}
\only<article>{Frequently the sequential dependency is not in the data itself, but in some hidden underlying markov process. In that case, the hidden variable $x_t$ is the \emph{state} of the process. The observed variable $y_t$ is simply an observation.}
  \begin{figure}[H]
    \centering
    \begin{tikzpicture}
      \node[RV, hidden] at (-2,0) (xt1) {$x_{t-1}$};
      \node[RV, hidden] at (0,0) (xt2) {$x_t$};
      \node[RV, hidden] at (2,0) (xt3) {$x_{t+1}$};
      \node[RV] at (-2,2) (yt1) {$y_{t-1}$};
      \node[RV] at (0,2) (yt2) {$y_t$};
      \node[RV] at (2,2) (yt3) {$y_{t+1}$};
      \draw[->] (xt1)--(xt2);
      \draw[->] (xt2)--(xt3);
      \draw[->] (xt1)--(yt1);
      \draw[->] (xt2)--(yt2);
      \draw[->] (xt3)--(yt3);
    \end{tikzpicture}
    \label{fig:markov-chain}
    \caption{Graphical model for a hidden Markov model.}
  \end{figure}
  \begin{align}
    \label{eq:hmm}
    P_\param(x_{t+1} \mid x_t) \tag{transition distribution}\\
    P_\param(y_t \mid x_t) \tag{emission distribution}
  \end{align}
  \only<article>{For any given parater value $\param$, it is easy to estimate the probability distribution over states given the observations $P_\param(x^T \mid y^T)$. As an example, if $y^T$ is raw speech data and $x^T$ is a sequence of words, and $\theta$ are the parameters of our speech model, then we can obtain probabilities for every possible sequence of words that was uttered. Frequently, though, in speech recognition we are only interested in the most likely seuence of words. This makes the problem simple enough to be solved instantaneously by modern cellphones.}
  \begin{block}{Application}
    \begin{itemize}
    \item Speech recognition.
    \item Filtering (Kalman Filter).
    \item DNA analysis.
    \end{itemize}
  \end{block}
\end{frame}




%%% Local Variables:
%%% mode: latex
%%% TeX-engine: xetex
%%% TeX-master: "notes.tex"
%%% End:

%\include{rnn}

\printindex


\bibliographystyle{plainnat}
\bibliography{../bibliography}


\end{document}
%%% Local Variables:
%%% mode: latex
%%% TeX-engine: xetex
%%% TeX-master: "book"
%%% End:
